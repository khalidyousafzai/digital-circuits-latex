%this file describes urdu commands for the commonly used english latex commands

%chapter, section etc
%  \newcommand*{newcommand}[arguments]{actual command}
\newcommand*{\باب}[1]{\chapter{#1}}                                      %defining commonly used commands
\newcommand*{\حصہ}[1]{\section{#1}}
\newcommand*{\جزوحصہ}[1]{\subsection{#1}}
\newcommand*{\جزوجزوحصہ}[1]{\subsubsection{#1}}

\newcommand*{\بابء}[1]{\chapter*{#1}}                                      %defining commonly used commands
\newcommand*{\حصہء}[1]{\section*{#1}}
\newcommand*{\جزوحصہء}[1]{\subsection*{#1}}
\newcommand*{\جزوجزوحصہء}[1]{\subsubsection*{#1}}


%english text in urdu mode
\newcommand*{\تحریر}[1]{\textenglish{#1}}	% english text in urduMode
%\newcommand*{\موٹا}[1]{\textbf{#1}}
%\newcommand*{\ترچھا}[1]{{\textit{#1}}}
\newcommand*{\موٹا}[1]{{\urduTechTermsfont{#1}}}
\newcommand*{\ترچھا}[1]{{\small{#1}}}

%%\newcommand*{\اصطلاح}[1]{{\color{red}{#1}}}   %colours spills to next word when there is index or footnote entry with the word
%%\newcommand{\اصطلاح}[1]{{\urdufontBig{#1}}}

\newcommand{\اصطلاح}[1]{{\urduTechTermsfont{#1}}}  %%%moved to main file


%end commands cannot be redefined and as such these two are not usable
\providecommand*{\ابتدا}[1]{\begin{#1}}
\providecommand*{\انتہا}[1]{\end{#1}}

%include and input directives for adding external files into the main document 
\newcommand*{\بشمول}[1]{\includeonly{#1}}
\newcommand*{\شامل}[1]{\include{#1}}
\newcommand*{\داخل}[1]{\input{#1}}

%to use extra latex packages
\newcommand*{\استعمال}[1]{\usepackage{#1}}

%footnotes and indexes
\newcommand*{\حاشیہب}[1]{{\raggedright{\footnote{\textenglish{#1}}}} }          %  moved to main file. footnote to the left hand side
\newcommand*{\حاشیہد}[1]{{\raggedleft{\footnote{#1}}}}
\newcommand*{\حاشیہط}[1]{\marginpar{#1}}

\newcommand*{\فرہنگ}[1]{\index{#1}}

%references and labels
\newcommand*{\شناخت}[1]{\label{#1}}
\newcommand*{\حوالہ}[1]{\ref{#1}}
\newcommand*{\حوالہء}[1]{#1}			%dummy to enter the figure number directly while the figure is not ready
\newcommand*{\حوالہصفحہ}[1]{\pageref{#1}}

%counters
\newcommand*{\فاصلہ}{\vspace*{10mm}}
\newcommand*{\فاصلہء}{\quad}

%itemize, bullets and numbered items   
\newcommand*{\اشیاء}{itemize}                               %used in   \begin{itemize}
\newcommand*{\شے}[1]{\item {#1}}			%used in    \item, \description
%description
\newcommand*{\جزو}[1]{\item[#1]}                      %used in \begin{description}
%maths commands
\newcommand*{\عددی}[1]{\: \ensuremath{{#1}} \:} % in-line math & inside math mode; {#1} ensures no splitting of inline eqs
\newcommand*{\عددیء}[1]{\ensuremath{{#1}}}
\newcommand*{\سمتیہ}[1]{\ensuremath{{\bf{#1}}}}
\newcommand*{\سمتیازیرنوشت}[2]{\ensuremath{{\boldsymbol{#1}}_{\textup{#2}}}}

\newcommand*{\ضرب}{\time}					%multiplication symbol
\newcommand*{\نکطہد}{\cdot}
\newcommand*{\نقطے}{\ensuremath{\cdots}}

\newcommand*{\زیرنوشت}[3]{\: \ensuremath{{#1_{#2 \textrm {#3}}}} \:}   %english+urdu subscript \زیرنوشت{V}{CE}{غیرافزائندہ}
\newcommand*{\سیدھازیرنوشت}[2]{\: \ensuremath{{#1_{\textup{#2}}}} \:} %RC

\newcommand*{\قریب}[1]{\mbox{#1}}  %dissallows splitting along two lines 
%\newcommand{\سن}[1]{؁\,\ensuremath{#1}}
\newcommand{\سن}[1]{
#1 ؁
}
\newcommand{\زور}[1]{\aemph{#1}} %overline urdu text to emphasize


\renewcommand{\indexname}{فرہنگ}        %does nothing here. must be placed within begin{urdufont} environment to be the last to take effect 
%===============================


%numbering scheme
\renewcommand*{\thefigure}{\thechapter.\arabic{figure}}
\renewcommand*{\thetable}{\thechapter.\arabic{table}}
\renewcommand*{\theequation}{\thechapter.\arabic{equation}}
\renewcommand*{\thesection}{\thechapter.\arabic{section}}
\renewcommand*{\thesubsection}{\thechapter.\arabic{section}.\arabic{subsection}}
\renewcommand*{\thesubsubsection}{\thechapter.\arabic{section}.\arabic{subsection}.\arabic{subsubsection}}
%=======================================
%the following pertains to theorem environment. added to maths sty
\renewcommand*{\thetheorem}{\thechapter.\arabic{theorem}}
\renewcommand*{\thecorollary}{\thechapter.\arabic{theorem}.\arabic{corollary}}
\renewcommand*{\thelemma}{\thechapter.\arabic{lemma}}
\renewcommand*{\thedefinition}{\thechapter.\arabic{definition}.}

%================
%my environments
%================

%environment for examples مثال
%\newcounter{examplecounter}[section]
%\renewcommand{\theexamplecounter}{\arabic{examplecounter}}
\newcounter{examplecounter}[chapter]
\renewcommand{\theexamplecounter}{\thechapter.\arabic{examplecounter}}

\newenvironment*{مثال}
{\par\noindent\ignorespaces    مثال \refstepcounter{examplecounter} \theexamplecounter :\quad}%
{\hfill\qedsymbol  \vspace{\baselineskip}\par}
%{\noindent\ignorespaces \vspace{\baselineskip} \hrule \vspace{\baselineskip}  مثال \refstepcounter{examplecounter} \theexamplecounter :}%
%{\par\noindent \hrule  \vspace{\baselineskip}}

%------
%practice problems environment مشق

%\newcounter{practicecounter}[section]                              %practice here means مشق
%\renewcommand{\thepracticecounter}{\arabic{practicecounter}}
\newcounter{practicecounter}[chapter]                              %practice here means مشق
\renewcommand{\thepracticecounter}{\thechapter.\arabic{practicecounter}}

\newenvironment*{مشق}
{\par\noindent\ignorespaces \vspace{\baselineskip} \hrule \vspace{\baselineskip} مشق \refstepcounter{practicecounter} \thepracticecounter :\quad}%
{\par\noindent \hrule  \vspace{\baselineskip}}
%---------

%end of chapter questions environment سوال

\newcounter{questioncounter}[chapter]                           %for reseting at every section. to be used only during writing stage
%\renewcommand{\thequestioncounter}{\arabic{questioncounter}}
%\newcounter{questioncounter}[chapter]						%when book is finished, use this instead of the above
\renewcommand{\thequestioncounter}{\thechapter.\arabic{questioncounter}}

\newenvironment*{سوال}				
{\noindent\ignorespaces  سوال \refstepcounter{questioncounter} \thequestioncounter :\quad}%
{\par\noindent\ignorespaces }
%--------------------

%defining a LAW   قانون

%\newcounter{lawcounter}[section]
%\renewcommand{\thelawcounter}{\arabic{lawcounter}}
\newcounter{lawcounter}[chapter]
\renewcommand{\thelawcounter}{\thechapter.\arabic{lawcounter}}

\newenvironment*{قانون}				
{\par\medskip \refstepcounter{lawcounter} \quad\nopagebreak}%
{\par\hfill\qedsymbol  \vspace{\baselineskip}\par }
%{\par\medskip }
%--------------------
%--------------------

%defining a THEOREM   مسئلہ

%\newcounter{kthcounter}[section]
%\renewcommand{\thekthcounter}{\arabic{kthcounter}}
\newcounter{kthcounter}[chapter]
\renewcommand{\thekthcounter}{\thechapter.\arabic{kthcounter}}

\newenvironment*{مسئلہ}				
{\par\noindent\ignorespaces  مسئلہ \refstepcounter{kthcounter} \thekthcounter :\quad}%
{\par\noindent }
%--------------------
%--------------------

%defining a Proof   ثبوت

%\newcounter{kprcounter}[chapter]
%\renewcommand{\thekprcounter}{\thechapter.\arabic{kprcounter}}

\newenvironment*{ثبوت}				
{\noindent\ignorespaces  ثبوت :\quad}%
{\par\hfill\qedsymbol  \vspace{\baselineskip}\par }
%{\par\noindent\qedsymbol  \vspace{\baselineskip} }
%--------------------
%--------------------
%++++++++++++++++++++++++++++++

%--------------------

%defining a COROLLARY   ضمنی نتیجہ


\newcounter{kcocounter}[chapter]
\renewcommand{\thekcocounter}{\thechapter.\arabic{kcocounter}}

\newenvironment*{ضمنی نتیجہ}				
{\par\noindent\ignorespaces  ضمنی نتیجہ \refstepcounter{kcocounter} \thekcocounter :\quad}%
{\par\noindent }
%--------------------
%--------------------

%defining a Proof   ثبوت ضمنی نتیجہ

%\newcounter{kprcocounter}[chapter]
%\renewcommand{\thekprcocounter}{\thechapter.\arabic{kprcocounter}}

\newenvironment*{ثبوت ضمنی نتیجہ}				
{\noindent\ignorespaces  ثبوت ضمنی نتیجہ :\quad}%
{\par\hfill\qedsymbol  \vspace{\baselineskip}\par }
%{\par\noindent\qedsymbol  \vspace{\baselineskip} }
%--------------------
%--------------------
%++++++++++++++++++++++++++++++++++++++++++++

%defining a Definition تعریف

%\newcounter{kdfcounter}[chapter]
%\renewcommand{\thedfrcounter}{\thechapter.\arabic{kdfcounter}}

\newenvironment*{تعریف}				
{\par\noindent\ignorespaces  تعریف :\quad}%
{\par\hfill\qedsymbol  \vspace{\baselineskip}\par }
%{\par\noindent }
%--------------------
%----------------------------
%defining a Definition تعریفات

%\newcounter{kdfcounter}[chapter]
%\renewcommand{\thedfrcounter}{\thechapter.\arabic{kdfcounter}}

\newenvironment*{تعریفات}				
{\par\noindent\ignorespaces  تعریفات :\quad}%
{\par\hfill\qedsymbol  \vspace{\baselineskip}\par }
%{\par\noindent }
%--------------------
%--------------------

%defining a Definition مفروضہ

%\newcounter{kAscounter}[chapter]
%\renewcommand{\theAsrcounter}{\thechapter.\arabic{kAscounter}}

\newenvironment*{مفروضہ}				
{\par\noindent\ignorespaces  مفروضہ \quad}%
{\par\hfill\qedsymbol  \vspace{\baselineskip}\par }
%{\par\noindent }
%--------------------
%--------------------

%defining a RULE   قاعدہ

%\newcounter{krucounter}[section]
%\renewcommand{\thekrucounter}{\arabic{krucounter}}
\newcounter{krucounter}[chapter]
\renewcommand{\thekrucounter}{\thechapter.\arabic{krucounter}}

\newenvironment*{قاعدہ}				
{\par\noindent\ignorespaces  قاعدہ \refstepcounter{krucounter} \thekrucounter :\quad}%
{\par\noindent }
%--------------------
%--------------------

%defining a Proof   ثبوت قاعدہ

%\newcounter{kprRcounter}[chapter]
%\renewcommand{\thekprRcounter}{\thechapter.\arabic{kprRcounter}}

\newenvironment*{ثبوت قاعدہ}				
{\noindent\ignorespaces  ثبوت قاعدہ :\quad}%
{\par\hfill\qedsymbol  \vspace{\baselineskip}\par }
%{\par\noindent\qedsymbol  \vspace{\baselineskip} }
%--------------------
%--------------------

%defining a TEST   پرکھ

%\newcounter{kttcounter}[section]
%\renewcommand{\thekttcounter}{\arabic{kttcounter}}
\newcounter{kttcounter}[chapter]
\renewcommand{\thekttcounter}{\thechapter.\arabic{kttcounter}}

\newenvironment*{پرکھ}				
{\par\noindent\ignorespaces  \refstepcounter{kttcounter} \quad \nopagebreak}%
{\par\hfill\qedsymbol  \vspace{\baselineskip}\par }
%--------------------

\newenvironment*{ثبوت پرکھ}				
{\noindent\ignorespaces  ثبوت پرکھ :\quad}%
{\par\hfill\qedsymbol  \vspace{\baselineskip}\par }
%{\par\noindent\qedsymbol  \vspace{\baselineskip} }

%---------

%questions environment جواب

%\newcounter{answercounter}[section]                                           %for reseting at every section. NOT NEEDED
%\renewcommand{\theanswercounter}{\arabic{answercounter}}
%\newcounter{answercounter}[chapter]						%when book is finished, use this instead of the above
%\renewcommand{\theanswercounter}{\thechapter.\arabic{answercounter}}

%\newenvironment*{جواب}				
%{\noindent\ignorespaces\wf{\thequestioncounter)\noexpand\quad}}%
%{\par\noindent\ignorespacesafterend}
%==================

\newenvironment*{سوالات}				
{\noindent\ignorespaces\wf{
\موٹا{حصہ} 
\thesection
\quad
\موٹا{صفحہ}
\thepage}
\wf{\unexpanded{\begin{description}\setlength{\parskip}{0pt} \setlength{\itemsep}{0pt plus 1pt}}}
}%
{\wf{\unexpanded{\end{description}}}\par\noindent\ignorespacesafterend}
%==================
\newenvironment*{جواب}  %the next is a better environment. it take less input		
{\noindent\ignorespaces\wf{\unexpanded{\item[}}\wf{\thequestioncounter)}\wf{\unexpanded{]}}}%
{\noindent\ignorespacesafterend}
%%==================
%%==================
%%this takes one input hence there is no need to write \wf{\unexpanded{...}} in every answer
%\newenvironment*{جواب}[1]			%put actual answer in {} examples {\(\sqrt{2}\)} 	
%{\noindent\ignorespaces\wf{\unexpanded{\item[}}\wf{\thequestioncounter)}\wf{\unexpanded{]}}\wf{\unexpanded{#1}}}%
%{\noindent\ignorespacesafterend}
%==================
%\newenvironment*{جوابء}				%i think is wrong and not used	
%{\noindent\ignorespaces\wf{\unexpanded{[}}\wf{\thequestioncounter)}\wf{\unexpanded{]}}}%
%{\par\noindent\ignorespacesafterend}
%==================


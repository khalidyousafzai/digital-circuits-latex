\باب{قابل تشکیل ترکیبی منطقی ادوار}
پختہ حافظہ استعمال کرتے ہوئے تفاعل کا حصول گزشتہ باب میں دکھایا گیا۔ \عددی{m} بِٹ پتہ پختہ حافظہ میں تمام ممکنہ \عددی{2^m} ارکان ضرب موجود ہوتے ہیں جنہیں جمع گیٹوں سے جوڑ کر درکار تفاعل حاصل کیے جا سکتے ہیں۔ پختہ حافظہ\اصطلاح{ قابل تشکیل ترکیبی منطقی ادوار }\فرہنگ{قابل تشکیل منطقی دور}\حاشیہب{programmable logic devices (PLDs)}\فرہنگ{PLD, programmable logic device}، جن پر یہاں غور کیا جائے گا، کی ایک قسم ہے۔

قابل تشکیل ترکیبی منطقی ادوار کی پہلی قسم \اصطلاح{ قابل تشکیل جمع ترکیبی منطقی ادوار }\فرہنگ{ترکیبی منطقی ادوار!قابل تشکیل جمع} \حاشیہب{programmable array logic (PAL)}\فرہنگ{PAL, programmable array logic} ہے ، جن میں پہلا صف ضرب گیٹ اور دوسرا جمع گیٹ کا ہوتا ہے اور جو مجموعہ ارکان ضرب کی صورت میں تفاعل دیتے ہیں۔ ضرب گیٹوں کی صف میں داخلی برقی جوڑ اٹل جبکہ دوسری صف کے جمع گیٹوں کے داخلی برقی جوڑ قابل تشکیل ہوتے ہیں۔پختہ حافظہ اس قسم میں شمار ہوتا ہے۔

قابل تشکیل ترکیبی منطقی ادوار کی دوسری قسم \اصطلاح{ قابل تشکیل ضرب ترکیبی منطقی ادوار }\فرہنگ{ترکیبی منطقی ادوار!قابل تشکیل ضرب} \حاشیہب{programmable logic array (PLA)}\فرہنگ{PLA, programmable logic array} ہے ، جن میں پہلا صف ضرب گیٹ اور دوسرا جمع گیٹ کا ہوتا ہے اور جو مجموعہ ارکان ضرب کی صورت میں تفاعل دیتے ہیں۔ پہلی صف کے ضرب گیٹوں کے داخلی برقی جوڑ قابل تشکیل جبکہ دوسری صف کے جمع گیٹوں کے داخلی برقی جوڑ اٹل ہوتے ہیں۔

تیسری اور سب سے زیادہ لچک دار قابل تشکیل ترکیبی منطقی ادوار کی قسم میں پہلی صف کے ضرب گیٹوں کے داخلی جوڑ اور دوسری صف کے جمع گیٹوں کے داخلی جوڑ دونوں قابل تشکیل ہوتے ہیں۔انہیں\اصطلاح{ قابل تشکیل ضرب و جمع ترکیبی منطقی ادوار }\فرہنگ{ترکیبی منطقی ادوار!قابل تشکیل ضرب و جمع}\حاشیہب{CPLD, complex programmable logic devices}\فرہنگ{CPLD} کہتے ہیں۔

مذکورہ بالا ادوار پروگرامر (مخلوط دور برنامہ نویس) سے تشکیل دیے جاتے ہیں۔


\جزوحصہ{قابل تشکیل ضرب ترکیبی منطقی ادوار}
قابل تشکیل ضرب ترکیبی منطقی ادوار کی عمومی ساخت شکل \حوالہء{ 10.1 } میں دکھائی گئی ہے جہاں دور کے چار مداخل اور تین مخارج ہیں۔ان ادوار میں عموماً کئی مخارج اشارے بھی بطور مداخل استعمال کیے جاتے ہیں جیسے یہاں \عددی{F_2} استعمال کیا گیا ہے۔

دکھائے گئے دور کے تین یکساں حصے ہیں۔ہر حصہ میں دس مداخل تین ضرب گیٹ ہیں جو تین مداخل ایک جمع گیٹ کو جاتے ہیں۔ضرب گیٹ کے مداخل قابل تشکیل جبکہ جمع گیٹ کے مداخل اٹل ہیں۔دور کے کُل چار مداخل ہیں جنہیں مستحکم کار سے گزار کر ان کے متمم بھی ضرب گیٹ کو مہیا کیے گئے ہیں۔اس دور میں \عددی{10} داخلی کُل\عددی{9} جمع گیٹ ہیں لہٰذا اس میں \عددی{9\times 10=90} فتیلے ہوں گے۔

عام دستیاب ادوار میں مداخل اور مخارج کی تعداد اس سے زیادہ ہو گی، مثلاً ان میں سولہ مداخل، آٹھ مخارج اور آٹھ یکساں اندرونی حصے ہو سکتے ہیں جن میں ہر حصہ آٹھ ضرب اور ایک جمع گیٹ پر مشتمل ہو گا۔مزید خارجی اشاروں پر مستحکم کار نصب ہو سکتے ہیں جنہیں بلند رکاوٹی حال کیا جا سکتا ہے۔

آئیں اس دور کو استعمال کرتے ہوئے درج ذیل تفاعل حاصل کرتے ہیں جو ارکان ضرب کے روپ میں دیے گئے ہیں۔
\begin{gather}
\begin{aligned}
F_0(A,B,C,D)&=\sum(4,5,10,14)\\
F_1(A,B,C,D)&=\sum(0,1,5,7,9,13,14,15)\\
F_2(A,B,C,D)&=\sum(0,1,5,7,14,15)
\end{aligned}
\end{gather}
ان تفاعل کا سادہ روپ درج ذیل ہے۔
\begin{gather}
\begin{aligned}
F_0&=\overline{A}B\overline{C}+AC\overline{D}\\
F_1&=\overline{A}\,\overline{B}\,\overline{C}+\overline{A}BD+ABC+A\overline{B}C=F_2+A\overline{B}C\\
F_2&=\overline{A}\,\overline{B}\,\overline{C}+\overline{A}BD+ABC
\end{aligned}
\end{gather}

ان مساواتوں میں کوئی بھی ضربی رکن تین سے زیادہ مداخل پر مشتمل نہیں لہٰذا درج بالا تفاعلات کو شکل \حوالہء{ 10.1 } میں پیش قابل تشکیل ترکیبی منطقی دور استعمال کر کے حاصل کیا جا سکتا ہے۔شکل \حوالہء{10.2 } میں تفاعلات کا دور دکھایا گیا ہے جہاں موجود جوڑ صلیبی نشان سے ظاہر کیے گئے ہیں۔ باقی جوڑ منقطع کیے گئے ہیں۔


\جزوحصہ{قابل تشکیل ضرب و جمع ترکیبی منطقی ادوار}
ان ادوار میں بھی پہلی صف ضرب گیٹ اور دوسری صف جمع گیٹوں کی ہوتی ہے البتہ ان میں ضرب گیٹوں اور جمع گیٹوں کے تمام جوڑ قابل تشکیل ہوتے ہیں۔یوں استعمال کے نکتہ نظر سے یہ نہایت لچک دار ہوتے ہیں۔ 


شکل\حوالہء{ 10.3 } میں قابل تشکیل ضرب و جمع ترکیبی منطقی دور دکھایا گیا ہے۔اس دور میں تمام ضرب گیٹوں کے داخلی جوڑ اور تمام جمع گیٹوں کے داخلی جوڑ قابل تشکیل ہیں۔اس دور میں آٹھ داخلی چھ ضرب گیٹ اور چھ داخلی تین جمع گیٹ ہیں۔یوں اس میں کُل جوڑ \عددی{66} ہوں گے۔

اس شکل میں درج ذیل تین تفاعل حاصل کیے گئے ہیں جہاں صلیبی نشان سلامت جوڑ کو ظاہر کرتے ہیں۔ ان تفاعل کے حصول میں چار ضرب گیٹ اور تینوں جمع گیٹ کی ضرورت پیش آئی، جبکہ دو ضرب گیٹ زیر استعمال نہیں آئے۔
\begin{gather}
\begin{aligned}
D_2&=\overline{A_0}\,\overline{A_1} A_2 \overline{A_3}\\
D_1&=\overline{A_1}A_2 \overline{A_3}\\
D_2&=A_0\overline{A_1}A_3+\overline{A_0}\,\overline{A_3}
\end{aligned}
\end{gather}
 یہاں دکھایا گیا قابل تشکیل ضرب و جمع ترکیبی منطقی دور صرف سمجھانے کی خاطر تھا۔حقیقی ادوار میں کئی گنا زیادہ مداخل،مخارج، اور گیٹ ہوں گے۔ثنائی تفاعل کی سادہ ترین صورت حاصل کر کے اسے مخلوط دور میں ڈالا جاتا ہے۔سادہ ترین روپ کا حصول ، جو عموماً ایک مشکل کام ہو گا ، کمپیوٹر کے ذریعے کیا جاتا ہے۔ منقطع ہونے والے فتیلوں کی معلومات بھی کمپیوٹر فراہم کرتا ہے۔فتیلے مخلوط ادوار کا پروگرامر منقطع کرتا ہے۔


\حصہ{ قابل تشکیل ترتیبی ادوار}
جیسا اس باب کی شروع میں ذکر ہوا، \اصطلاح{ وسیع پیمانے کے مخلوط ادوار}\فرہنگ{مخلوط دور!وسیع پیمانہ}\حاشیہب{large scale integration (LSI)}\فرہنگ{LSI, large scale integration} ترتیبی بناوٹ رکھتے ہیں۔قابل تشکیل ترکیبی ادوار کے ساتھ پلٹ منسلک کر کے قابل تشکیل ترتیبی ادوار حاصل کیے جاتے ہیں۔اس طرح کے یکساں کئی حصے ایک مخلوط دور پر میں ڈال کر \اصطلاح{پیچیدہ قابل تشکیل ترتیبی ادوار }\فرہنگ{قابل تشکیل!پیچیدہ ترتیبی دور}\حاشیہب{complex PLD (CPLD)}\فرہنگ{CPLD, complex PLD} بنائے جاتے ہیں۔ان ادوار میں تمام انفرادی حصوں کے مابین، قابل تشکیل ترکیبی ادوار کی طرح، برقی جوڑوں (فتیلوں) کا جال بچھایا جاتا ہے ، اور بیرونی مداخل کے ساتھ ساتھ دور کے مخارج بطور مداخل استعمال کیے جا سکتے ہیں۔

\اصطلاح{انتہائی وسیع پیمانے کے مخلوط ادوار }\فرہنگ{مخلوط دور!انتہائی وسیع پیمانہ}\حاشیہب{very large scale integration (VLSI)}\فرہنگ{VLSI, very large scale integration} کی بناوٹ صف در صف گیٹوں پر مبنی ہوتی ہے۔ایسے جدید مخلوط ادوار میں گیٹوں کی تعداد اربوں میں ہوتی ہے۔

انتہائی وسیع پیمانے کے مخلوط ادوار کا ذکر کرتے ہوئے \موٹا{ مُور } کی پیشن گوئی کا ذکر کرنا لازم ہے جنہوں نے \سن{1965 } میں پیشن گوئی کی کہ مخلوط ادوار میں گیٹوں کی تعداد ہر دو سال میں دگنی ہو گی۔یہ پیشن گوئی جسے \اصطلاح{ مُور کا قانون }\فرہنگ{قانون!مور}\حاشیہب{Moore's law}\فرہنگ{Moore's law} کہتے ہیں اب تک درست ثابت ہوتا آ رہا ہے۔

انتہائی وسیع پیمانہ مخلوط دور تشکیل دینے کی خاطر تفاعل میں مستعمل گیٹ اور ان کے بیچ جوڑ کی معلومات مخلوط دور تیار کرنے والے صنعت کار کو فراہم کیا جاتا ہے۔ مخلوط دور بناتے وقت اس معلومات کے تحت گیٹوں کے بیچ درکار جوڑ بنا دیے جاتے ہیں۔کبھی کبھار صنعت کار صارف کے ضرورت کے مطابق مخلوط دور تیار کرتا ہے۔ایسے تیار کیے جانے والے ادوار کو\اصطلاح{ خصوصی استعمال کے مخلوط ادوار }\فرہنگ{مخلوط دور!خصوصی استعمال}\حاشیہب{application specific integrated circuit (ASIC)}\فرہنگ{ASIC} کہتے ہیں۔

اس سلسلہ کی آخری قسم \اصطلاح{ موقع پر قابل تشکیل گیٹ صف }\فرہنگ{موقع پر قابل تشکیل گیٹ صف}\حاشیہب{field programmable gate array (FPGA)}\فرہنگ{FPGA} ہے جو دراصل انتہائی وسیع پیمانہ مخلوط ادوار کی وہ قسم ہے جسے صارف خود تشکیل دے سکتا ہے۔انہیں بار بار تشکیل دیا جا سکتا ہے۔ ان ادوار میں گیٹ، پلٹ، شناخت کار، عارضی حافظہ اور اس قسم کے دیگر ادوار پائے جاتے ہیں۔موقع پر قابل تشکیل گیٹ صف استعمال کرنے کی خاطر کمپیوٹر کا بھرپور استعمال کیا جاتا ہے۔ \اصطلاح{کمپیوٹر کی مدد سے تیار }\فرہنگ{کمپیوٹر کی مدد سے تیار}\حاشیہب{computer aided design (CAD)}\فرہنگ{CAD} کرنے کی خاطر کئی کمپیوٹر پروگرام استعمال کیے جا سکتے ہیں۔ 

\ابتدا{مشق}
انٹرنیٹ سے \عددی{EPM7032} مخلوط دور کے معلوماتی صفحات حاصل کریں۔(ا) اس میں کتنے یکساں حصے ہیں؟ (ب) کیا ہر حصے میں پلٹ بھی پایا جاتا ہے؟
\انتہا{مشق}

\باب{قابل تشکیل ترکیبی منطقی ادوار}
	پختہ حافظہ، قابلِ تشکیل ترکیبی منطقی ادوار 1 کی پہلی قسم ہے۔بِٹ پتہ کے پختہ حافظہ میں تمام ممکنہارکانِ ضرب موجود ہوتے ہیں جنہیں جمع گیٹوں سے جوڑ کر درکار تفاعل حاصل کیا جاتا ہے۔پختہ حافظہ میں ضرب گیٹوں کے داخلی  برقی جوڑ مقررہ جبکہ جمع گیٹوں کے داخلی برقی جوڑ قابلِ تشکیل ہوتے ہیں۔
	پختہ حافظہ کا دورانیہ رسائی 2، ترکیبی ادوار کے دورانیہ ردِ عمل سے کئی گنا زیادہ ہوتا ہے۔یوں حافظہ کا بطور قابلِ تشکیل ترکیبی منطقی ادوار کے استعمال میں آہستا آہستا کمی آ رہی ہے اور اس کی جگہ ایسے مربوط ادوار کا استعمال بڑھ رہا ہے جو خاص اسی مقصد سے بنائے گئے ہوں۔اس حصہ میں انہیں ادوار پر بحث ہو گا۔
	قابلِ تشکیل ترکیبی منطقی ادوار میں پہلے ضرب گیٹوں کی ایک صف اور اس کے بعد، جمع گیٹوں کی ایک صف ہوتی ہے جن کی مدد سے درکار تفاعل کو مجموعہ ارکانِ ضرب کی صورت میں حاصل کیا جاتا ہے۔ایسے قابلِ تشکیل ترکیبی منطقی ادوار کی پہلی قسم میں ضرب گیٹوں کے صف میں داخلی برقی جوڑ مقررہ ہوتے ہیں جبکہ دوسری صف کے جمع گیٹوں کے داخلی برقی جوڑ قابلِ تشکیل ہوتے ہیں۔پختہ حافظہ بھی اسی قسم میں شمار ہوتا ہے۔ایسے ادوار کو قابلِ تشکیل جمع، ترکیبی منطقی ادوار کہتے ہیں۔ 
	قابلِ تشکیل ترکیبی منطقی ادوار کی دوسری قسم میں پہلی صف کے ضرب گیٹوں کے داخلی برقی جوڑ، قابلِ تشکیل ہوتے ہیں جبکہ  دوسری صف کے جمع گیٹوں کے داخلی برقی جوڑ مقررہ ہوتے ہیں۔انہیں قابلِ تشکیل ضرب، ترکیبی منطقی ادوار 3 کہتے ہیں۔انہیں مخلوط ادوار کے پروگرامر کی مدد سے تشکیل دیا جاتا  ہے۔
	تیسری اور سب سے لچک دار، قابلِ تشکیل ترکیبی منطقی ادوار کی قسم میں پہلے صف کے ضرب گیٹوں کے داخلی جوڑ اور دوسرے صف کے جمع گیٹوں کے داخلی جوڑ تمام کے تمام قابلِ تشکیل ہوتے ہیں۔انہیں  قابلِ تشکیل ضرب- جمع ترکیبی منطقی ادوار 4 کہتے ہیں۔

10.1.1 قابلِ تشکیل ضرب ترکیبی منطقی ادوار
	قابلِ تشکیل ضرب جوڑ کے ترکیبی منطقی ادوار کی عمومی ساخت شکل 10.1 میں دکھائی گئی ہے جہاں دور کے چار مداخل اور تین مخارج ہیں۔ان ادوار میں عموماً کئی مخارج اشارات، اسی دور کو بطور مداخل بھی فراہم کئے جاتے ہیں جیسا یہاںکے ساتھ کیا گیا ہے۔


	دکھائے دور کے تین یکساں حصے ہیں۔ہر حصہ میں دس داخلی تین ضرب گیٹ ہیں جو ایک جمع گیٹ کو جاتے ہیں۔ضرب گیٹ کے مداخل قابلِ تشکیل ہیں جبکہ جمع گیٹ کے مداخل مقررہ ہیں۔دور کے کُل چار مداخل ہیں جنہیں وسطی ادوار سے گزار کر ان اشارت کے تکملہ 5 بھی حاصل کر کے ضرب گیٹوں کو مہیا کئے گئے ہیں۔اس دور میں داخلی کُلجمع گیٹ ہیں۔یوں اس میں یعنیفیوز ہیں۔
	عام دستیاب ادوار، زیادہ مداخل اور مخارج رکھتے ہیں، مثلاً ان کے سولہ مداخل، آٹھ مخارج اور آٹھ یکساں اندرونی حصے ہو سکتے ہیں جن میں ہر حصہ آٹھ ضرب اور ایک جمع گیٹ پر مشتمل ہو گا۔مزید یہ کہ خارجی اشارت پر وسطی ادوار نصب ہو سکتے ہیں جن کو کثیر مقاومت حالت کیا جا سکتا ہے۔
	آئیں اس دور کو استعمال کرتے مندرجہ ذیل تفاعل حاصل کرتے ہیں جنہیں ارکانِ ضرب کی صورت میں دیا گیا ہے۔

 
(10.1)

	ان تفاعل کی سادہ اشکال یہ ہے۔

 
(10.2)

ان مساواتوں میں کوئی بھی ضربی رکن تین سے زیادہ مداخل پر مشتمل نہیں۔یوں اس قابلِ تشکیل ترکیبی منطقی دور کو یہ تفاعل حاصل کرنے کے لئے استعمال کیا جا سکتا ہے۔شکل 10.2 میں تفاعل کا دور دکھایا گیا ہے۔


اس شکل میں صلیبی نشان لگے جوڑ موجود ہیں جبکہ بقایا تمام جوڑ منقطع کر دئے گئے ہیں۔
10.1.2 قابلِ تشکیل ضرب- جمع ترکیبی منطقی ادوار
	ان ادوار میں بھی پہلے صف ضرب گیٹوں اور دوسری صف جمع گیٹوں کی ہوتی ہے البتہ ان میں ضرب گیٹوں اور جمع گیٹوں کے تمام جوڑ قابلِ تشکیل ہوتے ہیں۔یوں یہ استعمال کے نکتہ نظر سے نہایت لچک دار ہوتے ہیں۔ 


	شکل 10.3 میں قابلِ تشکیل ضرب-جمع، ترکیبی منطقی دور دکھایا گیا ہے۔جیسے آپ دیکھ سکتے ہیں اس دور میں تمام ضرب گیٹوں کے داخلی جوڑ قابلِ تشکیل ہیں اور اسی طرح اس کے تمام جمع گیٹوں کے داخلی جوڑ قابلِ تشکیل ہیں۔اس دور میں آٹھ داخلی چہ ضرب گیٹ اور چہ داخلی تین جمع گیٹ ہیں۔یوں اس میں کُل جوڑہیں۔
	شکل میں صلیبی نشانات سے سالم جوڑ دکھائے گئے ہیں۔یوں اسے استعمال کرتے تین تفاعل حاصل کئے گئے ہیں۔ایسا کرتے چار ضرب گیٹ اور تینوں جمع گیٹ کی ضرورت پڑی ہے جبکہ دو ضرب گیٹ زیرِ استعمال نہیں آئے۔حاصل کردہ تفاعل مندرجہ زیل ہیں۔

 
(10.3)
 
	یہاں دکھلایا قابلِ تشکیل ایڈ-جمع، ترکیبی منطقی دور صرف سمجھانے کی خاطر تھا۔حقیقی ادوار میں کئی گنا زیادہ مداخل،مخارج اور گیٹ ہوں گے۔ثنائی تفاعل کی سادہ ترین شکل حاصل کر کے اسے مخلوط دور میں ڈالا جاتا ہے۔سادہ ترین شکل ازخود حاصل کرنا عموماً خاصہ مشکل ہوتا ہے اور یہ  کمپیوٹر کی مدد سے کیا جاتا ہے۔کمپیوٹر ہی منقطع ہونے والے فیوز کی معلومات فراہم کرتا ہے۔فیوز مخلوط ادوار کے پروگرامر 6 کی مدد سے منقطع کئے جاتے ہیں۔
10.2 قابلِ تشکیل ترتیبی ادوار
	جیسا اس باب کے شروع میں ذکر ہوا، وسیع پیمانے کے مخلوط ادوار 7 ترتیبی بناوٹ رکھتے ہیں۔قابلِ تشکیل ترکیبی ادوار کے ساتھ پلٹ منسلک کر کے قابلِ تشکیل ترتیبی ادوار حاصل کئے جاتے ہیں۔اس طرح کے کئی یکساں حصے ایک ہی مخلوط دور پر بنا کر مخلوط قابلِ تشکیل ترتیبی ادوار 8 بنائے جاتے ہیں۔ایسے ادوار میں تمام انفرادی حصوں کے مابین، قابلِ تشکیل ترکیبی ادوار کی طرح، برقی جوڑوں کا جال بچایا جاتا ہے۔یوں بیرونی مداخل کے ساتھ ساتھ کسی بھی حصہ کا مخارج بطورِ مداخل استعمال کیا جا سکتا ہے۔
	انتہائی وسیع پیمانے کے مخلوط ادوار 9 کی بناوٹ، صف در صف گیٹوں پر مبنی ہوتی ہے۔ایسے جدید مخلوط ادوار میں گیٹوں کی تعداد اربوں10 میں گنی جاتی ہے۔
	انتہائی وسیع پیمانے کے مخلوط ادوار کا ذکر کرتے مُور11 کی پیشن گوئی کا ذکر کرنا لازم ہے جنہوں نے سن 1965 میں پیشن گوئی کی کہ مخلوط ادوار میں گیٹوں کی تعداد ہر دو سالوں میں دگنی ہو گی۔یہ پیشن گوئی جسے مُور کا قانون 12 کہتے ہیں اب تک درست ثابت ہوتا آ رہا ہے۔
	انتہائی وسیع پیمانے کے مخلوط ادوار تشکیل دینے کی خاطر، صارف درکار تفاعل میں گیٹوں کے آپس میں جوڑ، مخلوط دور تیار کرنے والے صنعت کار کو فراہم کرتا ہے جو اس معلومات سے مخلوط دور بناتے وقت اس میں درکار جوڑ بنا دیتا ہے۔کبھی کبار صنعت کار صارف کے ضرورت کے مطابق مخلوط دور تیار کرتا ہے۔ایسے تیار کئے جانے والے ادوار کو خصوصی استعمال کے مخلوط ادوار  13 کہتے ہیں۔
	اس سلسلہ کی آخری قسم جائے استعمال پر تشکیل کے قابل گیٹوں کے صف 14 ہے جو دراصل انتہائی وسیع پیمانے کے مخلوط ادوار کی وہ قسم ہے جنہیں استعمال کرنے والا ازخود تشکیل دے سکتا ہے۔یہ بار بار تشکیل دئے جانے کی صلاحیت رکھتے ہیں۔
	ان  ادوار میں گیٹ، پلٹ، شناخت کار، عارضی حافظہ اور اس قسم کے دیگر ادوار پائے جاتے ہیں۔جائے استعمال پر تشکیل کے قابل گیٹوں کے صف استعمال کرنے کی خاطر کمپیوٹر کا بھرپور استعمال کیا جاتا ہے۔کمپیوٹر کی مدد سے تیار کرنے 15 کی خاطر کئی کمپیوٹر پروگرام استعمال کئے جاتے ہیں۔ 

مشق:	انٹرنیٹ سےمخلوط دور کے معلوماتی صفحات حاصل کریں۔(ا) اس میں کتنے یکساں حصے ہیں۔ (ب) کیا ہر حصے میں پلٹ بھی پایا جاتا ہے۔

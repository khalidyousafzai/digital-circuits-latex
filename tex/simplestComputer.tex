\باب{سادہ ترین کمپیوٹر}
اس باب میں کمپیوٹر کی سادہ ترین ساخت پر غور کیا جائے گا۔ سادہ ہونے کے باوجود اس میں   کئی   اعلٰی  تصورات شامل ہیں۔ اس باب کو پڑھنے اور   سمجھنے  کے بعد آپ جدید کمپیوٹر کی بناؤٹ سمجھ پائیں گے۔

\حصہ{بناؤٹ}
سادہ کمپیوٹر کی  بناوٹ شکل \حوالہ{شکل_کمپیوٹر_سادہ_ترین} میں پیش ہے۔ یہ ایک مکمل کمپیوٹر ہے۔   دفاتر کے    وہ خروج جو آٹھ بِٹ گزر گاہ     سے جڑے ہیں،  \اصطلاح{سہ حالی}\فرہنگ{حال!سہ}\حاشیہب{tri-state}\فرہنگ{state!tri}    ہیں؛  جو  مواد کی  منظم ترسیل ممکن بناتا  ہے۔ آٹھ بِٹ  گزر گاہ سے مراد آٹھ برقی تاریں ہیں جو    ذیلی ادوار (مثلاً حافظہ، جمع و منفی کار)  کے مابین مواد کی ترسیل ممکن بناتے ہیں۔ دفاتر کے باقی خروج \اصطلاح{   دو حالی  }\فرہنگ{حال!دو}\حاشیہب{two-state}\فرہنگ{state!two} ہیں؛ یہ خروج   ان ڈبہ ادوار کو مسلسل   معلومات (مواد، پتہ، شمار وغیرہ)  فراہم کرتے ہیں جن سے یہ منسلک ہیں۔

\begin{figure}
\centering
\begin{tikzpicture}
\pgfmathsetmacro{\klshift}{0.25}
\pgfmathsetmacro{\knshift}{0.07}
\pgfmathsetmacro{\kmv}{0.15}
\pgfmathsetmacro{\knshift}{0.07}
\pgfmathsetmacro{\kpin}{0.50}
\pgfmathsetmacro{\kpina}{1}
\pgfmathsetmacro{\kpsep}{0.40}			%pin to pin distance
\pgfmathsetmacro{\kW}{\kpsep}
\pgfmathsetmacro{\kulV}{0.40}			%edge clearance along vertical edge
\pgfmathsetmacro{\kulH}{0.50}
\pgfmathsetmacro{\kdimX}{2*\kulH+4*\kpsep}
\pgfmathsetmacro{\kdimY}{2*\kulV+4*\kpsep}		%two spaces between 3 pins
\pgfmathsetmacro{\ksepX}{\kdimX+2*\kpina+1*\kpsep+\kW}
\pgfmathsetmacro{\ksepY}{\kdimY+\kpin+0.5*\kpsep}
\draw(0,0) rectangle ++(\kdimX,\kdimY)node[pos=0.5,rectangle,inner sep=0pt,text width=1.5cm,align=center]{\text{\RL{ہدایت}} \text{\RL{گنت کار}}};
\draw(0,\kulV+0*\kpsep)--++(-\kpin,0)node[left]{$E_P$};
\draw(0,\kulV+1*\kpsep)--++(-\kpin,0)node[left]{$\overline{CLR}$};
\draw(0,\kulV+2*\kpsep)--++(-\kpin,0)node[left]{$\overline{CLK}$};
\draw(0,\kulV+4*\kpsep)--++(-\kpin,0)node[left]{$C_P$};
\draw(0,\kulV+2*\kpsep-\kmv)--++(\kmv,\kmv)--++(-\kmv,\kmv);
\draw(0,\kulV+2*\kpsep)++(-\knshift,0)node[ocirc]{};
\draw(0,\kulV+1*\kpsep)++(-\knshift,0)node[ocirc]{};
\draw(\kdimX,\kulV+1.5*\kpsep)--++(\kpina,0);
\draw(\kdimX,\kulV+2.5*\kpsep)--++(\kpina,0);
\draw(\kdimX,\kulV+1.5*\kpsep)++(\kpina,0)++(-0.5*\kpsep,-0.5*\kpsep)--++(1*\kpsep,1*\kpsep)--++(-1*\kpsep,1*\kpsep);
\draw(\kdimX,\kulV+2*\kpsep)++(0.5*\kpina,0)node[]{$4$};

\draw(0,-\ksepY) rectangle ++(\kdimX,\kdimY)node[pos=0.5,rectangle,inner sep=0pt,text width=2cm,align=center]{\text{\RL{درآمدی سوئچ  اور }} \text{\RL{دفتر پتہ}}};
\draw(0,-\ksepY+\kulV+2*\kpsep)--++(-\kpina,0)node[left]{$CLK$};
\draw(0,-\ksepY+\kulV+2*\kpsep-\kmv)--++(\kmv,\kmv)--++(-\kmv,\kmv);
\draw(0,-\ksepY+\kulV+4*\kpsep)--++(-\kpina,0)node[left]{$\overline{L}_M$};
\draw(0,-\ksepY+\kulV+4*\kpsep)++(-\knshift,0)node[ocirc]{};
\draw(\kulH+0*\kpsep,-\ksepY)--++(0,-\kpin);
\draw(\kulH+1*\kpsep,-\ksepY)--++(0,-\kpin);
\draw(\kulH+0*\kpsep,-\ksepY)++(0,-\kpin)++(-0.5*\kpsep,0.5*\kpsep)--++(\kpsep,-\kpsep)--++(\kpsep,\kpsep);
\draw(\kulH+0.5*\kpsep,-\ksepY)++(0,-0.5*\kpin)node[]{$4$};
\draw(\kulH+3*\kpsep,-\ksepY)--++(0,-\kpin);
\draw(\kulH+4*\kpsep,-\ksepY)--++(0,-\kpin);
\draw(\kulH+3*\kpsep,-\ksepY)++(0,-\kpin)++(-0.5*\kpsep,0.5*\kpsep)--++(\kpsep,-\kpsep)--++(\kpsep,\kpsep);
\draw(\kulH+3.5*\kpsep,-\ksepY)++(0,-0.5*\kpin)node[]{$4$};
\draw(\kdimX,-\ksepY+\kulV+1.5*\kpsep)++(0.5*\kpsep,0)--++(\kpina,0);
\draw(\kdimX,-\ksepY+\kulV+2.5*\kpsep)++(0.5*\kpsep,0)--++(\kpina,0);
\draw(\kdimX,-\ksepY+\kulV+2*\kpsep)--++(1*\kpsep,-1*\kpsep);
\draw(\kdimX,-\ksepY+\kulV+2*\kpsep)--++(1*\kpsep,1*\kpsep);
\draw(\kdimX,-\ksepY+\kulV+2*\kpsep)++(0.5*\kpina,0)node[]{$4$};

\draw(0,-2*\ksepY) rectangle ++(\kdimX,\kdimY)node[pos=0.5,rectangle,inner sep=0pt,text width=2cm,align=center]{$16\times 8$ \text{\RL{عارضی حافظہ}}};
\draw(0,-2*\ksepY+\kulV)--++(-\kpina,0)node[left]{$\overline{CE}$}; 
\draw(0,-2*\ksepY+\kulV)++(-\knshift,0)node[ocirc]{};
\draw(\kdimX,-2*\ksepY+\kulV+1.5*\kpsep)--++(\kpina,0);
\draw(\kdimX,-2*\ksepY+\kulV+2.5*\kpsep)--++(\kpina,0);
\draw(\kdimX,-2*\ksepY+\kulV+1.5*\kpsep)++(\kpina,0)++(-0.5*\kpsep,-0.5*\kpsep)--++(1*\kpsep,1*\kpsep)--++(-1*\kpsep,1*\kpsep);
\draw(\kdimX,-2*\ksepY+\kulV+2*\kpsep)++(0.5*\kpina,0)node[]{$8$};

\draw(0,-3*\ksepY) rectangle ++(\kdimX,\kdimY)node[pos=0.5,rectangle,inner sep=0pt,text width=2cm,align=center]{ \text{\RL{دفتر ہدایت}}};
\draw(0,-3*\ksepY+\kulV+0*\kpsep)--++(-\kpina,0)node[left]{$\overline{E}_I$};
\draw(0,-3*\ksepY+\kulV+1*\kpsep)--++(-\kpina,0)node[left]{$CLR$};
\draw(0,-3*\ksepY+\kulV+2*\kpsep)--++(-\kpina,0)node[left]{$CLK$};
\draw(0,-3*\ksepY+\kulV+4*\kpsep)--++(-\kpina,0)node[left]{$\overline{L}_I$};
\draw(0,-3*\ksepY+\kulV+0*\kpsep)++(-\knshift,0)node[ocirc]{};
\draw(0,-3*\ksepY+\kulV+4*\kpsep)++(-\knshift,0)node[ocirc]{};
\draw(\kdimX,-3*\ksepY+\kulV+1*\kpsep)--++(\kpina,0);
\draw(\kdimX,-3*\ksepY+\kulV+2*\kpsep)--++(\kpina,0);
\draw(\kdimX,-3*\ksepY+\kulV+1*\kpsep)++(\kpina,0)++(-0.5*\kpsep,-0.5*\kpsep)--++(1*\kpsep,1*\kpsep)--++(-1*\kpsep,1*\kpsep);
\draw(\kdimX,-3*\ksepY+\kulV+1.5*\kpsep)++(0.5*\kpina,0)node[]{$4$};
\draw(\kdimX,-3*\ksepY+\kulV+3*\kpsep)++(0.5*\kpsep,0)--++(\kpina,0);
\draw(\kdimX,-3*\ksepY+\kulV+4*\kpsep)++(0.5*\kpsep,0)--++(\kpina,0);
\draw(\kdimX,-3*\ksepY+\kulV+3.5*\kpsep)--++(1*\kpsep,-1*\kpsep);
\draw(\kdimX,-3*\ksepY+\kulV+3.5*\kpsep)--++(1*\kpsep,1*\kpsep);
\draw(\kdimX,-3*\ksepY+\kulV+3.5*\kpsep)++(0.5*\kpina,0)node[]{$8$};
\draw(\kulH+1.5*\kpsep,-3*\ksepY)--++(0,-\kpin);
\draw(\kulH+2.5*\kpsep,-3*\ksepY)--++(0,-\kpin);
\draw(\kulH+1.5*\kpsep,-3*\ksepY)++(0,-\kpin)++(-0.5*\kpsep,0.5*\kpsep)--++(\kpsep,-\kpsep)--++(\kpsep,\kpsep);
\draw(\kulH+2*\kpsep,-3*\ksepY)++(0,-0.5*\kpin)node[]{$4$};

\draw(0,-4*\ksepY) rectangle ++(\kdimX,\kdimY)node[pos=0.5,rectangle,inner sep=0pt,text width=3cm,align=center]{ \text{\RL{قابو  و ترتیب کار}}};
\draw[-latex](\kdimX,-4*\ksepY+\kulV+0*\kpsep)--++(\kpina,0)node[right]{$\overline{CLR}$}; 
\draw[-latex](\kdimX,-4*\ksepY+\kulV+1*\kpsep)--++(\kpina,0)node[right]{$CLR$}; 
\draw[-latex](\kdimX,-4*\ksepY+\kulV+2*\kpsep)--++(\kpina,0)node[right]{$\overline{CLK}$}; 
\draw[-latex](\kdimX,-4*\ksepY+\kulV+3*\kpsep)--++(\kpina,0)node[right]{$CLK$}; 
\draw(\kulH+1.5*\kpsep,-4*\ksepY)--++(0,-\kpin);
\draw(\kulH+2.5*\kpsep,-4*\ksepY)--++(0,-\kpin);
\draw(\kulH+1.5*\kpsep,-4*\ksepY)++(0,-\kpin)++(-0.5*\kpsep,0.5*\kpsep)--++(\kpsep,-\kpsep)--++(\kpsep,\kpsep);
\draw(\kulH+2*\kpsep,-4*\ksepY)++(0,-0.5*\kpin)node[]{$12$};
\draw(\kulH+2*\kpsep,-4*\ksepY)++(0,-1*\kpin-\kpsep)node[below]{$C_PE_P\overline{L}_M\overline{CE}\,\,\, \overline{L}_I\overline{E}_I\overline{L}_AE_A\,\,S_UE_U\overline{L}_B\overline{L}_O$};
\draw(\kdimX,-4*\ksepY+\kulV+0*\kpsep)++(\knshift,0)node[ocirc]{};
\draw(\kdimX,-4*\ksepY+\kulV+2*\kpsep)++(\knshift,0)node[ocirc]{};

\draw(\kdimX+\kpina+0.5*\kpsep,\kdimY) rectangle ++(\kW,-4*\ksepY+\kpin);
\draw(\kdimX+\kpina+0.5*\kpsep,\kdimY)++(0.5*\kW,0)node[above, text width=1.5cm,align=center]{\text{\RL{گزرگاہ}} $W$}node[below]{$8$};

\draw(\ksepX,-0*\ksepY) rectangle ++(\kdimX,\kdimY)node[pos=0.5,rectangle,inner sep=0pt,text width=1cm,align=center]{ \text{\RL{دفتر}} $A$};
\draw(\ksepX,-0*\ksepY+\kulV+0*\kpsep)--++(-\kpina,0); 
\draw(\ksepX,-0*\ksepY+\kulV+1*\kpsep)--++(-\kpina,0); 
\draw(\ksepX,-0*\ksepY+\kulV+0*\kpsep)++(-\kpina,0)++(0.5*\kpsep,-0.5*\kpsep)--++(-\kpsep,\kpsep)--++(\kpsep,\kpsep);
\draw(\ksepX,-0*\ksepY+\kulV+0.5*\kpsep)++(-0.5*\kpina,0)node[]{$8$};
\draw(\ksepX,-0*\ksepY+\kulV+3*\kpsep)++(-0.5*\kpsep,0)--++(-\kpina,0); 
\draw(\ksepX,-0*\ksepY+\kulV+4*\kpsep)++(-0.5*\kpsep,0)--++(-\kpina,0); 
\draw(\ksepX,-0*\ksepY+\kulV+3*\kpsep)++(-1*\kpsep,-0.5*\kpsep)--++(\kpsep,\kpsep)--++(-\kpsep,\kpsep);
\draw(\ksepX,-0*\ksepY+\kulV+3.5*\kpsep)++(-0.5*\kpina,0)node[]{$8$};
\draw(\ksepX+\kulH+1.5*\kpsep,-0*\ksepY)--++(0,-\kpin);
\draw(\ksepX+\kulH+2.5*\kpsep,-0*\ksepY)--++(0,-\kpin);
\draw(\ksepX+\kulH+1.5*\kpsep,-0*\ksepY)++(0,-\kpin)++(-0.5*\kpsep,0.5*\kpsep)--++(\kpsep,-\kpsep)--++(\kpsep,\kpsep);
\draw(\ksepX+\kulH+2*\kpsep,-0*\ksepY)++(0,-0.5*\kpin)node[]{$8$};
\draw(\ksepX+\kdimX,-0*\ksepY+\kulV+0*\kpsep)--++(\kpin,0)node[right]{$E_A$};
\draw(\ksepX+\kdimX,-0*\ksepY+\kulV+2*\kpsep)--++(\kpin,0)node[right]{$CLK$};
\draw(\ksepX+\kdimX,-0*\ksepY+\kulV+4*\kpsep)--++(\kpin,0)node[right]{$\overline{L}_A$};
\draw(\ksepX+\kdimX,-0*\ksepY+\kulV+2*\kpsep)++(0,-\kmv)--++(-\kmv,\kmv)--++(\kmv,\kmv);
\draw(\ksepX+\kdimX,-0*\ksepY+\kulV+4*\kpsep)++(\knshift,0)node[ocirc]{};
\draw(\ksepX,-1*\ksepY) rectangle ++(\kdimX,\kdimY)node[pos=0.5,rectangle,inner sep=0pt,text width=2cm,align=center]{ \text{\RL{جمع و منفی کار}}};
\draw(\ksepX+\kdimX,-1*\ksepY+\kulV+0*\kpsep)--++(\kpin,0)node[right]{$E_U$};
\draw(\ksepX+\kdimX,-1*\ksepY+\kulV+4*\kpsep)--++(\kpin,0)node[right]{$S_U$};
\draw(\ksepX,-1*\ksepY+\kulV+1.5*\kpsep)--++(-\kpina,0); 
\draw(\ksepX,-1*\ksepY+\kulV+2.5*\kpsep)--++(-\kpina,0); 
\draw(\ksepX,-1*\ksepY+\kulV+1.5*\kpsep)++(-\kpina,0)++(0.5*\kpsep,-0.5*\kpsep)--++(-\kpsep,\kpsep)--++(\kpsep,\kpsep);
\draw(\ksepX,-1*\ksepY+\kulV+2*\kpsep)++(-0.5*\kpina,0)node[]{$8$};
\draw(\ksepX+\kulH+1.5*\kpsep,-1*\ksepY)++(0,-0.5*\kpsep)--++(0,-\kpin);
\draw(\ksepX+\kulH+2.5*\kpsep,-1*\ksepY)++(0,-0.5*\kpsep)--++(0,-\kpin);
\draw(\ksepX+\kulH+1.5*\kpsep,-1*\ksepY)++(-0.5*\kpsep,-\kpsep)--++(\kpsep,\kpsep)--++(\kpsep,-\kpsep);
\draw(\ksepX+\kulH+2*\kpsep,-1*\ksepY)++(0,-0.5*\kpsep-0.5*\kpin)node[]{$8$};


\draw(\ksepX,-2*\ksepY) rectangle ++(\kdimX,\kdimY)node[pos=0.5,rectangle,inner sep=0pt,text width=1cm,align=center]{ \text{\RL{دفتر}} $B$};
\draw(\ksepX,-2*\ksepY+\kulV+1.5*\kpsep)++(-0.5*\kpsep,0)--++(-\kpina,0); 
\draw(\ksepX,-2*\ksepY+\kulV+2.5*\kpsep)++(-0.5*\kpsep,0)--++(-\kpina,0); 
\draw(\ksepX,-2*\ksepY+\kulV+1.5*\kpsep)++(-1*\kpsep,-0.5*\kpsep)--++(\kpsep,\kpsep)--++(-\kpsep,\kpsep);
\draw(\ksepX,-2*\ksepY+\kulV+2*\kpsep)++(-0.5*\kpina-0.5*\kpsep,0)node[]{$8$};
\draw(\ksepX+\kdimX,-2*\ksepY+\kulV+2*\kpsep)--++(\kpina,0)node[right]{$CLK$}; 
\draw(\ksepX+\kdimX,-2*\ksepY+\kulV+4*\kpsep)--++(\kpina,0)node[right]{$\overline{L}_B$}; 
\draw(\ksepX+\kdimX,-2*\ksepY+\kulV+2*\kpsep)++(0,-\kmv)--++(-\kmv,\kmv)--++(\kmv,\kmv);
\draw(\ksepX+\kdimX,-2*\ksepY+\kulV+4*\kpsep)++(\knshift,0)node[ocirc]{};

\draw(\ksepX,-3*\ksepY) rectangle ++(\kdimX,\kdimY)node[pos=0.5,rectangle,inner sep=0pt,text width=2cm,align=center]{ \text{\RL{خارجی دفتر}}};
\draw(\ksepX,-3*\ksepY+\kulV+1.5*\kpsep)++(-0.5*\kpsep,0)--++(-\kpina,0); 
\draw(\ksepX,-3*\ksepY+\kulV+2.5*\kpsep)++(-0.5*\kpsep,0)--++(-\kpina,0); 
\draw(\ksepX,-3*\ksepY+\kulV+1.5*\kpsep)++(-1*\kpsep,-0.5*\kpsep)--++(\kpsep,\kpsep)--++(-\kpsep,\kpsep);
\draw(\ksepX,-3*\ksepY+\kulV+2*\kpsep)++(-0.5*\kpina-0.5*\kpsep,0)node[]{$8$};
\draw(\ksepX+\kdimX,-3*\ksepY+\kulV+2*\kpsep)--++(\kpina,0)node[right]{$CLK$}; 
\draw(\ksepX+\kdimX,-3*\ksepY+\kulV+4*\kpsep)--++(\kpina,0)node[right]{$\overline{L}_O$}; 
\draw(\ksepX+\kdimX,-3*\ksepY+\kulV+2*\kpsep)++(0,-\kmv)--++(-\kmv,\kmv)--++(\kmv,\kmv);
\draw(\ksepX,-3*\ksepY)++(\kulH+1.5*\kpsep,0)--++(0,-\kpin);
\draw(\ksepX,-3*\ksepY)++(\kulH+2.5*\kpsep,0)--++(0,-\kpin);
\draw(\ksepX,-3*\ksepY)++(\kulH+2*\kpsep,0)++(0,-0.5*\kpin)node[]{$8$};
\draw(\ksepX,-3*\ksepY)++(\kulH+1.5*\kpsep,0)++(0,-\kpin)++(-0.5*\kpsep,0.5*\kpsep)--++(\kpsep,-\kpsep)--++(\kpsep,\kpsep);
\draw(\ksepX+\kdimX,-3*\ksepY+\kulV+4*\kpsep)++(\knshift,0)node[ocirc]{};

\draw(\ksepX,-4*\ksepY) rectangle ++(\kdimX,\kdimY)node[pos=0.5,rectangle,inner sep=0pt,text width=2cm,align=center]{ \text{\RL{ثنائی نمائش}}};
\end{tikzpicture}
\caption{سادہ ترین کمپیوٹر کی بناوٹ}
\label{شکل_کمپیوٹر_سادہ_ترین}
\end{figure}

سادہ ترین کمپیوٹر کے مختلف حصے واضح کرنے کی غرض سے شکل \حوالہ{شکل_کمپیوٹر_سادہ_ترین}  بنایا گیا ہے۔ اسی لئے تمام    قابو اشارات  ایک ڈبہ جسے \اصطلاح{ قابو مرکز }\فرہنگ{قابو مرکز}\حاشیہب{control unit}\فرہنگ{control unit}کہتے ہیں ، تمام داخلی  اور خارجی ادوار ایک ڈبہ جسے\اصطلاح{ دخول و خروج مرکز }\فرہنگ{دخول و خروج مرکز}\حاشیہب{input-output unit}\فرہنگ{input output unit} کہتے ہیں، وغیرہ،   میں نہیں رکھے گئے ہیں۔

شکل \حوالہ{شکل_کمپیوٹر_سادہ_ترین} میں پیش کئی دفاتر  آپ پہلے سے  جانتے ہیں۔  ہر ڈبے کی مختصر  خصوصیات بیان کرتے ہیں؛ ان پر تفصیلی گفتگو بعد میں کی جائے گی۔

\جزوحصہء{ہدایت گنت کار}
حافظہ  کے شروع میں برنامہ (پروگرام)  رکھا جاتا ہے۔ پہلا ہدایت ثنائی پتہ \عددی{0000} پر، دوسرا ہدایت پتہ \عددی{0001}، اور تیسرا ہدایت \عددی{0010} پر ہو گا۔\اصطلاح{ ہدایت گنت کار }\فرہنگ{ہدایت گنت کار}\حاشیہب{program counter}\فرہنگ{program counter}،  جو قابو مرکز کا حصہ ہے، \عددی{0000} تا \عددی{1111} گردان کرتا ہے۔ اس کا کام حافظہ کو وہ پتہ فراہم کرنا ہے جس سے اگلا ہدایت پڑھ کر عمل میں لایا جائے گا۔ یہ کام درج ذیل طریقے سے سرانجام ہو گا۔

کمپیوٹر کی ہر دوڑ  سے قبل ہدایت گنت کار   \عددی{0000}  کر دیا جاتا ہے۔ جب کمپیوٹر کی دوڑ شروع  ہوتی ہے ہدایت گنت کار   حافظہ کو پتہ \عددی{0000} فراہم کرتا ہے۔ اس کے بعد ہدایت گنت کار   ایک قدم بڑھا کر \عددی{0001}  کر دیا جاتا ہے۔ پہلا ہدایت (مقام \عددی{0000} سے) پڑھ کر اس پر عمل کیا جاتا ہے، جس کے بعد ہدایت گنت کار حافظہ کو پتہ \عددی{0001} بھیجتا ہے اور   ہدایت گنت کار ایک قدم بڑھا کر \عددی{0010} کر دیا جاتا ہے۔ دوسرا ہدایت پڑھنے اور اس پر عمل کرنے کے بعد ہدایت گنت کار حافظہ کو \عددی{0010} پتہ بھیجتا ہے۔ اس طرح، ہدایت گنت کار ہر وقت اگلی  ہدایت  پر نظر جمائے رکھتا ہے۔

گویا ہدایت گنت کار اس شخص کی طرح ہے جو ہدایت کی فہرست  کی طرف اشارہ کرتے ہوئے کہتا ہے یہ کام  پہلے کریں، یہ کام دوسرے نمبر پر کریں، یہ تیسرے نمبر پر کریں، وغیرہ۔ اسی لئے ہدایت گنت کار بعض اوقات \اصطلاح{  اشارہ گر   }\فرہنگ{اشارہ گر}\حاشیہب{pointer}\فرہنگ{pointer} کہلاتا ہے؛ یہ حافظہ میں اس مقام کی طرف اشارہ کرتا ہے جہاں کوئی   اہم معلومات درج ہو گی۔

\جزوحصہء{درآمدی سوئچ  اور  دفتر پتہ}
ہدایت گنت کار کے نیچے درآمدی سوئچ اور دفتر پتہ کا  ڈبہ ہے۔عارضی حافظہ کو  \عددی{4} پتہ اور \عددی{8} مواد   بِٹ فراہم کرنے والا دور،  جو درآمدی سوئچ پر مبنی ہے اور جس کے ذریعہ عارضی حافظہ میں برنامہ بھرا جاتا ہے،  اسی کا حصہ ہے۔یاد رہے کمپیوٹر کی  (با مقصد)دوڑ سے قبل عارضی حافظہ میں برنامہ  لکھنا لازمی ہے۔ یہ دور جو 

حافظے کے پتہ  کا دفتر  (دفتر پتہ  حافظہ)  اس کمپیوٹر کے عارضی حافظے کا حصہ ہے۔ کمپیوٹر  کی دوڑ کے دوران، ہدایت گنت کار  میں موجود پتہ  اس میں  (دفتر پتہ میں)    نقل  کیا جاتا ہے۔ چند لمحوں بعد دفتر پتہ  یہ پتہ عارضی  حافظہ کو فراہم کرتا ہے، جہاں سے   اگلا ہدایت پڑھا جاتا ہے۔

\جزوحصہء{عارضی حافظہ}
کمپیوٹر کی دوڑ سے قبل   \عددی{16\times 8} عارضی  حافظہ  میں  ہدایت اور درکار مواد لکھا جاتا ہے۔ کمپیوٹر کی دوڑ کے دوران، حافظہ کو دفتر پتہ \عددی{4} بِٹ پتہ فراہم کرتا ہے ؛    جہاں سے ہدایت یا مواد  پڑھ  کر \عددی{W} گزرگاہ پر رکھ دیا جاتا ہے جسے  کمپیوٹر کا کوئی دوسرا حصے استعمال کر سکتا ہے۔

\جزوحصہء{دفتر ہدایت}
 قابو مرکز کا ایک حصہ  \اصطلاح{دفتر ہدایت  }\فرہنگ{دفتر ہدایت}\حاشیہب{instruction register}\فرہنگ{instruction register} ہے۔حافظہ سے ہدایت پڑھنے کی خاطر کمپیوٹر   جو عمل  سرانجام دیتا ہے اس کو \اصطلاح{ہدایت پڑھ عمل }\فرہنگ{ہدایت پڑھ عمل}\حاشیہب{memory read operation}\فرہنگ{operation!memory read} کہتے ہیں۔  حافظہ کے   مخاطب  مقام   پر موجود ہدایت (یا مواد) کو  یہ عمل \عددی{W} گزرگاہ پر رکھتا ہے۔ ساتھ ہی   ساعت کے اگلے مثبت کنارے پر  دفتر  ہدایت بھرائی کے لئے تیار کر دیا جاتا ہے۔
 
 دفتر ہدایت    میں موجود معلومات کو دو حصوں میں تقسیم کیا جاتا ہے۔  نچلے   (زیریں) چار بِٹ سہ حالی مخارج ہے جو بوقت ضرورت \عددی{W} گزرگاہ پر ڈال دیا جاتا ہے جبکہ   بالا چار بِٹ  دو حالی مخارج ہے جو سیدھا  قابو و ترتیب کار  کو مہیا کیا جاتا ہے۔
 
 \جزوحصہء{قابو و ترتیب کار}
 کمپیوٹر کی ہر دوڑ سے قبل  ہدایت گنت کار کو \عددی{\overline{CLR}} اور دفتر   ہدایت کو \عددی{CLR}  اشارہ بھیجا جاتا ہے ، جو ہدایت گنت کار   \عددی{0000}     کرتا ہے اور دفتر ہدایت   میں موجود ہدایت  زائل  کرتا  ہے۔

تمام مستحکم کار دفاتر کو ساعتی اشارہ \عددی{CLK} بھیجا جاتا ہے جو کمپیوٹر کے  مختلف اعمال   ہم قدم    کرتے ہوئے یقینی بناتا ہے کہ سب کچھ اپنے اپنے  وقت پر  ہو۔ دوسرے لفظوں میں،   دفاتر کے مابین معلومات کا تبادلہ مشترک ساعت \عددی{CLK} کے مثبت کنارے پر ہو۔ دھیان رہے، ہدایت گنت کار کو \عددی{\overline{CLK}} اشارہ بھی فراہم کیا گیا ہے۔

قابو و ترتیب کار \عددی{12} بِٹ  لفظ خارج کرتا ہے جو باقی کمپیوٹر کو قابو کرتا ہے۔ وہ \عددی{12} برقی تار جن  پر یہ لفظ ترسیل ہوتا ہے \اصطلاح{ قابو گزرگاہ }\فرہنگ{گزرگاہ!قابو}\حاشیہب{control bus}\فرہنگ{bus!control} کہلاتا ہے۔

بارہ بِٹ قابو  لفظ   درج ذیل ہے۔
\begin{align*}
\text{\RL{قابو}}=C_PE_P\overline{L}_M\overline{CE}\,\,\, \overline{L}_I\overline{E}_I\overline{L}_AE_A\,\,S_UE_U\overline{L}_B\overline{L}_O
\end{align*}

 ساعت \عددی{CLK} کے  اگلے   مثبت کنارے پر  دفاتر کا عمل اس لفظ کے تحت ہو گا۔ مثلاً، بلند \عددی{E_P} اور پست \عددی{\overline{L}_M} کی صورت میں ساعت کے اگلے مثبت کنارے پر ہدایت گنت کار کی معلومات   دفتر پتہ میں نقل ہو گی۔ اسی طرح، پست \عددی{\overline{CE}} اور پست \عددی{\overline{L}_A} کی صورت میں  ساعت کے اگلے مثبت کنارے پر  دفتر \عددی{A}   میں عارضی حافظہ کا مخاطب   لفظ نقل ہو گا۔انتقال    مواد  کے اوقات کار  کی ترسیمات پر غور  ( جس سے ہم جان پائیں گے   یہ انتقال  کیسے اور کب ہو ں گے) بعد میں کیا جائے گا ۔

\جزوحصہء{دفتر \عددی{A}}
 کمپیوٹر کی دوڑ کے دوران   حاصل نتائج دفتر \عددی{A} میں ذخیرہ کیے جاتے ہیں۔ شکل \حوالہ{شکل_کمپیوٹر_سادہ_ترین} میں \عددی{A} کے دو مخارج  دکھائے گئے ہیں۔اس کا دو حالی مخارج   سیدھا جمع و منفی کار کو جاتا ہے۔ تین حالی مخارج \عددی{W} گزرگاہ کو جاتا ہے۔ یوں \عددی{A} کا آٹھ بِٹ لفظ  جمع و منفی کار کو    مسلسل فراہم ہو گا؛ یہی لفظ  بلند \عددی{E_A} کی صورت میں \عددی{W} گزرگاہ  پر بھی ڈالا جائے گا۔
 
 \جزوحصہء{جمع و منفی کار}
 یہاں  تکملہ  \عددی{2} کا   جمع و منفی کار مستعمل ہے۔ پست \عددی{S_U}  کی صورت میں شکل \حوالہ{شکل_کمپیوٹر_سادہ_ترین} میں جمع و منفی کار کا مخارج درج ذیل ہو گا۔
 \begin{align*}
 S=A+B
 \end{align*}
 بلند \عددی{S_U} کی صورت میں  جمع و منفی کار درج ذیل دیگا جہاں \عددی{B'} سے مراد \عددی{B} کا  اساس \عددی{2} تکملہ ہے۔(یاد رہے، \عددی{2} کا تکملہ علامت تبدیل کرنے کے مترادف ہے۔)
  \begin{align*}
 S=A+B'
 \end{align*}
 جمع و منفی کار غیر معاصر ہے (یعنی اس کی کارکردگی ساعت پر منحصر نہیں)؛  یوں   جیسے ہی داخلی الفاظ تبدیل ہوں، اس کا مخارج تبدیل ہو گا۔ بلند \عددی{E_U} کی صورت میں یہ مخارج \عددی{W} گزرگاہ پر ڈالا جائے گا۔
 
\جزوحصہء{دفتر \عددی{B}}
دفتر \عددی{B} حسابی   اعمال میں  استعمال کیا جاتا ہے۔ پست \عددی{\overline{L}_B} کی صورت میں ساعت کے مثبت کنارے  پر \عددی{W} گزرگاہ پر موجود لفظ \عددی{B} میں نقل ہو گا۔ دفتر \عددی{B} کا دو حالی مخارج مسلسل جمع و منفی کار کو فراہم کیا جاتا ہے ۔ یہ عدد \عددی{A} میں موجود عدد کے ساتھ جمع یا اس سے منفی ہو گا۔

\جزوحصہء{خارجی دفتر}
کسی بھی مسئلے کو حل کرنے کے بعد حاصل نتیجہ دفتر \عددی{A} میں     ہو گا۔ یہ نتیجہ بیرونی دنیا کو بتانا مقصود ہو گا۔یہ کام \اصطلاح{ خارجی دفتر }\فرہنگ{خارجی دفتر}\حاشیہب{output register}\فرہنگ{register!output} کے سپرد ہے۔ بلند \عددی{E_A} اور پست \عددی{\overline{L}_O} کی صورت میں ساعت کے  اگلے  مثبت کنارے پر \عددی{A} میں موجود معلومات خارجی دفتر میں نقل کی جاتی ہے۔

چونکہ خارجی دفتر کے ذریعہ    مواد کمپیوٹر سے باہر منتقل ہوتا ہے لہٰذا اسے عموماً\اصطلاح{ خارجی  روزن }\فرہنگ{خارجی روزن}\حاشیہب{output port}\فرہنگ{port!output} بھی  کہتے ہیں۔ خارجی روزن  \اصطلاح{ ملاپی ادوار }\فرہنگ{دور!ملاپی}\حاشیہب{interface circuits}\فرہنگ{interface circuit} سے منسلک ہو گا جو بیرونی آلات  مثلاً  \اصطلاح{      پرنٹر}\فرہنگ{پرنٹر}\حاشیہب{printer}\فرہنگ{printer}،   سات کلی نمائشی تختی،  کمپیوٹر کا شیشہ، وغیرہ چلاتے ہیں۔

\جزوحصہء{ثنائی نمائش}
ثنائی نمائش آٹھ \اصطلاح{  نوری ڈایوڈ  }\فرہنگ{نوری ڈایوڈ}\حاشیہب{LED}\فرہنگ{LED} پر مبنی ہے۔خارجی  روزن کے ہر بِٹ کے ساتھ ایک نوری ڈایوڈ منسلک ہے۔ یوں ثنائی نمائش پر خارجی دفتر میں موجود  معلومات    ثنائی روپ میں نظر آئے گی۔

\جزوحصہء{خلاصہ}
اس کمپیوٹر کا قابو مرکز ہدایت گنت کار، ہدایت دفتر، اور قابو و ترتیب کار  (جو قابو لفظ، ساعت \عددی{CLK}، اور زائل  اشارہ \عددی{CLR}  پیدا کرتا ہے) پر مشتمل ہے۔ کمپیوٹر کا\اصطلاح{ حسابی  مرکز }\فرہنگ{حسابی  مرکز}\حاشیہب{arithmetic logic unit, ALU}\فرہنگ{ALU}  دفتر \عددی{A}، دفتر \عددی{B}، اور جمع و منفی کار پر مشتمل ہے۔کمپیوٹر کا حافظہ  دفتر پتہ اور \عددی{16\times 8} عارضی حافظہ پر مشتمل ہے۔  درآمدی  سوئچ، خارجی روزن، اور ثنائی نمائش مل کر   دخول و خروج مرکز دیتے ہیں۔

\حصہ{ہدایات کی فہرست} 
کمپیوٹر کی با مقصد دوڑ سے قبل اس کے حافظہ  میں ہدایات قدم با قدم  بھرنا لازم ہے۔البتہ، ایسا کرنے سے پہلے  آپ کو اس کی  ہدایات  جاننی  ہو گی۔ان ہدایات سے مراد وہ اعمال ہیں جو یہ کمپیوٹر سرانجام دے سکتا ہے۔ اس کمپیوٹر کی ہدایات کی فہرست  پر اب غور کرتے ہیں۔

\جزوحصہء{نقل الف}


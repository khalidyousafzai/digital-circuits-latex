\باب{سادہ ترین کمپیوٹر}
اس باب میں کمپیوٹر کی سادہ ترین ساخت پر غور کیا جائے گا۔ سادہ ہونے کے باوجود اس میں   کئی   اعلٰی  تصورات شامل ہیں۔ اس باب کو پڑھنے اور   سمجھنے  کے بعد آپ جدید کمپیوٹر کی بناؤٹ سمجھ پائیں گے۔

\حصہ{بناؤٹ}
سادہ کمپیوٹر کی  بناوٹ شکل \حوالہ{شکل_کمپیوٹر_سادہ_ترین} میں پیش ہے۔ یہ ایک مکمل کمپیوٹر ہے۔   دفاتر کے    وہ خروج جو آٹھ بِٹ گزر گاہ     سے جڑے ہیں،  \اصطلاح{سہ حالی}\فرہنگ{حال!سہ}\حاشیہب{tri-state}\فرہنگ{state!tri}    ہیں؛  جو  مواد کی  منظم ترسیل ممکن بناتا  ہے۔ آٹھ بِٹ  گزر گاہ سے مراد آٹھ برقی تاریں ہیں جو    ذیلی ادوار (مثلاً حافظہ، جمع و منفی کار)  کے مابین مواد کی ترسیل ممکن بناتے ہیں۔ دفاتر کے باقی خروج \اصطلاح{   دو حالی  }\فرہنگ{حال!دو}\حاشیہب{two-state}\فرہنگ{state!two} ہیں؛ یہ خروج   ان ڈبہ ادوار کو مسلسل   معلومات (مواد، پتہ، شمار وغیرہ)  فراہم کرتے ہیں جن سے یہ منسلک ہیں۔

\begin{figure}
\centering
\begin{tikzpicture}
\pgfmathsetmacro{\klshift}{0.25}
\pgfmathsetmacro{\knshift}{0.07}
\pgfmathsetmacro{\kmv}{0.15}
\pgfmathsetmacro{\knshift}{0.07}
\pgfmathsetmacro{\kpin}{0.50}
\pgfmathsetmacro{\kpina}{1}
\pgfmathsetmacro{\kpsep}{0.40}			%pin to pin distance
\pgfmathsetmacro{\kW}{\kpsep}
\pgfmathsetmacro{\kulV}{0.40}			%edge clearance along vertical edge
\pgfmathsetmacro{\kulH}{0.50}
\pgfmathsetmacro{\kdimX}{2*\kulH+4*\kpsep}
\pgfmathsetmacro{\kdimY}{2*\kulV+4*\kpsep}		%two spaces between 3 pins
\pgfmathsetmacro{\ksepX}{\kdimX+2*\kpina+1*\kpsep+\kW}
\pgfmathsetmacro{\ksepY}{\kdimY+\kpin+0.5*\kpsep}
\draw(0,0)  [thick] rectangle ++(\kdimX,\kdimY)node[pos=0.5,rectangle,inner sep=0pt,text width=1.5cm,align=center]{\RTL{ہدایت گنت کار}};
\draw(0,\kulV+0*\kpsep)--++(-\kpin,0)node[left]{$E_P$};
\draw(0,\kulV+1*\kpsep)--++(-\kpin,0)node[left]{$\overline{CLR}$};
\draw(0,\kulV+2*\kpsep)--++(-\kpin,0)node[left]{$\overline{CLK}$};
\draw(0,\kulV+4*\kpsep)--++(-\kpin,0)node[left]{$C_P$};
\draw(0,\kulV+2*\kpsep-\kmv)--++(\kmv,\kmv)--++(-\kmv,\kmv);
\draw(0,\kulV+2*\kpsep)++(-\knshift,0)node[ocirc]{};
\draw(0,\kulV+1*\kpsep)++(-\knshift,0)node[ocirc]{};
\draw(\kdimX,\kulV+1.5*\kpsep)--++(\kpina,0);
\draw(\kdimX,\kulV+2.5*\kpsep)--++(\kpina,0);
\draw(\kdimX,\kulV+1.5*\kpsep)++(\kpina,0)++(-0.5*\kpsep,-0.5*\kpsep)--++(1*\kpsep,1*\kpsep)--++(-1*\kpsep,1*\kpsep);
\draw(\kdimX,\kulV+2*\kpsep)++(0.5*\kpina,0)node[]{$4$};

\draw(0,-\ksepY)  [thick] rectangle ++(\kdimX,\kdimY)node[pos=0.5,rectangle,inner sep=0pt,text width=2cm,align=center]{\RTL{برنامہ نویس\\ اور\\ دفتر پتہ}};
\draw(0,-\ksepY+\kulV+2*\kpsep)--++(-\kpina,0)node[left]{$CLK$};
\draw(0,-\ksepY+\kulV+2*\kpsep-\kmv)--++(\kmv,\kmv)--++(-\kmv,\kmv);
\draw(0,-\ksepY+\kulV+4*\kpsep)--++(-\kpina,0)node[left]{$\overline{L}_M$};
\draw(0,-\ksepY+\kulV+4*\kpsep)++(-\knshift,0)node[ocirc]{};
\draw(\kulH+0*\kpsep,-\ksepY)--++(0,-\kpin);
\draw(\kulH+1*\kpsep,-\ksepY)--++(0,-\kpin);
\draw(\kulH+0*\kpsep,-\ksepY)++(0,-\kpin)++(-0.5*\kpsep,0.5*\kpsep)--++(\kpsep,-\kpsep)--++(\kpsep,\kpsep);
\draw(\kulH+0.5*\kpsep,-\ksepY)++(0,-0.5*\kpin)node[]{$4$};
\draw(\kulH+3*\kpsep,-\ksepY)--++(0,-\kpin);
\draw(\kulH+4*\kpsep,-\ksepY)--++(0,-\kpin);
\draw(\kulH+3*\kpsep,-\ksepY)++(0,-\kpin)++(-0.5*\kpsep,0.5*\kpsep)--++(\kpsep,-\kpsep)--++(\kpsep,\kpsep);
\draw(\kulH+3.5*\kpsep,-\ksepY)++(0,-0.5*\kpin)node[]{$4$};
\draw(\kdimX,-\ksepY+\kulV+1.5*\kpsep)++(0.5*\kpsep,0)--++(\kpina,0);
\draw(\kdimX,-\ksepY+\kulV+2.5*\kpsep)++(0.5*\kpsep,0)--++(\kpina,0);
\draw(\kdimX,-\ksepY+\kulV+2*\kpsep)--++(1*\kpsep,-1*\kpsep);
\draw(\kdimX,-\ksepY+\kulV+2*\kpsep)--++(1*\kpsep,1*\kpsep);
\draw(\kdimX,-\ksepY+\kulV+2*\kpsep)++(0.5*\kpina,0)node[]{$4$};

\draw(0,-2*\ksepY)  [thick] rectangle ++(\kdimX,\kdimY)node[pos=0.5,rectangle,inner sep=0pt,text width=2.5cm,align=center]{\RTL{$16\times 8$\\ عارضی حافظہ}};
\draw(0,-2*\ksepY+\kulV)--++(-\kpina,0)node[left]{$\overline{CE}$}; 
\draw(0,-2*\ksepY+\kulV)++(-\knshift,0)node[ocirc]{};
\draw(\kdimX,-2*\ksepY+\kulV+1.5*\kpsep)--++(\kpina,0);
\draw(\kdimX,-2*\ksepY+\kulV+2.5*\kpsep)--++(\kpina,0);
\draw(\kdimX,-2*\ksepY+\kulV+1.5*\kpsep)++(\kpina,0)++(-0.5*\kpsep,-0.5*\kpsep)--++(1*\kpsep,1*\kpsep)--++(-1*\kpsep,1*\kpsep);
\draw(\kdimX,-2*\ksepY+\kulV+2*\kpsep)++(0.5*\kpina,0)node[]{$8$};

\draw(0,-3*\ksepY)  [thick] rectangle ++(\kdimX,\kdimY)node[pos=0.5,rectangle,inner sep=0pt,text width=2cm,align=center]{\RTL{دفتر ہدایت}};
\draw(0,-3*\ksepY+\kulV+0*\kpsep)--++(-\kpina,0)node[left]{$\overline{E}_I$};
\draw(0,-3*\ksepY+\kulV+1*\kpsep)--++(-\kpina,0)node[left]{$CLR$};
\draw(0,-3*\ksepY+\kulV+2*\kpsep)--++(-\kpina,0)node[left]{$CLK$};
\draw(0,-3*\ksepY+\kulV+4*\kpsep)--++(-\kpina,0)node[left]{$\overline{L}_I$};
\draw(0,-3*\ksepY+\kulV+0*\kpsep)++(-\knshift,0)node[ocirc]{};
\draw(0,-3*\ksepY+\kulV+4*\kpsep)++(-\knshift,0)node[ocirc]{};
\draw(\kdimX,-3*\ksepY+\kulV+0*\kpsep)--++(\kpina,0);
\draw(\kdimX,-3*\ksepY+\kulV+1*\kpsep)--++(\kpina,0);
\draw(\kdimX,-3*\ksepY+\kulV+0*\kpsep)++(\kpina,0)++(-0.5*\kpsep,-0.5*\kpsep)--++(1*\kpsep,1*\kpsep)--++(-1*\kpsep,1*\kpsep);
\draw(\kdimX,-3*\ksepY+\kulV+0.5*\kpsep)++(0.5*\kpina,0)node[]{$4$};
\draw(\kdimX,-3*\ksepY+\kulV+3*\kpsep)++(0.5*\kpsep,0)--++(\kpina,0);
\draw(\kdimX,-3*\ksepY+\kulV+4*\kpsep)++(0.5*\kpsep,0)--++(\kpina,0);
\draw(\kdimX,-3*\ksepY+\kulV+3.5*\kpsep)--++(1*\kpsep,-1*\kpsep);
\draw(\kdimX,-3*\ksepY+\kulV+3.5*\kpsep)--++(1*\kpsep,1*\kpsep);
\draw(\kdimX,-3*\ksepY+\kulV+3.5*\kpsep)++(0.5*\kpina,0)node[]{$8$};
\draw(\kulH+1.5*\kpsep,-3*\ksepY)--++(0,-\kpin);
\draw(\kulH+2.5*\kpsep,-3*\ksepY)--++(0,-\kpin);
\draw(\kulH+1.5*\kpsep,-3*\ksepY)++(0,-\kpin)++(-0.5*\kpsep,0.5*\kpsep)--++(\kpsep,-\kpsep)--++(\kpsep,\kpsep);
\draw(\kulH+2*\kpsep,-3*\ksepY)++(0,-0.5*\kpin)node[]{$4$};

\draw(0,-4*\ksepY)  [thick] rectangle ++(\kdimX,\kdimY)node[pos=0.5,rectangle,inner sep=0pt,text width=3cm,align=center]{\RTL{قابو  و ترتیب کار}};
\draw[-latex](\kdimX,-4*\ksepY+\kulV+0*\kpsep)--++(\kpina,0)node[right]{$\overline{CLR}$}; 
\draw[-latex](\kdimX,-4*\ksepY+\kulV+1*\kpsep)--++(\kpina,0)node[right]{$CLR$}; 
\draw[-latex](\kdimX,-4*\ksepY+\kulV+2*\kpsep)--++(\kpina,0)node[right]{$\overline{CLK}$}; 
\draw[-latex](\kdimX,-4*\ksepY+\kulV+3*\kpsep)--++(\kpina,0)node[right]{$CLK$}; 
\draw(\kulH+1.5*\kpsep,-4*\ksepY)--++(0,-\kpin);
\draw(\kulH+2.5*\kpsep,-4*\ksepY)--++(0,-\kpin);
\draw(\kulH+1.5*\kpsep,-4*\ksepY)++(0,-\kpin)++(-0.5*\kpsep,0.5*\kpsep)--++(\kpsep,-\kpsep)--++(\kpsep,\kpsep);
\draw(\kulH+2*\kpsep,-4*\ksepY)++(0,-0.5*\kpin)node[]{$12$};
\draw(\kulH+2*\kpsep,-4*\ksepY)++(0,-1*\kpin-\kpsep)node[below]{$C_PE_P\overline{L}_M\overline{CE}\,\,\, \overline{L}_I\overline{E}_I\overline{L}_AE_A\,\,S_UE_U\overline{L}_B\overline{L}_O$};
\draw(\kdimX,-4*\ksepY+\kulV+0*\kpsep)++(\knshift,0)node[ocirc]{};
\draw(\kdimX,-4*\ksepY+\kulV+2*\kpsep)++(\knshift,0)node[ocirc]{};

\draw(\kdimX+\kpina+0.5*\kpsep,\kdimY)  [thick] rectangle ++(\kW,-4*\ksepY+\kpin);
\draw(\kdimX+\kpina+0.5*\kpsep,\kdimY)++(0.5*\kW,0)node[above, text width=1.5cm,align=center]{\RTL{$W$ گزر گاہ}}node[below]{$8$};

\draw(\ksepX,-0*\ksepY)  [thick] rectangle ++(\kdimX,\kdimY)node[pos=0.5,rectangle,inner sep=0pt,text width=1cm,align=center]{\RTL{دفتر\\ $A$}};
\draw(\ksepX,-0*\ksepY+\kulV+0*\kpsep)--++(-\kpina,0); 
\draw(\ksepX,-0*\ksepY+\kulV+1*\kpsep)--++(-\kpina,0); 
\draw(\ksepX,-0*\ksepY+\kulV+0*\kpsep)++(-\kpina,0)++(0.5*\kpsep,-0.5*\kpsep)--++(-\kpsep,\kpsep)--++(\kpsep,\kpsep);
\draw(\ksepX,-0*\ksepY+\kulV+0.5*\kpsep)++(-0.5*\kpina,0)node[]{$8$};
\draw(\ksepX,-0*\ksepY+\kulV+3*\kpsep)++(-0.5*\kpsep,0)--++(-\kpina,0); 
\draw(\ksepX,-0*\ksepY+\kulV+4*\kpsep)++(-0.5*\kpsep,0)--++(-\kpina,0); 
\draw(\ksepX,-0*\ksepY+\kulV+3*\kpsep)++(-1*\kpsep,-0.5*\kpsep)--++(\kpsep,\kpsep)--++(-\kpsep,\kpsep);
\draw(\ksepX,-0*\ksepY+\kulV+3.5*\kpsep)++(-0.5*\kpina,0)node[]{$8$};
\draw(\ksepX+\kulH+1.5*\kpsep,-0*\ksepY)--++(0,-\kpin);
\draw(\ksepX+\kulH+2.5*\kpsep,-0*\ksepY)--++(0,-\kpin);
\draw(\ksepX+\kulH+1.5*\kpsep,-0*\ksepY)++(0,-\kpin)++(-0.5*\kpsep,0.5*\kpsep)--++(\kpsep,-\kpsep)--++(\kpsep,\kpsep);
\draw(\ksepX+\kulH+2*\kpsep,-0*\ksepY)++(0,-0.5*\kpin)node[]{$8$};
\draw(\ksepX+\kdimX,-0*\ksepY+\kulV+0*\kpsep)--++(\kpin,0)node[right]{$E_A$};
\draw(\ksepX+\kdimX,-0*\ksepY+\kulV+2*\kpsep)--++(\kpin,0)node[right]{$CLK$};
\draw(\ksepX+\kdimX,-0*\ksepY+\kulV+4*\kpsep)--++(\kpin,0)node[right]{$\overline{L}_A$};
\draw(\ksepX+\kdimX,-0*\ksepY+\kulV+2*\kpsep)++(0,-\kmv)--++(-\kmv,\kmv)--++(\kmv,\kmv);
\draw(\ksepX+\kdimX,-0*\ksepY+\kulV+4*\kpsep)++(\knshift,0)node[ocirc]{};
\draw(\ksepX,-1*\ksepY)  [thick] rectangle ++(\kdimX,\kdimY)node[pos=0.5,rectangle,inner sep=0pt,text width=1cm,align=center]{\RTL{جمع و منفی کار}};
\draw(\ksepX+\kdimX,-1*\ksepY+\kulV+0*\kpsep)--++(\kpin,0)node[right]{$E_U$};
\draw(\ksepX+\kdimX,-1*\ksepY+\kulV+4*\kpsep)--++(\kpin,0)node[right]{$S_U$};
\draw(\ksepX,-1*\ksepY+\kulV+1.5*\kpsep)--++(-\kpina,0); 
\draw(\ksepX,-1*\ksepY+\kulV+2.5*\kpsep)--++(-\kpina,0); 
\draw(\ksepX,-1*\ksepY+\kulV+1.5*\kpsep)++(-\kpina,0)++(0.5*\kpsep,-0.5*\kpsep)--++(-\kpsep,\kpsep)--++(\kpsep,\kpsep);
\draw(\ksepX,-1*\ksepY+\kulV+2*\kpsep)++(-0.5*\kpina,0)node[]{$8$};
\draw(\ksepX+\kulH+1.5*\kpsep,-1*\ksepY)++(0,-0.5*\kpsep)--++(0,-\kpin);
\draw(\ksepX+\kulH+2.5*\kpsep,-1*\ksepY)++(0,-0.5*\kpsep)--++(0,-\kpin);
\draw(\ksepX+\kulH+1.5*\kpsep,-1*\ksepY)++(-0.5*\kpsep,-\kpsep)--++(\kpsep,\kpsep)--++(\kpsep,-\kpsep);
\draw(\ksepX+\kulH+2*\kpsep,-1*\ksepY)++(0,-0.5*\kpsep-0.5*\kpin)node[]{$8$};


\draw(\ksepX,-2*\ksepY)  [thick] rectangle ++(\kdimX,\kdimY)node[pos=0.5,rectangle,inner sep=0pt,text width=1cm,align=center]{\RTL{دفتر\\ $B$}};
\draw(\ksepX,-2*\ksepY+\kulV+1.5*\kpsep)++(-0.5*\kpsep,0)--++(-\kpina,0); 
\draw(\ksepX,-2*\ksepY+\kulV+2.5*\kpsep)++(-0.5*\kpsep,0)--++(-\kpina,0); 
\draw(\ksepX,-2*\ksepY+\kulV+1.5*\kpsep)++(-1*\kpsep,-0.5*\kpsep)--++(\kpsep,\kpsep)--++(-\kpsep,\kpsep);
\draw(\ksepX,-2*\ksepY+\kulV+2*\kpsep)++(-0.5*\kpina-0.5*\kpsep,0)node[]{$8$};
\draw(\ksepX+\kdimX,-2*\ksepY+\kulV+2*\kpsep)--++(\kpina,0)node[right]{$CLK$}; 
\draw(\ksepX+\kdimX,-2*\ksepY+\kulV+4*\kpsep)--++(\kpina,0)node[right]{$\overline{L}_B$}; 
\draw(\ksepX+\kdimX,-2*\ksepY+\kulV+2*\kpsep)++(0,-\kmv)--++(-\kmv,\kmv)--++(\kmv,\kmv);
\draw(\ksepX+\kdimX,-2*\ksepY+\kulV+4*\kpsep)++(\knshift,0)node[ocirc]{};

\draw(\ksepX,-3*\ksepY)  [thick] rectangle ++(\kdimX,\kdimY)node[pos=0.5,rectangle,inner sep=0pt,text width=2cm,align=center]{\RTL{خارجی دفتر}};
\draw(\ksepX,-3*\ksepY+\kulV+1.5*\kpsep)++(-0.5*\kpsep,0)--++(-\kpina,0); 
\draw(\ksepX,-3*\ksepY+\kulV+2.5*\kpsep)++(-0.5*\kpsep,0)--++(-\kpina,0); 
\draw(\ksepX,-3*\ksepY+\kulV+1.5*\kpsep)++(-1*\kpsep,-0.5*\kpsep)--++(\kpsep,\kpsep)--++(-\kpsep,\kpsep);
\draw(\ksepX,-3*\ksepY+\kulV+2*\kpsep)++(-0.5*\kpina-0.5*\kpsep,0)node[]{$8$};
\draw(\ksepX+\kdimX,-3*\ksepY+\kulV+2*\kpsep)--++(\kpina,0)node[right]{$CLK$}; 
\draw(\ksepX+\kdimX,-3*\ksepY+\kulV+4*\kpsep)--++(\kpina,0)node[right]{$\overline{L}_O$}; 
\draw(\ksepX+\kdimX,-3*\ksepY+\kulV+2*\kpsep)++(0,-\kmv)--++(-\kmv,\kmv)--++(\kmv,\kmv);
\draw(\ksepX,-3*\ksepY)++(\kulH+1.5*\kpsep,0)--++(0,-\kpin);
\draw(\ksepX,-3*\ksepY)++(\kulH+2.5*\kpsep,0)--++(0,-\kpin);
\draw(\ksepX,-3*\ksepY)++(\kulH+2*\kpsep,0)++(0,-0.5*\kpin)node[]{$8$};
\draw(\ksepX,-3*\ksepY)++(\kulH+1.5*\kpsep,0)++(0,-\kpin)++(-0.5*\kpsep,0.5*\kpsep)--++(\kpsep,-\kpsep)--++(\kpsep,\kpsep);
\draw(\ksepX+\kdimX,-3*\ksepY+\kulV+4*\kpsep)++(\knshift,0)node[ocirc]{};

\draw(\ksepX,-4*\ksepY)  [thick] rectangle ++(\kdimX,\kdimY)node[pos=0.5,rectangle,inner sep=0pt,text width=2cm,align=center]{\RTL{ثنائی نمائشی تختی}};
\end{tikzpicture}
\caption{سادہ ترین کمپیوٹر کی بناوٹ}
\label{شکل_کمپیوٹر_سادہ_ترین}
\end{figure}

سادہ ترین کمپیوٹر کے مختلف حصے واضح کرنے کی غرض سے شکل \حوالہ{شکل_کمپیوٹر_سادہ_ترین}  بنایا گیا ہے۔ اسی لئے تمام    قابو اشارات  ایک ڈبہ جسے \اصطلاح{ قابو مرکز }\فرہنگ{قابو مرکز}\حاشیہب{control unit}\فرہنگ{control unit}کہتے ہیں ، تمام داخلی  اور خارجی ادوار ایک ڈبہ جسے\اصطلاح{ دخول و خروج مرکز }\فرہنگ{دخول و خروج مرکز}\حاشیہب{input-output unit}\فرہنگ{input output unit} کہتے ہیں، وغیرہ،   میں نہیں رکھے گئے ہیں۔

شکل \حوالہ{شکل_کمپیوٹر_سادہ_ترین} میں پیش کئی دفاتر  آپ پہلے سے  جانتے ہیں۔  ہر ڈبے کی مختصر  خصوصیات بیان کرتے ہیں؛ ان پر تفصیلی گفتگو بعد میں کی جائے گی۔

\جزوحصہء{ہدایت گنت کار}
حافظہ  کے شروع میں\اصطلاح{ برنامہ }\فرہنگ{برنامہ}\حاشیہب{program}\فرہنگ{program} (پروگرام)  رکھا جاتا ہے۔ پہلا ہدایت ثنائی پتہ \عددی{0000} پر، دوسرا ہدایت پتہ \عددی{0001}، اور تیسرا ہدایت \عددی{0010} پر ہو گا۔\اصطلاح{ ہدایت گنت کار }\فرہنگ{ہدایت گنت کار}\حاشیہب{program counter}\فرہنگ{program counter}،  جو قابو مرکز کا حصہ ہے، \عددی{0000} تا \عددی{1111} گردان کرتا ہے۔ اس کا کام حافظہ کو وہ پتہ فراہم کرنا ہے جس سے اگلا ہدایت پڑھ کر عمل میں لایا جائے گا۔ یہ کام درج ذیل طریقے سے سرانجام ہو گا۔

کمپیوٹر کی ہر دوڑ  سے قبل ہدایت گنت کار   \عددی{0000}  کر دیا جاتا ہے۔ جب کمپیوٹر کی دوڑ شروع  ہوتی ہے ہدایت گنت کار   حافظہ کو پتہ \عددی{0000} فراہم کرتا ہے۔ اس کے بعد ہدایت گنت کار   ایک قدم بڑھا کر \عددی{0001}  کر دیا جاتا ہے۔ پہلا ہدایت (مقام \عددی{0000} سے) پڑھ کر اس پر عمل کیا جاتا ہے، جس کے بعد ہدایت گنت کار حافظہ کو پتہ \عددی{0001} بھیجتا ہے اور   ہدایت گنت کار ایک قدم بڑھا کر \عددی{0010} کر دیا جاتا ہے۔ دوسرا ہدایت پڑھنے اور اس پر عمل کرنے کے بعد ہدایت گنت کار حافظہ کو \عددی{0010} پتہ بھیجتا ہے۔ اس طرح، ہدایت گنت کار ہر وقت اگلی  ہدایت  پر نظر جمائے رکھتا ہے۔

گویا ہدایت گنت کار اس شخص کی طرح ہے جو ہدایت کی فہرست  کی طرف اشارہ کرتے ہوئے کہتا ہے یہ کام  پہلے کریں، یہ کام دوسرے نمبر پر کریں، یہ تیسرے نمبر پر کریں، وغیرہ۔ اسی لئے ہدایت گنت کار بعض اوقات \اصطلاح{  اشارہ گر   }\فرہنگ{اشارہ گر}\حاشیہب{pointer}\فرہنگ{pointer} کہلاتا ہے؛ یہ حافظہ میں اس مقام کی طرف اشارہ کرتا ہے جہاں کوئی   اہم معلومات درج ہو گی۔

\جزوحصہء{برنامہ نویس   اور  دفتر پتہ}
ہدایت گنت کار کے نیچے برنامہ نویس   اور دفتر پتہ کا  ڈبہ ہے۔ شکل \حوالہ{شکل_کمپیوٹر_برنامہ_نویسی} میں برنامہ نویس پیش ہے (     صفحہ \حوالہصفحہ{شکل_حافظہ_میں_مواد_کی_لکھائی} پر شکل \حوالہ{شکل_حافظہ_میں_مواد_کی_لکھائی} سمجھیں)   جس  کے ذریعہ سوئچوں  کی مدد سے   عارضی حافظہ کو  \عددی{4} پتہ اور \عددی{8} مواد   بِٹ فراہم کر  کے بھرا جاتا ہے۔ یاد رہے کمپیوٹر کی  (با مقصد)دوڑ سے قبل عارضی حافظہ میں برنامہ  لکھنا لازمی ہے۔ یہ دور جو 

حافظے کے پتہ  کا دفتر  (دفتر پتہ  )  اس کمپیوٹر کے عارضی حافظے کا حصہ ہے۔ کمپیوٹر  کی دوڑ کے دوران، ہدایت گنت کار  میں موجود پتہ  اس   (دفتر پتہ)     میں نقل  کیا جاتا ہے۔  دفتر پتہ   چند لمحوں بعد یہ پتہ عارضی  حافظہ کو فراہم کرتا ہے، جہاں سے   اگلی ہدایت پڑھی جاتی ہے۔

\begin{figure}
\centering
 \ctikzset{bipoles/resistor/height=0.15}
 \ctikzset{bipoles/resistor/width=0.4}
  \ctikzset{bipoles/nos/height=0.15}
 \ctikzset{bipoles/nos/width=0.4}
\begin{circuitikz}
\pgfmathsetmacro{\ksepX}{6}
\pgfmathsetmacro{\ksepY}{8}
\pgfmathsetmacro{\kr}{1}
\pgfmathsetmacro{\knshift}{0.07}
\pgfmathsetmacro{\kpsep}{0.50}
\pgfmathsetmacro{\kpsepr}{0.25}
\pgfmathsetmacro{\kpin}{0.5}
\pgfmathsetmacro{\kpina}{2.5}
\pgfmathsetmacro{\kpinb}{\kpsep+2*\kpsepr}%{4}
\pgfmathsetmacro{\kul}{0.50}
\pgfmathsetmacro{\kmv}{0.15}
\pgfmathsetmacro{\kxdim}{2*\kul+3*\kpsep}
\pgfmathsetmacro{\kydim}{2*\kul+5*\kpsep}
\draw[thick](0,0)rectangle++(\kxdim,\kydim)node[pos=0.5,below]{$74189$}node[pos=0.5,above]{\text{\RL{حافظہ}}};
\draw(1*\kul+0*\kpsep,\kydim)node[below]{$A_3$}node[shift={(-1.5ex,2ex)}]{$13$}--++(0,\kpina)to [R]++(0,\kr)coordinate(kpA);
\draw(1*\kul+1*\kpsep,\kydim)node[below]{$A_2$}node[shift={(-1.5ex,2ex)}]{$14$}--++(0,\kpina)to [R]++(0,\kr);
\draw(1*\kul+2*\kpsep,\kydim)node[below]{$A_1$}node[shift={(-1.5ex,2ex)}]{$15$}--++(0,\kpina)to [R]++(0,\kr);
\draw(1*\kul+3*\kpsep,\kydim)node[below]{$A_0$}node[shift={(-1ex,2ex)}]{$1$}--++(0,\kpina)to [R]++(0,\kr)coordinate(kpB);
\draw(kpA)--(kpB)--++(\kpin,0)node[right]{$\SI{+5}{\volt}$};
\draw(1*\kul+0*\kpsep,\kydim)++(0,3*\kpsep+3*\kpsepr)coordinate(kaa)--++(-1*\kpsep,0) to [nos,o-o,invert,mirror]++(-\kpin,0)--++(-0.5*\kpsep,0)coordinate(gndA)node[left]{$a_3$};
\draw(1*\kul+1*\kpsep,\kydim)++(0,3*\kpsep+2*\kpsepr)coordinate(kab)--++(-2*\kpsep,0) to [nos,o-o,invert,mirror]++(-\kpin,0)--++(-0.5*\kpsep,0)node[left]{$a_2$};
\draw(1*\kul+2*\kpsep,\kydim)++(0,3*\kpsep+1*\kpsepr)coordinate(kac)--++(-3*\kpsep,0) to [nos,o-o,invert,mirror]++(-\kpin,0)--++(-0.5*\kpsep,0)node[left]{$a_1$};
\draw(1*\kul+3*\kpsep,\kydim)++(0,3*\kpsep+0*\kpsepr)coordinate(kad)--++(-4*\kpsep,0) to [nos,o-o,invert,mirror]++(-\kpin,0)--++(-0.5*\kpsep,0)coordinate(gndB)node[left]{$a_0$};
\draw(gndA)--(gndB)node [ground]{};
\draw(0,\kul+1*\kpsep)node[right]{$D_4$}node[left,xshift=-0.5ex,yshift=1.5ex]{$4$}--++(-\kpinb-\kpsepr,0)--++(0,-2*\kpsep) to [nos,o-o,invert,mirror]++(0,-\kpin)--++(0,-\kpin)coordinate(kA);%
\draw(0,\kul+2*\kpsep)node[right]{$D_5$}node[left,xshift=-0.5ex,yshift=1.5ex]{$6$}--++(-\kpinb-2*\kpsepr,0)--++(0,-3*\kpsep) to [nos,o-o,invert,mirror]++(0,-\kpin)--++(0,-\kpin);
\draw(0,\kul+3*\kpsep)node[right]{$D_6$}node[left,xshift=-0.5ex,yshift=1.5ex]{$10$}--++(-\kpinb-3*\kpsepr,0)--++(0,-4*\kpsep) to [nos,o-o,invert,mirror]++(0,-\kpin)--++(0,-\kpin);
\draw(0,\kul+4*\kpsep)node[right]{$D_7$}node[left,xshift=-0.5ex,yshift=1.5ex]{$12$}--++(-\kpinb-4*\kpsepr,0)--++(0,-5*\kpsep) to [nos,o-o,invert,mirror]++(0,-\kpin)node[above left]{$d_7$}--++(0,-\kpin)coordinate(kB);
\draw(kA)--(kB)node[ground]{};
\draw(-\kpsep-1*\kpsepr,\kul+0*\kpsep)--++(0,4*\kpsep+\kul)to [R]++(0,\kr)coordinate(kpC);
\draw(-\kpsep-3*\kpsepr,\kul+1*\kpsep)--++(0,3*\kpsep+\kul)to [R]++(0,\kr);
\draw(-\kpsep-4*\kpsepr,\kul+2*\kpsep)--++(0,2*\kpsep+\kul)to [R]++(0,\kr);
\draw(-\kpsep-5*\kpsepr,\kul+3*\kpsep)--++(0,1*\kpsep+\kul)to [R]++(0,\kr);
\draw(-\kpsep-6*\kpsepr,\kul+4*\kpsep)--++(0,0*\kpsep+\kul)to [R]++(0,\kr)coordinate(kpD);
\draw(kpC)--(kpD)--++(-0.5*\kpin,0)node[left]{$\SI{5}{\volt}$};
\draw(0,\kul+0*\kpsep)node[right]{$\overline{\text{\RL{لکھ}}}$}node[left,xshift=-0.5ex,yshift=1.5ex]{$3$}--++(-\kpinb+\kpsepr,0)--++(0,-5.5*\kpsep)coordinate(kkWW) to [nos,o-o,invert,mirror]++(0,-\kpin)node[right]{$S_1$}--++(0,-\kpin)node[ground]{};
\draw(0,\kul+0*\kpsep)++(-\knshift,0)node[ocirc]{};
\draw(1.5*\kul,0)node[below right]{$16$}--++(0,-\kpin)--++(-1*\kpin,0)node[left]{$\SI{5}{\volt}$} (1.5*\kul+\kpsep,0)node[below right]{$8$}--++(0,-\kpin)node[ground]{};
\draw(1*\kul+3*\kpsep,0)node[below right]{$2$}node[above]{$\overline{\text{\RL{بیدار}}}$}
--++(0,-0.5*\kpin)coordinate(kCE);
\draw(1*\kul+3*\kpsep,0)++(0,-\knshift)node[ocirc]{};
\draw(\kxdim,\kul+1*\kpsep)node[right,xshift=0.5ex,yshift=1.5ex]{$5$}--++(1*\kpin,0)node[right]{$\overline{O}_4$};
\draw(\kxdim,\kul+2*\kpsep)node[right,xshift=0.5ex,yshift=1.5ex]{$7$}--++(1*\kpin,0)node[right]{$\overline{O}_5$};
\draw(\kxdim,\kul+3*\kpsep)node[right,xshift=0.5ex,yshift=1.5ex]{$9$}--++(1*\kpin,0)node[right]{$\overline{O}_6$};
\draw(\kxdim,\kul+4*\kpsep)node[right,xshift=0.5ex,yshift=1.5ex]{$11$}--++(1*\kpin,0)node[right]{$\overline{O}_7$};
\draw(\kxdim,\kul+1*\kpsep)++(\knshift,0)node[ocirc]{};
\draw(\kxdim,\kul+2*\kpsep)++(\knshift,0)node[ocirc]{};
\draw(\kxdim,\kul+3*\kpsep)++(\knshift,0)node[ocirc]{};
\draw(\kxdim,\kul+4*\kpsep)++(\knshift,0)node[ocirc]{};

\draw[thick](\ksepX,0*\ksepY)rectangle++(\kxdim,\kydim)node[pos=0.5,below]{$74189$}node[pos=0.5,above]{\text{\RL{حافظہ}}};
\draw(1*\kul+0*\kpsep+\ksepX,\kydim+0*\ksepY)node[below]{$A_3$}node[shift={(-1.5ex,2ex)}]{$13$}|-(kaa)node[draw,circle,fill=black,inner sep=1.5pt]{};
\draw(1*\kul+1*\kpsep+\ksepX,\kydim+0*\ksepY)node[below]{$A_2$}node[shift={(-1.5ex,2ex)}]{$14$}|-(kab)node[draw,circle,fill=black,inner sep=1.5pt]{};
\draw(1*\kul+2*\kpsep+\ksepX,\kydim+0*\ksepY)node[below]{$A_1$}node[shift={(-1.5ex,2ex)}]{$15$}|-(kac)node[draw,circle,fill=black,inner sep=1.5pt]{};
\draw(1*\kul+3*\kpsep+\ksepX,\kydim+0*\ksepY)node[below]{$A_0$}node[shift={(-1ex,2ex)}]{$1$}|-(kad)node[draw,circle,fill=black,inner sep=1.5pt]{};
\draw(\ksepX,\kul+1*\kpsep+0*\ksepY)node[right]{$D_0$}node[left,xshift=-0.5ex,yshift=1.5ex]{$4$}--++(-\kpinb+\kpsepr,0)--++(0,-2*\kpsep) to [nos,o-o,invert,mirror]++(0,-\kpin)--++(0,-\kpin)coordinate(kA);%
\draw(\ksepX,\kul+2*\kpsep+0*\ksepY)node[right]{$D_1$}node[left,xshift=-0.5ex,yshift=1.5ex]{$6$}--++(-\kpinb-0*\kpsepr,0)--++(0,-3*\kpsep) to [nos,o-o,invert,mirror]++(0,-\kpin)--++(0,-\kpin);
\draw(\ksepX,\kul+3*\kpsep+0*\ksepY)node[right]{$D_2$}node[left,xshift=-0.5ex,yshift=1.5ex]{$10$}--++(-\kpinb-1*\kpsepr,0)--++(0,-4*\kpsep) to [nos,o-o,invert,mirror]++(0,-\kpin)--++(0,-\kpin);
\draw(\ksepX,\kul+4*\kpsep+0*\ksepY)node[right]{$D_3$}node[left,xshift=-0.5ex,yshift=1.5ex]{$12$}--++(-\kpinb-2*\kpsepr,0)--++(0,-5*\kpsep) to [nos,o-o,invert,mirror]++(0,-\kpin)node[above left]{$d_3$}--++(0,-\kpin)coordinate(kB);
\draw(\ksepX-\kpsep-1*\kpsepr,\kul+1*\kpsep+0*\ksepY)--++(0,3*\kpsep+\kul)to [R]++(0,\kr)coordinate(kpC);
\draw(\ksepX-\kpsep-2*\kpsepr,\kul+2*\kpsep+0*\ksepY)--++(0,2*\kpsep+\kul)to [R]++(0,\kr);
\draw(\ksepX-\kpsep-3*\kpsepr,\kul+3*\kpsep+0*\ksepY)--++(0,1*\kpsep+\kul)to [R]++(0,\kr);
\draw(\ksepX-\kpsep-4*\kpsepr,\kul+4*\kpsep+0*\ksepY)--++(0,0*\kpsep+\kul)to [R]++(0,\kr)coordinate(kpD);
\draw(kpC)--(kpD)--++(-0.5*\kpin,0)node[left]{$\SI{5}{\volt}$};
\draw(\ksepX,\kul+0*\kpsep+0*\ksepY)node[right]{$\overline{\text{\RL{لکھ}}}$}node[left,xshift=-0.5ex,yshift=1.5ex]{$3$}coordinate(kwr);%--++(-\kpinb+\kpsepr,0)--++(0,-3*\kpsep) to [nos,o-o,invert,mirror]++(0,-\kpin)node[right]{$S_1$}--++(0,-\kpin)coordinate(kA);
\draw(kwr)--++(-\kpin,0)|-($(kkWW)+(0,0.3)$);
\draw(\ksepX,\kul+0*\kpsep+0*\ksepY)++(-\knshift,0)node[ocirc]{};
\draw(kA)--(kB)node[ground]{};
\draw(1.5*\kul+\ksepX,0*\ksepY)node[below right]{$16$}--++(0,-\kpin)--++(-0.5*\kpin,0)node[left]{$\SI{5}{\volt}$} (1.5*\kul+\kpsep+\ksepX,0)node[below right]{$8$}--++(0,-\kpin)node[ground]{};
\draw(1*\kul+3*\kpsep+\ksepX,0*\ksepY)node[below right]{$2$}node[above]{$\overline{\text{\RL{بیدار}}}$}
--++(0,-5*\kpin)node[spdt,xscale=-1,rotate=-90,anchor=in](sw){};
\draw(1*\kul+3*\kpsep+\ksepX,0*\ksepY)++(0,-\knshift)node[ocirc]{};
\draw(sw.out 1)node[ground]{} (sw.out 2)node[above right,yshift=3ex]{$S_2$}--++(2*\kpin,0)node[right]{$\overline{E}_R$};
\draw(sw)++(0,-1.5cm)node[]{\text{\RL{دوڑ/برنامہ نویسی}}};
\draw(sw.in)++(0,0.2)-|(kCE);
\draw(\kxdim+\ksepX,\kul+1*\kpsep+0*\ksepY)node[right,xshift=0.5ex,yshift=1.5ex]{$5$}--++(1*\kpin,0)node[right]{$\overline{O}_0$};
\draw(\kxdim+\ksepX,\kul+2*\kpsep+0*\ksepY)node[right,xshift=0.5ex,yshift=1.5ex]{$7$}--++(1*\kpin,0)node[right]{$\overline{O}_1$};
\draw(\kxdim+\ksepX,\kul+3*\kpsep+0*\ksepY)node[right,xshift=0.5ex,yshift=1.5ex]{$9$}--++(1*\kpin,0)node[right]{$\overline{O}_2$};
\draw(\kxdim+\ksepX,\kul+4*\kpsep+0*\ksepY)node[right,xshift=0.5ex,yshift=1.5ex]{$11$}--++(1*\kpin,0)node[right]{$\overline{O}_3$};
\draw(\kxdim+\ksepX,\kul+1*\kpsep+0*\ksepY)++(\knshift,0)node[ocirc]{};
\draw(\kxdim+\ksepX,\kul+2*\kpsep+0*\ksepY)++(\knshift,0)node[ocirc]{};
\draw(\kxdim+\ksepX,\kul+3*\kpsep+0*\ksepY)++(\knshift,0)node[ocirc]{};
\draw(\kxdim+\ksepX,\kul+4*\kpsep+0*\ksepY)++(\knshift,0)node[ocirc]{};
\end{circuitikz}
\caption{برنامہ نویس}
\label{شکل_کمپیوٹر_برنامہ_نویسی}
\end{figure}

\جزوحصہء{عارضی حافظہ}
کمپیوٹر کی دوڑ سے قبل   \عددی{16\times 8} عارضی  حافظہ  میں  ہدایت اور درکار مواد لکھا جاتا ہے۔ کمپیوٹر کی دوڑ کے دوران، حافظہ کو دفتر پتہ \عددی{4} بِٹ پتہ فراہم کرتا ہے ؛    جہاں سے ہدایت یا مواد  پڑھ  کر \عددی{W} گزرگاہ پر رکھ دیا جاتا ہے جسے  کمپیوٹر کا کوئی دوسرا حصے استعمال کر سکتا ہے۔عارضی حافظہ کے مخارج \عددی{\overline{O}_0} تا \عددی{\overline{O}_7} آٹھ برقی تاروں کے ذریعہ کمپیوٹر کے  باقی  حصوں کے ساتھ جڑا ہے۔ ان آٹھ تاروں کو  \عددی{W} گزرگاہ کہتے ہیں۔

\جزوحصہء{دفتر ہدایت}
 قابو مرکز کا ایک حصہ  \اصطلاح{دفتر ہدایت  }\فرہنگ{دفتر ہدایت}\حاشیہب{instruction register}\فرہنگ{instruction register} ہے۔حافظہ سے ہدایت پڑھنے کی خاطر کمپیوٹر   جو عمل  سرانجام دیتا ہے اس کو \اصطلاح{ہدایت پڑھ عمل }\فرہنگ{ہدایت پڑھ عمل}\حاشیہب{memory read operation}\فرہنگ{operation!memory read} کہتے ہیں۔  حافظہ کے   مخاطب  مقام   پر موجود ہدایت (یا مواد) کو  یہ عمل \عددی{W} گزرگاہ پر رکھتا ہے۔ ساتھ ہی   ساعت کے اگلے مثبت کنارے پر  دفتر  ہدایت بھرائی کے لئے تیار کر دیا جاتا ہے۔
 
 دفتر ہدایت    میں موجود معلومات کو دو حصوں میں تقسیم کیا جاتا ہے۔  نچلے   (زیریں) چار بِٹ سہ حالی مخارج ہے جو بوقت ضرورت \عددی{W} گزرگاہ پر ڈال دیا جاتا ہے جبکہ   بالا چار بِٹ  دو حالی مخارج ہے جو سیدھا  قابو و ترتیب کار  کو مہیا کیا جاتا ہے۔
 
 \جزوحصہء{قابو و ترتیب کار}
 کمپیوٹر کی ہر دوڑ سے قبل  ہدایت گنت کار کو \عددی{\overline{CLR}} اور دفتر   ہدایت کو \عددی{CLR}  اشارہ بھیجا جاتا ہے ، جو ہدایت گنت کار   \عددی{0000}     کرتا ہے اور دفتر ہدایت   میں موجود ہدایت  زائل  کرتا  ہے۔

تمام مستحکم کار دفاتر کو ساعتی اشارہ \عددی{CLK} بھیجا جاتا ہے جو کمپیوٹر کے  مختلف اعمال   ہم قدم    کرتے ہوئے یقینی بناتا ہے کہ سب کچھ اپنے اپنے  وقت پر  ہو۔ دوسرے لفظوں میں،   دفاتر کے مابین معلومات کا تبادلہ مشترک ساعت \عددی{CLK} کے مثبت کنارے پر ہو۔ دھیان رہے، ہدایت گنت کار کو \عددی{\overline{CLK}} اشارہ بھی فراہم کیا گیا ہے۔

قابو و ترتیب کار \عددی{12} بِٹ  لفظ خارج کرتا ہے جو باقی کمپیوٹر کو قابو کرتا ہے۔ وہ \عددی{12} برقی تار جن  پر یہ لفظ ترسیل ہوتا ہے \اصطلاح{ قابو گزرگاہ }\فرہنگ{گزرگاہ!قابو}\حاشیہب{control bus}\فرہنگ{bus!control} کہلاتا ہے۔

بارہ بِٹ قابو  لفظ   درج ذیل ہے۔
\begin{align*}
\text{\RL{قابو}}=C_PE_P\overline{L}_M\overline{CE}\,\,\, \overline{L}_I\overline{E}_I\overline{L}_AE_A\,\,\,S_UE_U\overline{L}_B\overline{L}_O
\end{align*}

 ساعت \عددی{CLK} کے  اگلے   مثبت کنارے پر  دفاتر کا عمل اس لفظ کے تحت ہو گا۔ مثلاً، بلند \عددی{E_P} اور پست \عددی{\overline{L}_M} کی صورت میں ساعت کے اگلے مثبت کنارے پر ہدایت گنت کار کی معلومات   دفتر پتہ میں نقل ہو گی۔ اسی طرح، پست \عددی{\overline{CE}} اور پست \عددی{\overline{L}_A} کی صورت میں  ساعت کے اگلے مثبت کنارے پر  دفتر \عددی{A}   میں عارضی حافظہ کا مخاطب   لفظ نقل ہو گا۔انتقال    مواد  کی وقتیہ  ترسیمات پر غور  ( جس سے ہم جان پائیں گے   یہ انتقال  کیسے اور کب ہو ں گے) بعد میں کیا جائے گا ۔

\جزوحصہء{دفتر \عددی{A}}
 کمپیوٹر کی دوڑ کے دوران   حاصل نتائج دفتر \عددی{A} میں ذخیرہ کیے جاتے ہیں۔ شکل \حوالہ{شکل_کمپیوٹر_سادہ_ترین} میں \عددی{A} کے دو مخارج  دکھائے گئے ہیں۔اس کا دو حالی مخارج   سیدھا جمع و منفی کار کو جاتا ہے۔ تین حالی مخارج \عددی{W} گزرگاہ کو جاتا ہے۔ یوں \عددی{A} کا آٹھ بِٹ لفظ  جمع و منفی کار کو    مسلسل فراہم ہو گا؛ یہی لفظ  بلند \عددی{E_A} کی صورت میں \عددی{W} گزرگاہ  پر بھی ڈالا جائے گا۔
 
 \جزوحصہء{جمع و منفی کار}
 یہاں  تکملہ  \عددی{2} کا   جمع و منفی کار مستعمل ہے۔ پست \عددی{S_U}  کی صورت میں شکل \حوالہ{شکل_کمپیوٹر_سادہ_ترین} میں جمع و منفی کار کا مخارج درج ذیل ہو گا۔
 \begin{align*}
 S=A+B
 \end{align*}
 بلند \عددی{S_U} کی صورت میں  جمع و منفی کار درج ذیل دیگا جہاں \عددی{B'} سے مراد \عددی{B} کا  اساس \عددی{2} تکملہ ہے۔(یاد رہے، \عددی{2} کا تکملہ علامت تبدیل کرنے کے مترادف ہے۔)
  \begin{align*}
 S=A+B'
 \end{align*}
 جمع و منفی کار غیر معاصر ہے (یعنی اس کی کارکردگی ساعت پر منحصر نہیں)؛  یوں   جیسے ہی داخلی الفاظ تبدیل ہوں، اس کا مخارج تبدیل ہو گا۔ بلند \عددی{E_U} کی صورت میں یہ مخارج \عددی{W} گزرگاہ پر ڈالا جائے گا۔
 
\جزوحصہء{دفتر \عددی{B}}
دفتر \عددی{B} حسابی   اعمال میں  استعمال کیا جاتا ہے۔ پست \عددی{\overline{L}_B} کی صورت میں ساعت کے مثبت کنارے  پر \عددی{W} گزرگاہ پر موجود لفظ \عددی{B} میں نقل ہو گا۔ دفتر \عددی{B} کا دو حالی مخارج مسلسل جمع و منفی کار کو فراہم کیا جاتا ہے ۔ یہ عدد \عددی{A} میں موجود عدد کے ساتھ جمع یا اس سے منفی ہو گا۔

\جزوحصہء{خارجی دفتر}
کسی بھی مسئلے کو حل کرنے کے بعد حاصل نتیجہ دفتر \عددی{A} میں     ہو گا۔ یہ نتیجہ بیرونی دنیا کو بتانا مقصود ہو گا۔یہ کام \اصطلاح{ خارجی دفتر }\فرہنگ{خارجی دفتر}\حاشیہب{output register}\فرہنگ{register!output} کے سپرد ہے۔ بلند \عددی{E_A} اور پست \عددی{\overline{L}_O} کی صورت میں ساعت کے  اگلے  مثبت کنارے پر \عددی{A} میں موجود معلومات خارجی دفتر میں نقل کی جاتی ہے۔

چونکہ خارجی دفتر کے ذریعہ    مواد کمپیوٹر سے باہر منتقل ہوتا ہے لہٰذا اسے عموماً\اصطلاح{ خارجی  روزن }\فرہنگ{خارجی روزن}\حاشیہب{output port}\فرہنگ{port!output} بھی  کہتے ہیں۔ خارجی روزن  \اصطلاح{ ملاپی ادوار }\فرہنگ{دور!ملاپی}\حاشیہب{interface circuits}\فرہنگ{interface circuit} سے منسلک ہو گا جو بیرونی آلات  مثلاً  \اصطلاح{      پرنٹر}\فرہنگ{پرنٹر}\حاشیہب{printer}\فرہنگ{printer}،   سات کلی نمائشی تختی،  کمپیوٹر کا شیشہ، وغیرہ چلاتے ہیں۔

\جزوحصہء{ثنائی نمائشی تختی}
ثنائی نمائشی تختی  آٹھ \اصطلاح{  نوری ڈایوڈ  }\فرہنگ{نوری ڈایوڈ}\حاشیہب{LED}\فرہنگ{LED} پر مبنی ہے۔خارجی  روزن کے ہر بِٹ کے ساتھ ایک نوری ڈایوڈ منسلک ہے۔ یوں ثنائی نمائشی تختی  پر خارجی دفتر میں موجود  معلومات    ثنائی روپ میں نظر آئے گی۔

\جزوحصہء{خلاصہ}
اس کمپیوٹر کا قابو مرکز ہدایت گنت کار، ہدایت دفتر، اور قابو و ترتیب کار  (جو قابو لفظ، ساعت \عددی{CLK}، اور زائل  اشارہ \عددی{CLR}  پیدا کرتا ہے) پر مشتمل ہے۔ کمپیوٹر کا\اصطلاح{ حسابی  مرکز }\فرہنگ{حسابی  مرکز}\حاشیہب{arithmetic logic unit, ALU}\فرہنگ{ALU}  دفتر \عددی{A}، دفتر \عددی{B}، اور جمع و منفی کار پر مشتمل ہے۔کمپیوٹر کا حافظہ  دفتر پتہ اور \عددی{16\times 8} عارضی حافظہ پر مشتمل ہے۔  درآمدی  سوئچ، خارجی روزن، اور ثنائی نمائشی تختی  مل کر   دخول و خروج مرکز دیتے ہیں۔

\حصہ{ہدایات کی فہرست} 
کمپیوٹر کی با مقصد دوڑ سے قبل اس کے حافظہ  میں ہدایات قدم با قدم  بھرنا لازم ہے۔البتہ، ایسا کرنے سے پہلے  آپ کو یہ ہدایات  جاننی  ہو گی۔ان ہدایات سے مراد وہ اعمال ہیں جو یہ کمپیوٹر سرانجام دے سکتا ہے۔ اس کمپیوٹر کی ہدایات کی فہرست  پر اب غور کرتے ہیں۔ہدایت کا مجموعہ کمپیوٹر کی \اصطلاح{ مادری زبان }\فرہنگ{زبان!مادری}\فرہنگ{مادری زبان}\حاشیہب{ assembly language}\فرہنگ{language!assembly} کہلاتی ہے۔

\جزوحصہء{نقل الف}
حافظہ کے مقام \عددی{0000_2} پر موجود معلومات کو ہم  \عددی{R_0} کہتے ہیں، مقام \عددی{0001_2} پر  \عددی{R_1} ہو گا، وغیرہ ۔ یوں \عددی{R_0} مقام \عددی{0H} پر محفوظ ہے ، \عددی{R_1} پتہ \عددی{1H} پر، \عددی{R_2} پتہ \عددی{2H} پر، وغیرہ، جہاں \عددی{0H} سے مراد   \عددی{0_{16}} ہے۔اساس \عددی{16} اعداد کے آخر میں  زیرنوشت \عددی{16} لکھنے کی بجائے ہم عدد کے آخر میں \عددی{H} لکھتے ہیں۔

\موٹا{نقل الف  } اس کمپیوٹر کی ایک ہدایت ہے جو  کہتی ہے دفتر ا  الف میں مواد نقل کریں۔ پوری ہدایت میں  اس مواد کا اساس سولہ پتہ بھی دیا جاتا ہے جو دفتر الف میں بھرا جائے گا، لہٰذا مکمل ہدایت درج ذیل ہے جو جدول \حوالہ{جدول_کمپیوٹر_ہدایات} میں  پیش ہے۔
\begin{align*}
\text{\RL{نقل  الف \فاصلہء   پتہ}}
\end{align*}
 یوں\قول{ نقل الف \عددی{8H}  } کہتی ہے کہ عارضی حافظہ کے پتہ \عددی{8H} پر درج معلومات کو دفتر الف میں نقل کریں۔ اس ہدایت پر عمل کرنے کے بعد دفتر الف میں  اور حافظہ کے مقام  \عددی{8H} پر ایک جیسا مواد  پایا جائے گا۔ یوں  درج ذیل صورت میں
\begin{align*}
R_8=1111\,0000
\end{align*}
جو کہتی ہے  مقام \عددی{R_8} پر  ثنائی معلومات  \عددی{1111\,0000} محفوظ ہے،  ذیل ہدایت
\begin{align*}
8H \quad \text{\RL{نقل الف }}
\end{align*}
پر عمل کرنے کے بعد درج ذیل ہو گا۔
\begin{align*}
\text{الف}=1111\,0000
\end{align*}
آپ نے دیکھا  یہ  ہدایت دفتر الف میں معلومات نقل کرتے ہوئے  حافظہ میں درج معلومات پر   اثر انداز نہیں ہوتی۔

اسی طرح \قول{نقل الف \عددی{AH}}  مقام \عددی{10_{10}} سے دفتر الف میں معلومات  نقل کرے گی، اور \قول{نقل الف \عددی{FH}} مقام \عددی{F_{16}} سے معلومات دفتر الف میں نقل کرے گی۔

\جزوحصہء{جمع}
کمپیوٹر کی یہ ہدایت دو اعداد جمع کرنے کو کہتی  ہے۔پہلا عدد دفتر الف میں ہو گا جبکہ دوسرے عدد کا پتہ مکمل ہدایت میں شامل ہو گا؛ نتیجہ دفتر الف میں محفوظ ہو گا، لہٰذا دفتر الف میں پہلے سے موجود مواد زائل ہو گا۔یوں   اگر دفتر الف میں \عددی{2_{10}} اور حافظہ کے مقام \عددی{9H} پر \عددی{3_{10}} ہو:
\begin{align*}
\text{الف}&=0000\,0010\\
R_9&=0000\,0011
\end{align*}
تب ذیل ہدایت
\begin{align*}
9H\quad \text{\RL{جمع}}
\end{align*} 
پر عمل کرنے کے لئے درج ذیل اقدام پر عمل کرنا ہو گا۔ پہلے قدم پر،  دفتر ب میں \عددی{R_9} ڈالا جائے گا:
\begin{align*}
\text{ب}=0000\,0011
\end{align*}
جس کے فوراً بعد  جمع و منفی کار   الف اور ب کا مجموعہ
\begin{align*}
\text{مجموعہ}=0000\,0101
\end{align*}
معلوم  کرتا ہے۔دوسرے قدم پر،   یہ مجموعہ دفتر الف میں ڈالا جاتا ہے۔
\begin{align*}
\text{الف}&=0000\,0101
\end{align*}

جب  بھی \قول{جمع}  کی ہدایت پر عمل کیا جائے درج بالا اقدام اٹھانے ہوں گے؛ دیے گئے پتہ سے مواد دفتر ب میں ڈال کر جمع  و منفی کار  سے مجموعہ حاصل کرنے کے بعد نتیجہ دفتر الف میں ڈالا جاتا ہے۔چونکہ   دفتر الف میں پہلے سے موجود مواد  کے اوپر  نیا مواد (حاصل جمع) لکھا جاتا ہے لہٰذا  دفتر الف کا پرانا  مواد زائل ہو گا۔اسی طرح چونکہ دفتر ب میں دیے گئے پتے کا مواد ڈالا  کیا جاتا ہے لہٰذا دفتر ب کا پرانا مواد بھی زائل ہو گا۔ اس طرح  \قول{جمع \عددی{9H}} پر عمل کرنے سے دفتر الف کا مواد اور \عددی{R_9} کا مجموعہ دفتر الف میں  حاصل ہو گا۔ \قول{جمع \عددی{FH}} پر عمل کے بعد دفتر الف میں \عددی{R_F} اور دفتر الف کا مجموعہ پایا جائے گا۔

\begin{table}
\caption{کمپیوٹر کی مادری زبان کی ہدایات}
\label{جدول_کمپیوٹر_ہدایات}
\centering
\begin{tabular}{r|r}
\toprule
\multicolumn{1}{c|}{ہدایت}& \multicolumn{1}{c}{عمل}\\
\midrule
نقل الف \فاصلہء پتہ& دفتر الف میں حافظہ سے مواد نقل کریں\\
جمع \فاصلہء پتہ& دفتر الف کے ساتھ حافظہ کا مواد جمع کریں\\
منفی\فاصلہء پتہ&دفتر الف سے حافظہ کا مواد منفی کریں\\
برآمد&دفتر الف کا مواد  ر خارجی  دفتر میں ڈالیں\\
رک& کام کرنا روک دیں\\
\bottomrule
\end{tabular}
\end{table}

\جزوحصہء{منفی}
دو اعداد منفی کرنے کے لئے کمپیوٹر کی ہدایت \اصطلاح{منفی} ہے جو دفتر الف میں موجود عدد سے  دیا گیا عدد منفی کر کے نتیجہ دفتر الف میں دے گی۔ مکمل ہدایت میں منفی ہونے والے عدد کے مقام کا پتہ بھی شامل ہو گا۔
\begin{align*}
\text{\RL{منفی\فاصلہء پتہ}}
\end{align*}
یوں \قول{منفی \عددی{CH}} کا مطلب ہے دفتر الف میں موجود مواد سے حافظہ  کے مقام \عددی{CH} پر موجود مواد \عددی{R_C} منفی کر کے نتیجہ دفتر الف میں ڈالیں۔

مثال کی خاطر فرض کریں دفتر الف میں  اعشاری \عددی{7} اور  حافظہ کے مقام \عددی{CH} پر اعشاری \عددی{3} پایا جاتا ہے۔
\begin{align*}
\text{الف}&=0000\,0111\\
R_C&=0000\,0011
\end{align*}
\قول{منفی\عددی{CH}} پر عمل درج ذیل اقدام اٹھانے سے ہو گا۔ پہلے قدم پر،  دفتر ب میں \عددی{R_C}  ڈالا  کیا جاتا ہے: 
\begin{align*}
\text{ب}=0000\,0011
\end{align*}
جس کے فوراً بعد جمع و منفی کار دفتر الف اور ب کا فرق:
\begin{align*}
\text{فرق}=0000\,0100
\end{align*}
  معلوم کرتا ہے۔دوسرے قدم پر یہ فرق  دفتر الف میں  ڈالا جاتا ہے۔
\begin{align*}
\text{الف}=0000\,0100
\end{align*}

منفی کی  تمام ہدایت  پر عمل درج بالا اقدام کے ذریعہ ہو گا؛ دیے گئے پتہ پر موجود مواد حافظہ سے دفتر ب میں ڈال کر جمع و منفی کار کو مہیا کیا جاتا ہے جو فوراً ان کا فرق معلوم کرتا ہے۔ یہ فرق دفتر الف میں ڈالا جاتا ہے۔ یوں \قول{منفی \عددی{CH}} پر عمل کرتے ہوئے \عددی{R_C} کو دفتر الف سے منفی کر کے نتیجہ دفتر الف میں ڈالا جائے گا۔ \قول{منفی \عددی{EH}} مقام \عددی{EH}  پر موجود مواد \عددی{R_E} کو دفتر الف سے منفی کر کے نتیجہ دفتر الف میں ڈالتا ہے۔

\جزوحصہء{برآمد}
کمپیوٹر کی ہدایت  \اصطلاح{برآمد}  کہتی ہے دفتر الف کا مواد خارجی دفتر میں ڈالیں۔اس ہدایت پر عمل کرنے کے بعد دفتر الف کا مواد کمپیوٹر سے باہر دستیاب ہو گا جہاں سے آپ نتیجہ دیکھ سکتے ہیں۔

اس ہدایت پر عمل کرنے کے لئے  حافظہ سے رجوع کرنے کی ضرورت نہیں لہٰذا اس ہدایت میں پتہ درکار نہیں ہے۔

\جزوحصہء{رک}
یہ ہدایت  ، جو برنامے کی آخری ہدایت ہو گی، کمپیوٹر کو  مزید ہدایات پر عمل کرنے ے روکتی ہے۔یہ ہدایت،  جملہ مکمل ہونے کے بعد    (جملے کے آخر میں) \اصطلاح{  ختمہ  }\فرہنگ{ختمہ}\حاشیہب{fullstop}\فرہنگ{fullstop} کے مترادف ہے۔ ہر برنامے کے آخر میں یہ ہدایت ضروری ہے ؛  ورنہ کمپیوٹر  بے باق   دوڑتا رہے گا  اور بے مقصد (اور غلط) نتائج فراہم کرتا رہے گا۔

رک کی ہدایت از خود مکمل ہے۔ اس پر عمل کرنے کی خاطر حافظہ سے رجوع کرنے کی ضرورت نہیں لہٰذا اس ہدایت میں پتے کی شمولیت نہیں ہو گی۔

\جزوحصہء{حافظہ سے رجوع کرنے والے  راجع  ہدایات}
نقل الف، جمع، اور منفی کی ہدایات حافظہ سے رجوع کرنی ہیں لہٰذا یہ\اصطلاح{  راجع ہدایات }\فرہنگ{راجع ہدایات}\حاشیہب{memory-reference instructions}\فرہنگ{memory reference instructions} کہلاتی ہیں۔ اس کے برعکس برآمد اور  رک حافظہ سے رجوع نہیں کرتی ہیں لہٰذا یہ ہدایات غیر راجع  ہیں۔

\جزوحصہء{\عددی{8080} اور \عددی{8085}}
وسیع  پیمانے پر استعمال ہونے والا  پہلا \اصطلاح{   خرد عامل  کار  }\فرہنگ{خرد عامل کار}\حاشیہب{microprocessor}\فرہنگ{microprocessor} (مائکروپراسیسر ) \عددی{8080}  تھا۔ اس کی کل \عددی{72} ہدایات ہیں۔اس خرد عامل کار \عددی{8085} ہے  جو انہیں ہدایات پر چلتا ہے۔  اس باب کے  سادہ ترین کمپیوٹر کو حقیقتاً قابل استعمال بنانے کی غرض سے ہم  اس کی ہدایات کو \عددی{8080/8085} کی ہدایت کے  ہم آہنگ بناتے ہیں۔ دوسرے لفظوں میں  نقل، جمع، منفی، برآمد، اور رک \عددی{8080/8085} کے بھی ہدایات ہیں۔

\ابتدا{مثال}\شناخت{مثال_کمپیوٹر_برنامہ_الف}
سادہ ترین کمپیوٹر کا ایک برنامہ پیش ہے۔
\begin{center}
\begin{tabular}{rr}
پتہ& ہدایات\\[0.5ex]
0H& نقل  \عددی{9H}\\
1H&جمع \عددی{AH}\\
2H&جمع \عددی{BH}\\
3H&منفی \عددی{CH}\\
4H&برآمد\\
5H&رک
\end{tabular}
\end{center}
حافظہ میں برنامہ سے اوپر درج ذیل مواد پایا جاتا ہے۔
\begin{center}
\begin{tabular}{rr}
پتہ& مواد\\[0.5ex]
6H&FFH\\
7H&FFH\\
8H&FFH\\
9H&01H\\
AH&02H\\
BH&03H\\
CH&04H\\
DH&FFH\\
EH&FFH\\
FH&FFH
\end{tabular}
\end{center}
یہ ہدایات  کیا کریں گے؟

حل:\quad
برنامہ نچلے حافظہ میں \عددی{0H} تا \عددی{5H} مقامات پر رکھا گیا ہے۔ پہلی ہدایت حافظہ کے مقام \عددی{9H} سے مواد \عددی{01H} دفتر الف میں نقل کرتی ہے۔
\begin{align*}
\text{الف}=01H
\end{align*}
دوسری ہدایت مقام \عددی{AH}  کا مواد دفتر الف کے ساتھ جمع کر کے نتیجہ دفتر الف میں ڈالتی ہے۔
\begin{align*}
\text{الف}=01H+02H=03H
\end{align*}
تیسری ہدایت حافظہ کے مقام \عددی{BH} کے مواد کو دفتر الف (جس میں اس وقت \عددی{03H} موجود ہے) کے ساتھ جمع کر کے نتیجہ دفتر الف منتقل کرتی ہے۔
\begin{align*}
\text{الف}=03H+03H=06H
\end{align*}
چوتھی ہدایت مقام \عددی{CH} کے مواد کو دفتر الف سے منفی کرکے نتیجہ دفتر الف میں ڈالتی ہے۔
\begin{align*}
\text{الف}=06H-04H=02H
\end{align*}
پانچویں ہدایت دفتر الف کے مواد کو خارجی دفتر میں منتقل کرتی ہے۔خارجی دفتر کے ساتھ ثنائی نمائشی تختی   منسلک ہے جس پر یہ مواد ثنائی روپ میں نظر آئے گا۔یوں  نوری ڈایوڈ درج ذیل دکھائیں گے۔
\begin{align*}
0000\,0010
\end{align*}
آخری ہدایت   \اصطلاح{رک}  ہے جو کمپیوٹر کر  کو مزید ہدایات پر عمل کرنے سے روکتی ہے۔
\انتہا{مثال}

\حصہ{کمپیوٹر  کی برنامہ  نویسی}
کمپیوٹر کے  حافظہ میں ہدایات اور مواد بھرنے کے لئے ہمیں ایسی زبان استعمال کرنی ہو گی جو کمپیوٹر  سمجھ سکے۔ جدول \حوالہ{جدول_کمپیوٹر_رموز}  میں  کمپیوٹر  کے \اصطلاح{رموز }\فرہنگ{رمز}\حاشیہب{operation codes, op codes}\فرہنگ{op code} پیش ہیں۔ یوں \قول{ نقل  الف} کی ہدایت کے لئے کمپیوٹر \عددی{0000} کا  ثنائی رمز استعمال کرتا ہے۔\قول{ جمع } کے لئے \عددی{0001}، \قول{ منفی } کے لئے \عددی{0010}، \قول{  برآمد }کے لئے \عددی{1110}، اور \قول{ رک } کے لئے \عددی{1111} استعمال ہو گا۔ 
\begin{table}
\caption{سادہ ترین کمپیوٹر کے رمز}
\label{جدول_کمپیوٹر_رموز}
\centering
\begin{tabular}{rr}
\toprule
ہدایت&رمز\\
\midrule
نقل&0000\\
جمع&0001\\
منفی&0010\\
برآمد&1110\\
رک&1111\\
\bottomrule
\end{tabular}
\end{table}

\begin{figure}

\end{figure}

جیسا پہلے ذکر کیا گیا،(صفحہ \حوالہصفحہ{مثال_حافظہ_برنامہ_نویس} پر مثال \حوالہ{مثال_حافظہ_برنامہ_نویس} دیکھیں) برنامہ نویس (شکل \حوالہ{شکل_کمپیوٹر_برنامہ_نویسی})  سوئچ  کے ذریعہ  حافظہ میں معلومات ڈالتا ہے۔ ان سوئچ کو یوں استعمال کیا گیا ہے کہ منقطع (کھڑا ) سوئچ \عددی{1} اور غیر منقطع ( بیٹھا  یا چالو) سوئچ \عددی{0} دیتا ہے۔ برنامہ نویسی کے دوران سوئچ \عددی{d_4} تا \عددی{d_7}   ہدایت  کے رمز کے مطابق رکھے جاتے ہیں جبکہ \عددی{d_0} تا \عددی{d_3}  ہدایت کے باقی\اصطلاح{ زیر عمل }\فرہنگ{زیر عمل}\حاشیہب{operand}\فرہنگ{operand} حصہ کے مطابق رکھے جاتے ہیں۔

مثلاً،  فرض کریں ہم درج ذیل  ہدایات حافظہ میں بھرنا چاہتے ہیں۔
\begin{center}
\begin{tabular}{rrr}
پتہ&\multicolumn{2}{c}{ہدایت}\\[1ex]
$0H$& نقل& $ّFH$\\
$1H$&جمع& $EH$\\
$2H$&رک&
\end{tabular}
\end{center}

سب سے پہلے ایک ایک ہدایت کا ثنائی روپ  حاصل کرتے ہیں۔
\begin{center}
\begin{tabular}{rrrr}
نقل&$FH$&=&$0000\,1111$\\
جمع & $EH$&=&$0001\,1110$\\
رک&&=&$1111\,xxxx$
\end{tabular}
\end{center}
پہلی ہدایت  \قول{نقل \عددی{FH}} ہے جس   کے دو حصے ہیں۔ اس کا پہلا حصہ  ہدایت \قول{نقل } ہے جس کا ثنائی رمز  \عددی{0000}   ہے؛ اس کا دوسرا حصہ   \عددی{FH} ہے جو     اس مقام کا پتہ ہے جہاں سے مواد لیا جائے گا۔یہ ہدایت کا  \اصطلاح{زیر عمل }\فرہنگ{زیر عمل}\حاشیہب{operand}\فرہنگ{operand} حصہ ہے۔ اس پتے کا ثنائی مماثل \عددی{1111} ہے۔   یوں \قول{جمع \عددی{FH}} کی جگہ  ان کے ثنائی مماثل جوڑ کر \عددی{0000\,1111} حاصل کیا گیا ہے۔ دوسری ہدایت میں جمع کا رمز \عددی{0001} اور زیر عمل حصہ \عددی{EH} کا ثنائی مماثل \عددی{1110} ہے۔ ان کو  ساتھ ساتھ لکھ کر \عددی{0001\,1110} حاصل کیا گیا ہے۔ آخری ہدایت  میں رک کا رمز \عددی{1111} ہے جبکہ اس کا کوئی زیر عمل حصہ نہیں پایا جاتا، لہٰذا زیر عمل حصہ غیر مطلوبہ ہے جس میں کچھ بھی لکھا جا سکتا ہے۔ اس غیر مطلوبہ حصہ کو \عددی{xxxx} سے ظاہر کیا گیا ہے۔یوں \عددی{1111\,xxxx} حاصل کیا گیا ہے۔

اب \عددی{S_2} کو بٹھا کر (زمین   سے  جوڑ کر)   پتہ  اور مواد کے  سوئچ    قدم با قدم   درج ذیل رکھیں، جہاں \قول{ک} سے مراد  کھڑا   یعنی منقطع  سوئچ ہے جو \عددی{1} کو ظاہر کرتا ہے،  \قول{ب} سے مراد بیٹھا یا غیر منقطع (چالو) سوئچ ہے کو \عددی{0}   دیگا، اور \قول{x} سے مراد یہ کہ  سوئچ کسی بھی حالت میں  (منقطع یا غیر منقطع)   ہو سکتا ہے۔
\begin{center}
\begin{tabular}{rr}
\multicolumn{1}{c}{پتہ}& \multicolumn{1}{c}{مواد}\\[1ex]
ب ب ب ب & ک ک ک ک   ب ب ب ب\\
ک ب ب ب & ب ک ک ک  ک ب ب ب\\
ب ک ب ب & \,\,\,\,\,
 x\,\quad  x\,\quad  x\, \quad x \quad
  ک ک ک ک
\end{tabular}
\end{center}

ہر قدم پر  پتہ اور  مواد سوئچ  مطلوبہ حالت میں رکھ کر   \عددی{S_1} کو بٹھا  کر دوبارہ کھڑا کریں۔ تینوں پتہ پر مواد لکھنے کے بعد \عددی{S_2} کو کھڑا کریں۔حافظہ کے ابتدائی تین  مقامات پر اب درج ذیل پایا جائے گا۔
\begin{center}
\begin{tabular}{LL}
\multicolumn{1}{c}{\text{\RL{پتہ}}}& \multicolumn{1}{c}{\text{\RL{مواد}}}\\[1ex]
0000&0000\,1111\\
0001&0001\,1110\\
0010&1111\,xxxx
\end{tabular}
\end{center}

آپ نے دیکھا کہ ہم کمپیوٹر کی \اصطلاح{ مادری زبان  } میں اردو  کے الفاظ مثلاً  \قول{نقل}، اور  \قول{جمع} استعمال کر کے کمپیوٹر  کو ہدایات جاری کرتے ہیں۔ کمپیوٹر از خود \قول{  ثنائی زبان  } سمجھتا ہے جو   \اصطلاح{مشینی زبان}\فرہنگ{زبان!مشینی}\فرہنگ{مشینی زبان}\حاشیہب{machine language}\فرہنگ{language!machine}\فرہنگ{machine language}    کہلاتی ہے۔  مشینی  زبان میں  \عددی{0} اور \عددی{1}  سے الفاظ بنائے جاتے ہیں۔ درج ذیل مثال  ان  زبانوں میں فرق اجاگر کرتا ہے۔

\ابتدا{مثال}
گزشتہ مثال میں دیے گئے  برنامے کا ترجمہ مشینی  زبان میں کریں۔

حل:\quad
مثال \حوالہ{مثال_کمپیوٹر_برنامہ_الف} کا برنامہ جو مادری زبان میں ہے  ذیل ہے۔
\begin{center}
\begin{tabular}{rr}
پتہ& ہدایات\\[0.5ex]
0H& نقل  \عددی{9H}\\
1H&جمع \عددی{AH}\\
2H&جمع \عددی{BH}\\
3H&منفی \عددی{CH}\\
4H&برآمد\\
5H&رک
\end{tabular}
\end{center}
اس کا ترجمہ مشینی  زبان میں کرتے ہیں۔
\begin{center}
\begin{tabular}{RR}
\multicolumn{1}{c}{\text{\RL{پتہ}}}& \multicolumn{1}{c}{\text{\RL{ہدایت}}}\\[0.5ex]
0000&0000\,1001\\
0001&0001\,1010\\
0010&0001\,1011\\
0011&0010\,1100\\
0100&1110\,xxxx\\
0101&1111\,xxxx
\end{tabular}
\end{center}

اس ثنائی برنامہ میں  ہدایت کے چار  بلند  تر رتبی بِٹ \قول{    عمل } کو ظاہر کرتے ہیں جبکہ چار کم تر رتبی بِٹ  \قول{پتہ } فراہم کرتے ہیں۔ بعض اوقات ہم  چار بلند تر رتبی بِٹ کو \اصطلاح{ جزو  ہدایت }\فرہنگ{جزو ہدایت}\حاشیہب{instruction field}\فرہنگ{instruction field} اور چار کم تر رتبی بِٹ کو  \اصطلاح{جزو  پتہ }\فرہنگ{جزو پتہ}\حاشیہب{address field}\فرہنگ{address field} کہتے ہیں۔
\begin{align*}
\underbrace{XXXX}_{\text{\RL{جزو ہدایت}}}\, \underbrace{YYYY}_{\text{\RL{جزو پتہ}}}=\text{\RL{ہدایت}}
\end{align*}
\انتہا{مثال}
\ابتدا{مثال}
درج  ذیل  حساب کرنے کے لئے کمپیوٹر کا برنامہ  لکھیں۔ تمام اعداد اعشاری ہیں۔
\begin{align*}
16+20+24-32
\end{align*}

حل:\quad
گزشتہ مثال کا برنامہ لے کر  حافظہ کے مقام \عددی{9H} تا \عددی{CH} میں بالترتیب  مواد \عددی{16}، \عددی{20}، \عددی{24}، اور \عددی{32}  کے اساس سولہ مماثل لکھ کر    درج ذیل مطلوبہ برنامہ حاصل ہو گا۔(اعشاری \عددی{16} کا اساس سولہ مماثل \عددی{10H} ہے۔)
\begin{center}
\begin{tabular}{RR}
\multicolumn{1}{c}{\text{\RL{پتہ}}}& \multicolumn{1}{c}{\text{\RL{ہدایت}}}\\[0.5ex]
0H& \LDA{9H}\\
1H&\ADD{AH}\\
2H&\ADD{BH}\\
3H&\SUB{CH}\\
4H&\OUT\\
5H&\HLT\\
6H&XX\\
7H&XX\\
8H&XX\\
9H&10H\\
AH&14H\\
BH&18H\\
CH&20H\\
DH&XX\\
EH&XX\\
FH&XX
\end{tabular}
\end{center}
اس کا ترجمہ مشینی  زبان میں کرتے ہیں۔
\begin{center}
\begin{tabular}{RR}
\multicolumn{1}{c}{\text{\RL{پتہ}}}& \multicolumn{1}{c}{\text{\RL{ہدایت}}}\\[0.5ex]
0000&0000\,1001 \\
0001&0001\,1010\\
0010&0001\,1011\\
0011&0010\,1100\\
0100&1110\,xxxx\\
0101&1111\,xxxx\\
0110&xxxx\,xxxx\\
0111&xxxx\,xxxx\\
1000&xxxx\,xxxx\\
1001&0001\,0000\\
1010&0001\,0100\\
1011&0001\,1000\\
1100&0010\,0000\\
1101&xxxx\,xxxx\\
1110&xxxx\,xxxx\\
1111&xxxx\,xxxx
\end{tabular}
\end{center}
یاد رہے برنامے کی پہلی ہدایت حافظہ کے مقام \عددی{0000}  سے پڑھی جاتی ہے،   دوسری  مقام \عددی{0001} سے پڑھی جاتی ہے، وغیرہ، لہٰذا  برنامہ زیریں حافظہ میں اور مواد بالا میں رکھا  گیا ہے۔ غیر مستعمل مقامات  میں معلومات   کو   \عددی{xxxx\,xxxx}  دکھایا گیا ہے۔
\انتہا{مثال}
\ابتدا{مثال}
درج بالا مثال میں حاصل ثنائی برنامہ کو اساس سولہ کے روپ میں لکھیں۔ ثنائی روپ کی بجائے ہم عموماً  برنامے کا اساس سولہ روپ استعمال کرتے ہیں۔

حل:\quad
\begin{center}
\begin{tabular}{RR}
\multicolumn{1}{c}{\text{\RL{پتہ}}}& \multicolumn{1}{c}{\text{\RL{ہدایت}}}\\[0.5ex]
0H&09H \\
1H&1AH\\
2H&1BH\\
3H&2CH\\
4H&EXH\\
5H&FXH\\
6H&XXH\\
7H&XXH\\
8H&XXH\\
9H&10H\\
AH&14H\\
BH&18H\\
CH&20H\\
DH&XXH\\
EH&XXH\\
FH&XXH
\end{tabular}
\end{center}
اساس سولہ میں لکھی گئی زبان بھی مشینی زبان کہلاتی ہے۔

  مشینی زبان میں منفی عدد کا اساس \عددی{2} تکملہ  استعمال کیا جاتا ہے۔ مثال کے طور پر ، \عددی{-03H}  کی بجائے \عددی{FDH} حافظہ میں ڈالا جائے گا۔
\انتہا{مثال}

\حصہ{بازیابی  پھیرا}
کمپیوٹر کی خودکار کارکردگی کا دارومدار \قول{ قابو مرکز } پر ہے۔ حافظہ سے باری باری ایک  ہدایت    اٹھانے اور اس پر عمل کرنے  کے  احکامات قابو مرکز جاری کرتا ہے۔ہدایت  اٹھانے اور اس پر عمل کرنے کے دوران کمپیوٹر مختلف \اصطلاح{   وقتیہ حال }\فرہنگ{وقتیہ حال}\حاشیہب{timing states}\فرہنگ{timing states} (\عددی{T} حال)   سے گزرتا ہے، جس میں دفاتر  کا  مواد  تبدیل ہوتا ہے۔ آئیں وقتیہ حال پر غور کریں۔

\جزوحصہء{چھلا گنت کار}
اس کمپیوٹر میں  چھلا گنت کار  مستعمل ہے جو شکل \حوالہ{شکل_کمپیوٹر_چھلا} میں پیش ہے۔ مخلوط دور \عددی{74107} میں دو عدد جے کے پلٹ کار  پائے جاتے ہیں لہٰذا تین مخلوط دور استعمال کیے گئے۔ اس  مخلوط دور میں زبردستی پست   کا مداخل موجود ہے، تاہم اس میں زبردستی بلند کا مداخل موجود نہیں۔  استعمال سے پہلا ایک مرتبہ  چھلا گنت کار کو ابتدائی حال میں لانا ضروری ہے جس میں صرف ایک مخارج بلند ہو۔  زبردستی پست مداخل پلٹ کے   مخارج پس کر تا ہے جبکہ ہمیں ایک مخارج بلند چاہیے۔ اسی لئے  بایاں ترین پلٹ باقی سے مختلف طریقے سے استعمال کیا گیا ہے۔ پست حال میں اس کا \عددی{\overline{Q}} بلند ہو گا جو ساعت کے کنارہ اترائی پر اگلی پلٹ کو منتقل ہو گا۔

\begin{figure}
\centering
\begin{subfigure}{1\textwidth}
\centering
\begin{tikzpicture}
\pgfmathsetmacro{\ksepX}{2}
\pgfmathsetmacro{\kpin}{0.4}
\def\pnd{pnd}
\def\p{p}			%pin tip
\def\pb{pb}			%pin base
\def\pbd{pbd}			%pin base
\def\pd{pd}
\def\pQ{p6}
\kJKFF[u0]{0}{0}
\kJKFF[u1]{-1*\ksepX}{0}
\kJKFF[u2]{-2*\ksepX}{0}
\kJKFF[u3]{-3*\ksepX}{0}
\kJKFF[u4]{-4*\ksepX}{0}
\kFF[u5]{-5*\ksepX}{0}
\def\kref{u5}
\draw(u0pn2)node[ocirc]{};
\draw(u1pn2)node[ocirc]{};
\draw(u2pn2)node[ocirc]{};
\draw(u3pn2)node[ocirc]{};
\draw(u4pn2)node[ocirc]{};
\foreach \n in {1,2,3,4,6}{\draw[thin](\kref\pb\n)--(\kref\p\n);}
\foreach \n/\lbl in {1/K,3/J,4/Q,6/{\overline{Q}}}{\draw(\kref\pl\n)node[]{$\lbl$};}
\foreach \n in {2}{\draw[thin](\kref\pcd\n)--(\kref\pcm\n)--(\kref\pcu\n);}
\foreach \n in {2,6}{\draw(\kref\pn\n)node[ocirc]{};}
\draw(u5p6)--(u4p1) (u4p6)--(u3p1) (u3p6)--(u2p1)   (u2p6)--(u1p1)  (u1p6)--(u0p1) (u0p6)--++(0,2*\kpin)-|(u5p1);
\draw(u5p4)--(u4p3) (u4p4)--(u3p3) (u3p4)--(u2p3)   (u2p4)--(u1p3)  (u1p4)--(u0p3) (u0p4)--++(0,-4*\kpin) -|(u5p3);
\draw(u0p2)--++(0,-4.5*\kpin)--++(-5*\ksepX-\kpin,0)coordinate(kBot)|-coordinate(kclk)(u5p2)   (kclk)--++(-\kpin,0)coordinate(klft)node[left]{$CLK$};
\draw(u1p2)--(u1p2 |- kBot) (u2p2)--(u2p2 |- kBot) (u3p2)--(u3p2 |- kBot) (u4p2)--(u4p2 |- kBot);
\foreach \n in {0,1,2,3,4,5}{\draw(u\n\pbd)--(u\n\pd) (u\n\pnd)node[ocirc]{};}
\draw(u0pd)--++(0,-3*\kpin)coordinate(kRES)--(kRES -|klft)coordinate(kR)node[left]{$\overline{CLR}$};
\draw(u1pd)--(u1pd |- kR) (u2pd)--(u2pd |- kR) (u3pd)--(u3pd |- kR) (u4pd)--(u4pd |- kR) (u5pd)--(u5pd |- kR);
\foreach \n/\a in {0/6,1/5,2/4,3/3,4/2,5/1}{\draw[thin](u\n\pQ)++(0.5*\kpin,0)--++(0,3*\kpin)node[above]{$T_{\a}$};}
\draw(u0p6)--++(0.5*\kpin,0);
\draw(u0pbd)node[below right]{\scriptsize{$13$}};
\draw(u2pbd)node[below right]{\scriptsize{$13$}};
\draw(u4pbd)node[below right]{\scriptsize{$13$}};
\draw(u1pbd)node[below right]{\scriptsize{$10$}};
\draw(u3pbd)node[below right]{\scriptsize{$10$}};
\draw(u5pbd)node[below right]{\scriptsize{$10$}};
\foreach \n/\a in {1/11,2/9,3/8}{\draw(u5pb\n)node[above,xshift=-1.25ex]{\scriptsize{$\a$}};}
\foreach \n/\a in {4/5,6/6}{\draw(u5pb\n)node[above,xshift=1.25ex]{\scriptsize{$\a$}};}
%
\foreach \n/\a in {1/1,2/12,3/4}{\draw(u0pb\n)node[above,xshift=-1.25ex]{\scriptsize{$\a$}};}
\foreach \n/\a in {4/2,6/3}{\draw(u0pb\n)node[above,xshift=1.25ex]{\scriptsize{$\a$}};}
\foreach \n/\a in {1/1,2/12,3/4}{\draw(u2pb\n)node[above,xshift=-1.25ex]{\scriptsize{$\a$}};}
\foreach \n/\a in {4/2,6/3}{\draw(u2pb\n)node[above,xshift=1.25ex]{\scriptsize{$\a$}};}
\foreach \n/\a in {1/1,2/12,3/4}{\draw(u4pb\n)node[above,xshift=-1.25ex]{\scriptsize{$\a$}};}
\foreach \n/\a in {4/2,6/3}{\draw(u4pb\n)node[above,xshift=1.25ex]{\scriptsize{$\a$}};}
%
\foreach \n/\a in {1/8,2/9,3/11}{\draw(u1pb\n)node[above,xshift=-1.25ex]{\scriptsize{$\a$}};}
\foreach \n/\a in {4/6,6/5}{\draw(u1pb\n)node[above,xshift=1.25ex]{\scriptsize{$\a$}};}
\foreach \n/\a in {1/8,2/9,3/11}{\draw(u3pb\n)node[above,xshift=-1.25ex]{\scriptsize{$\a$}};}
\foreach \n/\a in {4/6,6/5}{\draw(u3pb\n)node[above,xshift=1.25ex]{\scriptsize{$\a$}};}
\draw(-0.25*\ksepX,-1.75cm)node[rectangle,inner sep=0pt,text width=0.5cm,align=center]{\small{$u36$\\ $74107$}};
\draw(-2.25*\ksepX,-1.75cm)node[rectangle,inner sep=0pt,text width=0.5cm,align=center]{\small{$u37$\\ $74107$}};
\draw(-4.25*\ksepX,-1.75cm)node[rectangle,inner sep=0pt,text width=0.5cm,align=center]{\small{$u38$\\ $74107$}};
\end{tikzpicture}
\caption{}
\end{subfigure}
\begin{subfigure}{1\textwidth}
\centering
\begin{tikzpicture}
\pgfmathsetmacro{\kpin}{0.5}
\pgfmathsetmacro{\kpina}{2.5}
\pgfmathsetmacro{\kpinb}{4}
\pgfmathsetmacro{\kpsep}{0.50}
\pgfmathsetmacro{\kul}{0.50}
\pgfmathsetmacro{\kmv}{0.15}
\pgfmathsetmacro{\kxdim}{2*\kul+5*\kpsep}
\pgfmathsetmacro{\kydim}{2*\kul+1*\kpsep}
\draw[thick](0,0)rectangle++(\kxdim,\kydim)node[pos=0.5]{\text{\RL{چھلا گنت کار}}};
\foreach \n/\a in {0/6,1/5,2/4,3/3,4/2,5/1}{\draw[thin](\kul+\n*\kpsep,0)--++(0,-\kpin)node[below]{$T_{\a}$};}
\foreach \n/\a in {0/{\overline{CLR}},1/{CLK}}{\draw[thin](\kxdim,\kul+\n*\kpsep)--++(\kpin,0)node[right]{$\a$};}
\draw[thin](\kxdim,\kul)++(0.07,0)node[ocirc]{}  (\kxdim,\kul+\kpsep)++(0.07,0)node[ocirc]{} 
(\kxdim,\kul+\kpsep-\kmv)--++(-\kmv,\kmv)--++(\kmv,\kmv);
\end{tikzpicture}
\caption{}
\end{subfigure}
\begin{subfigure}{1\textwidth}
\centering
\begin{otherlanguage}{english}
 \begin{tikztimingtable}[%
timing/.style={x=4ex,y=3ex},
timing/rowdist=6ex,
every node/.style={inner sep=0,outer sep=0},
%timing/c/arrow tip=latex, %and this set the style
%timing/c/rising arrows,
timing/slope=0, %0.1 is good
timing/dslope=0,
thick,
]
%\tikztimingmetachar{R}{[|/utils/exec=\setcounter{new}{0}|]}
%\usetikztiminglibrary[new={char=Q,reset char=R}]{counters}
%[timing/counter/new={char=c, base=2,digits=3,max value=7, wraps ,text style={font=\normalsize}}] 12{2c} \\ 
%$C$& H22{C}\\
%$\texturdu{\RL{پتہ}}$&3D{} 20D{[scale=1.5]\texturdu{\RL{درست پتہ}}} 3D{}\\
$CLK$&HN(a)LHN(b)LHN(c)LHN(d)LHN(e)LHN(f)LHN(g)LHN(h)L\\
\\
$T_1$&LHHLLLLLLLLLLHHL\\
$T_2$&LLLHHLLLLLLLLLLL\\
$T_3$&LLLLLHHLLLLLLLLL\\
$T_4$&LLLLLLLHHLLLLLLL\\
$T_5$&LLLLLLLLLHHLLLLL\\
$T_6$&LLLLLLLLLLLHHLLL\\
\extracode
\begin{pgfonlayer}{background}
\begin{scope}[]
\draw [latex-latex] (a|-row2.north) --node[fill=white]{$T_1$}node[below,yshift=-1ex]{\texturdu{حال}} (b|-row2.north);
\draw [latex-latex] (b|-row2.north) --node[fill=white]{$T_2$}node[below,yshift=-1ex]{\texturdu{حال}} (c|-row2.north);
\draw [latex-latex] (c|-row2.north) --node[fill=white]{$T_3$}node[below,yshift=-1ex]{\texturdu{حال}} (d|-row2.north);
\draw [latex-latex] (d|-row2.north) --node[fill=white]{$T_4$}node[below,yshift=-1ex]{\texturdu{حال}} (e|-row2.north);
\draw [latex-latex] (e|-row2.north) --node[fill=white]{$T_5$}node[below,yshift=-1ex]{\texturdu{حال}} (f|-row2.north);
\draw [latex-latex] (f|-row2.north) --node[fill=white]{$T_6$}node[below,yshift=-1ex]{\texturdu{حال}} (g|-row2.north);
\draw [latex-latex] (g|-row2.north) --node[fill=white]{$T_1$}node[below,yshift=-1ex]{\texturdu{حال}} (h|-row2.north);
\foreach \n in {a,b,c,d,e,f,g,h}{\draw[thin]($(\n|-row1.south)+(0,-1ex)$)--++(0,-4ex);}
%%\vertlines[darkgray,dotted]{3.6,7.5,11.5,15.5,19.5,23.5,27.5}
%\foreach \n in {1,3,...,11} \draw(4*\n ex-2ex,-4ex+1.25ex)node[]{$0$};
%\foreach \n in {2,4,...,12} \draw(4*\n ex-2ex,-4ex+1.25ex)node[]{$1$};
%\foreach \n in {1,2,5,6,9,10} \draw(4*\n ex-2ex,-12 ex+1.25ex)node[]{$0$};
%\foreach \n in {3,4,7,8,11,12} \draw(4*\n ex-2ex,-12 ex+1.25ex)node[]{$1$};
%\foreach \n in {1,2,3,4,9,10,11,12} \draw(4*\n ex-2ex,-20 ex+1.25ex)node[]{$0$};
%\foreach \n in {5,6,7,8} \draw(4*\n ex-2ex,-20 ex+1.25ex)node[]{$1$};
%\draw(4*4 ex-2ex,-12 ex+1.25ex) circle (0.25cm and 1.75cm);
%\draw(4*4 ex-2ex,-7*\rowdist+1.25ex) circle (0.25cm and 0.5cm);
%%\foreach \n in {1}\draw(B\n.south)--(F\n.north);
%%\foreach \n in {1}\draw(C\n.south)--(G\n.north);
\end{scope}
\end{pgfonlayer}
\end{tikztimingtable}
\end{otherlanguage}
\caption{}
\end{subfigure}
\caption{(ا) چھلا گنت کار،  (ب) ڈبہ شکل،  (ج) ساعت،اور وقتیہ ترسیمات۔}
\label{شکل_کمپیوٹر_چھلا}
\end{figure}

شکل \حوالہ{شکل_کمپیوٹر_چھلا}  -ب میں   گنت کار کی ڈبہ شکل   جبکہ  شکل-د میں ساعت اور وقتیہ ترسیمات  پیش ہیں۔ چھلا گنت کار کا مخارج درج ذیل ہے۔
\begin{align*}
\bold{T}=T_6T_5T_4T_3T_2T_1
\end{align*}
کمپیوٹر کی دوڑ کے آغاز میں چھلا لفظ درج ذیل ہو گا۔
\begin{align*}
\bold{T}=000001
\end{align*}
یک بعد دیگرے ساعت   کی دھڑکن  ذیل چھلا الفاظ پیدا کرتا ہے۔
\begin{align*}
\bold{T}&=000010\\
\bold{T}&=000100\\
\bold{T}&=001000\\
\bold{T}&=010000\\
\bold{T}&=100000\\
\end{align*}
اس کے بعد چھلا گنت کار \عددی{000001}  پہنچتا ہے اور دوبارہ چکر کاٹنا  شروع کرتا  ہے۔ یہ عمل مسلسل چلتا ہے۔ ہر ایک چھلا لفظ ایک \عددی{T} پھیرا ظاہر کرتا ہے۔

شکل-ج میں وقتیہ ترسیمات پیش ہیں۔ ابتدائی \عددی{T_1} حال  کا آغاز ساعت کے پہلے کنارہ اترائی پر  اور اختتام اگلے کنارہ اترائی پر  ہو گا۔ اس \عددی{T} حال میں چھلا گنت کار کا \عددی{T_1} بِٹ بلند  رہے گا۔

اگلے حال میں \عددی{T_2} بلند ہو گا؛ اس سے اگلے میں \عددی{T_3}؛ اس کے بعد \عددی{T_4}؛ وغیرہ۔ جیسا آپ دیکھ سکتے ہیں چھلا گنت کار چھ \عددی{T} حال پیدا کرتا ہے۔ ان چھ \عددی{T} حال کے دوران   (ہر) ایک ہدایت اٹھایا جاتا ہے اور اس پر عمل کیا جاتا ہے۔

جیسا دکھایا گیا ہے، ساعت کا کنارہ  چڑھائی   نصف \عددی{T} حال  گزرنے کے بعد (یعنی وسط میں )  آتا ہے۔ یہ ایک اہم حقیقت ہے جس پر جلد روشنی ڈالی جائے گی۔

\جزوحصہء{پتہ حال}
برنامہ گنت کار سے حافظہ کو پتہ  \عددی{T_1} حال   کے دوران منتقل ہوتا ہے، لہٰذا یہ \اصطلاح{پتہ حال}\فرہنگ{حال!پتہ}\حاشیہب{address state}\فرہنگ{state!address}  کہلاتا ہے۔شکل  \حوالہ{شکل_کمپیوٹر_اجاگر_حصے_بازیابی_پھیرا}-الف میں کمپیوٹر کے وہ حصے گہری سیاہی سے  اجاگر کیے گئے ہیں جو  \عددی{T_1}  حال  کے دوران  فعال ہیں (غیر فعال حصے ہلکی سیاہی میں دکھائے گئے ہیں؛ مزید،  ڈبہ ادوار  کے مختصر  نام لکھ گئے ہیں)۔

پتہ حال کے دوران \عددی{E_P} اور \عددی{\overline{L}_M} فعال جبکہ باقی تمام بِٹ غیر فعال ہوں گے۔ یوں اس حال کے دوران  قابو و ترتیب کار  درج ذیل قابو لفظ خارج کرتا ہے۔
\begin{align*}
\text{\RL{قابو}}&=C_PE_P\overline{L}_M\overline{CE}\quad  \overline{L}_I\overline{E}_I\overline{L}_AE_A\quad S_UE_U\overline{L}_B\overline{L}_O\\
&=\,\,0\,\,\,\,\,1\,\,\,\,\,0\,\,\,\,\,1\quad\,\, 1\,\,\,\,1\,\,\,\,1\,\,\,\,0\quad \quad 0\,\,\,\,0\,\,\,\,1\,\,\,\,1
\end{align*}

\جزوحصہء{بڑھوتری حال}
شکل  \حوالہ{شکل_کمپیوٹر_اجاگر_حصے_بازیابی_پھیرا}-ب میں کمپیوٹر کے وہ حصے اجاگر کیے گئے ہیں جو \عددی{T_2} حال کے دوران فعال ہیں۔ اس  حال میں گنت کار  کا  شمار (گنتی ) ایک قدم بڑھایا جاتا ہے لہٰذا اس کو\اصطلاح{ بڑھوتری حال}\فرہنگ{حال!بڑھوتری}\حاشیہب{increment state}\فرہنگ{state!increment} کہتے ہیں۔ بڑھوتری حال کے دوران قابو و ترتیب کار درج ذیل قابو لفظ خارج کرتا ہے۔
\begin{align*}
\text{\RL{قابو}}&=C_PE_P\overline{L}_M\overline{CE}\quad  \overline{L}_I\overline{E}_I\overline{L}_AE_A\quad S_UE_U\overline{L}_B\overline{L}_O\\
&=\,\,1\,\,\,\,\,0\,\,\,\,\,1\,\,\,\,\,1\quad\,\, 1\,\,\,\,1\,\,\,\,1\,\,\,\,0\quad \quad 0\,\,\,\,0\,\,\,\,1\,\,\,\,1
\end{align*}
جیسا آپ دیکھ سکتے ہیں \عددی{C_P} فعال ہو گا۔

\جزوحصہء{حافظہ حال}
  حافظہ سے ہدایت دفتر کو \عددی{T_3} حال کے دوران ہدایت منتقل کی جاتی ہے۔ یہ ہدایت فراہم کردہ پتہ کے مقام  سے پڑھی جاتی ہے۔اس حال کے دوران فعال حصے  شکل  \حوالہ{شکل_کمپیوٹر_اجاگر_حصے_بازیابی_پھیرا}-ج میں دکھائے گئے ہیں۔ اس حال میں صرف \عددی{\overline{CE}} اور \عددی{\overline{L}_I} قابو بِٹ فعال ہوں گے۔اس حال کے دوران قابو و ترتیب کار  درج ذیل قابو لفظ خارج کرتا ہے۔
  \begin{align*}
\text{\RL{قابو}}&=C_PE_P\overline{L}_M\overline{CE}\quad  \overline{L}_I\overline{E}_I\overline{L}_AE_A\quad S_UE_U\overline{L}_B\overline{L}_O\\
&=\,\,0\,\,\,\,\,0\,\,\,\,\,1\,\,\,\,\,0\quad\,\, 0\,\,\,\,1\,\,\,\,1\,\,\,\,0\quad \quad 0\,\,\,\,0\,\,\,\,1\,\,\,\,1
\end{align*}

\begin{figure}
\centering
\begin{subfigure}{0.30\textwidth}
\centering
\begin{tikzpicture}
\pgfmathsetmacro{\klshift}{0.25}
\pgfmathsetmacro{\knshift}{0.07}
\pgfmathsetmacro{\kmv}{0.15}
\pgfmathsetmacro{\knshift}{0.07}
\pgfmathsetmacro{\kpin}{0.30}
\pgfmathsetmacro{\kpina}{0.30}
\pgfmathsetmacro{\kpsep}{0.15}			%pin to pin distance
\pgfmathsetmacro{\kW}{\kpsep}
\pgfmathsetmacro{\kulV}{0.20}			%edge clearance along vertical edge
\pgfmathsetmacro{\kulH}{0.20}
\pgfmathsetmacro{\kdimX}{2*\kulH+4*\kpsep}
\pgfmathsetmacro{\kdimY}{2*\kulV+4*\kpsep}		%two spaces between 3 pins
\pgfmathsetmacro{\ksepX}{\kdimX+2*\kpina+1*\kpsep+\kW}
\pgfmathsetmacro{\ksepY}{\kdimY+\kpin+0.5*\kpsep}
\draw(0,0)  [thick] rectangle ++(\kdimX,\kdimY)node[pos=0.5,rectangle,inner sep=0pt,text width=1.5cm,align=center]{\RTL{گنتکار}};
\draw(0,\kulV+0*\kpsep)--++(-\kpin,0)node[left]{$E_P$};
\draw(\kdimX,\kulV+1.5*\kpsep)--++(\kpina,0);
\draw(\kdimX,\kulV+2.5*\kpsep)--++(\kpina,0);
\draw(\kdimX,\kulV+1.5*\kpsep)++(\kpina,0)++(-0.5*\kpsep,-0.5*\kpsep)--++(1*\kpsep,1*\kpsep)--++(-1*\kpsep,1*\kpsep);

\draw(0,-\ksepY)  [thick] rectangle ++(\kdimX,\kdimY)node[pos=0.5,rectangle,inner sep=0pt,text width=1cm,align=center]{\RTL{دفتر پتہ}};
\draw(0,-\ksepY+\kulV+4*\kpsep)--++(-\kpin,0)node[left]{$\overline{L}_M$};
\draw(0,-\ksepY+\kulV+4*\kpsep)++(-\knshift,0)node[ocirc]{};
\draw[inactivePart](\kulH+0*\kpsep,-\ksepY)--++(0,-\kpin);
\draw[inactivePart](\kulH+1*\kpsep,-\ksepY)--++(0,-\kpin);
\draw[inactivePart](\kulH+0*\kpsep,-\ksepY)++(0,-\kpin)++(-0.5*\kpsep,0.5*\kpsep)--++(\kpsep,-\kpsep)--++(\kpsep,\kpsep);
\draw[inactivePart](\kulH+3*\kpsep,-\ksepY)--++(0,-\kpin);
\draw[inactivePart](\kulH+4*\kpsep,-\ksepY)--++(0,-\kpin);
\draw[inactivePart](\kulH+3*\kpsep,-\ksepY)++(0,-\kpin)++(-0.5*\kpsep,0.5*\kpsep)--++(\kpsep,-\kpsep)--++(\kpsep,\kpsep);
\draw(\kdimX,-\ksepY+\kulV+1.5*\kpsep)++(0.5*\kpsep,0)--++(\kpina,0);
\draw(\kdimX,-\ksepY+\kulV+2.5*\kpsep)++(0.5*\kpsep,0)--++(\kpina,0);
\draw(\kdimX,-\ksepY+\kulV+2*\kpsep)--++(1*\kpsep,-1*\kpsep);
\draw(\kdimX,-\ksepY+\kulV+2*\kpsep)--++(1*\kpsep,1*\kpsep);

\draw [thick,inactivePart](0,-2*\ksepY)  rectangle ++(\kdimX,\kdimY);%node[pos=0.5,rectangle,inner sep=0pt,text width=1cm,align=center]{\footnotesize{\RTL{حافظہ}}};
\draw[inactivePart](\kdimX,-2*\ksepY+\kulV+1.5*\kpsep)--++(\kpina,0);
\draw[inactivePart](\kdimX,-2*\ksepY+\kulV+2.5*\kpsep)--++(\kpina,0);
\draw[inactivePart](\kdimX,-2*\ksepY+\kulV+1.5*\kpsep)++(\kpina,0)++(-0.5*\kpsep,-0.5*\kpsep)--++(1*\kpsep,1*\kpsep)--++(-1*\kpsep,1*\kpsep);

\draw[thick,inactivePart](0,-3*\ksepY)  rectangle ++(\kdimX,\kdimY);%node[pos=0.5,rectangle,inner sep=0pt,text width=1cm,align=center]{\RTL{دفتر ہدایت}};
\draw[inactivePart](\kdimX,-3*\ksepY+\kulV+0*\kpsep)--++(\kpina,0);
\draw[inactivePart](\kdimX,-3*\ksepY+\kulV+1*\kpsep)--++(\kpina,0);
\draw[inactivePart](\kdimX,-3*\ksepY+\kulV+0*\kpsep)++(\kpina,0)++(-0.5*\kpsep,-0.5*\kpsep)--++(1*\kpsep,1*\kpsep)--++(-1*\kpsep,1*\kpsep);
\draw[inactivePart](\kdimX,-3*\ksepY+\kulV+3*\kpsep)++(0.5*\kpsep,0)--++(\kpina,0);
\draw[inactivePart](\kdimX,-3*\ksepY+\kulV+4*\kpsep)++(0.5*\kpsep,0)--++(\kpina,0);
\draw[inactivePart](\kdimX,-3*\ksepY+\kulV+3.5*\kpsep)--++(1*\kpsep,-1*\kpsep);
\draw[inactivePart](\kdimX,-3*\ksepY+\kulV+3.5*\kpsep)--++(1*\kpsep,1*\kpsep);
\draw[inactivePart](\kulH+1.5*\kpsep,-3*\ksepY)--++(0,-\kpin);
\draw[inactivePart](\kulH+2.5*\kpsep,-3*\ksepY)--++(0,-\kpin);
\draw[inactivePart](\kulH+1.5*\kpsep,-3*\ksepY)++(0,-\kpin)++(-0.5*\kpsep,0.5*\kpsep)--++(\kpsep,-\kpsep)--++(\kpsep,\kpsep);

\draw(0,-4*\ksepY)  [thick] rectangle ++(\kdimX,\kdimY)node[pos=0.5,rectangle,inner sep=0pt,text width=1.5cm,align=center]{\RTL{قابو    }};
\draw(\kulH+1.5*\kpsep,-4*\ksepY)--++(0,-\kpin);
\draw(\kulH+2.5*\kpsep,-4*\ksepY)--++(0,-\kpin);
\draw(\kulH+1.5*\kpsep,-4*\ksepY)++(0,-\kpin)++(-0.5*\kpsep,0.5*\kpsep)--++(\kpsep,-\kpsep)--++(\kpsep,\kpsep);
\draw(\kulH+2*\kpsep,-4*\ksepY)++(0,-1*\kpin-\kpsep)node[below,rectangle,inner sep=0pt, text width=1cm]{\RTL{قابو لفظ}};

\draw(\kdimX+\kpina+0.5*\kpsep,\kdimY)  [thick] rectangle ++(\kW,-4*\ksepY+\kpin);

\begin{scope}[inactivePart]
\draw(\ksepX,-0*\ksepY)  [thick] rectangle ++(\kdimX,\kdimY);%node[pos=0.5,rectangle,inner sep=0pt,text width=1cm,align=center]{\RTL{$A$}};
\draw(\ksepX,-0*\ksepY+\kulV+0*\kpsep)--++(-\kpina,0); 
\draw(\ksepX,-0*\ksepY+\kulV+1*\kpsep)--++(-\kpina,0); 
\draw(\ksepX,-0*\ksepY+\kulV+0*\kpsep)++(-\kpina,0)++(0.5*\kpsep,-0.5*\kpsep)--++(-\kpsep,\kpsep)--++(\kpsep,\kpsep);
\draw(\ksepX,-0*\ksepY+\kulV+3*\kpsep)++(-0.5*\kpsep,0)--++(-\kpina,0); 
\draw(\ksepX,-0*\ksepY+\kulV+4*\kpsep)++(-0.5*\kpsep,0)--++(-\kpina,0); 
\draw(\ksepX,-0*\ksepY+\kulV+3*\kpsep)++(-1*\kpsep,-0.5*\kpsep)--++(\kpsep,\kpsep)--++(-\kpsep,\kpsep);
\draw(\ksepX+\kulH+1.5*\kpsep,-0*\ksepY)--++(0,-\kpin);
\draw(\ksepX+\kulH+2.5*\kpsep,-0*\ksepY)--++(0,-\kpin);
\draw(\ksepX+\kulH+1.5*\kpsep,-0*\ksepY)++(0,-\kpin)++(-0.5*\kpsep,0.5*\kpsep)--++(\kpsep,-\kpsep)--++(\kpsep,\kpsep);

\draw(\ksepX,-1*\ksepY)  [thick] rectangle ++(\kdimX,\kdimY);%node[pos=0.5,rectangle,inner sep=0pt,text width=1cm,align=center]{\RTL{جمع و منفی کار}};
\draw(\ksepX,-1*\ksepY+\kulV+1.5*\kpsep)--++(-\kpina,0); 
\draw(\ksepX,-1*\ksepY+\kulV+2.5*\kpsep)--++(-\kpina,0); 
\draw(\ksepX,-1*\ksepY+\kulV+1.5*\kpsep)++(-\kpina,0)++(0.5*\kpsep,-0.5*\kpsep)--++(-\kpsep,\kpsep)--++(\kpsep,\kpsep);
\draw(\ksepX+\kulH+1.5*\kpsep,-1*\ksepY)++(0,-0.5*\kpsep)--++(0,-\kpin);
\draw(\ksepX+\kulH+2.5*\kpsep,-1*\ksepY)++(0,-0.5*\kpsep)--++(0,-\kpin);
\draw(\ksepX+\kulH+1.5*\kpsep,-1*\ksepY)++(-0.5*\kpsep,-\kpsep)--++(\kpsep,\kpsep)--++(\kpsep,-\kpsep);


\draw(\ksepX,-2*\ksepY)  [thick] rectangle ++(\kdimX,\kdimY);%node[pos=0.5,rectangle,inner sep=0pt,text width=1cm,align=center]{\RTL{$B$}};
\draw(\ksepX,-2*\ksepY+\kulV+1.5*\kpsep)++(-0.5*\kpsep,0)--++(-\kpina,0); 
\draw(\ksepX,-2*\ksepY+\kulV+2.5*\kpsep)++(-0.5*\kpsep,0)--++(-\kpina,0); 
\draw(\ksepX,-2*\ksepY+\kulV+1.5*\kpsep)++(-1*\kpsep,-0.5*\kpsep)--++(\kpsep,\kpsep)--++(-\kpsep,\kpsep);

\draw(\ksepX,-3*\ksepY)  [thick] rectangle ++(\kdimX,\kdimY);%node[pos=0.5,rectangle,inner sep=0pt,text width=1cm,align=center]{\RTL{خارجی دفتر}};
\draw(\ksepX,-3*\ksepY+\kulV+1.5*\kpsep)++(-0.5*\kpsep,0)--++(-\kpina,0); 
\draw(\ksepX,-3*\ksepY+\kulV+2.5*\kpsep)++(-0.5*\kpsep,0)--++(-\kpina,0); 
\draw(\ksepX,-3*\ksepY+\kulV+1.5*\kpsep)++(-1*\kpsep,-0.5*\kpsep)--++(\kpsep,\kpsep)--++(-\kpsep,\kpsep);
\draw(\ksepX,-3*\ksepY)++(\kulH+1.5*\kpsep,0)--++(0,-\kpin);
\draw(\ksepX,-3*\ksepY)++(\kulH+2.5*\kpsep,0)--++(0,-\kpin);
\draw(\ksepX,-3*\ksepY)++(\kulH+1.5*\kpsep,0)++(0,-\kpin)++(-0.5*\kpsep,0.5*\kpsep)--++(\kpsep,-\kpsep)--++(\kpsep,\kpsep);

\draw(\ksepX,-4*\ksepY)  [thick] rectangle ++(\kdimX,\kdimY);%node[pos=0.5,rectangle,inner sep=0pt,text width=1cm,align=center]{\RTL{تختی}};
\end{scope}
\end{tikzpicture}
\caption{}
\end{subfigure}\hfill
\begin{subfigure}{0.30\textwidth}
\centering
\begin{tikzpicture}
\pgfmathsetmacro{\klshift}{0.25}
\pgfmathsetmacro{\knshift}{0.07}
\pgfmathsetmacro{\kmv}{0.15}
\pgfmathsetmacro{\knshift}{0.07}
\pgfmathsetmacro{\kpin}{0.30}
\pgfmathsetmacro{\kpina}{0.30}
\pgfmathsetmacro{\kpsep}{0.15}			%pin to pin distance
\pgfmathsetmacro{\kW}{\kpsep}
\pgfmathsetmacro{\kulV}{0.20}			%edge clearance along vertical edge
\pgfmathsetmacro{\kulH}{0.20}
\pgfmathsetmacro{\kdimX}{2*\kulH+4*\kpsep}
\pgfmathsetmacro{\kdimY}{2*\kulV+4*\kpsep}		%two spaces between 3 pins
\pgfmathsetmacro{\ksepX}{\kdimX+2*\kpina+1*\kpsep+\kW}
\pgfmathsetmacro{\ksepY}{\kdimY+\kpin+0.5*\kpsep}
\draw(0,0)  [thick] rectangle ++(\kdimX,\kdimY)node[pos=0.5,rectangle,inner sep=0pt,text width=1.5cm,align=center]{\RTL{گنتکار}};
\draw(0,\kulV+4*\kpsep)--++(-\kpin,0)node[left]{$C_P$};
\draw[inactivePart](\kdimX,\kulV+1.5*\kpsep)--++(\kpina,0);
\draw[inactivePart](\kdimX,\kulV+2.5*\kpsep)--++(\kpina,0);
\draw[inactivePart](\kdimX,\kulV+1.5*\kpsep)++(\kpina,0)++(-0.5*\kpsep,-0.5*\kpsep)--++(1*\kpsep,1*\kpsep)--++(-1*\kpsep,1*\kpsep);

\begin{scope}[inactivePart]
\draw(0,-\ksepY)  [thick] rectangle ++(\kdimX,\kdimY);%node[pos=0.5,rectangle,inner sep=0pt,text width=1cm,align=center]{\RTL{دفتر پتہ}};
\draw(\kulH+0*\kpsep,-\ksepY)--++(0,-\kpin);
\draw(\kulH+1*\kpsep,-\ksepY)--++(0,-\kpin);
\draw(\kulH+0*\kpsep,-\ksepY)++(0,-\kpin)++(-0.5*\kpsep,0.5*\kpsep)--++(\kpsep,-\kpsep)--++(\kpsep,\kpsep);
\draw(\kulH+3*\kpsep,-\ksepY)--++(0,-\kpin);
\draw(\kulH+4*\kpsep,-\ksepY)--++(0,-\kpin);
\draw(\kulH+3*\kpsep,-\ksepY)++(0,-\kpin)++(-0.5*\kpsep,0.5*\kpsep)--++(\kpsep,-\kpsep)--++(\kpsep,\kpsep);
\draw(\kdimX,-\ksepY+\kulV+1.5*\kpsep)++(0.5*\kpsep,0)--++(\kpina,0);
\draw(\kdimX,-\ksepY+\kulV+2.5*\kpsep)++(0.5*\kpsep,0)--++(\kpina,0);
\draw(\kdimX,-\ksepY+\kulV+2*\kpsep)--++(1*\kpsep,-1*\kpsep);
\draw(\kdimX,-\ksepY+\kulV+2*\kpsep)--++(1*\kpsep,1*\kpsep);

\draw(0,-2*\ksepY)  [thick] rectangle ++(\kdimX,\kdimY);%node[pos=0.5,rectangle,inner sep=0pt,text width=1cm,align=center]{\footnotesize{\RTL{حافظہ}}};
\draw(\kdimX,-2*\ksepY+\kulV+1.5*\kpsep)--++(\kpina,0);
\draw(\kdimX,-2*\ksepY+\kulV+2.5*\kpsep)--++(\kpina,0);
\draw(\kdimX,-2*\ksepY+\kulV+1.5*\kpsep)++(\kpina,0)++(-0.5*\kpsep,-0.5*\kpsep)--++(1*\kpsep,1*\kpsep)--++(-1*\kpsep,1*\kpsep);

\draw(0,-3*\ksepY)  [thick] rectangle ++(\kdimX,\kdimY);%node[pos=0.5,rectangle,inner sep=0pt,text width=1cm,align=center]{ \RTL{دفتر ہدایت}};
\draw(\kdimX,-3*\ksepY+\kulV+0*\kpsep)--++(\kpina,0);
\draw(\kdimX,-3*\ksepY+\kulV+1*\kpsep)--++(\kpina,0);
\draw(\kdimX,-3*\ksepY+\kulV+0*\kpsep)++(\kpina,0)++(-0.5*\kpsep,-0.5*\kpsep)--++(1*\kpsep,1*\kpsep)--++(-1*\kpsep,1*\kpsep);
\draw(\kdimX,-3*\ksepY+\kulV+3*\kpsep)++(0.5*\kpsep,0)--++(\kpina,0);
\draw(\kdimX,-3*\ksepY+\kulV+4*\kpsep)++(0.5*\kpsep,0)--++(\kpina,0);
\draw(\kdimX,-3*\ksepY+\kulV+3.5*\kpsep)--++(1*\kpsep,-1*\kpsep);
\draw(\kdimX,-3*\ksepY+\kulV+3.5*\kpsep)--++(1*\kpsep,1*\kpsep);
\draw(\kulH+1.5*\kpsep,-3*\ksepY)--++(0,-\kpin);
\draw(\kulH+2.5*\kpsep,-3*\ksepY)--++(0,-\kpin);
\draw(\kulH+1.5*\kpsep,-3*\ksepY)++(0,-\kpin)++(-0.5*\kpsep,0.5*\kpsep)--++(\kpsep,-\kpsep)--++(\kpsep,\kpsep);
\end{scope}

\draw(0,-4*\ksepY)  [thick] rectangle ++(\kdimX,\kdimY)node[pos=0.5,rectangle,inner sep=0pt,text width=1.5cm,align=center]{\RTL{قابو    }};
\draw(\kulH+1.5*\kpsep,-4*\ksepY)--++(0,-\kpin);
\draw(\kulH+2.5*\kpsep,-4*\ksepY)--++(0,-\kpin);
\draw(\kulH+1.5*\kpsep,-4*\ksepY)++(0,-\kpin)++(-0.5*\kpsep,0.5*\kpsep)--++(\kpsep,-\kpsep)--++(\kpsep,\kpsep);
\draw(\kulH+2*\kpsep,-4*\ksepY)++(0,-1*\kpin-\kpsep)node[below,rectangle,inner sep=0pt, text width=1cm]{\RTL{قابو لفظ}};

\begin{scope}[inactivePart]
\draw(\kdimX+\kpina+0.5*\kpsep,\kdimY)  [thick] rectangle ++(\kW,-4*\ksepY+\kpin);

\draw(\ksepX,-0*\ksepY)  [thick] rectangle ++(\kdimX,\kdimY);%node[pos=0.5,rectangle,inner sep=0pt,text width=1cm,align=center]{\RTL{$A$}};
\draw(\ksepX,-0*\ksepY+\kulV+0*\kpsep)--++(-\kpina,0); 
\draw(\ksepX,-0*\ksepY+\kulV+1*\kpsep)--++(-\kpina,0); 
\draw(\ksepX,-0*\ksepY+\kulV+0*\kpsep)++(-\kpina,0)++(0.5*\kpsep,-0.5*\kpsep)--++(-\kpsep,\kpsep)--++(\kpsep,\kpsep);
\draw(\ksepX,-0*\ksepY+\kulV+3*\kpsep)++(-0.5*\kpsep,0)--++(-\kpina,0); 
\draw(\ksepX,-0*\ksepY+\kulV+4*\kpsep)++(-0.5*\kpsep,0)--++(-\kpina,0); 
\draw(\ksepX,-0*\ksepY+\kulV+3*\kpsep)++(-1*\kpsep,-0.5*\kpsep)--++(\kpsep,\kpsep)--++(-\kpsep,\kpsep);
\draw(\ksepX+\kulH+1.5*\kpsep,-0*\ksepY)--++(0,-\kpin);
\draw(\ksepX+\kulH+2.5*\kpsep,-0*\ksepY)--++(0,-\kpin);
\draw(\ksepX+\kulH+1.5*\kpsep,-0*\ksepY)++(0,-\kpin)++(-0.5*\kpsep,0.5*\kpsep)--++(\kpsep,-\kpsep)--++(\kpsep,\kpsep);

\draw(\ksepX,-1*\ksepY)  [thick] rectangle ++(\kdimX,\kdimY);%node[pos=0.5,rectangle,inner sep=0pt,text width=1cm,align=center]{\RTL{جمع و منفی کار}};
\draw(\ksepX,-1*\ksepY+\kulV+1.5*\kpsep)--++(-\kpina,0); 
\draw(\ksepX,-1*\ksepY+\kulV+2.5*\kpsep)--++(-\kpina,0); 
\draw(\ksepX,-1*\ksepY+\kulV+1.5*\kpsep)++(-\kpina,0)++(0.5*\kpsep,-0.5*\kpsep)--++(-\kpsep,\kpsep)--++(\kpsep,\kpsep);
\draw(\ksepX+\kulH+1.5*\kpsep,-1*\ksepY)++(0,-0.5*\kpsep)--++(0,-\kpin);
\draw(\ksepX+\kulH+2.5*\kpsep,-1*\ksepY)++(0,-0.5*\kpsep)--++(0,-\kpin);
\draw(\ksepX+\kulH+1.5*\kpsep,-1*\ksepY)++(-0.5*\kpsep,-\kpsep)--++(\kpsep,\kpsep)--++(\kpsep,-\kpsep);
\draw(\ksepX,-2*\ksepY)  [thick] rectangle ++(\kdimX,\kdimY);%node[pos=0.5,rectangle,inner sep=0pt,text width=1cm,align=center]{\RTL{$B$}};
\draw(\ksepX,-2*\ksepY+\kulV+1.5*\kpsep)++(-0.5*\kpsep,0)--++(-\kpina,0); 
\draw(\ksepX,-2*\ksepY+\kulV+2.5*\kpsep)++(-0.5*\kpsep,0)--++(-\kpina,0); 
\draw(\ksepX,-2*\ksepY+\kulV+1.5*\kpsep)++(-1*\kpsep,-0.5*\kpsep)--++(\kpsep,\kpsep)--++(-\kpsep,\kpsep);

\draw(\ksepX,-3*\ksepY)  [thick] rectangle ++(\kdimX,\kdimY);%node[pos=0.5,rectangle,inner sep=0pt,text width=1cm,align=center]{\RTL{خارجی دفتر}};
\draw(\ksepX,-3*\ksepY+\kulV+1.5*\kpsep)++(-0.5*\kpsep,0)--++(-\kpina,0); 
\draw(\ksepX,-3*\ksepY+\kulV+2.5*\kpsep)++(-0.5*\kpsep,0)--++(-\kpina,0); 
\draw(\ksepX,-3*\ksepY+\kulV+1.5*\kpsep)++(-1*\kpsep,-0.5*\kpsep)--++(\kpsep,\kpsep)--++(-\kpsep,\kpsep);
\draw(\ksepX,-3*\ksepY)++(\kulH+1.5*\kpsep,0)--++(0,-\kpin);
\draw(\ksepX,-3*\ksepY)++(\kulH+2.5*\kpsep,0)--++(0,-\kpin);
\draw(\ksepX,-3*\ksepY)++(\kulH+1.5*\kpsep,0)++(0,-\kpin)++(-0.5*\kpsep,0.5*\kpsep)--++(\kpsep,-\kpsep)--++(\kpsep,\kpsep);

\draw(\ksepX,-4*\ksepY)  [thick] rectangle ++(\kdimX,\kdimY);%node[pos=0.5,rectangle,inner sep=0pt,text width=1cm,align=center]{\RTL{تختی}};
\end{scope}
\end{tikzpicture}
\caption{}
\end{subfigure}\hfill
\begin{subfigure}{0.30\textwidth}
\centering
\begin{tikzpicture}
\pgfmathsetmacro{\klshift}{0.25}
\pgfmathsetmacro{\knshift}{0.07}
\pgfmathsetmacro{\kmv}{0.15}
\pgfmathsetmacro{\knshift}{0.07}
\pgfmathsetmacro{\kpin}{0.30}
\pgfmathsetmacro{\kpina}{0.30}
\pgfmathsetmacro{\kpsep}{0.15}			%pin to pin distance
\pgfmathsetmacro{\kW}{\kpsep}
\pgfmathsetmacro{\kulV}{0.20}			%edge clearance along vertical edge
\pgfmathsetmacro{\kulH}{0.20}
\pgfmathsetmacro{\kdimX}{2*\kulH+4*\kpsep}
\pgfmathsetmacro{\kdimY}{2*\kulV+4*\kpsep}		%two spaces between 3 pins
\pgfmathsetmacro{\ksepX}{\kdimX+2*\kpina+1*\kpsep+\kW}
\pgfmathsetmacro{\ksepY}{\kdimY+\kpin+0.5*\kpsep}

\begin{scope}[inactivePart]
\draw(0,0)  [thick] rectangle ++(\kdimX,\kdimY);%node[pos=0.5,rectangle,inner sep=0pt,text width=1.5cm,align=center]{\RTL{گنتکار}};
\draw(\kdimX,\kulV+1.5*\kpsep)--++(\kpina,0);
\draw(\kdimX,\kulV+2.5*\kpsep)--++(\kpina,0);
\draw(\kdimX,\kulV+1.5*\kpsep)++(\kpina,0)++(-0.5*\kpsep,-0.5*\kpsep)--++(1*\kpsep,1*\kpsep)--++(-1*\kpsep,1*\kpsep);
\end{scope}

\draw(0,-\ksepY)  [thick] rectangle ++(\kdimX,\kdimY)node[pos=0.5,rectangle,inner sep=0pt,text width=1cm,align=center]{\RTL{دفتر پتہ}};
\draw(\kulH+0*\kpsep,-\ksepY)--++(0,-\kpin);
\draw(\kulH+1*\kpsep,-\ksepY)--++(0,-\kpin);
\draw(\kulH+0*\kpsep,-\ksepY)++(0,-\kpin)++(-0.5*\kpsep,0.5*\kpsep)--++(\kpsep,-\kpsep)--++(\kpsep,\kpsep);
\draw(\kulH+3*\kpsep,-\ksepY)--++(0,-\kpin);
\draw(\kulH+4*\kpsep,-\ksepY)--++(0,-\kpin);
\draw(\kulH+3*\kpsep,-\ksepY)++(0,-\kpin)++(-0.5*\kpsep,0.5*\kpsep)--++(\kpsep,-\kpsep)--++(\kpsep,\kpsep);
\draw[inactivePart](\kdimX,-\ksepY+\kulV+1.5*\kpsep)++(0.5*\kpsep,0)--++(\kpina,0);
\draw[inactivePart](\kdimX,-\ksepY+\kulV+2.5*\kpsep)++(0.5*\kpsep,0)--++(\kpina,0);
\draw[inactivePart](\kdimX,-\ksepY+\kulV+2*\kpsep)--++(1*\kpsep,-1*\kpsep);
\draw[inactivePart](\kdimX,-\ksepY+\kulV+2*\kpsep)--++(1*\kpsep,1*\kpsep);

\draw(0,-2*\ksepY)  [thick] rectangle ++(\kdimX,\kdimY)node[pos=0.5,rectangle,inner sep=0pt,text width=1cm,align=center]{\footnotesize{\RTL{حافظہ}}};
\draw(0,-2*\ksepY+\kulV)--++(-\kpin,0)node[left]{$\overline{CE}$}; 
\draw(0,-2*\ksepY+\kulV)++(-\knshift,0)node[ocirc]{};
\draw(\kdimX,-2*\ksepY+\kulV+1.5*\kpsep)--++(\kpina,0);
\draw(\kdimX,-2*\ksepY+\kulV+2.5*\kpsep)--++(\kpina,0);
\draw(\kdimX,-2*\ksepY+\kulV+1.5*\kpsep)++(\kpina,0)++(-0.5*\kpsep,-0.5*\kpsep)--++(1*\kpsep,1*\kpsep)--++(-1*\kpsep,1*\kpsep);

\draw(0,-3*\ksepY)  [thick] rectangle ++(\kdimX,\kdimY)node[pos=0.5,rectangle,inner sep=0pt,text width=1cm,align=center]{ \RTL{دفتر ہدایت}};
\draw(0,-3*\ksepY+\kulV+4*\kpsep)--++(-\kpin,0)node[left]{$\overline{L}_I$};
\draw(0,-3*\ksepY+\kulV+4*\kpsep)++(-\knshift,0)node[ocirc]{};
\draw[inactivePart](\kdimX,-3*\ksepY+\kulV+0*\kpsep)--++(\kpina,0);
\draw[inactivePart](\kdimX,-3*\ksepY+\kulV+1*\kpsep)--++(\kpina,0);
\draw[inactivePart](\kdimX,-3*\ksepY+\kulV+0*\kpsep)++(\kpina,0)++(-0.5*\kpsep,-0.5*\kpsep)--++(1*\kpsep,1*\kpsep)--++(-1*\kpsep,1*\kpsep);
\draw(\kdimX,-3*\ksepY+\kulV+3*\kpsep)++(0.5*\kpsep,0)--++(\kpina,0);
\draw(\kdimX,-3*\ksepY+\kulV+4*\kpsep)++(0.5*\kpsep,0)--++(\kpina,0);
\draw(\kdimX,-3*\ksepY+\kulV+3.5*\kpsep)--++(1*\kpsep,-1*\kpsep);
\draw(\kdimX,-3*\ksepY+\kulV+3.5*\kpsep)--++(1*\kpsep,1*\kpsep);
\draw[inactivePart](\kulH+1.5*\kpsep,-3*\ksepY)--++(0,-\kpin);
\draw[inactivePart](\kulH+2.5*\kpsep,-3*\ksepY)--++(0,-\kpin);
\draw[inactivePart](\kulH+1.5*\kpsep,-3*\ksepY)++(0,-\kpin)++(-0.5*\kpsep,0.5*\kpsep)--++(\kpsep,-\kpsep)--++(\kpsep,\kpsep);

\draw(0,-4*\ksepY)  [thick] rectangle ++(\kdimX,\kdimY)node[pos=0.5,rectangle,inner sep=0pt,text width=1.5cm,align=center]{\RTL{قابو    }};
\draw(\kulH+1.5*\kpsep,-4*\ksepY)--++(0,-\kpin);
\draw(\kulH+2.5*\kpsep,-4*\ksepY)--++(0,-\kpin);
\draw(\kulH+1.5*\kpsep,-4*\ksepY)++(0,-\kpin)++(-0.5*\kpsep,0.5*\kpsep)--++(\kpsep,-\kpsep)--++(\kpsep,\kpsep);
\draw(\kulH+2*\kpsep,-4*\ksepY)++(0,-1*\kpin-\kpsep)node[below,rectangle,inner sep=0pt, text width=1cm]{\RTL{قابو لفظ}};

\draw(\kdimX+\kpina+0.5*\kpsep,\kdimY)  [thick] rectangle ++(\kW,-4*\ksepY+\kpin);

\begin{scope}[inactivePart]
\draw(\ksepX,-0*\ksepY)  [thick] rectangle ++(\kdimX,\kdimY);%node[pos=0.5,rectangle,inner sep=0pt,text width=1cm,align=center]{\RTL{$A$}};
\draw(\ksepX,-0*\ksepY+\kulV+0*\kpsep)--++(-\kpina,0); 
\draw(\ksepX,-0*\ksepY+\kulV+1*\kpsep)--++(-\kpina,0); 
\draw(\ksepX,-0*\ksepY+\kulV+0*\kpsep)++(-\kpina,0)++(0.5*\kpsep,-0.5*\kpsep)--++(-\kpsep,\kpsep)--++(\kpsep,\kpsep);
\draw(\ksepX,-0*\ksepY+\kulV+3*\kpsep)++(-0.5*\kpsep,0)--++(-\kpina,0); 
\draw(\ksepX,-0*\ksepY+\kulV+4*\kpsep)++(-0.5*\kpsep,0)--++(-\kpina,0); 
\draw(\ksepX,-0*\ksepY+\kulV+3*\kpsep)++(-1*\kpsep,-0.5*\kpsep)--++(\kpsep,\kpsep)--++(-\kpsep,\kpsep);
\draw(\ksepX+\kulH+1.5*\kpsep,-0*\ksepY)--++(0,-\kpin);
\draw(\ksepX+\kulH+2.5*\kpsep,-0*\ksepY)--++(0,-\kpin);
\draw(\ksepX+\kulH+1.5*\kpsep,-0*\ksepY)++(0,-\kpin)++(-0.5*\kpsep,0.5*\kpsep)--++(\kpsep,-\kpsep)--++(\kpsep,\kpsep);
\draw(\ksepX,-1*\ksepY)  [thick] rectangle ++(\kdimX,\kdimY);%node[pos=0.5,rectangle,inner sep=0pt,text width=1cm,align=center]{\RTL{جمع و منفی کار}};
\draw(\ksepX,-1*\ksepY+\kulV+1.5*\kpsep)--++(-\kpina,0); 
\draw(\ksepX,-1*\ksepY+\kulV+2.5*\kpsep)--++(-\kpina,0); 
\draw(\ksepX,-1*\ksepY+\kulV+1.5*\kpsep)++(-\kpina,0)++(0.5*\kpsep,-0.5*\kpsep)--++(-\kpsep,\kpsep)--++(\kpsep,\kpsep);
\draw(\ksepX+\kulH+1.5*\kpsep,-1*\ksepY)++(0,-0.5*\kpsep)--++(0,-\kpin);
\draw(\ksepX+\kulH+2.5*\kpsep,-1*\ksepY)++(0,-0.5*\kpsep)--++(0,-\kpin);
\draw(\ksepX+\kulH+1.5*\kpsep,-1*\ksepY)++(-0.5*\kpsep,-\kpsep)--++(\kpsep,\kpsep)--++(\kpsep,-\kpsep);


\draw(\ksepX,-2*\ksepY)  [thick] rectangle ++(\kdimX,\kdimY);%node[pos=0.5,rectangle,inner sep=0pt,text width=1cm,align=center]{\RTL{$B$}};
\draw(\ksepX,-2*\ksepY+\kulV+1.5*\kpsep)++(-0.5*\kpsep,0)--++(-\kpina,0); 
\draw(\ksepX,-2*\ksepY+\kulV+2.5*\kpsep)++(-0.5*\kpsep,0)--++(-\kpina,0); 
\draw(\ksepX,-2*\ksepY+\kulV+1.5*\kpsep)++(-1*\kpsep,-0.5*\kpsep)--++(\kpsep,\kpsep)--++(-\kpsep,\kpsep);

\draw(\ksepX,-3*\ksepY)  [thick] rectangle ++(\kdimX,\kdimY);%node[pos=0.5,rectangle,inner sep=0pt,text width=1cm,align=center]{\RTL{خارجی دفتر}};
\draw(\ksepX,-3*\ksepY+\kulV+1.5*\kpsep)++(-0.5*\kpsep,0)--++(-\kpina,0); 
\draw(\ksepX,-3*\ksepY+\kulV+2.5*\kpsep)++(-0.5*\kpsep,0)--++(-\kpina,0); 
\draw(\ksepX,-3*\ksepY+\kulV+1.5*\kpsep)++(-1*\kpsep,-0.5*\kpsep)--++(\kpsep,\kpsep)--++(-\kpsep,\kpsep);
\draw(\ksepX,-3*\ksepY)++(\kulH+1.5*\kpsep,0)--++(0,-\kpin);
\draw(\ksepX,-3*\ksepY)++(\kulH+2.5*\kpsep,0)--++(0,-\kpin);
\draw(\ksepX,-3*\ksepY)++(\kulH+1.5*\kpsep,0)++(0,-\kpin)++(-0.5*\kpsep,0.5*\kpsep)--++(\kpsep,-\kpsep)--++(\kpsep,\kpsep);

\draw(\ksepX,-4*\ksepY)  [thick] rectangle ++(\kdimX,\kdimY);%node[pos=0.5,rectangle,inner sep=0pt,text width=1cm,align=center]{\RTL{تختی}};
\end{scope}
\end{tikzpicture}
\caption{}
\end{subfigure}
\caption{
بازیابی پھیرا: (ا) \عددی{T_1} حال؛ (ب) \عددی{T_2} حال؛ (ج) \عددی{T_3} حال۔
}
\label{شکل_کمپیوٹر_اجاگر_حصے_بازیابی_پھیرا}
\end{figure}


\جزوحصہء{بازیابی پھیرا}
پتہ حال، بڑھوتری حال، اور حافظہ حال مل کر \اصطلاح{ بازیابی پھیرا }\فرہنگ{پھیرا!بازیابی}\حاشیہب{fetch cycle}\فرہنگ{cycle!fetch} دیتے ہیں۔ پتہ حال کے دوران \عددی{E_P} اور \عددی{\overline{L}_M} فعال ہوں گے؛ یوں برنامہ گنت کار   \عددی{W} گزرگاہ کے ذریعہ  دفتر پتہ کو  تیار کرتا ہے۔ جیسا شکل  \حوالہ{شکل_کمپیوٹر_چھلا}-ج   میں دکھایا گیا،  ساعت کا مثبت کنارہ  نصف  پتہ حال  گزرنے کے بعد (یعنی  پتہ حال کے وسط میں)  آ تا ہے؛ اور یوں  گنت کار کی معلومات دفتر پتہ میں  درج کرتا ہے۔

بڑھوتری حال کے دوران صرف \عددی{C_P} قابو بِٹ فعال ہو گا۔ یہ بِٹ برنامہ گنت کار  کو ساعت کے مثبت کنارہ گننے کی اجازت دیتا ہے۔ بڑھوتری حال کے وسط میں ساعت کا مثبت کنارہ آئے گا ، جو برنامہ گنت کار کی گنتی میں  \عددی{1}  کا اضافہ کرے گا۔

حافظہ حال کے دوران \عددی{\overline{CE}} اور \عددی{\overline{L}_I} فعال ہوں گے۔یوں،  حافظہ کے مقام پتہ پر موجود  لفظ کی رسائی ، \عددی{W} گزرگاہ کے ذریعہ ، دفتر ہدایت تک    ہو گی۔ حافظہ حال کے وسط میں   ساعت کا  آنے والا مثبت کنارہ  دفتر ہدایت میں یہ لفظ  درج کرتا ہے۔

\حصہ{تعمیلی پھیرا}
اگلے تین حال ( \عددی{T_4}، \عددی{T_5}، اور \عددی{T_6} ) کمپیوٹر کا\اصطلاح{ تعمیلی پھیرا  }\فرہنگ{پھیرا!تعمیلی}\حاشیہب{execution cycle}\فرہنگ{cycle!execution}کہلاتے   ہیں۔تعمیلی پھیرا  کے دوران دفاتر میں معلومات کا انتقال  اس ہدایت پر منحصر ہے جس کی تعمیل کی جا رہی ہو۔ مثلاً،  \قول{نقل \عددی{9H}}  کی تعمیل کے دوران دفاتر میں معلومات کا انتقال \قول{جمع \عددی{BH}}  کی تعمیل کے دوران دفاتر میں معلومات کے انتقال  سے مختلف ہو گا۔آئیں     اب مختلف ہدایات  کی تعمیل  کے لئے\قول{ قابو  طریقہ کار } پر غور کریں۔

\جزوحصہء{طریق نقل}
اس گفتگو کو آگے بڑھانے کے لئے فرض کریں دفتر ہدایت میں نقل \عددی{9H} بھرا گیا ہے۔
\begin{align*}
0000\, 1001=\text{\RL{دفتر ہدایت}}
\end{align*}
جزو ہدایت \عددی{0000}  قابو و ترتیب کار  کو   \عددی{T_4} حال کے دوران جاتا ہے، جہاں  اس کی  رمز کشائی  ہو گی؛  جزو پتہ \عددی{1001} دفتر پتہ  میں ڈالا جاتا ہے۔ شکل   \حوالہ{شکل_کمپیوٹر_اجاگر_حصے_تعمیلی_پھیرا}-الف میں \عددی{T_4} حال کے دوران فعال حصے اجاگر کیے گئے ہیں۔ جیسا آپ دیکھ سکتے ہیں، \عددی{\overline{E}_I} اور \عددی{\overline{L}_M} فعال ہیں، جبکہ باقی تمام قابو بِٹ غیر فعال ہیں۔

دوران \عددی{T_5} حال ، \عددی{\overline{CE}} اور \عددی{\overline{L}_A} پست ہوں گے۔ یوں ساعت کے اگلے کنارہ  چڑھائی پر  حافظہ کے مقام پتہ  سے  مواد  کا لفظ دفتر الف  میں نقل  ہو گا (شکل  \حوالہ{شکل_کمپیوٹر_اجاگر_حصے_تعمیلی_پھیرا}-ب  دیکھیں)۔

\عددی{T_6}\اصطلاح{ فارغ حال }\فرہنگ{حال!فارغ}\حاشیہب{nop, no operation}\فرہنگ{nop} ہے۔اس (تیسرے تعمیلی ) حال کے دوران تمام دفاتر غیر فعال ہیں (شکل  \حوالہ{شکل_کمپیوٹر_اجاگر_حصے_تعمیلی_پھیرا}-ج دیکھیں)۔ یوں قابو و ترتیب کار ایسا قابو لفظ خارج کرتا ہے جس کے تمام بِٹ غیر فعال ہوں گے۔  فارغ حال میں  کوئی کام سرانجام نہیں ہو گا۔

\begin{figure}
\centering
\begin{subfigure}{0.30\textwidth}
\centering
\begin{tikzpicture}
\pgfmathsetmacro{\klshift}{0.25}
\pgfmathsetmacro{\knshift}{0.07}
\pgfmathsetmacro{\kmv}{0.15}
\pgfmathsetmacro{\knshift}{0.07}
\pgfmathsetmacro{\kpin}{0.30}
\pgfmathsetmacro{\kpina}{0.30}
\pgfmathsetmacro{\kpsep}{0.15}			%pin to pin distance
\pgfmathsetmacro{\kW}{\kpsep}
\pgfmathsetmacro{\kulV}{0.20}			%edge clearance along vertical edge
\pgfmathsetmacro{\kulH}{0.20}
\pgfmathsetmacro{\kdimX}{2*\kulH+4*\kpsep}
\pgfmathsetmacro{\kdimY}{2*\kulV+4*\kpsep}		%two spaces between 3 pins
\pgfmathsetmacro{\ksepX}{\kdimX+2*\kpina+1*\kpsep+\kW}
\pgfmathsetmacro{\ksepY}{\kdimY+\kpin+0.5*\kpsep}

\begin{scope}[inactivePart]
\draw(0,0)  [thick] rectangle ++(\kdimX,\kdimY);%node[pos=0.5,rectangle,inner sep=0pt,text width=1.5cm,align=center]{\RTL{گنتکار}};
\draw(\kdimX,\kulV+1.5*\kpsep)--++(\kpina,0);
\draw(\kdimX,\kulV+2.5*\kpsep)--++(\kpina,0);
\draw(\kdimX,\kulV+1.5*\kpsep)++(\kpina,0)++(-0.5*\kpsep,-0.5*\kpsep)--++(1*\kpsep,1*\kpsep)--++(-1*\kpsep,1*\kpsep);
\end{scope}

\draw(0,-\ksepY)  [thick] rectangle ++(\kdimX,\kdimY)node[pos=0.5,rectangle,inner sep=0pt,text width=1cm,align=center]{\RTL{دفتر پتہ}};
\draw(0,-\ksepY+\kulV+4*\kpsep)--++(-\kpin,0)node[left]{$\overline{L}_M$};
\draw(0,-\ksepY+\kulV+4*\kpsep)++(-\knshift,0)node[ocirc]{};
\draw[inactivePart](\kulH+0*\kpsep,-\ksepY)--++(0,-\kpin);
\draw[inactivePart](\kulH+1*\kpsep,-\ksepY)--++(0,-\kpin);
\draw[inactivePart](\kulH+0*\kpsep,-\ksepY)++(0,-\kpin)++(-0.5*\kpsep,0.5*\kpsep)--++(\kpsep,-\kpsep)--++(\kpsep,\kpsep);
\draw[inactivePart](\kulH+3*\kpsep,-\ksepY)--++(0,-\kpin);
\draw[inactivePart](\kulH+4*\kpsep,-\ksepY)--++(0,-\kpin);
\draw[inactivePart](\kulH+3*\kpsep,-\ksepY)++(0,-\kpin)++(-0.5*\kpsep,0.5*\kpsep)--++(\kpsep,-\kpsep)--++(\kpsep,\kpsep);
\draw(\kdimX,-\ksepY+\kulV+1.5*\kpsep)++(0.5*\kpsep,0)--++(\kpina,0);
\draw(\kdimX,-\ksepY+\kulV+2.5*\kpsep)++(0.5*\kpsep,0)--++(\kpina,0);
\draw(\kdimX,-\ksepY+\kulV+2*\kpsep)--++(1*\kpsep,-1*\kpsep);
\draw(\kdimX,-\ksepY+\kulV+2*\kpsep)--++(1*\kpsep,1*\kpsep);

\begin{scope}[inactivePart]
\draw [thick,](0,-2*\ksepY)  rectangle ++(\kdimX,\kdimY);%node[pos=0.5,rectangle,inner sep=0pt,text width=1cm,align=center]{\footnotesize{\RTL{حافظہ}}};
\draw(\kdimX,-2*\ksepY+\kulV+1.5*\kpsep)--++(\kpina,0);
\draw(\kdimX,-2*\ksepY+\kulV+2.5*\kpsep)--++(\kpina,0);
\draw(\kdimX,-2*\ksepY+\kulV+1.5*\kpsep)++(\kpina,0)++(-0.5*\kpsep,-0.5*\kpsep)--++(1*\kpsep,1*\kpsep)--++(-1*\kpsep,1*\kpsep);
\end{scope}

\draw[thick,](0,-3*\ksepY)  rectangle ++(\kdimX,\kdimY)node[pos=0.5,rectangle,inner sep=0pt,text width=1cm,align=center]{\RTL{دفتر ہدایت}};
\draw(0,-3*\ksepY+\kulV+0*\kpsep)--++(-\kpin,0)node[left]{$\overline{E}_I$};
\draw(0,-3*\ksepY+\kulV+0*\kpsep)++(-0.07,0)node[ocirc]{};
\draw(\kdimX,-3*\ksepY+\kulV+0*\kpsep)--++(\kpina,0);
\draw(\kdimX,-3*\ksepY+\kulV+1*\kpsep)--++(\kpina,0);
\draw(\kdimX,-3*\ksepY+\kulV+0*\kpsep)++(\kpina,0)++(-0.5*\kpsep,-0.5*\kpsep)--++(1*\kpsep,1*\kpsep)--++(-1*\kpsep,1*\kpsep);
\draw[inactivePart](\kdimX,-3*\ksepY+\kulV+3*\kpsep)++(0.5*\kpsep,0)--++(\kpina,0);
\draw[inactivePart](\kdimX,-3*\ksepY+\kulV+4*\kpsep)++(0.5*\kpsep,0)--++(\kpina,0);
\draw[inactivePart](\kdimX,-3*\ksepY+\kulV+3.5*\kpsep)--++(1*\kpsep,-1*\kpsep);
\draw[inactivePart](\kdimX,-3*\ksepY+\kulV+3.5*\kpsep)--++(1*\kpsep,1*\kpsep);
\draw(\kulH+1.5*\kpsep,-3*\ksepY)--++(0,-\kpin);
\draw(\kulH+2.5*\kpsep,-3*\ksepY)--++(0,-\kpin);
\draw(\kulH+1.5*\kpsep,-3*\ksepY)++(0,-\kpin)++(-0.5*\kpsep,0.5*\kpsep)--++(\kpsep,-\kpsep)--++(\kpsep,\kpsep);

\draw(0,-4*\ksepY)  [thick] rectangle ++(\kdimX,\kdimY)node[pos=0.5,rectangle,inner sep=0pt,text width=1.5cm,align=center]{\RTL{قابو    }};
\draw(\kulH+1.5*\kpsep,-4*\ksepY)--++(0,-\kpin);
\draw(\kulH+2.5*\kpsep,-4*\ksepY)--++(0,-\kpin);
\draw(\kulH+1.5*\kpsep,-4*\ksepY)++(0,-\kpin)++(-0.5*\kpsep,0.5*\kpsep)--++(\kpsep,-\kpsep)--++(\kpsep,\kpsep);
\draw(\kulH+2*\kpsep,-4*\ksepY)++(0,-1*\kpin-\kpsep)node[below,rectangle,inner sep=0pt, text width=1cm]{\RTL{قابو لفظ}};

\draw(\kdimX+\kpina+0.5*\kpsep,\kdimY)  [thick] rectangle ++(\kW,-4*\ksepY+\kpin); %گزرگاہ

\begin{scope}[inactivePart]
\draw(\ksepX,-0*\ksepY)  [thick] rectangle ++(\kdimX,\kdimY);%node[pos=0.5,rectangle,inner sep=0pt,text width=1cm,align=center]{\RTL{$A$}};
\draw(\ksepX,-0*\ksepY+\kulV+0*\kpsep)--++(-\kpina,0); 
\draw(\ksepX,-0*\ksepY+\kulV+1*\kpsep)--++(-\kpina,0); 
\draw(\ksepX,-0*\ksepY+\kulV+0*\kpsep)++(-\kpina,0)++(0.5*\kpsep,-0.5*\kpsep)--++(-\kpsep,\kpsep)--++(\kpsep,\kpsep);
\draw(\ksepX,-0*\ksepY+\kulV+3*\kpsep)++(-0.5*\kpsep,0)--++(-\kpina,0); 
\draw(\ksepX,-0*\ksepY+\kulV+4*\kpsep)++(-0.5*\kpsep,0)--++(-\kpina,0); 
\draw(\ksepX,-0*\ksepY+\kulV+3*\kpsep)++(-1*\kpsep,-0.5*\kpsep)--++(\kpsep,\kpsep)--++(-\kpsep,\kpsep);
\draw(\ksepX+\kulH+1.5*\kpsep,-0*\ksepY)--++(0,-\kpin);
\draw(\ksepX+\kulH+2.5*\kpsep,-0*\ksepY)--++(0,-\kpin);
\draw(\ksepX+\kulH+1.5*\kpsep,-0*\ksepY)++(0,-\kpin)++(-0.5*\kpsep,0.5*\kpsep)--++(\kpsep,-\kpsep)--++(\kpsep,\kpsep);

\draw(\ksepX,-1*\ksepY)  [thick] rectangle ++(\kdimX,\kdimY);%node[pos=0.5,rectangle,inner sep=0pt,text width=1cm,align=center]{\RTL{جمع و منفی کار}};
\draw(\ksepX,-1*\ksepY+\kulV+1.5*\kpsep)--++(-\kpina,0); 
\draw(\ksepX,-1*\ksepY+\kulV+2.5*\kpsep)--++(-\kpina,0); 
\draw(\ksepX,-1*\ksepY+\kulV+1.5*\kpsep)++(-\kpina,0)++(0.5*\kpsep,-0.5*\kpsep)--++(-\kpsep,\kpsep)--++(\kpsep,\kpsep);
\draw(\ksepX+\kulH+1.5*\kpsep,-1*\ksepY)++(0,-0.5*\kpsep)--++(0,-\kpin);
\draw(\ksepX+\kulH+2.5*\kpsep,-1*\ksepY)++(0,-0.5*\kpsep)--++(0,-\kpin);
\draw(\ksepX+\kulH+1.5*\kpsep,-1*\ksepY)++(-0.5*\kpsep,-\kpsep)--++(\kpsep,\kpsep)--++(\kpsep,-\kpsep);


\draw(\ksepX,-2*\ksepY)  [thick] rectangle ++(\kdimX,\kdimY);%node[pos=0.5,rectangle,inner sep=0pt,text width=1cm,align=center]{\RTL{$B$}};
\draw(\ksepX,-2*\ksepY+\kulV+1.5*\kpsep)++(-0.5*\kpsep,0)--++(-\kpina,0); 
\draw(\ksepX,-2*\ksepY+\kulV+2.5*\kpsep)++(-0.5*\kpsep,0)--++(-\kpina,0); 
\draw(\ksepX,-2*\ksepY+\kulV+1.5*\kpsep)++(-1*\kpsep,-0.5*\kpsep)--++(\kpsep,\kpsep)--++(-\kpsep,\kpsep);

\draw(\ksepX,-3*\ksepY)  [thick] rectangle ++(\kdimX,\kdimY);%node[pos=0.5,rectangle,inner sep=0pt,text width=1cm,align=center]{\RTL{خارجی دفتر}};
\draw(\ksepX,-3*\ksepY+\kulV+1.5*\kpsep)++(-0.5*\kpsep,0)--++(-\kpina,0); 
\draw(\ksepX,-3*\ksepY+\kulV+2.5*\kpsep)++(-0.5*\kpsep,0)--++(-\kpina,0); 
\draw(\ksepX,-3*\ksepY+\kulV+1.5*\kpsep)++(-1*\kpsep,-0.5*\kpsep)--++(\kpsep,\kpsep)--++(-\kpsep,\kpsep);
\draw(\ksepX,-3*\ksepY)++(\kulH+1.5*\kpsep,0)--++(0,-\kpin);
\draw(\ksepX,-3*\ksepY)++(\kulH+2.5*\kpsep,0)--++(0,-\kpin);
\draw(\ksepX,-3*\ksepY)++(\kulH+1.5*\kpsep,0)++(0,-\kpin)++(-0.5*\kpsep,0.5*\kpsep)--++(\kpsep,-\kpsep)--++(\kpsep,\kpsep);

\draw(\ksepX,-4*\ksepY)  [thick] rectangle ++(\kdimX,\kdimY);%node[pos=0.5,rectangle,inner sep=0pt,text width=1cm,align=center]{\RTL{تختی}};
\end{scope}
\end{tikzpicture}
\caption{}
\end{subfigure}\hfill
\begin{subfigure}{0.30\textwidth}
\centering
\begin{tikzpicture}
\pgfmathsetmacro{\klshift}{0.25}
\pgfmathsetmacro{\knshift}{0.07}
\pgfmathsetmacro{\kmv}{0.15}
\pgfmathsetmacro{\knshift}{0.07}
\pgfmathsetmacro{\kpin}{0.30}
\pgfmathsetmacro{\kpina}{0.30}
\pgfmathsetmacro{\kpsep}{0.15}			%pin to pin distance
\pgfmathsetmacro{\kW}{\kpsep}
\pgfmathsetmacro{\kulV}{0.20}			%edge clearance along vertical edge
\pgfmathsetmacro{\kulH}{0.20}
\pgfmathsetmacro{\kdimX}{2*\kulH+4*\kpsep}
\pgfmathsetmacro{\kdimY}{2*\kulV+4*\kpsep}		%two spaces between 3 pins
\pgfmathsetmacro{\ksepX}{\kdimX+2*\kpina+1*\kpsep+\kW}
\pgfmathsetmacro{\ksepY}{\kdimY+\kpin+0.5*\kpsep}

\begin{scope}[inactivePart]
\draw(0,0)  [thick] rectangle ++(\kdimX,\kdimY);%node[pos=0.5,rectangle,inner sep=0pt,text width=1.5cm,align=center]{\RTL{گنتکار}};
\draw(\kdimX,\kulV+1.5*\kpsep)--++(\kpina,0);
\draw(\kdimX,\kulV+2.5*\kpsep)--++(\kpina,0);
\draw(\kdimX,\kulV+1.5*\kpsep)++(\kpina,0)++(-0.5*\kpsep,-0.5*\kpsep)--++(1*\kpsep,1*\kpsep)--++(-1*\kpsep,1*\kpsep);
\end{scope}

\draw(0,-\ksepY)  [thick] rectangle ++(\kdimX,\kdimY)node[pos=0.5,rectangle,inner sep=0pt,text width=1cm,align=center]{\RTL{دفتر پتہ}};
\draw(\kulH+0*\kpsep,-\ksepY)--++(0,-\kpin);
\draw(\kulH+1*\kpsep,-\ksepY)--++(0,-\kpin);
\draw(\kulH+0*\kpsep,-\ksepY)++(0,-\kpin)++(-0.5*\kpsep,0.5*\kpsep)--++(\kpsep,-\kpsep)--++(\kpsep,\kpsep);
\draw(\kulH+3*\kpsep,-\ksepY)--++(0,-\kpin);
\draw(\kulH+4*\kpsep,-\ksepY)--++(0,-\kpin);
\draw(\kulH+3*\kpsep,-\ksepY)++(0,-\kpin)++(-0.5*\kpsep,0.5*\kpsep)--++(\kpsep,-\kpsep)--++(\kpsep,\kpsep);
\draw[inactivePart](\kdimX,-\ksepY+\kulV+1.5*\kpsep)++(0.5*\kpsep,0)--++(\kpina,0);
\draw[inactivePart](\kdimX,-\ksepY+\kulV+2.5*\kpsep)++(0.5*\kpsep,0)--++(\kpina,0);
\draw[inactivePart](\kdimX,-\ksepY+\kulV+2*\kpsep)--++(1*\kpsep,-1*\kpsep);
\draw[inactivePart](\kdimX,-\ksepY+\kulV+2*\kpsep)--++(1*\kpsep,1*\kpsep);

\draw(0,-2*\ksepY)  [thick] rectangle ++(\kdimX,\kdimY)node[pos=0.5,rectangle,inner sep=0pt,text width=1cm,align=center]{\footnotesize{\RTL{حافظہ}}};
\draw(0,-2*\ksepY+\kulV+0*\kpsep)--++(-\kpin,0)node[left]{$\overline{CE}$};
\draw(0,-2*\ksepY+\kulV+0*\kpsep)++(-0.07,0)node[ocirc]{};
\draw(\kdimX,-2*\ksepY+\kulV+1.5*\kpsep)--++(\kpina,0);
\draw(\kdimX,-2*\ksepY+\kulV+2.5*\kpsep)--++(\kpina,0);
\draw(\kdimX,-2*\ksepY+\kulV+1.5*\kpsep)++(\kpina,0)++(-0.5*\kpsep,-0.5*\kpsep)--++(1*\kpsep,1*\kpsep)--++(-1*\kpsep,1*\kpsep);

\draw(0,-3*\ksepY)  [thick] rectangle ++(\kdimX,\kdimY)node[pos=0.5,rectangle,inner sep=0pt,text width=1cm,align=center]{ \RTL{دفتر ہدایت}};
\draw[inactivePart](\kdimX,-3*\ksepY+\kulV+0*\kpsep)--++(\kpina,0);
\draw[inactivePart](\kdimX,-3*\ksepY+\kulV+1*\kpsep)--++(\kpina,0);
\draw[inactivePart](\kdimX,-3*\ksepY+\kulV+0*\kpsep)++(\kpina,0)++(-0.5*\kpsep,-0.5*\kpsep)--++(1*\kpsep,1*\kpsep)--++(-1*\kpsep,1*\kpsep);
\draw[inactivePart](\kdimX,-3*\ksepY+\kulV+3*\kpsep)++(0.5*\kpsep,0)--++(\kpina,0);
\draw[inactivePart](\kdimX,-3*\ksepY+\kulV+4*\kpsep)++(0.5*\kpsep,0)--++(\kpina,0);
\draw[inactivePart](\kdimX,-3*\ksepY+\kulV+3.5*\kpsep)--++(1*\kpsep,-1*\kpsep);
\draw[inactivePart](\kdimX,-3*\ksepY+\kulV+3.5*\kpsep)--++(1*\kpsep,1*\kpsep);
\draw(\kulH+1.5*\kpsep,-3*\ksepY)--++(0,-\kpin);
\draw(\kulH+2.5*\kpsep,-3*\ksepY)--++(0,-\kpin);
\draw(\kulH+1.5*\kpsep,-3*\ksepY)++(0,-\kpin)++(-0.5*\kpsep,0.5*\kpsep)--++(\kpsep,-\kpsep)--++(\kpsep,\kpsep);


\draw(0,-4*\ksepY)  [thick] rectangle ++(\kdimX,\kdimY)node[pos=0.5,rectangle,inner sep=0pt,text width=1.5cm,align=center]{\RTL{قابو    }};
\draw(\kulH+1.5*\kpsep,-4*\ksepY)--++(0,-\kpin);
\draw(\kulH+2.5*\kpsep,-4*\ksepY)--++(0,-\kpin);
\draw(\kulH+1.5*\kpsep,-4*\ksepY)++(0,-\kpin)++(-0.5*\kpsep,0.5*\kpsep)--++(\kpsep,-\kpsep)--++(\kpsep,\kpsep);
\draw(\kulH+2*\kpsep,-4*\ksepY)++(0,-1*\kpin-\kpsep)node[below,rectangle,inner sep=0pt, text width=1cm]{\RTL{قابو لفظ}};


\draw(\kdimX+\kpina+0.5*\kpsep,\kdimY)  [thick] rectangle ++(\kW,-4*\ksepY+\kpin);

\draw(\ksepX,-0*\ksepY)  [thick] rectangle ++(\kdimX,\kdimY)node[pos=0.5,rectangle,inner sep=0pt,text width=1cm,align=center]{\RTL{$A$}};
\draw(\ksepX+\kdimX,-0*\ksepY+\kulV+4*\kpsep)--++(\kpin,0)node[right]{$\overline{L}_A$}; 
\draw(\ksepX+\kdimX,-0*\ksepY+\kulV+4*\kpsep)++(0.07,0)node[ocirc]{};
\draw[inactivePart](\ksepX,-0*\ksepY+\kulV+0*\kpsep)--++(-\kpina,0); 
\draw[inactivePart](\ksepX,-0*\ksepY+\kulV+1*\kpsep)--++(-\kpina,0); 
\draw[inactivePart](\ksepX,-0*\ksepY+\kulV+0*\kpsep)++(-\kpina,0)++(0.5*\kpsep,-0.5*\kpsep)--++(-\kpsep,\kpsep)--++(\kpsep,\kpsep);
\draw(\ksepX,-0*\ksepY+\kulV+3*\kpsep)++(-0.5*\kpsep,0)--++(-\kpina,0); 
\draw(\ksepX,-0*\ksepY+\kulV+4*\kpsep)++(-0.5*\kpsep,0)--++(-\kpina,0); 
\draw(\ksepX,-0*\ksepY+\kulV+3*\kpsep)++(-1*\kpsep,-0.5*\kpsep)--++(\kpsep,\kpsep)--++(-\kpsep,\kpsep);
\draw[inactivePart](\ksepX+\kulH+1.5*\kpsep,-0*\ksepY)--++(0,-\kpin);
\draw[inactivePart](\ksepX+\kulH+2.5*\kpsep,-0*\ksepY)--++(0,-\kpin);
\draw[inactivePart](\ksepX+\kulH+1.5*\kpsep,-0*\ksepY)++(0,-\kpin)++(-0.5*\kpsep,0.5*\kpsep)--++(\kpsep,-\kpsep)--++(\kpsep,\kpsep);

\begin{scope}[inactivePart]
\draw(\ksepX,-1*\ksepY)  [thick] rectangle ++(\kdimX,\kdimY);%node[pos=0.5,rectangle,inner sep=0pt,text width=1cm,align=center]{\RTL{جمع و منفی کار}};
\draw(\ksepX,-1*\ksepY+\kulV+1.5*\kpsep)--++(-\kpina,0); 
\draw(\ksepX,-1*\ksepY+\kulV+2.5*\kpsep)--++(-\kpina,0); 
\draw(\ksepX,-1*\ksepY+\kulV+1.5*\kpsep)++(-\kpina,0)++(0.5*\kpsep,-0.5*\kpsep)--++(-\kpsep,\kpsep)--++(\kpsep,\kpsep);
\draw(\ksepX+\kulH+1.5*\kpsep,-1*\ksepY)++(0,-0.5*\kpsep)--++(0,-\kpin);
\draw(\ksepX+\kulH+2.5*\kpsep,-1*\ksepY)++(0,-0.5*\kpsep)--++(0,-\kpin);
\draw(\ksepX+\kulH+1.5*\kpsep,-1*\ksepY)++(-0.5*\kpsep,-\kpsep)--++(\kpsep,\kpsep)--++(\kpsep,-\kpsep);
\draw(\ksepX,-2*\ksepY)  [thick] rectangle ++(\kdimX,\kdimY);%node[pos=0.5,rectangle,inner sep=0pt,text width=1cm,align=center]{\RTL{$B$}};
\draw(\ksepX,-2*\ksepY+\kulV+1.5*\kpsep)++(-0.5*\kpsep,0)--++(-\kpina,0); 
\draw(\ksepX,-2*\ksepY+\kulV+2.5*\kpsep)++(-0.5*\kpsep,0)--++(-\kpina,0); 
\draw(\ksepX,-2*\ksepY+\kulV+1.5*\kpsep)++(-1*\kpsep,-0.5*\kpsep)--++(\kpsep,\kpsep)--++(-\kpsep,\kpsep);

\draw(\ksepX,-3*\ksepY)  [thick] rectangle ++(\kdimX,\kdimY);%node[pos=0.5,rectangle,inner sep=0pt,text width=1cm,align=center]{\RTL{خارجی دفتر}};
\draw(\ksepX,-3*\ksepY+\kulV+1.5*\kpsep)++(-0.5*\kpsep,0)--++(-\kpina,0); 
\draw(\ksepX,-3*\ksepY+\kulV+2.5*\kpsep)++(-0.5*\kpsep,0)--++(-\kpina,0); 
\draw(\ksepX,-3*\ksepY+\kulV+1.5*\kpsep)++(-1*\kpsep,-0.5*\kpsep)--++(\kpsep,\kpsep)--++(-\kpsep,\kpsep);
\draw(\ksepX,-3*\ksepY)++(\kulH+1.5*\kpsep,0)--++(0,-\kpin);
\draw(\ksepX,-3*\ksepY)++(\kulH+2.5*\kpsep,0)--++(0,-\kpin);
\draw(\ksepX,-3*\ksepY)++(\kulH+1.5*\kpsep,0)++(0,-\kpin)++(-0.5*\kpsep,0.5*\kpsep)--++(\kpsep,-\kpsep)--++(\kpsep,\kpsep);

\draw(\ksepX,-4*\ksepY)  [thick] rectangle ++(\kdimX,\kdimY);%node[pos=0.5,rectangle,inner sep=0pt,text width=1cm,align=center]{\RTL{تختی}};
\end{scope}
\end{tikzpicture}
\caption{}
\end{subfigure}\hfill
\begin{subfigure}{0.30\textwidth}
\centering
\begin{tikzpicture}
\pgfmathsetmacro{\klshift}{0.25}
\pgfmathsetmacro{\knshift}{0.07}
\pgfmathsetmacro{\kmv}{0.15}
\pgfmathsetmacro{\knshift}{0.07}
\pgfmathsetmacro{\kpin}{0.30}
\pgfmathsetmacro{\kpina}{0.30}
\pgfmathsetmacro{\kpsep}{0.15}			%pin to pin distance
\pgfmathsetmacro{\kW}{\kpsep}
\pgfmathsetmacro{\kulV}{0.20}			%edge clearance along vertical edge
\pgfmathsetmacro{\kulH}{0.20}
\pgfmathsetmacro{\kdimX}{2*\kulH+4*\kpsep}
\pgfmathsetmacro{\kdimY}{2*\kulV+4*\kpsep}		%two spaces between 3 pins
\pgfmathsetmacro{\ksepX}{\kdimX+2*\kpina+1*\kpsep+\kW}
\pgfmathsetmacro{\ksepY}{\kdimY+\kpin+0.5*\kpsep}

\begin{scope}[inactivePart]
\draw(0,0)  [thick] rectangle ++(\kdimX,\kdimY);%node[pos=0.5,rectangle,inner sep=0pt,text width=1.5cm,align=center]{\RTL{گنتکار}};
\draw(\kdimX,\kulV+1.5*\kpsep)--++(\kpina,0);
\draw(\kdimX,\kulV+2.5*\kpsep)--++(\kpina,0);
\draw(\kdimX,\kulV+1.5*\kpsep)++(\kpina,0)++(-0.5*\kpsep,-0.5*\kpsep)--++(1*\kpsep,1*\kpsep)--++(-1*\kpsep,1*\kpsep);


\draw(0,-\ksepY)  [thick] rectangle ++(\kdimX,\kdimY);%node[pos=0.5,rectangle,inner sep=0pt,text width=1cm,align=center]{\RTL{دفتر پتہ}};
\draw(\kulH+0*\kpsep,-\ksepY)--++(0,-\kpin);
\draw(\kulH+1*\kpsep,-\ksepY)--++(0,-\kpin);
\draw(\kulH+0*\kpsep,-\ksepY)++(0,-\kpin)++(-0.5*\kpsep,0.5*\kpsep)--++(\kpsep,-\kpsep)--++(\kpsep,\kpsep);
\draw(\kulH+3*\kpsep,-\ksepY)--++(0,-\kpin);
\draw(\kulH+4*\kpsep,-\ksepY)--++(0,-\kpin);
\draw(\kulH+3*\kpsep,-\ksepY)++(0,-\kpin)++(-0.5*\kpsep,0.5*\kpsep)--++(\kpsep,-\kpsep)--++(\kpsep,\kpsep);
\draw(\kdimX,-\ksepY+\kulV+1.5*\kpsep)++(0.5*\kpsep,0)--++(\kpina,0);
\draw(\kdimX,-\ksepY+\kulV+2.5*\kpsep)++(0.5*\kpsep,0)--++(\kpina,0);
\draw(\kdimX,-\ksepY+\kulV+2*\kpsep)--++(1*\kpsep,-1*\kpsep);
\draw(\kdimX,-\ksepY+\kulV+2*\kpsep)--++(1*\kpsep,1*\kpsep);

\draw(0,-2*\ksepY)  [thick] rectangle ++(\kdimX,\kdimY);%node[pos=0.5,rectangle,inner sep=0pt,text width=1cm,align=center]{\footnotesize{\RTL{حافظہ}}};
\draw(\kdimX,-2*\ksepY+\kulV+1.5*\kpsep)--++(\kpina,0);
\draw(\kdimX,-2*\ksepY+\kulV+2.5*\kpsep)--++(\kpina,0);
\draw(\kdimX,-2*\ksepY+\kulV+1.5*\kpsep)++(\kpina,0)++(-0.5*\kpsep,-0.5*\kpsep)--++(1*\kpsep,1*\kpsep)--++(-1*\kpsep,1*\kpsep);

\draw(0,-3*\ksepY)  [thick] rectangle ++(\kdimX,\kdimY);%node[pos=0.5,rectangle,inner sep=0pt,text width=1cm,align=center]{ \RTL{دفتر ہدایت}};
\draw(\kdimX,-3*\ksepY+\kulV+0*\kpsep)--++(\kpina,0);
\draw(\kdimX,-3*\ksepY+\kulV+1*\kpsep)--++(\kpina,0);
\draw(\kdimX,-3*\ksepY+\kulV+0*\kpsep)++(\kpina,0)++(-0.5*\kpsep,-0.5*\kpsep)--++(1*\kpsep,1*\kpsep)--++(-1*\kpsep,1*\kpsep);
\draw(\kdimX,-3*\ksepY+\kulV+3*\kpsep)++(0.5*\kpsep,0)--++(\kpina,0);
\draw(\kdimX,-3*\ksepY+\kulV+4*\kpsep)++(0.5*\kpsep,0)--++(\kpina,0);
\draw(\kdimX,-3*\ksepY+\kulV+3.5*\kpsep)--++(1*\kpsep,-1*\kpsep);
\draw(\kdimX,-3*\ksepY+\kulV+3.5*\kpsep)--++(1*\kpsep,1*\kpsep);
\draw(\kulH+1.5*\kpsep,-3*\ksepY)--++(0,-\kpin);
\draw(\kulH+2.5*\kpsep,-3*\ksepY)--++(0,-\kpin);
\draw(\kulH+1.5*\kpsep,-3*\ksepY)++(0,-\kpin)++(-0.5*\kpsep,0.5*\kpsep)--++(\kpsep,-\kpsep)--++(\kpsep,\kpsep);

\draw(0,-4*\ksepY)  [thick] rectangle ++(\kdimX,\kdimY);%node[pos=0.5,rectangle,inner sep=0pt,text width=1.5cm,align=center]{\RTL{قابو    }};
\end{scope}

\draw(\kulH+1.5*\kpsep,-4*\ksepY)--++(0,-\kpin);
\draw(\kulH+2.5*\kpsep,-4*\ksepY)--++(0,-\kpin);
\draw(\kulH+1.5*\kpsep,-4*\ksepY)++(0,-\kpin)++(-0.5*\kpsep,0.5*\kpsep)--++(\kpsep,-\kpsep)--++(\kpsep,\kpsep);
\draw(\kulH+2*\kpsep,-4*\ksepY)++(0,-1*\kpin-\kpsep)node[below,rectangle,inner sep=0pt, text width=1cm]{\RTL{قابو لفظ}};

\begin{scope}[inactivePart]
\draw(\kdimX+\kpina+0.5*\kpsep,\kdimY)  [thick] rectangle ++(\kW,-4*\ksepY+\kpin);


\draw(\ksepX,-0*\ksepY)  [thick] rectangle ++(\kdimX,\kdimY);%node[pos=0.5,rectangle,inner sep=0pt,text width=1cm,align=center]{\RTL{$A$}};
\draw(\ksepX,-0*\ksepY+\kulV+0*\kpsep)--++(-\kpina,0); 
\draw(\ksepX,-0*\ksepY+\kulV+1*\kpsep)--++(-\kpina,0); 
\draw(\ksepX,-0*\ksepY+\kulV+0*\kpsep)++(-\kpina,0)++(0.5*\kpsep,-0.5*\kpsep)--++(-\kpsep,\kpsep)--++(\kpsep,\kpsep);
\draw(\ksepX,-0*\ksepY+\kulV+3*\kpsep)++(-0.5*\kpsep,0)--++(-\kpina,0); 
\draw(\ksepX,-0*\ksepY+\kulV+4*\kpsep)++(-0.5*\kpsep,0)--++(-\kpina,0); 
\draw(\ksepX,-0*\ksepY+\kulV+3*\kpsep)++(-1*\kpsep,-0.5*\kpsep)--++(\kpsep,\kpsep)--++(-\kpsep,\kpsep);
\draw(\ksepX+\kulH+1.5*\kpsep,-0*\ksepY)--++(0,-\kpin);
\draw(\ksepX+\kulH+2.5*\kpsep,-0*\ksepY)--++(0,-\kpin);
\draw(\ksepX+\kulH+1.5*\kpsep,-0*\ksepY)++(0,-\kpin)++(-0.5*\kpsep,0.5*\kpsep)--++(\kpsep,-\kpsep)--++(\kpsep,\kpsep);
\draw(\ksepX,-1*\ksepY)  [thick] rectangle ++(\kdimX,\kdimY);%node[pos=0.5,rectangle,inner sep=0pt,text width=1cm,align=center]{\RTL{جمع و منفی کار}};
\draw(\ksepX,-1*\ksepY+\kulV+1.5*\kpsep)--++(-\kpina,0); 
\draw(\ksepX,-1*\ksepY+\kulV+2.5*\kpsep)--++(-\kpina,0); 
\draw(\ksepX,-1*\ksepY+\kulV+1.5*\kpsep)++(-\kpina,0)++(0.5*\kpsep,-0.5*\kpsep)--++(-\kpsep,\kpsep)--++(\kpsep,\kpsep);
\draw(\ksepX+\kulH+1.5*\kpsep,-1*\ksepY)++(0,-0.5*\kpsep)--++(0,-\kpin);
\draw(\ksepX+\kulH+2.5*\kpsep,-1*\ksepY)++(0,-0.5*\kpsep)--++(0,-\kpin);
\draw(\ksepX+\kulH+1.5*\kpsep,-1*\ksepY)++(-0.5*\kpsep,-\kpsep)--++(\kpsep,\kpsep)--++(\kpsep,-\kpsep);


\draw(\ksepX,-2*\ksepY)  [thick] rectangle ++(\kdimX,\kdimY);%node[pos=0.5,rectangle,inner sep=0pt,text width=1cm,align=center]{\RTL{$B$}};
\draw(\ksepX,-2*\ksepY+\kulV+1.5*\kpsep)++(-0.5*\kpsep,0)--++(-\kpina,0); 
\draw(\ksepX,-2*\ksepY+\kulV+2.5*\kpsep)++(-0.5*\kpsep,0)--++(-\kpina,0); 
\draw(\ksepX,-2*\ksepY+\kulV+1.5*\kpsep)++(-1*\kpsep,-0.5*\kpsep)--++(\kpsep,\kpsep)--++(-\kpsep,\kpsep);

\draw(\ksepX,-3*\ksepY)  [thick] rectangle ++(\kdimX,\kdimY);%node[pos=0.5,rectangle,inner sep=0pt,text width=1cm,align=center]{\RTL{خارجی دفتر}};
\draw(\ksepX,-3*\ksepY+\kulV+1.5*\kpsep)++(-0.5*\kpsep,0)--++(-\kpina,0); 
\draw(\ksepX,-3*\ksepY+\kulV+2.5*\kpsep)++(-0.5*\kpsep,0)--++(-\kpina,0); 
\draw(\ksepX,-3*\ksepY+\kulV+1.5*\kpsep)++(-1*\kpsep,-0.5*\kpsep)--++(\kpsep,\kpsep)--++(-\kpsep,\kpsep);
\draw(\ksepX,-3*\ksepY)++(\kulH+1.5*\kpsep,0)--++(0,-\kpin);
\draw(\ksepX,-3*\ksepY)++(\kulH+2.5*\kpsep,0)--++(0,-\kpin);
\draw(\ksepX,-3*\ksepY)++(\kulH+1.5*\kpsep,0)++(0,-\kpin)++(-0.5*\kpsep,0.5*\kpsep)--++(\kpsep,-\kpsep)--++(\kpsep,\kpsep);

\draw(\ksepX,-4*\ksepY)  [thick] rectangle ++(\kdimX,\kdimY);%node[pos=0.5,rectangle,inner sep=0pt,text width=1cm,align=center]{\RTL{تختی}};
\end{scope}
\end{tikzpicture}
\caption{}
\end{subfigure}
\caption{
طریق نقل: (ا) \عددی{T_4} حال؛ (ب) \عددی{T_5} حال؛ (ج) \عددی{T_6} حال۔
}
\label{شکل_کمپیوٹر_اجاگر_حصے_تعمیلی_پھیرا}
\end{figure}
%
\begin{figure}
\centering
\begin{otherlanguage}{english}
 \begin{tikztimingtable}[%
timing/.style={x=4ex,y=3ex},
timing/rowdist=6ex,
every node/.style={inner sep=0,outer sep=0},
%timing/c/arrow tip=latex, %and this set the style
%timing/c/rising arrows,
timing/slope=0, %0.1 is good
timing/dslope=0,
thick,
]
%\tikztimingmetachar{R}{[|/utils/exec=\setcounter{new}{0}|]}
%\usetikztiminglibrary[new={char=Q,reset char=R}]{counters}
%[timing/counter/new={char=c, base=2,digits=3,max value=7, wraps ,text style={font=\normalsize}}] 12{2c} \\ 
%$C$& H22{C}\\
%$\texturdu{\RL{پتہ}}$&3D{} 20D{[scale=1.5]\texturdu{\RL{درست پتہ}}} 3D{}\\
$CLK$&HN(a)LN(ca)HN(b)LN(cb)HN(c)LN(cc)HN(d)LN(cd)HN(e)LN(ce)HN(f)LN(cf)HN(g)L\\
$E_P$&LHHLLLLLLLLLLH\\
$\overline{L}_M$&HLL4{H}2{L}5{H}\\
$C_P$&LLLHHLLLLLLLLL\\
$\overline{CE}$&HHHHHLLHHLLHHH\\
$\overline{L}_I$&HHHHHLLHHHHHHH\\
$\overline{E}_I$&HHHHHHHLLHHHHH\\
$\overline{L}_A$&HHHHHHHHHLLHHH\\
\extracode
\begin{pgfonlayer}{background}
\begin{scope}[]
\draw [latex-latex] ($(a|-row1.north)+(0,2ex)$) --node[fill=white]{$T_1$} ($(b|-row1.north)+(0,2ex)$);
\draw [latex-latex] ($(b|-row1.north)+(0,2ex)$) --node[fill=white]{$T_2$} ($(c|-row1.north)+(0,2ex)$);
\draw [latex-latex] ($(c|-row1.north)+(0,2ex)$) --node[fill=white]{$T_3$} ($(d|-row1.north)+(0,2ex)$);
\draw [latex-latex] ($(d|-row1.north)+(0,2ex)$) --node[fill=white]{$T_4$} ($(e|-row1.north)+(0,2ex)$);
\draw [latex-latex] ($(e|-row1.north)+(0,2ex)$) --node[fill=white]{$T_5$} ($(f|-row1.north)+(0,2ex)$);
\draw [latex-latex] ($(f|-row1.north)+(0,2ex)$) --node[fill=white]{$T_6$} ($(g|-row1.north)+(0,2ex)$);
\foreach \n in {a,b,c,d,e,f,g}{\draw[thin]($(\n|-row1.north)+(0,1ex)$)--++(0,2ex);}
\draw[dashed] (ca)--(ca |-row3.south);
\draw[dashed] (cb)--(cb |-row4.north);
\draw[dashed] (cc)--(cc |-row6.south);
\draw[dashed] (cd)--(cd |-row7.south);
\draw[dashed] (ce)--(ce |-row8.south);
%%\vertlines[darkgray,dotted]{3.6,7.5,11.5,15.5,19.5,23.5,27.5}
%\foreach \n in {1,3,...,11} \draw(4*\n ex-2ex,-4ex+1.25ex)node[]{$0$};
%\foreach \n in {2,4,...,12} \draw(4*\n ex-2ex,-4ex+1.25ex)node[]{$1$};
%\foreach \n in {1,2,5,6,9,10} \draw(4*\n ex-2ex,-12 ex+1.25ex)node[]{$0$};
%\foreach \n in {3,4,7,8,11,12} \draw(4*\n ex-2ex,-12 ex+1.25ex)node[]{$1$};
%\foreach \n in {1,2,3,4,9,10,11,12} \draw(4*\n ex-2ex,-20 ex+1.25ex)node[]{$0$};
%\foreach \n in {5,6,7,8} \draw(4*\n ex-2ex,-20 ex+1.25ex)node[]{$1$};
%\draw(4*4 ex-2ex,-12 ex+1.25ex) circle (0.25cm and 1.75cm);
%\draw(4*4 ex-2ex,-7*\rowdist+1.25ex) circle (0.25cm and 0.5cm);
%%\foreach \n in {1}\draw(B\n.south)--(F\n.north);
%%\foreach \n in {1}\draw(C\n.south)--(G\n.north);
\end{scope}
\end{pgfonlayer}
\end{tikztimingtable}
\end{otherlanguage}
\caption{بازیابی اور نقل کی وقتیہ ترسیمات۔}
\label{شکل_کمپیوٹر_بازیابی_وقتیہ}
\end{figure}

شکل  \حوالہ{شکل_کمپیوٹر_بازیابی_وقتیہ}  میں بازیابی اور نقل طریق کی وقتیہ ترسیمات پیش ہیں۔ \عددی{T_1} حال  کے دوران \عددی{E_P} اور \عددی{\overline{L}_M}  فعال ہیں؛  اس حال کے وسط میں  ساعت کا آنے والا  کنارہ چڑھائی ، دفتر پتہ میں   برنامہ گنت کار سے پتہ  منتقل کرتا ہے۔ \عددی{T_2} حال  کے دوران  \عددی{C_P} فعال ہے لہٰذا ساعت کے کنارہ چڑھائی پر  برنامہ گنت کار  کی گنتی میں \عددی{1} کا اضافہ  ہو گا۔ \عددی{T_3} حال کے دوران \عددی{\overline{CE}} اور \عددی{\overline{L}_I} فعال ہیں؛ ساعت کے کنارہ  چڑھائی پر  دفتر ہدایت میں ، پتہ  کی نشاندہی پر حافظہ  کے مطلوبہ (    نشان زد )مقام سے ، لفظ بھرا جائے گا۔ \قول{نقل} کی ہدایت پر عمل درآمد   \عددی{T_4} حال سے شروع ہو گی، جہاں \عددی{\overline{L}_M} اور \عددی{\overline{E}_I} فعال  ہیں؛ دفتر ہدایت میں موجود   جزو پتہ ،  ساعت کے کنارہ چڑھائی پر،  دفتر پتہ میں  منتقل ہو گا۔ \عددی{T_5} حال کے دوران \عددی{\overline{CE}} اور \عددی{\overline{L}_A} فعال ہیں؛ دفتر الف میں، ساعت کے کنارہ چڑھائی پر،  حافظہ کے مطلوبہ مقام سے مواد کا لفظ بھرا جائے گا۔ \قول{نقل} ہدایت میں \عددی{T_6} حال   کچھ نہیں کرتا۔ ہم کہتے ہیں یہ فارغ حال ہے۔

\جزوحصہء{طریق جمع}
فرض کریں بازیابی پھیرا کے اختتام پر دفتر ہدایت میں \قول{جمع \عددی{BH}} پایا جاتا ہے۔
\begin{align*}
0001\,1011=\text{\RL{دفتر ہدایت}}
\end{align*}
دوران \عددی{T_4} حال قابو و ترتیب کار کو   جزو ہدایت  اور دفتر پتہ کو جزو پتہ جائے گا (شکل \حوالہ{شکل_کمپیوٹر_جمع_اور_منفی}-الف  دیکھیں)۔ اس حال کے دوران \عددی{\overline{E}_I} اور \عددی{\overline{L}_M} فعال ہوں گے۔

\عددی{T_5} حال کے دوران قابو بِٹ \عددی{\overline{CE}} اور \عددی{\overline{L}_B} فعال ہوں گے۔یوں  پتہ  کی نشاندہی کے مقام پر لفظ    حافظہ  سے  دفتر \عددی{B} میں  لکھا جا سکتا ہے (شکل \حوالہ{شکل_کمپیوٹر_جمع_اور_منفی}-ب)۔ ہمیشہ کی طرح ،   اس حال کے وسط میں آنے والے ساعت کے کنارہ چڑھائی پر مواد دفتر \عددی{B} میں منتقل ہو گا۔

\عددی{T_6} حال  کے دوران، \عددی{E_U} اور \عددی{\overline{L}_A} فعال ہو ں گے؛ لہٰذا دفتر \عددی{A} تک جمع و منفی کار کا مخارج پہنچے گا (شکل  \حوالہ{شکل_کمپیوٹر_جمع_اور_منفی}-ج)۔ اس حال کے وسط میں جمع و منفی کار  کا مخارج دفتر \عددی{A} منتقل ہو گا۔

\begin{figure}
\centering
\begin{subfigure}{0.30\textwidth}
\centering
\begin{tikzpicture}
\pgfmathsetmacro{\klshift}{0.25}
\pgfmathsetmacro{\knshift}{0.07}
\pgfmathsetmacro{\kmv}{0.15}
\pgfmathsetmacro{\knshift}{0.07}
\pgfmathsetmacro{\kpin}{0.30}
\pgfmathsetmacro{\kpina}{0.30}
\pgfmathsetmacro{\kpsep}{0.15}			%pin to pin distance
\pgfmathsetmacro{\kW}{\kpsep}
\pgfmathsetmacro{\kulV}{0.20}			%edge clearance along vertical edge
\pgfmathsetmacro{\kulH}{0.20}
\pgfmathsetmacro{\kdimX}{2*\kulH+4*\kpsep}
\pgfmathsetmacro{\kdimY}{2*\kulV+4*\kpsep}		%two spaces between 3 pins
\pgfmathsetmacro{\ksepX}{\kdimX+2*\kpina+1*\kpsep+\kW}
\pgfmathsetmacro{\ksepY}{\kdimY+\kpin+0.5*\kpsep}

\begin{scope}[inactivePart]
\draw(0,0)  [thick] rectangle ++(\kdimX,\kdimY);%node[pos=0.5,rectangle,inner sep=0pt,text width=1.5cm,align=center]{\RTL{گنتکار}};
\draw(\kdimX,\kulV+1.5*\kpsep)--++(\kpina,0);
\draw(\kdimX,\kulV+2.5*\kpsep)--++(\kpina,0);
\draw(\kdimX,\kulV+1.5*\kpsep)++(\kpina,0)++(-0.5*\kpsep,-0.5*\kpsep)--++(1*\kpsep,1*\kpsep)--++(-1*\kpsep,1*\kpsep);
\end{scope}

\draw(0,-\ksepY)  [thick] rectangle ++(\kdimX,\kdimY)node[pos=0.5,rectangle,inner sep=0pt,text width=1cm,align=center]{\RTL{دفتر پتہ}};
\draw(0,-\ksepY+\kulV+4*\kpsep)--++(-\kpin,0)node[left]{$\overline{L}_M$};
\draw(0,-\ksepY+\kulV+4*\kpsep)++(-\knshift,0)node[ocirc]{};
\draw[inactivePart](\kulH+0*\kpsep,-\ksepY)--++(0,-\kpin);
\draw[inactivePart](\kulH+1*\kpsep,-\ksepY)--++(0,-\kpin);
\draw[inactivePart](\kulH+0*\kpsep,-\ksepY)++(0,-\kpin)++(-0.5*\kpsep,0.5*\kpsep)--++(\kpsep,-\kpsep)--++(\kpsep,\kpsep);
\draw[inactivePart](\kulH+3*\kpsep,-\ksepY)--++(0,-\kpin);
\draw[inactivePart](\kulH+4*\kpsep,-\ksepY)--++(0,-\kpin);
\draw[inactivePart](\kulH+3*\kpsep,-\ksepY)++(0,-\kpin)++(-0.5*\kpsep,0.5*\kpsep)--++(\kpsep,-\kpsep)--++(\kpsep,\kpsep);
\draw(\kdimX,-\ksepY+\kulV+1.5*\kpsep)++(0.5*\kpsep,0)--++(\kpina,0);
\draw(\kdimX,-\ksepY+\kulV+2.5*\kpsep)++(0.5*\kpsep,0)--++(\kpina,0);
\draw(\kdimX,-\ksepY+\kulV+2*\kpsep)--++(1*\kpsep,-1*\kpsep);
\draw(\kdimX,-\ksepY+\kulV+2*\kpsep)--++(1*\kpsep,1*\kpsep);

\begin{scope}[inactivePart]
\draw [thick,](0,-2*\ksepY)  rectangle ++(\kdimX,\kdimY);%node[pos=0.5,rectangle,inner sep=0pt,text width=1cm,align=center]{\footnotesize{\RTL{حافظہ}}};
\draw(\kdimX,-2*\ksepY+\kulV+1.5*\kpsep)--++(\kpina,0);
\draw(\kdimX,-2*\ksepY+\kulV+2.5*\kpsep)--++(\kpina,0);
\draw(\kdimX,-2*\ksepY+\kulV+1.5*\kpsep)++(\kpina,0)++(-0.5*\kpsep,-0.5*\kpsep)--++(1*\kpsep,1*\kpsep)--++(-1*\kpsep,1*\kpsep);
\end{scope}

\draw[thick,](0,-3*\ksepY)  rectangle ++(\kdimX,\kdimY)node[pos=0.5,rectangle,inner sep=0pt,text width=1cm,align=center]{\RTL{دفتر ہدایت}};
\draw(0,-3*\ksepY+\kulV+0*\kpsep)--++(-\kpin,0)node[left]{$\overline{E}_I$};
\draw(0,-3*\ksepY+\kulV+0*\kpsep)++(-0.07,0)node[ocirc]{};
\draw(\kdimX,-3*\ksepY+\kulV+0*\kpsep)--++(\kpina,0);
\draw(\kdimX,-3*\ksepY+\kulV+1*\kpsep)--++(\kpina,0);
\draw(\kdimX,-3*\ksepY+\kulV+0*\kpsep)++(\kpina,0)++(-0.5*\kpsep,-0.5*\kpsep)--++(1*\kpsep,1*\kpsep)--++(-1*\kpsep,1*\kpsep);
\draw[inactivePart](\kdimX,-3*\ksepY+\kulV+3*\kpsep)++(0.5*\kpsep,0)--++(\kpina,0);
\draw[inactivePart](\kdimX,-3*\ksepY+\kulV+4*\kpsep)++(0.5*\kpsep,0)--++(\kpina,0);
\draw[inactivePart](\kdimX,-3*\ksepY+\kulV+3.5*\kpsep)--++(1*\kpsep,-1*\kpsep);
\draw[inactivePart](\kdimX,-3*\ksepY+\kulV+3.5*\kpsep)--++(1*\kpsep,1*\kpsep);
\draw(\kulH+1.5*\kpsep,-3*\ksepY)--++(0,-\kpin);
\draw(\kulH+2.5*\kpsep,-3*\ksepY)--++(0,-\kpin);
\draw(\kulH+1.5*\kpsep,-3*\ksepY)++(0,-\kpin)++(-0.5*\kpsep,0.5*\kpsep)--++(\kpsep,-\kpsep)--++(\kpsep,\kpsep);

\draw(0,-4*\ksepY)  [thick] rectangle ++(\kdimX,\kdimY)node[pos=0.5,rectangle,inner sep=0pt,text width=1.5cm,align=center]{\RTL{قابو    }};
\draw(\kulH+1.5*\kpsep,-4*\ksepY)--++(0,-\kpin);
\draw(\kulH+2.5*\kpsep,-4*\ksepY)--++(0,-\kpin);
\draw(\kulH+1.5*\kpsep,-4*\ksepY)++(0,-\kpin)++(-0.5*\kpsep,0.5*\kpsep)--++(\kpsep,-\kpsep)--++(\kpsep,\kpsep);
\draw(\kulH+2*\kpsep,-4*\ksepY)++(0,-1*\kpin-\kpsep)node[below,rectangle,inner sep=0pt, text width=1cm]{\RTL{قابو لفظ}};

\draw(\kdimX+\kpina+0.5*\kpsep,\kdimY)  [thick] rectangle ++(\kW,-4*\ksepY+\kpin); %گزرگاہ

\begin{scope}[inactivePart]
\draw(\ksepX,-0*\ksepY)  [thick] rectangle ++(\kdimX,\kdimY);%node[pos=0.5,rectangle,inner sep=0pt,text width=1cm,align=center]{\RTL{$A$}};
\draw(\ksepX,-0*\ksepY+\kulV+0*\kpsep)--++(-\kpina,0); 
\draw(\ksepX,-0*\ksepY+\kulV+1*\kpsep)--++(-\kpina,0); 
\draw(\ksepX,-0*\ksepY+\kulV+0*\kpsep)++(-\kpina,0)++(0.5*\kpsep,-0.5*\kpsep)--++(-\kpsep,\kpsep)--++(\kpsep,\kpsep);
\draw(\ksepX,-0*\ksepY+\kulV+3*\kpsep)++(-0.5*\kpsep,0)--++(-\kpina,0); 
\draw(\ksepX,-0*\ksepY+\kulV+4*\kpsep)++(-0.5*\kpsep,0)--++(-\kpina,0); 
\draw(\ksepX,-0*\ksepY+\kulV+3*\kpsep)++(-1*\kpsep,-0.5*\kpsep)--++(\kpsep,\kpsep)--++(-\kpsep,\kpsep);
\draw(\ksepX+\kulH+1.5*\kpsep,-0*\ksepY)--++(0,-\kpin);
\draw(\ksepX+\kulH+2.5*\kpsep,-0*\ksepY)--++(0,-\kpin);
\draw(\ksepX+\kulH+1.5*\kpsep,-0*\ksepY)++(0,-\kpin)++(-0.5*\kpsep,0.5*\kpsep)--++(\kpsep,-\kpsep)--++(\kpsep,\kpsep);

\draw(\ksepX,-1*\ksepY)  [thick] rectangle ++(\kdimX,\kdimY);%node[pos=0.5,rectangle,inner sep=0pt,text width=1cm,align=center]{\RTL{جمع و منفی کار}};
\draw(\ksepX,-1*\ksepY+\kulV+1.5*\kpsep)--++(-\kpina,0); 
\draw(\ksepX,-1*\ksepY+\kulV+2.5*\kpsep)--++(-\kpina,0); 
\draw(\ksepX,-1*\ksepY+\kulV+1.5*\kpsep)++(-\kpina,0)++(0.5*\kpsep,-0.5*\kpsep)--++(-\kpsep,\kpsep)--++(\kpsep,\kpsep);
\draw(\ksepX+\kulH+1.5*\kpsep,-1*\ksepY)++(0,-0.5*\kpsep)--++(0,-\kpin);
\draw(\ksepX+\kulH+2.5*\kpsep,-1*\ksepY)++(0,-0.5*\kpsep)--++(0,-\kpin);
\draw(\ksepX+\kulH+1.5*\kpsep,-1*\ksepY)++(-0.5*\kpsep,-\kpsep)--++(\kpsep,\kpsep)--++(\kpsep,-\kpsep);


\draw(\ksepX,-2*\ksepY)  [thick] rectangle ++(\kdimX,\kdimY);%node[pos=0.5,rectangle,inner sep=0pt,text width=1cm,align=center]{\RTL{$B$}};
\draw(\ksepX,-2*\ksepY+\kulV+1.5*\kpsep)++(-0.5*\kpsep,0)--++(-\kpina,0); 
\draw(\ksepX,-2*\ksepY+\kulV+2.5*\kpsep)++(-0.5*\kpsep,0)--++(-\kpina,0); 
\draw(\ksepX,-2*\ksepY+\kulV+1.5*\kpsep)++(-1*\kpsep,-0.5*\kpsep)--++(\kpsep,\kpsep)--++(-\kpsep,\kpsep);

\draw(\ksepX,-3*\ksepY)  [thick] rectangle ++(\kdimX,\kdimY);%node[pos=0.5,rectangle,inner sep=0pt,text width=1cm,align=center]{\RTL{خارجی دفتر}};
\draw(\ksepX,-3*\ksepY+\kulV+1.5*\kpsep)++(-0.5*\kpsep,0)--++(-\kpina,0); 
\draw(\ksepX,-3*\ksepY+\kulV+2.5*\kpsep)++(-0.5*\kpsep,0)--++(-\kpina,0); 
\draw(\ksepX,-3*\ksepY+\kulV+1.5*\kpsep)++(-1*\kpsep,-0.5*\kpsep)--++(\kpsep,\kpsep)--++(-\kpsep,\kpsep);
\draw(\ksepX,-3*\ksepY)++(\kulH+1.5*\kpsep,0)--++(0,-\kpin);
\draw(\ksepX,-3*\ksepY)++(\kulH+2.5*\kpsep,0)--++(0,-\kpin);
\draw(\ksepX,-3*\ksepY)++(\kulH+1.5*\kpsep,0)++(0,-\kpin)++(-0.5*\kpsep,0.5*\kpsep)--++(\kpsep,-\kpsep)--++(\kpsep,\kpsep);

\draw(\ksepX,-4*\ksepY)  [thick] rectangle ++(\kdimX,\kdimY);%node[pos=0.5,rectangle,inner sep=0pt,text width=1cm,align=center]{\RTL{تختی}};
\end{scope}
\end{tikzpicture}
\caption{}
\end{subfigure}\hfill
\begin{subfigure}{0.30\textwidth}
\centering
\begin{tikzpicture}
\pgfmathsetmacro{\klshift}{0.25}
\pgfmathsetmacro{\knshift}{0.07}
\pgfmathsetmacro{\kmv}{0.15}
\pgfmathsetmacro{\knshift}{0.07}
\pgfmathsetmacro{\kpin}{0.30}
\pgfmathsetmacro{\kpina}{0.30}
\pgfmathsetmacro{\kpsep}{0.15}			%pin to pin distance
\pgfmathsetmacro{\kW}{\kpsep}
\pgfmathsetmacro{\kulV}{0.20}			%edge clearance along vertical edge
\pgfmathsetmacro{\kulH}{0.20}
\pgfmathsetmacro{\kdimX}{2*\kulH+4*\kpsep}
\pgfmathsetmacro{\kdimY}{2*\kulV+4*\kpsep}		%two spaces between 3 pins
\pgfmathsetmacro{\ksepX}{\kdimX+2*\kpina+1*\kpsep+\kW}
\pgfmathsetmacro{\ksepY}{\kdimY+\kpin+0.5*\kpsep}

\begin{scope}[inactivePart]
\draw(0,0)  [thick] rectangle ++(\kdimX,\kdimY);%node[pos=0.5,rectangle,inner sep=0pt,text width=1.5cm,align=center]{\RTL{گنتکار}};
\draw(\kdimX,\kulV+1.5*\kpsep)--++(\kpina,0);
\draw(\kdimX,\kulV+2.5*\kpsep)--++(\kpina,0);
\draw(\kdimX,\kulV+1.5*\kpsep)++(\kpina,0)++(-0.5*\kpsep,-0.5*\kpsep)--++(1*\kpsep,1*\kpsep)--++(-1*\kpsep,1*\kpsep);
\end{scope}

\draw(0,-\ksepY)  [thick] rectangle ++(\kdimX,\kdimY)node[pos=0.5,rectangle,inner sep=0pt,text width=1cm,align=center]{\RTL{دفتر پتہ}};
\draw(\kulH+0*\kpsep,-\ksepY)--++(0,-\kpin);
\draw(\kulH+1*\kpsep,-\ksepY)--++(0,-\kpin);
\draw(\kulH+0*\kpsep,-\ksepY)++(0,-\kpin)++(-0.5*\kpsep,0.5*\kpsep)--++(\kpsep,-\kpsep)--++(\kpsep,\kpsep);
\draw(\kulH+3*\kpsep,-\ksepY)--++(0,-\kpin);
\draw(\kulH+4*\kpsep,-\ksepY)--++(0,-\kpin);
\draw(\kulH+3*\kpsep,-\ksepY)++(0,-\kpin)++(-0.5*\kpsep,0.5*\kpsep)--++(\kpsep,-\kpsep)--++(\kpsep,\kpsep);
\draw[inactivePart](\kdimX,-\ksepY+\kulV+1.5*\kpsep)++(0.5*\kpsep,0)--++(\kpina,0);
\draw[inactivePart](\kdimX,-\ksepY+\kulV+2.5*\kpsep)++(0.5*\kpsep,0)--++(\kpina,0);
\draw[inactivePart](\kdimX,-\ksepY+\kulV+2*\kpsep)--++(1*\kpsep,-1*\kpsep);
\draw[inactivePart](\kdimX,-\ksepY+\kulV+2*\kpsep)--++(1*\kpsep,1*\kpsep);

\draw(0,-2*\ksepY)  [thick] rectangle ++(\kdimX,\kdimY)node[pos=0.5,rectangle,inner sep=0pt,text width=1cm,align=center]{\footnotesize{\RTL{حافظہ}}};
\draw(0,-2*\ksepY+\kulV+0*\kpsep)--++(-\kpin,0)node[left]{$\overline{CE}$};
\draw(0,-2*\ksepY+\kulV+0*\kpsep)++(-0.07,0)node[ocirc]{};
\draw(\kdimX,-2*\ksepY+\kulV+1.5*\kpsep)--++(\kpina,0);
\draw(\kdimX,-2*\ksepY+\kulV+2.5*\kpsep)--++(\kpina,0);
\draw(\kdimX,-2*\ksepY+\kulV+1.5*\kpsep)++(\kpina,0)++(-0.5*\kpsep,-0.5*\kpsep)--++(1*\kpsep,1*\kpsep)--++(-1*\kpsep,1*\kpsep);

\draw(0,-3*\ksepY)  [thick] rectangle ++(\kdimX,\kdimY)node[pos=0.5,rectangle,inner sep=0pt,text width=1cm,align=center]{ \RTL{دفتر ہدایت}};
\draw[inactivePart](\kdimX,-3*\ksepY+\kulV+0*\kpsep)--++(\kpina,0);
\draw[inactivePart](\kdimX,-3*\ksepY+\kulV+1*\kpsep)--++(\kpina,0);
\draw[inactivePart](\kdimX,-3*\ksepY+\kulV+0*\kpsep)++(\kpina,0)++(-0.5*\kpsep,-0.5*\kpsep)--++(1*\kpsep,1*\kpsep)--++(-1*\kpsep,1*\kpsep);
\draw[inactivePart](\kdimX,-3*\ksepY+\kulV+3*\kpsep)++(0.5*\kpsep,0)--++(\kpina,0);
\draw[inactivePart](\kdimX,-3*\ksepY+\kulV+4*\kpsep)++(0.5*\kpsep,0)--++(\kpina,0);
\draw[inactivePart](\kdimX,-3*\ksepY+\kulV+3.5*\kpsep)--++(1*\kpsep,-1*\kpsep);
\draw[inactivePart](\kdimX,-3*\ksepY+\kulV+3.5*\kpsep)--++(1*\kpsep,1*\kpsep);
\draw(\kulH+1.5*\kpsep,-3*\ksepY)--++(0,-\kpin);
\draw(\kulH+2.5*\kpsep,-3*\ksepY)--++(0,-\kpin);
\draw(\kulH+1.5*\kpsep,-3*\ksepY)++(0,-\kpin)++(-0.5*\kpsep,0.5*\kpsep)--++(\kpsep,-\kpsep)--++(\kpsep,\kpsep);


\draw(0,-4*\ksepY)  [thick] rectangle ++(\kdimX,\kdimY)node[pos=0.5,rectangle,inner sep=0pt,text width=1.5cm,align=center]{\RTL{قابو    }};
\draw(\kulH+1.5*\kpsep,-4*\ksepY)--++(0,-\kpin);
\draw(\kulH+2.5*\kpsep,-4*\ksepY)--++(0,-\kpin);
\draw(\kulH+1.5*\kpsep,-4*\ksepY)++(0,-\kpin)++(-0.5*\kpsep,0.5*\kpsep)--++(\kpsep,-\kpsep)--++(\kpsep,\kpsep);
\draw(\kulH+2*\kpsep,-4*\ksepY)++(0,-1*\kpin-\kpsep)node[below,rectangle,inner sep=0pt, text width=1cm]{\RTL{قابو لفظ}};


\draw(\kdimX+\kpina+0.5*\kpsep,\kdimY)  [thick] rectangle ++(\kW,-4*\ksepY+\kpin);

\begin{scope}[inactivePart]
\draw(\ksepX,-0*\ksepY)  [thick] rectangle ++(\kdimX,\kdimY);%node[pos=0.5,rectangle,inner sep=0pt,text width=1cm,align=center]{\RTL{$A$}};
\draw[inactivePart](\ksepX,-0*\ksepY+\kulV+0*\kpsep)--++(-\kpina,0); 
\draw[inactivePart](\ksepX,-0*\ksepY+\kulV+1*\kpsep)--++(-\kpina,0); 
\draw[inactivePart](\ksepX,-0*\ksepY+\kulV+0*\kpsep)++(-\kpina,0)++(0.5*\kpsep,-0.5*\kpsep)--++(-\kpsep,\kpsep)--++(\kpsep,\kpsep);
\draw(\ksepX,-0*\ksepY+\kulV+3*\kpsep)++(-0.5*\kpsep,0)--++(-\kpina,0); 
\draw(\ksepX,-0*\ksepY+\kulV+4*\kpsep)++(-0.5*\kpsep,0)--++(-\kpina,0); 
\draw(\ksepX,-0*\ksepY+\kulV+3*\kpsep)++(-1*\kpsep,-0.5*\kpsep)--++(\kpsep,\kpsep)--++(-\kpsep,\kpsep);
\draw[inactivePart](\ksepX+\kulH+1.5*\kpsep,-0*\ksepY)--++(0,-\kpin);
\draw[inactivePart](\ksepX+\kulH+2.5*\kpsep,-0*\ksepY)--++(0,-\kpin);
\draw[inactivePart](\ksepX+\kulH+1.5*\kpsep,-0*\ksepY)++(0,-\kpin)++(-0.5*\kpsep,0.5*\kpsep)--++(\kpsep,-\kpsep)--++(\kpsep,\kpsep);

\draw(\ksepX,-1*\ksepY)  [thick] rectangle ++(\kdimX,\kdimY);%node[pos=0.5,rectangle,inner sep=0pt,text width=1cm,align=center]{\RTL{جمع و منفی کار}};
\draw(\ksepX,-1*\ksepY+\kulV+1.5*\kpsep)--++(-\kpina,0); 
\draw(\ksepX,-1*\ksepY+\kulV+2.5*\kpsep)--++(-\kpina,0); 
\draw(\ksepX,-1*\ksepY+\kulV+1.5*\kpsep)++(-\kpina,0)++(0.5*\kpsep,-0.5*\kpsep)--++(-\kpsep,\kpsep)--++(\kpsep,\kpsep);
\draw(\ksepX+\kulH+1.5*\kpsep,-1*\ksepY)++(0,-0.5*\kpsep)--++(0,-\kpin);
\draw(\ksepX+\kulH+2.5*\kpsep,-1*\ksepY)++(0,-0.5*\kpsep)--++(0,-\kpin);
\draw(\ksepX+\kulH+1.5*\kpsep,-1*\ksepY)++(-0.5*\kpsep,-\kpsep)--++(\kpsep,\kpsep)--++(\kpsep,-\kpsep);
\end{scope} 

\draw(\ksepX,-2*\ksepY)  [thick] rectangle ++(\kdimX,\kdimY)node[pos=0.5,rectangle,inner sep=0pt,text width=1cm,align=center]{\RTL{$B$}};
\draw(\ksepX+\kdimX,-2*\ksepY+\kulV+4*\kpsep)--++(\kpin,0)node[right]{$\overline{L}_B$};
\draw(\ksepX+\kdimX,-2*\ksepY+\kulV+4*\kpsep)++(\knshift,0)node[ocirc]{};
\draw(\ksepX,-2*\ksepY+\kulV+1.5*\kpsep)++(-0.5*\kpsep,0)--++(-\kpina,0); 
\draw(\ksepX,-2*\ksepY+\kulV+2.5*\kpsep)++(-0.5*\kpsep,0)--++(-\kpina,0); 
\draw(\ksepX,-2*\ksepY+\kulV+1.5*\kpsep)++(-1*\kpsep,-0.5*\kpsep)--++(\kpsep,\kpsep)--++(-\kpsep,\kpsep);

\begin{scope}[inactivePart]
\draw(\ksepX,-3*\ksepY)  [thick] rectangle ++(\kdimX,\kdimY);%node[pos=0.5,rectangle,inner sep=0pt,text width=1cm,align=center]{\RTL{خارجی دفتر}};
\draw(\ksepX,-3*\ksepY+\kulV+1.5*\kpsep)++(-0.5*\kpsep,0)--++(-\kpina,0); 
\draw(\ksepX,-3*\ksepY+\kulV+2.5*\kpsep)++(-0.5*\kpsep,0)--++(-\kpina,0); 
\draw(\ksepX,-3*\ksepY+\kulV+1.5*\kpsep)++(-1*\kpsep,-0.5*\kpsep)--++(\kpsep,\kpsep)--++(-\kpsep,\kpsep);
\draw(\ksepX,-3*\ksepY)++(\kulH+1.5*\kpsep,0)--++(0,-\kpin);
\draw(\ksepX,-3*\ksepY)++(\kulH+2.5*\kpsep,0)--++(0,-\kpin);
\draw(\ksepX,-3*\ksepY)++(\kulH+1.5*\kpsep,0)++(0,-\kpin)++(-0.5*\kpsep,0.5*\kpsep)--++(\kpsep,-\kpsep)--++(\kpsep,\kpsep);

\draw(\ksepX,-4*\ksepY)  [thick] rectangle ++(\kdimX,\kdimY);%node[pos=0.5,rectangle,inner sep=0pt,text width=1cm,align=center]{\RTL{تختی}};
\end{scope}
\end{tikzpicture}
\caption{}
\end{subfigure}\hfill
\begin{subfigure}{0.30\textwidth}
\centering
\begin{tikzpicture}
\pgfmathsetmacro{\klshift}{0.25}
\pgfmathsetmacro{\knshift}{0.07}
\pgfmathsetmacro{\kmv}{0.15}
\pgfmathsetmacro{\knshift}{0.07}
\pgfmathsetmacro{\kpin}{0.30}
\pgfmathsetmacro{\kpina}{0.30}
\pgfmathsetmacro{\kpsep}{0.15}			%pin to pin distance
\pgfmathsetmacro{\kW}{\kpsep}
\pgfmathsetmacro{\kulV}{0.20}			%edge clearance along vertical edge
\pgfmathsetmacro{\kulH}{0.20}
\pgfmathsetmacro{\kdimX}{2*\kulH+4*\kpsep}
\pgfmathsetmacro{\kdimY}{2*\kulV+4*\kpsep}		%two spaces between 3 pins
\pgfmathsetmacro{\ksepX}{\kdimX+2*\kpina+1*\kpsep+\kW}
\pgfmathsetmacro{\ksepY}{\kdimY+\kpin+0.5*\kpsep}

\begin{scope}[inactivePart]
\draw(0,0)  [thick] rectangle ++(\kdimX,\kdimY);%node[pos=0.5,rectangle,inner sep=0pt,text width=1.5cm,align=center]{\RTL{گنتکار}};
\draw(\kdimX,\kulV+1.5*\kpsep)--++(\kpina,0);
\draw(\kdimX,\kulV+2.5*\kpsep)--++(\kpina,0);
\draw(\kdimX,\kulV+1.5*\kpsep)++(\kpina,0)++(-0.5*\kpsep,-0.5*\kpsep)--++(1*\kpsep,1*\kpsep)--++(-1*\kpsep,1*\kpsep);


\draw(0,-\ksepY)  [thick] rectangle ++(\kdimX,\kdimY);%node[pos=0.5,rectangle,inner sep=0pt,text width=1cm,align=center]{\RTL{دفتر پتہ}};
\draw(\kulH+0*\kpsep,-\ksepY)--++(0,-\kpin);
\draw(\kulH+1*\kpsep,-\ksepY)--++(0,-\kpin);
\draw(\kulH+0*\kpsep,-\ksepY)++(0,-\kpin)++(-0.5*\kpsep,0.5*\kpsep)--++(\kpsep,-\kpsep)--++(\kpsep,\kpsep);
\draw(\kulH+3*\kpsep,-\ksepY)--++(0,-\kpin);
\draw(\kulH+4*\kpsep,-\ksepY)--++(0,-\kpin);
\draw(\kulH+3*\kpsep,-\ksepY)++(0,-\kpin)++(-0.5*\kpsep,0.5*\kpsep)--++(\kpsep,-\kpsep)--++(\kpsep,\kpsep);
\draw(\kdimX,-\ksepY+\kulV+1.5*\kpsep)++(0.5*\kpsep,0)--++(\kpina,0);
\draw(\kdimX,-\ksepY+\kulV+2.5*\kpsep)++(0.5*\kpsep,0)--++(\kpina,0);
\draw(\kdimX,-\ksepY+\kulV+2*\kpsep)--++(1*\kpsep,-1*\kpsep);
\draw(\kdimX,-\ksepY+\kulV+2*\kpsep)--++(1*\kpsep,1*\kpsep);

\draw(0,-2*\ksepY)  [thick] rectangle ++(\kdimX,\kdimY);%node[pos=0.5,rectangle,inner sep=0pt,text width=1cm,align=center]{\footnotesize{\RTL{حافظہ}}};
\draw(\kdimX,-2*\ksepY+\kulV+1.5*\kpsep)--++(\kpina,0);
\draw(\kdimX,-2*\ksepY+\kulV+2.5*\kpsep)--++(\kpina,0);
\draw(\kdimX,-2*\ksepY+\kulV+1.5*\kpsep)++(\kpina,0)++(-0.5*\kpsep,-0.5*\kpsep)--++(1*\kpsep,1*\kpsep)--++(-1*\kpsep,1*\kpsep);
\end{scope}

\draw(0,-3*\ksepY)  [thick] rectangle ++(\kdimX,\kdimY)node[pos=0.5,rectangle,inner sep=0pt,text width=1cm,align=center]{ \RTL{دفتر ہدایت}};
\draw[inactivePart](\kdimX,-3*\ksepY+\kulV+0*\kpsep)--++(\kpina,0);
\draw[inactivePart](\kdimX,-3*\ksepY+\kulV+1*\kpsep)--++(\kpina,0);
\draw[inactivePart](\kdimX,-3*\ksepY+\kulV+0*\kpsep)++(\kpina,0)++(-0.5*\kpsep,-0.5*\kpsep)--++(1*\kpsep,1*\kpsep)--++(-1*\kpsep,1*\kpsep);
\draw[inactivePart](\kdimX,-3*\ksepY+\kulV+3*\kpsep)++(0.5*\kpsep,0)--++(\kpina,0);
\draw[inactivePart](\kdimX,-3*\ksepY+\kulV+4*\kpsep)++(0.5*\kpsep,0)--++(\kpina,0);
\draw[inactivePart](\kdimX,-3*\ksepY+\kulV+3.5*\kpsep)--++(1*\kpsep,-1*\kpsep);
\draw[inactivePart](\kdimX,-3*\ksepY+\kulV+3.5*\kpsep)--++(1*\kpsep,1*\kpsep);
\draw(\kulH+1.5*\kpsep,-3*\ksepY)--++(0,-\kpin);
\draw(\kulH+2.5*\kpsep,-3*\ksepY)--++(0,-\kpin);
\draw(\kulH+1.5*\kpsep,-3*\ksepY)++(0,-\kpin)++(-0.5*\kpsep,0.5*\kpsep)--++(\kpsep,-\kpsep)--++(\kpsep,\kpsep);

\draw(0,-4*\ksepY)  [thick] rectangle ++(\kdimX,\kdimY)node[pos=0.5,rectangle,inner sep=0pt,text width=1.5cm,align=center]{\RTL{قابو    }};

\draw(\kulH+1.5*\kpsep,-4*\ksepY)--++(0,-\kpin);
\draw(\kulH+2.5*\kpsep,-4*\ksepY)--++(0,-\kpin);
\draw(\kulH+1.5*\kpsep,-4*\ksepY)++(0,-\kpin)++(-0.5*\kpsep,0.5*\kpsep)--++(\kpsep,-\kpsep)--++(\kpsep,\kpsep);
\draw(\kulH+2*\kpsep,-4*\ksepY)++(0,-1*\kpin-\kpsep)node[below,rectangle,inner sep=0pt, text width=1cm]{\RTL{قابو لفظ}};


\draw(\kdimX+\kpina+0.5*\kpsep,\kdimY)  [thick] rectangle ++(\kW,-4*\ksepY+\kpin);


\draw(\ksepX,-0*\ksepY)  [thick] rectangle ++(\kdimX,\kdimY)node[pos=0.5,rectangle,inner sep=0pt,text width=1cm,align=center]{\RTL{$A$}};
\draw(\ksepX,-0*\ksepY)++(\kdimX,\kulV+4*\kpsep)--++(\kpin,0)node[right]{$\overline{L}_A$};
\draw(\ksepX,-0*\ksepY)++(\kdimX,\kulV+4*\kpsep)++(\knshift,0)node[ocirc]{};
\draw[inactivePart](\ksepX,-0*\ksepY+\kulV+0*\kpsep)--++(-\kpina,0); 
\draw[inactivePart](\ksepX,-0*\ksepY+\kulV+1*\kpsep)--++(-\kpina,0); 
\draw[inactivePart](\ksepX,-0*\ksepY+\kulV+0*\kpsep)++(-\kpina,0)++(0.5*\kpsep,-0.5*\kpsep)--++(-\kpsep,\kpsep)--++(\kpsep,\kpsep);
\draw[inactivePart](\ksepX,-0*\ksepY+\kulV+3*\kpsep)++(-0.5*\kpsep,0)--++(-\kpina,0); 
\draw[inactivePart](\ksepX,-0*\ksepY+\kulV+4*\kpsep)++(-0.5*\kpsep,0)--++(-\kpina,0); 
\draw[inactivePart](\ksepX,-0*\ksepY+\kulV+3*\kpsep)++(-1*\kpsep,-0.5*\kpsep)--++(\kpsep,\kpsep)--++(-\kpsep,\kpsep);
\draw(\ksepX+\kulH+1.5*\kpsep,-0*\ksepY)--++(0,-\kpin);
\draw(\ksepX+\kulH+2.5*\kpsep,-0*\ksepY)--++(0,-\kpin);
\draw(\ksepX+\kulH+1.5*\kpsep,-0*\ksepY)++(0,-\kpin)++(-0.5*\kpsep,0.5*\kpsep)--++(\kpsep,-\kpsep)--++(\kpsep,\kpsep);
\draw(\ksepX,-1*\ksepY)  [thick] rectangle ++(\kdimX,\kdimY)node[pos=0.5,rectangle,inner sep=0pt,text width=1cm,align=center]{\RTL{جمع و منفی کار}};
\draw(\ksepX,-1*\ksepY)++(\kdimX,\kulV+0*\kpsep)--++(\kpin,0)node[right]{$E_U$};
\draw(\ksepX,-1*\ksepY+\kulV+1.5*\kpsep)--++(-\kpina,0); 
\draw(\ksepX,-1*\ksepY+\kulV+2.5*\kpsep)--++(-\kpina,0); 
\draw(\ksepX,-1*\ksepY+\kulV+1.5*\kpsep)++(-\kpina,0)++(0.5*\kpsep,-0.5*\kpsep)--++(-\kpsep,\kpsep)--++(\kpsep,\kpsep);
\draw(\ksepX+\kulH+1.5*\kpsep,-1*\ksepY)++(0,-0.5*\kpsep)--++(0,-\kpin);
\draw(\ksepX+\kulH+2.5*\kpsep,-1*\ksepY)++(0,-0.5*\kpsep)--++(0,-\kpin);
\draw(\ksepX+\kulH+1.5*\kpsep,-1*\ksepY)++(-0.5*\kpsep,-\kpsep)--++(\kpsep,\kpsep)--++(\kpsep,-\kpsep);


\draw(\ksepX,-2*\ksepY)  [thick] rectangle ++(\kdimX,\kdimY)node[pos=0.5,rectangle,inner sep=0pt,text width=1cm,align=center]{\RTL{$B$}};
\draw[inactivePart](\ksepX,-2*\ksepY+\kulV+1.5*\kpsep)++(-0.5*\kpsep,0)--++(-\kpina,0); 
\draw[inactivePart](\ksepX,-2*\ksepY+\kulV+2.5*\kpsep)++(-0.5*\kpsep,0)--++(-\kpina,0); 
\draw[inactivePart](\ksepX,-2*\ksepY+\kulV+1.5*\kpsep)++(-1*\kpsep,-0.5*\kpsep)--++(\kpsep,\kpsep)--++(-\kpsep,\kpsep);

\begin{scope}[inactivePart]
\draw(\ksepX,-3*\ksepY)  [thick] rectangle ++(\kdimX,\kdimY);%node[pos=0.5,rectangle,inner sep=0pt,text width=1cm,align=center]{\RTL{خارجی دفتر}};
\draw(\ksepX,-3*\ksepY+\kulV+1.5*\kpsep)++(-0.5*\kpsep,0)--++(-\kpina,0); 
\draw(\ksepX,-3*\ksepY+\kulV+2.5*\kpsep)++(-0.5*\kpsep,0)--++(-\kpina,0); 
\draw(\ksepX,-3*\ksepY+\kulV+1.5*\kpsep)++(-1*\kpsep,-0.5*\kpsep)--++(\kpsep,\kpsep)--++(-\kpsep,\kpsep);
\draw(\ksepX,-3*\ksepY)++(\kulH+1.5*\kpsep,0)--++(0,-\kpin);
\draw(\ksepX,-3*\ksepY)++(\kulH+2.5*\kpsep,0)--++(0,-\kpin);
\draw(\ksepX,-3*\ksepY)++(\kulH+1.5*\kpsep,0)++(0,-\kpin)++(-0.5*\kpsep,0.5*\kpsep)--++(\kpsep,-\kpsep)--++(\kpsep,\kpsep);

\draw(\ksepX,-4*\ksepY)  [thick] rectangle ++(\kdimX,\kdimY);%node[pos=0.5,rectangle,inner sep=0pt,text width=1cm,align=center]{\RTL{تختی}};
\end{scope}
\end{tikzpicture}
\caption{}
\end{subfigure}
\caption{
طریق جمع و منفی؛ (ا) \عددی{T_4} حال؛ (ب) \عددی{T_5} حال؛ (ج) \عددی{T_6} حال۔
}
\label{شکل_کمپیوٹر_جمع_اور_منفی}
\end{figure}

اتفاق سے،   دورانیہ تیاری اور  دورانیہ ردعمل  کی بدولت دفتر \عددی{A} حالت دوڑ سے دو چار نہیں ہوتا۔ شکل \حوالہء{10.6c} میں  ساعت کے  کنارہ چڑھائی  پر دفتر \عددی{A} کا  مواد تبدیل ہو گا، جس کی وجہ سے جمع و منفی کار کا مخارج تبدیل ہو گا۔ یہ نیا مواد دفتر \عددی{A}  کے مداخل تک پہنچتا ہے، تاہم یہ مواد  ساعت کے کنارہ چڑھائی کے دو  تاخیر  بعد  یہاں پہنچے گا ( پہلی تاخیر   دفتر \عددی{A} اور دوسری  تاخیر جمع و منفی کار کی بدولت ہو گی)۔ اس وقت تک دفتر \عددی{A}  میں مواد لکھنے کا لمحہ گزر چکا ہوگا۔ یوں دفتر \عددی{A} حالت دوڑ (جس میں ساعت کے  ایک  ہی کنارے پر  ایک سے زیادہ مرتبہ مواد بھرا جاتا ہو)  سے دو چار نہیں ہو گا۔

شکل  \حوالہ{شکل_کمپیوٹر_بازیابی_جمع_وقتیہ}  میں بازیابی اور  \قول{طریق جمع  } کی وقتیہ ترسیمات  پیش ہیں۔طریق  بازیابی  ہمیشہ کی طرح \عددی{T_1} حال میں  دفتر پتہ میں برنامہ گنت کار کا مواد منتقل کرتا ہے؛ \عددی{T_2} حال میں  گنت کار کی گنتی میں ایک کا اضافہ کیا جاتا ہے؛ \عددی{T_3} حال میں دفتر ہدایت  کو ،  پتہ کی نشاندہی پر ، حافظہ سے ہدایت منتقل کی جاتی ہے۔

\begin{figure}
\centering
\begin{otherlanguage}{english}
 \begin{tikztimingtable}[%
timing/.style={x=4ex,y=3ex},
timing/rowdist=6ex,
every node/.style={inner sep=0,outer sep=0},
%timing/c/arrow tip=latex, %and this set the style
%timing/c/rising arrows,
timing/slope=0, %0.1 is good
timing/dslope=0,
thick,
]
%\tikztimingmetachar{R}{[|/utils/exec=\setcounter{new}{0}|]}
%\usetikztiminglibrary[new={char=Q,reset char=R}]{counters}
%[timing/counter/new={char=c, base=2,digits=3,max value=7, wraps ,text style={font=\normalsize}}] 12{2c} \\ 
%$C$& H22{C}\\
%$\texturdu{\RL{پتہ}}$&3D{} 20D{[scale=1.5]\texturdu{\RL{درست پتہ}}} 3D{}\\
$CLK$&HN(a)LN(ca)HN(b)LN(cb)HN(c)LN(cc)HN(d)LN(cd)HN(e)LN(ce)HN(f)LN(cf)HN(g)L\\
$E_P$&LHHLLLLLLLLLLH\\
$\overline{L}_M$&HLL4{H}2{L}5{H}\\
$C_P$&LLLHHLLLLLLLLL\\
$\overline{CE}$&HHHHHLLHHLLHHH\\
$\overline{L}_I$&HHHHHLLHHHHHHH\\
$\overline{E}_I$&HHHHHHHLLHHHHH\\
$\overline{L}_B$&HHHHHHHHHLLHHH\\
$E_U$&LLLLLLLLLLLHHL\\
$\overline{L}_A$&HHHHHHHHHHHLLH\\
\extracode
\begin{pgfonlayer}{background}
\begin{scope}[]
\draw [latex-latex] ($(a|-row1.north)+(0,2ex)$) --node[fill=white]{$T_1$} ($(b|-row1.north)+(0,2ex)$);
\draw [latex-latex] ($(b|-row1.north)+(0,2ex)$) --node[fill=white]{$T_2$} ($(c|-row1.north)+(0,2ex)$);
\draw [latex-latex] ($(c|-row1.north)+(0,2ex)$) --node[fill=white]{$T_3$} ($(d|-row1.north)+(0,2ex)$);
\draw [latex-latex] ($(d|-row1.north)+(0,2ex)$) --node[fill=white]{$T_4$} ($(e|-row1.north)+(0,2ex)$);
\draw [latex-latex] ($(e|-row1.north)+(0,2ex)$) --node[fill=white]{$T_5$} ($(f|-row1.north)+(0,2ex)$);
\draw [latex-latex] ($(f|-row1.north)+(0,2ex)$) --node[fill=white]{$T_6$} ($(g|-row1.north)+(0,2ex)$);
\foreach \n in {a,b,c,d,e,f,g}{\draw[thin]($(\n|-row1.north)+(0,1ex)$)--++(0,2ex);}
\draw[dashed] (ca)--(ca |-row3.south);
\draw[dashed] (cb)--(cb |-row4.north);
\draw[dashed] (cc)--(cc |-row6.south);
\draw[dashed] (cd)--(cd |-row7.south);
\draw[dashed] (ce)--(ce |-row8.south);
\draw[dashed] (cf)--(cf |-row10.south);
%%\vertlines[darkgray,dotted]{3.6,7.5,11.5,15.5,19.5,23.5,27.5}
%\foreach \n in {1,3,...,11} \draw(4*\n ex-2ex,-4ex+1.25ex)node[]{$0$};
%\foreach \n in {2,4,...,12} \draw(4*\n ex-2ex,-4ex+1.25ex)node[]{$1$};
%\foreach \n in {1,2,5,6,9,10} \draw(4*\n ex-2ex,-12 ex+1.25ex)node[]{$0$};
%\foreach \n in {3,4,7,8,11,12} \draw(4*\n ex-2ex,-12 ex+1.25ex)node[]{$1$};
%\foreach \n in {1,2,3,4,9,10,11,12} \draw(4*\n ex-2ex,-20 ex+1.25ex)node[]{$0$};
%\foreach \n in {5,6,7,8} \draw(4*\n ex-2ex,-20 ex+1.25ex)node[]{$1$};
%\draw(4*4 ex-2ex,-12 ex+1.25ex) circle (0.25cm and 1.75cm);
%\draw(4*4 ex-2ex,-7*\rowdist+1.25ex) circle (0.25cm and 0.5cm);
%%\foreach \n in {1}\draw(B\n.south)--(F\n.north);
%%\foreach \n in {1}\draw(C\n.south)--(G\n.north);
\end{scope}
\end{pgfonlayer}
\end{tikztimingtable}
\end{otherlanguage}
\caption{بازیابی اور جمع  وقتیہ ترسیمات۔}
\label{شکل_کمپیوٹر_بازیابی_جمع_وقتیہ}
\end{figure}

\عددی{T_4} حال کے دوران، \عددی{\overline{E}_I} اور \عددی{\overline{L}_M} فعال  ہوں گے؛ ساعت کے اگلے کنارہ چڑھائی پر، دفتر پتہ کو   دفتر  ہدایت سے جزو  پتہ منتقل ہو گا۔ \عددی{T_5} حال کے دوران، \عددی{\overline{CE}} اور \عددی{\overline{L}_B} فعال ہوں گے؛  لہٰذا ساعت کے کنارہ چڑھائی پر   دفتر \عددی{B} میں  پتہ کی نشاندہی پر حافظہ سے لفظ منتقل ہو گا۔ \عددی{T_6} حال کے دوران، \عددی{E_U} اور \عددی{\overline{L}_A} فعال ہوں گے؛ دفتر \عددی{A} میں، ساعت کے کنارہ چڑھائی پر، جمع  و منفی کار کا  حاصل نتیجہ منتقل ہو گا۔

\جزوحصہء{طریق منفی}
طریق منفی اور طریق جمع ملتے جلتے ہیں۔ شکل \حوالہ{شکل_کمپیوٹر_جمع_اور_منفی}-الف اور ب میں طریق منفی کے لئے  \عددی{T_4} اور \عددی{T_5}  حال کے دوران  فعال حصے دکھائے گئے ہیں۔ \عددی{T_6} حال کے دوران شکل \حوالہ{شکل_کمپیوٹر_جمع_اور_منفی}-ج کے جمو ع منفی حصے کو بلند \عددی{S_U} بھیجا جاتا ہے۔ وقتیہ ترسیم شکل \حوالہ{شکل_کمپیوٹر_بازیابی_جمع_وقتیہ}    سے تقریباً  مکمل مماثلت رکھتی ہے۔ \عددی{T_1}  تا \عددی{T_5} حال کے دوران پست \عددی{S_U} اور \عددی{T_6} حال کے دوران بلند \عددی{S_U} تصور کریں۔

\جزوحصہء{طریق برآمد}
فرض کریں  بازیابی پھیرا کے آخر میں دفتر ہدایت میں برآمد کی ہدایت موجود ہو۔ یوں درج ذیل ہو گا۔
\begin{align*}
1110\,xxxx=\text{\RL{دفتر ہدایت}}
\end{align*}
قابو و ترتیب کار کو رمز کشائی کے لئے  جزو ہدایت  بھیجا جاتا ہے۔ رمز کشائی کے بعد   قابو و ترتیب کار   خارجی دفتر میں دفتر \عددی{A} کا مواد منتقل کرنے کے لئے قابو لفظ جاری کرتا ہے۔

برآمد کی ہدایت کے دوران فعال حصے شکل  \حوالہ{شکل_کمپیوٹر_برآمد_ہدایت}  میں پیش ہیں۔ چونکہ \عددی{E_A} اور \عددی{\overline{L}_O} فعال ہیں، لہٰذا ساعت کے اگلے کنارہ چڑھائی پر دفتر \عددی{A} کی معلومات  خارجی دفتر میں  ، \عددی{T_4} حال کے دوران ، منتقل ہو گی۔ \عددی{T_5} اور \عددی{T_6} حال فارغ  ہیں۔

\begin{figure}
\centering
\begin{tikzpicture}
\pgfmathsetmacro{\klshift}{0.25}
\pgfmathsetmacro{\knshift}{0.07}
\pgfmathsetmacro{\kmv}{0.15}
\pgfmathsetmacro{\knshift}{0.07}
\pgfmathsetmacro{\kpin}{0.30}
\pgfmathsetmacro{\kpina}{0.30}
\pgfmathsetmacro{\kpsep}{0.15}			%pin to pin distance
\pgfmathsetmacro{\kW}{\kpsep}
\pgfmathsetmacro{\kulV}{0.20}			%edge clearance along vertical edge
\pgfmathsetmacro{\kulH}{0.20}
\pgfmathsetmacro{\kdimX}{2*\kulH+4*\kpsep}
\pgfmathsetmacro{\kdimY}{2*\kulV+4*\kpsep}		%two spaces between 3 pins
\pgfmathsetmacro{\ksepX}{\kdimX+2*\kpina+1*\kpsep+\kW}
\pgfmathsetmacro{\ksepY}{\kdimY+\kpin+0.5*\kpsep}

\begin{scope}[inactivePart]
\draw(0,0)  [thick] rectangle ++(\kdimX,\kdimY);%node[pos=0.5,rectangle,inner sep=0pt,text width=1.5cm,align=center]{\RTL{گنتکار}};
\draw(\kdimX,\kulV+1.5*\kpsep)--++(\kpina,0);
\draw(\kdimX,\kulV+2.5*\kpsep)--++(\kpina,0);
\draw(\kdimX,\kulV+1.5*\kpsep)++(\kpina,0)++(-0.5*\kpsep,-0.5*\kpsep)--++(1*\kpsep,1*\kpsep)--++(-1*\kpsep,1*\kpsep);


\draw(0,-\ksepY)  [thick] rectangle ++(\kdimX,\kdimY);%node[pos=0.5,rectangle,inner sep=0pt,text width=1cm,align=center]{\RTL{دفتر پتہ}};
\draw(\kulH+0*\kpsep,-\ksepY)--++(0,-\kpin);
\draw(\kulH+1*\kpsep,-\ksepY)--++(0,-\kpin);
\draw(\kulH+0*\kpsep,-\ksepY)++(0,-\kpin)++(-0.5*\kpsep,0.5*\kpsep)--++(\kpsep,-\kpsep)--++(\kpsep,\kpsep);
\draw(\kulH+3*\kpsep,-\ksepY)--++(0,-\kpin);
\draw(\kulH+4*\kpsep,-\ksepY)--++(0,-\kpin);
\draw(\kulH+3*\kpsep,-\ksepY)++(0,-\kpin)++(-0.5*\kpsep,0.5*\kpsep)--++(\kpsep,-\kpsep)--++(\kpsep,\kpsep);
\draw(\kdimX,-\ksepY+\kulV+1.5*\kpsep)++(0.5*\kpsep,0)--++(\kpina,0);
\draw(\kdimX,-\ksepY+\kulV+2.5*\kpsep)++(0.5*\kpsep,0)--++(\kpina,0);
\draw(\kdimX,-\ksepY+\kulV+2*\kpsep)--++(1*\kpsep,-1*\kpsep);
\draw(\kdimX,-\ksepY+\kulV+2*\kpsep)--++(1*\kpsep,1*\kpsep);

\draw(0,-2*\ksepY)  [thick] rectangle ++(\kdimX,\kdimY);%node[pos=0.5,rectangle,inner sep=0pt,text width=1cm,align=center]{\footnotesize{\RTL{حافظہ}}};
\draw(\kdimX,-2*\ksepY+\kulV+1.5*\kpsep)--++(\kpina,0);
\draw(\kdimX,-2*\ksepY+\kulV+2.5*\kpsep)--++(\kpina,0);
\draw(\kdimX,-2*\ksepY+\kulV+1.5*\kpsep)++(\kpina,0)++(-0.5*\kpsep,-0.5*\kpsep)--++(1*\kpsep,1*\kpsep)--++(-1*\kpsep,1*\kpsep);
\end{scope}

\draw(0,-3*\ksepY)  [thick] rectangle ++(\kdimX,\kdimY)node[pos=0.5,rectangle,inner sep=0pt,text width=1cm,align=center]{ \RTL{دفتر ہدایت}};
\draw[inactivePart](\kdimX,-3*\ksepY+\kulV+0*\kpsep)--++(\kpina,0);
\draw[inactivePart](\kdimX,-3*\ksepY+\kulV+1*\kpsep)--++(\kpina,0);
\draw[inactivePart](\kdimX,-3*\ksepY+\kulV+0*\kpsep)++(\kpina,0)++(-0.5*\kpsep,-0.5*\kpsep)--++(1*\kpsep,1*\kpsep)--++(-1*\kpsep,1*\kpsep);
\draw[inactivePart](\kdimX,-3*\ksepY+\kulV+3*\kpsep)++(0.5*\kpsep,0)--++(\kpina,0);
\draw[inactivePart](\kdimX,-3*\ksepY+\kulV+4*\kpsep)++(0.5*\kpsep,0)--++(\kpina,0);
\draw[inactivePart](\kdimX,-3*\ksepY+\kulV+3.5*\kpsep)--++(1*\kpsep,-1*\kpsep);
\draw[inactivePart](\kdimX,-3*\ksepY+\kulV+3.5*\kpsep)--++(1*\kpsep,1*\kpsep);
\draw(\kulH+1.5*\kpsep,-3*\ksepY)--++(0,-\kpin);
\draw(\kulH+2.5*\kpsep,-3*\ksepY)--++(0,-\kpin);
\draw(\kulH+1.5*\kpsep,-3*\ksepY)++(0,-\kpin)++(-0.5*\kpsep,0.5*\kpsep)--++(\kpsep,-\kpsep)--++(\kpsep,\kpsep);

\draw(0,-4*\ksepY)  [thick] rectangle ++(\kdimX,\kdimY)node[pos=0.5,rectangle,inner sep=0pt,text width=1.5cm,align=center]{\RTL{قابو    }};

\draw(\kulH+1.5*\kpsep,-4*\ksepY)--++(0,-\kpin);
\draw(\kulH+2.5*\kpsep,-4*\ksepY)--++(0,-\kpin);
\draw(\kulH+1.5*\kpsep,-4*\ksepY)++(0,-\kpin)++(-0.5*\kpsep,0.5*\kpsep)--++(\kpsep,-\kpsep)--++(\kpsep,\kpsep);
\draw(\kulH+2*\kpsep,-4*\ksepY)++(0,-1*\kpin-\kpsep)node[below,rectangle,inner sep=0pt, text width=1cm]{\RTL{قابو لفظ}};


\draw(\kdimX+\kpina+0.5*\kpsep,\kdimY)  [thick] rectangle ++(\kW,-4*\ksepY+\kpin);


\draw(\ksepX,-0*\ksepY)  [thick] rectangle ++(\kdimX,\kdimY)node[pos=0.5,rectangle,inner sep=0pt,text width=1cm,align=center]{\RTL{$A$}};
\draw(\ksepX,-0*\ksepY)++(\kdimX,\kulV+0*\kpsep)--++(\kpin,0)node[right]{$E_A$};
\draw(\ksepX,-0*\ksepY+\kulV+0*\kpsep)--++(-\kpina,0); 
\draw(\ksepX,-0*\ksepY+\kulV+1*\kpsep)--++(-\kpina,0); 
\draw(\ksepX,-0*\ksepY+\kulV+0*\kpsep)++(-\kpina,0)++(0.5*\kpsep,-0.5*\kpsep)--++(-\kpsep,\kpsep)--++(\kpsep,\kpsep);
\draw[inactivePart](\ksepX,-0*\ksepY+\kulV+3*\kpsep)++(-0.5*\kpsep,0)--++(-\kpina,0); 
\draw[inactivePart](\ksepX,-0*\ksepY+\kulV+4*\kpsep)++(-0.5*\kpsep,0)--++(-\kpina,0); 
\draw[inactivePart](\ksepX,-0*\ksepY+\kulV+3*\kpsep)++(-1*\kpsep,-0.5*\kpsep)--++(\kpsep,\kpsep)--++(-\kpsep,\kpsep);
\draw[inactivePart](\ksepX+\kulH+1.5*\kpsep,-0*\ksepY)--++(0,-\kpin);
\draw[inactivePart](\ksepX+\kulH+2.5*\kpsep,-0*\ksepY)--++(0,-\kpin);
\draw[inactivePart](\ksepX+\kulH+1.5*\kpsep,-0*\ksepY)++(0,-\kpin)++(-0.5*\kpsep,0.5*\kpsep)--++(\kpsep,-\kpsep)--++(\kpsep,\kpsep);

\begin{scope}[inactivePart]
\draw(\ksepX,-1*\ksepY)  [thick] rectangle ++(\kdimX,\kdimY);%node[pos=0.5,rectangle,inner sep=0pt,text width=1cm,align=center]{\RTL{جمع و منفی کار}};
\draw(\ksepX,-1*\ksepY)++(\kdimX,\kulV+0*\kpsep)--++(\kpin,0)node[right]{$E_U$};
\draw(\ksepX,-1*\ksepY+\kulV+1.5*\kpsep)--++(-\kpina,0); 
\draw(\ksepX,-1*\ksepY+\kulV+2.5*\kpsep)--++(-\kpina,0); 
\draw(\ksepX,-1*\ksepY+\kulV+1.5*\kpsep)++(-\kpina,0)++(0.5*\kpsep,-0.5*\kpsep)--++(-\kpsep,\kpsep)--++(\kpsep,\kpsep);
\draw(\ksepX+\kulH+1.5*\kpsep,-1*\ksepY)++(0,-0.5*\kpsep)--++(0,-\kpin);
\draw(\ksepX+\kulH+2.5*\kpsep,-1*\ksepY)++(0,-0.5*\kpsep)--++(0,-\kpin);
\draw(\ksepX+\kulH+1.5*\kpsep,-1*\ksepY)++(-0.5*\kpsep,-\kpsep)--++(\kpsep,\kpsep)--++(\kpsep,-\kpsep);


\draw(\ksepX,-2*\ksepY)  [thick] rectangle ++(\kdimX,\kdimY);%node[pos=0.5,rectangle,inner sep=0pt,text width=1cm,align=center]{\RTL{$B$}};
\draw[inactivePart](\ksepX,-2*\ksepY+\kulV+1.5*\kpsep)++(-0.5*\kpsep,0)--++(-\kpina,0); 
\draw[inactivePart](\ksepX,-2*\ksepY+\kulV+2.5*\kpsep)++(-0.5*\kpsep,0)--++(-\kpina,0); 
\draw[inactivePart](\ksepX,-2*\ksepY+\kulV+1.5*\kpsep)++(-1*\kpsep,-0.5*\kpsep)--++(\kpsep,\kpsep)--++(-\kpsep,\kpsep);
\end{scope}

\draw(\ksepX,-3*\ksepY)  [thick] rectangle ++(\kdimX,\kdimY)node[pos=0.5,rectangle,inner sep=0pt,text width=1cm,align=center]{\RTL{خارجی دفتر}};
\draw(\ksepX,-3*\ksepY)++(\kdimX,\kulV+4*\kpsep)--++(\kpin,0)node[right]{$\overline{L}_O$};
\draw(\ksepX,-3*\ksepY)++(\kdimX,\kulV+4*\kpsep)++(\knshift,0)node[ocirc]{};
\draw(\ksepX,-3*\ksepY+\kulV+1.5*\kpsep)++(-0.5*\kpsep,0)--++(-\kpina,0); 
\draw(\ksepX,-3*\ksepY+\kulV+2.5*\kpsep)++(-0.5*\kpsep,0)--++(-\kpina,0); 
\draw(\ksepX,-3*\ksepY+\kulV+1.5*\kpsep)++(-1*\kpsep,-0.5*\kpsep)--++(\kpsep,\kpsep)--++(-\kpsep,\kpsep);
\draw(\ksepX,-3*\ksepY)++(\kulH+1.5*\kpsep,0)--++(0,-\kpin);
\draw(\ksepX,-3*\ksepY)++(\kulH+2.5*\kpsep,0)--++(0,-\kpin);
\draw(\ksepX,-3*\ksepY)++(\kulH+1.5*\kpsep,0)++(0,-\kpin)++(-0.5*\kpsep,0.5*\kpsep)--++(\kpsep,-\kpsep)--++(\kpsep,\kpsep);

\draw(\ksepX,-4*\ksepY)  [thick] rectangle ++(\kdimX,\kdimY)node[pos=0.5,rectangle,inner sep=0pt,text width=1cm,align=center]{\RTL{تختی}};
\end{tikzpicture}
\caption{برآمد ہدایت کے دوران \عددی{T_4} حال۔}
\label{شکل_کمپیوٹر_برآمد_ہدایت}
\end{figure}

شکل \حوالہ{شکل_کمپیوٹر_بازیابی_برآمد_وقتیہ} میں  بازیابی اور برآمد وقتیہ ترسیمات پیش ہیں۔  بازیابی حال   ہمیشہ کی طرح پتہ حال، بڑھوتری حال، اور حافظہ حال پر مشتمل ہو گا۔ \عددی{T_4} حال کے دوران، \عددی{E_A} اور \عددی{\overline{L}_O} فعال ہوں گے؛ لہٰذا ساعت کے اگلے کنارہ چڑھائی پر دفتر \عددی{A} کی معلومات خارجی دفتر کو منتقل ہو گی۔

\begin{figure}
\centering
\begin{otherlanguage}{english}
 \begin{tikztimingtable}[%
timing/.style={x=4ex,y=3ex},
timing/rowdist=6ex,
every node/.style={inner sep=0,outer sep=0},
%timing/c/arrow tip=latex, %and this set the style
%timing/c/rising arrows,
timing/slope=0, %0.1 is good
timing/dslope=0,
thick,
]
%\tikztimingmetachar{R}{[|/utils/exec=\setcounter{new}{0}|]}
%\usetikztiminglibrary[new={char=Q,reset char=R}]{counters}
%[timing/counter/new={char=c, base=2,digits=3,max value=7, wraps ,text style={font=\normalsize}}] 12{2c} \\ 
%$C$& H22{C}\\
%$\texturdu{\RL{پتہ}}$&3D{} 20D{[scale=1.5]\texturdu{\RL{درست پتہ}}} 3D{}\\
$CLK$&HN(a)LN(ca)HN(b)LN(cb)HN(c)LN(cc)HN(d)LN(cd)HN(e)LN(ce)HN(f)LN(cf)HN(g)L\\
$E_P$&LHHLLLLLLLLLLH\\
$\overline{L}_M$&HLL4{H}2{H}5{H}\\
$C_P$&LLLHHLLLLLLLLL\\
$\overline{CE}$&HHHHHLLHHHHHHH\\
$\overline{L}_I$&HHHHHLLHHHHHHH\\
$E_A$&LLLLLLLHHLLLLL\\
$\overline{L}_O$&HHHHHHHLLHHHHH\\
\extracode
\begin{pgfonlayer}{background}
\begin{scope}[]
\draw [latex-latex] ($(a|-row1.north)+(0,2ex)$) --node[fill=white]{$T_1$} ($(b|-row1.north)+(0,2ex)$);
\draw [latex-latex] ($(b|-row1.north)+(0,2ex)$) --node[fill=white]{$T_2$} ($(c|-row1.north)+(0,2ex)$);
\draw [latex-latex] ($(c|-row1.north)+(0,2ex)$) --node[fill=white]{$T_3$} ($(d|-row1.north)+(0,2ex)$);
\draw [latex-latex] ($(d|-row1.north)+(0,2ex)$) --node[fill=white]{$T_4$} ($(e|-row1.north)+(0,2ex)$);
\draw [latex-latex] ($(e|-row1.north)+(0,2ex)$) --node[fill=white]{$T_5$} ($(f|-row1.north)+(0,2ex)$);
\draw [latex-latex] ($(f|-row1.north)+(0,2ex)$) --node[fill=white]{$T_6$} ($(g|-row1.north)+(0,2ex)$);
\foreach \n in {a,b,c,d,e,f,g}{\draw[thin]($(\n|-row1.north)+(0,1ex)$)--++(0,2ex);}
\draw[dashed] (ca)--(ca |-row3.south);
\draw[dashed] (cb)--(cb |-row4.north);
\draw[dashed] (cc)--(cc |-row6.south);
\draw[dashed] (cd)--(cd |-row8.south);
%%\vertlines[darkgray,dotted]{3.6,7.5,11.5,15.5,19.5,23.5,27.5}
%\foreach \n in {1,3,...,11} \draw(4*\n ex-2ex,-4ex+1.25ex)node[]{$0$};
%\foreach \n in {2,4,...,12} \draw(4*\n ex-2ex,-4ex+1.25ex)node[]{$1$};
%\foreach \n in {1,2,5,6,9,10} \draw(4*\n ex-2ex,-12 ex+1.25ex)node[]{$0$};
%\foreach \n in {3,4,7,8,11,12} \draw(4*\n ex-2ex,-12 ex+1.25ex)node[]{$1$};
%\foreach \n in {1,2,3,4,9,10,11,12} \draw(4*\n ex-2ex,-20 ex+1.25ex)node[]{$0$};
%\foreach \n in {5,6,7,8} \draw(4*\n ex-2ex,-20 ex+1.25ex)node[]{$1$};
%\draw(4*4 ex-2ex,-12 ex+1.25ex) circle (0.25cm and 1.75cm);
%\draw(4*4 ex-2ex,-7*\rowdist+1.25ex) circle (0.25cm and 0.5cm);
%%\foreach \n in {1}\draw(B\n.south)--(F\n.north);
%%\foreach \n in {1}\draw(C\n.south)--(G\n.north);
\end{scope}
\end{pgfonlayer}
\end{tikztimingtable}
\end{otherlanguage}
\caption{بازیابی اور برآمد   وقتیہ  ترسیمات۔}
\label{شکل_کمپیوٹر_بازیابی_برآمد_وقتیہ}
\end{figure}

\جزوحصہء{رک}
رک کی ہدایت  پر عمل در آمد  کے دوران کسی دفتر کی ضرورت پیش نہیں آتی، لہٰذا  اس کے لئے طریق  قابو  درکار نہیں ہو گا۔ جب دفتر ہدایت میں درج ذیل موجود ہو
\begin{align*}
1111\, xxxx=\text{\RL{دفتر ہدایت}}
\end{align*}
جزو ہدایت \عددی{1111} قابو و ترتیب کار کو  مواد پر عمل  نہ کرنے کا اشارہ کرتا ہے۔ قابو و ترتیب کار ساعت (جس کے دور پر کچھ دیر میں غور کیا جائے گا)   روک کر کمپیوٹر کو  مزید کام کرنے سے روک لیتا ہے۔

\جزوحصہء{مشینی پھیرا اور ہدایتی پھیرا}
اس سادہ  کمپیوٹر  کے چھ \عددی{T} حال ہیں، جن میں سے تین بازیابی اور تین تعمیلی ہیں ۔ ان چھ حال کو\اصطلاح{ مشینی پھیرا }\فرہنگ{پھیرا!مشینی}\حاشیہب{machine cycle}\فرہنگ{cycle!machine} کہتے ہیں (شکل  \حوالہ{شکل_کمپیوٹر_مشینی_پھیرے}-الف  دیکھیں)۔ ایک مشینی پھیرے میں ایک ہدایت  کی بازیابی اور تعمیل کی جاتی ہے۔ اس سادہ ترین کمپیوٹر کی ساعت  کا تعدد \عددی{\SI{1}{\kilo\hertz}}  ہے، لہٰذا اس کا دوری عرصہ \عددی{\SI{1}{\milli\second}} ہو گا۔ یوں ہر  مشینی پھیرا \عددی{\SI{6}{\milli\second}} لیگا۔

کئی کمپیوٹر میں  ہدایت کی بازیابی اور تعمیل کرنا  ایک سے زائد  مشینی پھیروں میں ممکن ہو گا۔ شکل  \حوالہ{شکل_کمپیوٹر_مشینی_پھیرے}-ب  میں دو  مشینی پھیروں  کی ہدایت کا وقتیہ ترسیم پیش ہے۔ اولین تین \عددی{T} حال  بازیابی پھیرا دیتے ہیں؛ تاہم تعمیلی پھیرے کو اگلے نو \عددی{T} حال درکار ہیں۔  دو مشینی پھیرے کی ہدایت  زیادہ پیچیدہ ہو گی جس کی تعمیل  کے لئے اضافی \عددی{T} حال درکار ہوں گے۔

ایک ہدایت کی بازیابی اور تعمیل کے لئے درکار \عددی{T} حال کو \اصطلاح{ ہدایتی پھیرا }\فرہنگ{پھیرا!ہدایتی}\حاشیہب{instruction cycle}\فرہنگ{cycle!instruction}کہتے ہیں۔ اس سادہ ترین کمپیوٹر میں ہدایتی پھیرا اور مشینی پھیرا ایک  برابر ہیں، جبکہ  شکل \حوالہ{شکل_کمپیوٹر_مشینی_پھیرے}-ب میں ہدایتی پھیرا دو مشینی پھیروں کے برابر ہے۔

\عددی{8080} اور \عددی{8085}  کے ہدایتی پھیرے ایک سے  پانچ مشینی پھیروں کے برابر ہو سکتے ہیں۔ 

\begin{figure}
\centering
\begin{subfigure}{1\textwidth}
\centering
\begin{otherlanguage}{english}
 \begin{tikztimingtable}[%
timing/.style={x=2.75ex,y=3ex},
timing/rowdist=6ex,
every node/.style={inner sep=0,outer sep=0},
%timing/c/arrow tip=latex, %and this set the style
%timing/c/rising arrows,
timing/slope=0, %0.1 is good
timing/dslope=0,
thick,
]
%\tikztimingmetachar{R}{[|/utils/exec=\setcounter{new}{0}|]}
%\usetikztiminglibrary[new={char=Q,reset char=R}]{counters}
%[timing/counter/new={char=c, base=2,digits=3,max value=7, wraps ,text style={font=\normalsize}}] 12{2c} \\ 
%$C$& H22{C}\\
%$\texturdu{\RL{پتہ}}$&3D{} 20D{[scale=1.5]\texturdu{\RL{درست پتہ}}} 3D{}\\
$CLK$&HN(a)LN(ca)HN(b)LN(cb)HN(c)LN(cc)HN(d)LN(cd)HN(e)LN(ce)HN(f)LN(cf)HN(g)L\\
\extracode
\begin{pgfonlayer}{background}
\begin{scope}[]
\foreach \n/\a in {ca/1,cb/2,cc/3,cd/4,ce/5,cf/6}\draw ($(\n |-row1.south)+(0,-3ex)$) node[]{$T_\a$};
\foreach \n in {b,c,e,f}{\draw[thin]($(\n|-row1.south)+(0,-1ex)$)--($(\n|-row2.mid)$);}
\draw($(a|-row1.south)+(0,-1ex)$)--++(0,-14ex);
\draw($(g|-row1.south)+(0,-1ex)$)--++(0,-14ex);
\draw($(d|-row1.south)+(0,-1ex)$)--($(d|-row3.north)+(0,1ex)$);
\draw[stealth-stealth]($(a|-row3.north)+(0,2ex)$)--node[fill=white]{\texturdu{\RL{بازیابی پھیرا}}}($(d|-row3.north)+(0,2ex)$);
\draw[stealth-stealth]($(d|-row3.north)+(0,2ex)$)--node[fill=white]{\texturdu{\RL{تعمیلی پھیرا}}}($(g|-row3.north)+(0,2ex)$);
\draw[stealth-stealth](a|-row3.mid)--node[fill=white]{\texturdu{\RL{مشینی پھیرا}}}(g|-row3.mid);
\draw[stealth-stealth]($(a|-row3.south)+(0,-2ex)$)--node[fill=white]{\texturdu{\RL{ہدایتی پھیرا}}}($(g|-row3.south)+(0,-2ex)$);
%\draw[dashed] (cd)--(cd |-row8.south);
%%\vertlines[darkgray,dotted]{3.6,7.5,11.5,15.5,19.5,23.5,27.5}
%\foreach \n in {1,3,...,11} \draw(4*\n ex-2ex,-4ex+1.25ex)node[]{$0$};
\end{scope}
\end{pgfonlayer}
\end{tikztimingtable}
\end{otherlanguage}
\caption{}
\end{subfigure}
\begin{subfigure}{1\textwidth}
\centering
\begin{otherlanguage}{english}
 \begin{tikztimingtable}[%
timing/.style={x=2.75ex,y=3ex},
timing/rowdist=6ex,
every node/.style={inner sep=0,outer sep=0},
%timing/c/arrow tip=latex, %and this set the style
%timing/c/rising arrows,
timing/slope=0, %0.1 is good
timing/dslope=0,
thick,
]
%\tikztimingmetachar{R}{[|/utils/exec=\setcounter{new}{0}|]}
%\usetikztiminglibrary[new={char=Q,reset char=R}]{counters}
%[timing/counter/new={char=c, base=2,digits=3,max value=7, wraps ,text style={font=\normalsize}}] 12{2c} \\ 
%$C$& H22{C}\\
%$\texturdu{\RL{پتہ}}$&3D{} 20D{[scale=1.5]\texturdu{\RL{درست پتہ}}} 3D{}\\
$CLK$&HN(a)LN(ca)HN(b)LN(cb)HN(c)LN(cc)HN(d)LN(cd)HN(e)LN(ce)HN(f)L
N(cf)HN(g)LN(cg)HN(h)LN(ch)HN(i)LN(ci)HN(j)LN(cj)HN(k)LN(ck)HN(l)LN(cl)HN(m)L\\
\extracode
\begin{pgfonlayer}{background}
\begin{scope}[]
\foreach \n/\a in {ca/1,cb/2,cc/3,cd/4,ce/5,cf/6}\draw ($(\n |-row1.south)+(0,-3ex)$) node[]{$T_\a$};
\foreach \n/\a in {cg/1,ch/2,ci/3,cj/4,ck/5,cl/6}\draw ($(\n |-row1.south)+(0,-3ex)$) node[]{$T_\a$};
\foreach \n in {h,i,j,k,l}{\draw[thin]($(\n|-row1.south)+(0,-1ex)$)--($(\n|-row2.mid)$);}
\draw($(a|-row1.south)+(0,-1ex)$)--++(0,-14ex);
\draw($(g|-row1.south)+(0,-1ex)$)--++(0,-10ex);
\draw($(m|-row1.south)+(0,-1ex)$)--++(0,-14ex);
\draw($(d|-row1.south)+(0,-1ex)$)--($(d|-row3.north)+(0,1ex)$);
\draw[stealth-stealth]($(a|-row3.north)+(0,2ex)$)--node[fill=white]{\texturdu{\RL{بازیابی پھیرا}}}($(d|-row3.north)+(0,2ex)$);
\draw[stealth-stealth]($(d|-row3.north)+(0,2ex)$)--node[fill=white]{\texturdu{\RL{تعمیلی پھیرا}}}($(g|-row3.north)+(0,2ex)$);
\draw[stealth-stealth]($(g|-row3.north)+(0,2ex)$)--node[fill=white]{\texturdu{\RL{تعمیلی پھیرا}}}($(m|-row3.north)+(0,2ex)$);
\draw[stealth-stealth](a|-row3.mid)--node[fill=white]{\texturdu{\RL{مشینی پھیرا}}}(g|-row3.mid);
\draw[stealth-stealth](g|-row3.mid)--node[fill=white]{\texturdu{\RL{مشینی پھیرا}}}(m|-row3.mid);
\draw[stealth-stealth]($(a|-row3.south)+(0,-2ex)$)--node[fill=white]{\texturdu{\RL{ہدایتی پھیرا}}}($(m|-row3.south)+(0,-2ex)$);
%\draw[dashed] (cd)--(cd |-row8.south);
%%\vertlines[darkgray,dotted]{3.6,7.5,11.5,15.5,19.5,23.5,27.5}
%\foreach \n in {1,3,...,11} \draw(4*\n ex-2ex,-4ex+1.25ex)node[]{$0$};
\end{scope}
\end{pgfonlayer}
\end{tikztimingtable}
\end{otherlanguage}
\caption{}
\end{subfigure}
\caption{(ا) ہدایتی پھیرا؛ (ب) دو مشینی پھیروں پر مبنی ہدایتی پھیرا۔}
\label{شکل_کمپیوٹر_مشینی_پھیرے}
\end{figure}

%--------------------------------
\ابتدا{مثال}
\عددی{8080/8085} کا معلوماتی کتابچہ  کہتا ہے\قول{ نقل}  کی ہدایت  کی بازیابی اور تعمیل کے لئے تیرہ \عددی{T} حال درکار ہوں گے۔ اگر کمپیوٹر کی ساعت کا تعدد \عددی{\SI{2.5}{\mega\hertz}} ہو، اس  ہدایت کو کتنا وقت درکار ہو گا؟

حل:\quad
ساعت کا دوری عرصہ درج ذیل ہو گا۔
\begin{align*}
T=\frac{1}{f}=\frac{1}{\SI{2.5}{\mega\hertz}}=\SI{400}{\nano\second}
\end{align*}
چونکہ  ہر ایک \عددی{T} حال کو   \عددی{\SI{400}{\nano\second}} درکار ہیں  اور \قول{نقل} کی ہدایت کی بازیابی اور تعمیل   تیرہ \عددی{T} حال    میں  ممکن  ہے لہٰذا اس ہدایت کو درج ذیل وقت درکار ہو گا۔
\begin{align*}
13\times \SI{400}{\nano\second}=\SI{5.2}{\micro\second}
\end{align*}
\انتہا{مثال}
%------------------------------------
\ابتدا{مثال}
شکل \حوالہ{شکل_کمپیوٹر_کنارہ_چڑھائی_وسط_میں}  میں سادہ کمپیوٹر کے چھ \عددی{T} حال دکھائے گئے ہیں۔ ساعت کا (تیر دار)   کنارہ چڑھائی   نصف حال گزر کر آتا ہے۔ ایسا کیوں ہے؟

\begin{figure}
\centering
\begin{otherlanguage}{english}
 \begin{tikztimingtable}[%
timing/.style={x=4ex,y=3ex},
timing/rowdist=6ex,
every node/.style={inner sep=0,outer sep=0},
timing/c/arrow tip=latex, %and this set the style
timing/c/rising arrows,
timing/slope=0, %0.1 is good
timing/dslope=0,
thick,
]
%\tikztimingmetachar{R}{[|/utils/exec=\setcounter{new}{0}|]}
%\usetikztiminglibrary[new={char=Q,reset char=R}]{counters}
%[timing/counter/new={char=c, base=2,digits=3,max value=7, wraps ,text style={font=\normalsize}}] 12{2c} \\ 
%$C$& H22{C}\\
%$\texturdu{\RL{پتہ}}$&3D{} 20D{[scale=1.5]\texturdu{\RL{درست پتہ}}} 3D{}\\
$CLK$&HN(a)CN(b)CN(c)CN(d)CN(e)CN(f)CN(g)CN(h)CN(i)CN(j)CN(k)CN(l)CN(m)L\\
\extracode
\begin{pgfonlayer}{background}
\begin{scope}[]
\foreach \n in {a,c,e,g,i,k,m}{\draw (\n|-row1.south)++(0,-1ex)--(\n|-row2.mid);}
\foreach \n/\a  in {b/1,d/2,f/3,h/4,j/5,l/6}{\draw(\n|-row1.south)++(0,-3ex)node[]{$T_{\a}$};}
\end{scope}
\end{pgfonlayer}
\end{tikztimingtable}
\end{otherlanguage}
\caption{ساعت کا کنارہ چڑھائی \عددی{T} حال کے وسط میں پایا جاتا ہے۔}
\label{شکل_کمپیوٹر_کنارہ_چڑھائی_وسط_میں}
\end{figure}

حل:\quad
جدید کمپیوٹر کی طرح اس کمپیوٹر میں مواد کا تبادلہ بذریعہ \عددی{W}  گزرگاہ ہوتا ہے۔ تاہم دفتر  کی بغیر مسئلہ  بھرائی اس صورت ممکن ہو گی  جب دورانیہ تیاری اور  دورانیہ  ٹھیراؤ  مطمئن ہوں۔ نصف  دوری عرصہ انتظار   کر  کے دفتر میں  بھرائی ، دورانیہ تیاری کو مطمئن کرتا ہے؛ بھرائی کے بعد نصف دوری عرصہ کا انتظار، دورانیہ ٹھیراؤ کو مطمئن کرتا ہے۔ اسی لئے ساعت کا کنارہ چڑھائی  \عددی{T} حال کے عین وسط میں رکھا جاتا ہے (شکل  \حوالہ{شکل_کمپیوٹر_کنارہ_چڑھائی_وسط_میں})۔

نصف دوری عرصہ انتظار کرنے کی  دوسری وجہ بھی ہے۔ مواد ترسیل کرنے والے دفتر کا \قول{مجاز}   اشارہ فعال کرنے  سے \عددی{W} گزرگاہ  پر مواد ایک دم ڈلتا ہے۔ غیر مطلوبہ  برقی گنجائش اور تاروں کے    امالہ کی بدولت  گزرگاہ  تاروں میں برقی دباو کی درست  سطح  کے حصول میں وقت درکار ہوتا ہے۔ دوسرے لفظوں میں   \عددی{W} گزرگاہ  پر عبوری حال   پیدا ہو گا؛بوقت بھرائی درست مواد یقینی بنانے کے لئے ضروری ہے کہ  ا عبوری حال  کے اختتام  کا  انتظار کیا جائے۔
\انتہا{مثال}
%----------------------------

\حصہ{خرد برنامہ}
ہم جلد اس سادہ کمپیوٹر   کے دوری  نقشہ  پر غور کریں گے، لیکن اس سے قبل بہتر ہو گا ہم اس کی ہدایات کی تعمیل کو ایک  جدول میں،  جسے\اصطلاح{ خرد برنامہ }\فرہنگ{خرد برنامہ}\حاشیہب{microprogram}\فرہنگ{microprogram} کہتے ہیں،  یکجا کریں۔

\جزوحصہء{خرد ہدایات}
ہر ایک \عددی{T} حال کے دوران قابو و ترتیب کار ایک قابو لفظ خارج کرتا ہے۔ یہ  لفظ  کمپیوٹر کے باقی حصوں کو  بتاتا ہے کہ ان  نے کیا کام سرانجام دینا ہے۔ چونکہ   یہ  لفظ مواد پر عمل  کا ایک چھوٹا قدم  پیدا کرتا ہے لہٰذا یہ \اصطلاح{ خرد ہدایت  }فرہنگ{خرد ہدایت}\حاشیہب{microinstruction}\فرہنگ{microinstruction} کہلاتا ہے۔ شکل \حوالہ{شکل_کمپیوٹر_سادہ_ترین} کو دیکھتے ہوئے   قابو و ترتیب کار سے باقی ادوار کو مسلسل خرد ہدایات جاری ہونا  ہم تصور کر سکتے ہیں۔

\جزوحصہء{کلاں ہدایات}
برنامے کی ہدایات (نقل، جمع، منفی، وغیرہ) کو  بعض اوقات\اصطلاح{ کلاں ہدایات }\فرہنگ{کلاں ہدایات}\حاشیہب{macroinstructions}\فرہنگ{macroinstructions} کہتے ہیں تا کہ ان میں اور خرد ہدایات میں تمیز    ہو۔ سادہ ترین کمپیوٹر کی ہر ایک کلاں ہدایت تین خرد ہدایات پر مشتمل ہے۔ مثلاً، نقل کی کلاں ہدایت جدول \حوالہ{جدول_کمپیوٹر_کلاں_اور_خرد} میں پیش تین خرد ہدایات پر مشتمل ہے۔  آسان بنانے کی غرض سے ہم  خرد ہدایات کو اساس سولہ میں لکھتے ہیں (جدول \حوالہ{جدول_کمپیوٹر_خرد_اساس_سولہ} دیکھیں)۔
\begin{table}
\caption{نقل ہدایت تین خرد ہدایات پر مشتمل ہے۔}
\label{جدول_کمپیوٹر_کلاں_اور_خرد}
\centering
\begin{tabular}{RRRR}
\toprule
\text{\RL{کلاں}}&\text{\RL{حال}}&
C_PE_P\overline{L}_M\overline{CE}\quad  \overline{L}_I\overline{E}_I\overline{L}AE_A\quad S_UE_U\overline{L}_B\overline{L}_O&\text{\RL{فعال}}\\
\midrule
\text{\RL{نقل}}& T_4& 
\,\,0\,\,\,\,\,0\,\,\,\,\,0\,\,\,\,\,1\quad\,\, 1\,\,\,\,0\,\,\,\,1\,\,\,\,0\quad \quad 0\,\,\,\,0\,\,\,\,1\,\,\,\,1&\overline{L}_M, \overline{E}_I\\
& T_5& 
\,\,0\,\,\,\,\,0\,\,\,\,\,1\,\,\,\,\,0\quad\,\, 1\,\,\,\,1\,\,\,\,0\,\,\,\,0\quad \quad 0\,\,\,\,0\,\,\,\,1\,\,\,\,1&\overline{CE}, \overline{L}_A\\
& T_6& 
\,\,0\,\,\,\,\,0\,\,\,\,\,1\,\,\,\,\,1\quad\,\, 1\,\,\,\,1\,\,\,\,1\,\,\,\,0\quad \quad 0\,\,\,\,0\,\,\,\,1\,\,\,\,1&\text{\RL{کوئی نہیں}}\\
\bottomrule
\end{tabular}
\end{table}

\begin{table}
\caption{نقل ہدایت  کی اساس سولہ  خرد ہدایات۔}
\label{جدول_کمپیوٹر_خرد_اساس_سولہ}
\centering
\begin{tabular}{RRRR}
\toprule
\text{\RL{کلاں}}&\text{\RL{حال}}&
\text{\RL{قابو لفظ}}&\text{\RL{فعال}}\\
\midrule
\text{\RL{نقل}}& T_4& 
1A3H&\overline{L}_M, \overline{E}_I\\
& T_5& 
2C3H&\overline{CE}, \overline{L}_A\\
& T_6& 
3E3H&\text{\RL{کوئی نہیں}}\\
\bottomrule
\end{tabular}
\end{table}

جدول \حوالہ{جدول_کمپیوٹر_خرد_برنامہ} میں سادہ کمپیوٹر کا خرد برنامہ پیش ہے، جس میں ہر کلاں ہدایت اور  اس کی تعمیل کے لئے درکار خرد ہدایات  دیے گئے ہیں۔ یہ  جدول  سادہ کمپیوٹر کے طریق تعمیل   کا خلاصہ ہے۔ زیادہ جدید ہدایات کے لئے بھی ایسا جدول لکھا جا سکتا ہے۔
\begin{table}
\caption{سادہ کمپیوٹر کا خرد  برنامہ}
\label{جدول_کمپیوٹر_خرد_برنامہ}
\centering
\begin{tabular}{RRRL}
\toprule
\text{\RL{کلاں}}&\text{\RL{حال}}&
\text{\RL{قابو لفظ}}&\text{\RL{فعال}}\\
\midrule
\text{\RL{نقل}}& T_4& 
1A3H&\overline{L}_M, \overline{E}_I\\
& T_5& 
2C3H&\overline{CE}, \overline{L}_A\\
& T_6& 
3E3H&\text{\RL{کوئی نہیں}}\\
\text{\RL{جمع}}& T_4 & 1A3H&\overline{L}_M, \overline{E}_I\\
&T_5&2E1H&\overline{CE},\overline{L}_B\\
&T_6&3C7H&\overline{L}_A, E_U\\
\text{\RL{منفی}}& T_4& 1A3H&\overline{L}_M, \overline{E}_I\\
&T_5&2E1H&\overline{CE},\overline{L}_B\\
&T_6&3CFH&\overline{L}_A, S_U,E_U\\
\text{\RL{بر آمد}}&T_4&3F2H&E_A,\overline{L}_O\\
&T_5&3E3H&\text{\RL{کوئی نہیں}}\\
&T_6&3E3H&\text{\RL{کوئی نہیں}}\\
\bottomrule
\end{tabular}
\end{table}

\حصہ{سادہ کمپیوٹر کا نقشہ دور}
اس حصے میں سادہ کمپیوٹر کے مکمل نقشہ دور پر غور کیا جائے گا۔ شکل \حوالہء{10.12} تا شکل \حوالہء{10.15} میں تمام مخلوط ادوار، برقی تاریں، اور اشارات دکھائے گئے ہیں۔ آگے پڑھتے ہوئے ان  اشکال سے رجوع کریں۔

\جزوحصہء{برنامہ گنت کار}


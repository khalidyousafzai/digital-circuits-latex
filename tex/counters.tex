\باب{ گنت کار}
ثنائی گنت کار آپ دیکھ چکے ہیں۔گنت کار کا بنیادی مقصد  داخلی \اصطلاح{ برقی اشارے  }\فرہنگ{اشارہ!برقی}\حاشیہب{electrical signal}\فرہنگ{signal!electrical}کی گنتی  کرنا ہے۔برقی اشارہ اسے بطور ساعت یا سادہ مداخل کے طور پر  مہیا کیا جا تا ہے۔

وہ  دفتر جس کے خارجی برقی اشارات  ثنائی گنتی کے تحت ترتیب وار حال تبدیل کرتے ہوں \اصطلاح{ ثنائی گنت کار } کہلاتا ہے۔ وہ  دفتر  جس کے  خارجی اشارات اعشاری گنتی کے تحت ترتیب  وار حال تبدیل کرتے ہوں \اصطلاح{ اعشاری گنت کار }  کہلاتا ہے۔

ان کے علاوہ،   کوئی بھی دور جو کسی متعین ترتیب کے تحت  متواتر  حال تبدیل کرتا ہو  گنت کار کہلائے گا۔

گنت کار ادوار پر اس باب میں غور کیا جائے گا۔

\حصہ{ثنائی گنت کار}
چار بِٹ ثنائی   سیدھی  گنتی  \عددی{0000_2}  تا \عددی{1111_2}   ممکن ہے۔اسی طرح الٹی گنتی  \عددی{1111_2} سے  شروع ہو کر    \عددی{0000_2}  پر ختم ہو گی۔دونوں صورتوں میں گنتی پوری ہونے کے بعد   عموماً دوبارہ نئے سرے سے شروع کی جاتی ہے۔ شکل \حوالہء{  8.1 }-الف  میں \اصطلاح{ چار بِٹ  ثنائی سیدھا گنت کار }\فرہنگ{گنت کار!چار بٹ  ثنائی سیدھا}\حاشیہب{four bit binary up counter}\فرہنگ{counter!four bit binary, up} اور شکل- ب  میں \اصطلاح{ چار بِٹ  ثنائی الٹ گنت کار  }\فرہنگ{گنت کار!چار بٹ ثنائی، الٹ}\حاشیہب{four bit binary down counter}\فرہنگ{counter!four bit binary, down}  پیش ہیں۔ان کی بناوٹ ملتی جلتی ہے۔


\اصطلاح{ثنائی گنت کار }\فرہنگ{گنت کار!ثنائی}\حاشیہب{binary counter}\فرہنگ{counter!binary} آپ پہلے بھی دیکھ چکے ہیں۔\اصطلاح{سیدھے گنت کار }میں   
\عددی{\overline{\text{\RL{زبردستی بلند}}}}   کو بلند  (1) یعنی غیر فعال رکھا جاتا ہے۔گنتی شروع کرنے سے قبل   \عددی{\overline{\text{\RL{زبردستی پست}}}}  کو لمحاتی  پست  (0) کر کے گنتی   (کی ابتدائی قیمت)  \عددی{0000_2} کی جاتی  ہے۔ گنتی کے دوران کسی بھی وقت  \عددی{\overline{\text{\RL{زبردستی پست}}}}  اشارہ  پست کر کے گنتی دوبارہ صفر سے شروع کی جا سکتی ہے۔

\اصطلاح{الٹ گنت کار } میں  \عددی{\overline{\text{\RL{زبردستی پست}}}} کو غیر فعال رکھا جاتا ہے جبکہ \عددی{\overline{\text{\RL{زبردستی بلند}}}}  اشارے کو گنتی شروع کرنے سے  قبل لمحاتی فعال کر کے گنتی   \عددی{1111_2} سے شروع  کی  جاتی ہے۔ گنتی کے دوران کسی بھی وقت  اس اشارے کو پست کر کے گنتی  دوبارہ  \عددی{1111_2} سے شروع کی  جا سکتی ہے۔

سیدھے گنت کار کو مثال  بناتے  ہوئے  ایک اہم صورت حال پر غور کرتے ہیں۔شکل میں بایاں ترین پلٹ، ساعت کے ( ہر ) کنارہ چڑھائی پر حال تبدیل کرتا ہے۔ساعت کے کنارہ چڑھائی کے کچھ دیر بعد 
\عددی{\overline{Q}_3} حال تبدیل کرے گا۔اس دورانیہ کو پلٹ کا \اصطلاح{  دورانیہ ردعمل }\فرہنگ{دورانیہ!رد عمل}\حاشیہب{propagation time}\فرہنگ{propagation time}کہتے ہیں۔یوں  اگلے پلٹ  کو، جسے \عددی{\overline{Q}_3}  بطور ساعت فراہم  کیا گیا ہے ،حال تبدیل کرنے کا خبر اصل ساعت (کے کنارہ چڑھائی)   سے کچھ دیر بعد  پہنچتا ہے۔اس پلٹ کو بھی مخارج  (\عددی{\overline{Q}_2})   تبدیل کرنے کے لئے   پلٹ کے  دورانیہ رد عمل جتنا وقت درکار ہو گا۔اسی طرح اس سے اگلے پلٹ کو ، جسے \عددی{\overline{Q}_2}  بطور  ساعت فراہم کیا گیا ہے ، حال تبدیل کرنے کا اشارہ، اصل ساعت (کے کنارہ چڑھائی)  سے دورانیہ رد عمل کے دگنے وقت کے برابر تاخیر سے ملے گا۔

آپ دیکھ سکتے ہیں اس دور میں تمام پلٹوں کے مخارج بیک وقت تبدیل نہیں ہوں گے  بلکہ مخارج کی تبدیلی بائیں پلٹ سے شروع ہوتی ہے اور بدستور دائیں جانب بڑھتی ہے۔مخارج کی تبدیلی اس دور میں لہر کی طرح گزرتی ہے۔یوں اس طرح ادوار کو \اصطلاح{ لہر نما گنت کار }\فرہنگ{گنت کار!لہر نما}\حاشیہب{ripple counter}\فرہنگ{counter!ripple} کہتے ہیں۔یوں  موجودہ دور\اصطلاح{ لہر نما ثنائی گنت کار }\فرہنگ{گنت کار!لہر نما، ثنائی}\حاشیہب{binary ripple counter}\فرہنگ{counter!binary, ripple} کہلاتا   ہے۔

عین ممکن ہے کہ آخری پلٹ تک سعت کی خبر پہنچنے سے قبل  سعت کا نیا اشارہ پہلی پلٹ کو ملے۔  یوں  آخری پلٹ گزشتہ ساعت گننے  کے مطابق جبکہ پہلی پلٹ نئی سعت   گننے کے مطابق ہو گا اور گنتی غلط ہو گی۔متعدد پلٹ پر مبنی لہر نما گنت کار میں اس مسئلہ کی توقع    رکھیں۔ 

 معاصر گنت کار اس مسئلہ سے پاک ہیں۔آئیں ان پر غور کرتے ہیں۔
 
\حصہ{معاصر گنت کار}
\اصطلاح{معاصر گنت کار } میں تمام پلٹ کو ایک ہی ساعت مہیا کی جاتی ہے لہٰذا  تمام پلٹ  بیکوقت نیا حال  اختیار کرتے ہیں۔ان   ادوار میں ہر پلٹ کے مداخل پر ترکیبی دور نسب کر  کے ، اسے اگلی  ساعت کے کنارے پر ، بلند یا پست ہونے کا   اشارہ مہیا کیا جاتا ہے۔پلٹ  اگلی ساعت کے کنارے پر  اس اشارے کے مطابق حال اختیار کرتا  ہے۔یہ فیصلہ کہ اگلی ساعت پر پلٹ بلند یا  پست حال اختیار کرے گا ، دور کے موجودہ حال کو دیکھ کر کیا جاتا ہے۔اس طریقہ کار کو چند مثالوں سے سمجھتے ہیں۔

\جزوحصہ{معاصر ثنائی گنت کار}
\اصطلاح{تین بِٹ معاصر ثنائی گنت کار }\فرہنگ{گنت کار!تین بِٹ، معاصر}\حاشیہب{three bit synchronous counter}\فرہنگ{counter!synchronous, three bit} شکل \حوالہء{ 8.3 } میں پیش ہے۔  مخارج  \عددی{Q_0} کمتر  رتبی بِٹ   جبکہ \عددی{Q_2} بلند تر رتبی بِٹ ہے۔اس دور کی بناوٹ سیکھتے  ہیں۔



شکل  \حوالہء{8.2 } میں \موٹا{  موجودہ حال } کی قطار میں تین بِٹ ثنائی گنتی لکھی گئی ہے جو کسی بھی لمحے پلٹ کا موجودہ  حال  پیش کرتی ہے۔  جدول کی پہلی صف پر غور کریں جہاں موجودہ گنتی یا موجودہ حال  \عددی{000_2} ہے۔ہم چاہتے ہیں کہ اگلا عدد \عددی{001_2} ہو ، لہٰذا \موٹا{ اگلے حال }کی پہلی صف  میں ہم  \عددی{001_2} لکھتے ہیں۔آخری صف میں موجودہ حال  \عددی{111_2} ہے۔تین بِٹ  استعمال کرتے ہوئے یہیں تک  گنتی ممکن ہے۔اس آخری گنتی تک پہنچ کر ہم دوبارہ  \عددی{000_2}  سے  گنتی شروع کرتے ہی، لہٰذا  آخری صف میں اگلا حال \عددی{000_2}  ہو گا۔


 پہلی صف کے کمتر رتبی بِٹ  \عددی{Q_0} پر   غور کرتے ہیں۔اس بِٹ کی موجودہ قیمت کو موجودہ \عددی{Q_0} ظاہر کرتا ہے جو  \عددی{0} ہے جبکہ اس کی اگلی قیمت  اگلا \عددی{Q_0} ظاہر کرتا ہے جو  \عددی{1} ہے۔ٹی پلٹ استعمال کرتے  ہوئے ساعت کے کنارہ چڑھائی پر پلٹ کا حال  \عددی{0} سے \عددی{1} کرنے کی خاطر پلٹ کے مخارج  \عددی{T_0} کو بلند کرنا ہو گا۔یہ معلومات نیچے  پیش ٹی پلٹ کے جدول سے حاصل  کی گئی۔ یوں اسی صف  میں  \عددی{T_0} کی قیمت  \عددی{1} لکھی گئی ہے۔شکل میں نکتہ دار لکیروں سے  اس کی وضاحت کی گئی ہے۔



اسی  (پہلی) صف میں اگلے بِٹ  \عددی{Q_1} پر غور کرتے ہیں۔اس بِٹ کی موجودہ قیمت \عددی{0} اور اگلی قیمت بھی \عددی{0} ہے، لہٰذا ساعت کے اگلے کنارہ پر  ہم نہیں چاہتے کہ یہ پلٹ اپنا حال تبدیل کرے۔یوں اس پلٹ کے مداخل  \عددی{T_1} کو پست رکھنا ہو گا۔اس طرح  \عددی{T_1} کے خانے  میں  \عددی{0} لکھا جائے گا۔اسی طرز پر تمام صفوں کے تمام مداخل کے لئے جدول کے   خانے پُر کیے گئے ہیں۔


دور بنانے کی خاطر شکل \حوالہء{8.2 } میں  \اصطلاح{داخلی مساوات }\فرہنگ{داخلی مساوات} کی قطار استعمال  میں  لائی  جاتی ہے۔مجموعہ ارکان ضرب کی ترکیب سے درج ذیل   لکھا جا سکتا ہے۔
\begin{gather}
\begin{aligned}
T_0&=1\\
T_1&=\overline{Q}_2\overline{Q_1} Q_0+\overline{Q}_2 Q_1 Q_0+Q_2\overline{Q_1} Q_0+Q_2Q_1Q_0\\
T_2&=\overline{Q}_2 Q_1 Q_0+Q_2Q_1Q_0
\end{aligned}
\end{gather}

%???KKK
	یہ مساوات موجودہ حالوں کی قیمتیں مدِ نظر رکھ کر لکھی گئی ہیں۔شکل 8.2 میں موجود مواد سے شکل 8.4 میں کارناف نقشوں کی مدد سے سادہ مساواتیں حاصل کی گئی ہیں جنہیں مساوات 8.2 میں دوبارہ دکھایا گیا ہے۔



 
(8.2)

	شکل 8.3 میں تین پلٹ لگا کر ان کو مساوات 8.2 سے حاصل برقی اشارات بطور مداخل دئے گئے ہیں۔اس طرح تین بِٹ معاصر ثنائی گنت کار  6حاصل کیا گیا ہے۔
	مساوات 8.2 بغیر حل کئے بھی شکل 8.2 میں دئے جدول سے حاصل کئے جا سکتے ہیں۔اس جدول پر غور کرنے سے دیکھا جاتا ہے کہہر ساعت کے کنارے تبدیل ہوتا ہے۔ پرمہیا کرنے سے ایسا کیا جا سکتا ہے۔کو دیکھتے یہ بات سامنے آتی ہے کہ جب بھیکی قیمت ہو اس سے اگلے ساعت کے کنارہکی قیمت تبدیل ہوتی ہے۔یوں کو فراہم کرنے سے ایسا حاصل کیا جا سکتا ہے۔ پر غور کرنے سے دیکھا جاتا ہے کہ جب بھی اور دونوں کی قیمتیںہوں، اس سے اگلی ساعت کے کنارہکی قیمت تبدیل ہوتی ہے۔یوںکو فراہم کیا جاتا ہے۔اگر زیادہ بِٹ پر مبنی ثنائی گنتی پر غور کیا جائے تو دیکھا جاتا ہے کہ، کوئی بھی مخارج، ساعت کے اگلے کنارے، اس وقت حال تبدیل کرتا ہے جب اس سے کمتر تمام مخارج کی قیمتہو جائے۔یوں چار بِٹ ثنائی گنت کار کے لئے ہم لکھ سکتے ہیں۔

 
(8.3)


8.2.2 ثنائی علامتی روپ کا معاصر اعشاری گنت کار
	پچھلے حصہ میں تین بِٹ ثنائی گنت کار پر غور ہوا جوسےتک گنتی کرنے کی صلاحیت رکھتا ہے۔اسی طرح چار بِٹ پر مبنی دورسےتک ثنائی گنتی کر سکتا ہے۔اگر ایسے دور کو سےتک گنتی کرنے پر پابند کیا جائے تو اس سے ثنائی علامتی روپ کا اعشاری کنت کار 7 حاصل ہو گا۔اس حصہ میں ایسا ہی کرتے ہیں۔شکل 8.5 میں اس دور کے حالوں کا جدول دیا گیا ہے۔جدول میں مخارج 8 کے قطار کا اضافہ کیا گیا ہے۔مخارج صفر سے نو تک گنتی پوری ہونے پر ساعت کے ایک دوری عرصہ9 کے لئے بلند ہوتا ہے۔ہم آگے دیکھیں گے کہ  کو استعمال کرتے زیادہ اعشاری ہندسوں پر مبنی گنتی کے دور بنائے جاتے ہیں۔




	اس شکل میںسےتک کے ترتیب استعمال نہیں ہوتے۔کارناف نقشوں کی مدد سے پلٹوں کے مداخلتااور مخارج کے مساواتوں کی سادہ شکل حاصل کرتے وقت انہیں غیر ضروری حالیں 10 تصور کیا جاتا ہے۔شکل 8.6 میں سادہ مساوات حاصل کرنا دکھایا گیا ہے۔


	ایسا کرتے داخلی مساوات کی سادہ اشکال یوں حاصل ہوتے ہیں۔

 
(8.4)

ان مساوات کی مدد سے حاصل دور شکل 8.7 میں دکھایا گیا ہے جہاں دور میں گنتی شروع اور بند کرنے کی اضافی صلاحیت بھی پیدا کی گئی ہے۔یہ صلاحیت تمام پلٹوں کے مداخل پر اضافی ضرب گیٹ نصب کرنے سے حاصل کی گئی ہے۔



	ان اضافی ضرب گیٹوں کو برقی اشارہ گِن مہیا کیا گیا ہے۔یہ اشارہ بلند ہونے کی صورت میں دور گنتی کرتا ہے اور اشارہ پست ہونے کی صورت میں دور گنتی کرنا بند کر دیتا ہے۔
	شکل 8.8 میں تین درجہ دور بنایا گیا ہے جوسےتک گنتی کرتا ہے۔اسے بنانے کی خاطر تین عدد ثنائی علامتی روپ کا اعشاری  گنت کار استعمال کئے گئے ہیں۔اسی طرح مزید درجات جوڑ کر درکار ہندسوں کا  گنت کار بنایا جاتا ہے۔ 



	اس دور کی کارکردگی کچھ یوں ہے۔گنتی شروع کرنے سے قبل کو ایک لمحہ کے لئے پست کر کے گنتیکر دی جاتی ہے۔ساعت کے کنارہ چڑھائی پر اکائی عدد کی گنتی بڑھتی ہے۔اکائی درجہ کا مخارج پست رہنے کی وجہ سے دہائی اور سینکڑا کی گنتی رکھی رہتی ہے۔گنتیتک پہنچتے ہی اکائی درجہ کا مخارج بلند ہو جاتا ہے۔یوں اگلے ساعت کے کنارہ پر اکائی درجہ کی گنتیسےہو جاتی ہے جبکہ دہائی درجہ کی گنتیسےہو جاتی ہے اور اسی وقت اکائی کا مخارج ایک مرتبہ پھر پست ہو جاتا ہے۔یوں اس سے اگلے ساعت کے کنارہ صرف اکائی درجہ کی گنتی چالو رہتی ہے جبکہ دہائی اور سینکڑا درجہ کی گنتی بند رہتی ہے۔اسی طرحتک گنتی کے بعد اکائی درجہ اور دہائی درجہ دونوں کے مخارج بلند ہوتے ہیں جس کی وجہ سے اگلے ساعت کے کنارہ پر سینکڑا درجہ کی گنتی سے بڑھ کرہو جاتی ہے جبکہ اکائی اور دہائی درجے دونوںسےہو جاتے ہیں اور ساتھ ہی ساتھ ان کے مخارج  دوبارہ پست ہو جاتے ہین۔

مشق:	انٹرنیٹ سےاورکے معلوماتی صفحات حاصل کریں۔انہیں استعمال کرتے ہوئے زیادہ بِٹ کے گنت کار حاصل کریں۔

8.3 دیگر گنت کار
8.3.1 متغیر گنت کار
	چار بِٹ ثنائی گنت کارسےتک گنتی کرتا ہے۔اس میں متوازی دخول کی صلاحیت استعمال کرتے اسے دو اعداد کے مابین گنتی کرنے پر مجبور کیا جا سکتا ہے۔ایسے گنت کار کو ہم متغیر لمبائی گنت کار11 کہیں گے۔جس عدد سے گنتی شروع کرنی ہو اس عدد کو متوازی فراہم کیا جاتا ہے۔جس عدد تک گنتی درکار ہو، اس عدد تک گنتی پہنچنے پر دور کو مجبور کیا جاتا ہے کہ وہ دوبارہ متوازی فراہم کردہ عدد داخل کر کے گنتی از سرے نو شروع کرے۔
	چار بِٹ معاصر ثنائی گنت کار کو مثال بناتے اسےسےتک گنتی کرنے والا دور بناتے ہیں۔شکل 8.9 میں ایسا دور دکھایا گیا ہے۔شکل میں نکتہ دار مستطیل میں مساوات 8.2 سے حاصل درکار مداخل کا دور دکھایا گیا ہے۔دور میں ہر پلٹ کی داخلی طرف دو ضرب گیٹ اور ایک جمع گیٹ نصب کر کے اس میں متوازی دخول کی صلاحیت پیدا کی گئی ہے۔

	اس دور میں گنتی کے شروع کا عدد متوازی داخل کیا جاتا ہے۔اس عدد کوسے ظاہر کیا گیا ہے اور اس کی قیمت ہے۔گنتی کا آخری عددہے۔اس عدد کو نکتہ دار دائرے میں بند ترکیبی دور پہچان کر اپنی مخارج پست کرتا ہے اور یوں ساعت کے اگلے کنارے،  متوازی طور دور میں داخل ہو جاتا ہے۔اس طرح یہ  گنت کاراورکے مابین گنتی کرتا ہے۔
	دور میں پہلی مرتبہداخل کرنے کا طریقہ نہیں دکھایا گیا۔

8.3.2 چھلا نما گنت کار
	بِٹ چھلا نما گنت کار12مخارج میں ایک ہی بلند بِٹ گھماتا ہے۔اس کے باقی تمام بِٹ پست رہتے ہیں۔ایک ہی بلند بِٹ کو ساعت کے کنارے ایک پلٹ سے دوسرے پلٹ منتقل کیا جاتا ہے۔شکل 8.10 میں ایک ایسا چار بِٹ دور دکھایا گیا ہے۔


8.3.3 دورانیہ پید اکار
	بعض اوقات ہمیں مقررہ دورانیہ کے لئے بلند یا پست اشارہ درکار ہوتا ہے۔تین بِٹ کا معاصر ثنائی الٹ گنت کار استعمال کرتے ہوئے ایک ایسے ہی دور کو تشکیل دیتے ہیں۔اس دور کو ہم دورانیہ پیدا کار13 کہیں گے۔ 
	تین بِٹ کا الٹ گنت کارتاکی گنتی دہراتا رہتا ہے۔شکل 8.11 میں متوازی لکھے جانے کی صلاحیت رکھنے والے تین بِٹ کے الٹ گنت کار کو استعمال کیا گیا ہے جو اس وقت گنتی کرتا ہے جب اس کا مداخل گن بلند ہو۔اسے تین بِٹ بطور درکار دورانیہ کے فراہم کئے جاتے ہیں۔متوازی لکھ کا مداخل لمحاتی طور بلند کرنے سے یہ تین بِٹ گنت کار میں لکھ لئے جاتے ہیں۔جب تک گنت کار کے تینوں خارجی بِٹ پست نہ ہوں جمع گیٹ بلند رہتا ہے اور یوں گنت کار الٹ گنتی جاری رکھتا ہے۔جیسے ہی گنت کارپہنچتا ہے جمع گیٹ کا مخارج پست ہو جاتا ہے اور یوں گنت کار گنتی روکھ دیتا ہے۔یوں تین بِٹ درکار دورانیہ کے برابر دورانیہ کے لئے جمع گیٹ کا مخارج یعنی دورانیہ بلند رہتا ہے۔


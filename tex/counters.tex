\باب{ گنت کار}
ثنائی گنت کار آپ دیکھ چکے ہیں۔گنت کار کا بنیادی مقصد  داخلی \اصطلاح{ برقی اشارے  }\فرہنگ{اشارہ!برقی}\حاشیہب{electrical signal}\فرہنگ{signal!electrical}کی گنتی  کرنا ہے۔برقی اشارہ اسے بطور ساعت یا سادہ مداخل کے طور پر  مہیا کیا جا تا ہے۔

وہ  دفتر جس کے خارجی برقی اشارات  ثنائی گنتی کے تحت ترتیب وار حال تبدیل کرتے ہوں \اصطلاح{ ثنائی گنت کار } کہلاتا ہے۔ وہ  دفتر  جس کے  خارجی اشارات اعشاری گنتی کے تحت ترتیب  وار حال تبدیل کرتے ہوں \اصطلاح{ اعشاری گنت کار }  کہلاتا ہے۔

ان کے علاوہ،   کوئی بھی دور جو کسی متعین ترتیب کے تحت  متواتر  حال تبدیل کرتا ہو  گنت کار کہلائے گا۔

گنت کار ادوار پر اس باب میں غور کیا جائے گا۔

\حصہ{ثنائی گنت کار}
چار بِٹ ثنائی   سیدھی  گنتی  \عددی{0000_2}  تا \عددی{1111_2}   ممکن ہے۔اسی طرح الٹی گنتی  \عددی{1111_2} سے  شروع ہو کر    \عددی{0000_2}  پر ختم ہو گی۔دونوں صورتوں میں گنتی پوری ہونے کے بعد   عموماً دوبارہ نئے سرے سے شروع کی جاتی ہے۔ شکل \حوالہء{  8.1 }-الف  میں \اصطلاح{ چار بِٹ  ثنائی سیدھا گنت کار }\فرہنگ{گنت کار!چار بٹ  ثنائی سیدھا}\حاشیہب{four bit binary up counter}\فرہنگ{counter!four bit binary, up} اور شکل- ب  میں \اصطلاح{ چار بِٹ  ثنائی الٹ گنت کار  }\فرہنگ{گنت کار!چار بٹ ثنائی، الٹ}\حاشیہب{four bit binary down counter}\فرہنگ{counter!four bit binary, down}  پیش ہیں۔ان کی بناوٹ ملتی جلتی ہے۔

%???KKK

	ثنائی گنت کار آپ پہلے بھی دیکھ چکے ہیں۔سیدھے گنت کار میں  کو بلند یعنی غیر فعال رکھا جاتا ہے۔ایسا کرنے کی خاطر اس پرمہیا کیا جاتا ہے۔گنتی شروع کرنے سے قبلکو ایک لمحہ کے لئے پست کر کے گنتیکر دی جاتی ہے۔اس کو گنتی کے دوران کسی بھی وقت پست کر کے گنتی دوبارہ صفر سے شروع کرائی جا سکتی ہے۔
	الٹ گنت کار میںکو غیر فعال رکھا جاتا ہے جبکہکو گنتی شروع کرنے سے پہلے لمحاتی طور فعال کر کے گنتیسے شروع کرائی جاتی ہے۔اس کو گنتی کے دوران کسے بھی وقت پست کر کے گنتی دوبارہسے شروع کرائی جا سکتی ہے۔
	سیدھے گنت کار کو مثال لیتے ایک اہم صورتِ حال پر غور کرتے ہیں۔شکل میں سب سے بائیں جانب پلٹ، ساعت کے ہر کنارہِ چڑھائی پر حالت تبدیل کرتا ہے۔ساعت کے کنارہِ چڑھائی کے کچھ دیر بعدحالت تبدیل کرتا ہے۔اس دورانیہ کو پلٹ کا دورانیہ ردِ عمل کہتے ہیں۔یوں اس سے اگلا پلٹ جس کوبطورِ ساعت فراہم ہوتا ہے کو حالت تبدیل کرنے کی خبر اصل ساعت سے کچھ دیر بعد ملتی ہے۔اس پلٹ کو بھی مخارج تبدیل کرنے کے لئے پلٹ کا دورانیہ ردِ عمل جتنا وقت درکار ہو گا۔یوں اس سے اگلا پلٹ جسے بطورِ ساعت فراہم کیا گیا ہے کو حالت تبدیل کرنے کا اشارہ، اصل ساعت سے دورانیہ ردِ عمل کے دگنے وقت کے برابر تاخیر سے ملے گا۔
	آپ دیکھ سکتے ہیں کہ اس دور میں تمام پلٹوں کے مخارج بیک وقت تبدیل نہیں ہوتے بلکہ مخارج کی تبدیلی بائیں پلٹ سے شروع ہوتی ہے اور بدستور دائیں جانب بڑھتی ہے۔مخارج کی تبدیلی اس دور میں لہر کی طرح گزرتی ہے۔یوں اس طرح ادوار کو لہر نما گنت کار 4 کہتے ہیں۔اس طرح موجودہ دور کو لہر نما ثنائی گنت کار 5 کہیں گے۔
	تیز رفتار یا زیادہ پلٹوں پر مبنی لہر نما گنت کار کو یہ مسئلہ درپیش ہو سکتا ہے کہ ساعت کا دوسرا کنارہ پہنچنے کے با وجود تمام پلٹوں کی مخارج پہلی ساعت کے مطابق حالتیں اختیار نہ کر سکے ہوں اور یوں ان گنت کار کی گنتی ایسی صورت میں غلط ہو گی۔
	 معاصر گنت کار اس مسئلہ سے پاک ہیں۔آئیں ان پر غور کریں۔
8.2 معاصر گنت کار
	معاصر گنت کار میں تمام پلٹوں کو ایک ہی ساعت مہیا کی جاتی ہے۔یوں تمام پلٹ نئی حالتیں بیک وقت اختیار کرتے ہیں۔اس طرح ادوار میں ہر پلٹ کے مداخل پر ترکیبی دور لگا کر اسے اگلے ساعت کے کنارہ پر بلند یا پست ہونے کا برقی اشارہ مہیا کیا جاتا ہے۔پلٹ اگلے ساعت کے کنارہ پر یہی حالت اختیار کر لیتا ہے۔یہ فیصلہ کرنا کہ اگلے ساعت پر پلٹ بلند کہ پست حالت اختیار کرے گا دور کے موجودہ حالت کو دیکھ کر کیا جاتا ہے۔اس طریقہ کار کو چند مثالوں سے سمجھتے ہیں۔
8.2.1 معاصر ثنائی گنت کار
	تین بِٹ معاصر ثنائی گنت کار کو شکل 8.3 میں دکھایا گیا ہے۔پلٹ نمبر صفر کی مخارج کمتر  رتبہ والا بِٹ  ہے جبکہ پلٹ نمبر دو کی مخارجبلند تر رتبہ والا بِٹ ہے۔اس دور کی بناوٹ کا طریقہ دیکھتے ہیں۔



	شکل 8.2 میں بائیں جانب موجودہ حالتوں کے نام کے نیچے تین بِٹ ثنائی گنتی دی گئی ہے۔یہ ساعت کے ساتھ تبدیل ہوتے پلٹوں کی مطلوبہ حالتیں ہیں۔  جدول میں پہلی صف پر غور کریں۔موجودہ گنتی یا موجودہ حالہے۔ہم چاہتے ہیں کہ اگلا عددہو لہٰذا اگلی حالتوں کے خانے میں ہملکھتے ہیں۔آخری صف میں موجودہ حالہے۔تین بِٹ میں یہیں تک گِنتی ممکن ہے۔گنتی کے آخر میں پہنچ کر ہم دوبارہ شروع سے گنتی شروع کرتے ہیں۔لہٰذا اگلا حال ہو گا۔
	اب کمتر رتبہ والے بِٹ پر غور کرتے ہیں۔اس بِٹ کی موجودہ قیمت کو موجودہظاہر کرتا ہے جو کہہے جبکہ اس کے اگلے قیمت کو اگلاظاہر کرتا ہے جو کہہے۔ٹی پلٹ استعمال کرتے ساعت کے کنارہ چڑھائی پر پلٹ کا حالسےکرنے کی خاطر پلٹ کی مخارج کو بلند کرنا ہو گا۔یہ معلومات نیچے دئے ٹی پلٹ کی خصوصیات کی جدول سے حاصل ہوتی ہے۔ یوں اسی صف میںکی قیمتلکھی گئی ہے۔یہی کچھ شکل میں نکتہ دار لکیروں سے واضح کیا گیا ہے۔


جدول 8.1: ٹی پلٹ کی خصوصیات کا جدول

	اسی صف میں اگلے بِٹ یعنیپر غور کرتے ہیں۔اس بِٹ کی موجودہ قیمتاور اگلی قیمت بھیہے۔یوں ساعت کے اگلے کنارہ ہم نہیں چاہتے کہ یہ پلٹ اپنی حالت تبدیل کرے۔یوں اس پلٹ کی مداخلکو پست رکھنا ہو گا۔اس طرح کے خانے میںلکھ لیتے ہیں۔اسی طرزِ پر تمام صفوں کے تمام مداخل کے لئے جدول کے بقایا خانے پُر کئے گئے ہیں۔


	دور بنانے کی خاطر شکل 8.2 میں داخلی مساوات کی قطار زیرِ استعمال لائی جاتی ہے۔مجموعہ ارکانِ ضرب کی ترکیب سے حاصل ہوتا ہے۔

 
(8.1)

	یہ مساوات موجودہ حالتوں کی قیمتیں مدِ نظر رکھ کر لکھی گئی ہیں۔شکل 8.2 میں موجود مواد سے شکل 8.4 میں کارناف نقشوں کی مدد سے سادہ مساواتیں حاصل کی گئی ہیں جنہیں مساوات 8.2 میں دوبارہ دکھایا گیا ہے۔



 
(8.2)

	شکل 8.3 میں تین پلٹ لگا کر ان کو مساوات 8.2 سے حاصل برقی اشارات بطور مداخل دئے گئے ہیں۔اس طرح تین بِٹ معاصر ثنائی گنت کار  6حاصل کیا گیا ہے۔
	مساوات 8.2 بغیر حل کئے بھی شکل 8.2 میں دئے جدول سے حاصل کئے جا سکتے ہیں۔اس جدول پر غور کرنے سے دیکھا جاتا ہے کہہر ساعت کے کنارے تبدیل ہوتا ہے۔ پرمہیا کرنے سے ایسا کیا جا سکتا ہے۔کو دیکھتے یہ بات سامنے آتی ہے کہ جب بھیکی قیمت ہو اس سے اگلے ساعت کے کنارہکی قیمت تبدیل ہوتی ہے۔یوں کو فراہم کرنے سے ایسا حاصل کیا جا سکتا ہے۔ پر غور کرنے سے دیکھا جاتا ہے کہ جب بھی اور دونوں کی قیمتیںہوں، اس سے اگلی ساعت کے کنارہکی قیمت تبدیل ہوتی ہے۔یوںکو فراہم کیا جاتا ہے۔اگر زیادہ بِٹ پر مبنی ثنائی گنتی پر غور کیا جائے تو دیکھا جاتا ہے کہ، کوئی بھی مخارج، ساعت کے اگلے کنارے، اس وقت حالت تبدیل کرتا ہے جب اس سے کمتر تمام مخارج کی قیمتہو جائے۔یوں چار بِٹ ثنائی گنت کار کے لئے ہم لکھ سکتے ہیں۔

 
(8.3)


8.2.2 ثنائی علامتی روپ کا معاصر اعشاری گنت کار
	پچھلے حصہ میں تین بِٹ ثنائی گنت کار پر غور ہوا جوسےتک گنتی کرنے کی صلاحیت رکھتا ہے۔اسی طرح چار بِٹ پر مبنی دورسےتک ثنائی گنتی کر سکتا ہے۔اگر ایسے دور کو سےتک گنتی کرنے پر پابند کیا جائے تو اس سے ثنائی علامتی روپ کا اعشاری کنت کار 7 حاصل ہو گا۔اس حصہ میں ایسا ہی کرتے ہیں۔شکل 8.5 میں اس دور کے حالتوں کا جدول دیا گیا ہے۔جدول میں مخارج 8 کے قطار کا اضافہ کیا گیا ہے۔مخارج صفر سے نو تک گنتی پوری ہونے پر ساعت کے ایک دوری عرصہ9 کے لئے بلند ہوتا ہے۔ہم آگے دیکھیں گے کہ  کو استعمال کرتے زیادہ اعشاری ہندسوں پر مبنی گنتی کے دور بنائے جاتے ہیں۔




	اس شکل میںسےتک کے ترتیب استعمال نہیں ہوتے۔کارناف نقشوں کی مدد سے پلٹوں کے مداخلتااور مخارج کے مساواتوں کی سادہ شکل حاصل کرتے وقت انہیں غیر ضروری حالتیں 10 تصور کیا جاتا ہے۔شکل 8.6 میں سادہ مساوات حاصل کرنا دکھایا گیا ہے۔


	ایسا کرتے داخلی مساوات کی سادہ اشکال یوں حاصل ہوتے ہیں۔

 
(8.4)

ان مساوات کی مدد سے حاصل دور شکل 8.7 میں دکھایا گیا ہے جہاں دور میں گنتی شروع اور بند کرنے کی اضافی صلاحیت بھی پیدا کی گئی ہے۔یہ صلاحیت تمام پلٹوں کے مداخل پر اضافی ضرب گیٹ نصب کرنے سے حاصل کی گئی ہے۔



	ان اضافی ضرب گیٹوں کو برقی اشارہ گِن مہیا کیا گیا ہے۔یہ اشارہ بلند ہونے کی صورت میں دور گنتی کرتا ہے اور اشارہ پست ہونے کی صورت میں دور گنتی کرنا بند کر دیتا ہے۔
	شکل 8.8 میں تین درجہ دور بنایا گیا ہے جوسےتک گنتی کرتا ہے۔اسے بنانے کی خاطر تین عدد ثنائی علامتی روپ کا اعشاری  گنت کار استعمال کئے گئے ہیں۔اسی طرح مزید درجات جوڑ کر درکار ہندسوں کا  گنت کار بنایا جاتا ہے۔ 



	اس دور کی کارکردگی کچھ یوں ہے۔گنتی شروع کرنے سے قبل کو ایک لمحہ کے لئے پست کر کے گنتیکر دی جاتی ہے۔ساعت کے کنارہ چڑھائی پر اکائی عدد کی گنتی بڑھتی ہے۔اکائی درجہ کا مخارج پست رہنے کی وجہ سے دہائی اور سینکڑا کی گنتی رکھی رہتی ہے۔گنتیتک پہنچتے ہی اکائی درجہ کا مخارج بلند ہو جاتا ہے۔یوں اگلے ساعت کے کنارہ پر اکائی درجہ کی گنتیسےہو جاتی ہے جبکہ دہائی درجہ کی گنتیسےہو جاتی ہے اور اسی وقت اکائی کا مخارج ایک مرتبہ پھر پست ہو جاتا ہے۔یوں اس سے اگلے ساعت کے کنارہ صرف اکائی درجہ کی گنتی چالو رہتی ہے جبکہ دہائی اور سینکڑا درجہ کی گنتی بند رہتی ہے۔اسی طرحتک گنتی کے بعد اکائی درجہ اور دہائی درجہ دونوں کے مخارج بلند ہوتے ہیں جس کی وجہ سے اگلے ساعت کے کنارہ پر سینکڑا درجہ کی گنتی سے بڑھ کرہو جاتی ہے جبکہ اکائی اور دہائی درجے دونوںسےہو جاتے ہیں اور ساتھ ہی ساتھ ان کے مخارج  دوبارہ پست ہو جاتے ہین۔

مشق:	انٹرنیٹ سےاورکے معلوماتی صفحات حاصل کریں۔انہیں استعمال کرتے ہوئے زیادہ بِٹ کے گنت کار حاصل کریں۔

8.3 دیگر گنت کار
8.3.1 متغیر گنت کار
	چار بِٹ ثنائی گنت کارسےتک گنتی کرتا ہے۔اس میں متوازی دخول کی صلاحیت استعمال کرتے اسے دو اعداد کے مابین گنتی کرنے پر مجبور کیا جا سکتا ہے۔ایسے گنت کار کو ہم متغیر لمبائی گنت کار11 کہیں گے۔جس عدد سے گنتی شروع کرنی ہو اس عدد کو متوازی فراہم کیا جاتا ہے۔جس عدد تک گنتی درکار ہو، اس عدد تک گنتی پہنچنے پر دور کو مجبور کیا جاتا ہے کہ وہ دوبارہ متوازی فراہم کردہ عدد داخل کر کے گنتی از سرے نو شروع کرے۔
	چار بِٹ معاصر ثنائی گنت کار کو مثال بناتے اسےسےتک گنتی کرنے والا دور بناتے ہیں۔شکل 8.9 میں ایسا دور دکھایا گیا ہے۔شکل میں نکتہ دار مستطیل میں مساوات 8.2 سے حاصل درکار مداخل کا دور دکھایا گیا ہے۔دور میں ہر پلٹ کی داخلی طرف دو ضرب گیٹ اور ایک جمع گیٹ نصب کر کے اس میں متوازی دخول کی صلاحیت پیدا کی گئی ہے۔

	اس دور میں گنتی کے شروع کا عدد متوازی داخل کیا جاتا ہے۔اس عدد کوسے ظاہر کیا گیا ہے اور اس کی قیمت ہے۔گنتی کا آخری عددہے۔اس عدد کو نکتہ دار دائرے میں بند ترکیبی دور پہچان کر اپنی مخارج پست کرتا ہے اور یوں ساعت کے اگلے کنارے،  متوازی طور دور میں داخل ہو جاتا ہے۔اس طرح یہ  گنت کاراورکے مابین گنتی کرتا ہے۔
	دور میں پہلی مرتبہداخل کرنے کا طریقہ نہیں دکھایا گیا۔

8.3.2 چھلا نما گنت کار
	بِٹ چھلا نما گنت کار12مخارج میں ایک ہی بلند بِٹ گھماتا ہے۔اس کے باقی تمام بِٹ پست رہتے ہیں۔ایک ہی بلند بِٹ کو ساعت کے کنارے ایک پلٹ سے دوسرے پلٹ منتقل کیا جاتا ہے۔شکل 8.10 میں ایک ایسا چار بِٹ دور دکھایا گیا ہے۔


8.3.3 دورانیہ پید اکار
	بعض اوقات ہمیں مقررہ دورانیہ کے لئے بلند یا پست اشارہ درکار ہوتا ہے۔تین بِٹ کا معاصر ثنائی الٹ گنت کار استعمال کرتے ہوئے ایک ایسے ہی دور کو تشکیل دیتے ہیں۔اس دور کو ہم دورانیہ پیدا کار13 کہیں گے۔ 
	تین بِٹ کا الٹ گنت کارتاکی گنتی دہراتا رہتا ہے۔شکل 8.11 میں متوازی لکھے جانے کی صلاحیت رکھنے والے تین بِٹ کے الٹ گنت کار کو استعمال کیا گیا ہے جو اس وقت گنتی کرتا ہے جب اس کا مداخل گن بلند ہو۔اسے تین بِٹ بطور درکار دورانیہ کے فراہم کئے جاتے ہیں۔متوازی لکھ کا مداخل لمحاتی طور بلند کرنے سے یہ تین بِٹ گنت کار میں لکھ لئے جاتے ہیں۔جب تک گنت کار کے تینوں خارجی بِٹ پست نہ ہوں جمع گیٹ بلند رہتا ہے اور یوں گنت کار الٹ گنتی جاری رکھتا ہے۔جیسے ہی گنت کارپہنچتا ہے جمع گیٹ کا مخارج پست ہو جاتا ہے اور یوں گنت کار گنتی روکھ دیتا ہے۔یوں تین بِٹ درکار دورانیہ کے برابر دورانیہ کے لئے جمع گیٹ کا مخارج یعنی دورانیہ بلند رہتا ہے۔


\باب{ گنت کار}
ثنائی گنت کار آپ دیکھ چکے ہیں۔گنت کار کا بنیادی مقصد داخلی \اصطلاح{ برقی اشارے }\فرہنگ{اشارہ!برقی}\حاشیہب{electrical signal}\فرہنگ{signal!electrical}کی گنتی کرنا ہے۔برقی اشارہ اسے بطور ساعت یا سادہ مداخل کے طور پر مہیا کیا جا تا ہے۔

وہ دفتر جس کے خارجی برقی اشارات ثنائی گنتی کے تحت ترتیب وار حال تبدیل کرتے ہوں \اصطلاح{ ثنائی گنت کار } کہلاتا ہے۔ وہ دفتر جس کے خارجی اشارات اعشاری گنتی کے تحت ترتیب وار حال تبدیل کرتے ہوں \اصطلاح{ اعشاری گنت کار } کہلاتا ہے۔

ان کے علاوہ، کوئی بھی دور جو کسی متعین ترتیب کے تحت متواتر حال تبدیل کرتا ہو گنت کار کہلائے گا۔

گنت کار ادوار پر اس باب میں غور کیا جائے گا۔

\حصہ{ثنائی گنت کار}
چار بِٹ ثنائی سیدھی گنتی \عددی{0000_2} تا \عددی{1111_2} ممکن ہے۔اسی طرح الٹی گنتی \عددی{1111_2} سے شروع ہو کر \عددی{0000_2} پر ختم ہو گی۔دونوں صورتوں میں گنتی پوری ہونے کے بعد عموماً دوبارہ نئے سرے سے شروع کی جاتی ہے۔ شکل \حوالہء{ 8.1 }-الف میں \اصطلاح{ چار بِٹ ثنائی سیدھا گنت کار }\فرہنگ{گنت کار!چار بٹ ثنائی سیدھا}\حاشیہب{four bit binary up counter}\فرہنگ{counter!four bit binary, up} اور شکل- ب میں \اصطلاح{ چار بِٹ ثنائی الٹ گنت کار }\فرہنگ{گنت کار!چار بٹ ثنائی، الٹ}\حاشیہب{four bit binary down counter}\فرہنگ{counter!four bit binary, down} پیش ہیں۔ان کی بناوٹ ملتی جلتی ہے۔


\اصطلاح{ثنائی گنت کار }\فرہنگ{گنت کار!ثنائی}\حاشیہب{binary counter}\فرہنگ{counter!binary} آپ پہلے بھی دیکھ چکے ہیں۔\اصطلاح{سیدھے گنت کار }میں 
\عددی{\overline{\text{\RL{زبردستی بلند}}}} کو بلند (1) یعنی غیر فعال رکھا جاتا ہے۔گنتی شروع کرنے سے قبل \عددی{\overline{\text{\RL{زبردستی پست}}}} کو لمحاتی پست (0) کر کے گنتی (کی ابتدائی قیمت) \عددی{0000_2} کی جاتی ہے۔ گنتی کے دوران کسی بھی وقت \عددی{\overline{\text{\RL{زبردستی پست}}}} اشارہ پست کر کے گنتی دوبارہ صفر سے شروع کی جا سکتی ہے۔

\اصطلاح{الٹ گنت کار } میں \عددی{\overline{\text{\RL{زبردستی پست}}}} کو غیر فعال رکھا جاتا ہے جبکہ \عددی{\overline{\text{\RL{زبردستی بلند}}}} اشارے کو گنتی شروع کرنے سے قبل لمحاتی فعال کر کے گنتی \عددی{1111_2} سے شروع کی جاتی ہے۔ گنتی کے دوران کسی بھی وقت اس اشارے کو پست کر کے گنتی دوبارہ \عددی{1111_2} سے شروع کی جا سکتی ہے۔

سیدھے گنت کار کو مثال بناتے ہوئے ایک اہم صورت حال پر غور کرتے ہیں۔شکل میں بایاں ترین پلٹ، ساعت کے ( ہر ) کنارہ چڑھائی پر حال تبدیل کرتا ہے۔ساعت کے کنارہ چڑھائی کے کچھ دیر بعد 
\عددی{\overline{Q}_3} حال تبدیل کرے گا۔اس دورانیہ کو پلٹ کا \اصطلاح{ دورانیہ ردعمل }\فرہنگ{دورانیہ!رد عمل}\حاشیہب{propagation time}\فرہنگ{propagation time}کہتے ہیں۔یوں اگلے پلٹ کو، جسے \عددی{\overline{Q}_3} بطور ساعت فراہم کیا گیا ہے ،حال تبدیل کرنے کا خبر اصل ساعت (کے کنارہ چڑھائی) سے کچھ دیر بعد پہنچتا ہے۔اس پلٹ کو بھی مخارج (\عددی{\overline{Q}_2}) تبدیل کرنے کے لئے پلٹ کے دورانیہ رد عمل جتنا وقت درکار ہو گا۔اسی طرح اس سے اگلے پلٹ کو ، جسے \عددی{\overline{Q}_2} بطور ساعت فراہم کیا گیا ہے ، حال تبدیل کرنے کا اشارہ، اصل ساعت (کے کنارہ چڑھائی) سے دورانیہ رد عمل کے دگنے وقت کے برابر تاخیر سے ملے گا۔

آپ دیکھ سکتے ہیں اس دور میں تمام پلٹوں کے مخارج بیک وقت تبدیل نہیں ہوں گے بلکہ مخارج کی تبدیلی بائیں پلٹ سے شروع ہوتی ہے اور بدستور دائیں جانب بڑھتی ہے۔مخارج کی تبدیلی اس دور میں لہر کی طرح گزرتی ہے۔یوں اس طرح ادوار کو \اصطلاح{ لہر نما گنت کار }\فرہنگ{گنت کار!لہر نما}\حاشیہب{ripple counter}\فرہنگ{counter!ripple} کہتے ہیں۔یوں موجودہ دور\اصطلاح{ لہر نما ثنائی گنت کار }\فرہنگ{گنت کار!لہر نما، ثنائی}\حاشیہب{binary ripple counter}\فرہنگ{counter!binary, ripple} کہلاتا ہے۔

عین ممکن ہے کہ آخری پلٹ تک سعت کی خبر پہنچنے سے قبل سعت کا نیا اشارہ پہلی پلٹ کو ملے۔ یوں آخری پلٹ گزشتہ ساعت گننے کے مطابق جبکہ پہلی پلٹ نئی سعت گننے کے مطابق ہو گا اور گنتی غلط ہو گی۔متعدد پلٹ پر مبنی لہر نما گنت کار میں اس مسئلہ کی توقع رکھیں۔ 

 معاصر گنت کار اس مسئلہ سے پاک ہیں۔آئیں ان پر غور کرتے ہیں۔
 
\حصہ{معاصر گنت کار}
\اصطلاح{معاصر گنت کار } میں تمام پلٹ کو ایک ہی ساعت مہیا کی جاتی ہے لہٰذا تمام پلٹ بیکوقت نیا حال اختیار کرتے ہیں۔ان ادوار میں ہر پلٹ کے مداخل پر ترکیبی دور نصب کر کے ، اسے اگلی ساعت کے کنارے پر ، بلند یا پست ہونے کا اشارہ مہیا کیا جاتا ہے۔پلٹ اگلی ساعت کے کنارے پر اس اشارے کے مطابق حال اختیار کرتا ہے۔یہ فیصلہ کہ اگلی ساعت پر پلٹ بلند یا پست حال اختیار کرے گا ، دور کے موجودہ حال کو دیکھ کر کیا جاتا ہے۔اس طریقہ کار کو چند مثالوں سے سمجھتے ہیں۔

\جزوحصہ{معاصر ثنائی گنت کار}
\اصطلاح{تین بِٹ معاصر ثنائی گنت کار }\فرہنگ{گنت کار!تین بِٹ، معاصر}\حاشیہب{three bit synchronous counter}\فرہنگ{counter!synchronous, three bit} شکل \حوالہء{ 8.3 } میں پیش ہے۔ مخارج \عددی{Q_0} کمتر رتبی بِٹ جبکہ \عددی{Q_2} بلند تر رتبی بِٹ ہے۔اس دور کی بناوٹ سیکھتے ہیں۔



جدول \حوالہ{جدول_گنت_کار_تین_بِٹ_معاصر} میں \موٹا{ موجودہ حال } کی قطار میں تین بِٹ ثنائی گنتی لکھی گئی ہے جو کسی بھی لمحے پلٹ کا موجودہ حال پیش کرتی ہے۔موجودہ حال استعمال کرتے ہوئے باقی جدول حاصل ہو گا۔ جدول کی پہلی صف پر غور کریں جہاں موجودہ گنتی یا موجودہ حال \عددی{000_2} ہے۔ہم چاہتے ہیں کہ اگلا عدد \عددی{001_2} ہو ، لہٰذا \موٹا{ اگلے حال }کی پہلی صف میں ہم \عددی{001_2} لکھتے ہیں۔آخری صف میں موجودہ حال \عددی{111_2} ہے۔تین بِٹ استعمال کرتے ہوئے یہیں تک گنتی ممکن ہے۔اس آخری گنتی تک پہنچ کر ہم دوبارہ \عددی{000_2} سے گنتی شروع کرتے ہی، لہٰذا آخری صف میں اگلا حال \عددی{000_2} ہو گا۔یوں موجودہ حال کی دوسری صف در حقیقت اگلے حال کی پہلی صف ہو گی۔اسی طرح موجودہ حال کی تیسری صف اگلے حال کی دوسری صف ہو گی، اور موجودہ حال کی پہلی صف اگلے حال کی آخری صف ہو گی۔
\begin{table}
\caption{معاصر ثنائی گنت کار کے حال}
\label{جدول_گنت_کار_تین_بِٹ_معاصر}
\centering
\begin{otherlanguage}{english}
\begin{tabular}{CCC|CCC|CCC}
\toprule
\multicolumn{3}{c}{\text{\RL{موجودہ حال}}} &\multicolumn{3}{|c|}{\text{\RL{اگلا حال}}} &
\multicolumn{3}{c}{\text{\RL{مداخل}}}\\
\midrule
Q_2&Q_1&Q_0&Q_2&Q_1&Q_0&T_2&T_1&T_0\\
\midrule
0&0&0&0&0&1&0&0&1\\
0&0&1&0&1&0&0&1&1\\
0&1&0&0&1&1&0&0&1\\
0&1&1&1&0&0&1&1&1\\
1&0&0&1&0&1&0&0&1\\
1&0&1&1&1&0&0&1&1\\
1&1&0&1&1&1&0&0&1\\
1&1&1&0&0&0&1&1&1\\
\bottomrule
\end{tabular}
\end{otherlanguage}
\end{table}

 پہلی صف کے کمتر رتبی بِٹ \عددی{Q_0} پر غور کرتے ہیں۔اس بِٹ کی موجودہ قیمت کو موجودہ حال \عددی{Q_0} ظاہر کرتا ہے جو \عددی{0} ہے جبکہ اس کی اگلی قیمت اگلا حال \عددی{Q_0} ظاہر کرتا ہے جو \عددی{1} ہے۔ٹی پلٹ استعمال کرتے ہوئے ساعت کے کنارہ چڑھائی پر پلٹ کا حال \عددی{0} سے \عددی{1} کرنے کی خاطر پلٹ کے مخارج \عددی{T_0} کو بلند کرنا ہو گا۔یہ معلومات جدول \حوالہ{جدول_گنت_کار_ٹی_پلٹ} سے حاصل کی گئی۔ یوں جدول میں \موٹا{مداخل} کا خانہ بنا کر اس کی پہلی صف میں \عددی{T_0} کی قیمت \عددی{1} لکھتے ہیں۔
\begin{table}
\caption{ٹی پلٹ کی کارکردگی}
\label{جدول_گنت_کار_ٹی_پلٹ}
\centering
\begin{otherlanguage}{english}
\begin{tabular}{CL}
T&Q_{n+1}\\
\midrule
0&Q_n\\
1&\overline{Q}_n
\end{tabular}
\end{otherlanguage}
\end{table}


اسی (پہلی) صف میں اگلے بِٹ \عددی{Q_1} پر غور کرتے ہیں۔اس بِٹ کی موجودہ قیمت \عددی{0} ہے اور اس کی اگلی قیمت بھی \عددی{0} ہے، لہٰذا ساعت کے اگلے کنارے پر ہم نہیں چاہتے کہ یہ پلٹ اپنا حال تبدیل کرے۔یوں اس پلٹ کے مداخل \عددی{T_1} کو پست رکھنا ہو گا۔اس طرح \عددی{T_1} کے خانے میں \عددی{0} لکھا جائے گا۔اسی طرز پر تمام صفوں کے تمام مداخل کے لئے جدول کے خانے پُر کیے گئے ہیں۔


دور بنانے کے لئے جدول\حوالہ{جدول_گنت_کار_تین_بِٹ_معاصر} میں \اصطلاح{مداخل }کی قطار استعمال ہو گی جس سے مجموعہ ارکان ضرب کی ترکیب سے درج ذیل مساوات لکھے جا سکتے ہیں۔
\begin{gather}
\begin{aligned}
T_0&=1\\
T_1&=\overline{Q}_2\overline{Q_1} Q_0+\overline{Q}_2 Q_1 Q_0+Q_2\overline{Q_1} Q_0+Q_2Q_1Q_0\\
T_2&=\overline{Q}_2 Q_1 Q_0+Q_2Q_1Q_0
\end{aligned}
\end{gather}


یہ مساوات موجودہ حال کی قیمتیں مدِ نظر رکھ کر لکھی گئی ہیں۔جدول \حوالہ{جدول_گنت_کار_تین_بِٹ_معاصر} میں موجود مواد سے شکل \حوالہء{8.4 } میں پیش کارناف نقشوں کی مدد سے درج ذیل سادہ مساواتیں حاصل کی گئی ہیں ۔
\begin{gather}
\begin{aligned}\label{مساوات_گنت_کار_تین_ثنائی}
T_0&=1\\
T_1&=Q_0\\
T_2&=Q_1Q_0
\end{aligned}
\end{gather}

شکل \حوالہء{ 8.3 } میں تین پلٹوں کو مساوات \حوالہ{مساوات_گنت_کار_تین_ثنائی} سے حاصل برقی اشارات بطور مداخل فراہم کر کے \اصطلاح{ تین بِٹ معاصر ثنائی گنت کار }\فرہنگ{گنت کار!معاصر، تین بِٹ ثنائی}\حاشیہب{three bit synchronous binary counter}\فرہنگ{counter!synchronous, three bit binary}حاصل کیا گیا ہے۔

 جدول \حوالہ{جدول_گنت_کار_تین_بِٹ_معاصر} دیکھ کر بھی مساوات \حوالہ{مساوات_گنت_کار_تین_ثنائی} حاصل کیے جا سکتے ہیں۔ اس جدول پر غور کرنے سے دیکھا جا سکتا ہے کہ \عددی{Q_0} ہر ساعت کے کنارے پر تبدیل ہوتا ہے۔ \عددی{T_0} پر \عددی{1} مہیا کرنے سے یہی حاصل ہو گا (جو مساوات \حوالہ{مساوات_گنت_کار_تین_ثنائی} کا پہلا جزو ہے)۔ جدول میں جب بھی \عددی{Q_0} کی قیمت \عددی{1} ہو ، اگلی ساعت کے کنارے پر \عددی{Q_1} کی قیمت تبدیل ہوتی ہے، جو \عددی{T_1} کو \عددی{Q_0} فراہم کرنے سے حاصل ہو گا (یہ درج بالا مساوات کا دوسرا جزو ہے)۔ اسی طرح جدول میں جب بھی \عددی{Q_0} اور \عددی{Q_1} کی قیمتیں بیکوقت \عددی{1} ہوں، اگلی ساعت کے کنارے پر \عددی{Q_2} کی قیمت تبدیل ہوتی ہے۔یوں \عددی{T_2} کو \عددی{Q_1Q_0} فراہم کرنا ہو گا (درج بالا مساوات کا تیسرا جزو)۔ متعدد بِٹ ثنائی گنتی پر غور کرنے سے دیکھا جا سکتا ہے کہ کوئی بھی مخارج، ساعت کے اگلے کنارے، تب حال تبدیل کرتا ہے جب اس سے کمتر تمام مخارج کی قیمتیں بیکوقت \عددی{1} ہوں۔یوں \اصطلاح{ چار بِٹ معاصر ثنائی گنت کار }\فرہنگ{گنت کار!ثنائی، معاصر، چار بِٹ}\حاشیہب{four bit synchronous binary counter}\فرہنگ{counter!synchronous, binary, four bit} کے لئے درج ذیل ہو گا۔
\begin{gather}
\begin{aligned}
T_0&=1\\
T_1&=Q_0\\
T_2&=Q_1Q_0\\
T_3&=Q_2Q_1Q_0
\end{aligned}
\end{gather}

\جزوحصہ{ثنائی علامتی روپ معاصر اعشاری گنت کار}
گزشتہ حصے میں تین بِٹ ثنائی گنت کار پر غور کیا گیا، جو \عددی{000_2} تا \عددی{111_8} گنتی کرنے کی صلاحیت رکھتا ہے۔چار بِٹ ثنائی گنت کار \عددی{0000_2} تا \عددی{1111_2} ثنائی گنتی کر سکتا ہے۔چار بِٹ ثنائی گنت کار کے دور کو \عددی{0000_2} تا \عددی{1001_2} گنتی کرنے کا پابند بنانے سے \اصطلاح{ثنائی علامتی روپ اعشاری گنت کار}\فرہنگ{گنت کار!اعشاری، ثنائی علامتی روپ}\حاشیہب{BCD decimal counter}\فرہنگ{counter!decimal, BCD} حاصل ہو گا، جس پر اس حصہ میں غور کیا جائے گا۔

جدول \حوالہ{جدول_گنت_کار_ثنائی_اعشاری} میں ثنائی علامتی روپ اعشاری گنت کار کے حال پیش ہیں۔جدول میں \موٹا{ مخارج } \عددی{y} کی قطار کا اضافہ کیا گیا ہے۔مخارج \عددی{y } صفر سے نو تک گنتی پوری ہونے پر ساعت کے ایک\اصطلاح{ دوری عرصہ }\فرہنگ{دوری عرصہ}\حاشیہب{time period}\فرہنگ{time period} کے لئے بلند ہوتا ہے۔ہم آگے دیکھیں گے کہ \عددی{y} استعمال کرتے ہوئے متعدد اعشاری ہندسوں کے گنت کار تخلیق دیے جاتے ہیں۔

\begin{table}
\caption{ثنائی علامتی روپ اعشاری گنت کار کے حال}
\label{جدول_گنت_کار_ثنائی_اعشاری}
\centering
\begin{otherlanguage}{english}
\begin{tabular}{CCCC|CCCC|C|CCCC}
\toprule
\multicolumn{4}{c}{\text{\RL{موجودہ حال}}} &\multicolumn{4}{|c|}{\text{\RL{اگلا حال}}} &\text{\RL{مخارج}}&
\multicolumn{4}{c}{\text{\RL{مداخل}}}\\
\midrule
Q_3&Q_2&Q_1&Q_0&Q_3&Q_2&Q_1&Q_0&y&T_3&T_2&T_1&T_0\\
\midrule
0&0&0&0&0&0&0&1&0&0&0&0&1\\
0&0&0&1&0&0&1&0&0&0&0&1&1\\
0&0&1&0&0&0&1&1&0&0&0&0&1\\
0&0&1&1&0&1&0&0&0&0&1&1&1\\
0&1&0&0&0&1&0&1&0&0&0&0&1\\
0&1&0&1&0&1&1&0&0&0&0&1&1\\
0&1&1&0&0&1&1&1&0&0&0&0&1\\
0&1&1&1&1&0&0&0&0&1&1&1&1\\
1&0&0&0&1&0&0&1&0&0&0&0&1\\
1&0&0&1&0&0&0&0&1&1&0&0&1\\
\bottomrule
\end{tabular}
\end{otherlanguage}
\end{table}




اس جدول میں \عددی{1010_2} تا \عددی{1111_2} ترتیب استعمال نہیں ہوتے، لہٰذا کارناف نقشوں کی مدد سے پلٹوں کے مداخل \عددی{T_0} تا \عددی{T_3} اور مخارج \عددی{y} کی سادہ مساوات حاصل کرتے وقت انہیں \اصطلاح{ غیر ضروری حال }تصور کیا جاتا ہے۔شکل \حوالہء{ 8.6 } میں درج ذیل سادہ مساوات حاصل کرنا دکھایا گیا ہے۔
\begin{gather}
\begin{aligned}
T_0&=1\\
T_1&=\overline{Q}_3Q_0\\
T_2&=Q_1Q_0\\
T_3&=Q_3Q_0+Q_2Q_1Q_0\\
y&=Q_3Q_0
\end{aligned}
\end{gather}

ان مساوات سے حاصل دور شکل\حوالہء{ 8.7 } میں پیش ہے ، جہاں تمام پلٹ کے مداخل پر اضافی ضرب گیٹ نصب کر کے گنتی شروع اور روکنے کی اضافی صلاحیت بھی پیدا کی گئی ہے۔ ان اضافی ضرب گیٹوں کو برقی اشارہ \موٹا{ گِن} مہیا کیا گیا ہے۔یہ اشارہ بلند ہونے کی صورت میں دور گنتی کرتا ہے اور اشارہ پست ہونے کی صورت میں گنتی روکتا ہے۔

شکل \حوالہء{8.8 } میں تین درجی دور بنایا گیا ہے جو \عددی{000_{10}} تا \عددی{999_{10}} گنتی کرتا ہے۔اسے بنانے کی خاطر تین عدد \اصطلاح{ثنائی علامتی روپ اعشاری گنت کار } استعمال کیے گئے۔اسی طرح مزید درجات جوڑ کر درکار ہندسوں کا گنت کار بنایا جاتا ہے۔ 


اس دور کی کارکردگی کچھ یوں ہے۔گنتی شروع کرنے سے قبل \عددی{\overline{\text{\RL{زبردستی پست}}}} کو لمحاتی پست کر کے گنتی \عددی{000_{10}} کر دی جاتی ہے۔ساعت کے کنارہ چڑھائی پر اکائی ہندسے کی گنتی بتدریج بڑھتی ہے؛ اکائی درجے کا مخارج \عددی{y} پست رہتا ہے جو دہائی اور سینکڑا کی گنتی روک کر رکھتا ہے۔گنتی \عددی{009_{10}} تک پہنچتے ہی اکائی درجہ کا مخارج \عددی{y} ایک دوری عرصہ کے لئے بلند ہو گا۔یوں اگلے ساعت کے کنارہ چڑھائی پر اکائی درجہ کا ہندسہ \عددی{9_{10}}سے \عددی{0_{10}} ہو جائے گا، جبکہ دہائی درجے کا ہندسہ \عددی{0_{10}} سے بڑھ کر \عددی{1_{10}} ہو جائے گا اور اسی وقت اکائی کا مخارج \عددی{y} واپس پست حال اختیار کرے گا۔یوں اس سے اگلے ساعت کے کنارے پر صرف اکائی درجہ کی گنتی چالو رہتی ہے جبکہ دہائی اور سینکڑا کی گنتی رکی رہتی ہے۔اسی طرح \عددی{099_{10}} کے بعد اکائی اور دہائی درجات کے مخارج \عددی{y} بلند ہوتے ہیں جس کی وجہ سے اگلے ساعت کے کنارہ چڑھائی پر سینکڑا \عددی{0_{10}} سے بڑھ کر \عددی{1_{10}} ہو جائے گا جبکہ اکائی اور دہائی درجات \عددی{9_{10}} سے \عددی{0_{10}} ہو جائیں گے اور ساتھ ہی ان کے مخارج \عددی{y} دوبارہ پست ہو جائیں گے۔

\ابتدا{مشق}
انٹرنیٹ سے \عددی{7493} اور \عددی{4516} کے معلوماتی صفحات حاصل کریں۔انہیں استعمال کرتے ہوئے متعدد بِٹ گنت کار تخلیق دیں۔
\انتہا{مشق}


\حصہ{دیگر گنت کار}
\جزوحصہ{متغیر لمبائی گنت کار}
چار بِٹ ثنائی گنت کار \عددی{0000_2} تا \عددی{1111_2} گنتی کرتا ہے۔ متوازی دخول استعمال کرکے اس کو دو اعداد کے بیچ گنتی کرنے پر مجبور کیا جا سکتا ہے۔ایسے گنت کار کو ہم \اصطلاح{متغیر لمبائی گنت کار}\فرہنگ{گنت کار!متغیر لمبائی}\حاشیہب{variable length counter}\فرہنگ{counter!variable length}کہیں گے۔جس عدد سے گنتی کا آغاز کرنا ہو وہ عدد دور کو متوازی فراہم کیا جاتا ہے اور جہاں گنتی کا اختتام کرنا ہو وہاں پہنچ کر دور کو مجبور کیا جاتا ہے کہ وہ دوبارہ متوازی فراہم کردہ عدد داخل کر کے گنتی از سرے نو شروع کرے۔

چار بِٹ معاصر ثنائی گنت کار مثال بناتے ہوئے \عددی{0110_2} سے \عددی{1100_2} تک گنتی کرنے والا گنت کار بناتے ہیں، جو شکل \حوالہء{8.9 }میں پیش ہے۔ نقطہ دار مستطیل میں مساوات \حوالہ{مساوات_گنت_کار_تین_ثنائی} سے حاصل دور دکھایا گیا ہے جس میں ہر پلٹ کے ساتھ اضافی دو ضرب گیٹ اور ایک جمع گیٹ جوڑ کر متوازی دخول کی صلاحیت پیدا کی گئی ہے۔

اس دور میں ابتدائی عدد ، جس کو \عددی{A} سے ظاہر کیا گیا ہے اور جس کی قیمت \عددی{0110_2} ہے، متوازی داخل کیا جاتا ہے۔ اختتامی عدد \عددی{1100_2} ہے جس کو نقطہ دار دائرے میں بند ترکیبی دور پہچان کر اپنا مخارج پست کرتا ہے جس کی بدولت ساعت کے اگلے کنارے پر \عددی{0110_2} دور میں متوازی داخل ہو گا۔اس طرح یہ گنت کار \عددی{0110_2} سے لے کر \عددی{1100_2} تک گنتا ہے۔

دور میں \عددی{0110_2}پہلی مرتبہ داخل کرنے کا طریقہ نہیں دکھایا گیا۔


\جزوحصہ{بے ترتیب گنت کار}
معاصر ثنائی گنت کار پر بحث کے دوران جدول \حوالہ{جدول_گنت_کار_تین_بِٹ_معاصر} پیش کیا گیا ۔ اس جدول کے \موٹا{موجودہ حال} خانوں میں \عددی{000_2}، \عددی{001_2}، \عددی{011_2}، وغیرہ پُر کر کے باقی جدول حاصل کیا گیا۔ یوں حاصل گنت کار \عددی{000_2} سے بتدریج بڑھتے ہوئے \عددی{111_2} تک گنتا ہے۔

یہ ضروری نہیں کہ گنت کار عام فہم گنتی کی ترتیب میں ہی گننے۔ \موٹا{ موجودہ حال} صفوں میں کوئی بھی ترتیب لکھی جا سکتی ہے۔ فقط اتنا خیال رکھنا ضروری ہے کہ ہر صف میں منفرد عدد لکھا جائے۔ باقی جدول ان اندراج کے مطابق پورا کرنے سے ایسا گنت کار حاصل ہو گا جو \موٹا{موجودہ حال} صفوں میں لکھے گئے اعداد کے مطابق گنتی کرے گا۔ ہم اس کو \اصطلاح{بے ترتیب گنت کار }\فرہنگ{گنت کار!بے ترتیب} پکار سکتے ہیں۔

\ابتدا{مشق}\شناخت{مشق_گنت_کار_بے_ترتیب}
ایسا\اصطلاح{ بے ترتیب گنت کار } تخلیق دیں جو جدول \حوالہ{جدول_گنت_کار_بلا_ترتیب} میں پیش اعداد کی ترتیب کے مطابق گنتا ہو۔ یہ گنت کار \عددی{101_2} سے آغاز کرے گا۔ پہلی ساعت پر \عددی{011_2} اور دوسری ساعت پر \عددی{110_2} دے گا اور \عددی{001_2} تک پہنچنے کے بعد دوبارہ \عددی{101_2} سے گننا شروع کرے گا۔
\begin{table}
\caption{بے ترتیب گنت کار، برائے مشق \حوالہ{مشق_گنت_کار_بے_ترتیب}}
\label{جدول_گنت_کار_بلا_ترتیب}
\centering
\begin{otherlanguage}{english}
\begin{tabular}{CCC}
\toprule
\multicolumn{3}{c}{\text{\RL{موجودہ حال}}}\\
\midrule
Q_2&Q_1&Q_0\\
\midrule
1&0&1\\
0&1&1\\
1&1&0\\
0&1&0\\
1&0&0\\
0&0&0\\
0&0&1\\
\bottomrule
\end{tabular}
\end{otherlanguage}
\end{table}

\انتہا{مشق}

\جزوحصہ{ چھلا گنت کار}
\عددی{n} \اصطلاح{بِٹ چھلا گنت کار}\فرہنگ{گنت کار!چھلا}\حاشیہب{ring counter}\فرہنگ{counter!ring} کے مخارج میں ایک ہی بلند بِٹ گھومتا ہے؛ باقی تمام بِٹ پست رہتے ہیں۔ایک ہی بلند بِٹ کو ساعت کے کنارے پر ایک پلٹ سے دوسرے پلٹ منتقل کیا جاتا ہے۔شکل \حوالہء{8.10 } میں چار بِٹ چھلا گنت کار پیش ہے، جبکہ جدول \حوالہ{جدول_گنت_کار_چھلا} میں اس کی گنتی پیش کی گئی ہے۔
\begin{table}
\caption{چار بِٹ چھلا گنت کار}
\label{جدول_گنت_کار_چھلا}
\centering
\begin{otherlanguage}{english}
\begin{tabular}{CCCC}
\toprule
\multicolumn{4}{c}{\text{\RL{موجودہ حال}}}\\
\midrule
Q_3&Q_2&Q_1&Q_0\\
\midrule
0&0&0&1\\
0&0&1&0\\
0&1&0&0\\
1&0&0&0\\
\bottomrule
\end{tabular}
\end{otherlanguage}
\end{table}


\جزوحصہ{دورانیہ پیدا کار}
بعض اوقات ہمیں مقررہ دورانیہ کا بلند یا پست اشارہ درکار ہوتا ہے۔تین بِٹ کا معاصر ثنائی الٹ گنت کار استعمال کرتے ہوئے ایسا دور تشکیل دیتے ہیں۔اس دور کو ہم \اصطلاح{ دورانیہ پیدا کار }\فرہنگ{دورانیہ پیدا کار}\حاشیہب{pulse generator}\فرہنگ{pulse generator} کہیں گے۔ 

تین بِٹ الٹ گنت کار \عددی{111_2} تا \عددی{000_2} دہراتا ہے۔شکل \حوالہء{8.11 } میں متوازی دخول صلاحیت رکھنے والا تین بِٹ الٹ گنت کار استعمال کیا گیا ہے جو اس دوران گنتی کرتا ہے جب مداخل \موٹا{گن} بلند ہو۔اس دور کو تین بِٹ بطور درکار دورانیہ کے فراہم کیے جاتے ہیں۔\موٹا{متوازی لکھ } مداخل لمحاتی بلند کرنے سے یہ تین بِٹ گنت کار میں لکھے جاتے ہیں۔جب تک گنت کار کے تینوں خارجی بِٹ پست نہ ہوں جمع گیٹ بلند رہتا ہے اور یوں گنت کار الٹ گنتی جاری رکھتا ہے۔جیسے ہی گنت کار \عددی{000_2} کو پہنچتا ہے ، جمع گیٹ کا مخارج پست ہو جاتا ہے اور یوں گنت کار گنتی روک دیتا ہے۔ اس طرح تین بِٹ میں پیش دورانیہ کے برابر دورانیے کے لئے جمع گیٹ کا مخارج یعنی \موٹا{دورانیہ } بلند رہتا ہے۔


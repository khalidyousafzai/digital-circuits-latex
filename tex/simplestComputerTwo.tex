\باب{کمپیوٹر با}
  ارتقائی طور پر کمپیوٹر الف  ایک قدیم  مشین ہے جو  چند سادہ ہدایت پر عمل درآمد کر سکتا ہے۔ اس باب میں  ارتقا کی اگلی کڑی  پر غور کیا جائے گا جسے ہم کمپیوٹر با کہیں گے۔ کمپیوٹر با چھلانگ  کی ہدایات  جانتا ہے جو   برنامہ  کے کسی   حصے پر دوبارہ  عمل کرنے  یا اس حصے کو نظر انداز کرنے پر کمپیوٹر کو  مجبور کر سکتی ہیں۔ جیسا آپ جلد جان پائیں گے،چھلانگ  ہدایات  کی بدولت   کمپیوٹر کی طاقت بہت زیادہ  بڑھتی ہے۔
  
  \حصہ{دو طرفہ دفاتر}
  تاروں کی  برقی گنجائش کم کرنے کی غرض سے ہم کمپیوٹر با   کے ہر ایک دفتر اور \عددی{W} گزرگاہ کے بیچ   تاروں کا صرف ایک سلسلہ بچھائیں گے۔  شکل \حوالہء{11.1a} میں   اس تصور کی وضاحت کی گئی ہے۔ درآمدی اور برآمدی پنیے آپس میں جوڑے گئے ہیں؛ گزرگاہ  تک  تاروں کا  صرف  ایک گروہ  جاتا ہے۔
  
کیا درآمدی اور برآمدی پنیے آپس میں جوڑنا کوئی  مسئلہ کھڑا کرتا ہے؟ جی نہیں۔ کمپیوٹر کی دوڑ کے دوران کسی ایک وقت پر  \قول{  لاد } اور\قول{  مجاز } میں  سے  صرف ایک  فعال ہو گا۔ فعال \قول{لاد}  کی صورت میں  ثنائی مواد گزرگاہ سے دفتر  کی  درآمد    کی جانب گامزن ہو گا؛ لاد  عمل کے دوران ، برآمدی راہیں \اصطلاح{   غیر وابسطہ   }\فرہنگ{غیر وابسطہ}\حاشیہب{floating}\فرہنگ{floating}ہوں گی۔ اس کے برعکس،  فعال \قول{مجاز} کی صورت میں ،  ثنائی مواد دفتر سے  گزرگاہ کی طرف گامزن ہو گا، اور درآمدی راہیں غیر وابسطہ ہوں گی۔

سہ  حال دفتر کے درآمدی اور برآمدی پنیوں کو   مخلوط  دور ساز   اندرونی طور  پر  آپس میں جوڑ  سکتا ہے۔ اس سے نا صرف تاروں کی برقی گنجائش کم ہو گی بلکہ  درآمدی و برآمدی   پنیوں کی تعداد بھی کم ہو گی۔ مثلاً، شکل \حوالہء{11.1b} میں آٹھ کی بجائے چار درآمدی و برآمدی پنیے ہیں۔ 

شکل \حوالہء{11.1c} میں سہ حال دفتر ، جس کے درآمدی اور برآمدی  راہ  اندرونی طور پر   آپس میں جڑے ہیں، کی علامت  پیش ہے۔ دو طرفہ تیر ہمیں یاد دلاتا ہے کہ یہ راہ\اصطلاح{ دو طرفہ }\فرہنگ{دو طرفہ}\حاشیہب{bidirectional}\فرہنگ{bidirectional} ہے؛  اس پر  مواد  کسی بھی  طرف  چل سکتا ہے۔

\حصہ{طرز تعمیر}
شکل  \حوالہء{11.2} میں کمپیوٹر با کی طرز تعمیر پیش ہے۔   دفاتر کے وہ برآمدات جو  گزرگاہ \عددی{W} سے منسلک   ہیں سہ حال ہیں؛ جو \عددی{W} گزرگاہ سے منسلک نہیں، وہ دو حال ہیں۔ یہاں بھی ہر ایک دفتر کو  قابو و ترتیب کار قابو اشارات (جو یہاں دکھائے نہیں گئے)  بھیجتا ہے۔  قابو اشارات ساعت کے   اگلے کنارہ چڑھائی پر دفتر کو لادنے،  یا مجاز  ہونے، یا کسی دوسرے مقصد کے لئے تیار کرتے ہیں۔ ہر ڈبے کی مختصر تفصیل درج ذیل ہے۔

\جزوحصہء{داخلی روزن}
کمپیوٹر با کے دو داخلی روزن ہیں جنہیں روزن \عددی{ 1 } اور روزن \عددی{ 2 } کہتے  ہیں۔  سادس عشری  مرموز  \اصطلاح{ٹائپ کار تختی   }\فرہنگ{ٹائپ کار تختی}\حاشیہب{keyboard}\فرہنگ{keyboard} روز ن \عددی{1 }کے ساتھ جڑی ہے۔ یوں ہم  روزن  \عددی{1 }کے ذریعے سادس عشری   برنامہ ہدایات اور مواد  داخل کر سکتے ہیں۔ جیسا آپ دیکھ سکتے ہیں، سادس عشری ٹائپ کار تختی روزن   \عددی{2 }کے بِٹ \عددی{0} کو \قول{تیار} کا اشارہ بھیجتی  ہے۔یہ اشارہ روزن \عددی{1} میں درست مواد کی نشاندہی کرتا ہے۔

روزن \عددی{2} کے  پنیا  \عددی{7} کو جاتا ہوا  \اصطلاح{سلسلہ وار مداخل}    \فرہنگ{سلسلہ وار!مداخل}\حاشیہب{serial in}\فرہنگ{serial!in} اشارے پر بھی  نظر ڈالیں۔  کچھ دیر بعد، ایک مثال کی مدد سے  ، سلسلہ وار  داخل مواد کو متوازی مواد میں تبدیل کرنا دکھایا جائے گا۔

\جزوحصہء{برنامہ گنت کار}
یہاں برنامہ گنتکار \عددی{16} (سولہ)  بِٹ ہے جو  
\begin{align*}
\text{\RL{برنامہ گنتکار}}=0000\,\,0000\,\,0000\,\,0000
\end{align*}
تا 
\begin{align*}
\text{\RL{برنامہ گنتکار}}=1111\,\,1111\,\,1111\,\,1111
\end{align*}
گن سکتا ہے، جو \عددی{0000H} تا \عددی{FFFFH}، یا اعشاری \عددی{0} تا \عددی{65535} کے برابر ہے۔

کمپیوٹر کی ہر دوڑ  سے قبل  پست \عددی{\overline{CLR}} اشارہ برنامہ گنتکار کو  زبردستی   صاف کرتا ہے؛ یوں حافظہ کے مقام \عددی{0000H} پر موجود  ہدایت سے عمل شروع ہو گا۔

\جزوحصہء{دفتر  پتہ اور حافظہ}
بازیابی پھیرے کے دوران، دفتر پتہ کو برنامہ گنت کار  \عددی{16} بِٹ پتہ فراہم کرے گا، جس کے بعد  حافظہ کے مطلوبہ مقام  سے    دو حال\قول{ دفتر پتہ}   مخاطب  ہو گا۔کمپیوٹر با میں \عددی{0000H} تا \عددی{07FFH}   پتہ \عددی{2K}  پختہ حافظہ  استعمال  کرتا ہے ۔ پختہ حافظہ  میں موجود برنامے کو  \اصطلاح{ نگران }\فرہنگ{نگران}\حاشیہب{monitor}\فرہنگ{monitor}کہتے ہیں۔    برقی طاقت  کی فراہمی  پر  کمپیوٹر کی ابتدائی  صورت طے کرنا، ٹائپ کار تختی  کے مواد کی   تشریح، اور ایسے دیگر کام \قول{ نگران برنامہ  } کی ذمہ داری ہے۔ باقی \عددی{62K}    عارضی حافظہ کے لئے  مختص ہے۔ یوں \عددی{0800H} تا \عددی{FFFFH} پتے عارضی حافظہ کے لئے استعمال ہوں گے۔

\جزوحصہء{دفتر مواد}
حافظہ کے مواد کا دفتر جس کو ہم مختصراً\اصطلاح{ دفتر مواد }\فرہنگ{دفتر مواد}\حاشیہب{memory data register}\فرہنگ{memory data register} کہیں گے آٹھ بِٹ مستحکم کا رہے۔ اس کا مخارج عارضی حافظہ    سے جڑا ہے۔یہ دفتر لکھ عمل  سے قبل  گزرگاہ سے  مواد حاصل کرتا ہے، اور   پڑھ  عمل کے بعد گزرگاہ کو مواد بھیجتا  ہے۔

\جزوحصہء{دفتر ہدایت}
کمپیوٹر با کی ہدایات کی تعداد  کمپیوٹر الف کی ہدایات کی تعداد سے زیادہ ہے لہٰذا اس کا دفتر ہدایت \عددی{4} بِٹ کی بجائے \عددی{8} بِٹ ہے۔ آٹھ بِٹ  میں \عددی{256} ہدایات  سموئے جا  سکتے ہیں۔ کمپیوٹر با کے کل \عددی{42} ہدایتی رمز ہیں جنہیں \عددی{8} بِٹ میں ڈالنا مسئلہ پیش نہیں کریگا۔ آٹھ بِٹ ہدایتی رمز استعمال کرتے ہوئے کمپیوٹر با کی ہدایات کو    \عددی{8080\! / \! 8085}  کی    ہدایات  (جو خود آٹھ بِٹ ہیں)کے ہم آہنگ رکھا  گیا ہے۔ کمپیوٹر با کی تمام ہدایات  \عددی{8080\! / \! 8085}  کی ہدایات کے عین مطابق ہیں۔

\جزوحصہء{قابو و ترتیب کار}
قابو و ترتیب کار وہ قابو الفاظ یا خرد ہدایات پیدا کرتا ہے جو    کمپیوٹر  کے باقی حصوں کو  ساتھ چلاتے اور ان سے کام   لیتے ہیں۔کمپیوٹر با کی ہدایات کی تعداد زیادہ ہے لہٰذا  اس کے قابو و ترتیب کار  کا دور بھی زیادہ بڑا ہو گا۔اگرچہ، قابو لفظ بڑا ہو گا، بنیادی تصور میں کوئی فرق نہیں: ساعت کے اگلے کنارہ چڑھائی پر دفاتر کا ردعمل   قابو لفظ یا خرد ہدایات کے تحت ہو گا۔

\جزوحصہء{دفتر \عددی{\regA}}
دفتر \عددی{\regA} کا دو حال مخارج \قول{ مرکز حساب و منطق  } کو جاتا ہے؛ اس کا سہ حال مخارج \عددی{W} گزرگاہ کو جاتا ہے۔ یوں دفتر \عددی{\regA} میں موجود \عددی{8} بِٹ لفظ مسلسل مرکز حساب و منطق کو چلاتا ہے، تاہم  یہی لفظ گزرگاہ پر صرف اس وقت  ڈالا جاتا ہے  جب \عددی{E_A} فعال ہو۔

\جزوحصہء{مرکز حساب و منطق اور جھنڈے}
معیاری  \اصطلاح{ مرکز  حساب و منطق  }\فرہنگ{مرکز حساب و منطق}\حاشیہب{ALU, arithmetic logic unit}\فرہنگ{ALU}    کے مخلوط ادوار عام دستیاب ہیں۔ان   \قول{مراکز حساب و منطق } میں عموماً \عددی{4} یا اس سے زیادہ  قابو بِٹ  ہوں گے ، جو  \عددی{\regA} اور \عددی{\regB} الفاظ پر درکار حسابی اور منطقی عمل  تعین کرتے ہیں۔  کمپیوٹر با  میں مستعمل  مرکز حساب و منطق ، حسابی  اور منطقی اعمال کرنے کی صلاحیت رکھتا  ہے۔

\اصطلاح{جھنڈا }\فرہنگ{جھنڈا}\حاشیہب{flag}\فرہنگ{flag} سے مراد    ایک   پلٹ کار  ہے، جو  کمپیوٹر دوڑ کے دوران بدلتے حالات   پر نظر رکھتا ہے۔ کمپیوٹر با میں دو جھنڈے پائے جاتے ہیں۔کسی ہدایت پر عمل کے دوران دفتر \عددی{\regA} کا مواد منفی ہونے  کی صورت میں  \اصطلاح{ جھنڈا علامت }\فرہنگ{جھنڈا !علامت}\حاشیہب{sign flag}\فرہنگ{flag!sign} بلند ہو گا۔ دفتر \عددی{\regA} کا مواد صفر ہونے پر \اصطلاح{ جھنڈا صفر }\فرہنگ{جھنڈا!صفر}\حاشیہب{zero flag}\فرہنگ{flag!zero} بلند ہو گا۔

\جزوحصہء{عارضی دفتر، دفتر \عددی{\regB}، اور دفتر \عددی{\regC}}
دفتر \عددی{\regA} کے ساتھ جمع  یا اس سے منفی ہونے والا مواد دفتر \عددی{\regB} کی بجائے \موٹا{ عارضی دفتر }میں رکھا جاتا ہے۔ یوں دفتر \عددی{\regB} دیگر کام کے لئے استعمال کیا جا سکتا ہے۔ عارضی دفتر اور دفتر \عددی{\regB} کے علاوہ کمپیوٹر با میں دفتر \عددی{\regC} بھی پایا جاتا ہے۔ یوں کمپیوٹر دوڑ کے دوران  مواد کی ترسیل میں ہم زیادہ لچک سے کام لے سکتے ہیں۔

\جزوحصہء{خارجی روزن}
کمپیوٹر  با میں دو خارجی روزن ہیں جنہیں روزن \عددی{3} اور روزن \عددی{4} کہا گیا ہے۔ دفتر \عددی{\regA} کے مواد کو روزن \عددی{3} پر لادا جا سکتا ہے،  جو سادس عشری نمائشی تختی کو چلاتا ہے۔ یوں ہم نتائج دیکھ سکتے ہیں۔

دفتر \عددی{\regA} کا مواد  روزن \عددی{4} پر بھی ڈالا جا سکتا ہے۔ روزن \عددی{4} کا پنیا \عددی{7} سادس عشری  مرموز کار کو \اصطلاح{  تشکر }\فرہنگ{تشکر}\حاشیہب{acknowledge}\فرہنگ{acknowledge} کا اشارہ  بھیجتا  ہے۔\قول{ تشکر اشارہ } اور  \اصطلاح{تیار }\فرہنگ{اشارہ!تیار}\حاشیہب{ready}\فرہنگ{signal!ready} اشارہ\اصطلاح{  مصافحہ  }\فرہنگ{مصافحہ}\حاشیہب{handshaking}\فرہنگ{handshaking}کے تصور کا حصہ ہیں، جس پر جلد غور کیا جائے گا۔

روزن \عددی{4} کے بِٹ \عددی{0} پر بھی نظر ڈالیں جو\اصطلاح{ سلسلہ وار  مخارج}\فرہنگ{سلسلہ وار!مخارج}\حاشیہب{serial out}\فرہنگ{serial!out}  اشارے کو ظاہر کرتا ہے۔ایک مثال میں ہم  دفتر \عددی{\regA} کے متوازی مواد کو سلسلہ وار خارجی مواد میں تبدیل کریں گے۔

\حصہ{حافظہ سے رجوع کرنے  والی  راجع ہدایات}
کمپیوٹر با کا بازیابی پھیرا   وہی ہے جو پہلے تھا۔ \عددی{T_1}اب بھی  پتہ حال  ، \عددی{T_2} بڑھوتری حال، اور \عددی{T_3} حافظہ حال ہے۔چونکہ  بازیابی پھیرا میں حافظہ سے دفتر ہدایت میں برنامہ ہدایت  ڈالی جاتی ہے لہٰذا   کمپیوٹر با کی تمام ہدایات حافظہ استعمال کرتی ہیں۔

تاہم تعمیلی پھیرا کے دوران حافظہ سے  رجوع بعض اوقات  کیا جاتا ہے اور بعض اوقات نہیں کیا جاتا؛ اس کا دارومدار ہدایت کی نوعیت پر ہے۔ \قول{راجع ہدایت } وہ ہدایت ہو گی جو  تعمیلی پھیرا کے دوران حافظہ سے رجوع کرے۔

کمپیوٹر با کی کل \عددی{42} ہدایات ہیں۔ آئیں ان میں سے  راجع ہدایات  پر غور کریں۔

\جزوحصہء{\LDA{} اور \STA{}}
\قول{نقل}   کی ہدایت  وہی ہے جو پہلے تھی: مخاطب مقام   (نشان زد مقام) سے دفتر \عددی{\regA} میں حافظہ سے مواد ڈالنا۔  فرق فقط  اتنا ہے کہ  کمپیوٹر با کی رسائی  \عددی{0000H} تا \عددی{FFFFH}  مقامات تک ممکن   ہے۔ مثال کے طور پر،    \قول{نقل \عددی{2000H}} سے مراد حافظہ کے مقام  \عددی{2000H} سے دفتر \عددی{\regA} میں مواد نقل کرنا ہے۔

ہدایت کے مختلف حصوں میں فرق کرنے کے لئے   بعض اوقات   ہدایت کے   پہلے حصے  کو \اصطلاح{ ہدایتی رمز }\فرہنگ{ہدایتی رمز}\حاشیہب{opcode}\فرہنگ{opcode} جبکہ باقی حصے کو  \اصطلاح{زیر عمل }\فرہنگ{زیر عمل}\حاشیہب{operand}\فرہنگ{operand} کہتے ہیں۔ یوں      \قول{نقل \عددی{2000H}} کی ہدایت  میں\قول{ نقل } کو  \موٹا{ہدایتی رمز }اور  \قول{\عددی{2000H} } کو  \موٹا{زیر عمل}  کہیں گے۔ یوں ہدایتی رمز کے دو مختلف معنی لئے جا سکتے ہیں؛ یہ ہدایت کے لئے یا ہدایت کے ثنائی رمز کے لئے استعمال کیا جا سکتا ہے۔ اصل معنی متن سے معلوم   ہو گی۔

\قول{ذخیرہ}  ایک ایسی ہدایت ہے جو دفتر \عددی{\regA} کے مواد کو حافظہ میں محفوظ کرتی ہے۔ اس ہدایت کو  پتہ درکار ہو گا۔ یوں \قول{ذخیرہ \عددی{7FFFH}} کی ہدایت دفتر \عددی{\regA} کے مواد کو حافظہ میں مقام \عددی{7FFFH} پر  رکھتی ہے۔  اگر 

\begin{align*}
\regA=\kop{8AH}
\end{align*}
ہو تب   \قول{ذخیرہ \عددی{7FFFH}} کی تعمیل  مقام \عددی{7FFFH} پر \عددی{8AH}    لکھے گی۔

\جزوحصہء{  \MVI{}{}}
\قول{ \MVI{}{}} ہدایت    دیے گئے دفتر میں متصل مواد  منتقل  کرتی ہے۔یہ کمپیوٹر سے کہتی ہے  کہ  دیے گئے دفتر میں ہدایت رمز کے بعد پیش مواد ڈالے۔ مثال کے طور پر، 
\begin{align*}
\MVI{\regA}{\kop{37H}}
\end{align*}
کمپیوٹر  کو کہتی ہے کہ دفتر \عددی{\regA} میں \عددی{37H} ڈالے۔ اس ہدایت کی تعمیل کے بعد دفتر \عددی{\regA} میں  درج ذیل ثنائی مواد ہو گا۔
\begin{align*}
\regA=\kopBinary{0011\,0111}
\end{align*}

آپ \قول{\MVI{}{}}  ہدایت کو   دفاتر \regA، \regB، اور \regC کے ساتھ ملا کر استعمال کر سکتے ہو۔ان ہدایات کی اشکال  درج ذیل ہیں۔
\begin{align*}
\MVI{\regA}{بائٹ}&\\
\MVI{\regB}{بائٹ}&\\
\MVI{\regC}{بائٹ}&
\end{align*}

\جزوحصہء{ہدایتی رمز}
جدول \حوالہ{شکل_کمپیوٹر_ہدایتی_رمز} میں کمپیوٹر با کی تمام ہدایات پیش ہیں۔ یہ \عددی{8080\! / \! 8085} کی ہدایتی رمز ہیں۔ جیسا آپ دیکھ سکتے ہیں\قول{  \LDA{}}  کا ہدایتی رمز \عددی{3A} ہے، \قول{\STA{}}  کا ہدایتی رمز \عددی{32} ہے، وغیرہ۔ اس باب کو پڑھتے ہوئے اس جدول سے رجوع کریں۔
\begin{table}
\caption{کمپیوٹر با کے ہدایتی رمز}
\label{شکل_کمپیوٹر_ہدایتی_رمز}
\centering
\begin{tabular}{rc|rc}
\toprule
ہدایت&ہدایتی رمز&ہدایت&ہدایتی رمز\\
\midrule
\ADD{\regB}&\kop{80}&\MOV{\regB}{\regA}&\kop{47}\\
\ADD{\regC}&\kop{81}&\MOV{\regB}{\regC}&\kop{41}\\
\ANA{\regB}&\kop{A0}&\MOV{\regC}{\regA}&\kop{4F}\\
\ANA{\regC}&\kop{A1}&\MOV{\regC}{\regB}&\kop{48}\\
\ANI{بائٹ}&\kop{E6}&\MVI{\regA}{بائٹ}&\kop{3E}\\
\CALL{پتہ}&\kop{CD}&\MVI{\regB}{بائٹ}&\kop{06}\\
\CMA &\kop{2F}&\MVI{\regC}{بائٹ}&\kop{0E}\\
\DCR{\regA}&\kop{3D}&\NOP&\kop{00}\\
\DCR{\regB}&\kop{05}&\ORA{\regB}&\kop{B0}\\
\DCR{\regC}&\kop{0D}&\ORA{\regC}&\kop{B1}\\
\HLT&\kop{76}&\ORI{بائٹ}&\kop{F6}\\
\IN{بائٹ}&\kop{DB}&\OUT{بائٹ}&\kop{D3}\\
\INR{\regA}&\kop{3C}&\RAL&\kop{17}\\
\INR{\regB}&\kop{04}&\RAR&\kop{1F}\\
\INR{\regC}&\kop{0C}&\RET&\kop{C9}\\
\JM{پتہ}&\kop{FA}&\STA{پتہ}&\kop{32}\\
\JMP{پتہ}&\kop{C3}&\SUB{\regB}&\kop{90}\\
\JNZ{پتہ}&\kop{C2}&\SUB{\regC}&\kop{91}\\
\JZ{پتہ}&\kop{CA}&\XRA{\regB}&\kop{A8}\\
\LDA{پتہ}&\kop{3A}&\XRA{\regC}&\kop{A9}\\
\MOV{\regA}{\regB}&\kop{78}&\XRI{بائٹ}&\kop{EE}\\
\MOV{\regA}{\regC}&\kop{79}&&\\
\bottomrule
\end{tabular}
\end{table}

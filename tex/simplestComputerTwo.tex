\باب{کمپیوٹر با}
  ارتقائی طور پر کمپیوٹر الف  ایک قدیم  مشین ہے جو  چند سادہ ہدایت پر عمل درآمد کر سکتا ہے۔ اس باب میں  ارتقا کی اگلی کڑی  پر غور کیا جائے گا جسے ہم کمپیوٹر با کہیں گے۔ کمپیوٹر با چھلانگ  کی ہدایات  جانتا ہے جو   برنامہ  کے کسی   حصے پر دوبارہ  عمل کرنے  یا اس حصے کو نظر انداز کرنے پر کمپیوٹر کو  مجبور کر سکتی ہیں۔ جیسا آپ جلد جان پائیں گے،چھلانگ  ہدایات  کی بدولت   کمپیوٹر کی طاقت بہت زیادہ  بڑھتی ہے۔
  
  \حصہ{دو طرفہ دفاتر}
  تاروں کی  برقی گنجائش کم کرنے کی غرض سے ہم کمپیوٹر با   کے ہر ایک دفتر اور \عددی{W} گزرگاہ کے بیچ   تاروں کا صرف ایک سلسلہ بچھائیں گے۔  شکل \حوالہء{11.1a} میں   اس تصور کی وضاحت کی گئی ہے۔ درآمدی اور برآمدی پنیے آپس میں جوڑے گئے ہیں؛ گزرگاہ  تک  تاروں کا  صرف  ایک گروہ  جاتا ہے۔
  
کیا درآمدی اور برآمدی پنیے آپس میں جوڑنا کوئی  مسئلہ کھڑا کرتا ہے؟ جی نہیں۔ کمپیوٹر کی دوڑ کے دوران کسی ایک وقت پر  \قول{  لاد } اور\قول{  مجاز } میں  سے  صرف ایک  فعال ہو گا۔ فعال \قول{لاد}  کی صورت میں  ثنائی مواد گزرگاہ سے دفتر  کی  درآمد    کی جانب گامزن ہو گا؛ لاد  عمل کے دوران ، برآمدی راہیں \اصطلاح{   غیر وابسطہ   }\فرہنگ{غیر وابسطہ}\حاشیہب{floating}\فرہنگ{floating}ہوں گی۔ اس کے برعکس،  فعال \قول{مجاز} کی صورت میں ،  ثنائی مواد دفتر سے  گزرگاہ کی طرف گامزن ہو گا، اور درآمدی راہیں غیر وابسطہ ہوں گی۔

سہ  حال دفتر کے درآمدی اور برآمدی پنیوں کو   مخلوط  دور ساز   اندرونی طور  پر  آپس میں جوڑ  سکتا ہے۔ اس سے نا صرف تاروں کی برقی گنجائش کم ہو گی بلکہ  درآمدی و برآمدی   پنیوں کی تعداد بھی کم ہو گی۔ مثلاً، شکل \حوالہء{11.1b} میں آٹھ کی بجائے چار درآمدی و برآمدی پنیے ہیں۔ 

شکل \حوالہء{11.1c} میں سہ حال دفتر ، جس کے درآمدی اور برآمدی  راہ  اندرونی طور پر   آپس میں جڑے ہیں، کی علامت  پیش ہے۔ دو طرفہ تیر ہمیں یاد دلاتا ہے کہ یہ راہ\اصطلاح{ دو طرفہ }\فرہنگ{دو طرفہ}\حاشیہب{bidirectional}\فرہنگ{bidirectional} ہے؛  اس پر  مواد  کسی بھی  طرف  چل سکتا ہے۔

\حصہ{طرز تعمیر}
شکل  \حوالہء{11.2} میں کمپیوٹر با کی طرز تعمیر پیش ہے۔   دفاتر کے وہ برآمدات جو  گزرگاہ \عددی{W} سے منسلک   ہیں سہ حال ہیں؛ جو \عددی{W} گزرگاہ سے منسلک نہیں، وہ دو حال ہیں۔ یہاں بھی ہر ایک دفتر کو  قابو و ترتیب کار قابو اشارات (جو یہاں دکھائے نہیں گئے)  بھیجتا ہے۔  قابو اشارات ساعت کے   اگلے کنارہ چڑھائی پر دفتر کو لادنے،  یا مجاز  ہونے، یا کسی دوسرے مقصد کے لئے تیار کرتے ہیں۔ ہر ڈبے کی مختصر تفصیل درج ذیل ہے۔

\جزوحصہء{داخلی روزن}
کمپیوٹر با کے دو داخلی روزن ہیں جنہیں روزن \عددی{ 1 } اور روزن \عددی{ 2 } کہتے  ہیں۔  سادس عشری  مرموز  \اصطلاح{ٹائپ کار تختی   }\فرہنگ{ٹائپ کار تختی}\حاشیہب{keyboard}\فرہنگ{keyboard} روز ن \عددی{1 }کے ساتھ جڑی ہے۔ یوں ہم  روزن  \عددی{1 }کے ذریعے سادس عشری   برنامہ ہدایات اور مواد  داخل کر سکتے ہیں۔ جیسا آپ دیکھ سکتے ہیں، سادس عشری ٹائپ کار تختی روزن   \عددی{2 }کے بِٹ \عددی{0} کو \قول{تیار} کا اشارہ بھیجتی  ہے۔یہ اشارہ روزن \عددی{1} میں درست مواد کی نشاندہی کرتا ہے۔

روزن \عددی{2} کے  پنیا  \عددی{7} کو جاتا ہوا  \اصطلاح{سلسلہ وار مداخل}    \فرہنگ{سلسلہ وار!مداخل}\حاشیہب{serial in}\فرہنگ{serial!in} اشارے پر بھی  نظر ڈالیں۔  کچھ دیر بعد، ایک مثال کی مدد سے  ، سلسلہ وار  داخل مواد کو متوازی مواد میں تبدیل کرنا دکھایا جائے گا۔

\جزوحصہء{برنامہ گنت کار}
یہاں برنامہ گنتکار \عددی{16} (سولہ)  بِٹ ہے  لہٰذا یہ
\begin{align*}
0000\,\,0000\,\,0000\,\,0000=\regPC
\end{align*}
تا 
\begin{align*}
1111\,\,1111\,\,1111\,\,1111=\regPC
\end{align*}
گن سکتا ہے، جو \عددی{0000H} تا \عددی{FFFFH}، یا اعشاری \عددی{0} تا \عددی{65535} کے برابر ہے۔

کمپیوٹر کی ہر دوڑ  سے قبل  پست \عددی{\overline{CLR}} اشارہ برنامہ گنتکار کو  زبردستی   صاف کرتا ہے؛ یوں حافظہ کے مقام \عددی{0000H} پر موجود  ہدایت سے عمل شروع ہو گا۔

\جزوحصہء{دفتر  پتہ اور حافظہ}
بازیابی پھیرے کے دوران، دفتر پتہ کو برنامہ گنت کار  \عددی{16} بِٹ پتہ فراہم کرے گا، جس کے بعد  حافظہ کے مطلوبہ مقام  سے    دو حال\قول{ دفتر پتہ}   مخاطب  ہو گا۔کمپیوٹر با میں \عددی{0000H} تا \عددی{07FFH}   پتہ \عددی{2K}  پختہ حافظہ  استعمال  کرتا ہے ۔ پختہ حافظہ  میں موجود برنامے کو  \اصطلاح{ نگران }\فرہنگ{نگران}\حاشیہب{monitor}\فرہنگ{monitor}کہتے ہیں۔    برقی طاقت  کی فراہمی  پر  کمپیوٹر کی ابتدائی  صورت طے کرنا، ٹائپ کار تختی  کے مواد کی   تشریح، اور ایسے دیگر کام \قول{ نگران برنامہ  } کی ذمہ داری ہے۔ باقی \عددی{62K}    عارضی حافظہ کے لئے  مختص ہے۔ یوں \عددی{0800H} تا \عددی{FFFFH} پتے عارضی حافظہ کے لئے استعمال ہوں گے۔

\جزوحصہء{دفتر مواد}
حافظہ کے مواد کا دفتر جس کو ہم مختصراً\اصطلاح{ دفتر مواد }\فرہنگ{دفتر مواد}\حاشیہب{memory data register}\فرہنگ{memory data register} کہیں گے آٹھ بِٹ مستحکم کا رہے۔ اس کا مخارج عارضی حافظہ    سے جڑا ہے۔یہ دفتر لکھ عمل  سے قبل  گزرگاہ سے  مواد حاصل کرتا ہے، اور   پڑھ  عمل کے بعد گزرگاہ کو مواد بھیجتا  ہے۔

\جزوحصہء{دفتر ہدایت}
کمپیوٹر با کی ہدایات کی تعداد  کمپیوٹر الف کی ہدایات کی تعداد سے زیادہ ہے لہٰذا اس کا دفتر ہدایت \عددی{4} بِٹ کی بجائے \عددی{8} بِٹ ہے۔ آٹھ بِٹ  میں \عددی{256} ہدایات  سموئے جا  سکتے ہیں۔ کمپیوٹر با کے کل \عددی{42} ہدایتی رمز ہیں جنہیں \عددی{8} بِٹ میں ڈالنا مسئلہ پیش نہیں کریگا۔ آٹھ بِٹ ہدایتی رمز استعمال کرتے ہوئے کمپیوٹر با کی ہدایات کو    \عددی{8080\! / \! 8085}  کی    ہدایات  (جو خود آٹھ بِٹ ہیں)کے ہم آہنگ رکھا  گیا ہے۔ کمپیوٹر با کی تمام ہدایات  \عددی{8080\! / \! 8085}  کی ہدایات کے عین مطابق ہیں۔

\جزوحصہء{قابو و ترتیب کار}
قابو و ترتیب کار وہ قابو الفاظ یا خرد ہدایات پیدا کرتا ہے جو    کمپیوٹر  کے باقی حصوں کو  ساتھ چلاتے اور ان سے کام   لیتے ہیں۔کمپیوٹر با کی ہدایات کی تعداد زیادہ ہے لہٰذا  اس کے قابو و ترتیب کار  کا دور بھی زیادہ بڑا ہو گا۔اگرچہ، قابو لفظ بڑا ہو گا، بنیادی تصور میں کوئی فرق نہیں: ساعت کے اگلے کنارہ چڑھائی پر دفاتر کا ردعمل   قابو لفظ یا خرد ہدایات کے تحت ہو گا۔

\جزوحصہء{دفتر \عددی{\regA}}
دفتر \عددی{\regA} کا دو حال مخارج \قول{ مرکز حساب و منطق  } کو جاتا ہے؛ اس کا سہ حال مخارج \عددی{W} گزرگاہ کو جاتا ہے۔ یوں دفتر \عددی{\regA} میں موجود \عددی{8} بِٹ لفظ مسلسل مرکز حساب و منطق کو چلاتا ہے، تاہم  یہی لفظ گزرگاہ پر صرف اس وقت  ڈالا جاتا ہے  جب \عددی{E_A} فعال ہو۔

\جزوحصہء{مرکز حساب و منطق اور جھنڈے}
معیاری  \اصطلاح{ مرکز  حساب و منطق  }\فرہنگ{مرکز حساب و منطق}\حاشیہب{ALU, arithmetic logic unit}\فرہنگ{ALU}    کے مخلوط ادوار عام دستیاب ہیں۔ان   \قول{مراکز حساب و منطق } میں عموماً \عددی{4} یا اس سے زیادہ  قابو بِٹ  ہوں گے ، جو  \عددی{\regA} اور \عددی{\regB} الفاظ پر درکار حسابی اور منطقی عمل  تعین کرتے ہیں۔  کمپیوٹر با  میں مستعمل  مرکز حساب و منطق ، حسابی  اور منطقی اعمال کرنے کی صلاحیت رکھتا  ہے۔

\اصطلاح{جھنڈا }\فرہنگ{جھنڈا}\حاشیہب{flag}\فرہنگ{flag} سے مراد    ایک   پلٹ کار  ہے، جو  کمپیوٹر دوڑ کے دوران بدلتے حالات   پر نظر رکھتا ہے۔ کمپیوٹر با میں دو جھنڈے پائے جاتے ہیں۔کسی ہدایت پر عمل کے دوران دفتر \عددی{\regA} کا مواد منفی ہونے  کی صورت میں  \اصطلاح{ جھنڈا علامت }\فرہنگ{جھنڈا !علامت}\حاشیہب{sign flag}\فرہنگ{flag!sign} بلند ہو گا۔ دفتر \عددی{\regA} کا مواد صفر ہونے پر \اصطلاح{ جھنڈا صفر }\فرہنگ{جھنڈا!صفر}\حاشیہب{zero flag}\فرہنگ{flag!zero} بلند ہو گا۔

\جزوحصہء{عارضی دفتر، دفتر \عددی{\regB}، اور دفتر \عددی{\regC}}
دفتر \عددی{\regA} کے ساتھ جمع  یا اس سے منفی ہونے والا مواد دفتر \عددی{\regB} کی بجائے \موٹا{ عارضی دفتر }میں رکھا جاتا ہے۔ یوں دفتر \عددی{\regB} دیگر کام کے لئے استعمال کیا جا سکتا ہے۔ عارضی دفتر اور دفتر \عددی{\regB} کے علاوہ کمپیوٹر با میں دفتر \عددی{\regC} بھی پایا جاتا ہے۔ یوں کمپیوٹر دوڑ کے دوران  مواد کی ترسیل میں ہم زیادہ لچک سے کام لے سکتے ہیں۔

\جزوحصہء{خارجی روزن}
کمپیوٹر  با میں دو خارجی روزن ہیں جنہیں روزن \عددی{3} اور روزن \عددی{4} کہا گیا ہے۔ دفتر \عددی{\regA} کے مواد کو روزن \عددی{3} پر لادا جا سکتا ہے،  جو سادس عشری نمائشی تختی کو چلاتا ہے۔ یوں ہم نتائج دیکھ سکتے ہیں۔

دفتر \عددی{\regA} کا مواد  روزن \عددی{4} پر بھی ڈالا جا سکتا ہے۔ روزن \عددی{4} کا پنیا \عددی{7} سادس عشری  مرموز کار کو \اصطلاح{  تشکر }\فرہنگ{تشکر}\حاشیہب{acknowledge}\فرہنگ{acknowledge} کا اشارہ  بھیجتا  ہے۔\قول{ تشکر اشارہ } اور  \اصطلاح{تیار }\فرہنگ{اشارہ!تیار}\حاشیہب{ready}\فرہنگ{signal!ready} اشارہ\اصطلاح{  مصافحہ  }\فرہنگ{مصافحہ}\حاشیہب{handshaking}\فرہنگ{handshaking}کے تصور کا حصہ ہیں، جس پر جلد غور کیا جائے گا۔

روزن \عددی{4} کے بِٹ \عددی{0} پر بھی نظر ڈالیں جو\اصطلاح{ سلسلہ وار  مخارج}\فرہنگ{سلسلہ وار!مخارج}\حاشیہب{serial out}\فرہنگ{serial!out}  اشارے کو ظاہر کرتا ہے۔ایک مثال میں ہم  دفتر \عددی{\regA} کے متوازی مواد کو سلسلہ وار خارجی مواد میں تبدیل کریں گے۔

\حصہ{حافظہ سے رجوع کرنے  والی  راجع ہدایات}
کمپیوٹر با کا بازیابی پھیرا   وہی ہے جو پہلے تھا۔ \عددی{T_1}اب بھی  پتہ حال  ، \عددی{T_2} بڑھوتری حال، اور \عددی{T_3} حافظہ حال ہے۔چونکہ  بازیابی پھیرا میں حافظہ سے دفتر ہدایت میں برنامہ ہدایت  ڈالی جاتی ہے لہٰذا   کمپیوٹر با کی تمام ہدایات حافظہ استعمال کرتی ہیں۔

تاہم تعمیلی پھیرا کے دوران حافظہ سے  رجوع بعض اوقات  کیا جاتا ہے اور بعض اوقات نہیں کیا جاتا؛ اس کا دارومدار ہدایت کی نوعیت پر ہے۔ \قول{راجع ہدایت } وہ ہدایت ہو گی جو  تعمیلی پھیرا کے دوران حافظہ سے رجوع کرے۔

کمپیوٹر با کی کل \عددی{42} ہدایات ہیں۔ آئیں ان میں سے  راجع ہدایات  پر غور کریں۔

\جزوحصہء{\sLDA اور \sSTA}
\قول{\sLDA}   کی ہدایت  وہی ہے جو پہلے تھی: مخاطب مقام   (نشان زد مقام) سے دفتر \عددی{\regA} میں حافظہ سے مواد ڈالنا۔  فرق فقط  اتنا ہے کہ  کمپیوٹر با کی رسائی  \عددی{0000H} تا \عددی{FFFFH}  مقامات تک ممکن   ہے۔ مثال کے طور پر،    \قول{\LDA{2000H}} سے مراد حافظہ کے مقام  \عددی{2000H} سے دفتر \عددی{\regA} میں مواد نقل کرنا ہے۔

ہدایت کے مختلف حصوں میں فرق کرنے کے لئے   بعض اوقات   ہدایت کے   پہلے حصے  کو \اصطلاح{ ہدایتی رمز }\فرہنگ{ہدایتی رمز}\حاشیہب{opcode}\فرہنگ{opcode} جبکہ باقی حصے کو  \اصطلاح{زیر عمل }\فرہنگ{زیر عمل}\حاشیہب{operand}\فرہنگ{operand} کہتے ہیں۔ یوں      \قول{\LDA{2000H}} کی ہدایت  میں\قول{ \sLDA } کو  \موٹا{ہدایتی رمز }اور  \قول{\عددی{2000H} } کو  \موٹا{زیر عمل}  کہیں گے۔ یوں ہدایتی رمز کے دو مختلف معنی لئے جا سکتے ہیں؛ یہ ہدایت کے لئے یا ہدایت کے ثنائی رمز کے لئے استعمال کیا جا سکتا ہے۔ اصل معنی متن سے   واضح    ہو گی۔

\قول{\sSTA}  ایک ایسی ہدایت ہے جو دفتر \عددی{\regA} کے مواد کو حافظہ میں محفوظ کرتی ہے۔ اس ہدایت کو  پتہ درکار ہو گا۔ یوں \قول{\STA{7FFFH}} کی ہدایت دفتر \عددی{\regA} کے مواد کو حافظہ میں مقام \عددی{7FFFH} پر  رکھتی ہے۔  اگر 

\begin{align*}
\kop{8AH}=\regA
\end{align*}
ہو تب \قول{\STA{7FFFH}}  کی تعمیل  مقام \عددی{7FFFH} پر \عددی{8AH}    لکھے گی۔

\جزوحصہء{  \sMVI}
\قول{\sMVI} ہدایت    دیے گئے دفتر میں متصل مواد  منتقل  کرتی ہے۔یہ کمپیوٹر سے کہتی ہے  کہ  ہدایت رمز کے بعد پیش مواد کو   دیے گئے دفتر میں ڈالے۔ مثال کے طور پر، 
\begin{align*}
\MVI{\regA}{\kop{37H}}
\end{align*}
کمپیوٹر  کو کہتی ہے کہ دفتر \عددی{\regA} میں \عددی{37H} ڈالے۔ اس ہدایت کی تعمیل کے بعد دفتر \عددی{\regA} میں  درج ذیل ثنائی مواد ہو گا۔
\begin{align*}
\kopBinary{0011\,0111}=\regA
\end{align*}

آپ \قول{\sMVI}  ہدایت کو   دفاتر \regA، \regB، اور \regC کے ساتھ ملا کر استعمال کر سکتے ہو۔ان ہدایات کی اشکال  درج ذیل ہیں۔
\begin{align*}
\MVI{\regA}{بائٹ}&\\
\MVI{\regB}{بائٹ}&\\
\MVI{\regC}{بائٹ}&
\end{align*}

\جزوحصہء{ہدایتی رمز}
جدول \حوالہ{شکل_کمپیوٹر_ہدایتی_رمز} میں کمپیوٹر با کی تمام ہدایات پیش ہیں۔ یہ \عددی{8080\! / \! 8085} کی ہدایتی رمز ہیں۔ جیسا آپ دیکھ سکتے ہیں\قول{  \sLDA}  کا ہدایتی رمز \عددی{3A} ہے، \قول{\sSTA}  کا ہدایتی رمز \عددی{32} ہے، وغیرہ۔ اس باب کو پڑھتے ہوئے اس جدول سے رجوع کریں۔
\begin{table}
\caption{کمپیوٹر با کے ہدایتی رمز}
\label{شکل_کمپیوٹر_ہدایتی_رمز}
\centering
\begin{tabular}{rc|rc}
\toprule
ہدایت&ہدایتی رمز&ہدایت&ہدایتی رمز\\
\midrule
\ADD{\regB}&\kop{80}&\MOV{\regB}{\regA}&\kop{47}\\
\ADD{\regC}&\kop{81}&\MOV{\regB}{\regC}&\kop{41}\\
\ANA{\regB}&\kop{A0}&\MOV{\regC}{\regA}&\kop{4F}\\
\ANA{\regC}&\kop{A1}&\MOV{\regC}{\regB}&\kop{48}\\
\ANI{بائٹ}&\kop{E6}&\MVI{\regA}{بائٹ}&\kop{3E}\\
\CALL{پتہ}&\kop{CD}&\MVI{\regB}{بائٹ}&\kop{06}\\
\CMA &\kop{2F}&\MVI{\regC}{بائٹ}&\kop{0E}\\
\DCR{\regA}&\kop{3D}&\NOP&\kop{00}\\
\DCR{\regB}&\kop{05}&\ORA{\regB}&\kop{B0}\\
\DCR{\regC}&\kop{0D}&\ORA{\regC}&\kop{B1}\\
\HLT&\kop{76}&\ORI{بائٹ}&\kop{F6}\\
\IN{بائٹ}&\kop{DB}&\OUT{بائٹ}&\kop{D3}\\
\INR{\regA}&\kop{3C}&\RAL&\kop{17}\\
\INR{\regB}&\kop{04}&\RAR&\kop{1F}\\
\INR{\regC}&\kop{0C}&\RET&\kop{C9}\\
\JM{پتہ}&\kop{FA}&\STA{پتہ}&\kop{32}\\
\JMP{پتہ}&\kop{C3}&\SUB{\regB}&\kop{90}\\
\JNZ{پتہ}&\kop{C2}&\SUB{\regC}&\kop{91}\\
\JZ{پتہ}&\kop{CA}&\XRA{\regB}&\kop{A8}\\
\LDA{پتہ}&\kop{3A}&\XRA{\regC}&\kop{A9}\\
\MOV{\regA}{\regB}&\kop{78}&\XRI{بائٹ}&\kop{EE}\\
\MOV{\regA}{\regC}&\kop{79}&&\\
\bottomrule
\end{tabular}
\end{table}

%-----------------------------
\ابتدا{مثال}
دفتر \دفترالف میں  \زیرعمل{49H}، دفتر \دفترب میں \زیرعمل{4AH}، اور دفتر \دفترج میں \زیرعمل{4BH}  ڈالنے کے لئے برنامہ لکھیں؛ اس کے بعد دفتر \دفترالف کا مواد حافظہ کے  مقام \زیرعمل{6285H} پر رکھیں۔

حل:\quad
 ایسا  ایک برنامہ درج ذیل ہے۔
 \begin{center}
 \begin{tabular}{r}
 \MVI{\regA}{\kop{49H}}\\
 \MVI{\regB}{\kop{4AH}}\\
 \MVI{\regC}{\kop{4BH}}\\
 \STA{\kop{6285H}}\\
 \HLT
 \end{tabular}
 \end{center}
 پہلی تین ہدایات \زیرعمل{49H}، \زیرعمل{4AH}، اور \زیرعمل{4BH} بالترتیب دفاتر \دفترالف، \دفترب، اور \دفترج میں ڈالتے ہیں۔   \STA{\kop{6285H}} ہدایت دفتر \دفترالف کا مواد حافظہ کے مقام \زیرعمل{6285H} میں رکھتی ہے۔
 
 برنامے کی آخری ہدایت \رک  ہے جو ہمیشہ کی طرح کمپیوٹر کو  مواد کی عمل کاری سے روکتی ہے۔
\انتہا{مثال}
%---------------------------------
\ابتدا{مثال}
درج بالا برنامے کا ترجمہ   ، جدول \حوالہ{شکل_کمپیوٹر_ہدایتی_رمز} کی مدد سے،  \عددی{8080\!/\!8085}   کی  مشینی زبان میں  کریں۔ پتہ \زیرعمل{2000H} سے  شروع کریں۔

حل:
\begin{center}
\begin{tabular}{ccr}
\toprule
پتہ & مواد& علامتی روپ\\[0.5ex]
2000H&3EH&\MVI{\regA}{\kop{49H}}\\
2001H&49H&\\
2002H&06H&\MVI{\regB}{\kop{4AH}}\\
2003H&4AH&\\
2004H&0EH&\MVI{\regC}{\kop{4BH}}\\
2005H&4BH&\\
2006H&32H&\STA{\kop{6285H}}\\
2007H&85H&\\
2008H&62H&\\
2009H&76H&\HLT\\
\bottomrule
\end{tabular}
\end{center}

مشینی زبان کے اس برنامہ میں کئی نئے تصور پیش ہیں۔ پہلی ہدایت
\begin{center}
\begin{tabular}{r}
\MVI{\regA}{\kop{49A}}
\end{tabular}
\end{center}
کا  ہدایتی رمز  پہلے پتہ پر  اور زیر عمل بائٹ دوسرے پتے  پر رکھا گیا ہے۔ تمام \عددی{2} بائٹ ہدایات کے لئے  ایسا ہو گا: ہدایتی رمز  پہلے دستیاب  پتے پر جبکہ زیر عمل بائٹ اگلے پتے پر رکھا  جائے گا۔ درج ذیل ہدایت   \عددی{3} بائٹ لمبی ہے (ہدایتی رمز \عددی{1} بائٹ جبکہ زیر عمل مواد \عددی{2} بائٹ ہے)۔
\begin{center}
\begin{tabular}{r}
\STA{\kop{6285H}}
\end{tabular}
\end{center}
ہدایت \sSTA کا ہدایتی رمز \زیرعمل{32H} ہے۔ یہ بائٹ پہلے دستیاب پتہ ، \زیرعمل{2006H} ، پر رکھا  گیا ہے۔ اس ہدایت  میں دیا گیا پتہ (\زیرعمل{6285H}) دو بائٹ لمبا ہے۔ زیریں بائٹ \زیرعمل{85H} اگلے پتہ  (\زیرعمل{2007H}) پر، اور بالا بائٹ \زیرعمل{62H}  اس سے اگلے پتے  (\زیرعمل{2008H}) پر رکھا گیا ہے۔
 
 پتہ  بظاہر الٹ کیوں رکھا گیا  (یعنی زیریں بائٹ  کے بعد بالا بائٹ )؟   اولین  \عددی{8080} میں ایسا کیا گیا۔ اس  (اولین)  خرد عمل کار کے ساتھ ہم آہنگی کی بنا  پر \عددی{8085}  اور  دیگر خرد عمل کار میں  یہی طریقہ اختیار کیا گیا۔ یوں   زیریں بائٹ زیریں پتے پر، اور بالا بائٹ بالا پتے پر رکھا جاتا ہے۔
 
 آخری ہدایت \رک  ہے جس کا ہدایتی رمز \زیرعمل{76H}  پتہ \زیرعمل{2009H} پر رکھا گیا ہے۔
 
 آپ نے دیکھا کہ \sMVI ہدایت \عددی{2} بائٹ،  \sSTA ہدایت \عددی{3} بائٹ،  اور  \sHLT ہدایت \عددی{1} بائٹ ہے۔
\انتہا{مثال}
%-------------------------------

\حصہ{دفتری ہدایات}  
 ہدایتی پھیرے کے دوران  راجع ہدایات  ایک سے زیادہ مرتبہ حافظہ سے رجوع کرتی ہیں، لہٰذا یہ ہدایات نسبتاً سست رفتار ہیں۔مزید،  کئی مرتبہ ہم چاہتے ہیں کہ حافظہ سے گزرے بغیر ایک دفتر سے مواد دوسرے دفتر منتقل ہو۔ آئیں کمپیوٹر با کی ایسی\عددی{2} بائٹ  ہدایات پر غور کریں جو کم سے کم وقت میں ایک دفتر سے دوسرے دفتر مواد منتقل کرتی ہیں۔
 
 \جزوحصہ{\sMOV}
 ہدایت \sMOV کو  \قول{لاد} پڑھیں (جیسا  گھوڑے پر بوجھ لادنا)۔ یہ کمپیوٹر سے کہتی ہے کہ ایک دفتر سے مواد دوسرے دفتر منتقل کرے۔ مثال کے طور پر،
 \begin{align*}
 \MOV{\regA}{\regB}
\end{align*}
کمپیوٹر سے کہتی ہے کہ دفتر  \دفترب سے مواد دفتر  \دفترالف منتقل کریں۔ یہ عمل غیر  تباہ کن ہے، یعنی دفتر \دفترب کا مواد  نقل ہو گا لیکن یہ مواد دفتر \دفترب میں بھی رہے گا۔ مثلاً، درج ذیل صورت میں
 \begin{align*}
\kop{9DH}&=\regB  &&\kop{34H}=\regA
\end{align*}
ہدایت \MOV{\regA}{\regB} کی تعمیل کے بعد نتائج درج ذیل ہوں گے۔
\begin{align*}
\kop{9DH}&=\regA\\
\kop{9DH}&=\regB
\end{align*}

آپ دفاتر \دفترالف، \دفترب، اور \دفترج کے بیچ مواد  کا انتقال کر سکتے ہیں۔ ان ہدایات کی شکل و صورت درج ذیل ہے۔
\begin{center}
\begin{tabular}{r}
\MOV{\regA}{\regB}\\
\MOV{\regA}{\regC}\\
\MOV{\regB}{\regA}\\
\MOV{\regB}{\regC}\\
\MOV{\regC}{\regA}\\
\MOV{\regC}{\regB}
\end{tabular}
\end{center}
یہ  کمپیوٹر با کی تیز ترین  ہدایات ہیں جنہیں محض ایک مشینی پھیرا درکار ہے۔

\جزوحصہ{\sADD اور \sSUB}
ہدایت \sADD کہتی ہے دفتر \دفترالف کے ساتھ دیے گئے دفتر کا مواد جمع کر کے نتیجہ دفتر \دفترالف میں ڈال۔ مثلاً،
\begin{center}
\begin{tabular}{r}
\ADD{\regB}
\end{tabular}
\end{center}
کمپیوٹر سے کہتی ہے دفتر \دفترب کا مواد دفتر \دفترالف کے مواد کے ساتھ جمع کر۔ یوں  اگر اس ہدایت  کی تعمیل سے قبل  ان دفاتر میں  درج ذیل ہو:
\begin{align*}
\kop{02H}&=\regB &&\kop{04H}=\regA 
\end{align*}
تب \ADD{\regB} کی تعمیل کے بعد ان دفاتر میں درج ذیل ہو گا۔
\begin{align*}
\kop{02H}&=\regB &&\kop{06H}=\regA 
\end{align*}
دفتر \دفترالف میں نتیجہ جبکہ دفتر \دفترب اپنا مواد برقرار رکھتا ہے۔

اسی طرح \sSUB کہتی ہے دیے گئے دفتر کا مواد دفتر \دفترالف سے منفی کر کے دفتر \دفترالف میں نتیجہ رکھ۔ دیے گئے دفتر کا مواد تبدیل نہیں ہو گا۔ \SUB{\regC} دفتر \دفترج کا مواد دفتر \دفترالف کے مواد سے منفی کر کے نتیجہ دفتر \دفترالف میں رکھے گی۔

ہدایات \sADD اور \sSUB کی مختلف شکل و صورتیں درج ذیل ہیں۔
\begin{center}
\begin{tabular}{r}
\ADD{\regB}\\
\ADD{\regC}\\
\SUB{\regB}\\
\SUB{\regC}
\end{tabular}
\end{center}

\جزوحصہء{\sINR اور \sDCR}
بعض اوقات ہم  دفتر کا مواد  بڑھانا یا گھٹانا چاہتے ہیں۔بڑھوتری کے لئے ہدایت \sINR ہے؛ یہ کمپیوٹر سے کہتی ہے ، دیے گئے دفتر کے مواد میں \عددی{1} کا اضافہ کر۔ دفتر کے مواد میں  کمی لانے کی ہدایت \sDCR ہے، جو دیے گئے دفتر کے مواد میں \عددی{1} کی کمی پیدا کرتی ہے۔ ان ہدایات کی مختلف اشکال درج ذیل ہیں۔
\begin{center}
\begin{tabular}{r}
\INR{\regA}\\
\INR{\regB}\\
\INR{\regC}\\
\DCR{\regA}\\
\DCR{\regB}\\
\DCR{\regC}
\end{tabular}
\end{center}
یوں اگر  دفاتر میں
\begin{center}
\begin{tabular}{r}
\regB = \kop{56H}		% a space on both sides of = sign solve the flipping issue
\end{tabular}\quad\quad
\begin{tabular}{r}
\regC = \kop{8AH}
\end{tabular}
\end{center}
ہو تب  \INR{\regB} کی تعمیل کے بعد  
\begin{center}
\begin{tabular}{r}
\regB = \kop{57H}		% a space on both sides of = sign solve the flipping issue
\end{tabular}
\end{center}
اور \DCR{\regC} کی تعمیل کے بعد درج ذیل ہو گا۔
\begin{center}
\begin{tabular}{r}
\regC = \kop{89H}
\end{tabular}
\end{center}

%----------------------
\ابتدا{مثال}
اعشاری \عددی{23} اور \عددی{45} جمع کرنے کی ہدایت لکھیں۔ نتیجہ حافظہ میں مقام \عددی{5600H} پر رکھیں۔ نتیجے  میں \عددی{1} کا اضافہ کر کے جواب دفتر \دفترج میں ڈالیں۔

حل:\quad
اعشاری \عددی{23} اور \عددی{45} کو  سادس  عشری  میں لکھنا ہو گا جو بالترتیب \عددی{17H} اور \عددی{2DH} ہیں۔ درج ذیل برنامہ اس کا م کو سرانجام دے سکتا ہے۔
\begin{center}
\begin{tabular}{r}
\MVI{\regA}{17H}\\
\MVI{\regB}{2DH}\\
\ADD{\regB}\\
\STA{5600H}\\
\INR{\regA}\\
\MOV{\regC}{\regA}\\
\HLT
\end{tabular}
\end{center}
\انتہا{مثال}
%---------------------------
\ابتدا{مثال}
\اصطلاح{ماخذ برنامے}\فرہنگ{برنامہ!ماخذ}\حاشیہب{source program}\فرہنگ{program!source} کا  مشینی زبان میں  ترجمہ عموماً کمپیوٹر کے مخصوص برنامے کی مدد سے کیا جاتا ہے جسے  \اصطلاح{مترجم برنامہ }\فرہنگ{برنامہ!مترجم} یا مختصراً   \اصطلاح{مترجم }\فرہنگ{مترجم}\حاشیہب{assembler}\فرہنگ{assembler} کہتے ہیں۔یہی کام دستی بھی کیا جا سکتا ہے۔درج بالا ماخذ برنامے کا   \اصطلاح{ دستی ترجمہ}\فرہنگ{دستی!ترجمہ}  مشینی زبان میں کریں۔

حل:\quad
\begin{center}
\begin{tabular}{ccr}
\toprule
پتہ& مواد& علامتی روپ\\[0.5ex]
2000H&3EH&\MVI{\regA}{17H}\\
2001H&17H&\\
2002H&06H&\MVI{\regB}{2DH}\\
2003H&2DH&\\
2004H&80H&\ADD{\regB}\\
2005H&32H&\STA{5600H}\\
2006H&00H&\\
2007H&56H&\\
2008H&3CH&\INR{\regA}\\
2009H&4FH&\MOV{\regC}{\regA}\\
200AH&76H&\HLT\\
\bottomrule
\end{tabular}
\end{center}
یاد رہے،  \sADD، \sINR، \sMOV، اور \sHLT ہدایات \عددی{1} بائٹ ہیں؛ \sMVI ہدایات \عددی{2} بائٹ، اور \sSTA ہدایت \عددی{3} بائٹ ہے۔
\انتہا{مثال}
%-------------------------
%show a tree with two branches with code on both. on the side show the code location in memory
\حصہ{شاخ اور طلبی  ہدایات}
کمپیوٹر با  کی چار ہدایات ایسی ہیں جو برنامے کی ترتیب  تبدیل کر سکتی ہیں۔دوسرے لفظوں میں، ہمیشہ کی طرح اگلی ہدایت بازیاب کرنے کی بجائے ، کمپیوٹر  برنامے   کے دوسرے حصے  پہنچ کر وہاں  سے اگلی ہدایت بازیاب کر تا ہے۔ہم کہتے ہیں کمپیوٹر دوسری  \اصطلاح{شاخ }\فرہنگ{شاخ}\حاشیہب{branch}\فرہنگ{branch}  پر چل پڑتا ہے۔

فرض کریں آپ چاہتے ہیں کہ  دفتر \دفترالف میں صفر \عددی{0} ہونے کی صورت میں ایک کام جبکہ اس میں غیر صفر ہونے کی صورت میں دوسرا کام سرانجام ہو۔ جس نقطہ پر کمپیوٹر کو ایسا  فیصلہ کرنا ہو وہاں سے برنامے کی دو شاخیں نکلیں گی۔ کمپیوٹر  کو فیصلہ کرنا ہو گا کہ وہ کس\قول{ شاخ } پر چلے۔

\جزوحصہء{\sJMP}
نئی شاخ پر چلنے کی ایک ہدایت \sJMP ہے؛ یہ کمپیوٹر کو اگلی ہدایت دئے گئے  پتے  سے بازیاب کرنے کو کہتی ہے۔\sJMP ہدایت کے ساتھ پتہ ہو گا جو برنامہ گنت کار میں ڈال دیا جاتا ہے۔ مثال کے طور پر،
\begin{center}
\begin{tabular}{c}
\JMP{\kop{3000H}}
\end{tabular}
\end{center}
کمپیوٹر  کو  اگلی ہدایت حافظہ کے مقام  \عددی{3000H} سے بازیاب  کرنے کو کہتی ہے۔

آئیں اس عمل  پر  غور کریں۔ فرض کریں، \JMP{3000H} مقام \عددی{2005H} پر  موجود ہے (شکل \حوالہء{11.3a} دیکھیں)۔ بازیابی پھیرے کے اختتام پر، برنامہ گنت کار میں درج ذیل ہو گا۔
\begin{center}
\begin{tabular}{c}
\regPC = \kop{2006H}
\end{tabular}
\end{center}
تعمیلی پھیرے کے دوران ، \JMP{3000H} برنامہ گنت کار میں مطلوبہ پتہ ڈالتی ہے۔
\begin{center}
\begin{tabular}{c}
\regPC = \kop{3000H}
\end{tabular}
\end{center}
اگلا  بازیابی پھیرا، اگلی ہدایت \عددی{2006H} کی بجائے \عددی{3000H} سے پڑھے گا (شکل \حوالہء{11.3a} دیکھیں)۔

\جزوحصہء{\sJM}
کمپیوٹر با  میں دو جھنڈے ہیں جنہیں \موٹا{جھنڈا علامت } اور  \موٹا{جھنڈا صفر } کہتے ہیں۔ بعض ہدایات کی تعمیل کے  دوران،دفتر \دفترالف کے مواد کو دیکھتے ہوئے  یہ جھنڈے  بلند یا پست ہوں گے۔ دفتر \دفترالف کے مواد   کی   علامت  منفی   \عددی{(-)}ہونے کی صورت میں   جھنڈا علامت بلند ہو گا؛ دیگر صورت یہ جھنڈا پست ہو گا۔ علامتی طور پر درج ذیل  لکھا جائے گا، جہاں \عددی{S} جھنڈا علامت کو ظاہر کرتا ہے۔
\begin{align*}
S=\begin{cases}
0 & A\ge 0\\
1 & A<0
\end{cases}
\end{align*}
جھنڈا علامت   اس وقت تک بلند یا پست  رہے گا جب تک کوئی دوسری ہدایت (جو اس جھنڈے کو تبدیل کر سکتی ہو)  اسے تبدیل نہ کرے۔

ہدایت \sJM کہتی ہے  ، \قول{منفی صورت میں شاخ}   (منفی کی صورت میں نئی شاخ ہر چل)؛  کمپیوٹر  نامزد پتے پر صرف اس صورت پہنچے گا جب جھنڈا علامت بلند ہو۔مثال کے طور پر، فرض کریں \JM{3000H} حافظہ میں  \عددی{2005H} پر موجود ہو۔اس ہدایت کی بازیابی  کے بعد درج ذیل ہو گا۔
\begin{center}
\begin{tabular}{c}
\regPC = \kop{2006H}
\end{tabular}
\end{center}
اگر \عددی{S=1} ہو، \JM{3000H} کی تعمیل برنامہ گنت کار میں \عددی{3000H} ڈالے گی۔
\begin{center}
\begin{tabular}{c}
\regPC = \kop{3000H}
\end{tabular}
\end{center}
چونکہ برنامہ گنت کار اب \عددی{3000H}    پر نظر جمائے ہوئے ہے لہٰذا اگلی ہدایت حافظہ سے مقام  \عددی{3000H} سے پڑھی جائے گی۔

اس کے برعکس، اگر \عددی{S=0} ہو، نئی شاخ پر چلنے  کا جواز موجود نہیں ہو گا،   لہٰذا برنامہ گنت کار کا مواد تبدیل نہیں  ہو  گا اور   اگلے بازیابی پھیرا  میں ہدایت \عددی{2006H} سے پڑھی جائے گی۔

شکل \حوالہء{11.3b}  میں   دونوں صورتوں کی وضاحت کی گئی ہے۔ اگر منفی کی شرط   مطمئن ہو، کمپیوٹر اگلی ہدایت کے لئے \عددی{3000H}  کو شاخ کرے گا۔ اگر منفی شرط مطمئن  نہ ہو، کمپیوٹر شاخ کئے\اصطلاح{ سیدھا گزر کر }\فرہنگ{سیدھا گزرنا}\حاشیہب{fall through}\فرہنگ{fall through}  بغیر اگلی ہدایت اٹھائے گا۔

\جزوحصہء{\sJZ}
دوسرا جھنڈا جو دفتر \دفترالف کے مواد سے  متاثر  ہو\قول{ جھنڈا صفر } ہے۔ بعض ہدایات کی تعمیل  پر دفتر \دفترالف کا مواد صفر \عددی{(0)}  ہو گا۔ اس واقع کو جھنڈا صفر  بلند ہو کر  یاد رکھتا ہے؛ اگر دفتر \دفترالف کا مواد  صفر  نہ ہو یہ جھنڈا پست ہو گا۔ علامتی طور پر درج ذیل ہو گا، جہاں \عددی{Z} جھنڈا صفر کو ظاہر کرتا ہے۔
\begin{align*}
Z=\begin{cases}
0 & A\ne 0\\
1 & A=0
\end{cases}
\end{align*}

ہدایت \sJZ کہتی ہے،  \قول{صفر کی صورت میں شاخ } (اگر دفتر  \دفترالف میں صفر ہو ، اگلی ہدایت کے لئے شاخ کر)؛ کمپیوٹر نئی  شاخ پر صرف اس صورت چلے گا جب دفتر \دفترالف کا مواد صفر کے برابر ہو۔ فرض کریں، \JZ{3000H} حافظہ میں مقام \عددی{2005H} پر موجود ہو۔ اس ہدایت کی تعمیل کے دوران اگر \عددی{Z=1} ہو، اگلی ہدایت \عددی{3000H} سے اٹھائی جائے گی۔  اس کے برعکس، اگر \عددی{Z=0} ہو، اگلی ہدایت \عددی{2006H} سے پڑھی جائے گی۔

\جزوحصہء{\sJNZ}
ہدایت \sJNZ کہتی ہے،\قول{ غیر صفر  صورت میں شاخ }۔یوں شاخ اس صورت ہو گی جب جھنڈا صفر پست ہو؛ بلند جھنڈے کی صورت میں شاخ نہیں  کی جائے گی۔ فرض کریں \JNZ{7800H} مقام \عددی{2100H} ہے۔ اگر \عددی{Z=0} ہو، اگلی ہدایت \عددی{7800H} سے اٹھائی جائے گی؛ تاہم \عددی{Z=1} کی صورت میں کمپیوٹر شاخ نہیں کرتا اور اگلی ہدایت \عددی{2101H} سے اٹھائی جائے گی۔

ہدایات \sJM، \sJZ، اور \sJNZ کو  \اصطلاح{مشروط شاخ }\فرہنگ{شاخ!مشروط}\حاشیہب{conditional jumps}\فرہنگ{jump!conditional} کہتے ہیں۔ کمپیوٹر صرف اس صورت شاخ کرتا ہے جب کوئی مخصوص شرط مطمئن ہو۔ اس کے برعکس، \sJMP \اصطلاح{غیر مشروط  }\فرہنگ{شاخ!غیر مشروط}\حاشیہب{unconditional jump}\فرہنگ{jump!unconditional} ہے؛  اس ہدایت کی بازیابی کے بعد کمپیوٹر لازماً شاخ کر کے   دئے گئے  پتے پر پہنچے گا۔

\جزوحصہء{\sCALL اور \sRET}
\اصطلاح{ذیلی   معمولہ  }\فرہنگ{ذیلی معمولہ}\حاشیہب{subroutine}\فرہنگ{subroutine} سے مراد ایسا برنامہ ہے جو حافظہ میں اس مقصد سے رکھا جاتا ہے کہ کوئی دوسرا برنامہ اسے استعمال  کر سکے۔ سائن، کوسائن، ٹینجنٹ، لوگارتھم، جذر، وغیرہ معلوم کرنے کے لئے  کئی خرد کمپیوٹر کے ذیلی معمولہ  موجود ہیں۔  یہ ذیلی معمولے   صارف کو کمپیوٹر  کے ساتھ فراہم کیے جاتے ہیں۔

\قول{ذیلی معمولہ  طلب کرنے } کی ہدایت \sCALL ہے۔ مطلوبہ ذیلی معمولہ کا ابتدائی پتہ \sCALL ہدایت کے ساتھ فراہم کیا جاتا ہے۔مثال کے طور پر، اگر جذر  کا ذیلی معمولہ پتہ  \عددی{5000H} سے  اور لوگارتھم کا ذیلی معمولہ \عددی{6000H} سے  آغاز کرتا ہو، درج ذیل کی تعمیل
\begin{center}
\begin{tabular}{c}
\CALL{5000H}
\end{tabular}
\end{center}
جذر ذیلی معمولہ کو شاخ کرے گا (ہم کہتے ہیں اختیار جذر ذیلی معمولہ کو دیا جائے گا)۔ اس کے برعکس ، 
\begin{center}
\begin{tabular}{c}
\CALL{6000H}
\end{tabular}
\end{center}
لوگارتھم کے  ذیلی معمولہ کو شاخ کرے گا۔

ہدایت \sRET سے مراد واپس  \قول{لوٹنا} ہے۔ ہر ذیلی معمولے  کا اختتام اس ہدایت پر ہو گا ، جو کمپیوٹر کو برنامے میں اس مقام پر واپس پہنچنے کو کہتی ہے جہاں سے  ذیلی معمولہ طلب کیا گیا۔ہر ذیلی معمولہ کے اختتام پر اس ہدایت کو شامل کرنا مت  بھولیں، ورنہ کمپیوٹر ذیلی معمولے کے اختتام پر پہنچ کر واپس جانے کی بجائے اگلے مقام سے ہدایت اٹھا کر  بے قابو ہو گا۔

کمپیوٹر با میں \sCALL کی تعمیل پر برنامہ گنت کار کا مواد  ( اگلی ہدایت کا پتہ)  حافظہ  کے آخری دو مقامات  \عددی{FFFEH} اور \عددی{FFFFH} پر  خود بہ خود    رکھ دیا جاتا ہے۔ اس کے بعد \sCALL میں دیا گیا پتہ برنامہ گنت کار میں ڈالا جاتا ہے، تا کہ ذیلی معمولہ کی پہلی ہدایت  اٹھائی جائے۔ ذیلی معمولہ  کے اختتام پر \sRET ہدایت  ہو گی، جو \عددی{FFFEH} اور \عددی{FFFFH}  پر رکھا گیا  پتہ برنامہ گنت کار میں ڈالتی ہے۔یوں اصل برنامے  کو اختیار   لوٹایا جاتا ہے۔ 

شکل \حوالہء{11.4}  میں  ذیلی معمولے  کے دوران برنامے کا چلن پیش ہے۔ \CALL{5000H} ہدایت  کمپیوٹر کو \عددی{5000H} پر موجود ذیلی معمولے  پر بھیجتی ہے۔اس ذیلی معمولہ کے اختتام پر \sRET کمپیوٹر کو  \sCALL کے بعد آنے والی ہدایت پر بھیجتی ہے۔

ہدایت \sJMP کی طرح \sCALL غیر مشروط ہے۔ہدایتی دفتر میں  \sCALL  ہدایت  پہنچنے پر کمپیوٹر لازماً ذیلی معمولے کی پہلی ہدایت کو شاخ کرے گا۔

\جزوحصہء{جھنڈوں پر مزید معلومات}
علامت اور صفر جھنڈا  بعض ہدایات کے دوران بلند یا پست ہو سکتے ہیں۔ جدول \حوالہ{شکل_کمپیوٹر_با_جھنڈوں_اثر_انداز}   میں ان ہدایات کی فہرست پیش ہے جو   جھنڈوں  کو متاثر کر سکتے ہیں۔یہ  ہدایات تعمیلی پھیرے کے دوران دفتر \دفترالف استعمال کرتی ہیں۔اگر ان ہدایات  میں سے کسی ایک کی تعمیل کے دوران دفتر \دفترالف کا مواد صفر یا منفی ہو، جھنڈا صفر یا جھنڈا علامت  بلند  ہو گا۔

مثلاً،فرض کریں  ہدایت \ADD{\regC} کی تعمیل جاری ہے۔ دفتر \دفترج کا مواد دفتر \دفترالف کے مواد کے ساتھ جمع ہو کر دفتر \دفترالف میں ڈالا جائے گا۔ اگر دفتر \دفترالف کا مواد صفر ہو، جھنڈا صفر بلند ہو گا (جبکہ جھنڈا علامت پست ہو گا)؛ اگر دفتر \دفترالف کا مواد منفی ہو، جھنڈا علامت بلند ہو گا (جبکہ جھنڈا صفر پست ہو گا)۔  اگر دفتر \دفترالف کا مواد مثبت ہو، دونوں جھنڈے پست ہوں گے۔
\begin{table}
\caption{جھنڈوں پر اثر انداز ہونے والی ہدایات۔}
\label{شکل_کمپیوٹر_با_جھنڈوں_اثر_انداز}
\begin{tabular}{rc}
\toprule
ہدایت&متاثر  جھنڈے \\
\midrule
\sADD&S ، \,Z\\	%keep the spaces before and after the comma as they are else flips unexpectedly
\sSUB&S ، \,Z\\
\sINR&S ، \,Z\\
\sDCR&S ، \,Z\\
\sANA&S ، \,Z\\
\sORA&S ، \,Z\\
\sXRA&S ، \,Z\\
\sANI&S ، \,Z\\
\sORI&S ، \,Z\\
\sXRI&S ، \,Z\\
\bottomrule
\end{tabular}
\end{table}

اب \sINR اور \sDCR ہدایات پر نظر ڈالتے ہیں۔ چونکہ یہ ہدایات دفتر \دفترالف کے ساتھ \عددی{1} جمع کرتے ہیں یا اس سے \عددی{1} منفی کرتے ہیں لہٰذا یہ ہدایات بھی دونوں جھنڈوں پر اثر انداز ہوں گی۔ مثال کے طور پر، \DCR{\regC} کی تعمیل میں،  دفتر \دفترج کا مواد دفتر \دفترالف بھیج کر اس سے \عددی{1}  منفی کر کے نتیجہ   (دفتر \دفترالف کا مواد) واپس دفتر \دفترج بھیجا جاتا ہے۔  اگر \sDCR کی تعمیل کے دوران دفتر \دفترالف کا مواد صفر  ہو، جھنڈا صفر بلند ہو گا؛ اگر دفتر \دفترالف کا مواد منفی ہو، جھنڈا علامت بلند ہو گا۔

%----------------------------------
%ex 11.5
\ابتدا{مثال}\شناخت{مثال_کمپیوٹر_با_دائرہ}
درج ذیل برنامے کا  دستی ترجمہ  مشینی زبان میں کریں۔ پتہ \عددی{2000H} سے آغاز کریں۔
\begin{center}
\begin{tabular}{r}
\MVI{\regC}{\kop{03H}}\\
\DCR{\regC}\\
\JZ{\kop{0009H}}\\
\JMP{\kop{0002H}}\\
\HLT
\end{tabular}
\end{center}
حل:\quad
\begin{center}
\begin{tabular}{ccr}
\toprule
پتہ&مواد&علامتی روپ\\[0.5ex]
2000H&0EH&\MVI{\regC}{\kop{03H}}\\
2001H&03H&\\
2002H&0DH&\DCR{\regC}\\\
2003H&CAH&\JZ{\kop{2009H}}\\
2004H&09H&\\
2005H&20H&\\
2006H&C3H&\JMP{\kop{2002H}}\\
2007H&02H&\\
2008H&20H&\\
2009H&76H&\HLT\\
\bottomrule
\end{tabular}
\end{center}
\انتہا{مثال}
%-------------------------
%ex 11.6
\ابتدا{مثال}
درج بالا برنامہ میں \sDCR ہدایت کی تعمیل  کتنی مرتبہ  ہو گی؟

حل:\quad
شکل \حوالہء{11.5} میں برنامے کا بہاو دکھایا گیا ہے۔ \MVI{\regC}{03H} ہدایت دفتر \دفترج میں \عددی{03H} ڈالتی ہے۔ \DCR{\regC}  اس مواد  کو گھٹا کر \عددی{02H} کرتی ہے۔ یہ صفر سے زیادہ ہے؛ لہٰذا جھنڈا صفر پست ہو گا، اور \JZ{2009H}  ہدایت نظرانداز ہو گی۔ \JMP{2002H} ہدایت کمپیوٹر کو واپس  \DCR{\regC}ہدایت پر بھیجتی ہے۔

ہدایت \DCR{\regC} کی تعمیل دوسری مرتبہ کرنے سے  مواد گھٹ کر \عددی{01H} ہو جائے گا؛ جھنڈا صفر اب بھی پست  ہو  گا، اور \JZ{2009H} نظرانداز ہو گی، اور \JMP{2002H} کمپیوٹر کو واپس \DCR{\regC} پر بھیجے گی۔

تیسری مرتبہ \DCR{\regC} کی تعمیل مواد کو صفر کرتی ہے لہٰذا  جھنڈا  صفر بلند ہو گا، اور  \JZ{2009H} کمپیوٹر کو  \HLT ہدایت پر بھیجے گی۔

برنامے کا وہ حصہ جو دہرایا جائے\اصطلاح{ دائرہ }فرہنگ{دائرہ}\حاشیہب{loop}\فرہنگ{loop} کہلاتا ہے۔جیسا شکل \حوالہء{11.5} میں دکھایا گیا ہے،  اس مثال میں  ہم دائرہ (\DCR{\regC} اور \JZ{2009H}) سے تین مرتبہ گرتے ہیں۔ آپ دیکھ سکتے ہیں کہ دائرے سے گزرنے کی تعداد اور  دفتر \دفترج کی ابتدائی قیمت برابر ہے۔ اگر ہم پہلی ہدایت کو تبدیل کر کے درج ذیل کر دیں
\begin{center}
\begin{tabular}{c}
\MVI{\regC}{\kop{07H}}
\end{tabular}
\end{center}
کمپیوٹر اس دائرے سے \عددی{7} مرتبہ گزرے گا۔ اسی طرح اگر ہم چاہتے ہوں کہ  دائرے سے \عددی{200} مرتبہ   (جو \عددی{C8H} کے برابر ہے)گزرا جائے ، پہلی ہدایت درج ذیل ہو گی۔
\begin{center}
\begin{tabular}{c}
\MVI{\regC}{\kop{C8H}}
\end{tabular}
\end{center}
دفتر  \دفترج بطور قابل  پیش قیمت بھرائی گنت کار کردار ادا کرتا ہے۔ اسی لئے بعض اوقات  ہم اسے \قول{گنت کار } کہتے ہیں۔

جو نقطہ یاد رکھنے کے قابل ہے، وہ یہ ہے۔ ہم \sMVI، \sDCR، \sJZ، اور \sJMP استعمال کر کے دائرہ  پیدا دے سکتے ہیں۔  نامزد دفتر (جو بطور گنتکار کام کرے گا)  میں وہ عدد ڈالا جائے گا جتنی مرتبہ دائرے سے گزرنا مقصود ہو۔ اس دائرے میں  جو جو  ہدایات ڈالی جائیں ،  ان تمام کی تعمیل اتنی مرتبہ ہو گی  جو عدد گنتکار دفتر میں ابتدائی طور ڈالا گیا ہو۔
\انتہا{مثال}
%-------------------
%example 11.7
\ابتدا{مثال}
کمپیوٹر خریدتے وقت  آپ اس کا   \اصطلاح{نرم افزار }\فرہنگ{نرم افزار}\حاشیہب{software}\فرہنگ{software} (سافٹ وئیر) بھی خریدیں گے۔ ایک برنامہ جو آپ خرید سکتے ہیں\اصطلاح{  مترجم   } ہے۔ آپ علامتی روپ میں برنامہ لکھ کر  مترجم کی مدد اس کا ترجمہ  مشینی زبان میں  کرتے ہیں۔ دوسرے لفظوں میں، اگر آپ کے پاس مترجم ہو، آپ کو دستی ترجمہ کرنے کی ضرورت  نہیں  ہو گی؛ کمپیوٹر آپ کے لئے کام کرے گا۔

 مثال \حوالہ{مثال_کمپیوٹر_با_دائرہ} میں  دیا گیا برنامہ   مادری زبان کے روپ میں لکھیں۔ \اصطلاح{سرخی }\فرہنگ{سرخی}\حاشیہب{labels}\فرہنگ{labels} اور \اصطلاح{تبصرہ }\فرہنگ{تبصرہ}\حاشیہب{comments}\فرہنگ{comments}شامل کریں۔
 
 حل:\quad
 \begin{center}
\begin{tabular}{rrr}
\toprule
سرخی&\multicolumn{1}{c}{ہدایت}&\multicolumn{1}{c}{تبصرہ}\\[1ex]
&\MVI{\regC}{\kop{03H}}&؛گنتکار میں اعشاری \عددی{3} ڈالیں\\
دوبارہ: & \DCR{\regC} &؛گنتکار گھٹائیں\\ 
&\JZ{\text{\RL{اختتام}}}& ؛صفر  کے لئے پرکھیں\\
&\JMP{دوبارہ}& ؛دائرے سے دوبارہ گزریں \\
اختتام:&\HLT&
\end{tabular}
\end{center}

برنامہ لکھتے وقت\قول{ تبصرہ } شامل کرنا سودمند ثابت ہوتا ہے۔ اس تبصرے میں آپ اپنا مقصد بیان کرتے ہیں جو  بعض اوقات کمپیوٹر کی ہدایت دیکھ کر واضح نہیں ہو گا۔ کئی مہینوں یا کئی برس  بعد یہ برنامہ پڑھتے ہوئے یہ تبصرے آپ کو اپنا لکھا ہوا برنامہ سمجھنے میں مدد دیں گے۔ پہلا تبصرہ ہمیں یاد دلاتا ہے کہ ہم دفتر \دفترج کو بطور گنتکار استعمال کرتے ہوئے دائرے سے تین مرتبہ گزرنا چاہتے ہیں۔ دوسرا تبصرہ کہتا ہے  کہ ایک مرتبہ دائرے سے گزرنے پر  گنتکار  کی گنتی کم کی جاتی ہے۔ تیسرا تبصرہ کہتا ہے کہ ہم جھنڈا صفر کو دیکھ کر شاخ  لیں گے۔ چوتھا تبصرہ کہتا ہے کہ دائرے سے دوبارہ گزریں۔

مشینی زبان میں ترجمہ کرتے ہوئے، وقف ناقص (؛)  اور  اس لکیر پر اس کے بعد جو کچھ ہو ، کو مترجم  نظر انداز کرتا ہے۔  کیوں؟ وجہ یہ ہے کہ مترجم برنامے اسی طرح لکھے جاتے ہیں۔ وقف ناقص کمپیوٹر کو بتاتا  ہے کہ جو کچھ  آگے لکھا گیا ہے، برنامہ نویس کے ذاتی استعمال  اور یاداشت کے لئے ہے۔ 

شاخ اور طلبی  کے ساتھ\قول{ سرخی  } کا استعمال مددگار ثابت ہوتا ہے۔ کمپیوٹر کی مادری زبان میں برنامہ لکھتے وقت  ہم عموماً نہیں جانتے کہ شاخ یا طلبی ہدایت کے ساتھ کیا پتہ شامل کریں۔اعدادی پتے کی بجائے سرخی استعمال کرنے سے  برنامے کا بہاو  سمجھنا  زیادہ  آسان ہو گا۔ مترجم ان سرخیوں کو دیکھتے ہوئے شاخ اور طلبی  ہدایات میں  درست پتے شامل کرتا ہے۔

مثال کے طور پر، درج بالا برنامے کو مشینی زبان میں  لکھتے ہوئے مترجم  \sJZ کی جگہ اس کا ہدایتی رمز  \عددی{CA} (جدول \حوالہ{شکل_کمپیوٹر_ہدایتی_رمز} سے رجوع کریں)  اور  \قول{اختتام}  کی جگہ \sHLT ہدایت کا پتہ ڈالے گا۔ اسی طرح \sJMP کی جگہ مترجم  ہدایتی رمز  \عددی{C3}  اور  \قول{دوبارہ}  کی جگہ   ہدایت \DCR{\regC} کا پتہ ڈالے گا۔ مترجم تمام ہدایات کو درکار بائٹ گن کر    مشینی برنامہ میں   \sHLT اور \sJMP ہدایات کے پتے جان پاتا ہے۔

آپ کو صرف اتنا یاد رکھنا ہو گا کہ شاخ اور طلبی ہدایات کے ساتھ استعمال کے لئے  آپ کوئی بھی  سرخی  استعمال کر سکتے ہیں۔ اسی  سرخی کے آخر میں  \عددی{:} چسپاں کر کے  اس ہدایت کے آگے لکھیں جس پر آپ شاخ کرنا  چاہتے ہیں۔ جب مترجم آپ کے برنامے کو پڑھتا ہے یہ نشان \عددی{(:)} مترجم کو خبردار کرتا ہے کہ اس جگہ سرخی مستعمل ہے۔

کمپیوٹر با میں  سرخی کے لئے ایک  تا  چھ    علامت (حرف یا ہندسے ) استعمال کیے جا سکتے ہیں، تاہم پہلی علامت کا  لازماً ایک  حرف  ہونا ہو گا۔ سرخی عموماً معنی خیز الفاظ ہوں گے،  تاہم ہندسوں کا استعمال جائز ہے۔ جائز سرخیوں کی مثال درج ذیل ہے۔
 \begin{center}
\begin{tabular}{r}
دوبارہ\\ 
یہاں\\
تختپڑھ\\
ب4053\\
ج34م22
\end{tabular}
\end{center}
پہلی دو سرخیاں  عام الفاظ ہیں؛ تیسری سرخی  \قول{تختی پڑھ}   کہنا چاہتی ہے؛ چوتھی اور پانچویں  سرخیاں بے معنی   ہیں، تاہم ان کا استعمال جائز ہے۔ سرخی کی لمبائی پر چھ علامتوں کی  پابندی اور پہلی علامت    پر  حرف ہونے کی   پابندی  ، عام  دستیاب مترجم    بھی عائد کرتے ہیں۔
\انتہا{مثال}
%----------------------------
%example 11.8
\ابتدا{مثال} 
ایسا برنامہ لکھیں جو عشری \عددی{12} اور \عددی{8} آپس میں جمع کرے۔

حل:\quad
 \begin{center}
\begin{tabular}{rrr}
\toprule
سرخی&\multicolumn{1}{c}{ہدایت}&\multicolumn{1}{c}{تبصرہ}\\[1ex]
&\MVI{\regA}{\kop{00H}}& ؛ دفتر \دفترالف صاف کریں\\
&\MVI{\regB}{\kop{0CH}}& ؛ دفتر \دفترب میں اعشاری \عددی{12} ڈالیں\\
&\MVI{\regC}{\kop{08C}}& ؛ گنتکار کو \عددی{8} پر رکھیں\\
دوبارہ: & \ADD{\regB}& ؛ اعشاری \عددی{12} جمع کریں\\
&\DCR{\regC}& ؛گنتکار گھٹائیں\\
&\JZ{ہوگیا}& ؛صفر کے لئے پرکھیں\\
 &\JMP{دوبارہ} & ؛ دوبارہ دائرے سے گزریں\\
 ہوگیا: & \HLT & ؛ کمپیوٹر روک دیں
\end{tabular}
\end{center}
برنامے میں کیا گیا تبصرہ ہمیں کم و بیش پوری کہانی بتا پاتا ہے۔ سب سے پہلے ہم دفتر \دفترالف کو صاف کرتے ہیں۔ اس کے بعد  عشری \عددی{12} دفتر \دفترب میں ڈالا جاتا ہے۔اس کے بعد    گنت کار میں \عددی{8}   ڈال کر تیار کیا  جاتا ہے۔ مذکورہ بالا تین  ہدایات،  دائرے میں داخل ہونے سے قبل،   ابتدائی  حالت تعین کرتے ہیں۔

دائرے کا آغاز \ADD{\regB} کرتی ہے جو دفتر \دفترالف کے ساتھ عشری \عددی{12} جمع کرتی ہے۔ گنتکار کی گنتی \DCR{\regC} گھٹا کر \عددی{7} کرتی ہے۔ جھنڈا صفر پست ہونے کی بدولت اس مرتبہ  \JZ{ہوگیا} نظر انداز ہو گا اور  کمپیوٹر سیدھا آگے بڑھتے ہوئے \JMP{دوبارہ} کی تعمیل کر کے \ADD{\regB} پہنچے گا۔

چونکہ  \ADD{\regB} دائرے کے اندر پایا جاتا ہے لہٰذا اس کی تعمیل \عددی{8} مرتبہ ہو گی اور یوں دفتر \دفترالف (جو آغاز میں خالی تھا) کے ساتھ \عددی{8}  مرتبہ \عددی{12} جمع ہو گا۔ یہی \عددی{8} اور \عددی{12}  ضرب کرنے سے حاصل ہو گا۔ دائرے کے \عددی{8} چکر کاٹنے کے  بعد گنتکار میں \عددی{0} ہو گا، لہٰذا  جھنڈا صفر بلند ہو  گا؛ یوں \JZ{ہوگیا} کی تعمیل ہو گی اور کمپیوٹر دائرے سے نکل کر \HLT کو شاخ کرے گا۔

چونکہ  عشری \عددی{12} کو \عددی{8} مرتبہ جمع کیا گیا لہٰذا دفتر \دفترالف میں ا درج ذیل ہو گا۔
\begin{align*}
12+12+12+12+12+12+12+12=96
\end{align*}
عشری \عددی{96} سادس عشری   \عددی{60} کے برابر ہے لہٰذا دفتر \دفترالف میں ثنائی  \عددی{01100000} ہو گا۔یوں بار بار جمع کرنا ضرب دینے کے مترادف ہے۔ دوسرے لفظوں میں  آٹھ مرتبہ \عددی{12} اور \عددی{12\times 8} برابر ہیں۔ 

آپ گنت کار میں عشری \عددی{12} اور  دفتر \دفترب میں \عددی{8}  ڈال کر بھی   ان اعداد کو ضرب کر سکتے ہیں۔

زیادہ تر خرد عمل کاروں    میں ضرب   کرنے کا  \اصطلاح{ سخت  افزار }\فرہنگ{سخت افزار}\حاشیہب{hardware}\فرہنگ{hardware} نہیں پایا جاتا؛ ان میں، کمپیوٹر الف کی طرح ،  صرف   جمع و منفی کار   ہو گا۔ یوں، عموماً خرد عمل کار استعمال کرتے ہوئے ضرب کرنے کی خاطر آپ کو کسی قسم کا  برنامہ  (مثلاً بار بار جمع کرنے کا برنامہ) لکھنا ہو گا۔
\انتہا{مثال}
%------------------------------------
%example 11.9
\ابتدا{مثال}
درج بالا برنامہ تبدیل  کر کے \sJZ کی جگہ \sJNZ ہدایت استعمال کریں۔

حل:\quad
 \begin{center}
\begin{tabular}{rrr}
\toprule
سرخی&\multicolumn{1}{c}{ہدایت}&\multicolumn{1}{c}{تبصرہ}\\[1ex]
&\MVI{\regA}{\kop{00H}}& ؛ دفتر \دفترالف صاف کریں\\
&\MVI{\regB}{\kop{0CH}}& ؛ دفتر \دفترب میں اعشاری \عددی{12} ڈالیں\\
&\MVI{\regC}{\kop{08C}}& ؛ گنتکار کو \عددی{8} پر رکھیں\\
دوبارہ: & \ADD{\regB}& ؛ اعشاری \عددی{12} جمع کریں\\
&\DCR{\regC}& ؛گنتکار گھٹائیں\\
&\JNZ{دوبارہ}& ؛صفر کے لئے پرکھیں\\
& \HLT & ؛ کمپیوٹر روک دیں
\end{tabular}
\end{center}

یہ برنامہ نسبتاً سادہ ہے۔ اس میں ایک \sJMP ہدایت اور ایک سرخی کم ہیں۔ جب تک گنتکار  صفر  سے بڑا ہو، \sJNZ کمپیوٹر کو واپس\قول{ دوبارہ } پر بھیجے گی۔ جب گنتکار صفر ہو جائے، برنامہ \sJNZ سے سیدھا گزر کر \sHLT تک پہنچے گا۔
\انتہا{مثال}
%-------------------------
%ex 11.10
\ابتدا{مثال}
درج بالا کا ترجمہ مشینی زبان میں دستی کریں۔ ابتدائی پتہ \عددی{2000H} رکھیں۔

حل:\quad
 \begin{center}
\begin{tabular}{rrr}
\toprule
پتہ&\multicolumn{1}{c}{مواد}&\multicolumn{1}{c}{علامتی روپ}\\[1ex]
2000H&3EH&\MVI{\regA}{00H}\\
2001H&00H&\\
2002H&06H&\MVI{\regB}{0CH}\\
2003H&0CH&\\
2004H&0EH&\MVI{\regC}{08H}\\
2005H&08H&\\
2006H&80H&\ADD{\regB}\\
2007H&0DH&\DCR{\regC}\\
2008H&C2H&\JNZ{2006H}\\
2009H&06H&\\
200AH&20H&\\
200BH&76H&\HLT
\end{tabular}
\end{center}

اولین تین ہدایات، ضرب شروع ہونے سے قبل ،دفاتر کی  ابتدائی حالت تعین کرتی ہیں۔ ابتدائی حالت تبدیل کرنے سے  ہم دیگر اعداد آپس میں ضرب کر سکتے ہیں۔
\انتہا{مثال}
%--------------------------
%ex 11.11
\ابتدا{مثال}
درج بالا برنامے میں ضرب کرنے والے حصے  کو ذیلی معمولہ  میں تبدیل کر کے پتہ \عددی{F006H} پر رکھیں۔

حل:\quad
 \begin{center}
\begin{tabular}{rrr}
\toprule
پتہ&\multicolumn{1}{c}{مواد}&\multicolumn{1}{c}{علامتی روپ}\\[1ex]

F006H&80H&\ADD{\regB}\\
F007H&0DH&\DCR{\regC}\\
F008H&C2H&\JNZ{F006H}\\
F009H&06H&\\
F00AH&F0H&\\
F00BH&C9H&\RET
\end{tabular}
\end{center}

اس طرح سوچیں:  ابتدائی حالت تعین کرنے والی ہدایات کا ضرب  دینے کے عمل سے کوئی تعلق نہیں۔ یہ صرف  ان اعداد سے تعلق رکھتی ہیں جنہیں ضرب کرنا مقصود ہو۔ ذیلی معمولہ صرف اس  حصے پر مشتمل ہو گا جس کا ضرب کرنے سے تعلق ہو۔

برنامے کو نئی جگہ منتقل کرتے ہوئے  ہم نے \عددی{2006H} تا \عددی{200BH} پتوں کو \عددی{F006H} تا \عددی{F00BH}   پر نقش کیا۔ ساتھ ہی \sHLT کی جگہ \sRET استعمال کیا، تا کہ اصل برنامے کو اختیار منتقل کرنا ممکن ہو۔
\انتہا{مثال}
%------------------------------
%ex 11.12
\ابتدا{مثال}
درج بالا ضرب کار ذیلی معمولہ  درج ذیل برنامے میں مستعمل ہے۔ یہ برنامہ کیا کرتا ہے؟
 \begin{center}
\begin{tabular}{r}
\MVI{\regA}{00H}\\
\MVI{\regB}{10H}\\
\MVI{\regC}{0EH}\\
\CALL{F006H}\\
\HLT
\end{tabular}
\end{center}
حل:\quad
سادس عشری \عددی{10H} اعشاری \عددی{16} کے برابر، اور سادس عشری \عددی{0EH} اعشاری \عددی{14} کے برابر ہے۔  اولین تین ہدایات  دفتر \دفترالف کو صاف کرتی ہے، دفتر \دفترب میں عشری \عددی{16}، اور دفتر \دفترج میں عشری \عددی{14} ڈالتی ہے۔ \sCALL ہدایت( گزشتہ مثال میں دیے  گئے ) ضرب کار  ذیلی معمولہ  کو طلب کرتی ہے۔  ضرب کے اختتام پر \sRET  کی تعمیل کے وقت دفتر \دفترالف میں \عددی{E0H} ہو گا جو عشری \عددی{224} کے برابر ہے، جو مطلوبہ جواب ہے۔

\اصطلاح{مقدار معلوم }\فرہنگ{مقدار معلوم}\حاشیہب{parameter}\فرہنگ{parameter}    اس معلومات کو کہتے ہیں جس کی بنا ذیلی معمولہ صحیح کام کرنے سے قاصر ہو گا۔پتہ  \عددی{F006H} پر رکھے  گئے ضرب کار ذیلی معمولہ  کو، صحیح کام کرنے کے لئے، تین مقدار معلوم  (\regA، \regB، \regC) درکار ہیں۔ دفتر \دفترالف کو صاف کر کے، دفتر  \دفترب میں  مضروب، اور دفتر \دفترج میں  ضارب ڈال کر ہم   یہ مقدار معلوم ذیلی معمولہ کو  مہیا کرتے ہیں۔ دوسرے لفظوں میں ہم \عددی{A=00H}، \عددی{B=10H}، اور \عددی{C=0EH} رکھ کر ذیلی معمولہ کو طلب کرتے ہیں۔ ذیلی معمولہ کو معلومات  دفاتر کے ذریعہ فراہم کرنے کو\قول{   دفتری مقدار معلوم کی فراہمی } کہتے ہیں۔
\انتہا{مثال}
%----------------------------------

\حصہ{منطقی ہدایات}

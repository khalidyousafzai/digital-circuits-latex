\باب{ ثنائی نظام}

\حصہ{اعشاری نظامِ گنتی}

عام زندگی میں اعشاری نظامِ گنتی  استعمال ہوتا ہے،  جو \عددی{0}  تا \عددی{9}  کے ہندسوں پر مبنی ہے۔کسی بھی گنتی کے نظام میں کل علامات کی تعداد کو اس نظام کی  اساس  کہتے ہیں۔اعشاری نظام   میں \عددی{0} تا \عددی{9}،  یعنی دس \عددی{10} علامات ہیں، یوں اعشاری نظام کی  اساس دس ہے اور اس  کو اساس  \عددی{10} کا نظام کہتے ہیں۔

	مساوات \حوالہ{مساوات_ثنائی_عدد}  میں \عددی{538.72}   کو  اعشاری نظام میں لکھتے ہوئے زیر نوشت  میں \عددی{10}   لکھا گیا ہے، جو  اس بات کی یاد دہانی کراتا ہے کہ یہ عدد  اساس دس کے نظام میں لکھا گیا ہے۔اس کتاب میں چونکہ کئی نظامِ گنتی استعمال ہوں گے،  لہٰذا جہاں متن سے واضح نہ ہو وہاں اعداد کے ساتھ ان کی  اساس زیر نوشت میں لکھی  جائے گا۔
\begin{align}\label{مساوات_ثنائی_عدد}
538.72_{10}
\end{align}
اس نظام میں اعشاریہ کی بائیں جانب پہلا ہندسہ اکائی وزن رکھتا ہے، دوسرا دہائی، تیسرا سینکڑا،  وغیرہ۔یوں مساوات  \حوالہ{مساوات_ثنائی_سینکڑا}  میں  دئے گئے ہندسوں میں \عددی{8}    کا 
مطلب \عددی{8\times 10^0=8\times 1=8_{10}}    ہے،  جبکہ  \عددی{3}     کا مطلب \عددی{3\times 10^1=30_{10}}     اور \عددی{5}    کا \عددی{5\times 10^2=500_{10}}     ہے۔اسی طرح اعشاریہ کے دائیں جانب پہلے  ہندسے   کا وزن ایک بٹہ دس ہے، دوسرے  ہندسے  کا ایک بٹہ سو،  اور تیسرے ہندسے کا ایک بٹہ ہزار،  وغیرہ۔یوں  اس عدد میں \عددی{7}   دراصل \عددی{7\times 10^{-1}=0.7_{10}}   جبکہ \عددی{2}   دراصل  \عددی{2\times 10^{-2}=0.02_{10}}  ہے۔
\begin{align}\label{مساوات_ثنائی_سینکڑا}
538.72_{10}=(5\times 10^2)+(3\times 10^1)+(8\times 10^0)+(7\times 10^{-1})+(2\times 10^{-2})
\end{align}

اس حقیقت  کو عمومی طور پر درج ذیل  لکھ سکتے ہیں۔
\begin{gather}
\begin{aligned}
\cdots+a_2\times 10^2+&a_1\times 10^1+a_0\times 10^0+a_{-1}\times 10^{-1}+a_{-2}\times 10^{-2}+\cdots\\
&=(\cdots a_2a_1a_0.a_{-1}a_{-2}\cdots)_{10}
\end{aligned}
\end{gather}
%?????KKKKK am here
	اگر عدد \عددی{538.72_{10}}  کو  \عددی{x} کا نام دیا جائے تو شکل 1.1  میں اس عدد کے مختلف ہندسوں کو پکارنے کا طریقہ دکھایا گیا ہے جس کے تحت  \عددی{5} کو \عددی{x_2}  جبکہ \عددی{3}   کو \عددی{x_3} کہیں گے، وغیرہ وغیرہ۔

	اس طرح کسی بھی عدد میں بائیں جانب والے ہندسے کا رتبہ 4 دائیں جانب والے ہندسے سے بلند ہوتا ہے۔مساوات 1.1 میں بلند تر رتبہ والا ہندسہ 5 ہے جبکہکم تر رتبہ والا ہندسہ 6 ہے۔
	مساوات 1.4 میں سات کو تین مختلف طریقوں سے لکھا گیا ہے۔عام زندگی میں سات کو مساوات کی پہلی طرز پر لکھا جاتا ہے۔یوں کاغذ پر لکھتے ہوئے کسی بھی عدد کے بائیں جانب صفر نہیں لکھے جاتے اور اس جانب کاغذ کو خالی چھوڑا جاتا ہے۔یہاں یہ بات سمجھنا ضروری ہے کہ عام زندگی میں اعداد لکھتے وقت ان کی لمبائی یا ان میں کُل ہندسوں کی تعداد پہلے سے متعین نہیں کی جاتی۔کمپیوٹر میں چیزیں قدرِ مختلف ہیں جہاں صرف صفر اور ایک کا وجود ممکن ہے۔کسی مقام پر اگرنہیں لکھا ہو تو اس پرلکھا ہوتا ہے۔یوں کسی بھی عدد کے بائیں جانب خالی جگہ کا کمپیوٹر میں کوئی مطلب نہیں۔یہاں یا تو اور یا پھرکا ہونا ضروری ہے۔کمپیوٹر میں ہر قسم کی معلومات لکھنے سے پہلے اس بات کا فیصلہ کیا جاتا ہے کہ اسے لکھنے کی خاطر کتنی جگہ  درکار ہو گی۔یوں اگر کسی عدد کو لکھنے کی خاطر تین ہندسوں کے لکھے جانے کے برابر جگہ تعین کی گئی ہو تو اس تمام جگہ کو ہر صورت استعمال کیا جائے گا او یوں صرف  لکھنے کی بجائے اسےلکھا جائے گا۔

 
(1.4)

	اعشاری نظام میں گنتی   سے شروع ہوتی ہے اور بتدریج بڑھتے ہوئے  تک پہنچتی ہے۔ اس دوران دہائی، سینکڑا وغیرہ کے مقام پر صفر رہتا ہے اور انہیں عام طور نہیں لکھا جاتا۔گنتیتک پہنچنے کے بعد دہائی یعنی  وزن رکھنے والے مقام پر  کی بجائےلکھا جاتا ہے اور اکائی یعنی  وزن رکھنے والے مقام پر دوبارہسے کی جانب گنتی شروع ہوتا ہے۔ 
	اگر آپ کو اس پیراگراف کی سمجھ نہیں آئی تو اسے دوبارہ پڑھیں۔اس میں سادہ گنتی کی وضاحت کی گئی ہے۔ 
	اعشاری نظام میں اگر اعداد کو ایک ہندسے تک محدود کر دیا جائے تو اس میں  سے   تک گنتی ممکن ہو گی۔اگر اعداد کو دو ہندسوں تک محدود کر دیا جائے، یعنی اس میں زیادہ سے زیادہ دو ہندسے ہوں، تو  سے   تک گنتی ممکن ہو گی، اسی طرح  تین ہندسوں تک کے عدد استعمال کرنے سے   سے تک گنتی کی جا سکتی ہے وغیرہ۔ 
1.2 ہشتمی نظامِ گنتی
	ہشتمی نظام میںتا کے کُل آٹھ ہندسے ہوتے ہیں ۔اس  نظام میں آٹھ ہندسے ہونے کی وجہ سے یہ اساس-آٹھ  کا نظام ہے۔بالکل اعشاری نظام کی طرح، اس نظام میں اعداد لکھتے ہوئے اعشاریہ کے بائیں جانب پہلے ہندسے  کا وزن  ہے، دوسرے ہندسے کا  ، تیسرے کا  وغیرہ جبکہ اعشاریہ کے دائیں جانب پہلے کسری ہندسے کا وزن ہے، دوسرے  کسری ہندسے کا وزن ہو گا وغیرہ۔

 
(1.5)

	مساوات 1.13 کو ہشتمی نظامِ گنتی کے لئے یوں لکھ سکتے ہیں۔

 
(1.6)

	مساوات 1.5 میں ہشتمی نظام میں دئے گئے عدد کو اعشاری نظام میں تبدیل کرنا دکھایا گیا ہے۔ہشتمی عدد کی دائیں جانب نیچے کر کے چھوٹی لکھائی میں اس بات کی یاد دہانی کرتا ہے کہ یہ عدد ہشتمی نظام میں لکھا گیا ہے۔ 
	اس نظام میں گنتی  سے شروع ہوتی ہے۔ تک پہنچنے کے  بعد   وزن رکھنے والے مقام پر  کی بجائے  لکھا جاتا ہے اور وزن رکھنے والے مقام پر دوبارہ  سے   کی جانب گنتی شروع ہوتی ہے۔
1.3 ثنائی نظامِ گنتی
	مائکرو کنٹرولر کی دنیا میں ثنائی نظامِ گنتی استعمال ہوتا ہے۔ثنائی نظام میں صرف دو ہندسے یعنیاوراستعمال ہوتے ہیں۔یوں یہ نظام اساس-دو کا نظامِ گنتی ہے۔
	اس نظام میں گنتیسے شروع ہوتی ہے۔ تک پہنچنے کے بعد   وزن رکھنے  والی مقام پرکی بجائےلکھا جاتا ہے اوروزن رکھنے والی مقام پر دوبارہسےکی جانب گنتی شروع ہوتی ہے۔اس نظام میں گنتی مساوات 1.7 میں دکھائی گئی ہے۔موازنہ کے لئے اعشاری گنتی بھی دی گئی ہے۔

 
(1.7)

	اس نظام میں اعداد لکھتے ہوئے اعشاریہ کے بائیں جانب پہلے ہندسے کا وزن   ہوتا ہے  دوسرے  ہندسے کا   ، تیسرے کا   وغیرہ جبکہ اعشاریہ کے دائیں جانب پہلے ہندسے کا وزن  ،دوسرے ہندسے کا وزنوغیرہ ہوتا ہے۔
	مساوات 1.3 کو ثنائی نظامِ گنتی کے لئے یوں لکھ سکتے ہیں۔

 
(1.8)

	مساوات 1.9 میں ثنائی نظام میں دئے گئے عدد کو اعشاری نظام میں تبدیل کرنا دکھایا گیا ہے۔ ثنائی عدد کی دائیں جانب نیچے کر کے چھوٹی لکھائی میںاس بات کی یاد دہانی کراتا ہے کہ یہ عدد ثنائی نظام میں لکھا گیا ہے۔ 

 
(1.9)

	شکل 1.2 میں کسی بھی ثنائی عدد کے ہندسوں کو پکارنے کا طریقہ دکھایا گیا ہے۔یوں شکل میں سب سے دائیں جانب ہندسے کو کم تر رتبہ والا بِٹ  7 یا کم تر رتبہ والا ثنائی ہندسہ یا بِٹ صفر یا بِٹکہیں گے ۔ اس سے اگلے کو بِٹ ایک یا بِٹ اور اس سے اگلے کو بِٹ دو یعنی بِٹ وغیرہ جبکہ سب سے بائیں جانب ہندسے کو بلند تر رتبہ والا ثنائی ہندسہ 8 یا بلند تر رتبہ والا بِٹ یا بِٹ سات یا بِٹکہیں گے۔


	اگر دئے گئے عدد میں اعشاریہ کے دائیں جانب کچھ نہ ہو تب اس عدد کو یوں بھی اعشاری نظام میں تبدیل کیا جا سکتا ہے۔

 
(1.10)

دئے گئے عدد میں جہاں جہاں ہندسوں کی قیمت  ہے وہاں کے وزن جمع کر دئے گئے ہیں۔
	چار ہندسوں کا عددسےتک کی گنتی کے لئے استعمال ہو سکتا ہے۔اگر اس سے بڑا عدد لکھنا ہو تو چار سے زیادہ ہندسے استعمال کرنا ضروری ہو گا۔مائکرو کنٹرولر آٹھ ثنائی ہندسوں کے اعداد استعمال کرتا ہے۔ آٹھ ہندسوں کو استعمال کرتے سے  تک کے اعداد ظاہر کئے جا سکتے ہیں۔
	عام زندگی میں اعشاری نظامِ گنتی استعمال کرتے ہوئے اعداد لکھتے وقت ان کے بائیں جانب صفر نہیں لکھے جاتے یعنیکونہیں لکھا جاتا۔کمپیوٹر کی دنیا میں اعداد عموماً آٹھ ہندسوں پر مبنی ثنائی عدد کی صورت میں لکھے جاتے ہیں۔آٹھ سے کم ثنائی ہندسوں پر مبنی اعداد لکھتے وقت ان کے بائیں جانب صفریں لکھ کر انہیں آٹھ ہندسوں کی شکل میں لکھا جاتا ہے۔یوںکو کی بجائے لکھا جاتا ہے۔
1.4 اعشاری نظام سے ثنائی نظام میں تبادلہ

	اعشاری نظام میں دئے گئے عدد کو ثنائی نظام میں لکھنے کی خاطر اس کے عدد کو بار بار 2 سے تقسیم کریں حتٰی کہ یہ مزید تقسیم نہ ہو سکے۔ہر مرتبہ تقسیم کے بعد حاصل باقی ہوگا۔پہلے حاصل باقی کو ثنائی عدد کی سب سے کم وزن والے مقام پر لکھیں۔اگلے حاصل باقی کو اس سے دگنے وزن کے مقام پر لکھیں۔اسی طرح آخری حاصل باقی کو عدد کے سب سے زیادہ وزن کے مقام پر لکھیں۔یوں حاصل شدہ عدد دئے گئے عدد کی ثنائی لکھائی ہوگی۔
	یہ طریقہ استعمال کرتے ہوئے کو ثنائی لکھائی میں لکھتے ہیں۔
	
121 تقسیم 2=60+بقایا 1
60 تقسیم 2 =30+بقایا 0
30 تقسیم 2=15+بقایا 0
15 تقسیم 2=7 +بقایا 1
7 تقسیم 2=3+بقایا 1
3 تقسیم 2=1+بقایا 1
1 تقسیم 1=0+بقایا 1

اب سب سے آخری بقایا کو سب سے زیادہ وزن کی مقام پر اور سب سے پہلے بقایا کو سب سے کم وزن کے مقام پر لکھتے ہیں۔یوں حاصل ہوتا ہے ۔
	لہٰذا
 

 ثنائی نظام سے اس عدد کو واپس اعشاری نظام میں منتقل کر کے ہم یہ یقین دہانی کر سکتے ہیں کہ یہی اصل جواب ہے۔ایسا کرتے ہوئے



اس طریقہ کار کو عموماً یوں لکھا جاتا ہے


	کسی بھی عدد کے اعشاریہ کے بائیں جانب والے حصہ کو حصہ صحیح 9 کہتے ہیں جبکہ دائیں جانب والے حصہ کو حصہ مسکور 10  یا کسری کہتے ہیں۔یعنی



مثلاًمیں  کو عدد صحیح اور کو عدد مسکور کہیں گے۔اس طرح کے اعشاری نظام میں دئے گئے عدد کے صحیح حصہ کو ثنائی نظام میں تبدیل کرنے کیلئے اسے اوپر دی گئی مثال کی طرح ہی حل کیا جاتا ہے البتہ حصہ مسکور کو مختلف طریقہ سے تبدیل کیا جاتا ہے۔دونوں حصوں کے جوابات کو آخر میں ایک ساتھ لکھ لیا جاتا ہے۔
	حصہ مسکور کو یوں حل کیا جاتا ہے۔حصہ مسکور کو بار بار  سے ضرب دیں۔اگر حاصلِ ضرب میں اعشاریہ کے بائیں جانب حاصل ہو تو اس  کو حاصلِ ضرب سے ہٹا کر اسے ثنائی عدد کے دائیں جانب منسلک کر دیں اور اگر حاصلِ ضرب میں اعشاریہ کے بائیں جانب  حاصل ہو تب ثنائی عدد کے دائیں جانب  منسلک کر دیں۔یہ عمل مثال سے دکھلاتے ہیں۔



یوںاور اب دونوں جواب جمع کر کے


کے برابر ہے۔

1.5 اساس سولہ  (سادس عشری) کا نظامِ گنتی

	اساس سولہ کے نظام میں اعداد کے سولہ علامتیں ہیں۔ان میں پہلی دس علامتیں تاہیں اور بقایا بڑی لکھائی میں انگریزی حروفِ تہجی کے پہلے چہ حروف یعنی   ہیں۔ ان میںدس کو ظاہر کرتا ہے یعنیجبکہگیارہ کو یعنیاورپندرہ کو ظاہر کرتا ہے۔مساوات 1.11 میں مختلف نظامِ گنتی آمنے سامنے لکھے دکھائے گئے ہیں۔ان پر غور کریں اور انہیں اچھی طرح سمجھیں۔انہیں بغیر سمجھے آگے مت بڑھیں۔  

 
(1.11)

	اس نظام میں اعداد لکھتے ہوئے دائیں جانب سے پہلے ہندسے کا وزن   ہے دوسرے ہندسے کا  ، تیسرے کا  وغیرہ وغیرہ۔ 


 
(1.12)

مساوات 1.12 میں سادس عشری یا اساس سولہ کے نظام میں دئے گئے عدد کو اعشاری نظام میں تبدیل کرنا دکھایا گیا ہے۔ایسا کرتے وقت اور لئے گئے ہیں۔
	مساوات 1.3 کو اساس-سولہ کے لئے یوں لکھ سکتے ہیں۔

 
(1.13)

1.6 اساس-دو کا اساس-آٹھ میں تبادلہ
	مساوات 1.14 میں    کا ثنائی عدد بائیں جانب دیا گیا ہے۔اس ثنائی عدد کو اساس آٹھ میں لکھنے کی خاطر پہلے اس کو اعشاریہ سے شروع کرتے ہوئے اعشاریہ کے دونوں جانب تین تین ہندسوں کے گروہ میں لکھیں۔اعشاریہ کے بائیں جانب اگر آخر میں تین ہندسوں کا گروہ پورا نہ ہو تو عدد کے بائیں جانب صفریں لگا کر تین ہندسوں کا گروہ  پورا کریں۔اسی طرح اعشاریہ کے دائیں جانب اگر آخر میں تین ہندسوں کا گروہ پورا نہ ہو تو عدد کے دائیں جانب صفریں لگا کر تین ہندسوں کا گروہ پورا کریں۔اب مساوات 1.11 کی مدد سے ان تین تین کے گروہ کی جگہ ان کا مساوی اساس آٹھ کا ہندسہ لکھیں۔مساوات 1.14 میں یوں دائیں جانب سے  دو مقام پر کی جگہ لکھا گیا ہے،   کی جگہ   اور   کی جگہ   لکھا گیا ہے۔یوں یہ عدد اساس آٹھ میں   لکھا جائے گا۔یاد رہے کہ ایسا کرتے وقت اعشاریہ اپنی جگہ برقرار رکھتا ہے۔

 
(1.14)

1.7 اساس-دو کا اساس-سولہ میں تبادلہ
	ثنائی عدد کو اساس سولہ میں لکھنے کی خاطر ثنائی عدد کو اعشاریہ سے شروع کرتے ہوئے اعشاریہ کے دونوں جانب  چار چار ہندسوں کے گروہ میں لکھیں۔اگر اعشاریہ کے بائیں جانب آخر میں چار ہندسوں کا گروہ پورا نہ ہو تو عدد کے  بائیں جانب صفریں لگا کر چار ہندسے پورا کریں۔اسی طرح اگر اعشاریہ کے دائیں جانب آخر میں چار ہندسے پورے نہ ہوں تو عدد کے دائیں جانب صفر جوڑ کر چار ہندسے پورا کریں۔اب مساوات 1.11  کی مدد سے ان چار چار کے گروہ کی جگہ ان کا مساوی اساس سولہ کا ہندسہ لکھیں۔مساوات 1.15 میں یوں دائیں جانب سےکی جگہلکھا گیا ہے، کی جگہ   لکھا گیا ہے اور   کی جگہ   لکھا گیا ہے۔یوں یہ عدد اساس سولہ میں   لکھا جائے گا۔ یہ سب کرتے وقت اعشاریہ اپنی جگہ برقرار رکھتا ہے۔
 
 
(1.15)

1.8 اساس-آٹھ اور اساس-سولہ سے اساس-دو میں تبادلہ
	انہیں طریقوں کو الٹ استعمال کرتے ہوئے اساس آٹھ اور اساس سولہ کے اعداد با آسانی اساس-دو میں لکھے جا سکتے ہیں۔مساوات 1.16 میں اساس آٹھ اور مساوات 1.17 میں اساس سولہ کو ثنائی عدد کی شکل میں لکھنا دکھایا گیا ہے۔ 
 
(1.16)



 
(1.17)

	مساوات 1.16 اور 1.17 کی آخری لکیروں میں ثنائی اعداد کو دیکھتے ہوئے بہت جلد انسان اکتا جاتا ہے البتہ انہیں مساوات میں جہاں ان اعداد کو گروہ کی شکل میں لکھا گیا ہے وہاں انہیں سمجھنا ممکن ہے۔ 
	ایک ہندسے پر مبنی ثنائی عدد کو ثنائی ہندسہ یا بِٹ 11 کہتے ہیں۔ثنائی اعداد کو جب  آٹھ ثنائی ہندسوں یعنی آٹھ بِٹ کے گروہ میں لکھا جائے  تو اسے ایک ہشتمی ثنائی عدد یا ایک بائٹ 12 کہتے ہیں۔بائٹ کو  عموماً دو چار چار ثنائی اعداد کی گروہ میں لکھا جاتا ہے۔یوں مساوات 1.17 میں دو بائٹ ہیں۔اسی مساوات کو الٹ چلاتے ہوئے یہ واضح ہے کہ ہشتمی ثنائی عدد کو چار-چار ثنائی اعداد کے  گروہ میں لکھ کر انہیں جلد اساس سولہ میں لکھا جا سکتا ہے۔

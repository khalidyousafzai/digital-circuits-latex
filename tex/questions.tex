% https://onlinetoolz.net/truth-tables#sym=ma2&exp1=A%20and%20(B%20or%20not%20C)
%Karnaugh questions pending

\باب{سوالات}
%Q1.1
\ابتدا{سوال}
درج ذیل اعشاری اعداد کو ثنائی روپ میں لکھیں۔
\begin{multicols}{4}
\begin{enumerate}[a.]

\item

 \(33\)

\item

 \(64\)

\item

 \(128\) 

\item

 \(256\)

\item

 \(4096\)

\item

 \(0.375\)

\item

 \(5.625\) 

\item

 \(13.6875\)
\end{enumerate}
\end{multicols}
جواب: \عددی{100001}، \عددی{1000000}، \عددی{10000000}، \عددی{100000000}، \عددی{1000000000000}، \عددی{0.011}، \عددی{101.101}، \عددی{1101.1011}
\انتہا{سوال}
\ابتدا{سوال}
%Q1.2
درج ذیل ثنائی اعداد کو اعشاری روپ میں لکھیں۔
\begin{multicols}{4}
\begin{enumerate}[a.]

\item  
 
 \(10\)  

\item  
 
 \(101\) 

\item  
 
 \(1101\)  

\item  
 
 \(11011\)

\item  
 
 \(101101011\) 

\item  
 
 \(11001010011\)  
\end{enumerate}
\end{multicols}
جواب: \عددی{2}، \عددی{5}، \عددی{13}، \عددی{27}، \عددی{363}، \عددی{1619}
\انتہا{سوال}
\ابتدا{سوال}
%Q1.3
درج ذیل ثنائی اعداد کو اعشاری روپ میں لکھیں۔
\begin{multicols}{4}
\begin{enumerate}[a.]

\item  
 
 \(10.1\)  
\item  
 
 \(101.01\)

\item  
 
 \(0.001101\) 
\item  
 
 \(1011.01101\)

\item  
 
 \(100.001\) 
\item  
 
 \(1111.1111\)
\end{enumerate}
\end{multicols}
جواب: \عددی{2.5}،   \عددی{5.25}،   \عددی{0.203125}،   \عددی{11.40625}،   \عددی{4.125}،   \عددی{15.9375}
\انتہا{سوال}
\ابتدا{سوال}
%Q1.4
درج ذیل اعشاری اعداد کو اساس سولہ اور اساس آٹھ میں تبدیل کریں۔
\begin{multicols}{4}
\begin{enumerate}[a.]

\item 
 \(7\) 
\item 
 \(23\) 
\item 
 \(32\) 
\item 
 \(64\)  

\item 
 \(1024\) 
\item 
 \(2048\)
\end{enumerate}
\end{multicols}
جواب:  اساس سولہ \عددی{7}، \عددی{17}، \عددی{20}، \عددی{40}، \عددی{400}، \عددی{800}؛ اساس آٹھ   \عددی{7}، \عددی{27}، \عددی{40}، \عددی{100}، \عددی{2000}، \عددی{4000}
\انتہا{سوال}
\ابتدا{سوال}
%Q1.5
درج ذیل اساس سولہ اعداد کو اساس آٹھ اور ثنائی روپ میں لکھیں۔
\begin{multicols}{4}
\begin{enumerate}[a.]

\item 
 \(7\) 
\item 
 \(10\) 
\item 
 \(1A\) 
\item 
 \(2B3\) 

\item 
 \(A.BC\) 
\item 
 \(0.12\) 
\item 
 \(F0\) 
\item 
 \(FFFF\)
\end{enumerate}
\end{multicols} 
جواب:اساس سولہ   \عددی{7}، \عددی{20}، \عددی{32}، \عددی{1263}، \عددی{12.57}، \عددی{0.044}، \عددی{360}، \عددی{177777}؛ ثنائی  \عددی{111}، \عددی{10000}، \عددی{11010}، \عددی{1010110011}، \عددی{1010.101111}، \عددی{0.0001001}، \عددی{11110000}، \عددی{1111111111111111}
\انتہا{سوال}
%=======================
%Q2.1
\ابتدا{سوال}
درج ذیل ثنائی مجموعے حاصل کریں۔ان سوالات کو اعشاری روپ میں بھی حل کریں۔جوابات کا موازنہ کریں۔
\begin{multicols}{4}
\begin{enumerate}[a.]

\item  
 \(110+101\)  
\item 
 \(11+101\) 

\item  
 \(1011+1101\)  
\item 
 \(1101+1001\)   

\item  
 \(101+1011\)  
\item 
 \(101+1111\)  
\end{enumerate}
\end{multicols}
جواب:ثنائی \عددی{1011}، \عددی{1000}، \عددی{11000}، \عددی{10110}، \عددی{10000}، \عددی{10100}؛ 
اعشاری   \عددی{11}، \عددی{8}، \عددی{24}، \عددی{22}، \عددی{16}، \عددی{20}
\انتہا{سوال}
\ابتدا{سوال}
%Q2.2
درج ذیل ثنائی اعداد کے سوالات حل کریں۔ان سوالات کو اعشاری روپ میں بھی حل کریں۔جوابات کا موازنہ کریں۔
\begin{multicols}{4}
\begin{enumerate}[a.]

\item  
 \(110-101\)  
\item 
 \(111-101\) 

\item  
 \(1111-1101\)  
\item 
 \(1101-1001\)   

\item  
 \(101-1011\)  
\item 
 \(101-1111\)
\end{enumerate}
\end{multicols}
جواب: ثنائی  \عددی{1}، \عددی{10}، \عددی{10}،  \عددی{100}، \عددی{-110}، \عددی{-1010}؛
 اعشاری  \عددی{1}، \عددی{2}، \عددی{2}، \عددی{4}،  \عددی{-6}، \عددی{-10}
\انتہا{سوال}
\ابتدا{سوال}
%Q2.3
درج ذیل ثنائی اعداد کے سوالات حل کریں۔انہیں سوالات کو اعشاری روپ میں بھی حل کریں۔جوابات کا موازنہ کریں۔
\begin{multicols}{4}
\begin{enumerate}[a.]
\item  
 \(110-10.1\)  
\item 
 \(101-10.1\) 

\item  
 \(11.11-1.101\)  
\item 
 \(110.1-10.01\)   

\item  
 \(101.011-10.11\) 
\item 
 \(111.1-11.01\)
\end{enumerate}
\end{multicols}
جواب: ثنائی  \عددی{11.1}، \عددی{10.1}، \عددی{10.001}، \عددی{100.01}، \عددی{10.101}، \عددی{100.01}
\انتہا{سوال}
\ابتدا{سوال}
%Q2.4
درج ذیل اعشاری سوالات کو ثنائی روپ میں تبدیل کر کے حل کریں۔
\begin{multicols}{4}
\begin{enumerate}[a.]

\item  
 \(64+32\)  
\item  
 \(256-128\) 
\item  
 \(121.2-94.3\) 
\item  
 \(36.09+22.24\)  
\item  
 \(1024-63\) 
\item  
 \(2056+1024\)    
\end{enumerate}
\end{multicols}
جواب:  \عددی{1100000}، \عددی{10000000}، \عددی{11010.1110}، \عددی{111010.010}، \عددی{1111000001}، \عددی{110000001000}
\انتہا{سوال}
\ابتدا{سوال}
%Q2.5
درج ذیل اعشاری اعداد کا تکملہ نو اور تکملہ دس حاصل کریں۔
\begin{multicols}{4}
\begin{enumerate}[a.]
\item  
 \(6\)  
\item   
 \(8\) 
\item  
 \(19\)  
\item   
 \(205\) 
\item  
 \(3160029\) 
\item   
 \(9807568\) 
\item  
 \(0.63\)  
\item   
 \(39.09\) 
\item  
 \(3093.9801\) 
\item   
 \(23409.65487\) 
\end{enumerate}
\end{multicols}
جواب:  تکملات نو  \عددی{3}، \عددی{1}، \عددی{80}، \عددی{794}، \عددی{6839970}، \عددی{0192431}؛ تکملات دس  \عددی{4}، \عددی{2}، \عددی{81}، \عددی{795}، \عددی{6839971}، \عددی{0192432}
\انتہا{سوال}
\ابتدا{سوال}
%Q2.6
درج ذیل ثنائی اعداد کا (اتنے  ہی ہندسوں میں)  تکملہ ایک اور تکملہ دو حاصل کریں۔
\begin{multicols}{4}
\begin{enumerate}[a.]

\item  
 \(1011\)  
\item   
 \(1001\) 
\item  
 \(111101\) 
\item   
 \(10101010\) 
\item  
 \(11.11\)  
\item   
 \(1101.0011\) 
\end{enumerate}
\end{multicols}
جواب: تکملات ایک  \عددی{0100}، \عددی{0110}، \عددی{000010}، \عددی{01010101}؛  تکملات دو  \عددی{0101}، \عددی{0111}، \عددی{000011}، \عددی{01010110}
\انتہا{سوال}
\ابتدا{سوال}
%Q2.7
درج ذیل اعشاری سوالات کو تکملہ نو اور تکملہ دس استعمال کرتے ہوئے حل کریں۔ سادہ طریقے سے حاصل جوابات کے ساتھ موازنہ کریں۔
\begin{multicols}{3}
\begin{enumerate}[a.]

\item  
 \(9-4\)  
\item   
 \(16-9\) 
\item  
 \(23.9-13\) 
\item  
 \(555.078-303.93\) 
\item  
 \(0.555-0.045\) 
\item  
 \(1000-909.5301\) 
\end{enumerate}
\end{multicols}
\انتہا{سوال}
\ابتدا{سوال}
%Q2.8
درج ذیل ثنائی سوالات کو تکملہ ایک اور تکملہ دو سے حل کریں۔ سادہ ثنائی طریقے سے حاصل جوابات کے ساتھ موازنہ کریں۔
\begin{multicols}{3}
\begin{enumerate}[a.]
\item  
 \(11-10\)  
\item  
 \(1101-1010\) 
\item  
 \(11.10-10.11\) 
\item  
 \(1101.01-1001.1\) 
\item  
 \(101-1010\) 
\item  
 \(0.11-1101.11\) 
\end{enumerate}
\end{multicols}
\انتہا{سوال}
\ابتدا{سوال}
%Q2.9
درج ذیل اعشاری سوالات کو ثنائی روپ میں تبدیل کر کے حل کریں- جواب کو واپس اعشاری روپ میں تبدیل کر کے اعشاری طریقے سے حاصل جواب کے ساتھ موازنہ کریں۔
\begin{multicols}{4}
\begin{enumerate}[a.]
\item 
 \(3\times 9\)   
\item 
 \(31\times 23\)   
\item 
 \(15\times 3.625\)  
\item  
 \(1024\times 16\) 
\item 
 \(2048\times 2048\) 
\item 
 \(65.75\times 11.625\) 
\end{enumerate}
\end{multicols}
\انتہا{سوال}
%================
%3.1
\ابتدا{سوال} 
درج ذیل بوولین مساوات کا جدول لکھیں۔
\begin{multicols}{2}
\begin{enumerate}[a.]
\item 
 \(XYZ+\overline{X}Y\overline{Z}\) 
\item 
 \(ABC+A\overline{B}C+\overline{A}\,\overline{B}C\)   

\item 
 \(A(B+\overline{C})\)  
\item 
 \((A+B)(AB+BC+\overline{C}A)\)  

\item 
 \(A\overline{B}+\overline{A}B\)  
\item 
 \(A\overline{B}+B\overline{C}\) 
\end{enumerate}
\end{multicols}
جواب:
\begin{otherlanguage}{english}
\begin{tabular}{CCC|C}
X&Y&Z&\text{الف}\\
\midrule
0&0&0&0\\
0&0&1&0\\
0&1&0&1\\
0&1&1&0\\
1&0&0&0\\
1&0&1&0\\
1&1&0&0\\
1&1&1&1
\end{tabular} \hfill
\begin{tabular}{CCC|C}
A&B&C&\text{ب}\\
\midrule
0&0&0&0\\
0&0&1&1\\
0&1&0&0\\
0&1&1&0\\
1&0&0&0\\
1&0&1&1\\
1&1&0&0\\
1&1&1&1
\end{tabular}\hfill
\begin{tabular}{CCC|C}
A&B&C&\text{ج}\\
\midrule
0&0&0&0\\
0&0&1&0\\
0&1&0&0\\
0&1&1&0\\
1&0&0&1\\
1&0&1&0\\
1&1&0&1\\
1&1&1&1
\end{tabular}
\end{otherlanguage}
\انتہا{سوال}
\ابتدا{سوال}
%3.2 
تفاعل 
 \(AB+C\overline{D}\)
  کا متمم   \(\overline{AB+C\overline{D}}=(\overline{A}+\overline{B})(\overline{C}+D)\) ہے۔ درج ذیل کا  متمم لکھیں۔
\begin{multicols}{2}
\begin{enumerate}[a.]
\item  
 \(X+YZ+XY\) 
\item 
 \(AB(C\overline{D}+\overline{C}D)\)
\item   
 \(\overline{A}\,\overline{B}+A\overline{B}\) 
\item  
 \(X\overline{Y}Z+\overline{X}Y\) 
\item 
 \((A+B)(B+C)(C+A)\)
\end{enumerate}
\end{multicols}
جواب:  (ا)  \عددی{\overline{X}(\overline{Y}+\overline{Z})(\overline{X}+\overline{Y})}،  (ب) \عددی{\overline{A}+\overline{B}+(\overline{C}+D)(C+\overline{D})}،  (ج) \عددی{(A+B)(\overline{A}+B)}
\انتہا{سوال}
\ابتدا{سوال} 
%3.3
درج ذیل کے ادوار جمع، ضرب اور نفی گیٹوں کی مدد سے بنائیں۔
\begin{multicols}{3}
\begin{enumerate}[a.]
\item 
 \(AB\overline{C}+\overline{A}\,\overline{B}C\) 
\item 
 \(A+B(A+\overline{C})\)
\item 
 \(\overline{X}\,\overline{Y}(X+\overline{Y})\) 
\item 
 \(AB+BC+CA\)
\item 
 \(ABC+\overline{A}B\overline{C}+AB\overline{C}\)
\end{enumerate}
\end{multicols}
جواب:
\begin{tikzpicture}
\pgfmathsetmacro{\kpin}{1}
\pgfmathsetmacro{\kpina}{1.25}
\draw(0,0)node[or port,number inputs=2,anchor=out](u0){};
\draw(u0.in 1)--++(0,\kpina)node[and port,number inputs=3,anchor=out](u1){};
\draw(u0.in 2)--++(0,-\kpina)node[and port,number inputs=3,anchor=out](u2){};
\draw(u1.in 3)--++(0,-\kpin)node[not port,scale=0.9,anchor=out](u3){};
\draw(u2.in 2)node[not port,scale=0.9,anchor=out](u4){};
\draw(u2.in 3)--++(0,-\kpin)node[not port,scale=0.9,anchor=out](u5){};
\draw(u1.in 1)--++(-2*\kpin,0)coordinate(klft)node[left]{$A$};
\draw(u1.in 2)--++(-2*\kpin,0)node[left]{$B$};
\draw(u3.in)--(u3.in -| klft)node[left]{$C$};
\draw(u4.in)--++(-0.25*\kpin,0)coordinate(aa)--(aa |- u1.in 1);
\draw(u5.in)--++(-0.5*\kpin,0)coordinate(aa)--(aa |- u1.in 2);
\draw(u2.in 1)--++(0,0.5*\kpin)-|(u3.in);
\draw(u2.out)node[below right,yshift=-0.5cm]{\text{\RL{(الف)}}};
\end{tikzpicture}\hfill
\begin{tikzpicture}
\pgfmathsetmacro{\kpin}{0.75}
\pgfmathsetmacro{\kpina}{1.25}
\draw(0,0)node[and port,number inputs=3,anchor=out](u0){};
\draw(u0.in 2)node[or port,number inputs=2,scale=0.8,anchor=out](u1){};
\draw(u0.in 3)--++(0,-\kpin)node[not port,scale=0.8,anchor=out](u2){};
\draw(u1.in 1)--++(0,\kpin)node[not port,scale=0.8,anchor=out](u3){};
\draw(u0.in 1)|-(u3.out);
\draw(u3.in)coordinate(klft)node[left]{$Y$};
\draw(u1.in 2)--(u1.in 2 -| klft)node[left]{$X$};
\draw(u2.in)--++(-1*\kpin,0)coordinate(aa)--(aa |- u1.in 2);
\draw(u0.out)node[shift={(-0.5cm,-1cm)}]{\text{\RL{(ج)}}};
\end{tikzpicture}
\انتہا{سوال}
\ابتدا{سوال}
%3.4 
ڈی مارگن کلیات کو بوولین جدول   سے ثابت کریں۔
\انتہا{سوال}
\ابتدا{سوال}
%3.5 
بوولین جدول سے درج ذیل ثابت کریں۔
\begin{multicols}{2}
\begin{enumerate}[a.]
\item
 \(X\overline{Y}+XY=X\)
 \item
 \(X+\overline{X}Y=X+Y\)
 \end{enumerate}
 \end{multicols}
 جواب:  درج ذیل جدول  کا دایاں اور بایاں قطار ایک جیسے ہیں لہٰذا جزو-ا ثابت ہوا۔
 \begin{center}
 \begin{otherlanguage}{english}
 \begin{tabular}{CC|C}
 X&Y&X\overline{Y}+XY\\
 \midrule
 0&0&0\\
 0&1&0\\
 1&0&1\\
 1&1&1
 \end{tabular}
 \end{otherlanguage}
 \end{center}
\انتہا{سوال}
\ابتدا{سوال}
%3.6 
درج ذیل کو مجموعہ ارکان ضرب کی شکل میں لکھیں۔  جدول لکھ کر درستگی ثابت کریں۔
\begin{multicols}{2}
\begin{enumerate}[a.]
\item
 \( (A+B)(C+D) \)
 \item
 \( (A+B)(\overline{B}+C)(A+\overline{C}) \)
\item
 \( (A+B)(A+B+C)(C+B) \)
 \item
 \( (A+B+C)(\overline{B}+\overline{C}) \)
  \end{enumerate}
 \end{multicols}
 جواب: (ا) \عددی{ AC+AD+BC+BD}،   (ب) \عددی{ A\overline{B}+A\overline{B}\,\overline{C}+AC+ABC}
\انتہا{سوال}
\ابتدا{سوال}
%3.7 
(ا) بوولین مماثل استعمال کرتے ہوئے  درج ذیل کو  ضرب بعد از جمع کی شکل میں لکھیں ۔  (ب) ان تفاعل کے جدول لکھ کر یہی جواب حاصل کریں۔ (ج) دیے گئے تفاعل  اور حاصل جواب کے جدول لکھ کر  جواب کی درستگی ثابت کریں۔
\begin{multicols}{2}
\begin{enumerate}[a.]
\item
 \( XYZ+X\overline{Y}+\overline{X}\,\overline{Y} \)
 \item
 \( XY+\overline{Z}X \)
\item
 \( X\overline{Y}(\overline{Y}\,\overline{Z}+YZ) \)
 \item
 \( (A+B\overline{C})(\overline{A}B+\overline{B}A) \)
  \end{enumerate}
 \end{multicols}
 جواب: (ا) \عددی{ (X+\overline{Y}+Z)(X+\overline{Y}+\overline{Z})(\overline{X}+\overline{Y}+Z)}
\انتہا{سوال}
\ابتدا{سوال}
%3.8 
 تفاعل \عددی{Y} درج ذیل صورتوں میں  \عددی{1} کے برابر ہے۔اگر   \عددی{A=0}،   \عددی{B=0}، اور   \عددی{C=1}ہو یا اگر   \عددی{A=1}، \عددی{B=1}، اور \عددی{C=0}ہو اور یا اگر \عددی{A=1}، \عددی{B=1}، اور \عددی{C=1}ہو۔دیگر صورت  تفاعل کی قیمت   \عددی{(0)} ہے۔ان معلومات کا جدول لکھ کر تفاعل کی سادہ  مساوات مجموعہ ارکان ضرب کے روپ میں حاصل کریں۔
 
 جواب: \عددی{ Y=\overline{A}\,\overline{B}C+AB\overline{C}+ABC}
\انتہا{سوال}
\ابتدا{سوال}
%3.9 
(ا) گزشتہ سوال میں دیے  تفاعل \عددی{Y} کا  \اصطلاح{ ضرب و جمع }\فرہنگ{ضرب و جمع}\حاشیہب{AND-OR}\فرہنگ{AND-OR} دور  بنائیں۔ (ب) اس تفاعل  کا \اصطلاح{ضرب متمم و ضرب متمم }\فرہنگ{ضرب متمم و ضرب متمم}\حاشیہب{NAND-NAND}\فرہنگ{NAND-NAND} دور  بنائیں۔  مداخل کے متمم  دستیاب ہیں۔

جواب:
\begin{tikzpicture}
\pgfmathsetmacro{\kpin}{1}
\pgfmathsetmacro{\kpina}{1.25}
\draw(0,0)node[or port,number inputs=3,anchor=out](u0){};
\draw(u0.in 1)--++(0,\kpin)node[and port,number inputs=3,anchor=out](u1){};
\draw(u0.in 2)node[and port,number inputs=3,anchor=out](u2){};
\draw(u0.in 3)--++(0,-\kpin)node[and port,number inputs=3,anchor=out](u3){};
\draw(u1.in 1)node[left]{$\overline{A}$} (u1.in 2)node[left]{$\overline{B}$} (u1.in 3)node[left]{$C$};
\draw(u2.in 1)node[left]{$A$} (u2.in 2)node[left]{$B$} (u2.in 3)node[left]{$\overline{C}$};
\draw(u3.in 1)node[left]{$A$} (u3.in 2)node[left]{$B$} (u3.in 3)node[left]{$C$};
\draw(u0.out)node[right]{$Y$};
\end{tikzpicture}\hfill
\begin{tikzpicture}
\pgfmathsetmacro{\kpin}{1}
\pgfmathsetmacro{\kpina}{1.25}
\draw(0,0)node[nand port,number inputs=3,anchor=out](u0){};
\draw(u0.in 1)--++(0,\kpin)node[nand port,number inputs=3,anchor=out](u1){};
\draw(u0.in 2)node[nand port,number inputs=3,anchor=out](u2){};
\draw(u0.in 3)--++(0,-\kpin)node[nand port,number inputs=3,anchor=out](u3){};
\draw(u1.in 1)node[left]{$\overline{A}$} (u1.in 2)node[left]{$\overline{B}$} (u1.in 3)node[left]{$C$};
\draw(u2.in 1)node[left]{$A$} (u2.in 2)node[left]{$B$} (u2.in 3)node[left]{$\overline{C}$};
\draw(u3.in 1)node[left]{$A$} (u3.in 2)node[left]{$B$} (u3.in 3)node[left]{$C$};
\draw(u0.out)node[right]{$Y$};
\end{tikzpicture}

\انتہا{سوال}
\ابتدا{سوال}
%3.10 
تفاعل \عددی{Z} کی قیمت درج ذیل صورتوں میں صفر \عددی{(0)} ہے۔اگر \عددی{A=0}، \عددی{B=0}، اور \عددی{C=0}ہو یا اگر \عددی{A=1}، \عددی{B=0}، اور \عددی{C=0}ہو یا اگر \عددی{A=1}، \عددی{B=1}، اور \عددی{C=0}ہو اور یا اگر \عددی{A=1}، \عددی{B=1}، اور \عددی{C=1}ہو۔ ان صورتوں کے علاوہ اس کی قیمت ایک \عددی{(1)} رہتی ہے۔ان معلومات کا جدول لکھ کر \عددی{Z} کی  ضرب بعد از جمع  مساوات  حاصل کریں۔

جواب:   \عددی{Z=(A+B+C)(\overline{A}+B+C)(\overline{A}+\overline{B}+C)(\overline{A}+\overline{B}+\overline{C})}
\انتہا{سوال}
\ابتدا{سوال}
%3.11
(ا) گزشتہ سوال میں دیے تفاعل \عددی{Z} کا جمع و ضرب دور بنائیں۔  (ب) اس تفاعل کا  \اصطلاح{جمع متمم و جمع متمم }\حاشیہب{NOR-NOR}دور بنائیں۔    مداخل کے متمم دستیاب ہیں۔

جواب:
\begin{tikzpicture}
\pgfmathsetmacro{\kpin}{0.4}
\pgfmathsetmacro{\kpina}{0.45}
\pgfmathsetmacro{\kpinb}{1.4}
\draw(0,0)node[and port,number inputs=4,anchor=out](u0){};
\draw(u0.in 1)--++(0,\kpinb)--++(-\kpin,0)node[or port,number inputs=3,anchor=out](u1){};
\draw(u0.in 2)--++(-\kpin,0)--++(0,\kpina)node[or port,number inputs=3,anchor=out](u2){};
\draw(u0.in 3)--++(-\kpin,0)--++(0,-\kpina)node[or port,number inputs=3,anchor=out](u3){};
\draw(u0.in 4)--++(0,-\kpinb)--++(-\kpin,0)node[or port,number inputs=3,anchor=out](u4){};
\draw(u1.in 1)node[left]{$A$} (u1.in 2)node[left]{$B$} (u1.in 3)node[left]{$C$};
\draw(u2.in 1)node[left]{$\overline{A}$} (u2.in 2)node[left]{$B$} (u2.in 3)node[left]{$C$};
\draw(u3.in 1)node[left]{$\overline{A}$} (u3.in 2)node[left]{$\overline{B}$} (u3.in 3)node[left]{$C$};
\draw(u4.in 1)node[left]{$\overline{A}$} (u4.in 2)node[left]{$\overline{B}$} (u4.in 3)node[left]{$\overline{C}$};
\draw(u0.out)node[right]{$Z$};
\end{tikzpicture}\hfill
\begin{tikzpicture}
\pgfmathsetmacro{\kpin}{0.4}
\pgfmathsetmacro{\kpina}{0.45}
\pgfmathsetmacro{\kpinb}{1.4}
\draw(0,0)node[nor port,number inputs=4,anchor=out](u0){};
\draw(u0.in 1)--++(0,\kpinb)--++(-\kpin,0)node[nor port,number inputs=3,anchor=out](u1){};
\draw(u0.in 2)--++(-\kpin,0)--++(0,\kpina)node[nor port,number inputs=3,anchor=out](u2){};
\draw(u0.in 3)--++(-\kpin,0)--++(0,-\kpina)node[nor port,number inputs=3,anchor=out](u3){};
\draw(u0.in 4)--++(0,-\kpinb)--++(-\kpin,0)node[nor port,number inputs=3,anchor=out](u4){};
\draw(u1.in 1)node[left]{$A$} (u1.in 2)node[left]{$B$} (u1.in 3)node[left]{$C$};
\draw(u2.in 1)node[left]{$\overline{A}$} (u2.in 2)node[left]{$B$} (u2.in 3)node[left]{$C$};
\draw(u3.in 1)node[left]{$\overline{A}$} (u3.in 2)node[left]{$\overline{B}$} (u3.in 3)node[left]{$C$};
\draw(u4.in 1)node[left]{$\overline{A}$} (u4.in 2)node[left]{$\overline{B}$} (u4.in 3)node[left]{$\overline{C}$};
\draw(u0.out)node[right]{$Z$};
\end{tikzpicture}
\انتہا{سوال}
\ابتدا{سوال}
%3.12
جدول میں \عددی{A}، \عددی{B}، اور \عددی{C} تین آزاد داخلی متغیرات جبکہ \عددی{F_0}، \عددی{F_1}, اور \عددی{F_2} تابع خارجی  متغیرات ہیں۔
\begin{center}
\begin{otherlanguage}{english}
\begin{tabular}{CCC|CCC}
\toprule
A&B&C&F_0&F_1&F_2\\
\midrule
0&0&0&0&1&1\\
0&0&1&1&0&1\\
0&1&0&1&1&0\\
0&1&1&0&0&0\\
1&0&0&1&1&1\\
1&0&1&0&0&1\\
1&1&0&0&0&0\\
1&1&1&0&1&1\\
\bottomrule
\end{tabular}
\end{otherlanguage}
\end{center}
\begin{enumerate}[a.]

\item
تابع متغیرات     مجموعہ ارکان ضرب   روپ میں لکھیں۔

\item
 ضرب گیٹ اور  جمع گیٹ استعمال کرتے ہوئے تابع متغیرات کے ضرب و جمع دور بنائیں۔

\item
ضرب و جمع ادوار سے تابع متغیرات کے ضرب متمم و  ضرب متمم ادوار حاصل کریں۔

\item
 تابع متغیرات  کو ضرب بعد از جمع  روپ میں لکھیں۔

\item
 جمع گیٹ اور ضرب گیٹ استعمال کرتے ہوئے تابع متغیرات کے جمع و ضرب  ادوار بنائیں۔

\item
جمع و ضرب ادوار سے تابع متغیرات کے جمع متمم و جمع متمم ادوار حاصل کریں۔
\end{enumerate}

جواب: (ا) \عددی{F_0=\overline{A}\,\overline{B}C+\overline{A}B\overline{C}+A\overline{B}\,\overline{C}}، \عددی{F_1=\overline{A}\,\overline{B}\,\overline{C}+\overline{A}B\overline{C}+A\overline{B}\,\overline{C}+ABC}، \عددی{F_2=\overline{A}\,\overline{B}\,\overline{C}+\overline{A}\,\overline{B}C+A\overline{B}\,\overline{C}+A\overline{B}C+ABC}؛  \\
(د) \عددی{
F_0=(A+B+C)(A+\overline{B}+\overline{C})(\overline{A}+B+\overline{C})(\overline{A}+\overline{B}+C)(\overline{A}+\overline{B}+\overline{C})
}
\انتہا{سوال}
\ابتدا{سوال}
%3.13
درج ذیل تفاعل مجموعہ ارکان ضرب  روپ میں ہیں۔انہیں  ضرب بعد از جمع روپ میں لکھیں۔
\begin{multicols}{2}
\begin{enumerate}[a.]
\item
 \عددی{Z(A,B)=\sum(0,1)} 
\item
 \عددی{F(A,B,C)=\sum(1,3,7)} 
\item
  \عددی{F(A,B,C)=\sum (0,5,7)} 
\item
  \عددی{Y(A,B,C)=\sum (0,7)}
\item
 \عددی{Z(A,B,C,D)=\sum (0,2,5,12)}
\end{enumerate}
\end{multicols}

جواب:  (ا) \عددی{Z=\prod(2,3)}، (ج) \عددی{F=\prod(1,2,3,4,6)}،  (ھ) \عددی{Z=\prod(1,3,4,6,7,8,9,10,11,13,14,15)}
\انتہا{سوال}
\ابتدا{سوال}
%3.14
درج ذیل تفاعل  ضرب بعد از جمع  روپ میں ہیں۔انہیں مجموعہ ارکان ضرب روپ میں لکھیں۔
\begin{multicols}{2}
\begin{enumerate}[a.]
\item
\( F(A,B)=\prod (1,3)\)
\item
\( Z(A,B,C)=\prod (0,4,7) \)
\item
\( Z(A,B,C,D)=\prod (0,1,5,7,13,15) \)
\end{enumerate}
\end{multicols}

جواب: (ا) \عددی{F=\sum(0,2)}، (ج) \عددی{Z=\sum(2,3,4,6,8,9,10,11,12,14)}
\انتہا{سوال}
\ابتدا{سوال}
%3.15
انٹرنیٹ  سے درج ذیل معلوماتی صفحات  حاصل کریں۔ یہ مخلوط ادوار پاکستان کے ہر شہر میں نہایت سستے دام  دستیاب ہیں۔
\begin{multicols}{5}
\begin{enumerate}[a.]
\item
 \( 7400 \) 
 \item
  \( 4011  \)
  \item
  \( 7408  \)
  \item
  \(  4081  \)
  \item
  \( 4000  \)
  \item
  \( 7432  \)
  \item
  \( 7404  \)
  \item
  \( 4049  \)
  \item
  \( 4070  \)
\end{enumerate}
\end{multicols}
\انتہا{سوال}
\ابتدا{سوال}
%3.16
گزشتہ سوال میں \عددی{7400} مخلوط دور  کے  معلومات صفحات سے   دریافت کریں اس میں موجود  چار گیٹوں کے مخارج کن پنیوں پر دستیاب ہیں۔

جواب: پنیے \عددی{3}، \عددی{6}، \عددی{8}، اور \عددی{11}
\انتہا{سوال}
\ابتدا{سوال}
%3.17
انٹرنیٹ سے تین مداخل  ضرب گیٹ اور چار مداخل جمع گیٹ کے مخلوط ادوار دریافت کریں۔
\انتہا{سوال}
%====================
%???KKK karnaugh pending
\ابتدا{سوال}
4.1 کارناف نقشے میں
\انتہا{سوال}
%=============
%???KKK  karnaugh pending
\ابتدا{سوال}\شناخت{سوال_چار_مداخل_دور}
%5.1 
شکل   میں چار مداخل  دور دیا گیا ہے۔
\begin{center}
\begin{tikzpicture}
\def\kpin{0.7}
\def\kpina{0.4}
\draw(0,0)node[xor port,anchor=out,number inputs=2](u0){};
\draw(u0.in 1)--++(0,\kpin)node[or port,number inputs=2,anchor=out](u1){};
\draw(u0.in 2)--++(0,-\kpin)node[or port,number inputs=2,anchor=out](u2){};
\draw(u1.in 2)--(u2.in 1)coordinate[pos=0.5](aa);
\draw(aa)--++(-\kpin,0)node[above]{$K_1$}node[and port,anchor=out](u3){}  (u2.in 2)--++(-\kpin,0)node[above]{$K_2$}node[and port,anchor=out](u4){};
\draw(u3.in 1)node[not port,anchor=out,scale=0.7](u5){}  (u4.in 2)node[not port,anchor=out,scale=0.7](u6){};
\draw(u6.in)|-(u3.in 2)   (u6.in)--++(-\kpina,0)coordinate(klft)node[left]{$D$};
\draw(u4.in 1)--(u4.in 1 -| klft)node[left]{$C$}  (u5.in)--(u5.in -| klft)node[left]{$B$}  (u1.in 1)--(u1.in 1 -| klft)node[left]{$A$}  (u0.out)node[right]{$F$};
\end{tikzpicture}
\end{center}
\begin{enumerate}[a.]
\item
اندرونی متغیرات  \عددی{K_1} اور \عددی{ K_2}کی بوولین مساوات  حاصل  کریں۔
\item
 خارجی تابع متغیر \عددی{F} کی بوولین مساوات حاصل کریں۔
\item
ایک بوولین جدول بنائیں جس میں چار آزاد متغیرات \عددی{A}، \عددی{B}،  \عددی{C}،  اور \عددی{D} کی تمام  ممکنہ ترتیب درج ہو۔اس جدول میں \عددی{K_1}، \عددی{K_2}،  اور \عددی{F} کے خانے بنا کر پُر کریں۔ 
\end{enumerate}

جواب: (ا) \عددی{K_1=\overline{B}D}، \عددی{K_2=C\overline{D}}؛ (ب) \عددی{F=(A+K_1)\oplus (K_1+K_2)}، 
\عددی{F=(A+\overline{B}D)\oplus (\overline{B}D+C\overline{D})}
\انتہا{سوال}
\ابتدا{سوال}
%5.2
 ایسا بوولین جدول بنائیں جس  میں تین مداخل اور ایک مخارج ہو۔جدول  یوں پُر کریں کہ مخارج کی قیمت صرف اُس صورت ایک  \عددی{(1)} ہو جب صرف  ایک مداخل کی قیمت صفر   \عددی{(0)} ہو۔اس جدول کی مدد سے مخارج کا ترکیبی دور تشکیل دیں۔
 
 جواب:
 \begin{center}
 \begin{otherlanguage}{english}
 \begin{tabular}{CCC|C}
 \toprule
 A&B&C&F\\
 \midrule
 0&0&0&0\\
 0&0&1&0\\
 0&1&0&0\\
 0&1&1&1\\
 1&0&0&0\\
 1&0&1&1\\
 1&1&0&1\\
 1&1&1&0\\
 \bottomrule
 \end{tabular}
 \end{otherlanguage}
 \end{center}
 \عددی{F=\sum(3,5,6)}، \عددی{F=\prod(0,1,2,4,7)}
\انتہا{سوال}
\ابتدا{سوال}
%5.3
چار مداخل کا  ایسا بوولین جدول بنائیں جس کا مخارج صرف اُس   صورت بلند ہو جب  داخلی ثنائی عدد کی قیمت اعشاری  نو  \عددی{9} سے  کم ہو تفاعل کا ترکیبی دور تشکیل دیں۔ 

جواب: \عددی{F=\sum(0,1,2,3,4,5,6,7,8)}
\انتہا{سوال}
%???KKK done till here
\ابتدا{سوال}
%5.4
تین مداخل اور تین مخارج  کا ایسا بوولین جدول تشکیل دیں جس میں  داخلی ثنائی عدد کی قیمت سات  \عددی{(7)} سے کم ہونے کی صورت میں مخارج کی قیمت مداخل سے ایک زیادہ  ہو  جبکہ  داخلی قیمت سات کے برابر ہونے کی صورت میں مخارج کی قیمت صفر \عددی{(000)}  ہو۔
\انتہا{سوال}
\ابتدا{سوال}
%5.5
\اصطلاح{اقلیتی دور }\فرہنگ{اقلیتی دور}\حاشیہب{minority circuit}\فرہنگ{minority circuit} ایسے ترکیبی دور کو کہتے ہیں جس کا مداخل اس صورت بلند ہوتا ہے جب اس کے زیادہ تر مداخل پست ہوں۔تین  مداخل  اقلیتی دور تشکیل دیں۔
\انتہا{سوال}
\ابتدا{سوال}
%5.6
ایک ترکیبی دور تشکیل دیں جو اعشاری ہندسے کا اساس نو خارج  کرے۔اس دور کے چار مداخل اور چار مخارج ہوں گے۔
\انتہا{سوال}
\ابتدا{سوال}
%5.7
 تین بٹ کے دو اعداد کا موازنہ کرنے والا ایسا ترکیبی دور تشکیل دیں  جس کا مخارج اس صورت بلند ہو جب دونوں اعداد کی قیمتیں  برابر ہوں۔
\انتہا{سوال}
\ابتدا{سوال}
%5.8
 چار بٹ کے دو ثنائی اعداد ضرب کرنے والا ترکیبی دور تشکیل دیں۔
\انتہا{سوال}
\ابتدا{سوال}
%5.9
 جمع متمم گیٹ استعمال کرتے ہوئے  شناخت کار تشکیل دیں۔
\انتہا{سوال}
\ابتدا{سوال}
%5.10
ایک عدد  \عددی{3\times 8} شناخت کار کی مدد سے   درج ذیل تین تفاعلات حاصل کریں۔اس دور کو شکل \حوالہء{ 5.25 } کی طرز پر تشکیل دیں۔
\begin{align*}
F_0(X,Y,Z)&=\sum(0,3,7)\\
F_1(X,Y,Z)&=\sum(1,2,5)\\
F_2(X,Y,Z)&=\sum(0,1,2,3,5,7)
\end{align*}
 \انتہا{سوال}
\ابتدا{سوال}
%5.11
 درج ذیل تفاعل کو  \عددی{16\times 1} داخلی منتخب کار کی مدد سے حاصل  کریں۔
 \begin{align*}
 F(A,B,C,D)=\sum(0,1,4,7,13,15)
 \end{align*}
\انتہا{سوال}
\ابتدا{سوال}
%5.12
 مکمل جمع کار کو دو عدد داخلی منتخب کار کی مدد سے حاصل کریں۔
\انتہا{سوال}
\ابتدا{سوال}
%5.13
 شکل\حوالہء{ 12.2 } میں اعشاری ہندسوں کی  \اصطلاح{سات  کلی  نمائشی تختی }\فرہنگ{سات کلی نمائشی تختی}\حاشیہب{seven segment display}\فرہنگ{seven segment display} دکھائی  گئی ہے جو سات قابل روشن حصوں پر مبنی ہے۔ان حصوں میں سے  کسی ایک یا ایک سے زیادہ حصوں کو بیک وقت روشن کیا جا سکتا  ہے۔یوں مختلف حصے روشن کرنے سے اعشاری ہندسے لکھے جا  سکتے ہیں۔مثلاً   حصہ ب اور پ  (یعنی بپ)  بیک وقت روشن کرنے سے \عددی{1}   لکھا جائے گا۔اسی طرح حصہ ا،ب،  پ، ت، ٹ، اور  ث (یعنی ابپتٹث)   بیک  وقت روشن کرنے سے \عددی{0} لکھا جا ئے گا۔   فرض کریں کسی حصے کو  روشن کرنے کے لئے اس حصہ  کو  بلند کیا جاتا ہے۔چار مداخل اور سات مخارج  کا   ترکیبی  دور تشکیل دیں جو مہیا کردہ اعشاری ہندسے کو اس تختی پر دکھائے۔ اعشاری ہندسہ  ثنائی علامتی روپ میں مہیا کیا جائے گا۔   مخلوط دور  \عددی{4511} یہی کام  سرانجام   دیتا ہے۔
\انتہا{سوال}
\ابتدا{سوال}
%5.14
 انٹرنیٹ سے سات کلی  نمائشی تختی کے معلوماتی صفحات حاصل کریں۔ یہ سات نوری ڈایوڈ  پر  مشتمل ہو گا۔ بعض ادوار میں تمام نوری ڈایوڈ کے منفی سر ایک ساتھ جوڑ  کر  مطلوبہ نوری ڈایوڈ کے مثبت سر پر \عددی{1} مہیا   کر کے روشن کیا جاتا ہے اور بعض میں تمام کے مثبت سر آپس میں جوڑ کر مطلوبہ نوری ڈایوڈ کا  منفی سر پست کر کے اسے روشن کیا جاتا ہے۔
\انتہا{سوال}
%=================
\ابتدا{سوال}
%6.1
 ثابت کریں  جے کے پلٹ کے مخارج  \عددی{\overline{Q}_{n+1}} کی مساوات    درج ذیل ہے۔
 \begin{align*}
 \overline{Q}_{n+1}=\overline{J}\,\overline{Q}+KQ
 \end{align*}
\انتہا{سوال}
\ابتدا{سوال}  
%6.2
 شکل میں ضرب گیٹ کا دورانیہ رد عمل  \عددی{10} نینو سیکنڈ جبکہ جمع گیٹ کا \عددی{15} نینو سیکنڈ ہے۔تینوں مداخل بیک وقت تبدیل کیے جاتے ہیں۔ کتنی دیر بعد مخارج  \عددی{F_1}اور \عددی{F_2}  مستحکم حالت میں ہوں گے۔ (جواب: \عددی{\SI{10}{\nano\second}} ، \عددی{\SI{25}{\nano\second}})
\انتہا{سوال}
\ابتدا{سوال}
%6.3
 ایک کمپیوٹر  \عددی{\SI{2}{\giga\hertz}}  ساعتی اشارے  سے چلتا ہے۔یہ  اشارہ تیس فی صد وقت بلند رہتا ہے جبکہ اس کا دورانیہ اترائی  پانچ فی صد اور دورانیہ چڑھائی پانچ فی صد وقت لیتے ہیں۔ساعتی اشارے  کا دوری عرصہ ،دورانیہ چڑھائی اور پست دورانیہ حاصل کریں۔ (جواب:  \عددی{\SI{5e-10}{\second}}،  \عددی{\SI{2.5e-11}{\second}}، \عددی{\SI{3e-10}{\second}})
\انتہا{سوال}
\ابتدا{سوال}
% 6.4
 جمع متمم گیٹ پر مبنی متعدد  مداخل ایس آر پلٹ کے مداخل  ترسیم کیے گئے ہیں۔اس کا مخارج  ترسیم کریں۔
\انتہا{سوال}
\ابتدا{سوال} 
%6.5
 آقا و غلام پلٹ کے مداخل  ترسیم  کیے گئے ہیں۔مخارج \عددی{Q_a}اور \عددی{Q} ترسیم کریں۔
\انتہا{سوال}
\ابتدا{سوال}
%6.6
 شکل \حوالہء{ 6.25 } میں سلسلہ وار ثنائی جمع کار  پیش ہے۔اسے استعمال کرتے ہوئے \عددی{10110011_2} اور  \عددی{00110011_2}  جمع کریں۔
\انتہا{سوال}
\ابتدا{سوال}
%6.7
ایک ترتیب دور جس میں دو ڈی پلٹ،  \عددی{A} اور \عددی{B}،  استعمال ہوئے ہیں کے مداخل \عددی{x} اور  \عددی{y} جبکہ مخارج \عددی{z} ہے۔ دور کی مساوات درج ذیل ہیں۔
\begin{align*}
A(t+1)&=\overline{x}y+xA\\
B(t+1)&=\overline{x}B+xA\\
z&=B
\end{align*}
\begin{enumerate}[a.]
\item
 ترتیبی دور  بنائیں۔ 
\item
  ان مساوات سے حال کا جدول حاصل کریں۔ 
\item
  حال  کے جدول سے حال کا خاکہ حاصل کریں۔
\end{enumerate}
\انتہا{سوال}
\ابتدا{سوال}
%6.8
 مداخل \عددی{x} اور دو جے کے پلٹ،  \عددی{A} اور \عددی{B} ، پر مبنی ترتیبی دور درج ذیل مساوات   پر پورا اترتا ہے۔
 \begin{align*}
 J_A&=\overline{B}\\
 K_A&=x\\
 J_B&=A\\
 K_B&=x
 \end{align*}
 \begin{enumerate}[a.]
\item
ان سے حال کی مساوات  \عددی{A(t+1)} اور  \عددی{B(t+1)} حاصل کریں۔
\item 
ان  مساوات سے  حال کا خاکہ بنائیں۔
\end{enumerate}
\انتہا{سوال}
\ابتدا{سوال}
%6.9
 دو  ڈی پلٹ، \عددی{A} اور \عددی{B}، استعمال کر کے مداخل \عددی{x} کا   ترتیبی دور تخلیق دیں جو بالترتیب  \عددی{00}، \عددی{01}، \عددی{10}،  اور \عددی{11} حال اختیار کر سکتا ہو۔بلند  مداخل  کی صورت میں بڑھتی  گنتی   اور پست مداخل  کی صورت میں گھٹتی گنتی  حاصل کرنی ہے۔ بڑھتی گنتی کی صورت میں \عددی{11} کو  پہنچنے کے بعد بلند  مداخل  کی صورت میں  دور اسی حال میں رہنا چاہیے ۔گھٹتی گنتی کرتے ہوئے \عددی{00} کو پہنچنے کے بعد پست مداخل کی صورت میں دور \عددی{00} میں رہنا چاہیے۔
\انتہا{سوال}
\ابتدا{سوال}
%6.10
 گزشتہ سوال میں  مداخل  \عددی{e} کا اضافہ کریں۔ بلند  \عددی{e} کی صورت میں دور جوں کا توں چلتا ہو جبکہ پست \عددی{e} کی صورت میں  دور اپنا حال برقرار رکھتا ہو۔
\انتہا{سوال}
\ابتدا{سوال}
%6.11
 پچھلے سوال   میں  مداخل کی تعداد میں مزید اضافہ کرتے ہوئے مداخل \عددی{s}  کا اضافہ کریں۔ مداخل \عددی{s}  بلند کرنے سے دور کو حال \عددی{00} اختیار کر لینا چاہیے جبکہ پست  \عددی{s} کی صورت میں دور  کو  پہلے کی طرح کام کرنا چاہیے۔
\انتہا{سوال}
%==============
\ابتدا{سوال}
%7.1
 چار بٹ  سلسلہ وار دائیں منتقل دفتر میں ابتدائی ثنائی مواد  \عددی{1011} موجود ہے۔دفتر کا مخارج  اسی دفتر کو بطور مداخل مہیا کیا جاتا ہے۔سات ساعت کے کنارے گزرنے کے بعد دفتر میں کیا عدد ہو گا؟
\انتہا{سوال}
\ابتدا{سوال}
%7.2
  گزشتہ سوال میں دائیں منتقل دفتر کے بجائے بائیں منتقل دفتر استعمال کرتے ہوئے  جواب معلوم کریں۔
\انتہا{سوال}
\ابتدا{سوال}
%7.3
 گزشتہ دو سوالات میں  ساعت  کے ہر کنارے پر دفتر میں ثنائی عدد  معلوم  کریں۔
\انتہا{سوال}
\ابتدا{سوال}
%7.4
 آٹھ بٹ  سلسلہ وار دائیں منتقل دفتر کا  مخارج  چار بٹ  سلسلہ وار دائیں منتقل دفتر  کو بطور مداخل فراہم کیا جاتا ہے۔آٹھ بٹ دفتر میں ابتدائی مواد \عددی{10110110} پایا جاتا ہے  اور  اسے \عددی{1010}   مداخل  فراہم کیا جاتا ہے۔ساعت کے سات کنارے گزرنے کے بعد ان دفتر  میں کیا اعداد پائے جائیں گے؟
\انتہا{سوال}
\ابتدا{سوال}
%7.5
 گزشتہ سوال میں چار بٹ سلسلہ وار بائیں منتقل دفتر استعمال کرتے  ہوئے جواب حاصل کریں۔
\انتہا{سوال}
\ابتدا{سوال}
%7.6
 آٹھ بٹ کے دو عدد عالمگیر دفتر استعمال کرتے ہوئے سولہ بٹ کا عالمگیر دفتر حاصل کریں۔
\انتہا{سوال}
\ابتدا{سوال}
%7.7
 شکل   \حوالہ{شکل_دفتر_متعدد_ہندسی_جمع_کار}  میں سلسلہ وار ثنائی جمع کار دکھایا گیا ہے۔ آٹھ بِٹ دفتر۔ا  میں \عددی{11001010}  اور  آٹھ بِٹ دفتر-ب   میں \عددی{11100001} پایا جاتا ہے۔تصور کریں  ساعت کے آٹھ کنارے گزرتے ہیں۔ساعت کا ہر کنارہ گزرنے کے بعد دفتر-ا میں  کیا مواد موجود ہو گا؟
\انتہا{سوال}
\ابتدا{سوال}
%7.8
 سلسلہ وار ثنائی جمع کار سے سلسلہ وار ثنائی منفی کار حاصل کریں۔ منفی  کردہ عدد کا تکملہ  دفتر-ب میں متوازی لکھنا بھی دکھائیں۔
\انتہا{سوال}
%===============
\ابتدا{سوال}
%8.1
 چار بٹ معاصر سیدھا گنت کار کی موجودہ گنتی  \عددی{0101_2} ہے۔ساعت کے کتنے کناروں بعد  \عددی{0000_2} ہو گا؟
\انتہا{سوال}
\ابتدا{سوال}
%8.2
 سولہ بٹ معاصر گنت کار کی موجودہ گنتی  \عددی{3FA7_{16}}ہے۔ساعت کے کتنے کنارے گزرنے کے بعد \عددی{0000_{16}} ہو گا؟ (ا) تصور کریں یہ سیدھا گنت کار ہے۔ (ب) تصور کریں یہ الٹ گنت کار ہے۔
\انتہا{سوال}
\ابتدا{سوال}
%8.3
 چار بٹ ثنائی لہریا گنت کار  استعمال کر کے  ثنائی مرموز اعشاری گنت کار  بنایا جا سکتا ہے۔ پس اتنا کرنا ہو گا کہ \عددی{1010_2} پر پہنچ کر  گنتی   فوراً زبردستی \عددی{0000_2} کی جائے ، جو ایک  ضرب متمم گیٹ استعمال  کرنے سے  ممکن ہے۔زبردستی پست صلاحیت رکھنے والے پلٹ استعمال کرتے ہوئے   دور تخلیق دیں۔ 
\انتہا{سوال}
\ابتدا{سوال}
%8.4
 ڈی پلٹ استعمال کرتے ہوئے چار بٹ معاصر ثنائی گنت کار تشکیل دیں۔ 
\انتہا{سوال}
\ابتدا{سوال}
%8.5
 جے کے پلٹ استعمال کر کے  ایسا معاصر گنت کار تشکیل دیں  جو   \عددی{0}، \عددی{2}، \عددی{3}، اور \عددی{7}   کا گردان کرے۔
\انتہا{سوال}
\ابتدا{سوال}
%8.6
 ٹی پلٹ استعمال کرتے ہوئے ایسا  چار بٹ ثنائی معاصر گنت کار تشکیل دیں جو صفر  \عددی{(0000_2)} سے چودہ  \عددی{(1110_2)} تک جفت گنتی کے بعد ایک  \عددی{(0001_2)} سے پندرہ  \عددی{(1111_2)} تک طاق گنتی کرے اور اس ترتیب کو دہراتا رہے۔ 
\انتہا{سوال}
\ابتدا{سوال}
%Q8.6b
\ابتدا{سوال}
ایسا چار بِٹ چھلا گنت کار تخلیق دیں جو بلند بِٹ کو \عددی{Q_0} سے \عددی{Q_1} رخ گھماتا ہو۔
\انتہا{سوال}
%8.7
 شکل \حوالہء{ 8.11 } میں دورانیہ پیدا کار دکھایا گیا ہے جہاں  ساعت کا تعدد \عددی{\SI{10}{\mega\hertz}} ہے اور درکار دورانیہ \عددی{\SI{500}{\nano\second}}  ہے۔درکار دورانیہ کے تین بٹ کیا ہوں گے؟
\انتہا{سوال}
\ابتدا{سوال}
%8.8
 کارناف نقشے  استعمال  کر کے  مساوات \حوالہ{مساوات_گنت_کار_چار_بِٹ} حاصل کریں۔
\انتہا{سوال}
\ابتدا{سوال}
%8.9
جےکے پلٹ استعمال کرتے ہوئے  مساوات  \حوالہ{مساوات_گنت_کار_چار_بِٹ} کی متبادل مساوات   کیا ہو گی؟
\انتہا{سوال}
%=============
\ابتدا{سوال}
%9.1
  مختلف جسامت کے حافظوں میں پتہ بِٹ  کی  اعشاری تعداد  (ا) \عددی{4}، (ب) \عددی{16}، (ج) \عددی{32}، اور  (د) \عددی{132} ہے۔  ان حافظوں میں الفاظ ذخیرہ کرنے کے  مقام کتنے  ہوں گے؟
\انتہا{سوال}
\ابتدا{سوال}
%9.2
 حافظہ کی  جسامت  عموماً \عددی{N\times D} لکھی اور پکاری جاتی ہے ، جہاں \عددی{N}  حافظہ میں الفاظ کی تعداد اور \عددی{D} ایک لفظ میں بِٹوں کی تعداد  ہے۔یوں  (ا) \عددی{64K\times 8}،  (ب)  \عددی{16K\times 4} ، (ج)  \عددی{256K\times 8}، اور (د)   \عددی{1G\times 32}  حافظوں میں پتہ پن  اور  مواد پن کتنے  ہوں گے؟
\انتہا{سوال}
\ابتدا{سوال}
%9.3
 کسی حافظہ کے  \عددی{50293_{10}} پتہ پر \عددی{172_{10}} مواد لکھا ہے۔اس تک رسائی کے لئے سولہ پتہ بٹ کیا ہوں گے اور اس  مقام سے کیا  آٹھ مواد  پڑھا جائے گا؟
\انتہا{سوال}
\ابتدا{سوال}
%9.4
 چار عدد \عددی{2K\times 9} حافظہ  اور ا یک عدد \عددی{2\times 4} شناخت کار کی مدد سے \عددی{8K\times 8} حافظہ حاصل کریں۔
\انتہا{سوال}
\ابتدا{سوال}
%9.5
 دو عدد \عددی{256K\times 8} حافظے  استعمال  کر کے \عددی{256K\times 16} حافظہ حاصل کریں۔
\انتہا{سوال}
\ابتدا{سوال}
%9.6
 چار پتہ  اور آٹھ مواد بٹ حافظہ استعمال کر کے  نو کا  پہاڑا حاصل کرنا ہے۔حافظہ کو  ثنائی مرموز اعشاری  روپ میں \عددی{0} تا \عددی{9} اعشاری عدد بطور پتہ فراہم کیا جائے گا۔حافظہ نے مواد  بِٹ پر جواب ثنائی  مرموز اعشاری  روپ میں پیش کرنا ہے۔مثلاً ا، گر اسے دو \عددی{(0010_2)} فراہم کیا جائے تو یہ اٹھارہ  \عددی{(00011000_2)} خارج کرے۔ (ا) حافظہ میں لکھا مواد جدول کی شکل میں لکھیں۔ (ب) حافظہ میں کتنے مقام غیر مستعمل ہوں گے؟ 
\انتہا{سوال}
\ابتدا{سوال}
%9.7
 چار بٹ ثنائی عدد  میں  \عددی{1} کی تعداد جاننا مقصود ہے۔ اس کام کے لئے \عددی{16\times 4} حافظہ استعمال کیا جاتا ہے۔حافظہ کو ثنائی عدد بطور پتہ مہیا کیا جاتا ہے۔حافظہ نے  اس عدد میں  \عددی{1} کی تعداد بطور مواد خارج کرنا ہے۔یوں ا گر   \عددی{1011} فراہم کیا جائے تو   \عددی{0011_2} وصول ہو گا۔ حافظہ میں لکھا گیا مواد جدول میں لکھیں۔
\انتہا{سوال}
\ابتدا{سوال}
%9.8
 انٹرنیٹ سے   (ا) \عددی{2708}، (ب) \عددی{2732}،  (ج) \عددی{2764}،  (د)  \عددی{27256}، (ہ)  \عددی{6116}، اور  (و)   \عددی{62256} حافظوں  کے  معلوماتی صفحات  حاصل کر کے ان کی قسم (یعنی پختہ یا عارضی)، جسامت اور دورانیہ رسائی دریافت کریں۔ (یہ حافظے مختلف دورانیہ رسائی کی صلاحیت کے لئے دستیاب ہیں۔)
\انتہا{سوال}

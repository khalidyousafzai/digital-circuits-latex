\باب{سوالات}
\ابتدا{سوال}
درج ذیل اعشاری اعداد کو ثنائی روپ  میں لکھیں۔
(ا)	\عددی{33}	(ب)	\عددی{64}	(پ)	\عددی{128}	(ت)	\عددی{256}
(ٹ)	\عددی{4096}	(ث)	\عددی{0.375}	(ج)	\عددی{5.625}	(چ)	\عددی{13.6875}
\انتہا{سوال}
\ابتدا{سوال}
درج ذیل ثنائی اعداد کو اعشاری روپ میں لکھیں۔
(ا)		\عددی{10}		(ب)		\عددی{101}	
(پ)		\عددی{1101}		(ت)		\عددی{11011}
(ٹ)		\عددی{101101011}	(ث)		\عددی{11001010011}		
\انتہا{سوال}
\ابتدا{سوال}
درج ذیل ثنائی اعداد کو اعشاری روپ  میں لکھیں۔
(ا)		\عددی{10.1}		(ب)		\عددی{101.01}
(پ)		\عددی{0.001101}	(ت)		\عددی{1011.01101}
(ٹ)		\عددی{100.001}	(ث)		\عددی{1111.1111}
\انتہا{سوال}
\ابتدا{سوال}
درج ذیل اعشاری اعداد کو اساس سولہ اور اساس آٹھ میں تبدیل کریں۔
(ا)	\عددی{7}	(ب)	\عددی{23}	(پ)	\عددی{32}	(ت)	\عددی{64}		
(ٹ)	\عددی{1024}	(ث)	\عددی{2048}
\انتہا{سوال}
\ابتدا{سوال}
درج ذیل اساس سولہ اعداد کو اساس آٹھ اور ثنائی   روپ  میں لکھیں۔
(ا)	\عددی{7}	(ب)	\عددی{10}	(پ)	\عددی{1A}	(ت)	\عددی{2B3}	
(ٹ)	\عددی{A.BC}	(ث)	\عددی{0.12}	(ج)	\عددی{F0}	(چ)	\عددی{FFFF}	
\انتہا{سوال}
%=======================
%2.1
\ابتدا{سوال}
درج ذیل ثنائی  مجموعے حاصل کریں۔ان سوالات کو اعشاری  روپ  میں بھی حل کریں۔جوابات کا موازنہ کریں۔
(ا)		\عددی{101+110}		(ب)	\عددی{101+11}	
(پ)		\عددی{1101+1011}		(ت)	\عددی{1001+1101}			
(ٹ)		\عددی{1011+101}		(ث)	\عددی{1111+101}		
\انتہا{سوال}
\ابتدا{سوال}
درج ذیل ثنائی اعداد کے سوالات حل کریں۔ان سوالات کو اعشاری  روپ میں بھی حل کریں۔جوابات کا موازنہ کریں۔
(ا)		\عددی{101-110}		(ب)	\عددی{101-111}	
(پ)		\عددی{1101-1111}		(ت)	\عددی{1001-1101}			
(ٹ)		\عددی{1011-101}		(ث)	\عددی{1111-101}
\انتہا{سوال}
\ابتدا{سوال}
درج ذیل ثنائی اعداد کے سوالات حل کریں۔انہیں سوالات کو اعشاری  روپ  میں بھی حل کریں۔جوابات کا موازنہ کریں۔
(ا)		\عددی{10.1-110}		(ب)	\عددی{10.1-101}	
(پ)		\عددی{1.101-11.11}		(ت)	\عددی{10.01-110.1}			
(ٹ)		\عددی{10.11-101.011}	(ث)	\عددی{11.01-111.1}
\انتہا{سوال}
\ابتدا{سوال}
درج ذیل اعشاری سوالات کو ثنائی روپ میں تبدیل کر کے حل کریں۔
(ا)		\عددی{32+64}		(ب)		\عددی{128-256}	
(پ)		\عددی{94.3-121.2}	(ت)		\عددی{22.24+36.09}		
(ٹ)		\عددی{63-1024}	(ث)		\عددی{1024+2056}				
\انتہا{سوال}
\ابتدا{سوال}
درج ذیل اعشاری اعداد کا تکملہ نو اور تکملہ دس حاصل کریں۔
(ا)		\عددی{6}		(ب)			\عددی{8}	
(پ)		\عددی{19}		(ت)			\عددی{205}	
(ٹ)		\عددی{3160029}	(ث)			\عددی{9807568}	
(ج)		\عددی{0.63}		(چ)			\عددی{39.09}	
(ح)		\عددی{3093.9801}	(خ)			\عددی{23409.65487}	
\انتہا{سوال}
\ابتدا{سوال}
درج ذیل ثنائی اعداد کا  تکملہ ایک اور تکملہ دو حاصل کریں۔
(ا)		\عددی{1011}		(ب)			\عددی{1001}	
(پ)		\عددی{111101}	(ت)			\عددی{10101010}	
(ٹ)		\عددی{11.11}		(ث)			\عددی{1101.0011}	
\انتہا{سوال}
\ابتدا{سوال}
درج ذیل اعشاری سوالات کو تکملہ نو اور تکملہ دس  استعمال کرتے ہوئے  حل کریں۔
(ا)		\عددی{4-9}		(ب)			\عددی{9-16}	
(پ)		\عددی{13-23.9}	(ت)		\عددی{303.93-555.078}	
(ٹ)		\عددی{0.045-0.555}	(ث)		\عددی{909.5301-1000}	
\انتہا{سوال}
\ابتدا{سوال}
درج ذیل ثنائی سوالات کو تکملہ ایک اور تکملہ دو سے حل کریں۔
(ا)		\عددی{10-11}		(ب)		\عددی{1010-1101}	
(پ)		\عددی{10.11-11.10}	(ت)		\عددی{1001.1-1101.01}	
(ٹ)		\عددی{1010-101}	(ث)		\عددی{1101.11-0.11}	
\انتہا{سوال}
\ابتدا{سوال}
درج ذیل اعشاری سوالات کو ثنائی روپ  میں تبدیل کر کے حل کریں-(ا) \عددی{3\times 9}			(ب) \عددی{31\times 23}			
(پ)	\عددی{15\times 3.625}		(ت)	 \عددی{1024\times 16}	
(ٹ)	\عددی{2048\times 2048}	(ث) \عددی{65.75\times 11.625}	
\انتہا{سوال}
%================
%3.1
\ابتدا{سوال}	
درج ذیل بوولین مساوات کا جدول لکھیں۔
(ا)	\عددی{XYZ+\overline{X}Y\overline{Z}}	(ب) \عددی{ABC+A\overline{B}C+\overline{A}\,\overline{B}C}			
(پ)	\عددی{A(B+\overline{C})}		(ت)	\عددی{(A+B)(AB+BC+\overline{C}A)}		
(ٹ)	\عددی{A\overline{B}+\overline{A}B}		(ث)	\عددی{A\overline{B}+B\overline{C}}	
\انتہا{سوال}
\ابتدا{سوال}
%3.2	
تفاعل   \عددی{AB+C\overline{D}} کی تکملی شکل \عددی{\overline{AB+C\overline{D}}=(\overline{A}+\overline{B})(\overline{C}+D)} ہے۔ درج  ذیل کی تکملی شکل معلوم کریں۔
(ا)	  \عددی{X+YZ+XY}	(ب)   \عددی{AB(C\overline{D}+\overline{C}D)}
(پ)		  \عددی{\overline{A}\,\overline{B}+A\overline{B}}	(ت)	  \عددی{X\overline{Y}Z+\overline{X}Y}	
(ٹ)   \عددی{(A+B)(B+C)(C+A)}
\انتہا{سوال}
\ابتدا{سوال}	
%3.3
درج ذیل کے ادوار جمع، ضرب اور نفی گیٹوں کی مدد سے بنائیں۔
(ا)	\عددی{AB\overline{C}+\overline{A}\,\overline{B}C}	(ب)	\عددی{A+B(A+\overline{C})}
(پ)	\عددی{\overline{X}\,\overline{Y}(X+\overline{Y})}	(ت)	\عددی{AB+BC+CA}
(ٹ)	\عددی{ABC+\overline{A}B\overline{C}+AB\overline{C}}
\انتہا{سوال}
\ابتدا{سوال}
%3.4	
ڈی مارگن  کلیات کو بوولین جدول سے اخذ کرنے کے طریقہ سے ثابت کریں۔
\انتہا{سوال}
\ابتدا{سوال}
%3.5	
بوولین جدول سے اخذ کرنے کے طریقہ سے درج ذیل ثابت کریں۔
(ا)	\عددی{X\overline{Y}+XY=X}	(ب)	\عددی{X+\overline{X}Y=X+Y}
\انتہا{سوال}
\ابتدا{سوال}
%3.6	
درج ذیل کو مجموعہ ارکان ضرب کی شکل میں لکھیں۔	
(ا)		\عددی{(A+B)(C+D)}	(ب)	\عددی{(A+B)(\overline{B}+C)(A+\overline{C})}
(پ) \عددی{(A+B)(A+B+C)(C+B)}	(ت)	 \عددی{(A+B+C)(\overline{B}+\overline{C})}	
\انتہا{سوال}
\ابتدا{سوال}
%3.7	
درج ذیل کو ضرب   ارکان جمع کی شکل میں لکھیں۔
(ا)	\عددی{X+\overline{Y}Z+\overline{X}\,\overline{Z}}	(ب)	\عددی{XY+\overline{Z}X}		
(پ)	\عددی{X\overline{Y}(\overline{Y}\,\overline{Z}+YZ)}	(ت) \عددی{(A+B\overline{C})(\overline{A}B+\overline{B}A)}
\انتہا{سوال}
\ابتدا{سوال}
%3.8	
 تفاعل \عددی{Y}  درج ذیل صورتوں میں ایک \عددی{(1)}  کے برابر ہو گا۔اگر \عددی{A=0}، \عددی{B=0}،  اور \عددی{C=1}ہو  یا اگر \عددی{A=1}، \عددی{B=1}، اور \عددی{C=0}ہو اور یا اگر \عددی{A=1}، \عددی{B=1}، اور \عددی{C=1}ہو۔ان صورتوں کے علاوہ اس تفاعل کی قیمت صفر  \عددی{(0)} رہتی ہے۔ان معلومات  کو جدول کی شکل میں لکھ کر تفاعل کی مساوات مجموعہ ارکان  ضرب کے   روپ میں حاصل کریں۔
\انتہا{سوال}
\ابتدا{سوال}
%3.9	
گزشتہ سوال میں دیے گئے تفاعل  \عددی{Y} کو \اصطلاح{ ضرب  و جمع  }\فرہنگ{ضرب و جمع}\حاشیہب{AND-OR}\فرہنگ{AND-OR} دور کی شکل میں بنائیں۔یہی  تفاعل \اصطلاح{ضرب متمم  و ضرب متمم  }\فرہنگ{ضرب متمم و ضرب متمم}\حاشیہب{NAND-NAND}\فرہنگ{NAND-NAND} دور سے حاصل کریں۔
\انتہا{سوال}
\ابتدا{سوال}
%3.10	
تفاعل \عددی{Z}   کی قیمت درج ذیل صورتوں میں صفر   \عددی{(0)} ہے۔اگر \عددی{A=0}، \عددی{B=0}، اور \عددی{C=0}ہو یا اگر \عددی{A=1}، \عددی{B=0}، اور  \عددی{C=0}ہو یا اگر \عددی{A=1}، \عددی{B=1}، اور \عددی{C=0}ہو اور یا اگر \عددی{A=1}، \عددی{B=1}، اور  \عددی{C=1}ہو۔ان صورتوں کے علاوہ اس کی قیمت ایک  \عددی{(1)} رہتی ہے۔ان معلومات کو جدول کی شکل میں لکھ کر \عددی{Z}  کی مساوات ضرب  ارکان  جمع کے روپ  میں حاصل کریں۔
\انتہا{سوال}
\ابتدا{سوال}
%3.11
گزشتہ سوال میں دیے  گئے تفاعل  \عددی{Z} کا  جمع  و ضرب دور  بنائیں۔ اسی تفاعل کا   جمع متمم  و جمع متمم دور  بنائیں۔
\انتہا{سوال}
\ابتدا{سوال}
%3.12
جدول میں  \عددی{A}، \عددی{B}، اور \عددی{C} تین آزاد داخلی متغیرات  جبکہ  \عددی{F0}، \عددی{F1}, اور \عددی{F2} تابع متغیرات ہیں۔
\begin{center}
\begin{otherlanguage}{english}
\begin{tabular}{CCC|CCC}
\toprule
A&B&C&F0&F1&F2\\
\midrule
0&0&0&0&1&1\\
0&0&1&1&0&1\\
0&1&0&1&1&0\\
0&1&1&0&0&0\\
1&0&0&1&1&1\\
1&0&1&0&0&1\\
1&1&0&0&0&0\\
1&1&1&0&1&1\\
\bottomrule
\end{tabular}
\end{otherlanguage}
\end{center}
\begin{enumerate}[a.]
\item
تابع متغیرات کو باری باری مجموعہ ارکان ضرب کے روپ  میں لکھیں۔
\item
 منطقی ضرب گیٹ اور منطقی جمع گیٹ استعمال کرتے ہوئے  تابع متغیرات کے ضرب  و جمع دور بنائیں۔
\item
ضرب  و جمع ادوار سے تابع متغیرات کے ضرب متمم  و 	ضرب  متمم ادوار حاصل کریں۔
\item
 تابع متغیرات کو باری باری ضرب ارکان جمع کے روپ میں لکھیں۔
\item
 منطقی جمع گیٹ اور منطقی ضرب گیٹ استعمال کرتے ہوئے  تابع متغیرات  کے جمع   و ضرب دور بنائیں۔
\item
جمع  و  ضرب ادوار سے  تابع متغیرات کے جمع متمم  و جمع متمم  ادوار حاصل کریں۔
\end{enumerate}
\انتہا{سوال}
\ابتدا{سوال}
%3.13
درج ذیل تفاعل مجموعہ ارکان ضرب کی شکل میں ہیں۔انہیں ضرب ارکان جمع کی شکل میں لکھیں۔
(ا)	     \عددی{Z(A,B)=\sum(0,1)}  (ب)  \عددی{F(A,B,C)=\sum(1,3,7)}		
(پ) \عددی{F(A,B,C)=\sum (0,5,7)} (ت)  \عددی{Y(A,B,C)=\sum (0,7)}
(ٹ) \عددی{Z(A,B,C,D)=\sum (0,2,5,12)}
\انتہا{سوال}
%???KKK
\ابتدا{سوال}
%3.14
	درج ذیل تفاعل ضرب ارکان جمع کی شکل میں ہیں۔انہیں مجموعہ ارکان ضرب کی شکل میں لکھیں۔
(ا)	(ب)			
(پ)	
\انتہا{سوال}
\ابتدا{سوال}
%3.15
	انٹرنیٹ 3 سے درج ذیل معلوماتی صفحات 4 حاصل کریں۔یہ گیٹوں کے مخلوط ادوار5  پاکستان کے ہر شہر میں نہایت سستے داموں دستیاب ہیں۔(ا) 7400 (ب) 4011 (پ) 7408 (ت)4081 (ٹ) 4000 (ث)7432 (ج)7404 (چ)4049 (ح)4070  (مثال:7400 کے معلوماتی صفحات حاصل کرنے کی خاطر انٹرنیٹ کے گوگل6 میںلکھیں)
\انتہا{سوال}
\ابتدا{سوال}
%3.16
	گزشتہ سوال میں حاصل کئے گئے معلوماتی صفحات سے 7400 مخلوط دور میں چار گیٹوں کے مخارج کن پنوں پر دستیاب ہیں۔
\انتہا{سوال}
\ابتدا{سوال}
%3.17
	انٹرنیٹ سے تین مداخل والے ضرب گیٹ اور چار مداخل والے جمع گیٹ کے مخلوط ادوار دریافت کریں۔
\انتہا{سوال}
%====================
\ابتدا{سوال}
4.1	کارناف نقشے میں
\انتہا{سوال}
%=============
\ابتدا{سوال}
5.1	شکل 12.1 میں چار مداخل والا دور دیا گیا ہے۔

(ا)	شکل 12.1 کے اندرونی متغیراتاورکے بوولین مساوات 	حاصل 	کریں۔
(ب)	اسی شکل میں خارجی تابع متغیرہکی بوولین مساوات حاصل کریں۔
(پ)	ایک بوولین جدول بنائیں جس میں چار آزاد متغیرات،، 	اورکے تمام ممکنہ ترتیب درج ہوں۔اس بوولین جدول میں،		اورکے خانے بنائیں اور انہیں پُر کریں۔    
\انتہا{سوال}
\ابتدا{سوال}
5.2	ایک ایسا بوولین جدول بنائیں جس کے تین مداخل اور ایک مخارج ہو۔	اس جدول کو یوں پُر کریں کہ مخارج کی قیمت صرف اُس صورت میں ایک 	 کے برابر ہو جب صرف ایک مداخل کی قیمت صفر 	ہو۔اس جدول کی مدد سے مخارج کا ترکیبی دور تشکیل دیں۔
\انتہا{سوال}
\ابتدا{سوال}
5.3	چار مداخل کا ایک ایسا بوولین جدول بنائیں جس کا مخارج صرف اُس 	صورت بلند  ہو جب مخارج ثنائی عدد کی قیمت نو  سے 	کم ہو۔اس تفاعل کا ترکیبی دور تشکیل دیں۔ 
\انتہا{سوال}
\ابتدا{سوال}
5.4	تین مداخل اور تین مخارج والا ایک ایسا بوولین جدول تشکیل دیں جس 	میں مداخل ثنائی عدد کی قیمت سات  سے کم ہونے کی صورت 	میں مخارج کی قیمت مداخل سے ایک زیادہ ہو جبکہ مداخل کی قیمت 	سات کے برابر ہونے کی صورت میں مخارج کی قیمت صفر 	ہو۔
\انتہا{سوال}
\ابتدا{سوال}
5.5	اقلیتی دور ایسے ترکیبی دور کو کہتے ہیں جس کا مداخل اس صورت بلند	 ہوتا ہے جب اس کے زیادہ تر مداخل پست ہوں۔تین 	مداخل والا اقلیتی دور تشکیل دیں۔
\انتہا{سوال}
\ابتدا{سوال}
5.6	ایک ترکیبی دور تشکیل دیں جو اعشاری ہندسے کا اساس نو خارج 	کرے۔ایسے دور کے چار مداخل اور چار مخارج ہوں گے۔
\انتہا{سوال}
\ابتدا{سوال}
5.7	تین بٹ کے دو اعداد کا موازنہ کرنے والا ایسا ترکیبی دور تشکیل دیں 	جس کا مخارج اس صورت بلند ہو جب دونوں اعداد کی قیمتیں 	برابر ہوں۔
\انتہا{سوال}
\ابتدا{سوال}
5.8	چار بٹ کے دو ثنائی اعداد ضرب کرنے والا ترکیبی دور تشکیل دیں۔
\انتہا{سوال}
\ابتدا{سوال}
5.9	جمع متمم گیٹ استعمال کرتے  شناخت کار تشکیل دیں۔
\انتہا{سوال}
\ابتدا{سوال}
5.10	درج ذیل تین تفاعل کو ایک عدد  شناخت کار کی مدد سے حاصل کریں۔اس دور کو شکل 5.25 کی طرز پر تشکیل دیں۔
 \انتہا{سوال}
\ابتدا{سوال}
\انتہا{سوال}
\ابتدا{سوال}
5.11	درج ذیل تفاعل کو داخلی منتخب کار  کی مدد سے حاصل 	کریں۔
\انتہا{سوال}
\ابتدا{سوال}
5.12	مکمل جمع کار کو دو عدد داخلی منتخب کار کی مدد سے حاصل کریں۔
\انتہا{سوال}
\ابتدا{سوال}
5.13	شکل 12.2 میں اعشاری ہندسے کی نمائش کرنے والی تختی7 دکھائی 	گئی ہے جو  سات قابل روشن حصوں پر مبنی ہے۔ان حصوں میں سے 	کسی ایک یا ایک سے زیادہ حصوں کو  بیک وقت روشن کیا جا سکتا 	ہے۔یوں مختلف	حصے روشن کرنے سے اعشاری ہندسے لکھے جا 	سکتے ہیں۔مثلاً	(ب) اور (پ)  بیک وقت روشن کرنے سے  	لکھا جائے گا۔اسی طرح (ا)، (ب)، (پ)، (ت)، (ٹ) اور (ث) بیک 	وقت روشن کرنے سےلکھا جا سکتا ہے۔
		فرض کریں کہ کسی بھی حصے کو بلند  کرنے سے یہ 	حصہ روشن ہو جاتا ہے۔چار مداخل اور سات مخارج والا  ایسا ترکیبی 	دور تشکیل دیں جو مہیا کردہ اعشاری ہندسے کو اس تختی پر دکھلائے۔	اعشاری ہندسہ کو ثنائی علامتی روپ میں مہیا کیا جائے گا۔
		مخلوط دور یہی کام سر انجام دیتا ہے۔
\انتہا{سوال}
\ابتدا{سوال}
5.14	انٹرنیٹ سے سات حصوں والی نمائشی تختی کے معلوماتی صفحات حاصل کریں۔ایس کرنے کی خاطر گوگل میںلکھیں۔
\انتہا{سوال}
%=================
\ابتدا{سوال}
6.1	ثابت کریں کہ جے-کے پلٹ کے مخارجکی مساوات 	یوں ہے
\انتہا{سوال}
\ابتدا{سوال}      	
6.2	شکل میں ضرب گیٹ کا دورانیہ رد عمل نینو سیکنڈ جبکہ جمع گیٹ کانینو سیکنڈ ہے۔تینوں مداخل بیک وقت  تبدیل کئے جاتے ہیں۔  کتنی دیر بعد مخارجاورمتوازن حالت میں ہوں گے۔ (جواب: ،)
\انتہا{سوال}
\ابتدا{سوال}
6.3	ایک کمپیوٹر کے ساعتی اشارہ سے چلتا ہے۔یہ اشارہ تیس فی صد وقت بلند رہتا ہے جبکہ اس کے دورانیہ اترائی اور دورانیہ چڑھائی دونوں پانچ پانچ فی صد وقت لیتے ہیں۔ساعتی اشارہ کا دوری عرصہ ،دورانیہ چڑھائی اور پست دورانیہ حاصل کریں۔ (جواب: ، ،)
\انتہا{سوال}
\ابتدا{سوال}
           6.4	جمع متمم گیٹوں پر مبنی زیادہ مداخل والے ایس-آر پلٹ کے مداخل کو گراف میں تبدیل ہوتے دکھایا گیا ہے۔اس کا مخارج کھینچے۔
\انتہا{سوال}
\ابتدا{سوال}                      
6.5	آقا-غلام پلٹ کے مداخل گراف کئے گئے ہیں۔اورگراف کریں۔
\انتہا{سوال}
\ابتدا{سوال}
6.6	شکل 6.25 میں سلسلہ وار ثنائی جمع کار دکھایا گیا ہے۔اسے استعمال کرتے ہوئے اور کو جمع کریں۔
\انتہا{سوال}
\ابتدا{سوال}
6.7	،مداخل اورمخارج والا ایک ترتیبی دور جس میں دو ڈی پلٹ، اوراستعمال ہوئے ہیں کے مساوات درج ذیل ہیں۔
\انتہا{سوال}
\ابتدا{سوال}
(ا) اس ترتیبی دور کی شکل بنائیں۔ (ب) ان مساوات سے حالات کا جدول حاصل کریں۔ (ج) حالات کے جدول سے حالات کا خاکہ حاصل کریں۔
\انتہا{سوال}
\ابتدا{سوال}
6.8	ایک مداخلاور دو عدد جے-کے پلٹاورپر مبنی ترتیبی دور کے درج ذیل مساوات ہیں۔
(ا) ان سے حالات کے مساواتاور حاصل کریں۔(ب) ان کے حالات کا خاکہ بنائیں۔
\انتہا{سوال}
\ابتدا{سوال}
6.9	دو عدد ڈی پلٹ،اور، استعمال کرتے ہوئے ایک مداخل،، والا ایسا ترتیبی دور تخلیق دیں جو بلترتیب  ،،اورحالتیں اختیار کر سکے۔اگر مداخل بلند ہو تو یہ اوپر جانب کی طرف بڑھے اور اگر مداخل پست ہو تو یہ نیچے کی جانب بڑھے۔اُوپر جانب تک پہنچنے کے بعد مداخل بلند ہونے کی صورت میں یہ اسی حالت میں رہے۔اسی طرح نیچے جانب حالت پہنچ کر مداخل پست ہونے کی صورت یہ اسی حالت میں رہے۔
\انتہا{سوال}
\ابتدا{سوال}
6.10	پچھلے سوال میں مداخلکا اضافہ کریں۔اگر یہ مداخل بلند ہو تو دور بالکل اسی طرح کام کرے جیسے پہلے کرتا تھا جبکہ اگر یہ مداخل پست ہو تو دور جس حالت میں ہو اسی میں رہے۔
\انتہا{سوال}
\ابتدا{سوال}
6.11	پچھلے سوال کے مداخل کی تعداد میں مزید اضافہ کرتے ہوئے مداخل کا اضافہ کریں۔ بلند کرنے سے دور کوحالت اختیار کر لینا چاہیے جبکہپست ہونے کی صورت میں دور بالکل پہلے کی طرح کام کرے۔
\انتہا{سوال}
%==============
\ابتدا{سوال}
7.1	چار بٹ کے سلسلہ وار دائیں منتقل کھاتے میں ابتدائی ثنائی موادموجود ہے۔اس کھاتے کے مخارج کو اسی کھاتے کو بطور مداخل مہیا کیا جاتا ہے۔سات گھڑی کے کنارے گزرنے کے بعد کھاتے میں کیا عدد ہو گا۔
\انتہا{سوال}
\ابتدا{سوال}
7.2	 گزشتہ سوال میں دائیں منتقل کھاتے کے بجائے بائیں منتقل کھاتا استعمال کرتے جواب معلوم کریں۔
\انتہا{سوال}
\ابتدا{سوال}
7.3	گزشتہ دو سوالات میں ہر کنارہ ساعت پر کھاتے میں ثنائی عدد حاصل کریں۔
\انتہا{سوال}
\ابتدا{سوال}
7.4	آٹھ بٹ کے سلسلہ وار دائیں منتقل کھاتے کے مخارج کو چار بٹ کے سلالہ وار دائیں منتقل کھاتے کو بطور مداخل فراہم کیا جاتا ہے۔آٹھ بٹ کھاتے میں ابتدائی موادپایا جاتا ہے جبکہ اسے بطور مداخل متواترفراہم کیا جاتا ہے۔ساعت کے سات کنارے گزرنے کے بعد ان کھاتوں میں کیا اعداد پائے جائیں گے۔
\انتہا{سوال}
\ابتدا{سوال}
7.5	گزشتہ سوال میں چار بٹ کا سلسلہ وار بائیں منتقل کھاتا استعمال کرتے 	ہوئے جواب حاصل کریں۔
\انتہا{سوال}
\ابتدا{سوال}
7.6	آٹھ بٹ کے دو عدد عالمگیر کھاتے استعمال کرتے ہوئے سولہ بٹ کا عالمگیر کھاتا حاصل کریں۔
\انتہا{سوال}
\ابتدا{سوال}
7.7	شکل 7.7 میں سلسلہ وار ثنائی جمع کار دکھایا گیا ہے۔اگر اس شکل میں کھاتا۔ا اور کھاتا-ب دونوں آٹھ آٹھ بٹ کے ہوں اور ان میں ابتدائی ثنائی مواداورپائے جائیں۔تصور کریں کہ ساعت کے آٹھ کنارے گزرتے ہیں۔ساعت کے ہر کنارہ گزرنے کے بعد کھاتا-ا  میں موجود مواد کیا ہو گا۔
\انتہا{سوال}
\ابتدا{سوال}
7.8	سلسلہ وار ثنائی جمع کار سے سلسلہ وار ثنائی منفی کار حاصل کریں۔ایسا کرنے کی خاطر منفی ہونے والے عدد کے تکملہ کو کھاتا-ب میں متوازی لکھنا بھی دکھائیں۔
\انتہا{سوال}
%===============
\ابتدا{سوال}
8.1	چار بٹ معاصر سیدھا گنت کار کی موجودہ گنتیہے۔ساعت کے کتنے کناروں کے بعد یہدے گا۔
\انتہا{سوال}
\ابتدا{سوال}
8.2	سولہ بٹ معاصر گنت کار کی موجودہ گنتیہے۔یہ ساعت کے کتنے کنارے گزرنے کے بعدپڑھے گا۔(ا) تصور کریں کہ یہ سیدھا گنت کار ہے۔ (ب) تصور کریں کہ یہ الٹ گنت کار ہے۔
\انتہا{سوال}
\ابتدا{سوال}
8.3	چار بٹ ثنائی لہر نما گنت کار کو استعمال کرتے ہوئے اعشاری اعداد کے ثنائی علامتی روپ8 کا گنت کار بنایا جا سکتا ہے۔ایسا کرنے کی خاطر ثنائی گنت کار کوکے گنتی پر پہنچتے ہی زبردستی پست کر دیا جاتا ہے۔ایک عدد ضرب متمم گیٹ کے استعمال سے ایسا کرنا ممکن ہوتا ہے۔زبردستی پست صلاحیت رکھنے والے پلٹ استعمال کرتے ہوئے یہ دور تخلیق دیں۔ 
\انتہا{سوال}
\ابتدا{سوال}
8.4	ڈی پلٹ استعمال کرتے ہوئے چار بٹ معاصر ثنائی گنت کار تشکیل دیں۔ 
\انتہا{سوال}
\ابتدا{سوال}
8.5	جے-کے پلٹ استعمال کرتے ہوئے ایسا معاصر گنت کار تشکیل دیں جو اس ترتیب کو دہرائے۔،،اور۔
\انتہا{سوال}
\ابتدا{سوال}
8.6	ٹی پلٹ استعمال کرتے ہوئے چار بٹ کا ثنائی معاصر گنت کار تشکیل دیں جو صفر سے چودہ تک کی جفت گنتی کے بعد ایک سے پندرہ تک طاق گنتی کرنے کے بعد اسی طرح ان دو ترتیب کو دہراتا رہے۔
\انتہا{سوال}
\ابتدا{سوال}
8.7	شکل 8.11 میں دورانیہ پیدا کار دکھایا گیا ہے۔اگر ساعت کا تعددہو تبکے دورانیہ کے لئے درکار دورانیہ کے تین بٹ کیا ہوں گے۔
\انتہا{سوال}
\ابتدا{سوال}
8.8	کارناف نقشوں کے استعمال سے مساوات 8.3 حاصل کریں۔
\انتہا{سوال}
\ابتدا{سوال}
8.9	مساوات 8.3 کے متبادل مساوات جے-کے پلٹ کی خاطر حاصل کریں۔ 
\انتہا{سوال}
%=============
\ابتدا{سوال}
9.1	درج ذیل مختلف جسامت کے حافظہ میں پتہ بٹوں کی تعداد درج ذیل ہے۔ان حافظہ میں الفاظ ذخیرہ کرنے کے کتنے مقام ہیں۔ (ا) (ب) (ج) (د)
\انتہا{سوال}
\ابتدا{سوال}
9.2	حافظہ  کے جسامت کو عموماًلکھا اور پکارا جاتا ہے جہاںاس حافظہ میں الفاظ کی تعداد اورایک لفظ میں بٹ کی تعداد بتلاتے ہیں۔یوں درج ذیل حافظہ میں پتہ کے لئے درکار پن اور ثنائی مواد کے لئے درکار پن کیا ہوں گے۔ (ا) (ب) (ج) (د)
\انتہا{سوال}
\ابتدا{سوال}
9.3	کسی حافظہ کے پتہ پرمواد لکھا ہوا ہے۔اس تک رسائی کے لئے سولہ پتہ بٹ کیا ہوں گے اور اس سے پڑھے جانے والے آٹھ مواد بٹ کیا ہوں گے۔
\انتہا{سوال}
\ابتدا{سوال}
9.4	چار عددحافظہ اور یک عددشناخت کار کی مدد سےحافظہ حاصل کریں۔
\انتہا{سوال}
\ابتدا{سوال}
9.5	دو عددحافظہ کے استعمال سےحافظہ حاصل کریں۔
\انتہا{سوال}
\ابتدا{سوال}
9.6	چار پتہ بٹ اور آٹھ مواد بٹ والے حافظہ کو استعمال کرتے ہوئے نو کا پھاڑا حاصل کرنا ہے۔حافظہ کو ثنائی علامتی روپ میںتاکا اعشاری عدد بطور پتہ فراہم کیا جائے گا۔حافظہ نے مواد بٹوں پر جواب ثنائی علامتی روپ کی شکل میں پیش کرنا ہے۔مثلاً اگر اسے دوفراہم کیا جائے تو یہ اٹھارہ خارج کرے۔ (ا) حافظہ میں لکھی جانے والے مواد کو جدول کی شکل میں لکھیں۔ (ب) حافظہ میں کتنی جگہ باقی رہ جائے گی۔ 
\انتہا{سوال}
\ابتدا{سوال}
9.7	حافظہ استعمال کرتے دئے گئے چار بٹ ثنائی عدد میںکی تعداد معلوم کرنی ہے۔حافظہ کو ثنائی عدد بطور پتہ مہیا کیا جاتا ہے۔حافظہ نے دئے گئے ثنائی عدد میںکی تعداد بطور مواد خارج کرنی ہے۔مثلاً اگر اسےفراہم کیا جائے تو یہیعنی تین خارج کرے۔
\انتہا{سوال}
\ابتدا{سوال}
9.8	انٹرنیٹ سے درج ذیل حافظہ کے معلوماتی  حاصل کر کے ان کی قسم (یعنی پختہ یا عارضی)، جسامت اور دورانیہ رسائی دریافت کریں۔یہ تمام حافظہ مختلف دورانیہ رسائی کی صلاحیت کے لئے دستیاب ہیں۔ (ا)(ب)(پ)(ت)  (ٹ) (ث)(مثال: انٹرنیٹ سےکی معلومات حاصل کرنے کی خاطر گوگل میںلکھیں)
\انتہا{سوال}

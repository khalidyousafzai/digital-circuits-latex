\باب{ حافظہ}
	پلٹ ایک بِٹ معلومات کو ذخیرہ کرنے کی صلاحیت رکھتا ہے۔یوں ایک پلٹ ایک بِٹ حافظہ 1 کے طور کام کر سکتا ہے۔آٹھ پلٹ جوڑ کر آٹھ بِٹ کا حافظہ حاصل کیا جا سکتا ہے۔اسی طرحبِٹ پلٹ سے بِٹ حافظہ بنایا جا سکتا ہے۔آٹھ بِٹ کو ایک ہشتمی عدد یا ایک بائٹ 2 کہتے ہیں۔حافظہ میں کسی بھی مقام پر رکھے جانے والے مواد کو لفظ 3 کہتے ہیں۔حافظہ میں الفاظ کی لمبائی قطعی ہوتی ہے۔ یوں آٹھ بِٹ لفظ ایک بائٹ پر مشتمل ہو گا جبکہ سولہ بِٹ لفظ دو بائٹ پر مشتمل ہو گا۔کمپیوٹر میں موجود کُل حافظہ کی جسامت بائٹ میں بیان کی جاتی ہے۔یوں آٹھ بِٹ لفظوں والی دو سو الفاظ کے جسامت والے حافظہ کودو سو بائٹ کا حافظہ کہیں گے۔حافظہ میں مواد داخل کرنے کو مواد لکھنا 4 کہتے ہیں جبکہ اس سے مواد کے حصول کو مواد پڑھنا 5 کہتے ہیں۔اس باب میں انہیں قسم کے الیکٹرانک حافظہ پر غور کیا جائے گا۔
	حافظہ کے دو اہم اقسام ہیں۔حافظہ کی پہلی قسم میں معلومات اس وقت تک محفوظ رہتی ہے جتنی دیر حافظہ کو درکار برقی طاقت مہیا کی جائے۔اس طرح کے حافظہ کو عارضی حافظہ 6 کہتے ہیں۔ عارضی حافظہ میں معلومات کسی بھی وقت، حافظہ کے اندر کسی بھی جگہ، لکھی جا سکتی ہے یا اسے یہاں سے پڑھا جا سکتا ہے۔معلومات کا، حافظہ میں کسی بھی جگہ، لکھنے یا یہاں سے پڑھنے کے لئے درکار وقت تمام جگہوں کے لئے تقریباً برابر ہوتا ہے۔اس دورانیہ کو حافظہ کا دورانیہ رسائی 7 یا صرف دورانیہ رسائی کہتے ہیں۔یوں عارضی حافظہ میں مواد لکھی بھی جا سکتی ہے اور اس سے پڑھی بھی جا سکتی ہے۔
	دوسری قسم کی حافظہ وہ ہے جس میں برقی طاقت کی عدم موجودگی میں بھی اس میں مواد محفوظ رہتا ہے تاہم اس میں معلومات پڑھنے کی خاطر حافظہ کو درکار برقی طاقت فراہم کرنا لازم  ہوتا ہے۔اس قسم کے حافظہ کو پختہ حافظہ 8 کہتے ہیں۔پختہ حافظہ میں معلومات کسی بھی وقت، حافظہ کے اندر کسی بھی جگہ سے، پڑھی جا سکتی ہے۔معلومات کا، حافظہ میں کسی بھی جگہ سے، حصول کا وقت تمام جگہوں کے لئے تقریباً برابر ہوتا ہے اور اسے حافظہ کا دورانیہ رسائی کہتے ہیں۔عام استعمال میں پختہ حافظہ سے معلومات صرف پڑھی جاتی ہے۔پختہ حافظہ کی مختلف اقسام میں معلومات محفوظ کرانے کے طریقے مختلف ہیں۔ایک قسم میں معلومات صرف اور صرف ایک مرتبہ لکھی جا سکتی ہے۔یوں یہ صرف ایک مرتبہ معلومات کی لکھائی کے لئے استعمال ہو سکتا ہے۔اسے ایک مرتبہ لکھنے کے قابل پختہ حافظہ 9 کہتے ہیں۔دوسری قسم کی پختہ حافظہ کو دوبارہ معلومات لکھنے کے لئے استعمال کیا جا سکتا ہے لیکن ایسا کرنے سے پہلے اس سے پرانی معلومات صاف کرنی ضروری ہے۔جدید پختہ حافظہ کو برقی دباؤ کی مدد سے صاف کیا جا سکتا ہے۔ایسے پختہ حافظہ کو برقی دباؤ سے صاف ہونے والا پختہ حافظہ 10 کہتے ہیں۔اس سے قبل پختہ حافظہ کی ایک قسم کو شعائیں کی مدد سے صاف کیا جاتا تھا۔اسے شعائیں سے صاف ہونے والا پختہ حافظہ 11 کہتے ہیں۔
9.1 عارضی حافظہ
	اس حصہ میں عارضی حافظہ کی بناوٹ پر غور کیا جائے گا۔ایک بِٹ حافظہ بنیادی طور ایک پلٹ ہوتا ہے جس میں مواد لکھنے اور اس میں سے مواد پڑھنے کی صلاحیت موجود ہو۔چونکہ حافظہ عموماً کثیر تعداد کے بٹوں پر مشتمل ہوتا ہے لہٰذا حافظہ میں ہر پلٹ تک لکھنے اور پڑھنے کی خاطر رسائی ضروری ہوتی ہے۔شکل 9.1 میں ثنائی عارضی حافظہ کے اکائی 12 کی بناوٹ اور علامت دکھائی گئی ہے جس میں مندرجہ بالا تمام خاصیت موجود ہیں۔شکل میں مواد ذخیرہ کرنے کے لئے ایس-آر پلٹ استعمال کیا دکھایا گیا ہے۔حقیقت میں کئی طریقہ استعمال کئے جاتے ہیں جنہیں بعد میں بتلایا جائے گا۔ 


	اس اکائی حافظہ سے رجوع کرنے کی خاطر منتخب اشارہ 13 کو بلند کیا جاتا ہے۔ایسا کرنے کے بعد، اس میں مواد لکھنے کی خاطرکو پست کر کے اسے داخلی مواد فراہم کیا جاتا ہے جبکہ اس سے مواد پڑھنے کی خاطر کو بلند کیا جاتا ہے اور اس سے مواد پڑھی جاتی ہے۔
	زیادہ بِٹ کا حافظہ اسی اکائی حافظہ کی مدد سے حاصل کیا جاتا ہے۔شکل 9.2 میں چار بِٹ کے ایک لفظ کا حافظہ دکھایا گیا ہے جس میں تمام اکائی حافظہ کے  اشارات ایک ساتھ جوڑے گئے ہیں اور اسی طرح تمام کے اشارات ایک ساتھ جوڑے گئے ہیں۔یوں اس لفظ کے چاروں بِٹ بیک وقت منتخب ہوتے ہیں اور اس میں موادبیک وقت لکھا یا اس میں ذخیرہ مواد بیک وقت پڑھا جاتا ہے۔

	ایک قدم اور آگے بڑھتے ہیں اور اس طرح کے کئی الفاظ جوڑ کر زیادہ الفاظ کا حافظہ حاصل کرتے ہیں۔شکل 9.3 میں چار الفاظ جوڑ کر حافظہ بنایا گیا ہے۔
	عام حالت میں تمام منتخب اشارات پست14 رہتے ہیں۔یوں حافظہ کے تمام الفاظ تک رسائی نا ممکن ہوتی ہے۔حافظہ میں مواد لکھنے کی خاطر مواد کو داخلی راستے فراہم کیا جاتا ہے،کو پست رکھ کر  مطلوبہ مقام کی منتخب اشارہ بلند کیا جاتا ہے۔یوں مواد مطلوبہ لفظ کے مقام پر لکھ لیا جاتا ہے۔مثلاً ہم چاہتے ہیں کہ اعشاری عدد تین یعنیکی ثنائی علامتی روپ یعنیکو حافظہ کے لفظ2 کے مقام پر لکھا جائے۔ایسا کرنے کی خاطر ہم مداخل پرمہیا کرتے ہوئے،کو پست کر دیں گے اور منتخب لفظ 2 کے اشارہ کو بلند کر دیں گے۔ایسا کرنے سے شکل میں لفظ 2 پر لکھ لیا جائے گا۔یاد رہے کہ اس دوران بقایا منتخب اشارات پست رہتے ہیں۔اسی لفظ کو پڑھنے کی خاطر ہم کو بلند رکھ کر منتخب لفظ 2 کا اشارہ بلند کریں گے۔ایسا کرتے ہی مخارجپرخارج ہو گا جسے یہاں سے پڑھا جا سکتا ہے۔

	حقیقی حافظہ میں الفاظ تک رسائی پتہ کے ذریعہ کیا جاتا ہے۔چار الفاظ تک رسائی، دو بِٹ پتہ  استعمال کرتے، دو سے چار شناخت کار کی مدد سے ممکن ہے۔شکل 9.4 میں ایسا ہی دکھایا گیا ہے۔


	حافظہ کے استعمال کو شکل 9.5 میں جدول کی صورت میں دکھایا گیا ہے۔مجاز پست ہونے کی صورت میں حافظہ کثیر مقاومت حالت اختیار کر کے بیرونی ادوار سے مکمل طور منقطع ہو جاتا ہے۔


	شکل 9.4 میں چار بِٹ جمع گیٹ کی ایک نئی علامت استعمال کی گئی ہے۔اس جمع گیٹ کی ایک ہی مداخل دکھائی گئی ہے جس پر چھوٹی ترچی لکیر کے ساتھلکھ کر اس بات کی وضاحت کی گئی ہے کہ دراصل یہ چار داخلی جمع گیٹ ہے۔اس طرح بنائے گئے ادوار میں گیٹوں کے مداخل کو علیحدہ علیحدہ نہیں دکھایا جاتا بلکہ اس کے تمام مداخل کو ایک ہی داخلی تار کے طور دکھایا جاتا ہے۔یوں دور کو کاغذ پر بناتے ہوئے تاروں کے ہجوم سے نجات حاصل ہو جاتی ہے اور شکل کچھ صاف ستری ہو جاتا ہے۔یاد رہے کہ ایسا صرف صاف شکل بنانے کی خاطر کیا جاتا ہے۔یوں حافظہ کے گزشتہ دو اشکال بالکل ایک ہی دور کو بنانے کے دو طریقے ہیں۔
	اسی طرز پر زیادہ الفاظ کے حافظہ بنائے جاتے ہیں۔دس بِٹ پتہ سے  یعنی مقام تک رسائی ممکن ہے۔کمپیوٹر میں اسی عدد کو ہزار کہتے ہیں۔یوں دو ہزار سے مراد ہو گا۔


	شکل 9.6 میں داخلی اور خارجی راہ کے مابین وسطی دور نصب کئے گئے ہیں۔یوں اگر مجاز اور اشارت دونوں بلند ہوں تب پر حافظہ میں ذخیرہ مواد خارج ہو گا جبکہ اگر مجاز بلند اور  پست ہو تبپر موجود مواد حافظہ میں لکھ لیا جائے گا۔یوںبطور مداخل-مخارج دونوں کام کرتا ہے۔
	جدید عارضی حافظہ میں لاتعداد الفاظ ذخیرہ کرنے کی گنجائش ہوتی ہے۔شکل 9.7 (ا) میں چار الفاظ حافظہ کے مخلوط دور 15 کی علامت دکھائی گئی ہے۔لفظ کے چار داخلی-خارجی بِٹوں 16 کوکے بجائےکہا گیا ہے۔
	شکل (ب) میں مجاز کی جگہاستعمال کیا گیا ہے۔ایسا شکل (ا) کے مجاز مداخل پر نفی گیٹ نصب کرنے سے حاصل کیا جا سکتا ہے۔مزید یہ کہکو ابکہا گیا ہے اور اس کے پن پر گول دائرہ لگا کر اس کے پست فعال 17 ہونے کو ظاہر کیا گیا ہے۔یوں اگرپست ہو تو حافظہ میں مواد لکھا جا ئے گا اور اگر یہ بلند ہو تب اس سے مواد پڑھا جائے گا۔
	شکل (ج) میں  بارہ بِٹ پتہ اور ایک بائٹ لمبے الفاظ کے عارضی حافظہ کی علامت دکھائی گئی ہے۔بارہ بِٹ پتہ سے بائٹ تک رسائی ممکن ہے۔یوں یہ چار کلو بائٹ18 کے عارضی حافظہ کے مخلوط دور کی علامت ہے۔اس مخلوط دور میں مداخل کا اضافہ کیا گیا ہے۔آئیں اس کو سمجھتے ہیں۔


	کسی بھی مخلوط دور میں لاتعداد گیٹ پائے جاتے ہیں اور کوئی بھی جدید الیکٹرانک آلا کئی مخلوط ادوار پر مشتمل ہوتا ہے۔یہ تمام کے تمام برقی طاقت سے چلتے ہیں۔ہم کہتے ہیں کہ برقی طاقت انہیں بیدار رکھتا ہے۔عام استعمال میں عموماً آلات بیٹری سے برقی طاقت حاصل کرتے ہیں اور اگر کسی طرح درکار برقی طاقت کو کم کیا جا سکے تو بیٹری زیادہ دیر کارآمد رہے گی۔
	کسی بھی الیکٹرانک آلا میں مختلف مخلوط ادوار کی مختلف لمحات پر ضرورت پڑتی ہے۔ان لمحات کے علاوہ اگر انہیں بیدار رکھا جائے تو یہ برقی توانائی کا استعمال کریں گے۔بلکہ ایسا کہنا بہتر ہوگا کہ اس دوران یہ برقی توانائی ضائع کریں گے۔ایسے اوقات نہ استعمال ہونے والے مخلوط ادوار کو برقی طاقت منقطع نہیں کیا جا سکتا۔عارضی حافظہ کی مثال لیتے ہم دیکھتے ہیں برقی طاقت نہ ملنے پر اس میں مواد محفوظ نہیں رہ سکتا البتہ ایسا ممکن ہے کہ عارضی حافظہ کو صرف اتنی برقی طاقت مہیا کی جائے کہ یہ صرف مواد محفوظ رکھنے کے قابل ہو یعنی اسے نڈھال سی کیفیت میں ڈالا جا سکتا ہے۔عارضی حافظہ کے مخلوط دور میںمداخل اس مقصد کے لئے مہیا کیا گیا ہے۔جس لمحہ مخلوط دور کی ضرورت ہو، اس لمحہ اس کےمداخل پست کر کے اسے بیدار کیا جاتا ہے اور استعمال کے بعد اسے فوراً دوبارہ نڈھال کر دیا جاتا ہے۔نڈھال صورت میں مخلوط دور بیرونی دنیا سے، دو طرفہ وسطی دور کی مدد سے، مکمل طور منقطع رہتا ہے اور اس میں نہ کچھ لکھا جا سکتا ہے اور نہ ہی اس سے کچھ پڑھا جا سکتا ہے۔اس دوران حافظہ کمتر برقی توانائی خرچنے والے حال میں ہوتا ہے۔عموماً بیدار کئے جانے والے مخلوط دور کی شناخت کار کی مدد سے شناخت کی جاتی ہے۔



	چار الفاظ حافظہ کا تصوراتی تصویر شکل 9.8 میں دکھایا گیا ہے جہاں دو بِٹ پتہ اور چار بِٹ مواد کو ثنائی شکل میں لکھا گیا ہے۔اسی شکل میں ایک کلو بائٹ حافظہ کا تصوراتی تصویر بھی دیا گیا ہے جس میں مواد کو ثنائی شکل جبکہ پتہ کو اعشاری شکل میں لکھا گیا ہے۔

مشق:	عارضی حافظہ کے معلوماتی صفحات سے اس کی جسامت حاصل کریں۔

9.2 پختہ حافظہ

	پختہ حافظہ سے مراد ایسا حافظہ ہے جس میں مواد برقی طاقت کی عدم موجودگی میں بھی محفوظ رہتا ہو۔پختہ حافظہ کا بنیادی استعمال وہاں ہوتا ہے جہاں مواد بار بار تبدیل نہ ہو۔
	عارضی حافظہ کی طرح پختہ حافظہ بھی مختلف لمبائی کے الفاظ پر مشتمل ہوتا ہے۔لفظوں تک رسائی پتہ کے ذریعہ کیا جاتا ہے۔یوںبِٹ پتہ والا پختہ حافظہ میںالفاظ ہوں گے۔
	بائٹ لمبے الفاظ والے چار الفاظ کے پختہ حافظہ کی اندرونی ساخت شکل 9.9 میں دکھائی گئی ہے۔اسی کو بہتر طور شکل 9.10 میں دکھایا گیا ہے جہاں چار داخلی جمع گیٹ کی صاف شکل استعمال کی گئی ہے۔دو سے چار شناخت کار، پتہ کے دو بِٹ سے چار مقام تک رسائی ممکن بناتا ہے۔یوں چار الفاظ تک رسائی ممکن ہوتی ہے۔


	شکل 9.9 میں بالکل نیا غیر استعمال شدہ پختہ حافظہ دکھایا گیا ہے۔پتہ  ہونے کی صورت میں،دو سے چار شناخت کار، کو بلند کر کے لفظچنے گا۔آپ دیکھ سکتے ہیں کہ اس طرح تمام جمع گیٹ کے مخارج بلند ہوں گے۔یوںمداخل پرخارج ہو گا۔پتہہونے کی صورت میں لفظچنا جائے گا اور ایک مرتبہ پھرپرخارج ہو گا۔آپ تسلی کر لیں کہ چاروں پتہ پر یہی مواد ملتا ہے۔کسی بھی نئے غیر استعمال شدہ پختہ حافظہ کے ہر لفظ کے تمام بِٹوں میںپایا جاتا ہے۔


	آپ نے دیکھا کہبلند ہونے سے تمام جمع گیٹوں کو یہی بلند اشارہ ملتا ہے اور یوں تمام جمع گیٹوں کے مخارج بلند ہو جاتے ہیں۔اگرکا کسی جمع گیٹ سے جوڑ منقطع کیا جائے توکا اشارہ اس جمع گیٹ تک نہ پہنچ پائے گا۔شکل 9.11 میں اس طرح دائیں جانب چار جمع گیٹوں کو سے منقطع کیا گیا ہے۔اس صورت لفظپڑھنے سےحاصل ہوتا ہے۔یہ ذہن میں رکھیں کہ جمع گیٹ کے، اس طرح کہیں نہ جڑے ہوئے مداخل، جمع گیٹ کے مخارج پر کوئی اثر نہیں رکھتے۔
	اس بحث سے آپ پختہ حافظہ میں لکھنے کے عمل کو بخوبی سمجھ گئے ہوں گے۔پختہ حافظہ میں اس طرح جوڑوں کو توڑ کر کے مواد لکھا جاتا ہے۔اس طرح کے حافظہ میں ہر جوڑ دراصل ایک برقی فیوز 19 ہوتا ہے۔کسی بھی جوڑ کو توڑنے کی خاطر اس جوڑ پر نصب فیوز20 میں اس کے استعداد سے زیادہ برقی رو گزار کے پگھلا کر توڑا جاتا ہے۔
	حافظہ میں مطلوبہ لکھے جانے والے مواد کو شکل 9.8 کی طرح جدول میں لکھا جاتا ہے۔جدول میں باری باری ایک ایک لفظ کو دیکھتے ہوئے، جس بِٹ کے مقام پرہو، حافظہ کے اندر اسی لفظ کے اسی بِٹ کا جوڑ تباہ کر دیا جاتا ہے۔
	شکل 9.11 میں  جمع گیٹوں کے مداخل اور دو سے چار شناخت کار کے مخارج کے مابین جوڑ، گول دائرہ سے دکھائے گئے ہیں۔شکل 9.12 میں لکھا گیا مواد جدول میں دیا گیا ہے۔اس طرح اشکال میں غیر تباہ شدہ جوڑوں کو صلیبی نشان سے ظاہر کیا جاتا ہے۔اس شکل کو بخوبی سمجھنا نہایت ضروری ہے۔

	اب تک حافظہ میں چار الفاظ ہونے کی وجہ سے داخلی جمع گیٹ استعمال کئے گئے۔ایک لفظبِٹ ہونے کی وجہ سے کُلجمع گیٹ استعمال کئے گئے۔یوں حافظہ میں کُل یعنی بتیس برقی جوڑ یا فیوز ہیں۔بِٹ پتہ والے حافظہ میں چونکہالفاظ ہوتے ہیں لہٰذا ایسے حافظہ میںداخلی جمع گیٹ استعمال کئے جاتے۔اگر حافظہ کا ایک لفظبِٹ پر مشتمل ہو تب جمع گیتوں کی تعدادہو گی۔یوں حافظہ میں کُل جوڑوں کی تعدادہو گی۔
	شعائیں سے صاف ہونے والا پختہ حافظہ میں بار بار لکھائی ممکن ہے۔اس کے برقی جوڑ، برقی فیوز سے نہیں بنائے جاتے بلکہ اس کے ہر جوڑ کو ایک سوئچ 21 تصور کیا جائے جسے مخصوص طریقہ سے برقی طاقت کے ذریعہ منقطع کیا جا سکتا ہے۔منقطع جوڑوں کو دوبارہ جوڑنے کی خاطر حافظہ کو شعائیں میں کچھ دیر رکھا جاتا ہے۔
	جدید برقی دباؤ سے صاف ہونے والا پختہ حافظہ میں بار بار لکھائی ممکن ہے۔اس طرز کے حافظہ میں لکھائی برقی دباؤ سے کی جاتی ہے اور اسے صاف بھی برقی دباؤ سے ہی کیا جاتا ہے۔
	پختہ حافظہ میں لکھائی مخلوط ادوار کے پروگرامر 22 سے کی جاتی ہے۔
9.3 حافظہ کی جسامت بڑھانے کے ترکیب
	عارضی حافظہ کے مخلوط دور کے قابو کرنے والے عمومی مداخل ،اور ہوتے ہیں جبکہ پختہ حافظہ کےاورہوتے ہیں۔اس حصہ میں تصور کیا گیا ہے کہ یہاں تمام استعمال کئے گئے حافظہ کے قابو مداخل صرفاورہیں۔انہیں کی مدد سے آپ ایک سے زیادہ حافظہ آپس میں جوڑنا سیکھیں گے۔حقیقت میں عموماًکے علاوہ تمام حافظہ کے ایک جیسے قابو مداخل اکٹھے جوڑے جاتے ہیں۔یوں تمام حافظہ کے مداخل اکٹھے جوڑے جائیں گے اور اسی طرح ان کے تماماکٹھے جوڑے جائیں گے۔

9.3.1 دو عددحافظہ کے سلسلہ وار جوڑنے سے ایک عددحافظہ کا حصول
	کبھی کبھار درکار جسامت کا حافظہ میسر نہیں ہوتا۔ایسی صورت میں مایک سے زیادہ حافظہ کو اکٹھے جوڑ کر درکار بائٹ ذخیرہ کرنا ممکن بنایا جاتا ہے۔شکل 9.13 (ا) میں دو عدد حافظہ جوڑ کر دگنے جسامت کا حافظہ حاصل کیا گیا ہے۔ان دو چھوٹے حافظہ کو حافظہ-0 اور حافظہ-1 کہا گیا ہے۔ آئیے اس شکل پر غور کرتے ہیں۔شکل (ا) میں دونوں حافظہ کے پتہ کے بِٹ آپس میں جوڑے گئے ہیں یعنی حافظہ-0 کا حافظہ-1 کےکے ساتھ جوڑا گیا ہے۔اسی طرح حافظہ-0 کاحافظہ-1 کے کے ساتھ جوڑا گیا ہے۔اسی طرح ان کے مواد کے بِٹ بھی آپس میں جوڑے گئے ہیں یعنی حافظہ-0 کے،، اور کو اسی ترتیب سے حافظہ-1 کے،، اور کے ساتھ جوڑا گیا ہے۔البتہ حافظہ-1  کے مداخل (جسے کہا گیا ہے)  کو نفی گیٹ کے ذریعہ کے ساتھ جوڑا گیا ہے جبکہ حافظہ-0 کے  مداخل(جسے کہا گیا ہے) کو سیدھا  کے ساتھ جوڑا گیا ہے۔



	شکل 9.14 (ا) میں پتہ کے تین بِٹوں کے تمام ترتیب دکھائے گئے ہیں۔پست ہونے سے پست ہوتا ہے جس سے حافظہ-0  جاگ اٹھتا ہے جبکہ حافظہ-1 نڈھال صورت میں رہتا ہے۔اسی طرحبلند ہونے سے پست ہوتا ہے جس سے حافظہ-1  جاگ اٹھتا ہے جبکہ حافظہ-0 نڈھال صورت اختیار کر لیتا ہے۔
	یوں اگر پست ہو تب پتہ کے بقایا دو بِٹ یعنیاورحافظہ-0 کے مختلف مقامات تک رسائی ممکن بناتا ہے۔پتہ کے تینوں بِٹ کو دیکھتے ہوئے اس طرح پتہ حافظہ-0 کے صفرویں مقام تک رسائی دیتا ہے جبکہ پتہحافظہ-0 کے تیسرے مقام تک رسائی دیتا ہے۔


	اسی طرح اگر بلند ہو تب پتہ کے بقایا دو بِٹ یعنیاورحافظہ-1 کے مختلف مقامات تک رسائی ممکن بناتے ہیں۔یوں پتہ حافظہ-1 کے صفرویں مقام تک رسائی دیتا ہے جبکہ پتہحافظہ-1 کے تیسرے مقام تک رسائی دیتا ہے۔
	گزشتہ دو پیراگراف کو اس طرح بھی دیکھا جا سکتا ہے کہ دئے گئے دو عدد چار الفاظ والے حافظہ مل کر ایک عدد آٹھ الفاظ حافظہ کے طور کام کرتے ہیں۔الفاظ کی لمبائی جوں کی توں چار بِٹ ہی رہتی ہے۔اس طرح دیکھتے ہوئے پتہکل حافظہ کے صفرویں مقام تک رسائی دیتا ہے، پتہکل حافظہ کے تیسرے مقام تک رسائی دیتا ہے، پتہکل حافظہ کے چوتھے مقام تک رسائی دیتا ہے اور پتہکل حافظہ کے ساتویں مقام تک رسائی دیتا ہے۔آپ نے دیکھا کہ یوں دو عدد حافظہ جوڑنے سے ایک عدد حافظہ حاصل کیا جا سکتا ہے اور آپ کو ان کے اندرونی ساخت پر ہر وقت دوبارہ غور کرنے کی ضرورت نہیں ہوتی۔شکل 9.13 (ب) میں اس حقیقت کو مدِ نظر رکھتے ہوئے ان دو حافظہ، بمع نفی گیٹ کے، کو بطور ایک ہیحافظہ کے دکھایا گیا ہے جس کے تین پتہ کے بِٹ اور چار مواد کے بِٹ ہیں۔اسی طرح  شکل 9.14 (ب) میں تیں بِٹ پتہ کی نسبت سے دونوں حافظہ کے مقامات دکھائے گئے ہیں۔اس شکل سے واضح ہے کہ ان دو چھوٹے حافظہ کو پتہ کے لحاظ سے علیحدہ علیحدہ مقامات پر رکھا گیا ہے اور حافظہ۔0 کے آخری لفظ کے اگلے مقام پر حافظہ۔1 کا صفرواں لفظ پایا جاتا ہے۔یوں پتہ کے لحاظ سے ان دو حافظہ کو سلسلہ وار قریب رکھا گئے ہیں۔ آپ بھی دو یا دو سے زیادہ حافظہ جوڑتے وقت اس طرح کی تصوراتی شکل ذہن میں بنایا کریں۔
	اس مثال میںجسامت کے حافظہ استعمال کئے گئے جنہیں دو پتہ کے بِٹ یعنیاوردرکار تھے۔یوں ان دو بِٹ کو استعمال کر کے بیدار حافظہ کے مختلف مقامات تک رسائی حاصل کی جاتی ہے جبکہ اگلے بِٹ یعنیکو استعمال کر کے ان دو حافظہ کو پتہ کے لحاظ سے مختلف مقامات پر رکھا گیا۔یہی طریقہ کار زیادہ جسامت کے حافظہ کے ساتھ بھی استعمال کیا جا سکتا ہے۔یوں دو عدد دس بِٹ پتہ والے حافظہ جوڑتے وقت تابِٹ بیدار حافظہ کے مختلف مقامات تک رسائی کے لئے استعمال کئے جائیں گے جبکہ اگلا بِٹ یعنیانہیں جداگانہ طور پرمداخل کی مدد سے بیدار کرنے کے لئے استعمال کیا جئے گا۔ 

9.3.2 تین عددحافظہ کے سلسلہ وار جوڑنے سے ایک عددحافظہ کا حصول
	شکل 9.15 (ا) میں پست مخارج والےشناخت کار کے استعمال سے تین عددجسامت کے حافظہ جوڑے گئے ہیں۔ان حافظہ کو حافظہ-0، حافظہ-1 اور حافظہ-3 کہا گیا ہے۔تینوں حافظہ کے پتہ بِٹآپس میں جوڑے گئے ہیں۔اسی طرح،اوربھی جوڑے گئے ہیں۔تینوں حافظہ کے مواد کے آٹھ مخارج بِٹ یعنی تابھی اسی طرح جوڑے گئے ہیں۔البتہ ان کےمداخل علیحدہ علیحدہ رکھے گئے ہیں۔اس طرح ایک وقت پر صرف ایک حافظہ کےمداخل کو پست کر کے بیدار کیا جاتا ہے اور اس کے سولہ مقامات تکتاکی مدد سے رسائی حاصل کی جاتی ہے۔
	 شناخت کار کو پتہ کے بِٹاوربطور مدخل مہیا کئے گئے ہیں جبکہ اس کے مخارج،،اورہیں۔شناخت کار ان دو پتہ کے مداخل بِٹوں کی مدد سے مطلوبہ حافظہ کی شناخت کرتا ہے۔شناخت کار کا نام یہی سے نکلا ہے۔


	جیسا کہ آپ جانتے ہیں، شناخت کار کے مداخل کے کسی بھی ترتیب اس کے مخارج میں سے صرف ایک کو چنتی ہے۔شکل (ب) میں شناخت کار کا جدول دکھایا گیا ہے جس میں دائیں جانب ایک اضافی قطار بنائی گئی ہے۔آئیں اس جدول پر غور کریں۔اورپست ہونے کی صورت میںپست ہو گا جو کہ حافظہ۔0 کےکے ساتھ جڑا ہے۔یوں سے حافظہ۔0 کی شناخت ہوتی ہے اور اسے بیدار کیا جاتا ہے۔رکھتے ہوئے بقایا چار پتہ کے بِٹ آزادانہ طور پر بلند یا پست ہو سکتے ہیں یعنیکی قیمتتاہو سکتی ہے۔یوں حافظہ۔0 کے سولہ مقامات تک رسائی کی جائے گی۔تمام پتہ بِٹوں کو اکٹھا لکھتے ہوئے ہم دیکھتے ہیں کہ اس حافظہ کے مختلف مقامات تک رسائی کرتے وقت کی قیمتتاہوتی ہے۔جدول کے دائیں قطار میں یہی حدیں لکھی گئی ہیں۔شکل (ج) میں نچلی جانب کے سولہ خانے انہیں مقامات کو ظاہر کرتے ہیں۔حافظہ۔0 کا آخری مقام، یعنی پندرواں مقام، کل حافظہ کے مقامپر پایا جاتا ہے۔
	بلند اورپست ہونے کی صورت میںپست ہو گا جو کہ حافظہ۔1 کےکے ساتھ جڑا ہے۔یوں سے حافظہ۔1 کی شناخت ہوتی ہے اور اسے بیدار کیا جاتا ہے۔رکھتے ہوئے بقایا چار پتہ کے بِٹ آزادانہ طور پر بلند یا پست ہو سکتے ہیں یعنیکی قیمتتاہو سکتی ہے۔یوں حافظہ۔1 کے سولہ مقامات تک رسائی کی جائے گی۔اس حافظہ کے مختلف مقامات تک رسائی کرتے وقت کی قیمتتاہوتی ہے۔جدول کے دائیں قطار میں یہی حدیں لکھی گئی ہیں۔شکل (ج) میں نچلی جانب سے سولہ خانے اوپر اگلے سولہ خانے انہیں مقامات کو ظاہر کرتے ہیں۔جیسا پہلے ذکر ہوا، حافظہ۔0 کا آخری مقام کل حافظہ کے مقامپر پایا جاتا ہے جبکہ حافظہ۔1 کا صفرواں مقام اس سے اگلے یعنیپر پایا جاتا ہے۔شکل (ج) میں صاف ظاہر ہے کہ جہاں حافظہ۔0 کا اختتام ہے وہیں سے حافظہ۔1 شروع ہوتا ہے۔
	پست اوربلند ہونے کی صورت میںپست ہو گا جو کہ کسی بھی حافظہ کے ساتھ نہیں جڑا۔ یوں سے کسی بھی حافظہ کی شناخت نہیں ہوتی ہے۔رکھتے ہوئے بقایا چار پتہ کے بِٹ آزادانہ طور پر بلند یا پست ہو سکتے ہیں یعنیکی قیمتتاہو سکتی ہے۔یوںکی قیمتتاہو گی لیکن ان تمام مقامات پر نہ تو کچھ لکھا جا سکتا ہے اور نہ ہی یہاں سے کچھ پڑھا جا سکتا ہے ۔جدول کے دائیں قطار میں یہی حدیں لکھی گئی ہیں۔شکل (ج) میں ان مقامات کو خالی مقامات لکھ کر ظاہر کیا گیا ہے۔	
	 اوردونوں بلند ہونے کی صورت میںپست ہو گا جو کہ حافظہ۔3 کےکے ساتھ جڑا ہے۔یوں سے حافظہ۔3 کی شناخت ہوتی ہے اور اسے بیدار کیا جاتا ہے۔رکھتے ہوئے بقایا چار پتہ کے بِٹ آزادانہ طور پر بلند یا پست ہو سکتے ہیں یعنیکی قیمتتاہو سکتی ہے۔یوں حافظہ۔3 کے سولہ مقامات تک رسائی کی جائے گی۔اس حافظہ کے مختلف مقامات تک رسائی کرتے وقت کی قیمتتاہوتی ہے۔جدول کے دائیں قطار میں یہی حدیں لکھی گئی ہیں۔شکل (ج) میں اوپر کے سولہ خانے انہیں مقامات کو ظاہر کرتے ہیں۔۔شکل (ج) میں صاف ظاہر ہے کہ جہاں خالی مقامات کا اختتام ہوتا ہے وہیں سے حافظہ۔3  شروع ہوتا ہے۔
	یہاں کُل چہ پتہ کے بِٹ، یعنیتا ، استعمال کئے گئے جو کہ چونسٹھمقامات تک رسائی دے سکتے ہیں۔ہم نے سولہ سولہ الفاظ کے تین حافظہ استعمال کرتے ہوئے اڑتالیس مقامات استعمال کئے جبکہ سولہ مقامات (خالی مقامات) کو استعمال نہیں کیا گیا۔اس طرح اگرچہ ان تین حافظہ کو سلسلہ وار جوڑا گیا ہے لیکن ان میں صرف حافظہ۔0 اور حافظہ۔1 قریب قریب رکھے گئے ہیں جبکہ حافظہ۔3 کو دور رکھا گیا ہے۔ہم مزید ایک اور سولہ الفاظ کے حافظہ کو شناخت کار کے کے ساتھ جوڑ کر تمام کے تمام چونسٹھ مقامات بھی استعمال کر سکتے ہیں۔
9.3.3 دو عددحافظہ متوازی جوڑ کرحافظہ کا حصول


	شکل 9.16 (ا) میں دو عددحافظہ کو متوازی جوڑ کر ایک عددحافظہ حاصل کیا گیا ہے۔یہ دونوں حافظہ بیک وقت بیدار ہوتے ہیں اور پتہ کے دو بِٹ اوران دونوں کے چاروں مقام تک رسائی ممکن بناتے ہیں۔اگر حافظہ۔0 کے مواد کوتاتصور کیا جائے جبکہ حافظہ۔1 کے مواد کوتاتصور کیا جائے تو یوں ان آٹھ بِٹوں کو ایک ہی بائٹ تصور کیا جا سکتا ہے۔اس طرح ان دو جڑے حافظہ کو ایک ہیجسامت کا حافظہ تصور کیا جا سکتا ہے جسے شکل (ب) میں تصوراتی شکل دی گئی ہے۔ 
9.4 حافظہ کے اوقاتِ کار
	حافظہ کو عموماً مائکرو پراسیسر 23 کے ساتھ منسلکہ طور پر استعمال کیا جاتا ہے۔عموماً مخلوط ادوار کسی ایک مقصد سرانجام دینے کی خاطر تخلیق کئے جاتے ہیں۔مائکرو پراسیسر قدرِ مختلف نوعیت کا مخلوط دور ہے جو احکامات پر چلتا ہے۔ان احکامات کو تبدیل کر کے  مائکرو پراسیسر کی کارکردگی تبدیل کی جا سکتی ہے۔ان احکامات کو عموماً پہلے سے پختہ حافظہ میں لکھ لیا جاتا ہے جہاں سے مائکرو پراسیسر انہیں پڑھ کر ان پر عمل درآمد کرتا ہے۔مائکرو پراسیسر کے ساتھ عموماً عارضی حافظہ بھی منسلک کیا جاتا ہے جہاں یہ عارضی مواد لکھ کر ذخیرہ کر سکتا ہے اور یہاں سے مواد پڑھ بھی سکتا ہے۔عموماً مختلف صنعت کاروں کے بنائے گئے مائکرو پراسیسر کے اپنے مخصوص احکامات ہوتے ہیں جنہیں یہ سمجھ کر ان پر عمل کر سکتا ہے۔کسی بھی مائکرو پراسیسر کے تمام احکامات کو اس مائکرو پراسیسر کی مادری زبان 24 کہا جاتا ہے جبکہ کسی ایک حکم کو اس زبان کا لفظ 25 کہا جاتا ہے۔
	مائکرو پراسیسر بیرونی جڑے مخلوط ادوار کے ساتھ گفتگو بذریعہ پتہ ، مواد اور قابو اشارات کے کرتا ہے۔شکل 9.17 (ا) میں مائکرو پراسیسر بیرونی جڑے عارضی حافظہ سے گفتگو کر رہا ہے۔اس گفتگو کا مقصد حافظہ میں مواد لکھنا ہے۔اس گفتگو کا آغاز اس وقت ہوتا ہے جب  مائکرو پراسیسر درکار عارضی حافظہ کا پتہ خارج کرتا ہے۔ایسے ادوار میں نسب شناخت کار چند ہی لمحوں میں پتہ کی مدد سے درکار مخلوط دور کی شناخت کر کے اسے بیدار کرتا ہے۔اس عمل کو شکل میں حافظہ کے قابو مداخلکے پست ہونے سے دکھایا گیا ہے۔مائکرو پراسیسر خارجی قابو اشارہکو پست کر کے حافظہ کو خبر دار کرتا ہے کہ مائکرو پراسیسر حافظہ میں مواد لکھنا چاہتا ہے اور ساتھ ہی اس مواد کو خارج کرتا ہے۔شکل میں اس مواد کو درست مواد لکھ کر ظاہر کیا گیا ہے۔حافظہ اس مواد کواشارہ کے کنارہِ چڑھائی پر مطلوبہ مقام پر محفوظ کرتا ہے۔مائکرو پراسیسر کسی بھی ایسے  عمل کے دوران  پتہ برقرار رکھتا ہے۔شکل میں پتہ کی تبدیلی کو دو لکیروں کی آپس میں جگہ بدلنے سے دکھایا گیا ہے۔
	شکل(ب) میں مائکرو پراسیسر حافظہ سے مواد پڑھنا چاہتا ہے۔اس گفتگو میں مائکرو پراسیسراشارہ کو بلند رکھ کر حافظہ کو خبردار کرتا ہے کہ مائکرو پراسیسر حافظہ سے مواد پڑھنا چاہتا ہے۔حافظہ بیدار ہوتے ہی اس کوشش میں لگ جاتا ہے کہ درکار مقام سے مواد حاصل کر کے مائکرو پراسیسر کے حوالے کرے۔ایسا کرنے کے لئے حافظہ کو کچھ وقت درکار ہوتا ہے جسے  حافظہ کا دورانیہ رسائی 26 کہتے ہیں۔حافظہ مطلوبہ مقام سے مواد حاصل کر کے خارج کرتا ہے۔شکل میں اس مواد کو درست مواد لکھ کر اس کی نشاندہی کی گئی ہے۔مائکرو پراسیسر اس مواد کو پڑھ کر آگے بڑھتا ہے۔

مشق:	انٹرنیٹ سےاورحافظہ کے دورانیہ رسائی حاصل کریں۔
 
9.5 پختہ حافظہ سے ترکیبی ادوار کا حصول
	اس کتاب کے حصہ 5.4 میں شناخت کار کی مدد سے تفاعل کے حصول کا طریقہ بیان کیا گیا جہاں دیکھا گیا کہ شناخت کار کے ساتھ جمع گیٹ نصب کرنے سے ایسا ممکن ہوتا ہے۔بِٹ پتہ والے شناخت کار کے مداخل، دراصل پتہ کے بٹوں کے تمام ممکنہ مجموعہ ارکانِ ضرب ہوتے ہیں۔  کسی بھی تفاعل کو مجموعہ ارکانِ ضرب کی صورت میں لکھ کر اسے شناخت کار کے مطلوبہ مخارج اور ایک  جمع گیٹ کی مدد سے حاصل کیا جا سکتا ہے۔ 
	بِٹ الفاظ کے پختہ حافظہ میں شناخت کار اور جمع گیٹ موجود ہوتے ہیں۔یوں اسے تفاعل کے حصول کے لئے  تشکیل 27 دیا جا سکتا ہے۔اس طرح شکل 9.12 کو آٹھ تفاعل حاصل کرنے والا دور سمجھا جا سکتا ہے جہاں یہ آٹھ تفاعل مندرجہ ذیل ہیں۔

 
(9.1)

	انہیں تفاعل کو ایک اور نظر سے دیکھتے ہیں۔کمتر دو بِٹ یعنیاورکو اکٹھے دیکھیں تو یہ مداخل  اورجمع کرنے والا نصف دور ہے۔اسی طرح دراصلاوردراصلہیں۔اسی طرحدراصل دونوں مداخل کا منطقی ضرب جبکہان کا منطقی جمع،ان کا منطقی بلا شرکت جمع اوران کا بلا شرکت منطقی نفی۔جمع ہے۔

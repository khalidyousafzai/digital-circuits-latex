\باب{ حافظہ}
ایک پلٹ ایک\اصطلاح{ ثنائی عدد   }\فرہنگ{ثنائی عدد} معلومات (مواد)  ذخیرہ کرنے کی صلاحیت رکھتا ہے۔ثنائی عدد کو  \اصطلاح{بِٹ }\فرہنگ{بِٹ}\حاشیہب{bit}\فرہنگ{bit} بھی  کہتے ہیں۔یوں ایک پلٹ ایک ثنائی عدد  \اصطلاح{حافظہ }\فرہنگ{حافظہ}\حاشیہب{memory}\فرہنگ{memory} کے طور  پر کام کر سکتا ہے۔آٹھ پلٹ جوڑ کر آٹھ ثنائی عدد حافظہ حاصل کیا جا سکتا ہے۔اسی طرح   \عددی{n} بِٹ پلٹ سے    \عددی{n} بِٹ حافظہ بنایا جا سکتا ہے۔آٹھ ثنائی  بِٹ کو ایک \اصطلاح{ ہشتمی  عدد }\فرہنگ{ہشتمی عدد} یا ایک \اصطلاح{ بائٹ   }\فرہنگ{بائٹ}\حاشیہب{byte}\فرہنگ{byte} کہتے ہیں۔حافظہ میں رکھے  گئے   مواد کو\اصطلاح{ لفظ }\فرہنگ{لفظ}\حاشیہب{word}\فرہنگ{word} کہتے ہیں۔حافظہ میں \اصطلاح{ الفاظ } کی لمبائی قطعی ہوتی ہے۔ یوں آٹھ بِٹ لفظ ایک بائٹ پر مشتمل ہو گا جبکہ سولہ بِٹ لفظ دو بائٹ پر مشتمل ہو گا۔کمپیوٹر میں موجود کل  حافظہ کی  پیمائش  بائٹ میں بیان کی جاتی ہے۔یوں  دو سو الفاظ کا حافظہ  جس میں ہر لفظ ایک بائٹ پر مشتمل ہو\موٹا{ دو سو بائٹ حافظہ }کہلائے گا۔حافظہ میں مواد داخل کرنے کو مواد\اصطلاح{  لکھنا  }\فرہنگ{لکھنا}\حاشیہب{write}\فرہنگ{write} یا حافظہ لکھنا کہتے ہیں جبکہ  حافظہ  سے مواد کے حصول کو مواد \اصطلاح{ پڑھنا }\فرہنگ{پڑھنا}\حاشیہب{read}\فرہنگ{read}  یا   حافظہ پڑھنا کہتے ہیں۔اس باب میں انہیں قسم کے  برقیاتی  حافظہ پر غور کیا جائے گا۔


حافظہ کی دو اہم  قسمیں ہیں۔حافظہ کی پہلی قسم ، جو \اصطلاح{عارضی حافظہ }\فرہنگ{حافظہ!عارضی}\حاشیہب{random access memory, RAM}\فرہنگ{memory!RAM} کہلاتا ہے، میں معلومات اس وقت تک محفوظ رہتی ہے جتنی دیر حافظہ کو درکار برقی طاقت مہیا کی جائے۔کسی بھی وقت   ،  عارضی حافظہ کے اندر  کسی بھی   مقام پر  معلومات   لکھی  یا اس مقام سے  معلومات پڑھی جا سکتی ہے۔ معلومات کا، حافظہ میں کسی بھی مقام پر لکھنے یا اس سے پڑھنے میں درکار وقت تمام  مقامات کے لئے تقریباً برابر ہوتا ہے۔اس دورانیہ کو\اصطلاح{ حافظہ کا دورانیہ رسائی }\فرہنگ{حافظہ!دورانیہ رسائی}یا  مختصراً  \اصطلاح{ دورانیہ رسائی }\فرہنگ{دورانیہ رسائی}\حاشیہب{access time}\فرہنگ{access time} کہتے ہیں۔


دوسری قسم کا حافظہ ، جو \اصطلاح{ پختہ حافظہ }\فرہنگ{حافظہ!پختہ}\حاشیہب{ROM, read only memory}\فرہنگ{memory!ROM} کہلاتا ہے،  میں برقی طاقت کی عدم موجودگی میں بھی   مواد محفوظ رہتا ہے تاہم اس سے  معلومات پڑھنے کی خاطر حافظہ کو درکار برقی طاقت فراہم کرنا لازم   ہے۔ پختہ حافظہ سے معلومات کسی بھی وقت کسی بھی  مقام سے پڑھی جا سکتی ہے۔حافظہ کے تمام مقامات سے مواد پڑھنے کے لئے درکار وقت، جو حافظہ کا \اصطلاح{ دورانیہ رسائی }کہلاتا ہے،  تقریباً ایک جیسا ہو گا۔ عام استعمال میں پختہ حافظہ سے معلومات صرف پڑھی جاتی ہے۔پختہ حافظہ کی مختلف اقسام میں معلومات محفوظ کرنے کے طریقے ایک دوسرے سے  مختلف ہیں۔ایک قسم  کے پختہ حافظہ میں معلومات صرف اور صرف ایک مرتبہ لکھی جا سکتی ہے، لہٰذا اسے صرف ایک مرتبہ معلومات کی لکھائی کے لئے استعمال کیا جا سکتا ہے۔اس کو  \اصطلاح{ایک مرتبہ قابل  لکھائی پختہ حافظہ }\فرہنگ{پختہ حافظہ!ایک مرتبہ قابل لکھائی}\حاشیہب{one time programmable read only memory, OTP}\فرہنگ{OTP} کہتے ہیں۔دوسری قسم کی پختہ حافظہ میں معلومات  بار بار لکھی  جا سکتی ہے تاہم ایسا کرنے سے پہلے اس  سے پرانی معلومات\موٹا{ مٹانی} ضروری ہے۔جدید پختہ حافظہ سے معلومات  برق  کی مدد سے مٹائی  جاتی ہے۔ایسے پختہ حافظہ کو \اصطلاح{برق مٹتا پختہ حافظہ }\فرہنگ{پختہ حافظہ!برق مٹتا}\حاشیہب{electrically erasable read only memory, EEROM, \(E^2PROM\)}\فرہنگ{ROM!EEROM} کہتے ہیں۔شروع میں پختہ حافظہ کی ایک قسم کو شعاع  سے مٹایا جاتا تھا۔اس کو \اصطلاح{ شعاع مٹتا  پختہ حافظہ }\فرہنگ{پختہ حافظہ!شعاع مٹتا}\حاشیہب{UV erasable read only memory, UV erasable ROM}\فرہنگ{ROM!UV erasable} کہتے ہیں۔

کاغذ پر لکھائی  کو مٹانے سے صاف ستھرا کاغذ ملتا ہے ۔   پلٹ ہر صورت بلند یا پست حال ہوتا ہے لہٰذا اس سے مواد  کاغذ کی طرح  نہیں مٹایا جا سکتا۔  لکھائی سے صاف حافظہ سے مراد وہ حافظہ ہو گا جس کے تمام بِٹ بلند \عددی{(1)} ہوں۔  جدول \حوالہ{جدول_حافظہ_خالی} میں  آٹھ بِٹ لمبائی کے چار  لفظ  حافظہ  استعمال کرتے  ہوئے مواد سے بھرے اور خالی حافظہ  کی وضاحت کی گئی ہے۔ یقیناً ، حافظہ کے تمام بِٹ پر \عددی{1} لکھنا اور حافظہ سے مواد مٹانا   ایک جیسا ہو گا۔
\begin{table}
\caption{حافظہ سے مواد مٹانے کا  مفہوم}
\label{جدول_حافظہ_خالی}
\centering
\begin{subtable}[t]{0.48\textwidth}
\centering
\begin{tabular}{C}
\toprule
1011\,0101\\
0000\,0000\\
1111\,1111\\
0110\,0110\\
\bottomrule
\end{tabular}
\caption{مواد سے بھرا حافظہ}
\end{subtable}%
\begin{subtable}[t]{0.48\textwidth}
\centering
\begin{tabular}{C}
\toprule
1111\,1111\\
1111\,1111\\
1111\,1111\\
1111\,1111\\
\bottomrule
\end{tabular}
\caption{مواد سے خالی حافظہ}
\end{subtable}
\end{table}


\حصہ{عارضی حافظہ}
اس حصے  میں عارضی حافظہ کی بناوٹ پر غور کیا جائے گا۔ایک بِٹ حافظہ بنیادی طور   ایک پلٹ ہوگا،جس میں مواد لکھنے اور پڑھنے کی صلاحیت موجود ہو گی۔ حافظہ عموماً کثیر تعداد بِٹوں پر مشتمل ہوگا لہٰذا حافظہ میں ہر پلٹ  تک، لکھنے اور پڑھنے کی خاطر ،  رسائی ضروری ہے۔شکل\حوالہء{ 9.1 } میں\اصطلاح{ ثنائی عارضی حافظہ کی اکائی }\فرہنگ{عارضی حافظہ!اکائی}\حاشیہب{binary memory cell}\فرہنگ{memory!binary cell} ، جس کو مختصراً\اصطلاح{ اکائی حافظہ }\فرہنگ{حافظہ!اکائی}\حاشیہب{unit memory}\فرہنگ{memory!unit} کہتے ہیں، کی بناوٹ اور علامت  پیش ہے، جہاں مواد ذخیرہ کرنے کے لئے ایس آر  پلٹ استعمال کیا گیا ہے۔حقیقت میں کئی طریقے مستعمل ہیں جن  پر  بعد میں غور کیا جائے گا۔ 


اس اکائی حافظہ سے رجوع کے لئے   \اصطلاح{ منتخب}  اشارہ بلند کیا جاتا ہے اور ساتھ ہی،     مواد لکھنے کی خاطر \عددی{\overline{\text{لکھ}}/\text{پڑھ}} پست کر کے  داخلی مواد فراہم کیا جاتا ہے جبکہ  مواد پڑھنے کی خاطر   \عددی{\overline{\text{لکھ}}/\text{پڑھ}} بلند کر کے مواد پڑھا جاتا ہے۔

متعدد  بِٹ حافظہ اسی اکائی حافظہ کی مدد سے حاصل  ہو گا۔شکل \حوالہء{9.2 } میں چار بِٹ کے ایک لفظ کا حافظہ  پیش ہے جہاں تمام اکائی حافظہ کے   "منتخب" قابو اشارے ایک ساتھ جوڑے گئے ہیں اور اسی طرح تمام کے قابو  اشارے   "\عددی{\overline{\text{لکھ}}/\text{پڑھ}}  " ایک ساتھ جوڑے گئے ہیں۔یوں اس لفظ کے چاروں بِٹ بیک وقت منتخب ہوتے ہیں اور اس میں مواد  \عددی{D} بیک وقت لکھا جا سکتا ہے،  یا اس میں ذخیرہ مواد بیک وقت پڑھا جا سکتا ہے۔

مزید ایک قدم   آگے بڑھ کر اس طرح کے کئی الفاظ جوڑ کر  متعدد الفاظ حافظہ حاصل  کیا جا سکتا ہے۔شکل \حوالہء{9.3 } میں چار الفاظ جوڑ کر حافظہ  تخلیق دیا گیا ہے۔

متعدد لفظ حافظہ  کی تمام  اکائیوں   کا "منتخب "اشارہ عام صورت میں  پست رہتا ہے۔یوں حافظہ کے  کسی بھی لفظ  تک رسائی  ممکن  نہیں ہو گی۔حافظہ میں مواد لکھنے کی خاطر مواد  \عددی{Z} داخلی راستے فراہم کر کے  \عددی{\overline{\text{لکھ}}/\text{پڑھ}} پست رکھ کر  مطلوبہ مقام کا  "منتخب " اشارہ بلند کیا جاتا ہے۔یوں مواد مطلوبہ لفظ کے مقام پر لکھا جاتا ہے۔ فرض کریں ہم  اعشاری تین  \عددی{(3_{10})} کے  ثنائی علامتی روپ  \عددی{0011_2}  کو حافظہ کے لفظ \عددی{2} کے مقام پر لکھنا چاہتے ہیں۔ہم مداخل پر \عددی{0011_2} مہیا کر کے   \عددی{\overline{\text{لکھ}}/\text{پڑھ}}  پست   رکھ کر لفظ  \عددی{2} کے  "منتخب" اشارے کو بلند   کریں گے۔ایسا کرنے سے شکل \حوالہء{9.3}  میں لفظ  \عددی{2} پر \عددی{0011_2}  لکھا  جائے گا۔یاد رہے کہ اس دوران باقی" منتخب " اشارے پست رہیں گے۔اسی لفظ کو پڑھنے کے لئے  ہم  \عددی{\overline{\text{لکھ}}/\text{پڑھ}}   بلند رکھ کر لفظ  \عددی{2} کا  "منتخب"  بلند کریں گے۔ایسا کرنے سے مخارج \عددی{D}  پر  \عددی{0011_2} خارج ہو گا  جہاں سے اسے  پڑھا جا سکتا ہے۔


حقیقی حافظہ میں الفاظ تک رسائی \موٹا{ پتہ } کے ذریعے  کی جاتی ہے۔چار لفظ حافظہ میں  الفاظ  تک رسائی  ، دو بِٹ پتہ  استعمال کرتے ہوئے دو سے چار شناخت کار کی مدد سے ممکن ہے۔شکل  \حوالہء{9.4 } میں یہ عمل پیش کیا گیا  ہے۔

عارضی حافظہ کا استعمال جدول \حوالہ{جدول_حافظہ_عارضی_استعمال} میں دکھایا گیا ہے۔ \موٹا{مجاز } پست ہونے کی صورت میں حافظہ  \اصطلاح{بلند رکاوٹی حال }\فرہنگ{حال!بلند رکاوٹی}\حاشیہب{high impedance state}\فرہنگ{state!high impedance} اختیار کر کے بیرونی ادوار سے مکمل منقطع ہو  گا۔
\begin{table}
\caption{عارضی حافظہ کا استعمال}
\label{جدول_حافظہ_عارضی_استعمال}
\centering
\begin{otherlanguage}{english}
\begin{tabular}{CCCC|R}
\toprule
\text{\RL{مجاز}} & \overline{\text{\RL{لکھ}}}/\text{\RL{پڑھ}} & A_1 & A_0 & \multicolumn{1}{c}{\text{\RL{عمل}}}\\
\midrule
0& \times &\times &\times &\text{\RL{بلند رکاوٹی حال}}\\
1&0&0&0&\text{\RL{لفظ \عددی{0} کے مقام پر لکھ}}\\
1&0&0&1&\text{\RL{لفظ \عددی{1} کے مقام پر لکھ}}\\
1&0&1&0&\text{\RL{لفظ \عددی{2} کے مقام پر لکھ}}\\
1&0&1&1&\text{\RL{لفظ \عددی{3} کے مقام پر لکھ}}\\
1&1&0&0&\text{\RL{لفظ \عددی{0} کے مقام سے پڑھ}}\\
1&1&0&1&\text{\RL{لفظ \عددی{1} کے مقام سے پڑھ}}\\
1&1&1&0&\text{\RL{لفظ \عددی{2} کے مقام سے پڑھ}}\\
1&1&1&1&\text{\RL{لفظ \عددی{3} کے مقام سے پڑھ}}\\
\bottomrule
\end{tabular}
\end{otherlanguage}
\end{table}


شکل \حوالہء{ 9.4 } میں چار بِٹ جمع گیٹ کی ایک نئی علامت استعمال کی گئی  ہے ۔گیٹ کا ایک مداخل دکھایا گیا ہے  جس پر چھوٹی ترچھی لکیر کے ساتھ   \عددی{4} لکھ کر اس بات کی وضاحت کی گئی ہے کہ دراصل یہ چار داخلی جمع گیٹ ہے۔اس طرح  کی علامت میں گیٹ کے مداخل  علیحدہ علیحدہ نہیں دکھائے جاتے بلکہ تمام مداخل ایک داخلی تار  سے ظاہر کیے جاتے ہیں۔یوں دور کا نقشہ کاغذ پر  کھینچتے ہوئے ہوئے تاروں کے ہجوم سے نجات حاصل ہوتی ہے اور دور   صاف ستھرا نظر آتا ہے۔یاد رہے کہ ایسا صرف دور  صاف  ستھرا نظر آنے کے لئے  کیا جاتا ہے۔یوں حافظہ کے گزشتہ دو اشکال ایک ہی دور بنانے کے دو طریقے ہیں۔

اسی طرز پر متعدد لفظ  حافظہ کی علامت   بھی بنائی جاتی ہے۔دس بِٹ پتہ سے   \عددی{2^{10}=1024_{10}}  یعنی  تقریباً ایک ہزار مقامات تک رسائی ممکن ہے۔کمپیوٹر کی دنیا  میں  کلو (ہزار) سے مراد     \عددی{1024_{10}}لیا جاتا ہے۔یوں دو  کلو سے مراد \عددی{2048_{10}}  ہو گا۔


شکل  \حوالہء{9.6 } میں   \اصطلاح{  وسطی دور } کے استعمال  پر غور کریں۔\موٹا{ مجاز } اور   \عددی{\overline{\text{\RL{لکھ}}}/\text{\RL{پڑھ}}}  دونوں بلند ہونے کی صورت میں  حافظہ میں ذخیرہ مواد  \عددی{D}  پر خارج ہو گا جبکہ  مجاز بلند اور   \عددی{\overline{\text{\RL{لکھ}}}/\text{\RL{پڑھ}}}  پست ہو نے  کی صورت میں  \عددی{D} پر  مہیا مواد حافظہ میں لکھا جائے گا۔یوں \عددی{D} بطور مداخل و مخارج   کام کرتا ہے۔

جدید عارضی حافظہ میں کثیر تعداد  کے الفاظ ذخیرہ کرنے کی گنجائش ہوتی ہے۔شکل\حوالہء{ 9.7 }-ا میں چار  لفظ  حافظہ کے \اصطلاح{ مخلوط دور }\فرہنگ{مخلوط دور}\حاشیہب{integrated circuit, IC}\فرہنگ{IC, integrated circuit} کی علامت دکھائی گئی ہے  جہاں لفظ کے چار داخلی  و خارجی بِٹوں  کو \عددی{D} کے بجائے \عددی{I/O} کہا گیا ہے۔

شکل-ب  میں مجاز کی جگہ \عددی{\overline{\text{\RL{مجاز}}}} استعمال کیا گیا ہے ، جو شکل -ا کے مجاز مداخل پر نفی گیٹ نصب کرنے سے حاصل  ہو گا؛  مزید  \عددی{\overline{\text{\RL{لکھ}}}/\text{\RL{پڑھ}}}   کو مختصراً    \عددی{\overline{\text{\RL{لکھ}}}} پکار کر   پنیا پر گول دائرہ  ڈال کر اس کا  پست فعال  پن ظاہر کیا گیا ہے۔یوں \عددی{\overline{\text{\RL{لکھ}}}} پست ہو نے کی صورت میں \موٹا{ حافظہ میں }مواد لکھا جا تا ہے اور بلند ہو نے کی صورت میں    \موٹا{حافظہ سے } مواد پڑھا جاتا ہے۔

شکل - ج میں  بارہ بِٹ پتہ ،  ایک بائٹ لمبے الفاظ کے عارضی حافظہ کی علامت دکھائی گئی ہے۔بارہ بِٹ پتہ سے \عددی{2^{12}=4096_{10}}  بائٹ تک رسائی ممکن ہے لہٰذا  یہ چار کلو بائٹ عارضی حافظہ کی   علامت ہے۔اس مخلوط دور میں  \عددی{\overline{\text{\RL{بیدار}}}} مداخل کا اضافہ کیا گیا ہے، جس پر اب بات کرتے ہیں۔


 مخلوط دور میں متعدد  گیٹ پائے جاتے ہیں اور جدید  برقیاتی  آلات کئی مخلوط ادوار پر مشتمل ہوتے ہیں۔یہ سب  برقی طاقت سے چلتے ہیں۔ہم کہتے ہیں برقی طاقت انہیں \اصطلاح{ بیدار } رکھتی  ہے۔برقیاتی  آلات عموماً  بیٹری سے برقی طاقت حاصل کرتے ہیں۔ درکار برقی طاقت کم  کرنے سے بیٹری زیادہ دیر کارآمد رہتی ہے۔
 
برقیاتی آلات میں مختلف مخلوط ادوار کی  ضرورت مختلف   لمحات  پر  ہو گی۔ان لمحات کے علاوہ انہیں بیدار رکھنے سے بلا ضرورت برقی توانائی   ضائع ہو گی۔غیر مستعمل  مخلوط ادوار کی برقی طاقت منقطع نہیں کی جا سکتی ہے۔عارضی حافظہ کی مثال لیتے  ہوئے ہم  جانتے ہیں  کہ برقی طاقت نہ ملنے پر ان میں مواد محفوظ نہیں رہتا ، البتہ یہ  ممکن ہے کہ عارضی حافظہ کو صرف اتنی برقی طاقت مہیا کی جائے کہ یہ صرف مواد محفوظ رکھنے کے قابل ہو ، یعنی اسے  نڈھال سی کیفیت میں ڈالا جا سکتا ہے۔عارضی حافظہ کے مخلوط دور میں  \عددی{\overline{\text{\RL{بیدار}}}} مداخل اس مقصد کے لئے مہیا کیا گیا ہے۔جس لمحے پر مخلوط دور کی ضرورت ہو،   \عددی{\overline{\text{\RL{بیدار}}}}   پست کر کے اسے بیدار کیا جاتا ہے اور استعمال کے بعد فوراً دوبارہ نڈھال کر دیا جاتا ہے۔نڈھال صورت میں مخلوط دور بیرونی دنیا سے، دو طرفہ وسطی دور کی مدد سے، مکمل طور  پر منقطع رہتا ہے اور اس میں نہ کچھ لکھا جا سکتا ہے اور نہ ہی اس سے کچھ پڑھا جا سکتا ہے۔نڈھال حال  میں حافظہ کمتر برقی توانائی صرف کرتا   ہے۔عام طور شناخت کار کی مدد سے   بیدار کیے جانے والے مخلوط دور کی  شناخت کی جاتی ہے۔



چار  لفظ  حافظہ کی تصوراتی تصویر شکل \حوالہء{9.8 } میں دکھائی گئی ہے جہاں دو بِٹ پتہ اور چار بِٹ مواد   ثنائی  روپ میں لکھے    گئے   ہیں۔اسی شکل میں ایک کلو بائٹ حافظہ کی تصوراتی تصویر بھی  پیش ہے جہاں مواد کو ثنائی جبکہ پتہ کو اعشاری  روپ میں لکھا گیا ہے۔

\ابتدا{مشق}
عارضی حافظہ  \عددی{6116 }کے معلوماتی صفحات سے اس کی  استعداد  کلو بائٹ میں   معلوم  کریں۔
\انتہا{مشق}



\حصہ{ پختہ حافظہ}
پختہ حافظہ سے مراد ا وہ حافظہ ہے جس میں مواد برقی طاقت کی عدم موجودگی میں بھی محفوظ رہتا ہو۔پختہ حافظہ کا بنیادی استعمال وہاں ہو گا جہاں مواد تبدیل نہ ہو۔

عارضی حافظہ کی طرح پختہ حافظہ بھی مختلف لمبائی کے الفاظ پر مشتمل ہو گا۔لفظوں تک رسائی   پتہ کے ذریعہ  ہو گی؛ \عددی{n} بِٹ پتہ کے پختہ حافظہ میں \عددی{2^n}  لفظ ہوں گے۔

بائٹ لمبائی  چار  لفظ  پختہ حافظہ کی اندرونی ساخت شکل \حوالہء{9.9 } میں دکھائی گئی ہے جس کی بہتر صورت  شکل \حوالہء{ 9.10 }  پیش کرتی ہے ،  جہاں چار داخلی جمع گیٹ کی صاف شکل استعمال کی گئی ہے۔دو سے چار شناخت کار، پتہ کے دو بِٹ سے چار مقامات  تک رسائی ممکن بناتا ہے۔یوں چار الفاظ تک رسائی ممکن ہو گی۔



شکل  \حوالہء{9.9 } میں بالکل نیا غیر استعمال شدہ پختہ حافظہ دکھایا گیا ہے۔پتہ    \عددی{00_2}  کی صورت میں دو سے چار شناخت کار \عددی{y_0}  بلند کر کے لفظ  \عددی{0} چنے گا۔ تمام جمع گیٹ بلند ہوں گے اور \عددی{D} پر \عددی{11111111_2} خارج ہو گا۔ پتہ  \عددی{01_2} لفظ  \عددی{1} چنے گا اور \عددی{D} پر \عددی{11111111_2} خارج ہو گا۔آپ تسلی کر لیں کہ چاروں پتہ پر یہی مواد ملتا ہے۔کسی بھی نئے غیر استعمال شدہ پختہ حافظہ کے ہر لفظ کے تمام بِٹ بلند  \عددی{(1)} ہوں گے۔


آپ نے دیکھا کہ    بلند \عددی{y_0} کی صورت میں تمام جمع گیٹ کو یہی بلند اشارہ ملتا ہے اور یوں تمام جمع گیٹ کے مخارج بلند ہو ں گے۔جمع گیٹ سے  \عددی{y_0}  کا جوڑ منقطع  کرنے سے  \عددی{y_0}   جمع گیٹ تک نہیں  پہنچے  گا۔شکل \حوالہء{ 9.11 } میں دائیں چار جمع گیٹ \عددی{y_0} سے منقطع  ہیں لہٰذا \عددی{y_0} بلند کر کے لفظ \عددی{0} پڑھنے سے   \عددی{D} پر \عددی{11110000_2} ملتا ہے۔یہاں ایک بات ذہن  نشین کریں :ایسے اشکال میں  جمع گیٹ کا منقطع مداخل جمع گیٹ کے مخارج پر اثر  انداز نہیں ہو گا۔

امید کی جاتی ہے آپ پختہ حافظہ میں لکھائی کا عمل بخوبی سمجھ گئے ہوں گے۔پختہ حافظہ میں  جوڑوں کو توڑ کر مواد لکھا جاتا ہے۔اس قسم حافظہ میں ہر جوڑ دراصل ایک\اصطلاح{ برقی  فتیلہ  }\فرہنگ{فتیلہ}\حاشیہب{electric fuse}\فرہنگ{fuse} (فیوز  ) ہو تا ہے ۔ فتیلے  کی استعداد  سے زیادہ برقی رو  فتیلے سے گزار کر اسے پگھلا کر  جوڑ   منقطع  کیا   جاتا ہے۔


حافظہ میں  لکھا  مواد شکل\حوالہء{ 9.8 } کی طرح جدول میں لکھا جاتا ہے۔اس جدول میں    باری باری ایک لفظ کو  دیکھتے ہوئے جس بِٹ کے مقام پر \عددی{0} ہو،  حافظہ کے اندر اس لفظ کے اس بِٹ کا جوڑ تباہ کیا جاتا ہے۔

% i donot see the round circles mentioned in this para and the data is wrong, y0 holds 10110010

شکل \حوالہء{ 9.11 } میں  جمع گیٹوں کے مداخل اور دو سے چار شناخت کار کے مخارج کے  بیچ جوڑ گول دائروں سے  ظاہر کیے  گئے ہیں۔شکل \حوالہء{ 9.12 } میں لکھا گیا مواد   بھی   پیش کیا گیا ہے۔ان  اشکال میں غیر تباہ شدہ جوڑ  صلیبی نشان \عددی{(\times)}  سے ظاہر کیے  جاتے ہیں۔اس شکل کو بخوبی سمجھنا  ضروری ہے۔

اب تک  چار لفظ  حافظہ  پر بات کی گئی جس کی وجہ سے \عددی{4}  داخلی جمع گیٹ استعمال کیے گئے۔ایک لفظ  \عددی{8} بِٹ ہونے کی وجہ سے کل  \عددی{8} جمع گیٹ استعمال کیے گئے۔یوں  ان حافظہ میں کل \عددی{8\times 4} یعنی  بتیس \عددی{(32)}   جوڑ یا  فتیلے  ہوں گے۔ آپ دیکھ سکتے ہیں کہ \عددی{n} بِٹ پتہ   کے حافظہ میں  \عددی{2^n} لفظ ہوں گے لہٰذا ایسے حافظہ میں  \عددی{2^n} داخلی جمع گیٹ  ہوں گے۔اگر حافظہ کا ایک لفظ \عددی{m} بِٹ ہو تب جمع گیٹوں کی تعداد \عددی{m} ہو گی۔یوں حافظہ میں جوڑوں کی تعداد  \عددی{m\times 2^n} ہو گی۔

\اصطلاح{شعاع  صاف پختہ حافظہ } میں بار بار لکھائی ممکن ہے۔ان  میں   جوڑ، برقی فتیلہ سے نہیں بنائے جاتے بلکہ ان   جوڑ کو ایک سوئچ  تصور  کریں جنہیں مخصوص طریقے سے برقی طاقت کے ذریعہ منقطع کیا جا تا ہے۔منقطع جوڑوں کو دوبارہ جوڑنے کی خاطر حافظہ کو شعاع  میں کچھ دیر رکھا جاتا ہے۔

جدید\اصطلاح{ برق صاف پختہ حافظہ } میں  بار بار لکھائی ممکن ہے۔ان حافظہ میں لکھائی برقی دباو سے کی جاتی ہے اور اسے صاف بھی برقی دباو سے  کیا جاتا ہے۔

پختہ حافظہ میں لکھائی \اصطلاح{ مخلوط ادوار     برنامہ نویس  }\فرہنگ{مخلوط دور!برنامہ نویس}\حاشیہب{IC programmer}\فرہنگ{IC!programmer} کی مدد سے کی جاتی ہے۔
	
%???KKK

\حصہ{حافظہ کی  استعداد  بڑھانے کی ترکیب}
عارضی حافظہ کے مخلوط دور کے قابو کرنے والے عمومی مداخل ،اور ہوتے ہیں جبکہ پختہ حافظہ کےاورہوتے ہیں۔اس حصہ میں تصور کیا گیا ہے کہ یہاں تمام استعمال کیے گئے حافظہ کے قابو مداخل صرفاورہیں۔انہیں کی مدد سے آپ ایک سے زیادہ حافظہ آپس میں جوڑنا سیکھیں گے۔حقیقت میں عموماًکے علاوہ تمام حافظہ کے ایک جیسے قابو مداخل اکٹھے جوڑے جاتے ہیں۔یوں تمام حافظہ کے مداخل اکٹھے جوڑے جائیں گے اور اسی طرح ان کے تماماکٹھے جوڑے جائیں گے۔

9.3.1 دو عددحافظہ کے سلسلہ وار جوڑنے سے ایک عددحافظہ کا حصول
	کبھی کبھار درکار جسامت کا حافظہ میسر نہیں ہوتا۔ایسی صورت میں مایک سے زیادہ حافظہ کو اکٹھے جوڑ کر درکار بائٹ ذخیرہ کرنا ممکن بنایا جاتا ہے۔شکل 9.13 (ا) میں دو عدد حافظہ جوڑ کر دگنے جسامت کا حافظہ حاصل کیا گیا ہے۔ان دو چھوٹے حافظہ کو حافظہ-0 اور حافظہ-1 کہا گیا ہے۔ آئیے اس شکل پر غور کرتے ہیں۔شکل (ا) میں دونوں حافظہ کے پتہ کے بِٹ آپس میں جوڑے گئے ہیں یعنی حافظہ-0 کا حافظہ-1 کےکے ساتھ جوڑا گیا ہے۔اسی طرح حافظہ-0 کاحافظہ-1 کے کے ساتھ جوڑا گیا ہے۔اسی طرح ان کے مواد کے بِٹ بھی آپس میں جوڑے گئے ہیں یعنی حافظہ-0 کے،، اور کو اسی ترتیب سے حافظہ-1 کے،، اور کے ساتھ جوڑا گیا ہے۔البتہ حافظہ-1  کے مداخل (جسے کہا گیا ہے)  کو نفی گیٹ کے ذریعہ کے ساتھ جوڑا گیا ہے جبکہ حافظہ-0 کے  مداخل(جسے کہا گیا ہے) کو سیدھا  کے ساتھ جوڑا گیا ہے۔



	شکل 9.14 (ا) میں پتہ کے تین بِٹوں کے تمام ترتیب دکھائے گئے ہیں۔پست ہونے سے پست ہوتا ہے جس سے حافظہ-0  جاگ اٹھتا ہے جبکہ حافظہ-1 نڈھال صورت میں رہتا ہے۔اسی طرحبلند ہونے سے پست ہوتا ہے جس سے حافظہ-1  جاگ اٹھتا ہے جبکہ حافظہ-0 نڈھال صورت اختیار کر لیتا ہے۔
	یوں اگر پست ہو تب پتہ کے بقایا دو بِٹ یعنیاورحافظہ-0 کے مختلف مقامات تک رسائی ممکن بناتا ہے۔پتہ کے تینوں بِٹ کو دیکھتے ہوئے اس طرح پتہ حافظہ-0 کے صفرویں مقام تک رسائی دیتا ہے جبکہ پتہحافظہ-0 کے تیسرے مقام تک رسائی دیتا ہے۔


	اسی طرح اگر بلند ہو تب پتہ کے بقایا دو بِٹ یعنیاورحافظہ-1 کے مختلف مقامات تک رسائی ممکن بناتے ہیں۔یوں پتہ حافظہ-1 کے صفرویں مقام تک رسائی دیتا ہے جبکہ پتہحافظہ-1 کے تیسرے مقام تک رسائی دیتا ہے۔
	گزشتہ دو پیراگراف کو اس طرح بھی دیکھا جا سکتا ہے کہ دئے گئے دو عدد چار الفاظ والے حافظہ مل کر ایک عدد آٹھ الفاظ حافظہ کے طور کام کرتے ہیں۔الفاظ کی لمبائی جوں کی توں چار بِٹ ہی رہتی ہے۔اس طرح دیکھتے ہوئے پتہکل حافظہ کے صفرویں مقام تک رسائی دیتا ہے، پتہکل حافظہ کے تیسرے مقام تک رسائی دیتا ہے، پتہکل حافظہ کے چوتھے مقام تک رسائی دیتا ہے اور پتہکل حافظہ کے ساتویں مقام تک رسائی دیتا ہے۔آپ نے دیکھا کہ یوں دو عدد حافظہ جوڑنے سے ایک عدد حافظہ حاصل کیا جا سکتا ہے اور آپ کو ان کے اندرونی ساخت پر ہر وقت دوبارہ غور کرنے کی ضرورت نہیں ہوتی۔شکل 9.13 (ب) میں اس حقیقت کو مدِ نظر رکھتے ہوئے ان دو حافظہ، بمع نفی گیٹ کے، کو بطور ایک ہیحافظہ کے دکھایا گیا ہے جس کے تین پتہ کے بِٹ اور چار مواد کے بِٹ ہیں۔اسی طرح  شکل 9.14 (ب) میں تیں بِٹ پتہ کی نسبت سے دونوں حافظہ کے مقامات دکھائے گئے ہیں۔اس شکل سے واضح ہے کہ ان دو چھوٹے حافظہ کو پتہ کے لحاظ سے علیحدہ علیحدہ مقامات پر رکھا گیا ہے اور حافظہ۔0 کے آخری لفظ کے اگلے مقام پر حافظہ۔1 کا صفرواں لفظ پایا جاتا ہے۔یوں پتہ کے لحاظ سے ان دو حافظہ کو سلسلہ وار قریب رکھا گئے ہیں۔ آپ بھی دو یا دو سے زیادہ حافظہ جوڑتے وقت اس طرح کی تصوراتی شکل ذہن میں بنایا کریں۔
	اس مثال میںجسامت کے حافظہ استعمال کیے گئے جنہیں دو پتہ کے بِٹ یعنیاوردرکار تھے۔یوں ان دو بِٹ کو استعمال کر کے بیدار حافظہ کے مختلف مقامات تک رسائی حاصل کی جاتی ہے جبکہ اگلے بِٹ یعنیکو استعمال کر کے ان دو حافظہ کو پتہ کے لحاظ سے مختلف مقامات پر رکھا گیا۔یہی طریقہ کار زیادہ جسامت کے حافظہ کے ساتھ بھی استعمال کیا جا سکتا ہے۔یوں دو عدد دس بِٹ پتہ والے حافظہ جوڑتے وقت تابِٹ بیدار حافظہ کے مختلف مقامات تک رسائی کے لئے استعمال کیے جائیں گے جبکہ اگلا بِٹ یعنیانہیں جداگانہ طور پرمداخل کی مدد سے بیدار کرنے کے لئے استعمال کیا جئے گا۔ 

9.3.2 تین عددحافظہ کے سلسلہ وار جوڑنے سے ایک عددحافظہ کا حصول
	شکل 9.15 (ا) میں پست مخارج والےشناخت کار کے استعمال سے تین عددجسامت کے حافظہ جوڑے گئے ہیں۔ان حافظہ کو حافظہ-0، حافظہ-1 اور حافظہ-3 کہا گیا ہے۔تینوں حافظہ کے پتہ بِٹآپس میں جوڑے گئے ہیں۔اسی طرح،اوربھی جوڑے گئے ہیں۔تینوں حافظہ کے مواد کے آٹھ مخارج بِٹ یعنی تابھی اسی طرح جوڑے گئے ہیں۔البتہ ان کےمداخل علیحدہ علیحدہ رکھے گئے ہیں۔اس طرح ایک وقت پر صرف ایک حافظہ کےمداخل کو پست کر کے بیدار کیا جاتا ہے اور اس کے سولہ مقامات تکتاکی مدد سے رسائی حاصل کی جاتی ہے۔
	 شناخت کار کو پتہ کے بِٹاوربطور مدخل مہیا کیے گئے ہیں جبکہ اس کے مخارج،،اورہیں۔شناخت کار ان دو پتہ کے مداخل بِٹوں کی مدد سے مطلوبہ حافظہ کی شناخت کرتا ہے۔شناخت کار کا نام یہی سے نکلا ہے۔


	جیسا کہ آپ جانتے ہیں، شناخت کار کے مداخل کے کسی بھی ترتیب اس کے مخارج میں سے صرف ایک کو چنتی ہے۔شکل (ب) میں شناخت کار کا جدول دکھایا گیا ہے جس میں دائیں جانب ایک اضافی قطار بنائی گئی ہے۔آئیں اس جدول پر غور کریں۔اورپست ہونے کی صورت میںپست ہو گا جو کہ حافظہ۔0 کےکے ساتھ جڑا ہے۔یوں سے حافظہ۔0 کی شناخت ہوتی ہے اور اسے بیدار کیا جاتا ہے۔رکھتے ہوئے بقایا چار پتہ کے بِٹ آزادانہ طور پر بلند یا پست ہو سکتے ہیں یعنیکی قیمتتاہو سکتی ہے۔یوں حافظہ۔0 کے سولہ مقامات تک رسائی کی جائے گی۔تمام پتہ بِٹوں کو اکٹھا لکھتے ہوئے ہم دیکھتے ہیں کہ اس حافظہ کے مختلف مقامات تک رسائی کرتے وقت کی قیمتتاہوتی ہے۔جدول کے دائیں قطار میں یہی حدیں لکھی گئی ہیں۔شکل (ج) میں نچلی جانب کے سولہ خانے انہیں مقامات کو ظاہر کرتے ہیں۔حافظہ۔0 کا آخری مقام، یعنی پندرواں مقام، کل حافظہ کے مقامپر پایا جاتا ہے۔
	بلند اورپست ہونے کی صورت میںپست ہو گا جو کہ حافظہ۔1 کےکے ساتھ جڑا ہے۔یوں سے حافظہ۔1 کی شناخت ہوتی ہے اور اسے بیدار کیا جاتا ہے۔رکھتے ہوئے بقایا چار پتہ کے بِٹ آزادانہ طور پر بلند یا پست ہو سکتے ہیں یعنیکی قیمتتاہو سکتی ہے۔یوں حافظہ۔1 کے سولہ مقامات تک رسائی کی جائے گی۔اس حافظہ کے مختلف مقامات تک رسائی کرتے وقت کی قیمتتاہوتی ہے۔جدول کے دائیں قطار میں یہی حدیں لکھی گئی ہیں۔شکل (ج) میں نچلی جانب سے سولہ خانے اوپر اگلے سولہ خانے انہیں مقامات کو ظاہر کرتے ہیں۔جیسا پہلے ذکر ہوا، حافظہ۔0 کا آخری مقام کل حافظہ کے مقامپر پایا جاتا ہے جبکہ حافظہ۔1 کا صفرواں مقام اس سے اگلے یعنیپر پایا جاتا ہے۔شکل (ج) میں صاف ظاہر ہے کہ جہاں حافظہ۔0 کا اختتام ہے وہیں سے حافظہ۔1 شروع ہوتا ہے۔
	پست اوربلند ہونے کی صورت میںپست ہو گا جو کہ کسی بھی حافظہ کے ساتھ نہیں جڑا۔ یوں سے کسی بھی حافظہ کی شناخت نہیں ہوتی ہے۔رکھتے ہوئے بقایا چار پتہ کے بِٹ آزادانہ طور پر بلند یا پست ہو سکتے ہیں یعنیکی قیمتتاہو سکتی ہے۔یوںکی قیمتتاہو گی لیکن ان تمام مقامات پر نہ تو کچھ لکھا جا سکتا ہے اور نہ ہی یہاں سے کچھ پڑھا جا سکتا ہے ۔جدول کے دائیں قطار میں یہی حدیں لکھی گئی ہیں۔شکل (ج) میں ان مقامات کو خالی مقامات لکھ کر ظاہر کیا گیا ہے۔	
	 اوردونوں بلند ہونے کی صورت میںپست ہو گا جو کہ حافظہ۔3 کےکے ساتھ جڑا ہے۔یوں سے حافظہ۔3 کی شناخت ہوتی ہے اور اسے بیدار کیا جاتا ہے۔رکھتے ہوئے بقایا چار پتہ کے بِٹ آزادانہ طور پر بلند یا پست ہو سکتے ہیں یعنیکی قیمتتاہو سکتی ہے۔یوں حافظہ۔3 کے سولہ مقامات تک رسائی کی جائے گی۔اس حافظہ کے مختلف مقامات تک رسائی کرتے وقت کی قیمتتاہوتی ہے۔جدول کے دائیں قطار میں یہی حدیں لکھی گئی ہیں۔شکل (ج) میں اوپر کے سولہ خانے انہیں مقامات کو ظاہر کرتے ہیں۔۔شکل (ج) میں صاف ظاہر ہے کہ جہاں خالی مقامات کا اختتام ہوتا ہے وہیں سے حافظہ۔3  شروع ہوتا ہے۔
	یہاں کُل چہ پتہ کے بِٹ، یعنیتا ، استعمال کیے گئے جو کہ چونسٹھمقامات تک رسائی دے سکتے ہیں۔ہم نے سولہ سولہ الفاظ کے تین حافظہ استعمال کرتے ہوئے اڑتالیس مقامات استعمال کیے جبکہ سولہ مقامات (خالی مقامات) کو استعمال نہیں کیا گیا۔اس طرح اگرچہ ان تین حافظہ کو سلسلہ وار جوڑا گیا ہے لیکن ان میں صرف حافظہ۔0 اور حافظہ۔1 قریب قریب رکھے گئے ہیں جبکہ حافظہ۔3 کو دور رکھا گیا ہے۔ہم مزید ایک اور سولہ الفاظ کے حافظہ کو شناخت کار کے کے ساتھ جوڑ کر تمام کے تمام چونسٹھ مقامات بھی استعمال کر سکتے ہیں۔
9.3.3 دو عددحافظہ متوازی جوڑ کرحافظہ کا حصول


	شکل 9.16 (ا) میں دو عددحافظہ کو متوازی جوڑ کر ایک عددحافظہ حاصل کیا گیا ہے۔یہ دونوں حافظہ بیک وقت بیدار ہوتے ہیں اور پتہ کے دو بِٹ اوران دونوں کے چاروں مقام تک رسائی ممکن بناتے ہیں۔اگر حافظہ۔0 کے مواد کوتاتصور کیا جائے جبکہ حافظہ۔1 کے مواد کوتاتصور کیا جائے تو یوں ان آٹھ بِٹوں کو ایک ہی بائٹ تصور کیا جا سکتا ہے۔اس طرح ان دو جڑے حافظہ کو ایک ہیجسامت کا حافظہ تصور کیا جا سکتا ہے جسے شکل (ب) میں تصوراتی شکل دی گئی ہے۔ 
9.4 حافظہ کے اوقاتِ کار
	حافظہ کو عموماً مائکرو پراسیسر 23 کے ساتھ منسلکہ طور پر استعمال کیا جاتا ہے۔عموماً مخلوط ادوار کسی ایک مقصد سرانجام دینے کی خاطر تخلیق کیے جاتے ہیں۔مائکرو پراسیسر قدرِ مختلف نوعیت کا مخلوط دور ہے جو احکامات پر چلتا ہے۔ان احکامات کو تبدیل کر کے  مائکرو پراسیسر کی کارکردگی تبدیل کی جا سکتی ہے۔ان احکامات کو عموماً پہلے سے پختہ حافظہ میں لکھ لیا جاتا ہے جہاں سے مائکرو پراسیسر انہیں پڑھ کر ان پر عمل درآمد کرتا ہے۔مائکرو پراسیسر کے ساتھ عموماً عارضی حافظہ بھی منسلک کیا جاتا ہے جہاں یہ عارضی مواد لکھ کر ذخیرہ کر سکتا ہے اور یہاں سے مواد پڑھ بھی سکتا ہے۔عموماً مختلف صنعت کاروں کے بنائے گئے مائکرو پراسیسر کے اپنے مخصوص احکامات ہوتے ہیں جنہیں یہ سمجھ کر ان پر عمل کر سکتا ہے۔کسی بھی مائکرو پراسیسر کے تمام احکامات کو اس مائکرو پراسیسر کی مادری زبان 24 کہا جاتا ہے جبکہ کسی ایک حکم کو اس زبان کا لفظ 25 کہا جاتا ہے۔
	مائکرو پراسیسر بیرونی جڑے مخلوط ادوار کے ساتھ گفتگو بذریعہ پتہ ، مواد اور قابو اشارات کے کرتا ہے۔شکل 9.17 (ا) میں مائکرو پراسیسر بیرونی جڑے عارضی حافظہ سے گفتگو کر رہا ہے۔اس گفتگو کا مقصد حافظہ میں مواد لکھنا ہے۔اس گفتگو کا آغاز اس وقت ہوتا ہے جب  مائکرو پراسیسر درکار عارضی حافظہ کا پتہ خارج کرتا ہے۔ایسے ادوار میں نسب شناخت کار چند ہی لمحوں میں پتہ کی مدد سے درکار مخلوط دور کی شناخت کر کے اسے بیدار کرتا ہے۔اس عمل کو شکل میں حافظہ کے قابو مداخلکے پست ہونے سے دکھایا گیا ہے۔مائکرو پراسیسر خارجی قابو اشارہکو پست کر کے حافظہ کو خبر دار کرتا ہے کہ مائکرو پراسیسر حافظہ میں مواد لکھنا چاہتا ہے اور ساتھ ہی اس مواد کو خارج کرتا ہے۔شکل میں اس مواد کو درست مواد لکھ کر ظاہر کیا گیا ہے۔حافظہ اس مواد کواشارہ کے کنارہِ چڑھائی پر مطلوبہ مقام پر محفوظ کرتا ہے۔مائکرو پراسیسر کسی بھی ایسے  عمل کے دوران  پتہ برقرار رکھتا ہے۔شکل میں پتہ کی تبدیلی کو دو لکیروں کی آپس میں جگہ بدلنے سے دکھایا گیا ہے۔
	شکل(ب) میں مائکرو پراسیسر حافظہ سے مواد پڑھنا چاہتا ہے۔اس گفتگو میں مائکرو پراسیسراشارہ کو بلند رکھ کر حافظہ کو خبردار کرتا ہے کہ مائکرو پراسیسر حافظہ سے مواد پڑھنا چاہتا ہے۔حافظہ بیدار ہوتے ہی اس کوشش میں لگ جاتا ہے کہ درکار مقام سے مواد حاصل کر کے مائکرو پراسیسر کے حوالے کرے۔ایسا کرنے کے لئے حافظہ کو کچھ وقت درکار ہوتا ہے جسے  حافظہ کا دورانیہ رسائی 26 کہتے ہیں۔حافظہ مطلوبہ مقام سے مواد حاصل کر کے خارج کرتا ہے۔شکل میں اس مواد کو درست مواد لکھ کر اس کی نشاندہی کی گئی ہے۔مائکرو پراسیسر اس مواد کو پڑھ کر آگے بڑھتا ہے۔

مشق:	انٹرنیٹ سےاورحافظہ کے دورانیہ رسائی حاصل کریں۔
 
9.5 پختہ حافظہ سے ترکیبی ادوار کا حصول
	اس کتاب کے حصہ 5.4 میں شناخت کار کی مدد سے تفاعل کے حصول کا طریقہ بیان کیا گیا جہاں دیکھا گیا کہ شناخت کار کے ساتھ جمع گیٹ نصب کرنے سے ایسا ممکن ہوتا ہے۔بِٹ پتہ والے شناخت کار کے مداخل، دراصل پتہ کے بٹوں کے تمام ممکنہ مجموعہ ارکانِ ضرب ہوتے ہیں۔  کسی بھی تفاعل کو مجموعہ ارکانِ ضرب کی صورت میں لکھ کر اسے شناخت کار کے مطلوبہ مخارج اور ایک  جمع گیٹ کی مدد سے حاصل کیا جا سکتا ہے۔ 
	بِٹ الفاظ کے پختہ حافظہ میں شناخت کار اور جمع گیٹ موجود ہوتے ہیں۔یوں اسے تفاعل کے حصول کے لئے  تشکیل 27 دیا جا سکتا ہے۔اس طرح شکل 9.12 کو آٹھ تفاعل حاصل کرنے والا دور سمجھا جا سکتا ہے جہاں یہ آٹھ تفاعل مندرجہ ذیل ہیں۔

 
(9.1)

	انہیں تفاعل کو ایک اور نظر سے دیکھتے ہیں۔کمتر دو بِٹ یعنیاورکو اکٹھے دیکھیں تو یہ مداخل  اورجمع کرنے والا نصف دور ہے۔اسی طرح دراصلاوردراصلہیں۔اسی طرحدراصل دونوں مداخل کا منطقی ضرب جبکہان کا منطقی جمع،ان کا منطقی بلا شرکت جمع اوران کا بلا شرکت منطقی نفی۔جمع ہے۔

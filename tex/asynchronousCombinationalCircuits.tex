\باب{ غیر معاصر ترتیبی ادوار }
وسیع پیمانہ عددی ادوار عموماً معاصر  ادوار کے  طرز پر بنائے جاتے ہیں۔ان کے اگلے حال مکمل طور  پر  موجودہ حال سے  حاصل ہوتے ہیں۔ حال صرف ساعت کے کنارے پر تبدیل ہوتے ہیں اور باقی  اوقات کے لئے  انہیں غیر متغیر تصور کیا جا سکتا ہے۔ساعت کے کنارے سے چند لمحات قبل  تا  چند لمحات بعد تک       تمام حال کا   پائیدار  ہونا  یقینی بنایا جاتا ہے۔یوں کنارہ  ساعت    پر  معلوم حال پائے جاتے ہیں جن سے اگلے  پر  یقین حال  حاصل  ہوتے ہیں۔ 

اس کے برعکس غیر معاصر ادوار کے حال کسی بھی لمحہ تبدیل ہو سکتے ہیں جس سے حالت  دوڑ اور دیگر مسائل کھڑے ہوتے ہیں جن پر اس باب میں غور کیا جائے گا۔

غیر معاصر ادوار کی اپنی ایک اہمیت ہے۔یہ ساعت کے کنارے کا انتظار کیے بغیر اشارہ  کو ردعمل کر سکتے ہیں۔عموماً کسی بھی عددی دور میں کچھ حصہ معاصر اور کچھ غیر معاصر ہو گا۔

شکل\حوالہء{ 11.1 } میں نہایت سادہ دور دکھایا گیا ہے جس کو سرسری نظر سے    دیکھ کر  یوں محسوس ہوتا ہے کہ ضرب گیٹ کا  مخارج کبھی بلند نہیں ہو سکتا۔غور کرنے سے ثابت ہوتا ہے کہ مسئلہ اتنا سادہ نہیں۔ جب بھی مداخل  \عددی{A} حال تبدیل کرے  اس کے  چند لمحوں بعد نفی گیٹ کا مخارج حال تبدیل کرے گا۔یہ \اصطلاح{ تاخیر }\فرہنگ{تاخیر}\حاشیہب{delay}\فرہنگ{delay} نفی گیٹ کے  \اصطلاح{دورانیہ ردِ عمل } کی بدولت ہے۔شکل میں \عددی{A}  اور \عددی{\overline{A}}  کے خط کھینچتے ہوئے یہ    تاخیر دکھائی گئی ہے۔اگر ضرب گیٹ کا دورانیہ ردِ عمل صفر ہوتا تب ضرب گیٹ کا مخارج ان دو مداخل کے مطابق حال \عددی{Y_0}  اختیار کرتا۔حقیقتاً  ضرب گیٹ کو بھی ردِ عمل کے لئے چند لمحات درکار ہوں گے لہٰذا ضرب گیٹ کا مخارج  \عددی{Y} ہو گا۔

آپ دیکھ سکتے ہیں  ضرب گیٹ کا مخارج غیر مطلوبہ طور پر ، نفی گیٹ کے دورانیہ ردِ عمل کے برابر  دورانیے کے لئے، بلند ہو گا۔اس طرح کے  ، غیر مطلوبہ نہایت کم دورانیہ کے لئے ،حال کی تبدیلی کو\اصطلاح{ برقی لرزش} یا مختصراً \اصطلاح{ لرزش}\فرہنگ{ لرزش}\حاشیہب{glitch}\فرہنگ{glitch}    کہتے ہیں۔ برقی لرزش مثبت یا منفی ہو سکتی ہے لہٰذا موجودہ لرزش کو مثبت لرزش کہیں گے۔  لرزش  نہایت کم دورانیے کی دھڑکن تصور کی جا سکتی ہے، تاہم  لرزش کی اصطلاح عموماً غیر مطلوبہ  دھڑکن کے لئے استعمال کی جاتی ہے  اور ان  سے معاصر ادوار   کو پاک رکھا جاتا ہے۔

 لرزش کی وجہ سے ادوار  \اصطلاح{عبوری حال  }\فرہنگ{عبوری حال } \حاشیہب{transition state}\فرہنگ{state!transition} اختیار کرتے ہیں۔اس باب میں عبوری حال   پر تفصیلاً بحث ہو گی۔  

آپ نے دیکھا کہ ضرب گیٹ تک   اشارہ \عددی{\overline{A}} پہنچنے میں تاخیر کی بدولت لرزش پیدا ہوئی۔تاخیر کی مزید  ایک  مثال دیکھتے ہیں۔

برقی تار میں برقی دباو کی رفتار تقریباً خلاء میں روشنی کی  رفتار \حاشیہب{خلاء میں روشنی کی رفتار \عددی{3\times 10^8} میٹر فی سیکنڈ ہے۔}کے برابر ہوتی ہے۔یوں ایک نینو      سیکنڈ  میں برقی دباو تقریباً  \عددی{3\times 10^8\times 10^{-9}=0.3} میٹر یعنی \عددی{30} سنٹی میٹر فاصلہ طے کرتا ہے۔آئیے دیکھتے ہیں  اگر پچھلی مثال  تبدیل کر کے نفی گیٹ کی جگہ \عددی{30}  سینٹی میٹر برقی تار لگائی جائے اور ضرب گیٹ کی جگہ بلا شرکت جمع گیٹ نصب کیا جائے تو دور کا ردِ عمل کیسا ہوگا ( شکل \حوالہء{11.2  }دیکھیں)۔


 اشارہ  \عددی{A} گیٹ کے ایک داخلی پن پر مہیا کیا گیا ہے جبکہ یہی  اشارہ تیس سنٹی میٹر  برقی تار سے  گزار کر دوسرے داخلی پن پر مہیا کیا گیا ہے جہاں اشارے کو \عددی{A_t} کہا گیا ہے۔ تار کو بل دار لکیر سے ظاہر کیا گیا ہے۔یوں اشارہ \عددی{A_t} گیٹ کے  دوسرے پن تک تاخیر سے پہنچتا ہے۔اشارہ  \عددی{A} بلند یا پست ہونے کے ایک نینو سیکنڈ بعد اشارہ  \عددی{A_t} بلند یا پست ہو گا۔گیٹ کا دورانیہ ردِ عمل نظر انداز کرتے ہوئے گیٹ کا  مخارج  \عددی{Y_0} ہوگا۔ گیٹ کا دورانیہ ردِ عمل مدِ نظر رکھتے ہوئے مخارج  \عددی{Y} ہوگا۔گیٹ کے خارجی اشارے میں دو بلند برقی لرزشیں  دیکھنے کو ملتی ہیں جن کے دورانیے برقی تار میں تاخیر کے برابر ہیں۔یوں اشارے  کی راہ میں  تاخیر،   حافظہ کی طرح، معلومات   لمحاتی طور یاد  رکھنے کی صلاحیت رکھتی ہیں۔

آپ نے دیکھا مختلف طرز کی  تاخیر  دور میں لرزشیں پیدا  کرتی ہیں۔جہاں\اصطلاح{ واپسیں اشارہ }\فرہنگ{واپسیں اشارہ}\حاشیہب{feedback signal}\فرہنگ{feedback signal} تاخیر سے پہنچ کر مخارج تبدیل کرتا ہو وہاں دوران تاخیر مخارج اور تاخیر کے بعد مخارج مختلف ہوں گے  جس سے  \اصطلاح{  نا پائیدار حالت  }\فرہنگ{حالت!نا پائیدار}\حاشیہب{unstable condition}\فرہنگ{unstable condition} پیدا ہو گی۔

 جب بھی ایک سے زیادہ اشارے بیک وقت تبدیل ہوں، گیٹ اور برقی تاروں میں نا قابل  معلوم  تاخیر کی بدولت   ، ان کے اثرات جاننا   تقریباً ناممکن ہو گا۔اس مسئلے سے بچنے کی خاطر  غیر معاصر ادوار  درج ذیل دو شرائط کے تحت  بنائے جاتے ہیں: (ا)  ایک وقت پر صرف ایک  اشارہ تبدیل ہو؛ (ب)  اشاروں  کی تبدیلی کے درمیان اتنا وقفہ دیا  جائے کہ  تاخیر کے باوجود دور  پائیدار حال اختیار کرتا ہو۔ان شرائط کے تحت  چلنے کو  \اصطلاح{بنیادی طریق  کار }\فرہنگ{بنیادی طریقہ کار}\حاشیہب{fundamental mode}\فرہنگ{fundamental mode} کے تحت چلنا کہتے ہیں۔

\حصہ{ تجزیہ}
\اصطلاح{غیر معاصر ترتیبی ادوار }\فرہنگ{ترتیبی دور!غیر معاصر}\حاشیہب{asynchronous combinational circuit}\فرہنگ{asynchronous!combinational circuit} سے مراد ایسے  ادوار ہیں جن میں  (ا) بغیر ساعت   والے پلٹ پائے جائیں اور یا   (ب) ان میں ایک یا ایک سے زیادہ مخارج بطور \اصطلاح{ واپسیں اشارات } استعمال ہوں۔جیسے اُوپر ذکر کیا گیا،مختلف نوعیت کی  تاخیر کی بنا پر  واپسیں اشارات لمحاتی طور پر    حافظہ کی صلاحیت رکھتے ہیں۔

جب   خارجی اشارہ، مثلاً \عددی{D}،   بطور داخلی اشارہ  استعمال  ہو کر  اپنی  ہی قیمت \عددی{(D)} تعین کرنے میں کردار ادا کر تا ہو،  یہ  \اصطلاح{واپسیں اشارہ  }\فرہنگ{واپسیں اشارہ}\حاشیہب{feedback signal}\فرہنگ{feedback signal} کہلاتا ہے۔

اس حصہ میں بغیر پلٹ  ادوار پر غور کیا جائے گا۔پلٹ والے دور پر اگلے حصہ میں غور کیا جائے گا۔

\جزوحصہ{عبوری جدول}
غیر معاصر ترتیبی ادوار پر غور ان کے \اصطلاح{ عبوری جدول }\فرہنگ{عبوری جدول}\حاشیہب{transition table}\فرہنگ{transition table} کی مدد سے کیا جاتا ہے۔یہ طریقہ  شکل \حوالہء{11.3}  میں دیے گئے دور کی مدد سے سیکھتے ہیں۔

پلٹ کی غیر موجودگی کے باوجود اس کو ترتیبی دور اس لئے کہیں گے کہ خارجی اشارے   \عددی{A} اور \عددی{B} بطور واپسیں اشارات، \عددی{a} اور \عددی{b}، استعمال کیے گئے ہیں۔دور سے خارجی حال  کی مساوات  لکھتے ہیں۔
\begin{gather}
\begin{aligned}
A&=(b+x)\cdot (a+\overline{x})\\
B&=(b+x)\cdot (\overline{a}+\overline{x})
\end{aligned}
\end{gather}
مساوات حاصل کرتے  وقت واپسیں اشاروں کو  عام مداخل تصور کریں۔یوں  \عددی{x} کو   بیرونی مداخل  جبکہ \عددی{a} اور \عددی{b}  کو اندرونی مداخل    تصور کریں۔ ان مساوات میں \عددی{a} اور \عددی{b} \موٹا{ موجودہ مخارج } جبکہ \عددی{A}  اور \عددی{B} \موٹا{ اگلے مخارج }ہیں۔ ان مساوات سے  جدول  \حوالہ{جدول_غیر_معاصر_واپسیں} حاصل  ہو گا جس سے عبوری جدول کا حصول شکل  \حوالہء{11.4 } میں دکھایا گیا ہے۔
\begin{table}
\caption{دور کا بوولین جدول}
\label{جدول_غیر_معاصر_واپسیں}
\centering
\begin{otherlanguage}{english}
\begin{tabular}{CCC|CC}
\toprule
a&b&x&A&B\\
\midrule
0&0&0&0&0\\
0&0&1&0&1\\
0&1&0&1&1\\
0&1&1&0&1\\
1&0&0&1&1\\
1&0&1&1&0\\
1&1&0&0&0\\
1&1&1&1&0\\
\bottomrule
\end{tabular}
\end{otherlanguage}
\end{table}

جدول  \حوالہ{جدول_غیر_معاصر_واپسیں}  میں پیش \اصطلاح{حال کے متغیرات}\فرہنگ{حال کے متغیرات}\حاشیہب{state variables}\فرہنگ{state variables} \عددی{A}    اور  \عددی{B} کی معلومات کو   علیحدہ علیحدہ کارناف نقشوں کی طرز پر لکھا گیا ہے جس سے  عبوری جدول  کے حصول  میں آسانی پیدا  ہوتی ہے۔کارناف نقشوں  کی  بائیں جانب قطار کی صورت  میں اندرونی مداخل \عددی{ab} کی قیمتیں  جبکہ   اوپر جانب صف  کی صورت میں بیرونی مداخل \عددی{x} کی قیمتیں لکھی جاتی ہیں۔

\اصطلاح{عبوری جدول }  میں     \عددی{A} اور \عددی{B} کی قیمتیں  ساتھ ساتھ   \عددی{AB}  لکھی جاتی ہیں۔کارناف نقشوں   کی آخری صفوں کی دائیں قطاروں  میں  \عددی{A} کی قیمت  \عددی{1} جبکہ  \عددی{B} کی  قیمت  \عددی{0} ہے۔عبوری جدول  کی نچلی صف  اور دائیں قطار  کے خانے میں ان قیمتوں کو ساتھ ساتھ  \عددی{10} لکھا گیا ہے۔ اس عمل کی وضاحت نکتہ دار لکیروں سے  کی گئی ہے۔

عبوری جدول میں صف در صف چلتے ہوئے    جب  بھی     صف میں موجودہ مخارج \عددی{ab} اور  اگلے مخارج \عددی{AB} کی قیمت   یکساں ہو، \عددی{AB} کی قیمت دائرے میں بند کریں۔دائرہ میں بند  حال    پائیدار   (مستحکم) جبکہ باقی نا پائیدار  یعنی  \اصطلاح{عبوری } \فرہنگ{عبوری حال }\حاشیہب{transient state}\فرہنگ{transient state} ہوں گے۔

شکل\حوالہء{ 11.5 } پر نظر رکھ  کر   \اصطلاح{عبوری جدول }کے استعمال پر غور کرتے ہیں۔جدول کی   \عددی{ab=00} صف  اور \عددی{x=0} قطار میں واقع خانے کو  \اصطلاح{ابتدائی خانہ}\حاشیہد{کسی بھی مستحکم حال خانے کو ابتدائی خانہ منتخب کیا جا سکتا ہے۔}کہا گیا ہے، جس میں \عددی{ab=00} اور \عددی{x=0} کی صورت میں   \عددی{AB}   کی قیمت  درج ہے۔ فرض  کریں ابتدائی خانہ دور کا ابتدائی حال ظاہر کرتا ہے۔

اب اگر \عددی{ab=00}  رکھتے ہوئے بیرونی مداخل \عددی{x} کی قیمت  \عددی{0} سے   \عددی{1} کر دی جائے تو عبوری جدول کے مطابق \عددی{AB} کی قیمت \عددی{00} سے   \عددی{01} ہو جائے گی۔ یوں   موجودہ حال \عددی{ab} اور  اگلے حال \عددی{AB}  کی قیمتیں مختلف ہوں گی جو   عبوری  حال  کی نشانی ہے   اور جس میں دور   زیادہ دیر نہیں رہ سکتا۔ برقی تاروں میں تاخیر کے بعد  \عددی{ab} کی قیمت  \عددی{01} ہو جائے گی جبکہ  \عددی{x} اپنی نئی قیمت  \عددی{(1)} برقرار رکھے گا۔یوں دور تاخیر کے بعد عبوری جدول کی  \عددی{x=1} قطار اور  \عددی{ab=01} صف  پر پائے جانے والے خانے تک پہنچے گا جہاں \عددی{AB} اور \عددی{ab} دونوں کی قیمت \عددی{01} ہے، جو   مستحکم حال کو ظاہر کرتا ہے ( اور اسی لئے دائرے میں بند دکھایا گیا ہے)۔ اس پورے مرحلہ کو" پہلا قدم "  کہا گیا ہے۔پہلے قدم کو تیر  دار لکیر  سے ظاہر کیا گیا ہے جو عبوری خانے سے گزر کر مستحکم خانے پر  اختتام پذیر ہوتا ہے۔

   مستحکم (پائیدار) حال سے ابتدا کرتے ہوئے \عددی{x} کی قیمت تبدیل کرنے سے دور  کچھ لمحوں کے لئے  عبوری حال اختیار کر گیا۔یہ صورت زیادہ دیر برقرار نہیں رہی۔تاروں میں  تاخیر کے بعد  واپسیں اشارے تبدیل ہوئے اور دور دوبارہ  مستحکم حال اختیار کر گیا۔عموماً ادوار کا عمل اسی طرح ہو گا۔

اسی طرح  \عددی{ab=01} رکھتے ہوئے   \عددی{x} کی قیمت \عددی{1} سے \عددی{0} کرنے سے  عبوری جدول کے مطابق دور  \عددی{x=0} قطار اور \عددی{ab=01}  صف  کے خانے میں درج حال  \عددی{AB=11} اختیار کرے گا۔اس مرتبہ  بھی \عددی{AB} اور \عددی{ab} مختلف ہیں (جو عبوری حال کو ظاہر کرتا ہے)   لہٰذا دور اس سے نکلنے کی کوشش کرے گا۔برقی تاروں میں تاخیر کے بعد \عددی{AB}  کی  نئی قیمتوں  کی خبر  \عددی{ab} کے مقام تک پہنچے گی   لہٰذا \عددی{ab} کی قیمت بھی  \عددی{11} ہو جائے گی۔یوں دور \عددی{x=0} قطار اور  \عددی{ab=11} صف میں درج (دائرے میں بند)    مستحکم  حال \عددی{AB=11} اختیار کر ے  گا۔اسی  طرح  چلاتے ہوئے \عددی{x} کی قیمت  بار بار تبدیل کرنے سے دور  بالترتیب \عددی{00} ، \عددی{01}،   \عددی{11}، اور \عددی{10} مستحکم حال اختیار کرے گا۔ہر مرتبہ \عددی{10} تک  پہنچ کر   یہی ترتیب  دوبارہ    دہرائی جائے گی۔شکل میں تیر  دار  لکیروں سے یہ مراحل دکھائے گئے ہیں۔

 دور کا حال  \عددی{AB} کی بجائے \عددی{ABx}  لکھا جاتا ہے۔ یوں  \عددی{000}، \عددی{011}، \عددی{110}، اور \عددی{101}  \اصطلاح{ مستحکم حال }جبکہ \عددی{001}، \عددی{010}، \عددی{111}، اور \عددی{100} \اصطلاح{ عبوری  حال }ہیں۔
 
عبوری جدول کی  ہر صف میں ،عموماً ، کم از کم ایک مستحکم حال ضرور پایا جاتا ہے۔ایسا نہ ہونے کی صورت میں  اس صف میں پہنچ کر  دور  عبوری حال اختیار کرے گا۔

عبوری جدول حاصل کرنے کا طریقہ  کار  یہاں بیان کرتے ہیں۔
\begin{itemize}
\item
دور میں      تمام   \اصطلاح{واپسیں اشاروں    }اور \اصطلاح{واپسیں دائروں }\حاشیہب{feedback loops} کی نشاندہی  کریں۔
\item
 کسی بھی ترتیب سے واپسیں دائروں کے مخارج کی شناخت \عددی{A}، \عددی{B}، \عددی{C}، وغیرہ جبکہ اسی ترتیب سے ان کے واپسیں اشارات کی شناخت \عددی{a}،  \عددی{b}، \عددی{c}،  وغیرہ سے  کریں۔
\item
 بیرونی اور اندرونی مداخل کی صورت میں        تمام مخارج کے بوولین تفاعل حاصل کریں۔
\item    
    ان تفاعل کے کارناف نقشے بنائیں۔
\item 
  تمام کارناف نقشوں کو ایک عبوری جدول میں یکجا کریں۔عبوری جدول کے خانوں میں   \عددی{ABC\cdots}   قیمتیں جبکہ جدول کے بائیں جانب ہر صف میں  \عددی{abc\cdots}  قیمتیں اسی ترتیب سے لکھیں۔
\item    
     جہاں \عددی{ABC\cdots} اور اسی صف میں \عددی{abc\cdots} کی قیمت یکساں ہو، وہاں \عددی{ABC\cdots} کو دائرے میں بند کریں۔
    \end{itemize}
عبوری جدول کے حصول کے بعد بیرونی مداخل تبدیل کر کے دور کے عبوری حال   پر غور کیا  جا سکتا ہے۔ 

\جزوحصہ{بہاو کا جدول}
شکل\حوالہء{ 11.4 } میں عبوری جدول لکھتے   ہوئے  خانوں میں بوولین طرز  پر حال  درج کیے  گئے۔دو مخارج کی صورت میں  چار  حال (  \عددی{00}، \عددی{01}، \عددی{10}، اور \عددی{11} ) ممکن ہیں جنہیں  نام بھی دیے جا سکتے ہیں۔مثلاً حال  \عددی{00} کو حال \عددی{a} پکارا جا سکتا ہے۔ اسی طرح \عددی{01} کو حال \عددی{b}، \عددی{10} کو حال \عددی{c}، اور \عددی{11} کو  حال \عددی{d} نام دیے جا سکتے ہیں۔عبوری جدول میں یہ نام استعمال کر کے، شکل \حوالہء{ 11.6}  میں پیش،   \اصطلاح{  بہاو کا جدول }\فرہنگ{جدول!بہاو کا}\حاشیہب{flow table}\فرہنگ{table!flow} حاصل ہو گا۔

شکل   \حوالہء{ 11.6}  میں پیش  بہاو کے جدول کے ہر صف میں صرف ایک  مستحکم حال پایا جاتا  ہے۔پہلی صف میں صرف  \عددی{000}   اور دوسری صف میں صرف  \عددی{011} مستحکم حال  پائے جاتے ہیں۔ ایسا جدول جس  کی ہر  صف میں صرف ایک مستحکم حال  پایا جاتا  ہو \اصطلاح{     اوّلی   بہاو کا جدول }\فرہنگ{بہاو کا  جدول!اوّلی}\حاشیہب{primitive flow table}\فرہنگ{flow table!primitive} کہلاتا ہے۔


شکل\حوالہء{ 11.7 } میں ایک ایسا بہاو کا جدول  پیش کیا  گیا ہے جس کی صفوں میں ایک سے زیادہ مستحکم حال پائے جاتے ہیں۔مثلاً ، پہلی صف میں  مستحکم حال  \عددی{000}، \عددی{011}، اور  \عددی{010} ہیں۔ایسے جدول کو \اصطلاح{ غیر  اوّلی بہاو کا جدول }\فرہنگ{بہاو کا جدول!غیر اوّلی}\حاشیہب{non primitive flow table}\فرہنگ{flow table!non primitive} کہتے ہیں۔

 بہاو کے جدول سے دور حاصل کرنے کے لئے پہلے عبوری جدول حاصل کیا جاتا ہے۔بہاو کے جدول کے دو صف  ہیں  لہٰذا دور کے دو  حال  ہوں گے۔دو ممکنہ صورتوں کو ایک بِٹ  عدد ظاہر  کر سکتا ہے۔یوں حال \عددی{a} کو  \عددی{0} اور حال  \عددی{b} کو  \عددی{1} لکھ  کر عبوری جدول حاصل کرتے ہیں، جو شکل\حوالہء{ 11.7 } میں  دکھایا گیا ہے۔ دور کے اگلے   مخارج  کو \عددی{Y}   اور موجودہ مخارج کو \عددی{y} سے ظاہر کر کے عبوری جدول سے  \عددی{Y}  کا تفاعل حاصل کرتے ہیں۔
\begin{align}
Y=\overline{x}_1x_0+x_1y
\end{align}

اس تفاعل کا دور شکل\حوالہء{ 11.8 } میں دکھایا گیا ہے۔

 شکل\حوالہء{ 11.7 }  میں پیش  بہاو کے جدول کے استعمال پر شکل \حوالہء{ 11.9 } کی مدد سے غور کرتے ہیں۔ فرض  کریں بیرونی مداخل  \عددی{x_1x_0} کی قیمت  \عددی{00} ہے،یعنی \عددی{x=00} ،   اور دور حال  \عددی{a} میں ہے۔اگر \عددی{x_1} تبدیل کیے بغیر \عددی{x_0} کی قیمت \عددی{1} کر دی جائے،یعنی \عددی{x=01} کر دی جائے، تو عبوری جدول کے مطابق دور  چند لمحوں کے لئے   عبوری   حال  \عددی{b} اختیار کرنے کے بعد  مستحکم حال \عددی{b} اختیار کر ے گا۔اب اگر \عددی{x_0} کی قیمت \عددی{1} رکھتے ہوئے \عددی{x_1} کی قیمت بھی  \عددی{1} کر دی جائے،یعنی  \عددی{x=11} کر دی جائے، تو  حال  \عددی{b} برقرار رہے گا۔اس اختتامی خانے کو \موٹا{ پہلا اختتامی خانہ }کہا گیا ہے۔ابتدائی خانے سے \موٹا{ پہلے اختتامی خانے }تک پہنچنے کا عمل تین تیر دار لکیروں سے ظاہر کیا گیا ہے جہاں پہلا تیر  مستحکم حال \عددی{a} سے عبوری حال  \عددی{b} کا حصول جبکہ دوسرا تیر یہاں سے مستحکم حال  \عددی{b} کا حصول   ظاہر کرتا ہے۔تیسرا تیر مستحکم حال  \عددی{b} سے مستحکم حال  \عددی{b} میں ہی رہنے کو ظاہر کرتا ہے۔

اس کے برعکس، ابتدائی خانے سے آغاز کرتے ہوئے  \عددی{x_1} برقرار اور   \عددی{x_0} تبدیل کرنے کی بجائے ہم  \عددی{x_0} کی قیمت  \عددی{0}رکھتے ہوئے \عددی{x_1} کی قیمت \عددی{1} کرتے ہیں، یعنی  \عددی{x=10} کرتے ہیں۔ بہاو کے جدول کے مطابق حال  \عددی{a} برقرار رہے گا۔اب اگر \عددی{x_0} کی قیمت بھی  \عددی{1} کر دی جائے، یعنی \عددی{x=11} کر دی جائے، تو اختتامی حال  برقرار  \عددی{a}   رہے گا۔اس اختتامی خانے کو  \موٹا{دوسرا اختتامی خانہ } کہا گیا ہے۔

آپ نے دیکھا  اختتامی حال بیرونی مداخل کی تبدیلی کی ترتیب پر منحصر ہے۔اس مثال میں ابتدائی بیرونی مداخل  \عددی{00} جبکہ اختتامی بیرونی مداخل  \عددی{11} ہیں۔ یاد رہے \موٹا{ بنیادی طریق  کار } کی شرائط کے تحت،     (دور کی درست   کارکردگی   کے لئے ضروری ہے کہ)  ایک سے زیادہ بیرونی مداخل بیک وقت تبدیل نہ کیے جائیں۔ یوں  \عددی{00} سے آغاز کر کے  ہم سیدھا \عددی{11} نہیں   کر سکتے۔ایسا کرنے سے ( ناقابل معلوم تاخیر کی بنا پر)  درست  اختتامی حال  جاننا ناممکن ہو گا۔


\جزوحصہ{ حالت دوڑ}\شناخت{حصہ_غیر_معاصر_حالت_دوڑ}
\اصطلاح{حالت دوڑ }\فرہنگ{حالت دوڑ}\حاشیہب{race condition}\فرہنگ{race condition} کا تذکرہ  ایس آر پلٹ پر تبصرے کے  دوران کیا گیا۔اس حصے  میں اس پر تفصیلاً گفتگو  کی جائے گی۔حالت دوڑ اس صورت کو کہتے ہیں جب  بیرونی اشارے  کی تبدیلی   ایک سے زیادہ حال تبدیل  کرتا ہو۔نا معلوم تاخیر کی بنا پر حال  کی تبدیلی مکمل طور پر جاننا ممکن  نہیں ہو گا۔مثلاً ،   فرض کریں دو حال   دور کا موجودہ مستحکم حال  \عددی{00} ہے اور  بیرونی مداخل تبدیلی  کرنے سے دونوں حال تبدیل ہوتے ہیں ، اور دور آخر کار  \عددی{11} مستحکم حال اختیار کر تا ہے۔ پہلی واپسیں راہ کی تاخیر دوسری واپسیں راہ کی تاخیر سے کم ہو نے کی صورت میں  دور مستحکم حال  \عددی{00} سے عبوری حال  \عددی{10}  اور آخر کار مستحکم حال \عددی{11} اختیار کرے گا جبکہ  دوسری راہ کی تاخیر پہلی راہ  کی تاخیر سے کم ہو نے کی صورت میں دور   عبوری حال  \عددی{01} سے گزر کر مستحکم حال  \عددی{11}  تک پہنچے گا۔آپ نے دیکھا کہ  (نامعلوم   تاخیر  کی بنا پر)  حال تبدیل  ہونے کی ترتیب جاننا  ممکن نہیں۔

جب  عبوری حال   کی تبدیلی کی ترتیب   اختتامی حال متعین کرنے میں کردار ادا کرتی ہو اور   دور دو مختلف  اختتامی مستحکم حال اختیار کر نے کی صلاحیت رکھتا  ہو وہاں  دوڑ کو  \اصطلاح{بحرانی دوڑ  }\فرہنگ{دوڑ!بحرانی}\حاشیہب{critical race}\فرہنگ{race!critical} کہیں گے۔   سودمند استعمال کے لئے ضروری ہے کہ دور میں بحرانی دوڑ   کی صورت  پیدا نہ ہوتی ہو۔جہاں عبوری حال  کی تبدیلی کی ترتیب  اختتامی مستحکم حال پر اثر انداز نہ ہوتی ہو وہاں  دوڑ کو\اصطلاح{ غیر بحرانی دوڑ  }\فرہنگ{دوڑ!غیر بحرانی}\حاشیہب{non-critical race}\فرہنگ{race!non-critical} کہیں گے۔



شکل\حوالہء{ 11.10 } میں بحرانی دوڑ کی ایک مثال   دکھائی گئی ہے  جہاں بیرونی مداخل \عددی{x}  اور  حال   \عددی{y_1y_0x}  ہے۔ حال \عددی{000} سے آغاز  کر کے بیرونی مداخل  \عددی{0} سے \عددی{1}   کرنے سے  دور  اختتامی  حال  کی جانب دوڑ لگائے گا۔نا معلوم تاخیر کی بنا پر   ہم نہیں جانتے دور تین ممکنہ حال   \عددی{011}، \عددی{111}، اور  \عددی{101} میں سے کس حال   کو پہلے پہنچے گا۔  یہ تینوں عبوری حال  پہلی صف میں دکھائے گئے ہیں۔  عبوری حال   \عددی{011}  پہلے پہنچنے کی صورت میں  دور   یہاں سے ہوتے ہوئے  اختتامی مستحکم حال  \عددی{011} اختیار کر ے گا، جس کو دوسری صف میں دائرے میں بند  دکھایا  گیا ہے۔اگر دونوں واپسیں راہ میں مائل تاخیر  برابر ہوں ، دور   عبوری حال   \عددی{111}   پہلے پہنچے گا اور یہاں سے ہوتے ہوئے  اختتامی مستحکم حال  \عددی{111} اختیار کر ے گا، جس کو تیسری صف میں دائرہ میں بند دکھایا گیا ہے۔تیسری صورت میں دور عبوری حال \عددی{101}  پہلے   پہنچتا ہے جہاں  سے یہ آخری صف کی جانب رواں ہو گا ، لیکن آخری صف ازخود عبوری حال ہے لہٰذا  دور  اس عبوری حال سے بھی گزر کر آخر کار تیسری  صف کے اختتامی  مستحکم حال  \عددی{111}  پہنچے گا۔اس مثال میں دو  اختتامی حال ممکن ہیں۔یہ دریافت کرنا ناممکن ہے کہ دور ان میں سے کس  اختتامی حال  کو پہنچے گا۔ شکل میں بائیں  جانب \عددی{x=0} کی قطار اس لئے  خالی  رکھی گئی ہے کہ ہم صرف \عددی{x=0}  سے  \عددی{x=1}   کرتے   ہوئے دور پر غور کر رہے ہیں جس میں بائیں قطار    کے اندراجات   درکار نہیں۔

شکل\حوالہء{ 11.11 } میں بحرانی دوڑ کی دوسری مثال دکھائی گئی ہے جہاں تین اختتامی  حال  ممکن ہیں۔ مستحکم حال  \عددی{000} سے آغاز کرتے ہوئے بیرونی مداخل  \عددی{x} کی قیمت  \عددی{1} کر نے سے دور اختتامی حال کی طرف دوڑ لگائے گا۔بالکل اُوپر مثال کی طرح،تین ممکنہ عبوری  حال  ہیں۔ایک عبوری  حال \عددی{011} ہے جہاں سے یہ دوسری صف میں دکھائے اختتامی مستحکم حال  \عددی{011} پہنچے گا۔دوسرا عبوری  حال  \عددی{111} ہے جہاں سے یہ تیسری صف کے اختتامی مستحکم حال  \عددی{111} پہنچے گا اور تیسرا عبوری  حال \عددی{101} ہے جہاں سے یہ آخری صف میں دکھائے اختتامی مستحکم حال  \عددی{101} پہنچے گا۔ نامعلوم  تاخیر کی بنا پر یہ جاننا ممکن نہیں کہ دور  حقیقت  میں کس اختتامی حال  کو پہنچے گا۔

اب غیر بحرانی دوڑ کی ایک مثال  دیکھتے ہیں جو شکل  \حوالہء{11.12 } میں دکھائی گئی ہے۔اس مثال میں  \عددی{000} سے  آغاز کرتے  ہوئے تین عبوری حال  ممکن ہیں۔ ایک عبوری حال   \عددی{011} ہے جہاں سے دور دوسری صف کے عبوری حال  \عددی{111} اور  اس کے بعد  تیسری صف کے عبوری حال \عددی{101}  سے  گزر کر آخر کار  چوتھی صف کے اختتامی مستحکم حال  \عددی{101}  پہنچے گا۔ دوسرا  عبوری حال  \عددی{111} ہے جہاں سے   دور تیسری صف کے عبوری حال  \عددی{101} سے ہوتے ہوئے آخر کار آخری صف کے اختتامی مستحکم حال   \عددی{101}   پہنچے گا۔  تیسرا عبوری حال  \عددی{101} ہے جہاں سے  گزر کر دور   آخری صف کے اختتامی مستحکم حال  \عددی{101} پہنچے گا۔

اس مثال میں اگرچہ تین مختلف ممکنات موجود ہیں تاہم  اختتامی مستحکم حال سب کا ایک   ہے لہٰذا  یہ غیر بحرانی دوڑ  ہو گی۔  

 مخصوص اور منفرد عبوری حال   سے گزر کر اختتامی مستحکم  حال  اختیار کرنے کو \اصطلاح{ پھیرا }\فرہنگ{پھیرا}\حاشیہب{cycle}\فرہنگ{cycle} لگانا کہتے ہیں۔اس کی  مثال شکل \حوالہء{11.13 } میں دی گئی ہے۔ان  اشکال میں حالت  دوڑ  نہیں پائی جاتی چونکہ ایک وقت میں صرف ایک مخارج حال تبدیل کرتا ہے ، البتہ اختتامی حال تک پہنچنے کی خاطر دور کو مخصوص اور منفرد عبوری حال   سے گزرنا ہو گا۔

شکل - الف  میں   مستحکم حال \عددی{00} سے  آغاز کرتے ہوئے عبوری حال   \عددی{10} کے بعد عبوری حال \عددی{11} سے  گزر کر اختتامی مستحکم  حال  \عددی{01}  پہنچا گیا۔   شکل-ب میں مستحکم حال   \عددی{00} سے  آغاز کرتے ہوئے عبوری حال  \عددی{10}  کے راستے  اختتامی  مستحکم حال \عددی{11}  اختیار کیا  گیا۔


\جزوحصہ{توازن اور ارتعاش }
 ایسا دور جو  \موٹا{پھیرے} لگاتے ہوئے کسی بھی اختتامی مستحکم حال تک نہ پہنچ پائے  \اصطلاح{غیر مستحکم دور  }\فرہنگ{غیر مستحکم دور}\حاشیہب{unstable circuit}\فرہنگ{unstable circuit} کہلاتا ہے۔شکل \حوالہء{ 11.14 } میں اس کی مثال دکھائی گئی ہے جہاں بیرونی مداخل  \عددی{1}  کرنے سے دور   مستحکم حال تک پہنچے بغیر عبوری حال  سے عبوری حال  منتقل ہو گا۔	ایسے  ادوار بطور\اصطلاح{ مرتعش }\فرہنگ{مرتعش}\حاشیہب{oscillator}\فرہنگ{oscillator} استعمال کیے جاتے ہیں۔ ادوار کو کبھی بھی غیر مستحکم نہیں ہونے دیا جاتا ماسوائے جب انہیں بطور مرتعش استعمال کرنا مقصد ہو۔

\حصہ{حالت دوڑ سے پاک ثنائی علامتوں کا تقرر}
حالت دوڑ کی صورت اس وقت پیدا ہو گی ہے جب ایک سے زیادہ مخارج بیک وقت حال تبدیل کرنے کی کوشش کریں۔بحرانی دوڑ کی صورت میں ادوار قابل استعمال نہیں رہتے۔اس حصے میں بحرانی دوڑ کے خاتمے پر غور کیا جائے گا۔یاد  رہے (\موٹا{بنیادی طریقہ کار } پر چلنے کے تحت)     ایک وقت  پر غیر معاصر   دور کا صرف ایک مداخل تبدیل    ہو سکتا ہے ، لہٰذا یہ  حصہ  پڑھتے ہوئے   ایک سے زیادہ   مداخل  کی تبدیلی  کی فکر مت کریں۔

جن ادوار میں ایک وقت پر صرف ایک مخارج حال تبدیل کرنے کی کوشش کرتا ہو،   وہ حالت دوڑ سے دو چار نہیں ہوتے۔اس حقیقت کو بروئے کار لاتے ہوئے حالت دوڑ ختم کی جاتی ہے۔

عبوری جدول کے حصول کے بعد اس میں درج حال  کو ثنائی علامتیں تعین کی جاتی ہیں۔ ان حال کو  \اصطلاح{ہمسایہ } ثنائی علامتیں مختص کرنے سے جن کے مابین  عبوری جدول میں تبادلہ پایا جاتا ہو   بحرانی دوڑ سے پاک دور حاصل ہو گا۔دو  ایسے ثنائی اعداد \اصطلاح{  ہمسایہ اعداد  }\فرہنگ{ہمسایہ اعداد}\حاشیہب{adjacent numbers}\فرہنگ{adjacent numbers} کہلاتے ہیں   جن  میں صرف ایک ہندسے کا فرق   ہو۔یوں \عددی{1010} اور \عددی{1110} ہمسایہ اعداد ہیں چونکہ ان  میں صرف ایک بِٹ مختلف ہے۔اسی طرح  \عددی{1110} اور \عددی{0110} آپس میں ہمسایہ  ہیں   جبکہ  \عددی{1010} اور \عددی{0110} آپس میں ہمسایہ نہیں۔

اس ترکیب کو شکل \حوالہء{11.15 }-ا میں دی  مثال کی مدد سے دیکھتے ہیں جس میں چار صف ہیں۔یوں دو بِٹ\اصطلاح{ حال کا متغیر  } \عددی{f_1f_0} اس کے چار ممکنہ حال بیان کر سکتا ہے۔ہم حال  \عددی{a} کے لئے \عددی{f=00}، حال \عددی{b}  کے لئے \عددی{f=01}،حال  \عددی{c} کے لئے \عددی{f=11}، اور حال  \عددی{d} کے لئے \عددی{f=10}  حال کے  متغیر منتخب کر کے دیکھتے ہیں    کیا نتائج رو نما ہوتے ہیں۔

 پہلی صف میں  \عددی{x} کی قیمت  \عددی{00} سے \عددی{01} کرنے سے  حال تبدیل ہو کر  \عددی{a}سے \عددی{b} ہو  گا،  لہٰذا حال  کا متغیر  \عددی{f}  تبدیل ہو کر  \عددی{00} سے \عددی{01} ہو  گا۔چونکہ  حال کے  متغیر کا صرف ایک بِٹ تبدیل ہو ا لہٰذا   حالت  دوڑ پیدا نہیں   ہو گی۔اس کے برعکس،  پہلی صف میں  \عددی{x} کی قیمت \عددی{00} سے  \عددی{10}  کرنے  سے حال تبدیل ہو کر \عددی{a} سے \عددی{c} ہو گا لہٰذا \عددی{f} کی قیمت \عددی{00} سے تبدیل ہو کر \عددی{11} ہو گی۔چونکہ \عددی{f} کے دو ہندسے بیکوقت  تبدیل ہونے کی کوشش کرتے ہیں لہٰذا حالت دوڑ پیدا ہو گی۔ یوں دو بِٹ حال کا متغیر تقرر  کرنے سے حالت دوڑ  پیدا ہو گی۔ایسی صورت میں دو سے زیادہ بِٹ حال کا متغیر استعمال کر کے دیکھا جاتا ہے کہ   آیا حالت دوڑ سے چھٹکارا ممکن ہے۔

کبھی کبھار  چار صف  عبوری جدول میں دو بِٹ حال کا متغیر  یوں تقرر  کرنا ممکن ہو گا کہ حالت دوڑ پیدا نہ ہو۔

شکل  \حوالہء{11.15 } -ب میں حال کے متغیر کی ترتیب بدل کر  حالت دوڑ  سے بچنے  کی  (نا کام) کوشش کی گئی ہے۔ یہاں \عددی{a}، \عددی{b}، \عددی{c}، اور \عددی{d} کے لئے بالترتیب \عددی{f=00}، \عددی{f=01}، \عددی{f=10}، اور \عددی{f=11} مختص کیے گئے۔پہلی صف  میں   \عددی{a}سے  \عددی{b}  کرنے سے \عددی{f} کی قیمت \عددی{00} سے تبدیل ہو کر \عددی{01} ، جبکہ \عددی{a} سے \عددی{c} کرنے سے \عددی{f} کی قیمت \عددی{00} سے \عددی{10} ہو گی۔  دونوں صورتوں میں \عددی{f} کا صرف ایک بِٹ تبدیل ہو گا، لہٰذا پہلی صف میں حالت دوڑ پیدا نہیں ہو گا۔ البتہ دوسری صف میں \عددی{x} کی قیمت \عددی{01} سے \عددی{11} کرنے سے حال تبدیل ہو کر \عددی{b} سے \عددی{c} ہو گا اور یوں \عددی{f} کی قیمت \عددی{01} سے \عددی{10} ہو گی۔ حال کے متغیر کے دو بِٹ کی تبدیلی سے مراد حالت دوڑ ہے۔

مذکورہ بالا دو مثالوں سے ظاہر ہے کہ موجودہ مسئلے میں دو بِٹ  حال کا متغیر  مختص کرنے  سے حالت دوڑ سے نجات حاصل کرنا ممکن نہیں۔ایسی صورت میں حالت دوڑ سے پاک حال کا متغیر منتخب کرنے  کے لئے ہم  \اصطلاح{ایک بلند بِٹ تقرری }\فرہنگ{ایک بلند بِٹ تقرری}\حاشیہب{one hot bit assignment}\فرہنگ{one hot bit assignment} کا طریقہ استعمال کرتے ہیں، جس کا استعمال  نہایت آسان ہے۔آئیے  اسی مثال پر اسے  استعمال کرتے  ہیں۔

شکل\حوالہء{ 11.16 } میں  حال کا متغیر چار بِٹ رکھا گیا ہے اور اس میں ایک وقت پر   صرف ایک  بِٹ بلند ہے۔یوں حال  \عددی{a} ، \عددی{b}، \عددی{c}، اور \عددی{d} کے لئے حال کے متغیر  بالترتیب \عددی{0001}، \عددی{0010}، \عددی{0100}، اور \عددی{1000}  مقرر کیے گئے۔


شکل   \حوالہء{11.16 } میں جدول کی پہلی صف میں  مداخل کی قیمت  \عددی{00} سے \عددی{01}  کرنے سے دور حال \عددی{a} سے حال  \عددی{b} منتقل ہوتا ہے۔یوں حال کا متغیر  \عددی{0001} سے \عددی{0010} ہو گا اور اس میں دو بِٹ کی تبدیلی حالت دوڑ  پیدا کرے گی۔اس سے  بچنے کے لئے جدول میں ایک نیا عبوری حال، \عددی{e}، شامل کیا جاتا ہے ۔حال  \عددی{a} سے  \عددی{b} پہنچنے کے لئے اس عبوری حال سے گزرنا  لازمی بنایا جاتا ہے۔عبوری حال  \عددی{e}   کے لئے حال کا متغیر یوں مقرر کیا جاتا ہے کہ یہ  \عددی{a} اور  \عددی{b}دونوں کا ہمسایہ عدد ہو۔ایسا عدد  \عددی{0011} ہے۔یوں  \عددی{e} کے لئے  حال کا متغیر \عددی{0011} مقرر کیا جاتا ہے اور جدول کو تبدیل کر کے  \عددی{x=01}کی قطار کے حال  \عددی{a} کی صف میں \عددی{b} کی بجائے \عددی{e} لکھا جاتا ہے جبکہ اسی قطار میں حال  \عددی{e} کی صف میں \عددی{b} لکھا جاتا ہے۔ایسا کرنے سے جدول تبدیل ہو کر شکل  \حوالہء{11.17 } اختیار کرتا ہے۔


اب پہلی صف میں مداخل  \عددی{00} سے \عددی{01}  کرنے سے  دور حال  \عددی{a} سے عبوری حال  \عددی{e} اختیار کرتے ہوئے آخر کار اختتامی مستحکم  حال \عددی{b}  پہنچتا ہے۔یہ عمل نکتہ دار تیر دار لکیروں سے  ظاہر کیا گیا ہے۔  اس پورے عمل میں  ہر قدم پر حال کے متغیر کا صرف ایک بِٹ تبدیل ہوتا ہے  لہٰذا حالت دوڑ  پیدا نہیں ہو گی ۔ عبوری حال \عددی{e} کی صف میں  باقی خانے خالی رکھے گئے ہیں۔ان میں سے کچھ خانے زیر استعمال آئیں گے اور کچھ نہیں۔استعمال  میں نہ آنے والے خانے خالی رکھے جاتے ہیں اور ان خانوں  کی قیمت\اصطلاح{ غیر ضروری  }\فرہنگ{غیر ضروری}\حاشیہب{don't care}\فرہنگ{don't care} ہو گی۔

پہلی صف میں مداخل  \عددی{00}سے \عددی{10}  کرنے سے شکل  \حوالہء{11.17 } میں حال  \عددی{a} سے حال  \عددی{c}حاصل   ہو گا۔حال کا متغیر  \عددی{0001} سے تبدیل ہو کر \عددی{0100} ہونا چاہے گا۔البتہ ایسا کرنے سے حالت دوڑ پیدا ہو گی، جس سے ہم  مذکورہ بالا  طریقے سے چھٹکارا حاصل کرتے ہیں۔

اس  حالت دوڑ سے  بچنے کے لئے  جدول میں عبوری حال، \عددی{f} ، شامل کیا جاتا ہے  اور   حال  \عددی{a} سے عبوری حال  \عددی{f} کے ذریعہ حال  \عددی{c}  پہنچا جاتا ہے۔عبوری حال   \عددی{f} کے  لئے حال کا متغیر یوں مقرر کیا جاتا ہے کہ یہ  \عددی{a} اور  \عددی{c}   دونوں کا ہمسایہ عدد ہو۔ایسا عدد  \عددی{0101} ہے۔یوں ح \عددی{f}  کے لئے حال  کا متغیر  \عددی{0101} مقرر کیا جاتا ہے اور جدول کو تبدیل کر کے  \عددی{x=10} کی قطار   میں  حال \عددی{a} کی صف   \عددی{c} کو تبدیل کر کے \عددی{f} لکھا جاتا ہے جبکہ اسی قطار میں حال \عددی{f} کی صف میں  \عددی{c} لکھا جاتا ہے۔ایسا کرنے سے شکل  \حوالہء{ 11.18}   ملتا ہے۔

یہی طریقہ کار تمام خانوں کے لئے دہرایا جاتا ہے۔ایسا کرنے سے شکل  \حوالہء{11.19 } حاصل ہو گا۔آپ سے  گزارش کی جاتی ہے کہ   یہ جدول خود حاصل کریں۔تسلی کر لیں کہ اس جدول میں کسی بھی حال سے دوسرے حال تک پہنچنے میں حالت دوڑ پیدا نہیں ہوتی۔


\حصہ{  عبوری جدول کی مدد سے  پلٹ کا تجزیہ}
عبوری جدول  استعمال  کر کے سے اس حصے  میں پلٹ کا تجزیہ کیا جائے گا۔چند مثالوں کے بعد حصہ \حوالہ{حصہ_غیر_معاصر_قدم_با_قدم} میں اس طریقہ کار پر قدم با قدم غور کیا  جائے گا۔

\جزوحصہ{ایس آر پلٹ}
عبوری جدول استعمال  کر کے سب سے پہلے ایس آر پلٹ پر غور کرتے ہیں۔شکل  \حوالہء{11.20 } میں اُوپر  ایس آر پلٹ  اور نیچے اسی کو بطور \موٹا{  واپسیں دور   } پیش کیا گیا ہے جہاں\اصطلاح{ واپسیں اشارہ } \عددی{q} کی پہچان آسان  ہے۔

 حال کے  متغیر  \عددی{Q} کو بطور واپسیں اشارہ  \عددی{q} استعمال کیا گیا ہے۔یوں حال کا متغیر \عددی{Q}، اندرونی مداخل  \عددی{q} جبکہ بیرونی مداخل  \عددی{S} اور \عددی{R} ہیں۔انہیں استعمال کرتے ہوئے عبوری جدول حاصل کی گئی ہے (شکل  \حوالہء{11.20 } دیکھیں)۔آئیے اس پلٹ کا تجزیہ اس کے عبوری جدول کی مدد سے کریں۔پلٹ کا جدول    صداقت مندرجہ ذیل ہے۔
%???KKK
 
(11.3)

	اس جدول سے ظاہر ہے کہ نفی۔جمع گیٹ پر مبنی ایس آر پلٹ کا صحیح استعمال تب ممکن ہے جب اس کے دونوں مداخل کسی صورت اکٹھے بلند نہ ہوں چونکہ ایسا ہونے سے پلٹ کے مخارجاوردونوں پست ہو جاتے ہیں جبکہ کسی بھی پلٹ کے مخارج کا ہر صورت آپس میں متضاد رہنا ضروری ہے۔اس شرط کو یوں بیان کیا جا سکتا ہے کہ نفی۔جمع گیٹ پر مبنی ایس آر پلٹ کے مداخل کو ہر صورت مندرجہ ذیل مساوات پر پورا اترنا چاہئے۔

 
(11.4)
 
	شکل 11.21 کو دیکھتے آگے پڑھیں۔عبوری جدول میںکی قطار میں متوازن حالکی صف میں پایا جاتا ہے جہاں حال کا متغیریعنی پست ہے۔اگر کیا جئے تو عبوری جدول کے مطابق حال کا متغیر پست ہی رہے گا۔اس عمل کو شکل کے حصہ الف میں نکتہ دار تیر سے دکھایا گیا ہے۔
	اسی طرح  کی صورت میں پلٹ کا متوازن بلند حال  کی صف میں پایا جاتا ہے۔اگرکیا جائے تو عبوری جدول کے مطابق پلٹ بلند حال میں ہی رہتا ہے جیسے شکل کے حصہ با میں دکھایا گیا ہے۔یہ دونوں اعمال پلٹ کے بوولین جدول سے بھی وضح ہے۔
	اب دیکھتے ہیں کہسےکرنے سے کیا صورت پیدا ہوتی ہے۔پہلے تو یاد دہانی کراتے چلیں کہ اس طرح کے ادوار کو بنیادی طریقِ کار 5 کے طرز پر چلایا جاتا ہے جہاں ایک سے زیادہ بیرونی مداخل تبدیل کرنے کی اجازت نہیں ہوتی۔بہرحال پھر بھی دیکھتے ہیں کہ ایسا کرنے سے کیا مسائل کھڑے ہوتے ہیں۔
	کرنے سے پہلے تو بوولین جدول کے مطابقاوردونوں پست ہوتے ہیں۔اس طرح یہ آپس میں متضاد حال میں نہیں ہوتے جبکہ کسی بھی پلٹ کے لئے یہ لازم ہے کہ اس کے دونوں مخارج ہر وقت متضاد حال میں ہوں۔دوسری بات یہ کہ عبوری جدول کو دیکھتے ہوئے اگرپہلے پست حال اختیار کر لے تو اختتامی حالہو گا جبکہ اگر پہلے پست ہو پائے تب اختتامی حالہو گا۔چونکہ یہ قبل از وقت معلوم کرنا ناممکن ہے کہ ان میں پہلے کون پست حال اختیار کرے گا لہٰذا یہ جاننا ناممکن ہے کہ اختتامی حال کیا ہو گا۔یوں اس طرح، دور کا استعمال غیر یقینی صورت پیدا کرتا ہے۔

11.3.2 ساعت کے کنارے چلتا ڈی پلٹ
	شکل 11.22 میں ساعت کے کنارہ چلتا ڈی پلٹ دکھایا گیا ہے۔ڈی پلٹ میں اندرونی واپسیں دور پایا جاتا ہے جس کے اندرونی حال کے متغیراتاورہیں6۔یوں اس کے واپسیں اشاراتاورہیں۔شکل میں دور کو دوبارہ واپسیں دور کی طرز پر بنایا گیا ہے تا کہ واپسیں اشارات اورکی پہچان آسان ہو۔



	اس دور کے اورحال کے متغیرات،اورواپسیں اشارات  جبکہاوربیرونی مداخل ہیں۔یوں ہم لکھ سکتے ہیں۔

 
(11.5)

	شکل 11.23 میں ان مساوات سے حاصلاورکے بوولین جدول  کو کارناف نقشہ کی طرح لکھ کر عبوری جدول حاصل کیا گیا ہے۔مکمل حال کو کی صورت میں لکھتے ہوئے اس جدول پر غور کرتے ہیں۔


	تصور کریں کہ جس لمحہ پلٹ کو برقی طاقت مہیا کر کے زندہ کیا جاتا ہے  اس لمحہ  ساعت،یعنی، اور بیرونی مداخل، یعنی،دونوں پست ہیں۔اس صورت عبوری جدول کے مطابق دورکی قطار میں ہوگا۔اس قطار میں تین خانے عبوری حال کا متغیر کو ظاہر کرتے ہیں۔یہ تین خانے،اورہیں۔ان تینوں خانوں میں عبوری حالہے۔چوتھا خانہ،یعنی، متوازن حال کو ظاہر کرتا ہے اور اس میں متوازن حالہے۔یوں اگر برقی طاقت کے فراہمی کے لمحہ تاخیرات ایسے ہوں کہ دور ان تین عبوری خانوں میں کسی ایک میں داخل ہوتا ہے تو یہاں سے جلد وہ کی صف پہنچ کر مستحکم حال اختیار کر لے گا۔اگر زندہ ہوتے ہی دور سیدھاخانہ میں داخل ہو تب یہ یہی رہے گا۔
	اس کے برعکس برقی طاقت مہیا کرنے کے لمحہ اگراورہوں تو عبوری جدول کے مطابق دوریاکے مستحکم حال تک پہنچ کر یہی رہے گا جبکہاورکی صورت میں دور یامیں ہو گا۔
	پست ساعت کی صورت میں حال کے متغیراتکی قیمترہتی ہے۔عبوری جدول میںاورکی دو قطاریں اس بات کو ظاہر کرتی ہیں جہاں تمامکی قیمتیںہیں۔ہم جانتے ہیں کہ ایس آر پلٹ کی دونوں مداخل بلند ہونے کی صورت میں پلٹ اپنی حال برقرار رکھتا ہے۔یوں شکل 11.22 میں اس صورت میں خارجی پلٹ اپنی حال برقرار رکھے گا۔
	پست ساعت، یعنی، اور پست،یعنی،کی صورت میں متوازن حال کا متغیر حاصل کرنے کی خاطر ہم عبوری جدول کےکی قطار میں دیکھتے ہیں جہاں ہمیں مکمل حالبطور مستحکم حال ملتا ہے۔جدول کے اس خانے میں لکھ کر اسے واضح کیا گیا ہے۔یہاںہونے کی وجہ سے خارجی پلٹ اپنی حال برقرار رکھے گا۔
	پست ساعت اور بلندکی صورت میںکی قطار میں متوازن حالپایا جاتا ہے جہاںہی ہے اور یوں خارجی پلٹ اپنی حال برقرار رکھے گا۔جدول کے اس خانے میں لکھ کر اسے واضح کیا گیا ہے۔
	تصور کریں کہ دور کے متوازن حال،یعنی خانہ،میں ہوتے ہوئے بیرونی مداخلبلند ہوتا ہے۔بیرونی مداخلجس لمحہ سےہوتا ہے اس لمحہ کو ساعت کا کنارہ چڑھائی 7 کہتے ہیں۔ یوںکی صورت میں ساعت کے کنارہ چڑھائی آنے سے دور خانہکی صف میں رہتے ہوئے،سےکی قطار میں داخل ہو کر عبوری صورتاختیار کرتا ہے۔اس عبوری حال کو خانہ کہا گیا ہے۔یہاں سے یہ جلد اختتامی مستحکم حالک پہنچتا ہے۔اس خانہ کوکہا گیا ہےحال میں حال کا متغیر ہیں۔خارجی پلٹکی صورت میں پست حال اختیار کر لے گا اور یوںہو جائے گا۔اس قدم کو شکل میں خانہ سے  خانہکے راستے خانہتک تیر والے لکیر سے دکھایا گیا ہے۔اس پورے کا نچوڑ یہ ہے کہ کی صورت میں ساعت کے کنارہ چڑھائی پر ہو جائے گا یعنی ڈی پلٹ پست حال اختیار کر لیتا ہے۔
	اس پورے عمل پر دوبارہ غور کریں۔ساعت کے کنارہ چڑھائی آتے ہی دور عبوری حالاور پھر متوازن حالاختیار کرتا ہے۔ان دونوں حال میںہی رہتے ہیں اور یوں عبوری حال سے گزرتے ہوئے کسی قسم کی لرزش پیدا نہیں ہوتی۔آپ نیچے پڑھتے ہوئے ہر قدم پر تسلی کر لیں کہ کسی بھی عبوری حال سے گزرتے وقتکی قیمت وہی ہوتی ہے جو اس قدم کے اختتامی حال میں گی۔یوں ایسے لمحات پر لرزش سے کسی قسم کی غیر یقینی صورت پیدا نہیں ہوتی۔
		اسی طرح مکمل حالمیں موجود دور، ساعت کے کنارہ چڑھائی آتے، عبوری حالسے ہوتے ہوئے متوازن حالاختیار کرے گا۔اس قدم کو شکل میں خانہ سے  خانہکے راستے خانہتک تیر والے لکیر سے دکھایا گیا ہے۔یہ قدم بلند بیرونی مداخل یعنیکی صورت میں ساعت کے کنارہ چڑھائی پر ہونے کا عمل ہے جس سے داخلی پلٹ بلند ہو جائے گا اور یوں ڈی پلٹ کاہو جائے گا۔
	ساعت کے کنارہ اترائی کے عمل کو نکتہ دار تیر والے لکیروں سے دکھایا گیا ہے۔انہیں آپ خود سمجھ سکتے ہیں۔یہ دونوں لکیریں اس بات کو واضح کرتی ہیں کہ ساعت کے کنارہ اترائی پر عبوری حال اور اختتامی مستحکم حال دونوں میں ہوتا ہے۔ہونے کی صورت میں بیرونی پلٹ اپنی حال برقرار رکھتا ہے اور یوں ساعت کے کنارہ اترائی پر ڈی پلٹ کے حال میں کسی قسم کی تبدیلی رو نما نہیں ہوتی۔
	ایک آخری بات اس پلٹ کے حوالہ سے کرتے ہیں۔شکل 11.22 میں اشارہکوپیدا کرنے والے نفی۔ضرب گیٹ کو داخلی اشارہ کے طور مہیا کیا گیا ہے۔اس بات سے اختتامی یقین کرایا جاتا ہے کہاورکسی صورت اکٹھے پست نہیں ہو سکتے۔یاد رہے کہ ایسا ہونے سے بیرونی پلٹ کے دونوں مخارج بلند ہو جائیں گے جو کہ ناقابلِ قبول صورت ہو گی۔یوں عبوری جدول میںاورکے خانے کوئی معنی نہیں رکھتے۔ان خانوں کولکھ کر واضح کیا گیا ہے۔

\جزوحصہ{ایس آر پلٹوں والے غیر معاصر ادوار کا قدم با قدم تجزیہ}\شناخت{حصہ_غیر_معاصر_قدم_با_قدم}
	اُوپر دیے مثالوں میں استعمال کئے طریقہ کار کو یہاں بیان کرتے ہیں۔پلٹ کے اپنے واپسیں اشارات کو نظر انداز کرتے ہیں۔
    • تمام پلٹوں کے مخارج کوسے ظاہر کریں اور اسی طرح ان میں سے جو واپسیں اشارات کے طور استعمال کئے گئے ہوں انہیںسے ظاہر کریں جہاںہے۔
    • تمام پلٹوں کے اورمداخل کے مساوات حاصل کریں۔
    • نفی۔جمع گیٹ پر مبنی ایس آر پلٹوں کے لئے تسلی کر لیں کہ ہے جبکہ نفی۔ضرب گیٹوں پر مبنی ایس آر پلٹوں کے لئےہونا ضروری ہے۔ایسا نہ ہونے کی صورت میں پلٹ غلط نتائج دے سکتا ہے۔
    • اورکو دیکھتے ہوئے تمام پلٹوں کےحاصل کریں۔
    • ہرکو کارناف نقشہ کے طرز پر بیان کریں۔ان نقشوں کے بائیں جانب قطار میں واپسیں اشاراتجبکہ نقشوں کے اُوپر صف میں بیرونی مداخللکھیں جہاںسے مرادجبکہسے مرادہے۔
    • ان تمام نقشوں کو عبوری جدول میں یکجا کریں۔نقشوں کے خانوں میںلکھیں، جہاں سے مرادہے۔
    • وہ خانے جن میںہے، مستحکم حال کو ظاہر کرتے ہیں۔انہیں دائرہ میں بند کر دیں۔یوں عبوری جدول حاصل ہوتا ہے۔



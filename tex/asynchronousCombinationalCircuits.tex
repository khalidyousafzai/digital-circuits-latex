\باب{ غیر معاصر ترتیبی ادوار }
	وسیع عددی ادوار عموماً معاصر ادوار کے طرز پر بنائے جاتے ہیں۔ایسے ادوار کے اگلی حالتیں مکمل طور ان کے موجودہ حالتوں سے متعین کئے جا سکتے ہیں۔حالات صرف ساعت کے کنارہ تبدیل ہوتے ہیں اور بقایا تمام وقت کے لئے غیر متغیر تصور کئے جا سکتے ہیں۔ایسے ادوار بناتے وقت اس بات کو یقینی بنایا جاتا ہے کہ ساعت کے لمحہ کنارہ سے قبل تمام حالتیں متوازن صورت اختیار کر چکھے ہوں۔یوں ساعت کے کنارہ پر متعین حالتیں پائی جاتی ہیں جن سے اگلے حالت مکمل طور حاصل کئے جا سکتے ہیں۔ 
	ان کے برعکس غیر معاصر ادوار کی حالتیں کسی بھی لمحہ تبدیل ہو سکتے ہیں۔اس سے حالتِ دوڑ اور دیگر مسائل کھڑے ہوتے ہیں جن پر اس باب میں غور کیا جائے گا۔
	غیر معاصر ادوار کی اپنی ایک اہمیت ہے۔یہ ساعت کے کنارہ کا انتظار کئے بغیر اشارہ پر ردِ عمل کر سکتے ہیں۔عموماً کسی بھی عددی دور میں کچھ حصہ معاصر اور کچھ غیر معاصر ہوتا ہے۔
	شکل 11.1 میں ایک نہایت سادہ دور دکھایا گیا ہے۔سرسری نظر سے یوں محسوس ہوتا ہے کہ اس شکل میں ضرب گیٹ کی مخارج کبھی بلند نہیں ہو سکتی۔غور کرنے سے ثابت ہوتا ہے کہ مسئلہ اتنا سادہ نہیں۔
	جب بھی مداخلحالت تبدیل کرتا ہے اس کے کچھ لمحہ بعد، نفی گیٹ کا مخارج حالت تبدیل کرتا ہے۔یہ تاخیر 1 نفی گیٹ کے دورانیہ ردِ عمل کی وجہ سے ہوتی ہے۔شکل میں اورکے خط بناتے وقت اس تاخیر کو دکھایا گیا ہے۔اگر ضرب گیٹ کا دورانیہ ردِ عمل صفر ہوتا تب ضرب گیٹ کا مخارج ان دو مداخل کے مطابق حالت اختیار کرتا۔اس کوسے دکھایا گیا ہے۔چونکہ ضرب گیٹ کو بھی ردِ عمل کے لئے کچھ دیر درکار ہوتا ہے لہٰذا ضرب گیٹ کی مخارج کچھ دیر کے بعد نظر آئے گی جیسا شکل میں دکھایا گیا ہے۔
	آپ دیکھ سکتے ہیں کہ ضرب گیٹ کی مخارج غیر مطلوبہ طور پر، نفی گیٹ کے دورانیہ ردِ عمل کے برابر وقت کے لئے، بلند ہو گئی ہے۔اس طرح غیر مطلوبہ، نہایت کم دورانیہ کے، حالت کی تبدیلی کو برقی لرزش 2 کہتے ہیں۔چونکہ برقی لرزش مثبت یا منفی ہو سکتی ہے لہٰذا موجودہ لرزش کو مثبت برقی لرزش کہیں گے۔
	برقی لرزش کی وجہ سے ادوار عبوری حالت 3 اختیار کرتے ہیں۔اس باب میں عبوری حالتوں پر تفصیلاً بحث ہو گی۔  

	آپ نے دیکھا کہ برقی لرزش، اشارہکے ضرب گیٹ تک پہنچنے میں تاخیر کی وجہ سے پیدا ہوئی۔اشارات میں تاخیر کی ایک اور مثال دیکھتے ہیں۔
	برقی تار میں برقی دباؤ کی رفتار تقریباً خلاء میں روشنی کے رفتار کے4 برابر ہوتی ہے۔یوں ایک نینو سیکنڈ () میں برقی دباؤ تقریباًمیٹر یعنی سنٹی میٹر فاصلہ طے کرتا ہے۔آئیے دیکھتے ہیں کہ اگر پچھلی مثال کو تبدیل کر کے اس میں نفی گیٹ کی جگہ سینٹی میٹر برقی تار لگائی جائے اور ضرب گیٹ کی جگہ بلا شرکت جمع گیٹ نصب کیا جئے تو دور کا ردِ عمل کیسا ہوگا۔اس مثال کو شکل11.2 میں دکھایا گیا ہے۔


	 اشارہگیٹ کے ایک داخلی پن پر مہیا کیا گیا ہے جبکہ اسی اشارہ کوسنٹی میٹر لمبی برقی تار کی مدد سے دوسرے داخلی پن پر بطور مہیا کیا گیا ہے۔شکل میں لمبے تار کو بل کھاتے لکیر سے دکھایا گیا ہے۔یوں اشارہ،گیٹ کے پن تک تاخیر سے پہنچتا ہے۔اشارہبلند یا پست ہونے کے ایک نینو سیکنڈ بعد اشارہبلند یا پست ہوتا ہے۔اگر گیٹ کے دورانیہ ردِ عمل کو نظر انداز کیا جائے تو گیٹ کی مخارجہوگی اور اگر گیٹ کے دورانیہ ردِ عمل کو بھی مدِ نظر رکھا جائے تو اس کی مخارجہوگی۔خارجی اشارہ میں دو بلند برقی لرزشیں دیکھنے کو آتی ہیں۔ان برقی لرزشوں کے دورانیہ، برقی تار میں اشارہ کے تاخیر کے برابر ہے۔یوں اشارات کی راہ میں مختلف تاخیرات،  معلومات کو ایک لمحہ برقرار رکھنے کی صلاحیت رکھتی ہیں۔یوں مائل تاخیرات کا کردار، حافظہ کی طرح کا ہے۔
	آپ نے دیکھا کہ مختلف طرز کے تاخیرات کی وجہ سے دور میں برقی لرزشیں پیدا ہوتی ہیں۔اگر واپسیں اشارہ تاخیر سے پہنچ کر مخارج تبدیل کرے، تو دورانِ تاخیر مخارج اور تاخیر کے بعد کا مخارج مختلف ہوں گے اور یوں یہ ایک غیر متوازن حالت کی صورت ہو گی۔
	گیٹوں اور برقی تاروں میں نا قابلِ معلوم تاخیرات کی وجہ سے جب بھی ایک سے زیادہ اشارات بیک وقت تبدیل ہوں، یہ دریافت کرنا تقریباً ناممکن ہو جاتا ہے کہ ان کے اثرات کیا مرتب ہوں گے۔یوں عموماً غیر معاصر ادوار بناتے وقت اس بات کو یقینی بنایا جاتا ہے کہ کسی بھی وقت صرف ایک ہی اشارہ تبدیل ہو۔مزید یہ کہ کسی بھی دو اشارت کے تبدیلی کے درمیان اتنا وقفہ دیا جاتا ہے کہ دور میں اشارات کے تمام تاخیر کے بعد دور متوازن صورت اختیار کر سکے۔ان دو شرائط کے تحت دور کے چلنے کو بنیادی طریقِ کار 5 کے تحت چلنا کہتے ہیں۔
11.1 تجزیہ
	غیر معاصر ترتیبی ادوار سے مراد ایسے ادوار ہیں جن میں یا تو بغیر ساعت والے پلٹ پائے جائیں اور یا پھر ان میں ایک یا ایک سے زیادہ مخارج کو بطور واپسیں اشارات استعمال کیا گیا ہو۔جیسے اُوپر دیکھا گیا، واپسیں اشارات ایک لمحہ کے لئے،مختلف تاخیرات کی بنا پر، حافظہ کی صلاحیت رکھتی ہے۔
	کسی بھی خارجی اشارہ، مثلاً،  کو جب اس طرح داخلی اشارہ کے طور استعمال  کیا جائے کہ یہکی قیمت کو متعین کرنے میں کردار ادا کر سکے تو اسے واپسیں اشارہ 6 کے طور استعمال کرنا کہتے ہیں۔
	اس حصہ میں بغیر پلٹ والے ادوار پر غور کیا جائے گا۔پلٹ والے دور پر اگلے حصہ میں غور کیا جائے گا۔
11.1.1 عبوری جدول
	غیر معاصر ترتیبی ادوار پر غور ان کے عبوری جدول 7 کی مدد سے کیا جاتا ہے۔اس طریقہ کو شکل 11.3 میں دئے دور کی مدد سے سیکھتے ہیں۔

	پلٹ کی غیر موجودگی کے باوجود اس دور کو ترتیبی دور اس لئے کہتے ہیں کہ خارجی اشاراتاورکو بطور واپسیں اشارات،اور، استعمال کیا گیا ہے۔اس دور سے خارجی حالتوں کے مساوات یوں حاصل ہوتے ہیں۔

 
(11.1)

	مساوات حاصل کرتے وقت واپسیں اشارات کو بالکل عام مداخل کی طرح سمجھا جاتا ہے۔یوں اس دور کی ایک بیرونی مداخل اور دو اندرونی مداخلاور ہیں۔ان دو مساوات سے بوولین جدول حاصل کرتے ہیں۔


جدول 11.1: دور کا بوولین جدول
اس جدول سے عبوری جدول کا حصول شکل 11.4 میں دکھایا گیا ہے۔
	جدول 11.1 میں متغیرہ حالتوں 8اورکی معلومات کو پہلے علیحدہ علیحدہ کارناف نقشہ 9 کی طرز پر لکھا گیا ہے۔بوولین جدول کی بجائے ان کا یوں لکھنا عبوری جدول لکھنے میں آسانی پیدا کرتا ہے۔کارناف نقشہ میں جدول کے بائیں جانب قطار میں اندرونی مداخل، یعنی،کی قیمتیں لکھی جاتی ہیں جبکہ جدول کی پہلی صف کے اوپر صف کی شکل میں بیرونی مداخل،یعنی، کی قیمت لکھی جاتی ہے۔عبوری جدول حاصل کرنے کے لئے متغیرہ حالاتاورکو ساتھ ساتھ، کر کے، لکھا جاتا ہے۔شکل میں متغیرہ حالتوں کے آخری صفوں کے دائیں قطاروں میںکی قیمت جبکہکی قیمتہے۔یوں عبوری جدول میں ان دو قیمتوں کو ساتھ ساتھ، یعنی، کر کے لکھا گیا ہے۔شکل میں نکتہ دار لکیروں سے اس عمل کو دکھایا گیا ہے۔عبوری جدول میں جہاں بھی جدول کے اندر متغیرہ حالت کی قیمت اور اسی صف میں بائیں جانبکی قیمت یکساں ہوں، وہاں جدول میںکی قیمت کو دائرہ میں بند کر دیا گیا ہے۔دائرہ میں بند متغیرہ حالت متوازن حالت ہے جبکہ بقایا حالتیں غیر متوازن یعنی عبوری ہیں۔
	شکل 11.5 پر نظر رکھتے ہوئے عبوری جدول کے استعمال پر غور کرتے ہیں۔اس کے کی صف اورکی قطار میں واقعہ خانے کو ابتدائی خانہ10 کہہ کر ظاہر کیا گیا ہے۔اس خانے میںاورکی صورت میں کی قیمت  درج ہے۔تصور کریں کہ ابتدائی خانہ، دور کی ابتدائی حالت کو ظاہر کرتا ہے۔
	عبوری جدول میں جس خانے میںاور اسی خانے کی صف میںکی قیمتیں یکساں ہوں وہاں دور متوازن حالت میں ہوتا ہے۔عبوری جدول میں ایسے تمام خانوں کے اندراج گول دائروں میں بند کئے جاتے ہیں۔ 
	اب اگر رکھتے ہوئے بیرونی مداخلکی قیمتسے  کر دی جائے تو عبوری جدول کے مطابقکی قیمتہو جائے گی اور یوں واپسیں اشاراتاور متغیرہ حالاتکی قیمتیں مختلف ہو جائیں گی۔یہ غیر متوازن صورتِ حال ہے اور دور اس میں زیادہ دیر نہیں رہ سکتا۔ برقی تاروں میں تاخیر کے بعد کی قیمت بھیہو جائے گی جبکہاپنی نئی قیمت برقرار رکھے گا۔یوں دور تاخیر کے بعد عبوری جدول کےکی قطار اور کی صف میں دئے خانے تک پہنچ جائے گا۔اس خانے میںاوردونوں کی قیمتیںہیں۔یہ ایک متوازن حالت کو ظاہر کرتا ہے اور اسی لئے دائرہ میں بند دکھایا گیا ہے۔شکل میں اس پورے مرحلہ کو پہلا قدم لکھ کر ظاہر کیا گیا ہے۔پہلے قدم کو ظاہر کرنے والا تیر کا نشان، غیر متوازن خانہ سے گزر کر متوازن خانے پر اختتام پذیر ہوا ہے۔
	آپ نے دیکھا کہ متوازن حالت سے شروع کرتے،کی قیمت تبدیل کرنے سے دور کچھ لمحات کے لئے غیر متوازن حالت اختیار کر گیا۔یہ صورت زیادہ دیر برقرار نہیں رہی۔تاروں میں تاخیر کے بعد، واپسیں اشارات تبدیل ہوئے اور دور ایک مرتبہ دوبارہ متوازن حالت اختیار کر گیا۔عموماً ادوار کا عمل اسی طرح ہوتا ہے۔
	اسی طرحرکھتے ہوئے  اگرکی قیمتسےکی جائے تو عبوری جدول کے مطابق دور کی قطار اورکے خانے میں درج حالت یعنیاختیار کرے گا۔ایک مرتبہ پھراورمختلف ہیں اور دور اس سے نکلنے کی کوشش کرے گا۔برقی تاروں میں تاخیر کے بعد کے نئے قیمتوں کی خبرکی مقام تک پہنچ جائے گی اورکی قیمت بھیہو جائے گی۔یوں دور،کی قطار اور کی صف میں درج،بند دائرہ میں دکھائے متوازن حال، یعنی، اختیار کر لے گا۔اسی سلسلہ کو چلاتے ہوئے بار بار کی قیمت تبدیل کرنے سے دور ،،  اورمتوازن حالت اختیار کرتا ہے۔کے بعد یہی ترتیب بار بار دہرائی جاتی ہے۔شکل میں تیر والے لکیروں سے یہ تمام مراحل دکھائے گئے ہیں۔
	کسی بھی دور کے متوازن حالت اور غیر متوازن حالتیں لکھتے وقت اس دور کے حالت کوکی بجائےلکھا جاتا ہے۔ اس طرح موجودہ دور کے متوازن حالتیں ،،اورہیں جبکہ اس کے غیر متوازن حالتیں ،،اورہیں۔
	عبوری جدول کے ہر صف میں عموماً کم از کم ایک متوازن حالت ضرور پایا جاتا ہے۔ایسا نہ ہونے کی صورت میں دور اس صف میں پہنچ کر غیر متوازن صورت اختیار کرے گا۔
	عبوری جدول حاصل کرنے کے طریقہ کو یہاں بیان کرتے ہیں۔
    • تمام واپسیں اشارت اور واپسیں دائروں کا تعین کریں
    • کسی بھی ترتیب سے واپسیں دائروں کے مخارج کو ،،وغیرہ جبکہ اسی ترتیب سے ان کے واپسیں اشارات کو، ،وغیرہ سے شناخت کریں۔
    • ان تمام مخارج کے بوولین تفاعل کو بیرونی اور اندرونی مداخل کی صورت میں حاصل کریں۔
    • ان تفاعل کے کارناف نقشے بنائیں۔
    • تمام کارناف نقشوں کو ایک عبوری جدول میں یکجا کریں۔جدول کے خانوں میں  لکھیں جبکہ جدول کے بائیں جانب ہر صف میں اسی ترتیب سے لکھیں۔
    • جہاںاور اسی صف میںکی قیمت یکساں ہو، وہاںکو دائرہ میں بند کر دیں۔
جدول کے حصول کے بعد بیرونی مداخل تبدیل کر کے دور کے عبوری حالتوں پر غور کیا جاسکتا ہے۔ 
11.1.2 جدولِ بہاو
	شکل 11.4 میں عبوری جدول لکھتے وقت خانوں میں دور کی حالتیں بوولین طرز پر لکھے گئے۔دو مخارج کی صورت میں یہاں چار ممکنہ حالتیں ہیں یعنی ،،اور۔ان چار حالتوں کو نام بھی دئے جا سکتے ہیں۔مثلاً جب دورحال میں ہو تو کہا جائے کہ دور حالمیں ہے۔اسی طرحکو حال،کو حالاورکو حالکہا جا سکتا ہے۔ایسا کرنے سے عبوری جدول سے حاصل جدول کو جدولِ بہاو 11 کہتے ہیں۔شکل 11.6 میں ایسا ہوا دکھایا گیا ہے۔
	شکل میں دئے جدولِ بہاو میں ہر صف میں صرف ایک ہی متوازن حالت ہے۔مثلاً پہلی صف میں صرف متوازن حالت جبکہ دوسری صف میں صرفمتوازن حالت ہیں۔ ایسے جدول جن کے صفوں میں صرف ایک ہی متوازن حالت ہو کو اول جدولِ بہاو 12 کہتے ہیں۔

	شکل 11.7 میں ایک ایسا جدولِ بہاو دکھایا گیا ہے جس کے صفوں میں ایک سے زیادہ متوازن حالت پائے جاتے ہیں۔مثلاً پہلی صف میں،اورمتوازن حالت ہیں۔ایسے جدول کو غیر اولین جدولِ بہاو کہتے ہیں۔
	 جدولِ بہاو سے دور حاصل کرنے کی خاطر پہلے عبوری جدول حاصل کرتے ہیں۔جدول کے دو صف سے ظاہر ہے کہ دور کے دو ممکنہ حالتیں ہیں۔دو ممکنہ صورتوں کو ایک بِٹ کے عدد سے ظاہر کیا جا سکتا ہے۔یوں حالتکو اور حالت کو لکھتے ہوئے عبوری جدول حاصل کیا جاتا ہے۔اسے شکل میں دکھایا گیا ہے۔

	دور کے مخارج کولکھتے ہوئے عبوری جدول سے اس کا تفاعل حاصل کرتے ہیں۔

 
(11.2)

اس تفاعل کا دور شکل 11.8 میں دکھایا گیا ہے۔


	 اس جدولِ بہاو کے استعمال پر شکل 11.9 کی مدد سے غور کرتے ہیں۔تصور کریں کہ بیرونی مداخلکی قیمتہے،یعنی،  اور دور حالمیں ہے۔اگرکو تبدیل کئے بغیرکی قیمتکر دی جائے،یعنیکر دیا جائے، تو عبوری جدول کے مطابق دور  عبوری طور پر غیر متوازن حالاختیار کرتے ہوئے آخر کار متوازن حالاختیار کر لے  گا۔اب اگرکی قیمترکھتے ہوئےکی قیمت بھیکر دی جائے،یعنیکر دیا جائے، تو دور حالمیں ہی رہے گا۔اس اختتامی خانہ کو شکل میں پہلا اختتامی خانہ کہا گیا ہے۔شکل میں ابتدائی خانہ سے پہلی اختتامی خانہ تک پہنچنے کا عمل تین تیر والے لکیروں سے دکھایا گیا ہے جہاں پہلا تیر  متوازن حالسے عبوری حالکا حصول جبکہ دوسرا تیر یہاں سے متوازن حالکا حصول دکھلاتا ہے۔تیسرا تیر متوازن حالسے متوازن حالمیں ہی رہنے کو ظاہر کرتا ہے۔


	دیکھتے ہیں کہ اگر ابتدائی خانہ سے شروع کرتے وقت بجائےبرقرار رکھتے اور تبدیل کرنے کے ہمکی قیمتہی رکھتے ہوئےکی قیمتکرتے ہیں، یعنیکرتے ہیں۔ایسے کرتے جدولِ بہاو کے مطابق نظام حالہی برقرار رکھے گا۔اب اگرکی قیمت بھیکر دی جائے۔یعنیکر دیا جائے، تو نظام کا اختتامی حالہی رہے گا۔اس اختتامی خانے کو شکل میں دوسرا اختتامی خانہ کہا گیا ہے۔
	آپ نے دیکھا کہ اختتامی حال، بیرونی مداخل کے تبدیلی کے ترتیب پر منحصر ہے۔اس مثال میں ابتدائی بیرونی مداخلجبکہ اختتامی بیرونی مداخلہیں۔بھولئے گا نہیں کہ ایسے ادوار کے صحیح استعمال میں ایک سے زیادہ بیرونی مداخل بیک وقت تبدیل نہیں کئے جا سکتے۔یوںسے ابتدا کرتے ہوئے ہم سیدھاپر نہیں جا سکتے۔ایسا کرنے سے، ناقابلِ معلوم تاخیرات کی بنا پر، اختتامی حالت دریافت کرنا ناممکن ہوتا ہے۔
11.1.3 حالتِ دوڑ
	حالتِ دوڑ 13 کا تذکرہ ایس-آر پلٹ پر تبصرے کے وقت آیا تھا۔اس حصہ میں اس پر تفصیلاً گفتگو ہو گی۔حالتِ دوڑ اس صورت کو کہتے ہیں جب کسی بیرونی اشارہ کے تبدیل ہونے سے دور کے ایک سے زیادہ حالتیں تبدیل ہوں۔تاخیرات کی وجہ سے ایسی صورت میں حالتوں کی تبدیلی مکمل طور پر جاننا ناممکن ہو جاتا ہے۔مثلاً دو حالتوں والے دور کی موجودہ متوازن حالتہے۔بیرونی مداخل کی تبدیلی  سے اس کے دونوں حالتیں تبدیل ہوتے ہیں اور آخر کار یہمتوازن حالت اختیار کر لیتا ہے۔اگر پہلی واپسیں راہ میں تاخیر دوسری واپسیں راہ کے تاخیر سے کم ہو تو دور متوازن حالت سے عبوری حالت اور آخر کار متوازن حالاختیار کرے گا جبکہ اگر دوسری راہ میں تاخیر پہلی راہ سے کم ہو تب حالسے عبوری حالاور پھرہو گا۔لہٰذا آپ نے دیکھا کہ جس ترتیب سے حالتیں تبدیل ہوتے ہیں اسے جاننا ممکن نہیں14۔
	اگر عبوری حالتوں کے تبدیلی کی ترتیب حتمی حالت متعین کرنے میں کردار ادا کرے اور یہ ممکن ہو کہ دور دو مختلف حتمی متوازن حالت اختیار کر سکے، اس صورت میں دوڑ کو بحرانی دوڑ 15 کہتے ہیں۔کسی بھی دور کے کارآمد استعمال کے لئے یہ اشد ضروری ہے کہ اس میں بحرانی دوڑ کی صورت ممکن نہ ہو۔اگر عبوری حالت کے تبدیلی کی ترتیب کا حتمی متوازن حالت پر کوئی اثر نہ ہو، اس صورت میں دوڑ کو غیر بحرانی دوڑ 16 کہتے ہیں۔



	بحرانی دوڑ کی ایک مثال شکل 11.10 میں دکھائی گئی ہے۔یہاں حالت کو مکمل حالت کے طور یعنیلکھتے ہوئے اگر سے ابتدا کی جائے اور بیرونی مداخلکوسےکیا جائے تو دور حتمی حالکی جانب دوڑ لگائے گا۔تاخیرات کی وجہ سے دور تین ممکنہ حالتوں یعنی،اور میں سے کسی بھی حالت تک پہلے پہنچ سکتا ہے۔شکل میں یہ تینوں عبوری حالت پہلے صف میں دکھائے گئے ہیں۔اگر دورکے عبوری حالت تک پہلے پہنچے تو یہ یہاں سے ہوتے ہوئے حتمی متوازن حالاختیار کر لے گا اور یہیں رہے گا۔اس حالت کو دوسری صف میں دائرہ میں بند دکھایا گیا ہے۔اگر دونوں واپسیں راہ میں مائل تاخیرات بالکل برابر ہوں تو دورعبوری حالت میں پہلے پہنچے گا اور یہاں سے ہوتے ہوئے حتمی متوازن حالاختیار کر لے گا اور یہیں رہے گا۔اس حتمی متوازن حالت کو تیسری صف میں دائرہ میں بند دکھایا گیا ہے۔تیسری صورت میں دور پہلے عبوری حالپہنچے گا۔یہاں سے یہ آخری صف کی جانب رواں ہو گا لیکن آخری صف ازخود عبوری حالہے۔یوں دور عبوری حالسے بھی گزر کر آخر کار تیسرے صف کے حتمی متوازن حالتک پہنچے گا۔اس مثال میں دو حتمی حالتیں ممکن ہیں۔یہ دریافت کرنا ناممکن ہے کہ دور ان میں سے کس حتمی حالت تک پہنچتا ہے۔اس شکل میں بائیں جانب کی قطار خالی اس لئے رکھی گئی ہے کہ ہم صرفسے کی جانب جاتے دور پر غور کر رہے ہیں۔اس صورت میں ان خانوں کے اندراج کی ہمیں ضرورت نہیں۔
	شکل 11.11 میں بحرانی دوڑ کی دوسری مثال دکھائی گئی ہے جہاں تین ممکنہ حتمی حالتیں پائے جاتے ہیں۔اگر مکمل متوازن حالسے ابتدا کرتے ہوئے بیرونی مداخلکی قیمتکر دی جائے تو یہ دور حتمی حالکی طرف دوڑ لگائے گا۔بالکل اُوپر مثال کی طرح، تین ممکنہ عبوری صورتیں ہیں۔ایک عبوری صورتہے جہاں سے یہ دوسری صف میں دکھائے حتمی متوازن حالتک پہنچتا ہے۔دوسری عبوری صورتہے جہاں سے یہ تیسری صف کے حتمی متوازن حالپہنچتا ہے اور تیسری عبوری صورتہے جہاں سے یہ آخری صف میں دکھائے حتمی متوازن حالت تک پہنچتا ہے۔اس مثال میں تین ممکنہ حتمی متوازن حالتیں ہیں۔یہ جاننا ناممکن ہے  کہ ان میں سے دور کس حالت کو اختیار کرتا ہے۔


	اب دیکھتے ہیں غیر بحرانی دوڑ کی ایک مثال جسے شکل 11.12 میں دکھایا گیا ہے۔اس مثال میں ابتداسے کرتے تین عبوری حالت ممکن ہیں۔


	ایک ممکنہ عبوری حال ہے جہاں سے دور دوسری صف کے ایک اور عبوری حالاور یہاں سے تیسری صف کے عبوری حالسے ہوتے ہوئے آخر کار  چوتھی صف کے حتمی متوازن حالتک پہنچتا ہے۔
	دوسری ممکنہ عبوری حالہے جہاں سے یہ تیسری صف کے عبوری حالسے ہوتے ہوئے آخر کار آخری صف کے حتمی متوازن حالتک پہنچتا ہے۔
	تیسری ممکنہ عبوری حالہے جہاں سے یہ ہوتے ہوئے آخری صف کے حتمی متوازن حالتک پہنچ جاتا ہے۔
	اس مثال میں اگرچہ تین مختلف ممکنات موجود ہیں لیکن حتمی متوازن حالت سب کا ایک ہی ہے۔یوں یہ غیر بحرانی دوڑ ہے۔  
	اگر دور مخصوص اور منفرد عبوری حالتوں سے گزر کر حتمی متوازن صورت اختیار کرتا ہو تو اسے پھیرا 17 لگانا کہتے ہیں۔اس کی مثال شکل 11.13 میں دی گئی ہے۔ 


	ان اشکال پر غور کریں۔ان میں دوڑ کی حالت نہیں پائی جاتی چونکہ ایک وقت میں صرف ایک مخارج حالت تبدیل کرتا ہے البتہ حتمی حالت تک پہنچنے کی خاطر دور کو مخصوص اور منفرد عبوری حالتوں سے گزرنا ہوتا ہے۔
	شکل کے حصہ الف میں دورسے عبوری حالت اور پھرسے حتمی حالتپہنچتا ہے۔اسی طرح حصہ با میںسے   عبوری حالتکے راستے حتمی حالت اختیار کرتا ہے۔

11.1.4 توازن اور ارتعاش 
	 ایسے دور جو پھیرے لگاتے ہوئے کسی بھی حتمی متوازن حالت تک نہ پہنچ پائے کو غیر متوازن دور 18 کہتے ہیں۔شکل 11.14 میں اس کی مثال دکھائی گئی ہے۔اگر بیرونی مداخل کوکیا جائے تو دور ایک عبوری حالت سے دوسری عبوری حالت منتقل ہوتا رہتا ہے مگر کسی متوازن حالت تک نہیں پہنچ پاتا۔
	اس طرح کے ادوار بطور مرتعش 19 استعمال کئے جاتے ہیں۔بطور مرتعش استعمال ہونے والے ادوار کے علاوہ ادوار کو کسی صورت غیر متوازن نہیں ہونے دیا جاتا۔

11.2 حالتِ دوڑ سے پاک ثنائی علامتوں کا تقرر
	حالتِ دوڑ کی صورت اس وقت پیدا ہوتی ہے جب ایک سے زیادہ مخارج بیک وقت حالت تبدیل کرنے کی کوشش کریں۔بحرانی دوڑ کی صورت میں ادوار قابلِ استعمال نہیں رہتے۔اس حصہ میں بحرانی دوڑ کے خاتمے پر غور کیا جائے گا۔یہ یاد دہانی کراتے چلیں کہ غیر معاصر ادوار کو استعمال کرتے وقت ان کے مداخل پر یہ شرط لاگو کی جاتی ہے کہ کسی بھی وقت صرف ایک مداخل حالت تبدیل کر سکتا ہے لہٰذا ایک سے زیادہ مداخل کی تبدیلی کی فقر اس حصہ کو پڑھتے ہوئے نہ کریں۔
	جن ادوار میں ایک وقت پر صرف ایک مخارج حالت تبدیل کرنے کی کوشش کرے، ایسے ادوار حالتِ دوڑ سے دوچار نہیں ہوتے۔اسی حقیقت کو بروئے کار لاتے ہوئے حالتِ دوڑ ختم کی جاتی ہے۔
	عبوری جدول کے حصول کے بعد اس میں درج حالتوں کو ثنائی علامتیں تعین کی جاتی ہیں۔اگر ایسے حالتیں جن کے مابین، عبوری جدول میں، تبادلہ پایا جائے کو  ہمسایہ ثنائی علامتیں تعین کی جائیں تو دور بحرانی دوڑ سے دو چار نہیں ہوگا۔دو ثنائی اعداد کو اس صورت ہمسایہ اعداد کہا جاتا ہے جب ان میں صرف ایک ہندسہ مختلف ہو۔یوںاورہمسایہ اعداد 20 ہیں چونکہ ان اعداد میں صرف ایک بِٹ مختلف ہے۔اسی طرحاورآپس میں ہمسایہ ہیں جبکہاورآپس میں ہمسایہ نہیں ہیں۔
	اس ترکیب کو شکل 11.15 کے حصہ الف میں دئے مثال کی مدد سے دیکھتے ہیں۔



	اس شکل میں کُل چار صف ہیں۔یوں دو بِٹ متغیرہ حالت 21 سے اس کے چار ممکنہ حالت بیان کئے جا سکتے ہیں۔ہم حالکے لئے، حالت کے لئے،حالتکے لئےاور حالتکے لئےکی متغیرہ حالتیں منتخب کرتے ہیں۔ایسا کرنے سے دیکھتے ہیں کہ کیا نتائج رو نما ہوتے ہیں۔
	اس شکل کی پہلی صف میں اگرکی قیمتسےکی جائے تو حالسے تبدیل ہو کرہو جائے گا۔ہم دیکھتے ہیں کہ متغیرہ حالاتکی قیمتسےہو جائے گی۔یوں متغیرہ حالت کی صرف ایک بِٹ تبدیل ہوتی ہے اور یوں اس صورت میں حالتِ دوڑ پیدا نہیں ہوتا۔اب دیکھتے ہیں کہ اگر شکل کی پہلی صف میںکی قیمتسے کی جائے تو حالسے تبدیل ہو کرہو جائے گا۔یوں ہم دیکھتے ہیں کہ متغیرہ حالاتکی قیمتسےہونے کی کوشش کرے گی جس سے حالتِ دوڑ پیدا ہوتا ہے۔یوں ہم دیکھتے ہیں کہ دو بِٹ کی متغیرہ حالت کے تقرر سے حالتِ دوڑ سے بچھنا ممکن نہیں۔ایسی صورت میں دو سے زیادہ بِٹ پر مبنی متغیرہ حالت استعمال کر کے دیکھا جاتا ہے کہ آیا حالتِ دوڑ سے چٹکارا ممکن ہے۔
	کبھی کبھار ایسا ممکن ہوتا ہے کہ چار صف کی عبوری جدول میں دو بِٹ متغیرہ حالت اس طرح تقرر کئے جائیں کہ حالتِ دوڑ پیدا نہ ہو۔شکل کے حصہ با میں متغیرہ حالت کی ترتیب بدل کر ایسا کرنے کی کوشش کی گئی ہے جہاں آپ دیکھ سکتے ہیں کہ پہلی صف سے شروع کرتے حالسے حالتقرری سے کی قیمتسے ہوتی ہے جبکہ حالسے حالتقرری سے کی قیمتسےہوتی ہے۔دونوں صورتوں میں چونکہ متغیرہ حالت کی صرف ایک بِٹ تبدیل ہوتی ہے لہٰذا پہلی صف میں حالتِ دوڑ کا کوئی امکان نہیں۔البتہ دوسری صف کو دیکھتے ہوئے اگر مداخل کی قیمتسےکی جائے تو حالسے تبدیل ہو کرہو جائے گا اور اس شکل میں متغیرہ حالتکی قیمتسےہو جائے گی۔اس صورت متغیرہ حالت کے دو بِٹ بیک وقت تبدیل ہوتے ہیں جو کہ حالتِ دوڑ پیدا کرتا ہے۔
	ان دو صورتوں سے ظاہر ہے کہ موجودہ مسئلہ میں دو بِٹ کے متغیرہ حالت کی تقرری سے حالتِ دوڑ سے نجات حاصل کرنا ممکن نہیں۔ایسی صورت میں حالتِ دوڑ سے پاک متغیرہ حالت کے لئے ہم ایک بلند بِٹ تقرری 22 کا طریقہ استعمال کریں گے۔یہ طریقہ استعمال میں نہایت آسان ہے۔آئیے اس طریقہ کو اسی مثال پر استعمال کرتے دیکھیں۔
	شکل 11.16 میں اسی مثال کو لیتے ہوئے، متغیرہ حالت کو چار بِٹ رکھا گیا ہے۔مزید یہ کہ ہر حالت کے متغیرہ حالت کی تقرری یوں کی گئی ہے کہ اس میں صرف ایک بلند بِٹ ہو۔یوں حالکا متغیرہ حالمقرر کیا گیا ہے جبکہ حالکا،حالکا  اور حالکا مقرر کیا گیا ہے۔


	شکل 11.16 میں جدول کی پہلی صف میں اگر مداخل کی قیمتسےکی جائے تو دور حالسے حالمنتقل ہوتا ہے۔یوں متغیرہ حالت کی قیمتسے تبدیل ہو کرہو گی جس سے دو بِٹ تبدیل ہوتے ہیں اور یوں یہ حالتِ دوڑ پیدا کرے گی۔اس صورت سے یوں بچا جا سکتا ہے کہ جدول میں ایک نیا عبوری حال،، شامل کیا جائے اور اس عبوری حالت کو استعمال کرتے، حالسے عبوری حالکے ذریعہ حالتک پہنچا جائے۔عبوری حالکا متغیرہ حالت یوں مقرر کیا جاتا ہے کہ یہ حالاور حالدونوں کا ہمسایہ عدد ہو۔ایسا عدد ہے۔یوں حالکا متغیرہ حالمقرر کیا جاتا ہے اور جدول کو تبدیل کر کےکی قطار کے حالکی صف میںکو تبدیل کر کےلکھ لیا جات ہے جبکہ اسی قطار میں حالکی صف میںلکھا جاتا ہے۔ایسا کرنے سے جدول تبدیل ہو کر شکل 11.17 کی شکل اختیار کر لے گا۔


	اس شکل کی پہلی صف میں مداخل کی سےتبدیلی سے مشین حالسے عبوری حالاختیار کرتے ہوئے آخر کار حتمی متوازن حالتک پہنچتا ہے۔شکل میں نکتہ دار تیر کی لکیروں سے یہ عمل دکھایا گیا ہے۔آپ دیکھ سکتے ہیں کہ اس پورے عمل میں کسی ایک قدم پر متغیرہ حالت کا صرف ایک بِٹ تبدیل ہوتا ہے اور یوں یہ حالتِ دوڑ سے پاک ہے۔شکل میںکے صف میں بقایا خانے خالی رکھے گئے ہیں۔ان میں سے کچھ خانے زیرِ استعمال آئیں گے اور کچھ نہیں۔زیرِ استعمال نہ آنے والے خانے خالی رکھے جاتے ہیں۔ان کی قیمت غیر ضروری 23 ہوتی ہے۔
	اسی سلسلہ کو پہلی صف میں مداخل کےسےکے تبادلہ کی صورت میں استعمال کرتے ہیں۔شکل 11.17 میں آپ دیکھ سکتے ہیں کہ ایسا کرنے سے مشین حالسے حالمنتقل ہونا چاہتا ہے۔متغیرہ حالت کو دیکھتے ہوئے یہ بات واضح ہے کہ یہسے تبدیل ہو کرہونا چاہتا ہے۔البتہ ایسا کرنے سے حالتِ دوڑ پیدا ہوتا ہے جسے ہم بالکل پچھلی صورت کی طرح حل کریں گے۔
	اس صورت سے یوں بچا جا سکتا ہے کہ جدول میں ایک نیا عبوری حال،، شامل کیا جائے اور اس عبوری حالت کو استعمال کرتے، حالسے عبوری حالکے ذریعہ حالتک پہنچا جائے۔عبوری حالکا متغیرہ حالت یوں مقرر کیا جاتا ہے کہ یہ حالاور حالدونوں کا ہمسایہ عدد ہو۔ایسا عدد ہے۔یوں حالکا متغیرہ حالمقرر کیا جاتا ہے اور جدول کو تبدیل کر کےکی قطار کے حالکی صف میںکو تبدیل کر کےلکھ لیا جاتا ہے جبکہ اسی قطار میں حالکی صف میںلکھا جاتا ہے۔ایسا کرنے سے جدول تبدیل ہو کر شکل 11.18 کی شکل اختیار کر لے گا۔

	یہی طریقہ کار تمام خانوں کے لئے دہرایا جاتا ہے۔ایسا کرنے سے شکل 11.19 حاصل ہوتا ہے۔طلبہ سے گزارش کی جاتی ہے کہ وہ اس جدول کو از خود حاصل کریں۔تسلی کر لیں کہ اس جدول میں کسی بھی حالت سے دوسرے حالت تک پہنچنے میں حالتِ دوڑ پیدا نہیں ہوتا۔


11.3 پلٹوں کا عبوری جدول کی مدد سے تجزیہ
	عبوری جدول کے استعمال سے اس حصہ میں پلٹوں والے ادوار کا تجزیہ کیا جائے گا۔چند مثالوں کے بعد حصہ 11.3.3 میں اس طریقہ کار کا قدم با قدم طریقہ دیا جائے گا۔
11.3.1 ایس آر پلٹ
	عبوری جدول کے استعمال سے  سب سے پہلے ایس-آر پلٹ پر غور کرتے ہیں۔شکل 11.20 میں اُوپر جانب ایس-آر پلٹ دکھایا گیا ہے۔اسی کے نیچے اسے واپسیں دور 24 کی طرح دکھایا گیا ہے جہاں واپسیں اشارہ کی پہچان آسانی سے ممکن ہے۔


	شکل میں متغیرہ حالکو بطور واپسیں اشارہاستعمال کیا گیا ہے۔یوں دور میں متغیرہ حال، اندرونی مداخل جبکہ اور  دو بیرونی مداخل ہیں۔انہیں استعمال کرتے، شکل میں دکھائی، عبوری جدول حاصل کی گئی ہے۔آئیے اس پلٹ کا تجزیہ اس کے عبوری جدول کی مدد سے کریں۔پلٹ کی جدولِ درسگی مندرجہ ذیل ہے۔

 
(11.3)

	اس جدول سے ظاہر ہے کہ نفی۔جمع گیٹ پر مبنی ایس-آر پلٹ کا صحیح استعمال تب ممکن ہے جب اس کے دونوں مداخل کسی صورت اکٹھے بلند نہ ہوں چونکہ ایسا ہونے سے پلٹ کے مخارجاوردونوں پست ہو جاتے ہیں جبکہ کسی بھی پلٹ کے مخارج کا ہر صورت آپس میں متضاد رہنا ضروری ہے۔اس شرط کو یوں بیان کیا جا سکتا ہے کہ نفی۔جمع گیٹ پر مبنی ایس-آر پلٹ کے مداخل کو ہر صورت مندرجہ ذیل مساوات پر پورا اترنا چاہئے۔

 
(11.4)
 
	شکل 11.21 کو دیکھتے آگے پڑھیں۔عبوری جدول میںکی قطار میں متوازن حالکی صف میں پایا جاتا ہے جہاں متغیرہ حالیعنی پست ہے۔اگر کیا جئے تو عبوری جدول کے مطابق متغیرہ حالت پست ہی رہے گا۔اس عمل کو شکل کے حصہ الف میں نکتہ دار تیر سے دکھایا گیا ہے۔
	اسی طرح  کی صورت میں پلٹ کا متوازن بلند حالت  کی صف میں پایا جاتا ہے۔اگرکیا جائے تو عبوری جدول کے مطابق پلٹ بلند حالت میں ہی رہتا ہے جیسے شکل کے حصہ با میں دکھایا گیا ہے۔یہ دونوں اعمال پلٹ کے بوولین جدول سے بھی وضح ہے۔
	اب دیکھتے ہیں کہسےکرنے سے کیا صورت پیدا ہوتی ہے۔پہلے تو یاد دہانی کراتے چلیں کہ اس طرح کے ادوار کو بنیادی طریقِ کار 25 کے طرز پر چلایا جاتا ہے جہاں ایک سے زیادہ بیرونی مداخل تبدیل کرنے کی اجازت نہیں ہوتی۔بہرحال پھر بھی دیکھتے ہیں کہ ایسا کرنے سے کیا مسائل کھڑے ہوتے ہیں۔
	کرنے سے پہلے تو بوولین جدول کے مطابقاوردونوں پست ہوتے ہیں۔اس طرح یہ آپس میں متضاد حالت میں نہیں ہوتے جبکہ کسی بھی پلٹ کے لئے یہ لازم ہے کہ اس کے دونوں مخارج ہر وقت متضاد حالت میں ہوں۔دوسری بات یہ کہ عبوری جدول کو دیکھتے ہوئے اگرپہلے پست حالت اختیار کر لے تو حتمی حالہو گا جبکہ اگر پہلے پست ہو پائے تب حتمی حالہو گا۔چونکہ یہ قبل از وقت معلوم کرنا ناممکن ہے کہ ان میں پہلے کون پست حالت اختیار کرے گا لہٰذا یہ جاننا ناممکن ہے کہ حتمی حالت کیا ہو گا۔یوں اس طرح، دور کا استعمال غیر یقینی صورت پیدا کرتا ہے۔

11.3.2 ساعت کے کنارے چلتا ڈی پلٹ
	شکل 11.22 میں ساعت کے کنارہ چلتا ڈی پلٹ دکھایا گیا ہے۔ڈی پلٹ میں اندرونی واپسیں دور پایا جاتا ہے جس کے اندرونی متغیرہ حالاتاورہیں26۔یوں اس کے واپسیں اشاراتاورہیں۔شکل میں دور کو دوبارہ واپسیں دور کی طرز پر بنایا گیا ہے تا کہ واپسیں اشارات اورکی پہچان آسان ہو۔



	اس دور کے اورمتغیرہ حالات،اورواپسیں اشارات  جبکہاوربیرونی مداخل ہیں۔یوں ہم لکھ سکتے ہیں۔

 
(11.5)

	شکل 11.23 میں ان مساوات سے حاصلاورکے بوولین جدول  کو کارناف نقشہ کی طرح لکھ کر عبوری جدول حاصل کیا گیا ہے۔مکمل حالت کو کی صورت میں لکھتے ہوئے اس جدول پر غور کرتے ہیں۔


	تصور کریں کہ جس لمحہ پلٹ کو برقی طاقت مہیا کر کے زندہ کیا جاتا ہے  اس لمحہ  ساعت،یعنی، اور بیرونی مداخل، یعنی،دونوں پست ہیں۔اس صورت عبوری جدول کے مطابق دورکی قطار میں ہوگا۔اس قطار میں تین خانے عبوری متغیرہ حالت کو ظاہر کرتے ہیں۔یہ تین خانے،اورہیں۔ان تینوں خانوں میں عبوری حالہے۔چوتھا خانہ،یعنی، متوازن حالت کو ظاہر کرتا ہے اور اس میں متوازن حالہے۔یوں اگر برقی طاقت کے فراہمی کے لمحہ تاخیرات ایسے ہوں کہ دور ان تین عبوری خانوں میں کسی ایک میں داخل ہوتا ہے تو یہاں سے جلد وہ کی صف پہنچ کر متوازن حالت اختیار کر لے گا۔اگر زندہ ہوتے ہی دور سیدھاخانہ میں داخل ہو تب یہ یہی رہے گا۔
	اس کے برعکس برقی طاقت مہیا کرنے کے لمحہ اگراورہوں تو عبوری جدول کے مطابق دوریاکے متوازن حالت تک پہنچ کر یہی رہے گا جبکہاورکی صورت میں دور یامیں ہو گا۔
	پست ساعت کی صورت میں متغیرہ حالاتکی قیمترہتی ہے۔عبوری جدول میںاورکی دو قطاریں اس بات کو ظاہر کرتی ہیں جہاں تمامکی قیمتیںہیں۔ہم جانتے ہیں کہ ایس-آر پلٹ کی دونوں مداخل بلند ہونے کی صورت میں پلٹ اپنی حالت برقرار رکھتا ہے۔یوں شکل 11.22 میں اس صورت میں خارجی پلٹ اپنی حالت برقرار رکھے گا۔
	پست ساعت، یعنی، اور پست،یعنی،کی صورت میں متوازن متغیرہ حالت حاصل کرنے کی خاطر ہم عبوری جدول کےکی قطار میں دیکھتے ہیں جہاں ہمیں مکمل حالبطور متوازن حالت ملتا ہے۔جدول کے اس خانے میں لکھ کر اسے واضح کیا گیا ہے۔یہاںہونے کی وجہ سے خارجی پلٹ اپنی حالت برقرار رکھے گا۔
	پست ساعت اور بلندکی صورت میںکی قطار میں متوازن حالپایا جاتا ہے جہاںہی ہے اور یوں خارجی پلٹ اپنی حالت برقرار رکھے گا۔جدول کے اس خانے میں لکھ کر اسے واضح کیا گیا ہے۔
	تصور کریں کہ دور کے متوازن حال،یعنی خانہ،میں ہوتے ہوئے بیرونی مداخلبلند ہوتا ہے۔بیرونی مداخلجس لمحہ سےہوتا ہے اس لمحہ کو ساعت کا کنارہ چڑھائی 27 کہتے ہیں۔ یوںکی صورت میں ساعت کے کنارہ چڑھائی آنے سے دور خانہکی صف میں رہتے ہوئے،سےکی قطار میں داخل ہو کر عبوری صورتاختیار کرتا ہے۔اس عبوری حالت کو خانہ کہا گیا ہے۔یہاں سے یہ جلد حتمی متوازن حالتک پہنچتا ہے۔اس خانہ کوکہا گیا ہےحال میں متغیرہ حالت ہیں۔خارجی پلٹکی صورت میں پست حالت اختیار کر لے گا اور یوںہو جائے گا۔اس قدم کو شکل میں خانہ سے  خانہکے راستے خانہتک تیر والے لکیر سے دکھایا گیا ہے۔اس پورے کا نچوڑ یہ ہے کہ کی صورت میں ساعت کے کنارہ چڑھائی پر ہو جائے گا یعنی ڈی پلٹ پست حالت اختیار کر لیتا ہے۔
	اس پورے عمل پر دوبارہ غور کریں۔ساعت کے کنارہ چڑھائی آتے ہی دور عبوری حالاور پھر متوازن حالاختیار کرتا ہے۔ان دونوں حالت میںہی رہتے ہیں اور یوں عبوری حالت سے گزرتے ہوئے کسی قسم کی لرزش پیدا نہیں ہوتی۔آپ نیچے پڑھتے ہوئے ہر قدم پر تسلی کر لیں کہ کسی بھی عبوری حالت سے گزرتے وقتکی قیمت وہی ہوتی ہے جو اس قدم کے حتمی حالت میں گی۔یوں ایسے لمحات پر لرزش سے کسی قسم کی غیر یقینی صورت پیدا نہیں ہوتی۔
		اسی طرح مکمل حالمیں موجود دور، ساعت کے کنارہ چڑھائی آتے، عبوری حالسے ہوتے ہوئے متوازن حالاختیار کرے گا۔اس قدم کو شکل میں خانہ سے  خانہکے راستے خانہتک تیر والے لکیر سے دکھایا گیا ہے۔یہ قدم بلند بیرونی مداخل یعنیکی صورت میں ساعت کے کنارہ چڑھائی پر ہونے کا عمل ہے جس سے داخلی پلٹ بلند ہو جائے گا اور یوں ڈی پلٹ کاہو جائے گا۔
	ساعت کے کنارہ اترائی کے عمل کو نکتہ دار تیر والے لکیروں سے دکھایا گیا ہے۔انہیں آپ خود سمجھ سکتے ہیں۔یہ دونوں لکیریں اس بات کو واضح کرتی ہیں کہ ساعت کے کنارہ اترائی پر عبوری حالت اور حتمی متوازن حالت دونوں میں ہوتا ہے۔ہونے کی صورت میں بیرونی پلٹ اپنی حالت برقرار رکھتا ہے اور یوں ساعت کے کنارہ اترائی پر ڈی پلٹ کے حال میں کسی قسم کی تبدیلی رو نما نہیں ہوتی۔
	ایک آخری بات اس پلٹ کے حوالہ سے کرتے ہیں۔شکل 11.22 میں اشارہکوپیدا کرنے والے نفی۔ضرب گیٹ کو داخلی اشارہ کے طور مہیا کیا گیا ہے۔اس بات سے حتمی یقین کرایا جاتا ہے کہاورکسی صورت اکٹھے پست نہیں ہو سکتے۔یاد رہے کہ ایسا ہونے سے بیرونی پلٹ کے دونوں مخارج بلند ہو جائیں گے جو کہ ناقابلِ قبول صورت ہو گی۔یوں عبوری جدول میںاورکے خانے کوئی معنی نہیں رکھتے۔ان خانوں کولکھ کر واضح کیا گیا ہے۔
11.3.3 ایس-آر پلٹوں والے غیر معاصر ادوار کا قدم با قدم تجزیہ
	اُوپر دئے مثالوں میں استعمال کئے طریقہ کار کو یہاں بیان کرتے ہیں۔پلٹ کے اپنے واپسیں اشارات کو نظر انداز کرتے ہیں۔
    • تمام پلٹوں کے مخارج کوسے ظاہر کریں اور اسی طرح ان میں سے جو واپسیں اشارات کے طور استعمال کئے گئے ہوں انہیںسے ظاہر کریں جہاںہے۔
    • تمام پلٹوں کے اورمداخل کے مساوات حاصل کریں۔
    • نفی۔جمع گیٹ پر مبنی ایس-آر پلٹوں کے لئے تسلی کر لیں کہ ہے جبکہ نفی۔ضرب گیٹوں پر مبنی ایس-آر پلٹوں کے لئےہونا ضروری ہے۔ایسا نہ ہونے کی صورت میں پلٹ غلط نتائج دے سکتا ہے۔
    • اورکو دیکھتے ہوئے تمام پلٹوں کےحاصل کریں۔
    • ہرکو کارناف نقشہ کے طرز پر بیان کریں۔ان نقشوں کے بائیں جانب قطار میں واپسیں اشاراتجبکہ نقشوں کے اُوپر صف میں بیرونی مداخللکھیں جہاںسے مرادجبکہسے مرادہے۔
    • ان تمام نقشوں کو عبوری جدول میں یکجا کریں۔نقشوں کے خانوں میںلکھیں، جہاں سے مرادہے۔
    • وہ خانے جن میںہے، متوازن حالت کو ظاہر کرتے ہیں۔انہیں دائرہ میں بند کر دیں۔یوں عبوری جدول حاصل ہوتا ہے۔



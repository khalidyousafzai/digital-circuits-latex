\باب{دفتر}
ایک پلٹ کار ایک ثنائی ہندسے (بِٹ) کی معلومات ذخیرہ کر سکتا ہے۔ آٹھ بِٹ معلومات ذخیرہ کرنے کے لئے آٹھ پلٹ کار درکار ہوں گے۔\اصطلاح{دفتر}\فرہنگ{دفتر}\حاشیہب{register}\فرہنگ{register} سے مراد وہ دور ہے جو معلومات ذخیرہ، اور ایک جگہ سے دوسری جگہ منتقل کر نے کی صلاحیت رکھتا ہو۔یوں، \عددی{n} بِٹ دفتر سے مراد \عددی{n} پلٹ کار پر مبنی وہ دور ہو گا، جو \عددی{n} بِٹ ذخیرہ اور منتقل کر سکے۔معلومات کے انتقال کا انداز (سلسلہ وار یا متوازی) دور کے ترکیبی حصہ پر منحصر ہو گا۔

\begin{figure}
\centering
\begin{subfigure}{0.45\textwidth}
\centering
\begin{tikzpicture}
\pgfmathsetmacro{\kshYa}{2.5}
\pgfmathsetmacro{\kshPa}{1.0}
\pgfmathsetmacro{\kshPb}{0.75}
\pgfmathsetmacro{\kshPc}{1.0}
\pgfmathsetmacro{\kshPd}{0.5}
\pgfmathsetmacro{\kshPe}{0.5}
\kDFF[u0]{0}{0};
\kDFF[u1]{0}{\kshYa};
\kDFF[u2]{0}{2*\kshYa};
\kDFF[u3]{0}{3*\kshYa};
\draw(u0p6)node[right]{$B0$} (u1p6)node[right]{$B1$} (u2p6)node[right]{$B2$} (u3p6)node[right]{$B3$};
\draw(u0p1)--++(-\kshPa,0)node[left]{$A0$} (u1p1)--++(-\kshPa,0)node[left]{$A1$} 
(u2p1)--++(-\kshPa,0)node[left]{$A2$} (u3p1)--++(-\kshPa,0)node[left]{$A3$};
\draw(u0p2)--++(-\kshPd,0) (u1p2)--++(-\kshPd,0) (u2p2)--++(-\kshPd,0) (u3p2)--++(-\kshPd,0);
\draw(u3p2)++(-\kshPd,0)--++(0,-3*\kshYa-2*\kshPd)node[below]{$C$}; 
\end{tikzpicture}
\caption{}
\end{subfigure}\hfill
\begin{subfigure}{0.45\textwidth}
\centering
\begin{tikzpicture}
\pgfmathsetmacro{\kshYa}{2.5}
\pgfmathsetmacro{\kshPa}{1.0}
\pgfmathsetmacro{\kshPb}{0.75}
\pgfmathsetmacro{\kshPc}{1.0}
\pgfmathsetmacro{\kshPd}{0.5}
\pgfmathsetmacro{\kshPe}{0.5}
\kDFFud[u0]{0}{0};
\kDFFud[u1]{0}{\kshYa};
\kDFFud[u2]{0}{2*\kshYa};
\kDFFud[u3]{0}{3*\kshYa};
\draw(u0p6)node[right]{$B0$} (u1p6)node[right]{$B1$} (u2p6)node[right]{$B2$} (u3p6)node[right]{$B3$};
\draw(u0p1)--++(-\kshPa,0)node[left]{$A0$} (u1p1)--++(-\kshPa,0)node[left]{$A1$} 
(u2p1)--++(-\kshPa,0)node[left]{$A2$} (u3p1)--++(-\kshPa,0)node[left]{$A3$};
\draw(u0pd)--++(-\kshPb,0) (u1pd)--++(-\kshPb,0) (u2pd)--++(-\kshPb,0) (u3pd)--++(-\kshPb,0);
\draw(u0pu)--++(-\kshPc,0) (u1pu)--++(-\kshPc,0) (u2pu)--++(-\kshPc,0) (u3pu)--++(-\kshPc,0);
\draw(u3pd)++(-\kshPb,0)--++(0,-3*\kshYa-\kshPd)coordinate(blw)node[below right]{$\overline{\text{\RL{بیٹھ}}}$};
\draw(u0pu)++(-\kshPc,0)--++(0,3*\kshYa+\kshPd)node[above]{$\overline{\text{\RL{اٹھ}}}$};
\draw(u0p2)--++(-\kshPd,0) (u1p2)--++(-\kshPd,0) (u2p2)--++(-\kshPd,0) (u3p2)--++(-\kshPd,0);
\draw(u3p2)++(-\kshPd,0)--++(0,-3*\kshYa)coordinate(cc)--(cc|-blw)node[below]{$C$}; 
\end{tikzpicture}
\caption{}
\end{subfigure}
\caption{چار بِٹ دفتر۔}
\label{شکل_دفتر_چار_بٹ_ڈی}
\end{figure}
 سادہ ترین چار بِٹ دفتر شکل \حوالہ{شکل_دفتر_چار_بٹ_ڈی} میں پیش ہے۔شکل-الف میں مداخل \عددی{A} جبکہ مخارج \عددی{B} ہے۔ مداخل کے چار بِٹ \عددی{A_0}، \عددی{A_1}، \عددی{A_2}، اور \عددی{A_3}،جبکہ مخارج کے چار بِٹ \عددی{B_0}، \عددی{B_1}، \عددی{B_2}، اور \عددی{B_3} ہیں ۔

ساعت کے کنارہ چڑھائی پر داخلی چار بِٹ پلٹ کار کو منتقل ہو جاتے ہیں۔ ہم کہتے ہیں دفتر میں مواد کا اندراج ہو گیا، یا مواد دفتر میں درج ہو گیا، یا مواد دفتر میں لکھ لیا گیا۔ساعت کے اگلے کنارہ چڑھائی تک یہ چار بِٹ معلومات دفتر میں محفوظ، اور مخارج پر دستیاب ہو گی۔

شکل \حوالہ{شکل_دفتر_چار_بٹ_ڈی}-ب میں بلند اور پست صلاحیت کا پلٹ کار استعمال کیا گیا۔یوں، ساعت کے کنارہ چڑھائی کا انتظار کیے بغیر، تمام خارجی بِٹ زبردستی بلند یا پست کیے جا سکتے ہیں۔ زبردستی پست کرنے سے دفتر صاف ہو کر \عددی{0000_2}، جبکہ زبردستی بلند کرنے سے \عددی{1111_2} خارج کرتا ہے۔ 

اس دور میں پلٹ کار کی تعداد \عددی{n} کر کے \عددی{n} بِٹ دفتر تشکیل دیا جا سکتا ہے۔ ہر بِٹ کا متمم بھی دفتر کے مخارج سے دستیاب ہو گا۔ یوں \عددی{B_0} کا متمم \عددی{\overline{B}_0} مطابقتی پلٹ کار کے \عددی{\overline{Q}} سے دستیاب ہو گا۔ 


\حصہ{سلسلہ وار دفتر}
\جزوحصہ{دائیں انتقال دفتر}
شکل \حوالہ{شکل_دفتر_دائیں_منتقل} میں (سلسلہ وار) \اصطلاح{ دائیں انتقال دفتر}\فرہنگ{دفتر!دائیں انتقال}\حاشیہب{shift right register}\فرہنگ{register!shift right} پیش ہے، جہاں (متواتر) ایک پلٹ کار کا مخارج، دوسرے کا مداخل ہے، اور ثنائی مواد، \عددی{x}، بائیں (جانب) سے مہیا کیا گیا ہے۔شکل میں زبردستی پست پن نہیں دکھایا گیا تا کہ اصل مضمون پر توجہ رہے، تاہم تصور کریں ساعت کے پہلے کنارہ چڑھائی سے قبل، تمام پلٹ کار زبردستی پست کیے گئے۔
\begin{figure}
\centering
\begin{tikzpicture}
\pgfmathsetmacro{\kshXa}{2.5}
\pgfmathsetmacro{\kshPa}{1.00}
\pgfmathsetmacro{\kshPb}{0.5}
\kDFF[u0]{0}{0};
\kDFF[u1]{-\kshXa}{0};
\kDFF[u2]{-2*\kshXa}{0};
\kDFF[u3]{-3*\kshXa}{0};
\draw(u3p6)node[above]{$Q_3$}--(u2p1) (u2p6)node[above]{$Q_2$}--(u1p1) (u1p6)node[above]{$Q_1$}--(u0p1) (u0p6)node[above]{$Q_0$} (u3p1)--++(-\kshPb,0)node[left]{$x$};
\draw(u0p2)--++(0,-\kshPa) (u1p2)--++(0,-\kshPa) (u2p2)--++(0,-\kshPa) (u3p2)--++(0,-\kshPa)coordinate(cc);
\draw(u0p2)++(0,-\kshPa)--(cc)--++(-\kshPb,0)node[left]{$C$};
\draw(u0-north-west)node[above]{$u_0$};
\draw(u1-north-west)node[above]{$u_1$};
\draw(u2-north-west)node[above]{$u_2$};
\draw(u3-north-west)node[above]{$u_3$};
\path(u1-south)++(0,-1);
\end{tikzpicture}
\caption{دائیں انتقال دفتر}
\label{شکل_دفتر_دائیں_منتقل}
\end{figure}


ساعت کے پہلے کنارہ چڑھائی پر \عددی{u_0} کو \عددی{Q_1=0}، \عددی{u_1} کو \عددی{Q_2=0}، \عددی{u_2} کو \عددی{Q_3=0}، اور \عددی{u_4} کو \عددی{x=1} مواد فراہم ہے، جنہیں پلٹ کار، ساعت کے کنارہ چڑھائی پر، مخارج منتقل کرتے ہیں۔یوں پہلے کنارہ چڑھائی گزرنے کے بعد \عددی{Q_0=0}، \عددی{Q_1=0}، \عددی{Q_2=0}، اور \عددی{Q_3=1} ہو گا۔یاد رہے، ساعت کے کنارہ چڑھائی کے دوران، پلٹ کار گزشتہ حال میں رہتا ہے، اور نیا مواد کنارہ گزرنے کے بعد مخارج کو پہنچتا ہے۔ آپ نے دیکھا، یہ دور، مواد کی دائیں رخ نقل مکانی کرتا ہے، جس کی وجہ سے اس کو\موٹا{دائیں انتقال دفتر} کہتے ہیں۔ 

ساعت کے دوسرے کنارہ چڑھائی کے وقت، \عددی{u_0} کو \عددی{Q_1=0}، \عددی{u_1} کو \عددی{Q_2=0}، \عددی{u_2} کو \عددی{Q_3=1}، اور \عددی{u_4} کو \عددی{x} (جو \عددی{0} یا \عددی{1} ہو گا) مواد فراہم ہے، لہٰذا ساعت کا دوسرا کنارہ چڑھائی گزرنے کے بعد \عددی{Q_0=0}، \عددی{Q_1=0}، \عددی{Q_2=1}، اور \عددی{Q_3=x} ہو گا۔

دور کو سلسلہ وار فراہم بائیں سے مواد، سلسلہ وار دائیں پلٹ کے مخارج \عددی{Q_0} سے اسی ترتیب میں حاصل کیا جا سکتا ہے۔

\جزوحصہ{بائیں انتقال دفتر}
شکل \حوالہ{شکل_دفتر_بائیں_منتقل} میں (سلسلہ وار) \اصطلاح{بائیں انتقال دفتر}\فرہنگ{دفتر! بائیں انتقال}\حاشیہب{shift left register}\فرہنگ{register!shift left} دکھایا گیا ہے، جو مواد کی بائیں نقل مکانی کرتا ہے۔اس کی بناوٹ بالکل دائیں انتقال دفتر کی طرح ہے۔فرق صرف اتنا ہے، بائیں انتقال دفتر میں دایاں پلٹ کار کا مخارج پڑوسی بایاں پلٹ کار کا مداخل ہے۔

ساعت کے کنارہ چڑھائی پر دایاں پلٹ کار فراہم کردہ مواد \عددی{x} کی نقل حاصل کر کے \عددی{Q_0} پر خارج کرتا ہے۔ اگلے کنارہ پر یہ مواد \عددی{Q_1} کو منتقل ہو گا۔ آپ دیکھ سکتے ہیں کہ یہاں مواد دائیں سے فراہم کیا گیا ہے، جو دور میں سے گزرتے ہوئے بائیں منتقل ہو گا۔

\begin{figure}
\centering
\begin{tikzpicture}
\pgfmathsetmacro{\kshXa}{2.5}
\pgfmathsetmacro{\kshPa}{1.0}
\pgfmathsetmacro{\kshPb}{0.3}
\pgfmathsetmacro{\kshPc}{0.3}
\kuDFF[u0]{0*\kshXa}{0}
\kuDFF[u1]{-1*\kshXa}{0}
\kuDFF[u2]{-2*\kshXa}{0}
\kuDFF[u3]{-3*\kshXa}{0}
\path(u0p6)--(u1p4)coordinate[pos=0.5](aau);
\path(u0p1)--(u1p3)coordinate[pos=0.5](aad);
\draw(u0p6)--(aau)--(aad)--(u1p1);
\path(u1p6)--(u2p4)coordinate[pos=0.5](aau);
\path(u1p1)--(u2p3)coordinate[pos=0.5](aad);
\draw(u1p6)--(aau)--(aad)--(u2p1);
\path(u2p6)--(u3p4)coordinate[pos=0.5](aau);
\path(u2p1)--(u3p3)coordinate[pos=0.5](aad);
\draw(u2p6)--(aau)--(aad)--(u3p1);
\draw(u0p1)--++(\kshPa,0)coordinate(rgtin);
\kinright[j1]{rgtin}
\draw(j1east)node[right]{$x$};
\draw(u0p6)--++(0,\kshPb)node[above]{$Q_0$};
\draw(u1p6)--++(0,\kshPb)node[above]{$Q_1$};
\draw(u2p6)--++(0,\kshPb)node[above]{$Q_2$};
\draw(u3p6)--++(0,\kshPb)node[above]{$Q_3$};
\draw(u0p2)--++(0,-\kshPc);
\draw(u1p2)--++(0,-\kshPc);
\draw(u2p2)--++(0,-\kshPc);
\draw(u3p2)--++(0,-\kshPc);
\draw(u0p2)++(0,-\kshPc)--($(u3p2)+(0,-\kshPc)$)--++(-\kshPa,0)node[left]{$C$};
\end{tikzpicture}
\caption{بائیں انتقال دفتر}
\label{شکل_دفتر_بائیں_منتقل}
\end{figure}


\جزوحصہ{ دائیں و بائیں انتقال دفتر}
شکل \حوالہ{شکل_دفتر_بائیں_و_دائیں} میں (سلسلہ وار) بائیں و دائیں انتقال دفتر پیش ہے جو مواد کی بائیں یا دائیں نقل مکانی کی صلاحیت رکھتا ہے۔ مخارج \عددی{Q_2} پلٹ کار کے مداخل \عددی{D} اور اس سے منسلک جمع گیٹ اور (دو) ضرب گیٹ پر توجہ رکھیں۔ قابو اشارہ \عددی{(\text{دائیں}/{\overline{\text{بائیں}}})} بلند ہونے کی صورت میں، دایاں ضرب گیٹ معذور جبکہ بایاں مجاز ہو کر، جمع گیٹ تک \عددی{Q_3} پہنچاتے ہیں جو \عددی{D} پر دستیاب اور ساعت کے اگلے کنارہ چڑھائی پر پلٹ کار میں درج ہو کر بطور \عددی{Q_2} خارج ہو گا۔یوں مواد \عددی{Q_3} سے \عددی{Q_2} یعنی دائیں منتقل ہوا۔اس کے برعکس قابو اشارہ پست ہونے کی صورت میں، دایاں ضرب گیٹ مجاز اور بایاں معذور ہو کر، جمع گیٹ تک \عددی{Q_1} پر موجود مواد پہنچاتے ہیں، جو آخر کار \عددی{Q_2} پہنچتا ہے، اور یوں مواد بائیں منتقل ہوتا ہے۔

 بائیں ترین پلٹ کار کو بیرونی مواد \عددی{y} جبکہ دائیں ترین کو \عددی{x} فراہم کیا گیا ہے۔ قابو اشارہ ان میں سے ایک منتخب کر تا ہے جو مطلوبہ سمت ( دائیں یا بائیں) منتقل ہو گا۔ 

بائیں نقل مکانی کے دوران \عددی{x} پر میسر مواد ساعت کے کنارہ چڑھائی پر \عددی{Q_0} پہنچتا ہے۔ اگلے کنارہ پر یہی مواد \عددی{Q_1}، اس سے اگلے پر \عددی{Q_2} اور آخر میں \عددی{Q_3} پہنچتا ہے۔ دائیں نقل مکانی کی صورت میں \عددی{y} پر موجود مواد الٹ رخ \عددی{Q_3} سے \عددی{Q_0} نقل مکانی کرتا ہے۔
\begin{figure}
\centering
\begin{tikzpicture}
\pgfmathsetmacro{\ksepYa}{2.75}
\pgfmathsetmacro{\kpinA}{0.5}
\pgfmathsetmacro{\kpinB}{0.4}
\kuDFF[u0]{0}{0}
\kuDFF[u1]{-1*\ksepYa}{0}
\kuDFF[u2]{-2*\ksepYa}{0}
\kuDFF[u3]{-3*\ksepYa}{0}
\draw(u0p1)node[or port,scale=1,number inputs=2,rotate=90,anchor=out](u5){};
\draw(u5.in 1)--++(-\kpinB,0)node[and port,scale=1,number inputs=2,rotate=90,anchor=out](u6){};
\draw(u5.in 2)--++(\kpinB,0)node[and port,scale=1,number inputs=2,rotate=90,anchor=out](u7){};
\draw(u1p1)node[or port,scale=1,number inputs=2,rotate=90,anchor=out](u8){};
\draw(u8.in 1)--++(-\kpinB,0)node[and port,scale=1,number inputs=2,rotate=90,anchor=out](u9){};
\draw(u8.in 2)--++(\kpinB,0)node[and port,scale=1,number inputs=2,rotate=90,anchor=out](u10){};
\draw(u2p1)node[or port,scale=1,number inputs=2,rotate=90,anchor=out](u11){};
\draw(u11.in 1)--++(-\kpinB,0)node[and port,scale=1,number inputs=2,rotate=90,anchor=out](u12){};
\draw(u11.in 2)--++(\kpinB,0)node[and port,scale=1,number inputs=2,rotate=90,anchor=out](u13){};
\draw(u3p1)node[or port,scale=1,number inputs=2,rotate=90,anchor=out](u14){};
\draw(u14.in 1)--++(-\kpinB,0)node[and port,scale=1,number inputs=2,rotate=90,anchor=out](u15){};
\draw(u14.in 2)--++(\kpinB,0)node[and port,scale=1,number inputs=2,rotate=90,anchor=out](u16){};
\draw(u0p6)node[above]{$Q_0$};
\draw(u1p6)node[above]{$Q_1$};
\draw(u2p6)node[above]{$Q_2$};
\draw(u3p6)node[above]{$Q_3$};
\draw(u7.in 2)--++(\kpinA,0)coordinate(rgtin) (u6.in 1)node[below]{$Q_1$};
\kinright[j2]{rgtin}
\draw(j2east)node[right]{$x$};
\draw(u10.in 2)node[below]{$Q_0$} (u9.in 1)node[below]{$Q_2$};
\draw(u13.in 2)node[below]{$Q_1$} (u12.in 1)node[below]{$Q_3$};
\draw(u16.in 2)node[below]{$Q_2$} (u15.in 1)--++(-\kpinA,0)coordinate(lft);
\kinleft[j1]{lft}
\draw(j1west)node[left]{$y$};
\draw(u0p2)--(u3p2)--(u3p2 -|lft)node[left]{$C$};
\draw(u16.in 1)--++(0,-3*\kpinA)node[not port,scale=1,anchor=out](u17){};
\draw(u17.out)-|(u7.in 1);
\draw(u10.in 1)coordinate(aa)--(aa |- u17.out);
\draw(u13.in 1)coordinate(aa)--(aa |- u17.out);
\draw(u6.in 2)--++(0,-1.3*\kpinA)-|coordinate(bb)(u15.in 2) (bb)-|(u17.in)--++(-\kpinA,0)node[left]{$\text{دائیں}/{\overline{\text{بائیں}}}$};
\draw(u9.in 2)coordinate(aa)--(aa |- bb);
\draw(u12.in 2)coordinate(aa)--(aa |- bb);
\end{tikzpicture}
\caption{بائیں و دائیں انتقال دفتر}
\label{شکل_دفتر_بائیں_و_دائیں}
\end{figure}


\حصہ{متوازی بھرائی دفتر}
بعض اوقات، دفتر میں بیک وقت مواد چڑھانے کی ضرورت پیش آتی ہے۔ شکل \حوالہ{شکل_دفتر_متوازی_انتقال_دفتر} میں \اصطلاح{دائیں انتقال، متوازی بھرائی دفتر}\فرہنگ{دفتر!متوازی بھرائی}\حاشیہب{parallel load, right shift register}\فرہنگ{register!parallel load} پیش ہے، جس میں متوازی مواد بیک وقت چڑھانا ممکن ہے۔یہ مختصراً \موٹا{متوازی دائیں انتقال دفتر} کہلاتا ہے۔

\begin{figure}
\centering
\begin{tikzpicture}
\pgfmathsetmacro{\ksepYa}{2.75}
\pgfmathsetmacro{\kpinA}{0.5}
\pgfmathsetmacro{\kpinB}{0.4}
\pgfmathsetmacro{\kpinC}{2.00}
\kuDFF[u0]{0}{0}
\kuDFF[u1]{-1*\ksepYa}{0}
\kuDFF[u2]{-2*\ksepYa}{0}
\kuDFF[u3]{-3*\ksepYa}{0}
\draw(u0p1)node[or port,scale=1,number inputs=2,rotate=90,anchor=out](u5){};
\draw(u5.in 1)--++(-\kpinB,0)node[and port,scale=1,number inputs=2,rotate=90,anchor=out](u6){};
\draw(u5.in 2)--++(\kpinB,0)node[and port,scale=1,number inputs=2,rotate=90,anchor=out](u7){};
\draw(u1p1)node[or port,scale=1,number inputs=2,rotate=90,anchor=out](u8){};
\draw(u8.in 1)--++(-\kpinB,0)node[and port,scale=1,number inputs=2,rotate=90,anchor=out](u9){};
\draw(u8.in 2)--++(\kpinB,0)node[and port,scale=1,number inputs=2,rotate=90,anchor=out](u10){};
\draw(u2p1)node[or port,scale=1,number inputs=2,rotate=90,anchor=out](u11){};
\draw(u11.in 1)--++(-\kpinB,0)node[and port,scale=1,number inputs=2,rotate=90,anchor=out](u12){};
\draw(u11.in 2)--++(\kpinB,0)node[and port,scale=1,number inputs=2,rotate=90,anchor=out](u13){};
\draw(u3p1)node[or port,scale=1,number inputs=2,rotate=90,anchor=out](u14){};
\draw(u14.in 1)--++(-\kpinB,0)node[and port,scale=1,number inputs=2,rotate=90,anchor=out](u15){};
\draw(u14.in 2)--++(\kpinB,0)node[and port,scale=1,number inputs=2,rotate=90,anchor=out](u16){};
\kupinAnchors
\kuBuffer[u17]{\pfx}{\kpinB+\pfy}
\draw(u17pout)--++(0,\kpinA)node[above]{$D_0$};
\kuBuffer[u18]{-1*\ksepYa+\pfx}{\kpinB+\pfy}
\draw(u18pout)--++(0,\kpinA)node[above]{$D_1$};
\kuBuffer[u19]{-2*\ksepYa+\pfx}{\kpinB+\pfy}
\draw(u19pout)--++(0,\kpinA)node[above]{$D_2$};
\kuBuffer[u20]{-3*\ksepYa+\pfx}{\kpinB+\pfy}
\draw(u20pout)--++(0,\kpinA)node[above]{$D_3$};
\draw(u0p6)++(0,0.25)--++(\kpinA,0)coordinate(kkt);
\koutright[j0]{kkt}
\draw(j0east)node[right]{\text{\RL{سلسلہ وار مخارج}}};
\draw(u0p6)--++(0,\kpinB);
\draw(u1p6)--++(0,\kpinB);
\draw(u2p6)--++(0,\kpinB);
\draw(u3p6)--++(0,\kpinB);
\draw(u0p6)node[above left]{$Q_0$};
\draw(u1p6)node[above left]{$Q_1$};
\draw(u2p6)node[above left]{$Q_2$};
\draw(u3p6)node[above left]{$Q_3$};
\draw(u7.in 2)--++(0,-\kpinC)node[below]{$z_0$} (u6.in 1)node[below]{$Q_1$};
\draw(u10.in 2)--++(0,-\kpinC)node[below]{$z_1$} (u9.in 1)node[below]{$Q_2$};
\draw(u13.in 2)--++(0,-\kpinC)node[below]{$z_2$} (u12.in 1)node[below]{$Q_3$};
\draw(u16.in 2)--++(0,-\kpinC)node[below]{$z_3$} (u15.in 1)--++(-\kpinA,0)coordinate(lft);
\kinleft[j1]{lft}
\draw(j1west)node[left]{$y$};
\draw(u0p2)--(u3p2)--(u3p2 -|lft)node[left]{$C$};
\draw(u16.in 1)--++(0,-3*\kpinA)node[not port,scale=1,anchor=out](u21){};
\draw(u21.out)-|(u7.in 1);
\draw(u10.in 1)coordinate(aa)--(aa |- u21.out);
\draw(u13.in 1)coordinate(aa)--(aa |- u21.out);
\draw(u6.in 2)--++(0,-1.3*\kpinA)-|coordinate(bb)(u15.in 2) (bb)-|(u21.in)--++(-\kpinA,0)
node[left]{$\overline{\text{متوازی بھرائی}}$};
\draw(u9.in 2)coordinate(aa)--(aa |- bb);
\draw(u12.in 2)coordinate(aa)--(aa |- bb);
\draw(u17pbu)--(u17u);
\draw(u17pnu)node[ocirc]{};
\draw(u18pbu)--(u18u);
\draw(u18pnu)node[ocirc]{};
\draw(u19pbu)--(u19u);
\draw(u19pnu)node[ocirc]{};
\draw(u20pbu)--(u20u);
\draw(u20pnu)node[ocirc]{};
\draw(u17u)--++(0,1.5*\kpinA)coordinate(aa);
\draw(u18u)--++(0,1.5*\kpinA);
\draw(u19u)--++(0,1.5*\kpinA);
\draw(u20u)--++(0,1.5*\kpinA)coordinate(bb);
\draw(aa)--(bb)--++(-\kpinA,0)node[left]{$\overline{\text{\RL{متوازی خارج}}}$};
%\draw(u20pbd)--(u20d);
%\draw(u20pnd)node[ocirc]{};
%\draw(u20pnpin)node[ocirc]{};
%\draw(u20pnpout)node[ocirc]{};
\end{tikzpicture}
\caption{متوازی دائیں انتقال دفتر}
\label{شکل_دفتر_متوازی_انتقال_دفتر}
\end{figure}

پلٹ کار کو جمع گیٹ معلومات فراہم کرتا ہے جس کو دو ضرب گیٹ مواد فراہم کرتے ہیں۔ قابو اشارہ \عددی{\overline{\text{متوازی بھرائی}}} عام طور غیر فعال (بلند) رکھا جاتا ہے۔یوں دایاں ضرب گیٹ معذور جبکہ بایاں گیٹ مجاز ہو کر، بائیں پلٹ کار کا مخارج، جمع گیٹ کے راستے پلٹ کار کو فراہم کرتا ہے، جو ساعت کے اگلے کنارہ چڑھائی پر پلٹ کار میں درج ہو گا۔

 مواد \عددی{z_0} تا \عددی{z_3} پلٹ کار میں چڑھانے کے لئے \عددی{\overline{\text{متوازی بھرائی}}} پست کیا جاتا ہے۔ یوں پلٹ کار کو مواد فراہم کرنے والا بایاں ضرب گیٹ معذور جبکہ دایاں مجاز ہو گا۔ مجاز گیٹ متوازی مواد کو جمع گیٹ کے راستہ پلٹ کار تک پہنچاتا ہے۔
 
یوں پلٹ کار میں مواد سلسلہ وار \عددی{(y)} یا متوازی (\عددی{z_0} تا \عددی{z_3}) بھرا جا سکتا ہے۔
 
شکل میں پلٹ کار کا مخارج، مجاز و معذور صلاحیت مستحکم کار سے منسلک کیا گیا ہے۔ قابو اشارہ \عددی{\overline{\text{\RL{متوازی خارج}}}} پست کر کے پلٹ کار کا مواد \عددی{Q_0} تا \عددی{Q_3} بطور \عددی{D_0} تا \عددی{D_3} حاصل کیا جا سکتا ہے۔قابو اشارہ معذور (بلند) ہونے کی صورت میں مستحکم کار کا مخارج بلند رکاوٹ حال میں ہو گا۔


\حصہ{عالمگیر انتقال دفتر}
ہم مختلف صلاحیت کے دفاتر پر غور کر چکے، جن کی خوبیاں ایک دور میں سموئی جا سکتی ہیں۔ایسا ایک \اصطلاح{عالمگیر انتقال دفتر }\فرہنگ{انتقال دفتر!عالمگیر}\حاشیہب{universal shift register}\فرہنگ{shift register!universal} شکل \حوالہ{شکل_دفتر_عالمگیر_انتقال_دفتر} میں پیش ہے۔

\begin{figure}
\centering
\begin{tikzpicture}
\pgfmathsetmacro{\ksepXa}{2.75}
\pgfmathsetmacro{\ksepXb}{0.20}
\pgfmathsetmacro{\ksepYa}{2.75}
\pgfmathsetmacro{\ksepYb}{1.25}
\pgfmathsetmacro{\ksepYc}{0.75}
\pgfmathsetmacro{\ksepYd}{0.85}
\pgfmathsetmacro{\ksepYe}{0.65}
\pgfmathsetmacro{\kpinA}{0.5}
\pgfmathsetmacro{\kpinB}{0.4}
\pgfmathsetmacro{\kpinC}{2.00}
\kuDFF[u0]{0}{0}
\kuDFF[u1]{-1*\ksepXa}{0}
\kuDFF[u2]{-2*\ksepXa}{0}
\kuDFF[u3]{-3*\ksepXa}{0}
\kupinAnchors
\kuBuffer[u17]{\pfx}{\kpinB+\pfy}
\draw(u17pout)--++(0,\kpinA/3)node[above]{$D_0$};
\kuBuffer[u18]{-1*\ksepXa+\pfx}{\kpinB+\pfy}
\draw(u18pout)--++(0,\kpinA/3)node[above]{$D_1$};
\kuBuffer[u19]{-2*\ksepXa+\pfx}{\kpinB+\pfy}
\draw(u19pout)--++(0,\kpinA/3)node[above]{$D_2$};
\kuBuffer[u20]{-3*\ksepXa+\pfx}{\kpinB+\pfy}
\draw(u20pout)--++(0,\kpinA/3)node[above]{$D_3$};
\draw(u17pbu)--++(-\kpin,0)--++(0,\ksepYe)coordinate(tt);
\draw(u17pnu)node[ocirc]{};
\draw(u18pbu)--++(-\kpin,0)--++(0,\ksepYe);
\draw(u18pnu)node[ocirc]{};
\draw(u19pbu)--++(-\kpin,0)--++(0,\ksepYe);
\draw(u19pnu)node[ocirc]{};
\draw(u20pbu)--++(-\kpin,0)--++(0,\ksepYe)coordinate(ttt);
\draw(u20pnu)node[ocirc]{};
\draw(u0p6)--++(0,\kpinB);
\draw(u1p6)--++(0,\kpinB);
\draw(u2p6)--++(0,\kpinB);
\draw(u3p6)--++(0,\kpinB);
\draw(u0p6)node[above left]{$Q_0$}++(0,0.25)coordinate(ka);
\draw(u1p6)node[above left]{$Q_1$};
\draw(u2p6)node[above left]{$Q_2$};
\draw(u3p6)node[above right]{$Q_3$}++(0,0.25)coordinate(kb);
\koutright[j0]{ka}
\draw(j0east)node[right]{\text{\RL{دائیں خروج}}};
\koutleft[j1]{kb}
\draw(j1west)node[left]{\text{\RL{بائیں خروج}}};
\kBOXanc
\kupinAnchors
\kmuxD[u5]{-0*\ksepXa-\spcx+\pax}{-\ksepYa}
\kmuxD[u6]{-1*\ksepXa-\spcx+\pax}{-\ksepYa}
\kmuxD[u7]{-2*\ksepXa-\spcx+\pax}{-\ksepYa}
\kmuxD[u8]{-3*\ksepXa-\spcx+\pax}{-\ksepYa}
\draw(u0pbd)--++(\kpin,0)--++(0,-\ksepYb)coordinate(ll);
\draw(u0pnd)node[ocirc]{};
\draw(u1pbd)--++(\kpin,0)--++(0,-\ksepYb);
\draw(u1pnd)node[ocirc]{};
\draw(u2pbd)--++(\kpin,0)--++(0,-\ksepYb);
\draw(u2pnd)node[ocirc]{};
\draw(u3pbd)--++(\kpin,0)--++(0,-\ksepYb)coordinate(rr);
\draw(u3pnd)node[ocirc]{};
\draw(rr)++(-\ksepXa-1,0)coordinate(lft);
\draw(rr)--(ll)--++(0,-0.5)node[below]{$\overline{\text{بیٹھ}}$};
\draw(u0p2)--(u3p2)--++(-1,0)node[left]{$C$};
\draw(tt)--(ttt)--++(-0.75,0)node[above]{$\overline{\text{\RL{متوازی خارج}}}$};
\draw(u0p1)--(u5dp2);
\draw(u1p1)--(u6dp2);
\draw(u2p1)--(u7dp2);
\draw(u3p1)--(u8dp2);
\draw(u5ap1)--++(0,\ksepYc)coordinate(aa);
\draw(u5ap0)--++(-\ksepXb,0)--++(0,\ksepYd)coordinate(bb);
\draw(u6ap1)--++(0,\ksepYc);
\draw(u6ap0)--++(-\ksepXb,0)--++(0,\ksepYd);
\draw(u7ap1)--++(0,\ksepYc);
\draw(u7ap0)--++(-\ksepXb,0)--++(0,\ksepYd);
\draw(u8ap1)--++(0,\ksepYc)coordinate(aaa);
\draw(u8ap0)--++(-\ksepXb,0)--++(0,\ksepYd)coordinate(bbb);
\draw(aa)--(aaa)--++(0,0.25)coordinate(abab)node[above, xshift=0.75ex]{$a_1$};
\draw(bb)--(bbb)--(bbb |-abab)node[above,xshift=-0.75ex]{$a_0$};
\draw(u5bp0)node[below]{$y$};
\draw(u5bp1)node[below]{$Q_1$};
\draw(u5bp2)node[below]{$z_0$};
\draw(u5bp3)node[below]{$Q_0$};
\draw(u6bp0)node[below]{$Q_0$};
\draw(u6bp1)node[below]{$Q_2$};
\draw(u6bp2)node[below]{$z_2$};
\draw(u6bp3)node[below]{$Q_1$};
\draw(u7bp0)node[below]{$Q_1$};
\draw(u7bp1)node[below]{$Q_3$};
\draw(u7bp2)node[below]{$z_2$};
\draw(u7bp3)node[below]{$Q_2$};
\draw(u8bp0)node[below]{$Q_2$};
\draw(u8bp1)node[below]{$x$};
\draw(u8bp2)node[below]{$z_3$};
\draw(u8bp3)node[below]{$Q_3$};
\end{tikzpicture}
\caption{چار بِٹ ، عالمگیر دائیں انتقال دفتر}
\label{شکل_دفتر_عالمگیر_انتقال_دفتر}
\end{figure}


بائیں انتقال کے دوران مواد \عددی{y} پر\اصطلاح{ سلسلہ وار داخل }\فرہنگ{سلسلہ وار!داخل}\حاشیہب{serial in}\فرہنگ{serial in} ہو کر آخر کار بائیں \اصطلاح{ خروج }\فرہنگ{خروج}\حاشیہب{output}\فرہنگ{output} سے \اصطلاح{ سلسلہ وار خارج }\فرہنگ{سلسلہ وار!خارج}\حاشیہب{serial out}\فرہنگ{serial out} ہو گا، جبکہ دائیں انتقال کے دوران مواد \عددی{x} سے سلسلہ وار داخل ہو کر آخر کار دائیں خروج سے سلسلہ وار خارج ہو گا۔

شکل \حوالہ{شکل_دفتر_عالمگیر_انتقال_دفتر} میں چار یکساں حصے ہیں، جن کی کارکردگی ایک جیسی ہے۔دایاں حصہ پر غور کرتے ہیں۔


پلٹ کار کے ساتھ \اصطلاح{ چار سے ایک منتخب کنندہ }جوڑا گیا ہے ۔ پتہ کے دو بِٹ \عددی{a_0} اور \عددی{a_1} مداخل میں سے ایک چن کر خارجی پن پہنچاتے ہیں۔مداخل کا انتخاب درج ذیل جدول کے تحت ہو گا۔
 \begin{align*}
 \begin{array}{cc|cr}
 a_1& a_0&D_0&\\
 \toprule
 0&0&Q_0&\text{\RL{حال برقرار}}\\
 0&1&z_0&\text{\RL{متوازی داخل}}\\
 1&0&Q_1&\text{\RL{دائیں انتقال}}\\
 1&1&y&\text{\RL{بائیں انتقال}}
 \end{array}
 \end{align*}

 پتہ \عددی{00_2} مواد \عددی{Q_0} منتخب کر کے پلٹ کار کے مداخل پر مہیا کرتا ہے جو اگلے کنارہ ساعت پر پلٹ کار کے خارجی پن پر خارج ہو گا۔یوں دفتر اپنا حال برقرار رکھے گا (اور مواد دائیں یا بائیں منتقل نہیں ہو گا)۔
 
پتہ \عددی{01_2} مواد \عددی{z_0} پلٹ کار کو مہیا کرے گا جو ساعت کے اگلے کنارہ پلٹ کار کے مخارج پر نمودار ہو گا۔چونکہ \عددی{z_0} متوازی مہیا کردہ مواد ہے لہٰذا متوازی مواد دفتر میں چڑھے گا۔

پتہ \عددی{10_2} پلٹ کار کو \عددی{Q_1} مہیا کرے گا۔یوں موجودہ \عددی{Q_1} ساعت کے اگلے کنارے پر بطور \عددی{Q_0}نمودار ہو گا۔یعنی دفتر مواد دائیں منتقل کرے گا۔

پتہ \عددی{11_2} سلسلہ وار مہیا کردہ مواد \عددی{y} منتخب کرے گا جو ساعت کے اگلے کنارہ پر بطور \عددی{Q_0} نمودار ہو گا۔یوں دفتر مواد بائیں منتقل کرے گا۔

مذکورہ بالا تجزیہ باقی تین حصوں پر لاگو کر کے عالم گیر دفتر کی کارکردگی جدول میں پیش کرتے ہیں۔
 \begin{align*}
 \begin{array}{cc|ccccr}
 a_1& a_0&D_3&D_2&D_1&D_0&\\
 \toprule
 0&0&Q_3&Q_2&Q_1&Q_0&\text{\RL{حال برقرار}}\\
 0&1&z_3&z_2&z_1&z_0&\text{\RL{متوازی داخل}}\\
 1&0&x&Q_3&Q_2&Q_1&\text{\RL{دائیں انتقال}}\\
 1&1&Q_2&Q_1&Q_0&y&\text{\RL{بائیں انتقال}}
 \end{array}
 \end{align*}

\ابتدا{مشق}
انٹرنیٹ سے عالمگیر دفتر \عددی{74194} کے معلوماتی صفحات حاصل کریں ۔(ا) یہ کتنے بِٹ کا عالمگیر دفتر ہے۔ (ب) اس کو استعمال کرتے ہوئے سولہ بِٹ عالم گیر دفتر حاصل کریں۔
\انتہا{مشق}

 
 \حصہ{ سلسلہ وار ثنائی جمع کار}
صفحہ \حوالہصفحہ{شکل_ترتیبی_ثنائی_سلسلہ_وار_جمع_کار} پر شکل \حوالہ{شکل_ترتیبی_ثنائی_سلسلہ_وار_جمع_کار} میں سلسلہ وار ثنائی جمع کار پیش ہے جس کو استعمال کرکے شکل\حوالہء{ 7.7 } میں پیش متعدد بِٹ کا سلسلہ وار ثنائی جمع کار حاصل کیا گیا ۔
یہاں دو عدد \عددی{n} بِٹ متوازی دائیں انتقال دفتر (ا اور ب) مستعمل ہیں۔
	
 ساعت کے پہلے کنارے سے قبل (یعنی مجموعہ لینے سے قبل)، دفتر۔ا میں ثنائی عدد \عددی{y}، دفتر-ب میں ثنائی عدد \عددی{z} متوازی منتقل کئے جاتے ہیں اور زبردستی پست اشارہ لمحاتی پست کر کے ڈی پلٹ کار پست کیا جاتا ہے ( تا کہ مکمل جمع کار کا داخلی حاصل \عددی{0} ہو)۔شکل میں متوازی چڑھائی نہیں دکھائی گئی تا کہ اصل موضوع پر توجہ رہے۔

مکمل جمع کار ان دو ثنائی اعداد کے کم تر رتبی بِٹ اور داخلی حاصل \عددی{0} جمع کر کے جمع \عددی{s_0} اور خارجی حاصل \عددی{c_1} خارج کرتا ہے۔ساعت کے پہلے کنارے پر \عددی{c_1} کو ڈی پلٹ کار محفوظ کر کے اگلے ثنائی بِٹ کی جمع کے دوران مکمل جمع کار کو بطور داخلی حاصل فراہم کرتا ہے جبکہ دفتر-ا اور دفتر-ب اگلے ثنائی بِٹ فراہم کرتے ہیں۔جمع \عددی{s_0} شکل میں دفتر-ا کو سلسلہ وار مداخل کے طور مہیا کیا گیا ہے۔یوں جیسے جیسے دفتر ثنائی عدد \عددی{y} دائیں جانب خارج کرتا ہے ویسے ویسے اس کی جگہ دو اعداد کا مجموعہ جگہ لیتا ہے۔ساعت کے \عددی{n} کنارے گزرنے کے بعد دو ثنائی اعداد کا مجموعہ دفتر-ا میں محفوظ ہو گا جہاں سے اسے متوازی پڑھا جا سکتا ہے جبکہ مجموعے کا آخری حاصل مکمل جمع کار کے مخارج \عددی{c} سے پڑھا جا سکتا ہے۔


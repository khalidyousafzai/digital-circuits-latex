 \باب{بوولین الجبرا}
بوولین الجبرا انگلستان کے   ریاضی  دان جارج بوولی کے نام سے جانا جاتا  ہے، جنہوں نے اس الجبرا کو دریافت کیا۔بوولین الجبرا ذہنی سوچ یعنی منطق کو الجبرائی روپ  میں لکھنے کی صلاحیت رکھتی ہے۔اس لئے  حیرانی کی بات نہیں کہ  کمپیوٹر اسی کو استعمال کرتا ہے۔

\حصہ{بوولین الجبرا کے بنیادی تصورات} \شناخت{حصہ_بوولین_الجبرا_بنیادی_تصورات}
عام الجبرا میں متغیرات استعمال کرتے  ہوئے  تصور کیا جاتا ہے کہ ان کی  قیمت کچھ بھی  ہو سکتی ہے۔مثلاً،  تفاعل   \عددی{z=f(x,y)}،  جہاں   \عددی{x} اور  \عددی{y}  آزاد متغیرات   جبکہ \عددی{z}  تابع متغیر ہے،  میں متغیرات کی چند   ممکنہ قیمتیں درج ذیل     ہیں۔
\begin{center}
\begin{otherlanguage}{english}
\begin{tabular}{CC|C}
x&y&z\\
\toprule
0&0&0\\
1&2&5\\
2&1&4\\
3&2&7\\
2&2&6\\
3&1&5
\end{tabular}
\end{otherlanguage}
\end{center}
اس تفاعل جس  کو ایک نا مکمل جدول کے روپ میں  پیش کیا گیا ہے  کا الجبرائی روپ   درج ذیل ہے۔
 \begin{align*}
 z=x+2y
 \end{align*}
اس کے برعکس،  بوولین الجبرا میں متغیرات کی صرف دو ممکنہ قیمتیں ہیں۔ ان دو قیمتوں کو عموماً   \عددی{0} (صفر)    اور   \عددی{1} (ایک)   سے ظاہر کیا جاتا ہے۔بوولین تفاعل کی چند مثالوں پر غور کرتے ہیں۔

\جزوحصہ{منطقی ضرب}
تصور کریں  \عددی{X}  اور \عددی{Y}  آزاد بوولین متغیرات ہیں،  جبکہ \عددی{Z}  ان کا تابع بوولین متغیر \عددی{Z=f(X,Y)} ہے۔ چونکہ \عددی{X}  بوولین متغیر ہے،  لہٰذا اس کی ممکنہ قیمتیں صرف \عددی{0} اور \عددی{1} ہیں۔اسی طرح \عددی{Y}  بھی بوولین متغیر ہے،  لہٰذا اس کی قیمت  بھی صرف  \عددی{0} اور \عددی{1} ہو سکتی ہے۔تابع متغیر \عددی{Z}بھی بوولین متغیر ہے۔اس طرح اگرچہ اس کی قیمت \عددی{X} اور \عددی{Y} کی تابع ہے،   اس کے باوجود \عددی{Z} کی قیمت  صرف  \عددی{0} یا \عددی{1}    ہی ہو سکتا ہے۔ متغیرات \عددی{X} اور \عددی{Y}  درج ذیل چار ممکنہ ترتیب میں پائے جا سکتے ہیں۔
\begin{center}
\begin{otherlanguage}{english}
\begin{tabular}{CC}
X&Y\\
\toprule
0&0\\
0&1\\
1&0\\
1&1
\end{tabular}
\end{otherlanguage}
\end{center}
ان چار ممکنہ صورتوں میں \عددی{Z} کی   قیمت \عددی{0} یا \عددی{1}  ہوگی۔

آئیں، جدول \حوالہ{جدول_بوولین_جمع} میں پیش کیے گئے منطقی تفاعل پر غور کرتے ہیں جس کی تمام ممکنہ قیمتیں اس جدول میں دی گئی ہیں۔
\begin{table}
\centering
\begin{otherlanguage}{english}
\begin{tabular}{CC|C}
X&Y&Z\\
\toprule
0&0&0\\
0&1&0\\
1&0&0\\
1&1&1
\end{tabular}
\end{otherlanguage}
\caption{دو   متغیر منطقی ضرب}
\label{جدول_بوولین_جمع}
\end{table}
اس مثال میں تابع متغیر \عددی{Z}   کی قیمت صرف   اس وقت   \عددی{1}  ہے جب \عددی{X}   اور \عددی{Y}  دونوں کی قیمت \عددی{1}   ہے۔یہی قیمتیں \عددی{X}  اور \عددی{Y}  کی سادہ ضرب \عددی{X\cdot Y}  سے بھی حاصل ہوتی ہیں (ذیل دیکھیں)۔
\begin{align*}
0\cdot 0&=0\\
0\cdot 1&=0\\
1\cdot 0&=0\\
1\cdot 1&=1
\end{align*}
اسی کی بنا پر جدول \حوالہ{جدول_بوولین_جمع} میں پیش     تفاعل (اور  عمل)  کو  بوولین ضرب یا منطقی ضرب   کہتے ہیں۔ بوولین ضرب کو آزاد متغیرات کے درمیان نقطہ    \قول{\عددی{\cdot}} سے یا    آزاد متغیرات کو قریب قریب  لکھنے سے ظاہر کیا جاتا ہے۔یوں بوولین ضرب   درج ذیل لکھا  جائے گا۔
\begin{gather}
\begin{aligned}
Z&=X\cdot Y\\
Z&=XY&\text{\RL{\small{(بوولین ضرب)}}}
\end{aligned}
\end{gather}
منطقی ضرب  کے  تصور کو وسعت دے کر    متعدد آزاد متغیرات کے لئے بیان کیا جا سکتا ہے۔ منطقی ضرب کی عمومی تعریف پیش کرتے ہیں۔

\ابتدا{تعریف}
منطقی ضرب  اس صورت \عددی{1} دیگا جب تمام آزاد متغیرات کی قیمت \عددی{1} ہو۔
\انتہا{تعریف}


تین آزاد متغیرات کے منطقی ضرب تفاعل \عددی{Z=ABC}  کو جدول \حوالہ{جدول_بوولین_تین_متغیر_بوولین_ضرب} میں پیش کیا گیا ہے۔
\begin{table}
\centering
\begin{otherlanguage}{english}
\begin{tabular}{CCC|C}
A&B&C&Z\\
\toprule
0&0&0&0\\
0&0&1&0\\
0&1&0&0\\
0&1&1&0\\
1&0&0&0\\
1&0&1&0\\
1&1&0&0\\
1&1&1&1
\end{tabular}
\end{otherlanguage}
\caption{تین متغیر بوولین ضرب}
\label{جدول_بوولین_تین_متغیر_بوولین_ضرب}
\end{table}

\جزوحصہ{منطقی جمع}
دو آزاد متغیرات کے بوولین تفاعل کی ایک اور مثال لیتے ہیں جس کو جدول \حوالہ{جدول_بوولین_جمع_منطقی}  میں پیش کیا گیا ہے۔
\begin{table}
\centering
\begin{minipage}[t]{0.45\textwidth}
\centering
\begin{otherlanguage}{english}
\begin{tabular}{CC|C}
X&Y&Z\\
\toprule
0&0&0\\
0&1&1\\
1&0&1\\
1&1&1
\end{tabular}
\end{otherlanguage}
\caption{دو متغیر منطقی جمع}
\label{جدول_بوولین_جمع_منطقی}
\end{minipage}\hfill
\begin{minipage}[t]{0.45\textwidth}
\centering
\begin{otherlanguage}{english}
\begin{tabular}{CC|C}
X&Y&S\\
\toprule
0&0&0\\
0&1&1\\
1&0&1\\
1&1&2
\end{tabular}
\end{otherlanguage}
\caption{دو ثنائی اعداد کا سادہ مجموعہ}
\label{جدول_بوولین_جمع_سادہ}
\end{minipage}
\end{table}
اب \عددی{Z}   اس صورت   \عددی{1} کے برابر ہے جب \عددی{X}  یا \عددی{Y} یا دونوں کی قیمت  \عددی{1}   ہو۔اس بوولین عمل کو  بوولین جمع یا منطقی جمع  کہتے ہیں۔

آزاد متغیرات  \عددی{X}  اور \عددی{Y} کا  (روز مرہ) سادہ    الجبرائی مجموعہ  \عددی{S=X+Y} جدول \حوالہ{جدول_بوولین_جمع_سادہ} میں پیش کیا گیا ہے۔

جدول   \حوالہ{جدول_بوولین_جمع_منطقی}  اور جدول  \حوالہ{جدول_بوولین_جمع_سادہ}  کے اولین تین نتائج ایک جیسے ہیں۔اس مشابہت کی  بنا   جدول \حوالہ{جدول_بوولین_جمع_منطقی}  میں دیے  گئے بوولین تفاعل کو بوولین جمع  یا منطقی جمع  کہتے ہیں اور اس بوولین تفاعل کو جمع کے نشان \قول{\عددی{+}} سے ہی ظاہر کیا جاتا ہے۔یوں جدول   \حوالہ{جدول_بوولین_جمع_منطقی}میں پیش بوولین جمع  تفاعل درج ذیل لکھا جائے گا۔
\begin{align}
Z&=X+Y&\text{\RL{\small{(بوولین جمع)}}}
\end{align}
یہ  بوولین تفاعل کی مساوات ہے جس کو عام الجبرائی جمع ہرگز نہ سمجھا جائے۔بالخصوص،   بوولین جمع کرتے وقت یاد رہے کہ \عددی{1+1=1}ہے۔

بوولین جمع کے  تصور کو وسعت دے کر   متعدد  آزاد متغیرات کے لئے  بیان کیا جا سکتا ہے۔ بوولین جمع کی عمومی تعریف درج ذیل ہے۔

\ابتدا{تعریف}
منطقی جمع   اس صورت \عددی{1} دیگا  جب  آزاد متغیرات میں کم  سے کم ایک متغیر کی قیمت \عددی{1} ہو۔
\انتہا{تعریف}

تین متغیر منطقی جمع تفاعل \عددی{Z=A+B+C}   جدول  \حوالہ{جدول_بوولین_تین_متغیر_جمع} میں پیش کیا گیا ہے۔
\begin{table}
\centering
\begin{minipage}[b]{0.45\textwidth}
\centering
\begin{otherlanguage}{english}
\begin{tabular}{CCC|C}
A&B&C&Z\\
\toprule
0&0&0&0\\
0&0&1&1\\
0&1&0&1\\
0&1&1&1\\
1&0&0&1\\
1&0&1&1\\
1&1&0&1\\
1&1&1&1
\end{tabular}
\end{otherlanguage}
\caption{تین متغیر منطقی جمع}
\label{جدول_بوولین_تین_متغیر_جمع}
\end{minipage}\hfill
\begin{minipage}[b]{0.45\textwidth}
\centering
\begin{otherlanguage}{english}
\begin{tabular}{C|C}
X&Z\\
\toprule
0&1\\
1&0
\end{tabular}
\end{otherlanguage}
\caption{منطقی نفی یا متمم}
\label{جدول_بوولین_نفی}
\end{minipage}
\end{table}
یاد رہے کہ  تین آزاد متغیرات کے منطقی جمع  کا الجبرائی جمع کے ساتھ کوئی تعلق نہیں۔یہاں جمع کی علامت  بوولین جمع کو ظاہر کرتی ہے لہٰذا یہاں    \عددی{1+1+1=1} ہو گا۔


\جزوحصہ{منطقی نفی}
بوولین تفاعل \عددی{ Z=f(X)} کی  تیسری مثال  لیتے ہیں جہاں  آزاد متغیر \عددی{X}  اور تابع متغیر \عددی{Z} کا تعلق جدول \حوالہ{جدول_بوولین_نفی} میں پیش کیا گیا ہے۔

  اس تفاعل کو بوولین نفی  کہتے ہیں۔آپ دیکھ سکتے ہیں کہ درحقیقت، تابع متغیر \عددی{Z}،  آزاد متغیر کا متمم ہے۔یوں بوولین نفی درج ذیل لکھا جا سکتا ہے۔

\begin{align}
Z&=\overline{X}&\text{\RL{\small{(بوولین نفی یا متمم)}}}
\end{align}
بوولین نفی     صرف   ایک آزاد متغیر کے لئے  بیان کیا جا سکتا ہے، اور اس کی تعریف درج ذیل ہے۔

\ابتدا{تعریف}
بوولین نفی آزاد متغیر کا متمم دیتا ہے۔
\انتہا{تعریف}

\جزوحصہ{منطقی بلا شرکت جمع}
دو آزاد متغیرات کا ایسا  بوولین تفاعل جدول \حوالہ{جدول_بوولین_دو_بلا_شرکت}  میں دکھایا گیا ہے، جس کا   تابع متغیر اس صورت  \عددی{1} ہے جب صرف ایک  آزاد متغیر \عددی{1} ہو۔ یہ  دو متغیر بوولین بلا شرکت جمع ہے۔
\begin{table}
\centering
\begin{minipage}[b]{0.45\textwidth}
\centering
\begin{otherlanguage}{english}
\begin{tabular}{CC|C}
A&B&Z\\
\toprule
0&0&0\\
0&1&1\\
1&0&1\\
1&1&0
\end{tabular}
\end{otherlanguage}
\caption{دو متغیر منطقی  بلا شرکت جمع}
\label{جدول_بوولین_دو_بلا_شرکت}
\end{minipage}\hfill
\begin{minipage}[b]{0.45\textwidth}
\centering
\begin{otherlanguage}{english}
\begin{tabular}{CCC|C}
A&B&C&Z\\
\toprule
0&0&0&0\\
0&0&1&1\\
0&1&0&1\\
0&1&1&0\\
1&0&0&1\\
1&0&1&0\\
1&1&0&0\\
1&1&1&1
\end{tabular}
\end{otherlanguage}
\caption{تین متغیر بوولین بلا شرکت جمع}
\label{جدول_بوولین_تین_متغیر_بلا_شرکت}
\end{minipage}
\end{table}
اس تصور کو  متعدد آزاد متغیرات تک وسعت دے کر بیان کرتے ہیں۔

\ابتدا{تعریف}
طاق تعداد کے آزاد متغیرات \عددی{1} ہو نے کی صورت میں بوولین بلا شرکت   کا تابع متغیر \عددی{1} ہو گا۔
\انتہا{تعریف}

تین آزاد متغیر  بلا شرکت جمع تفاعل کو جدول  \حوالہ{جدول_بوولین_تین_متغیر_بلا_شرکت} میں پیش کیا گیا ہے۔



دو اور تین آزاد متغیر بوولین بلا شرکت  کی مساوات درج ذیل ہوں گی۔
\begin{gather}
\begin{aligned}
Z&=A\oplus B&\text{\RL{\small{(دو آزاد متغیر بلا شرکت جمع)}}}\\
Z&=A\oplus B\oplus C&\text{\RL{\small{(تین آزاد متغیر بلا شرکت جمع)}}}
\end{aligned}
\end{gather}

%???KKK
%==================================================================================
\جزوحصہ{منطقی ضد  بلا شرکت  جمع}
 بوولین بلا شرکت جمع تفاعل کا نفی  (یعنی متمم)  لینے سے بوولین  ضد   بلا شرکت  جمع    حاصل ہو گا،  جو دو اور تین آزاد متغیرات کے لئے درج ذیل لکھا جاتا ہے۔
\begin{gather}
\begin{aligned}
Z&=\overline{A\oplus B}\\
Z&=\overline{A\oplus B\oplus C}&\text{\RL{\small{(تین متغیر منطقی ضد بلا شرکت جمع)}}}
\end{aligned}
\end{gather}
 جدول \حوالہ{جدول_بوولین_دو_بلا_شرکت} اور جدول \حوالہ{جدول_بوولین_تین_متغیر_بلا_شرکت} میں تابع متغیر نفی کرنے سے   بالترتیب دو اور تین  بوولین  ضد بلا شرکت  تفاعل حاصل ہوں گے جنہیں جدول \حوالہ{جدول_بوولین_دو_متمم_بلا_شرکت} اور جدول \حوالہ{جدول_بوولین_تین_متغیرمتمم_بلا_شرکت} میں پیش کیا گیا ہے۔
\begin{table}
\centering
\begin{minipage}[b]{0.45\textwidth}
\centering
\begin{otherlanguage}{english}
\begin{tabular}{CC|C}
A&B&Z\\
\toprule
0&0&0\\
0&1&1\\
1&0&1\\
1&1&0
\end{tabular}
\end{otherlanguage}
\caption{دو متغیر منطقی ضد   بلا شرکت جمع}
\label{جدول_بوولین_دو_متمم_بلا_شرکت}
\end{minipage}\hfill
\begin{minipage}[b]{0.45\textwidth}
\centering
\begin{otherlanguage}{english}
\begin{tabular}{CCC|C}
A&B&C&Z\\
\toprule
0&0&0&0\\
0&0&1&1\\
0&1&0&1\\
0&1&1&0\\
1&0&0&1\\
1&0&1&0\\
1&1&0&0\\
1&1&1&1
\end{tabular}
\end{otherlanguage}
\caption{تین متغیر بوولین ضد  بلا شرکت جمع}
\label{جدول_بوولین_تین_متغیرمتمم_بلا_شرکت}
\end{minipage}
\end{table}
\حصہ{برقی تاروں میں جوڑ کی وضاحت}
  درج ذیل  شکل  پر غور کریں جس  میں دو برقی تاروں کے بیچ  جوڑ کی وضاحت کی گئی ہے۔
  
  جہاں ایک تار دوسری تار کے   اوپر سے گزرتی ہو  اور  دونوں آپس میں  جڑی ہوں،   وہاں  جوڑ کے مقام پر نقطے  کا نشان لگایا جاتا ہے۔ایسی صورت میں انہیں ایک  تار تصور کیا جائے۔
  
   جہاں تاریں  آپس میں جڑی نہ ہوں وہاں انہیں بغیر نقطے کے نشان سے   ایک دوسری کے اوپر سے گزرتا دکھایا جاتا ہے۔ نقطہ کے نشان کی غیر موجودگی میں ان تاروں کو دو علیحدہ اور بلا جوڑ  تاریں سمجھا جائے۔

 تیسری صورت بھی شکل میں دکھائی گئی ہے جہاں غلط فہمی کا  امکان نہیں پایا جاتا۔اس میں ایک تار کا سر دوسری تار پر ختم ہو تا ہے۔ ایسی صورت میں انہیں ایک  تار تصور کیا جائے  (یعنی یہ دونوں آپس میں جڑی ہیں) ۔
\begin{center}
\begin{tikzpicture}
\draw(-1,0)--(1.5,0);
\draw(0,0)--++(0,-0.75)--++(1,0);
\draw(0,-1)node[above left]{\text{\RL{جوڑ}}};
\end{tikzpicture}\quad\quad
\begin{tikzpicture}
\draw(-1,0)--(1.5,0);
\draw(0,1)--(0,-1);
\draw(0,-1)node[above right]{\text{\RL{بلا جوڑ}}};
\end{tikzpicture}\quad \quad 
\begin{tikzpicture}
\draw(-1,0)--(1.5,0);
\draw(0,1)--(0,0)node[circ]{}--(0,-1);
\draw(0,-1)node[above right]{\text{\RL{جوڑ}}};
\end{tikzpicture}
\end{center}

\حصہ{عددی گیٹ}
 بوولین الجبرا کے تین اہم ترین تفاعل پر حصہ \حوالہ{حصہ_بوولین_الجبرا_بنیادی_تصورات} میں  غور  کیا گیا۔یہ  تفاعلات  عددی  برقیات  میں کلیدی کردار ادا کرتے ہیں، جہاں انہیں  عددی ادوار کی مدد سے جامہ عمل  پہنایا  جاتا ہے۔یہ مخصوص عددی ادوار عددی گیٹ  کہلاتے ہیں۔
 
\جزوحصہ{ضرب گیٹ}
	منطقی (بوولین)  ضرب   تفاعل کو   ضرب گیٹ سے حاصل کیا جاتا ہے،  جو شکل 3.2 میں دکھایا گیا ہے۔ آزاد متغیرات،  \عددی{X}  اور \عددی{Y}،   ضرب گیٹ کی  بائیں  جبکہ تابع متغیر دائیں جانب  ہے۔  آزاد متغیرات کو مداخل  جبکہ تابع متغیرات کو مخارج کہتے ہیں۔دو متغیر  ضرب گیٹ کے دو مداخل اور ایک مخارج ہوں  گا۔

شکل 3.3 میں ضرب گیٹ کی کارکردگی  ترسیم  کی گئی ہے۔آپ دیکھ سکتے ہیں کہ مخارج صرف اور صرف اُس صورت بلند  ہوتا ہے جب ضرب گیٹ کے تمام مداخل بلند ہوں۔اس شکل میں  مداخل کو کسی خاص ترتیب سے  تبدیل نہیں  کیا گیا ہے۔

ضرب گیٹ کو شکل 3.4 میں بطور  عددی گیٹ دکھایا گیا ہے جہاں  ایک داخلی پینا  کو قابو پینا  کا نام دیا گیا ہے جبکہ دوسرا مداخل کہلاتا ہے۔ضرب گیٹ کے جدول سے واضح ہے کہ جب تک قابو پینا  \عددی{0} ہو،   خارجی پینا   \عددی{0}  رہتا ہے۔اس صورت میں   مداخل  پر موجود مواد، خارجی پینا تک نہیں پہنچ سکتا،  یعنی   اس   پر \عددی{0} یا \عددی{1} کرنے کا مخارج پر کوئی اثر نہیں ہوتا؛  ہم کہتے  ہیں قابو پینا نے ضرب گیٹ کو معذور   کر دیا ۔اس کے برعکس اگر قابو پینا  \عددی{1} ہو تب خارجی پینا پر وہی کچھ ہوگا جو مداخل پر ہے؛ ہم کہتے  ہیں  ضرب گیٹ مجاز   کر دیا گیا ہے۔قابو پینا پر ایک یا صفر دینے سے داخلی اشارہ  (مواد) کو خارجی پینا تک پہنچنا ممکن یا ناممکن بنایا جا سکتا ہے۔یوں یہ ایک دروازے کی طرح کام کرتا ہے۔اسی  کی بنا  سے یہ  گیٹ  کہلاتا ہے۔قابو پینا کو معذور اور مجاز بنانے والا پینا بھی کہتے ہیں۔شکل 3.5 میں اس  گیٹ کی  کارکردگی دکھائی گئی ہے۔آپ دیکھ سکتے ہیں کہ جب تک گیٹ کو معذور رکھا جائے اتنی دیر یہ مداخل کو  روکھے رکھتا ہے اور جیسے ہی  گیٹ مجاز کیا جاتا ہے   مداخل پر موجود اشارہ  مخارج پر خارج ہوتا ہے۔






3.3.2 جمع گیٹ
	منطقی جمع یعنی بوولین جمع کے تفاعل کو جمع گیٹ 13 سے حاصل کیا جاتا ہے جسے شکل 3.6 میں دکھلایا گیا ہے۔ 

	جمع گیٹ میں اگر ایک پینا کو قابو کی پینا سمجھا جائے تو قابو پینا پر صفر  دینے سے داخلی مواد کا خارجی پینا تک پہنچنا ممکن بنایا جاتا ہے جبکہ اس پر ایک دینے سے یہ ناممکن بنایا جاتا ہے۔
	جمع گیٹ کی کارکردگی شکل 3.7 میں گراف کے شکل میں دکھائی گئی ہے۔آپ دیکھ سکتے ہیں کہ جمع گیٹ کا مخارج اُس وقت بلند ہوتا ہے جب جمع گیٹ کے مداخل میں کم از کم ایک مداخل بلند ہو۔

3.3.3 نفی گیٹ
	نفی کے تفاعل کو نفی گیٹ سے حاصل کیا جاتا ہے جس کی علامت شکل 3.8 میں دکھائی گئی ہے۔

	نفی تفاعل صرف ایک ہی آزاد اور ایک ہی تابع متغیرہ کے لئے ممکن ہے۔اسی وجہ سے نفی گیٹ کا ایک ہی مداخل اور ایک ہی مخارج ہوتا ہے جبکہ ضرب گیٹ اور جمع گیٹ دو یا دو سے زیادہ مداخل کے بھی ہو سکتے ہیں۔شکل 3.10 میں تین مداخل کے ضرب اور جمع گیٹ دکھائے گئے ہیں۔

	نفی گیٹ کی کارکردگی شکل 3.9 میں گراف کے شکل میں دکھائی گئی ہے۔آپ دیکھ سکتے ہیں کہ نفی گیٹ کا مخارج اس کے مداخل کے اُلٹ رہتا ہے۔


	ضرب گیٹ کی مخارج اس وقتکے برابر ہوتی ہے جب اس کے تمام مداخلہوں۔جبکہ جمع گیٹ کی مخارج اس وقتہوتی ہے جب اس کے مداخل میں سے کوئی بھی مداخلہو۔

	شکل 3.11 کے حصہ (ا) میں دو عدد ضرب گیٹ جوڑے گئے ہیں۔ساتھ ہی اس دور کا بوولین جدول دیا گیا ہے۔آپ دیکھ سکتے ہیں کہ یہ دور تین داخلی ضرب گیٹ کا کردار ادا کر رہا ہے۔ یوں دو داخلی ضرب گیٹوں کی مدد سے زیادہ مداخل کا ضرب گیٹ حاصل کیا جا سکتا ہے۔اسی طرح شکل کے حصہ (ب) میں تین داخلی جمع گیٹ کا حصول دکھایا گیا ہے۔ 
	شکل 3.12 اور شکل 3.13 میں ان گیٹوں پر مبنی ادوار کے چند مثالیں اور ان کو حل کرنا دکھایا گیا ہے۔


	شکل 3.12 میں سب سے اوپر دو جمع گیٹوں کی خارجی پیناوں  کو اس کے سامنے ایک جمع گیٹ کی داخلی پیناوں  سے لکیروں (تاروں) کے ذریعہ جوڑا گیا ہے۔اس طرح کی لکیریں ایک خارجی پینا سے شروع اور ایک یا ایک سے زیادہ داخلی پیناوں  پر ختم ہوتی ہیں۔یوں جڑے تار خارجی پینا پر موجود سگنل یعنییاکو سامنے گیٹ کے داخلی پینا یا پیناوں  تک پہنچاتی ہیں۔اس طرح سب سے اوپر والی تار (لکیر) کا مطلب یہ ہوا کہ بائیں جانب جمع گیٹ کی مخارج یعنیدائیں جانب جمع گیٹ کی مداخل بن گئی ہے۔

	اس شکل میں اوپر سے دوسرے گیٹ یعنی جمع گیٹ کی مخارج یعنیدائیں جانب ضرب گیٹ اور نفی گیٹ دونوں کی مداخل بنی ہے۔
3.3.4 نفی۔جمع گیٹ اور نفی۔ضرب گیٹ
	شکل 3.14 (ا) میں تین داخلی نفی۔جمع گیٹ اور اس کا بوولین جدول دکھایا گیا ہے۔شکل (ب) میں تین داخلی جمع گیٹ کے ساتھ نفی گیٹ جوڑا گیا ہے۔ان جڑواں گیٹوں کے دور کا بوولین جدول بھی یہی حاصل ہوتا ہے گویا شکل کے دونوں حصے ایک ہی تفاعل کو ظاہر کرتے ہیں۔اسی مشابہت سے نفی اور جمع گیٹوں کے نام جوڑ کر اس گیٹ کا نام نفی۔جمع گیٹ 14 رکھا گیا ہے۔



	اسی طرح شکل 3.15 میں تین داخلی نفی۔ضرب گیٹ دکھایا گیا ہے جسے نفی اور ضرب کے لفظ جوڑ کر نفی۔ضرب گیٹ 15 کا نام دیا گیا ہے۔
	بالکل ضرب اور جمع گیٹوں کی طرح یہ دو قسم کے گیٹ بھی دو، تین یا ان سے زیادہ مداخل والے ہو سکتے ہیں۔


	کسی بھی نفی۔جمع گیٹ کی مخارج صرف اُسی صورتہوتا ہے جب اس کے تمام مداخلہوں جبکہ کسی بھی نفی۔ضرب گیٹ کی مخارج اُس وقت تک  رہتا ہے جب تک اس کے تمام مداخلنہ ہوں۔

	شکل 3.16 میں باری باری  نفی۔جمع گیٹ اور نفی۔ضرب گیٹ کی مدد سے نفی گیٹ کا عمل حاصل کرنا دکھایا گیا ہے۔یوں نفی گیٹ کی جگہ نفی۔جمع گیٹ استعمال کیا جا سکتا ہے یا پھر اس کی جگہ نفی۔ضرب گیٹ استعمال کیا جا سکتا ہے۔
	اسی طرح شکل 3.17 میں نفی۔جمع گیٹ کی مدد سے جمع گیٹ اور ضرب گیٹ کا عمل حاصل کیا گیا ہے جبکہ شکل 3.18 میں نفی۔ضرب گیٹ استعمال کرتے ہوئے جمع گیٹ اور ضرب گیٹ کا عمل حاصل کیا گیا ہے۔



	اس شکل میں ضرب گیٹ بناتے وقت بائیں جانب سب سے نیچے نفی۔جمع گیٹ کے دونوں مداخل آپس میں جوڑ کر انہیںمتغیرہ سے منسلک کیا گیا ہے۔

	اس حصہ کے شروع میں دیکھا گیا کہ جمع، ضرب اور نفی گیٹوں کی مدد سے نفی۔جمع گیٹ اور نفی۔ضرب گیٹ حاصل کئے جا سکتے ہیں جبکہ اس حصہ کے آخر میں نفی۔جمع گیٹوں اور نفی۔ضرب گیٹوں کی مدد سے نفی گیٹ، جمع گیٹ اور ضرب گیٹ حاصل کرنا دکھلایا گیا۔

3.3.5 بلا شرکت جمع گیٹ اور نفی بلا شرکت جمع گیٹ
	بلا شرکت جمع تفاعل کو بلا شرکت جمع گیٹ 16 سے حاصل کیا جاتا ہے جس کی علامت شکل 3.19 (ا) میں دکھائی گئی ہے۔اسی طرح بلا شرکت نار تفاعل کو نفی بلا شرکت جمع گیٹ 17 کی مدد سے حاصل کیا جاتا ہے جس کی علامت شکل (ب) میں دکھائی گئی ہے۔بلا شرکت جمع گیٹ کی مخارج کے ساتھ نفی گیٹ منسلک کرنے سے بلا شرکت نفی۔جمع گیٹ حاصل کیا جا سکتا ہے۔بلا شرکت گیٹ کی کارکردگی گراف کے شکل میں شکل 3.20 میں دکھائی گئی ہے۔

	
	تین مداخل والے بلا شرکت جمع گیٹ کا مخارج حاصل کرتے وقت اس کے کسی دو مداخل کا بلا شرکت جمع حاصل کریں اور حاصل جواب کا تیسرے مداخل کے ساتھ بلا شرکت جمع حاصل کریں۔یہی ان تین مداخل کا بلا شرکت جمع ہے۔مساوات 3.19 میں تین مداخل والے بلا شرکت جمع گیٹ کا بوولین جدول دکھایا گیا ہے۔جیسے آپ اس جدول سے دیکھ سکتے ہیں، کسی بھی بلا شرکت جمع گیٹ کا مخارج اُس صورت بلند ہوتا ہے جب اس کے بلند مداخل کی تعداد طاق ہو۔

 
(3.19)



	طلبہ سے گزارش کی جاتی ہے کہ وہ یہاں رُک کر ان اعمال کو اچھی طرح سمجھ لیں۔


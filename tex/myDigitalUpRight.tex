\tikzset{flipflop SR/.style={flipflop,
flipflop def={width=1.0,pin spacing=0.4,font=\normalsize,t1={\,$S$},t3={\,$ R$},t6={$Q$\,} ,t4={$\overline{Q}$\,}}}}

\tikzset{flipflop D/.style={flipflop, 
flipflop def={width=1.0,pin spacing=0.4,font=\normalsize,t1={\,$D$},t2={\,$C$},c2=1,n4=1,
t4={$\overline{Q}$\,} ,t6={$Q$\,} ,}}}

\tikzset{flipflop adder/.style={flipflop, 
flipflop def={width=1.0,pin spacing=0.4,font=\normalsize,t1={\,$y$},t2={\,$z$},td={\normalsize $c_{\text{خ}}$} ,t6={$s$\,} , 
tu={\normalsize$c_{\text{د}}$}}}}

%=============================
%=============================
%   \kfulladder[u1]{xshift}{yshift}     %called with ref desig, used to provide pin anchor naming (such as u1S, u1Q, u1C)
							% and location (xshift,yshift) where the comp is to be placed
							% default ref des is "u0"
\newcommand{\kfulladder}[3][u0]{
\def\kshiftX{#2}
\def\kshiftY{#3}
\pgfmathsetmacro{\kpin}{0.50}
\pgfmathsetmacro{\kpsep}{0.60}			%pin to pin distance
\pgfmathsetmacro{\kulV}{0.50}			%edge clearance
\pgfmathsetmacro{\kdimY}{2*\kulV+0*\kpsep}
\pgfmathsetmacro{\kdimX}{2*\kulV+2*\kpsep}		%two spaces between 3 pins

\def\leftPins{1/y/y,2/z/z}
\def\rightPins{2/s/s}
\def\cdelX{0}
\def\cdelY{0}
\def\upClk{0/c_i/ci}
\def\downClk{0/c_o/co}
\draw[thick](\kshiftX,\kshiftY) rectangle ++(\kdimX,\kdimY);
\foreach \yLoc/\lbl/\anchor in \leftPins {\draw[thin](\kshiftX+\kulV+\yLoc*\kpsep,\kshiftY+\kdimY)node[below]{$\lbl$}--++(0,\kpin,0)coordinate(#1\anchor);}
\foreach \yLoc/\lbl/\anchor in \rightPins{\draw[thin](\kshiftX+\kulV+\yLoc*\kpsep,\kshiftY)node[above]{$\lbl$}--++(0,-\kpin,0)coordinate(#1\anchor);}
\foreach \yLoc/\lbl/\anchor in \upClk{\draw[thin](\kshiftX+\kdimX,\kshiftY+\kulV+\yLoc*\kpsep)coordinate(cc)--++(\kpin,0)coordinate(#1\anchor)  (cc)++(0,-\cdelY)--++(-\cdelX,\cdelY)node[left]{$\lbl$}--++(\cdelX,\cdelY);}
\foreach \yLoc/\lbl/\anchor in \downClk{\draw[thin](\kshiftX,\kshiftY+\kulV+\yLoc*\kpsep)coordinate(cc)--++(-\kpin,0)coordinate(#1\anchor)  (cc)++(0,-\cdelY)--++(\cdelX,\cdelY)node[right]{$\lbl$}--++(-\cdelX,\cdelY);}
}
%==================================
%%%%%%%%%%%%%%%%%%%%%%%%%%%%%
%%%%%%%%%%%%%%%%%%%%%%%%%%%%%%%
%=============================
%for FlipFlops
%gives pin tip's locations as (pax,pay), pins are pa,pb,...,pf, pup,pdown corresponding to p1,..,p6,pu,pd
%needed when anchoring at a certain pin
\newcommand{\kpinAnchors}{
\def\pax{-\kpin}
\def\pay{\kulV+2*\kpsep}
\def\pbx{-\kpin}
\def\pby{\kulV+1*\kpsep}
\def\pcx{-\kpin}
\def\pcy{\kulV+0*\kpsep}
\def\pdx{\kdimX+\kpin}
\def\pdy{\kulV+0*\kpsep}
\def\pex{\kdimX+\kpin}
\def\pey{\kulV+1*\kpsep}
\def\pfx{\kdimX+\kpin}
\def\pfy{\kulV+2*\kpsep}
\def\pupx{\kdimX/2}
\def\pupy{\kpin+\kdimY}
\def\pdownx{\kdimX/2}
\def\pdowny{-\kpin}
}
%====================================
%puts labels of pins
%\kFFlabelPins[u1]{1/D,2/C,4/{\overline{Q}},6/Q}
\newcommand{\kFFlabelPins}[2][u0]{
\def\kref{#1}
\def\p{p}			%pin tip
\def\pb{pb}			%pin base
\def\pl{pl}			%pin label
\def\pn{pn}			%pin 'ocirc'
\def\px{x}			%pinAnchorX
\def\py{y}			%pinAnchorY
\foreach \n/\lbl in {#2}{\draw(\kref\pl\n)node[]{$\lbl$};}
}
%====================================
%====================================
%draws triangular edge of the clock. with the edge, clock should nt have label "C"
%\kFFclockEdge[u1]{2}
\newcommand{\kFFclockEdge}[2][u0]{
\def\kref{#1}
\def\cdelX{0.15}
\def\cdelY{0.15}
\def\p{p}			%pin tip
\def\pb{pb}			%pin base
\def\pl{pl}			%pin label
\def\pn{pn}			%pin 'ocirc'
\def\px{x}			%pinAnchorX
\def\py{y}			%pinAnchorY
\foreach \n in {#2}{\draw(\kref\pcd\n)--(\kref\pcm\n)--(\kref\pcu\n);}
}
%==================================
%draws small circle on the "Active Low Pins".    pins are counted like dip6 IC. package. 
%\kFFnegativePin[u1]{2,4,6} 
\newcommand{\kFFnegatePin}[2][u0]{
\def\kref{#1}
\def\p{p}			%pin tip
\def\pb{pb}			%pin base
\def\pl{pl}			%pin label
\def\pn{pn}			%pin 'ocirc'
\def\px{x}			%pinAnchorX
\def\py{y}			%pinAnchorY
\foreach \n in {#2} {\draw[thin](\kref\pn\n)node[ocirc]{};}
}
%====================================
%draws NOTHING, only gives all pin related ANCHORS
% generic comp for extracting all pin anchors such as p1,p2,..,p6,pu,pd,   pb1,pb2,...pb6,pbu,pbd,   pa1,pa2,...,pa6,pau,pad,   %  pcu1,pcm1,pcd1,...pcu6,pcm6,pcd6

%   \kgenSRFF[u1]{xshift}{yshift}     %called with ref desig, used to provide pin anchor naming (such as u1S, u1Q, u1C)
							% and location (xshift,yshift) where the comp is to be placed
							% default ref des is "u0"
\newcommand{\kFFa}[1][u0]{
\def\kref{#1}
\def\kshiftX{0}
\def\kshiftY{0}
\def\p{p}			%pin tip
\def\pb{pb}			%pin base
\def\pl{pl}			%pin label
\def\pn{pn}			%pin 'ocirc'
\def\px{x}			%pinAnchorX
\def\py{y}			%pinAnchorY
\def\pcu{pcu}			%pin tip
\def\pcm{pcm}			%pin tip
\def\pcd{pcd}			%pin tip
\def\pnum{pnum}			%pin number
\pgfmathsetmacro{\klshift}{0.25}
\pgfmathsetmacro{\knumshift}{0.15}   %half of pin length
\pgfmathsetmacro{\knshift}{0.07}
\pgfmathsetmacro{\kpin}{0.30}
\pgfmathsetmacro{\kpsep}{0.40}			%pin to pin distance
\pgfmathsetmacro{\kulV}{0.40}			%edge clearance along vertical edges
\pgfmathsetmacro{\kulH}{0.50}
\pgfmathsetmacro{\kdimX}{2*\kulH+0*\kpsep}
\pgfmathsetmacro{\kdimY}{2*\kulV+2*\kpsep}		%two spaces between 3 pins
\def\cdelX{0.15}
\def\cdelY{0.15}
%%\def\leftPins{0/R/R,2/S/S}
%%\def\rightPins{0/{\overline{Q}}/QN,2/Q/Q}
%\draw[thick](\kshiftX,\kshiftY) rectangle ++(\kdimX,\kdimY);
\draw(\kshiftX,\kshiftY)coordinate(\kref-south-west);
\draw(\kshiftX+\kdimX,\kshiftY)coordinate(\kref-south-east);
\draw(\kshiftX+\kdimX,\kdimY+\kshiftY)coordinate(\kref-north-east);
\draw(\kshiftX,\kdimY+\kshiftY)coordinate(\kref-north-west);
\draw(\kshiftX+\kdimX/2,\kshiftY)coordinate(\kref-south);
\draw(\kshiftX+\kdimX/2,\kdimY+\kshiftY)coordinate(\kref-north);
\draw(\kshiftX,\kdimY/2+\kshiftY)coordinate(\kref-west);
\draw(\kshiftX+\kdimX,\kdimY/2+\kshiftY)coordinate(\kref-east);
\draw(\kshiftX+\kdimX/2,\kdimY/2+\kshiftY)coordinate(\kref-center);
%%left pins
\foreach \n/\m in {0/3,1/2,2/1}{\draw(\kshiftX,\kshiftY+\kulV+\n*\kpsep)
coordinate(\kref\pb\m)+(\klshift,0)coordinate(\kref\pl\m)+(-\knshift,0)coordinate(\kref\pn\m)+(-\kpin,0)coordinate(\kref\p\m)+(-\knumshift,0.5*\kpsep)coordinate(\kref\pnum\m);}
%%right pins
\foreach \n/\m in {0/4,1/5,2/6}{\draw(\kshiftX+\kdimX,\kshiftY+\kulV+\n*\kpsep)
coordinate(\kref\pb\m)+(-\klshift,0)coordinate(\kref\pl\m)+(\knshift,0)coordinate(\kref\pn\m)+(\kpin,0)coordinate(\kref\p\m)+
(\knumshift,0.5*\kpsep)coordinate(\kref\pnum\m);}
%%up pin
\foreach \n/\m in {0/u}{\draw(\kshiftX+\kulH+\n*\kpsep,\kdimY+\kshiftY)
coordinate(\kref\pb\m)+(0,-\klshift)coordinate(\kref\pl\m)+(0,\knshift)coordinate(\kref\pn\m)+(0,\kpin)coordinate(\kref\p\m)+
(0.5*\kpsep,\knumshift)coordinate(\kref\pnum\m);}
%%down pin
\foreach \n/\m in {0/d}{\draw(\kshiftX+\kulH+\n*\kpsep,\kshiftY)
coordinate(\kref\pb\m)+(0,\klshift)coordinate(\kref\pl\m)+(0,-\knshift)coordinate(\kref\pn\m)+(0,-\kpin)coordinate(\kref\p\m)+(0.5*\kpsep,-\knumshift)coordinate(\kref\pnum\m);}
%%clock edges left pin
\foreach \n/\m in {0/3,1/2,2/1}{\draw(\kshiftX,\kshiftY+\kulV+\n*\kpsep)+(0,-\cdelY)
coordinate(\kref\pcd\m)+(\cdelX,0)coordinate(\kref\pcm\m)+(0,\cdelY)coordinate(\kref\pcu\m);}
%%clock edges right pin
\foreach \n/\m in {0/4,1/5,2/6}{\draw(\kshiftX+\kdimX,\kshiftY+\kulV+\n*\kpsep)
+(0,-\cdelY)coordinate(\kref\pcd\m)+(-\cdelX,0)coordinate(\kref\pcm\m)+(0,\cdelY)coordinate(\kref\pcu\m);}
}
%%%%%%%%%%%%%%%%%%%%%%%%
%=============================
%draws general purpose FF's SKELETON ONLY
%labels, pins, clock edge needs to be added separately
%can be anchored with pin at specific coordinate(x,y) as explained down
%%   \kSRFF[u1]{xshift}{yshift}   
%   to anchor pin2 at location (x,y) use
%	\kSRFF[u1]{x-\pbx}{y-\pby}
%pin 1,2,3,4,5,6,u,d are in this context written \pba, ...\pbf,\pbup,\pbdown

\newcommand{\kFF}[3][u0]{
\def\kref{#1}
\def\kshiftX{#2}
\def\kshiftY{#3}

\def\p{p}			%pin tip
\def\pb{pb}			%pin base
\def\pl{pl}			%pin label
\def\pcu{pcu}			%clock edge, upper corner
\def\pcm{pcm}			%clock edge middle corner
\def\pcd{pcd}			%clock edge lower corner
\def\pn{pn}			%pin 'ocirc'
\def\pnum{pnum}			%pin number
\pgfmathsetmacro{\klshift}{0.25}
\pgfmathsetmacro{\knshift}{0.07}
\pgfmathsetmacro{\knumshift}{0.15}   %half of pin length
\pgfmathsetmacro{\kpin}{0.30}
\pgfmathsetmacro{\kpsep}{0.40}			%pin to pin distance
\pgfmathsetmacro{\kulV}{0.40}			%edge clearance along vertical edge
\pgfmathsetmacro{\kulH}{0.50}
\pgfmathsetmacro{\kdimX}{2*\kulH+0*\kpsep}
\pgfmathsetmacro{\kdimY}{2*\kulV+2*\kpsep}		%two spaces between 3 pins
\def\cdelX{0.15}
\def\cdelY{0.15}
%%\def\leftPins{0/R/R,2/S/S}
%%\def\rightPins{0/{\overline{Q}}/QN,2/Q/Q}
\draw[thick](\kshiftX,\kshiftY) rectangle ++(\kdimX,\kdimY);
\draw(\kshiftX,\kshiftY)coordinate(\kref-south-west);
\draw(\kshiftX+\kdimX,\kshiftY)coordinate(\kref-south-east);
\draw(\kshiftX+\kdimX,\kdimY+\kshiftY)coordinate(\kref-north-east);
\draw(\kshiftX,\kdimY+\kshiftY)coordinate(\kref-north-west);
\draw(\kshiftX+\kdimX/2,\kshiftY)coordinate(\kref-south);
\draw(\kshiftX+\kdimX/2,\kdimY+\kshiftY)coordinate(\kref-north);
\draw(\kshiftX,\kdimY/2+\kshiftY)coordinate(\kref-west);
\draw(\kshiftX+\kdimX,\kdimY/2+\kshiftY)coordinate(\kref-east);
\draw(\kshiftX+\kdimX/2,\kdimY/2+\kshiftY)coordinate(\kref-center);
%%left pins
\foreach \n/\m in {0/3,1/2,2/1}{\draw(\kshiftX,\kshiftY+\kulV+\n*\kpsep)
coordinate(\kref\pb\m)+(\klshift,0)coordinate(\kref\pl\m)+(-\knshift,0)coordinate(\kref\pn\m)+(-\kpin,0)coordinate(\kref\p\m)
+(-\knumshift,0.5*\kpsep)coordinate(\kref\pnum\m);}
%%right pins
\foreach \n/\m in {0/4,1/5,2/6}{\draw(\kshiftX+\kdimX,\kshiftY+\kulV+\n*\kpsep)
coordinate(\kref\pb\m)+(-\klshift,0)coordinate(\kref\pl\m)+(\knshift,0)coordinate(\kref\pn\m)+(\kpin,0)coordinate(\kref\p\m)
+(\knumshift,0.5*\kpsep)coordinate(\kref\pnum\m);}
%%up pin
\foreach \n/\m in {0/u}{\draw(\kshiftX+\kulH+\n*\kpsep,\kdimY+\kshiftY)
coordinate(\kref\pb\m)+(0,-\klshift)coordinate(\kref\pl\m)+(0,\knshift)coordinate(\kref\pn\m)+(0,\kpin)coordinate(\kref\p\m)
+(0.5*\kpsep,\knumshift)coordinate(\kref\pnum\m);}
%%down pin
\foreach \n/\m in {0/d}{\draw(\kshiftX+\kulH+\n*\kpsep,\kshiftY)
coordinate(\kref\pb\m)+(0,\klshift)coordinate(\kref\pl\m)+(0,-\knshift)coordinate(\kref\pn\m)+(0,-\kpin)coordinate(\kref\p\m)
+(0.5*\kpsep,-\knumshift)coordinate(\kref\pnum\m);}
%%clock edges left pin
\foreach \n/\m in {0/3,1/2,2/1}{\draw(\kshiftX,\kshiftY+\kulV+\n*\kpsep)+(0,-\cdelY)
coordinate(\kref\pcd\m)+(\cdelX,0)coordinate(\kref\pcm\m)+(0,\cdelY)coordinate(\kref\pcu\m);}
%%clock edges right pin
\foreach \n/\m in {0/4,1/5,2/6}{\draw(\kshiftX+\kdimX,\kshiftY+\kulV+\n*\kpsep)
+(0,-\cdelY)coordinate(\kref\pcd\m)+(-\cdelX,0)coordinate(\kref\pcm\m)+(0,\cdelY)coordinate(\kref\pcu\m);}
}
%%%%%%%%%%%%%%%%%%%%%%%%%%%%%%%
%   to anchor pin2 at location (x,y) use
%	\kgenSRFF[u1]{x-\pbx}{y-\pby}
%pin 1,2,3,4,5,6,u,d are in this context written \pba, ...\pbf,\pbup,\pbdown

\newcommand{\kSRFF}[3][u0]{
\kFF[#1]{#2}{#3}
\def\kref{#1}
%\def\p{p}			%pin tip
%\def\pb{pb}			%pin base
%\def\pl{pl}			%pin label
%\def\pcu{pcu}			%pin edge, upper corner
%\def\pcm{pcm}			%pin edge middle corner
%\def\pcd{pcd}			%pin edge lower corner
%\def\pn{pn}			%pin 'ocirc'
\foreach \n in {1,2,3,4,6}{\draw[thin](\kref\pb\n)--(\kref\p\n);}
\foreach \n /\lbl in {1/S,2/C,3/R,4/{\overline{Q}},6/Q}{\draw[thin](\kref\pl\n)node[]{$\lbl$};}
\foreach \n in {4}{\draw[thin](\kref\pn\n)node[ocirc]{};}
}
%%%%%%%%%%%%%%%%%%%%%%%%%%%%%%%
\newcommand{\kDFF}[3][u0]{
\def\kref{#1}
%\def\p{p}			%pin tip
%\def\pb{pb}			%pin base
%\def\pl{pl}			%pin label
%\def\pcu{pcu}			%pin edge, upper corner
%\def\pcm{pcm}			%pin edge middle corner
%\def\pcd{pcd}			%pin edge lower corner
%\def\pn{pn}			%pin 'ocirc'
\kFF[#1]{#2}{#3}
\foreach \n in {1,2,4,6}{\draw[thin](\kref\pb\n)--(\kref\p\n);}
\foreach \n/\lbl in {1/D,4/{\overline{Q}},6/Q}{\draw[thin](\kref\pl\n)node[]{$\lbl$};}
\foreach \n in {2}{\draw[thin](\kref\pcd\n)--(\kref\pcm\n)--(\kref\pcu\n);}
\foreach \n in {4}{\draw[thin](\kref\pn\n)node[ocirc]{};}
}
%%%%%%%%%%%%%%%%%%%%%%%%%%%%%%%
%%%%%%%%%%%%%%%%%%%%%%%%%%%%%%%
\newcommand{\kDFFud}[3][u0]{
\def\kref{#1}
%\def\p{p}			%pin tip
%\def\pb{pb}			%pin base
%\def\pl{pl}			%pin label
%\def\pcu{pcu}			%pin edge, upper corner
%\def\pcm{pcm}			%pin edge middle corner
%\def\pcd{pcd}			%pin edge lower corner
%\def\pn{pn}			%pin 'ocirc'
\kFF[#1]{#2}{#3}
\foreach \n in {1,2,4,6,u,d}{\draw[thin](\kref\pb\n)--(\kref\p\n);}
\foreach \n/\lbl in {1/D,4/{\overline{Q}},6/Q}{\draw[thin](\kref\pl\n)node[]{$\lbl$};}
\foreach \n in {2}{\draw[thin](\kref\pcd\n)--(\kref\pcm\n)--(\kref\pcu\n);}
\foreach \n in {4,u,d}{\draw[thin](\kref\pn\n)node[ocirc]{};}
}
%%%%%%%%%%%%%%%%%%%%%%%%%%%%%%%
\newcommand{\kJKFF}[3][u0]{
\def\kref{#1}
%\def\p{p}			%pin tip
%\def\pb{pb}			%pin base
%\def\pl{pl}			%pin label
%\def\pcu{pcu}			%pin edge, upper corner
%\def\pcm{pcm}			%pin edge middle corner
%\def\pcd{pcd}			%pin edge lower corner
%\def\pn{pn}			%pin 'ocirc'
\kFF[#1]{#2}{#3}
\foreach \n in {1,2,3,4,6}{\draw[thin](\kref\pb\n)--(\kref\p\n);}
\foreach \n/\lbl in {1/J,3/K,4/{\overline{Q}},6/Q}{\draw(\kref\pl\n)node[]{$\lbl$};}
\foreach \n in {2}{\draw[thin](\kref\pcd\n)--(\kref\pcm\n)--(\kref\pcu\n);}
\foreach \n in {4}{\draw[thin](\kref\pn\n)node[ocirc]{};}
}
%%%%%%%%%%%%%%%%%%%%%%
%%%%%%%%%%%%%%%%%%%%%%%%%%%%%%%
\newcommand{\kJKFFcd}[3][u0]{
\def\kref{#1}
%\def\p{p}			%pin tip
%\def\pb{pb}			%pin base
%\def\pl{pl}			%pin label
%\def\pcu{pcu}			%pin edge, upper corner
%\def\pcm{pcm}			%pin edge middle corner
%\def\pcd{pcd}			%pin edge lower corner
%\def\pn{pn}			%pin 'ocirc'
\kFF[#1]{#2}{#3}
\foreach \n in {1,2,3,6,d}{\draw[thin](\kref\pb\n)--(\kref\p\n);}
\foreach \n/\lbl in {1/J,3/K,6/Q}{\draw[thin](\kref\pl\n)node[]{$\lbl$};}
\foreach \n in {2}{\draw[thin](\kref\pcd\n)--(\kref\pcm\n)--(\kref\pcu\n);}
\foreach \n in {2,d}{\draw[thin](\kref\pn\n)node[ocirc]{};}
}
%%%%%%%%%%%%%%%%%%%%%%%%%%%%%%%
%%%%%%%%%%%%%%%%%%%%%%%%%%%%%%%
\newcommand{\kTFF}[3][u0]{
\def\kref{#1}
%\def\p{p}			%pin tip
%\def\pb{pb}			%pin base
%\def\pl{pl}			%pin label
%\def\pcu{pcu}			%pin edge, upper corner
%\def\pcm{pcm}			%pin edge middle corner
%\def\pcd{pcd}			%pin edge lower corner
%\def\pn{pn}			%pin 'ocirc'
\kFF[#1]{#2}{#3}
\foreach \n in {1,2,4,6}{\draw[thin](\kref\pb\n)--(\kref\p\n);}
\foreach \n/\lbl in {1/T,4/{\overline{Q}},6/Q}{\draw[thin](\kref\pl\n)node[]{$\lbl$};}
\foreach \n in {2}{\draw[thin](\kref\pcd\n)--(\kref\pcm\n)--(\kref\pcu\n);}
\foreach \n in {4}{\draw[thin](\kref\pn\n)node[ocirc]{};}
}
%%%%%%%%%%%%%%%%%%%%%%
%%%%%%%%%%%%%%%%%%%%%%%%%%%%%%%
\newcommand{\kTFFu}[3][u0]{
\def\kref{#1}
%\def\p{p}			%pin tip
%\def\pb{pb}			%pin base
%\def\pl{pl}			%pin label
%\def\pcu{pcu}			%pin edge, upper corner
%\def\pcm{pcm}			%pin edge middle corner
%\def\pcd{pcd}			%pin edge lower corner
%\def\pn{pn}			%pin 'ocirc'
\kFF[#1]{#2}{#3}
\foreach \n in {1,2,4,6,u}{\draw[thin](\kref\pb\n)--(\kref\p\n);}
\foreach \n/\lbl in {1/T,4/{\overline{Q}},6/Q}{\draw[thin](\kref\pl\n)node[]{$\lbl$};}
\foreach \n in {2}{\draw[thin](\kref\pcd\n)--(\kref\pcm\n)--(\kref\pcu\n);}
\foreach \n in {4,u}{\draw[thin](\kref\pn\n)node[ocirc]{};}
}
%%%%%%%%%%%%%%%%%%%%%%
%%%%%%%%%%%%%%%%%%%%%%%%%%%%%%%
\newcommand{\kTFFd}[3][u0]{
\def\kref{#1}
%\def\p{p}			%pin tip
%\def\pb{pb}			%pin base
%\def\pl{pl}			%pin label
%\def\pcu{pcu}			%pin edge, upper corner
%\def\pcm{pcm}			%pin edge middle corner
%\def\pcd{pcd}			%pin edge lower corner
%\def\pn{pn}			%pin 'ocirc'
\kFF[#1]{#2}{#3}
\foreach \n in {1,2,4,6,d}{\draw[thin](\kref\pb\n)--(\kref\p\n);}
\foreach \n/\lbl in {1/T,4/{\overline{Q}},6/Q}{\draw[thin](\kref\pl\n)node[]{$\lbl$};}
\foreach \n in {2}{\draw[thin](\kref\pcd\n)--(\kref\pcm\n)--(\kref\pcu\n);}
\foreach \n in {4,d}{\draw[thin](\kref\pn\n)node[ocirc]{};}
}
%%%%%%%%%%%%%%%%%%%%%%
%%%%%%%%%%%%%%%%%%%%%%%%%%%%%%
\newcommand{\kTFFud}[3][u0]{
\def\kref{#1}
%\def\p{p}			%pin tip
%\def\pb{pb}			%pin base
%\def\pl{pl}			%pin label
%\def\pcu{pcu}			%pin edge, upper corner
%\def\pcm{pcm}			%pin edge middle corner
%\def\pcd{pcd}			%pin edge lower corner
%\def\pn{pn}			%pin 'ocirc'
\kFF[#1]{#2}{#3}
\foreach \n in {1,2,4,6,u,d}{\draw[thin](\kref\pb\n)--(\kref\p\n);}
\foreach \n/\lbl in {1/T,4/{\overline{Q}},6/Q}{\draw[thin](\kref\pl\n)node[]{$\lbl$};}
\foreach \n in {2}{\draw[thin](\kref\pcd\n)--(\kref\pcm\n)--(\kref\pcu\n);}
\foreach \n in {4,u,d}{\draw[thin](\kref\pn\n)node[ocirc]{};}
}
%%%%%%%%%%%%%%%%%%%%%%
% buffer that gives both the Signal and its Complement is  \kBusBuffer
\newcommand{\kBuffer}[3][u0]{
\def\kref{#1}
\def\kshiftX{#2}
\def\kshiftY{#3}

\def\p{p}			%pin tip
\def\pb{pb}			%pin base
\def\pl{pl}			%pin label
\def\pn{pn}			%pin 'ocirc'
\def\pnum{pnum}			%pin number  
\def\pin{pin}			%pin tip
\def\pout{pout}			%pin tip
\def\pu{u}			%pin tip
\def\pd{d}			%pin tip
\pgfmathsetmacro{\klshift}{0.25}
\pgfmathsetmacro{\knshift}{0.07}
\pgfmathsetmacro{\knumshift}{0.15}  %half of pin length
\pgfmathsetmacro{\kpin}{0.30}
\pgfmathsetmacro{\kpsep}{0.40}			%pin to pin distance
\pgfmathsetmacro{\kulV}{0.40}			%edge clearance along vertical edge
\pgfmathsetmacro{\kulH}{0.50}
\pgfmathsetmacro{\kdimX}{2*\kulH+0*\kpsep}
\pgfmathsetmacro{\kdimY}{\kdimX}		%two spaces between 3 pins
\draw[thin](\kshiftX,\kshiftY)coordinate(\kref\pin)--++(\kpin,0)coordinate(\kref\pb\pin);
\draw(\kref\pb\pin)[thick]++(0,-\kdimY/2)coordinate(aa)--++(0,\kdimY)coordinate(bb)--++(\kdimX,-\kdimY/2)coordinate(cc)coordinate(\kref\pb\pout)--++(-\kdimX,-\kdimY/2);
\draw[thin](\kref\pb\pout)--++(\kpin,0)coordinate(\kref\pout);
\draw($(bb)!0.5!(cc)$)coordinate(\kref\pb\pu)+(0,0.07)coordinate(\kref\pn\pu)+(0,\kpin)coordinate(\kref\pu)
+(0.5*\kpin,\knumshift)coordinate(\kref\pnum\pu);
\draw($(aa)!0.5!(cc)$)coordinate(\kref\pb\pd)+(0,-0.07)coordinate(\kref\pn\pd)+(0,-\kpin)coordinate(\kref\pd)
+(0.5*\kpin,-\knumshift)coordinate(\kref\pnum\pd);
\draw(\kref\pb\pin)+(-0.07,0)coordinate(\kref\pn\pin)+(-\knumshift,0.5*\kpsep)coordinate(\kref\pnum\pin);
\draw(\kref\pb\pout)+(0.07,0)coordinate(\kref\pn\pout)+(\knumshift,0.5*\kpsep)coordinate(\kref\pnum\pout);
}
%=====================================
%========================================
%input-output pins. if needs drawing at "coordinate(aa) at (2,3)" use the following command.
%%  \koutleft[j1]{aa}     
%%  \koutleft[j1]{2,3}
\newcommand{\koutleft}[2][j0]{
\def\kref{#1}
\def\keast{east}
\def\kwest{west}
\def\knorth{north}
\def\ksouth{south}
\pgfmathsetmacro{\kpin}{0.30}
\pgfmathsetmacro{\kw}{0.20}
\pgfmathsetmacro{\klen}{0.50}
\pgfmathsetmacro{\kdelX}{0.10}
\draw(#2)--++(-\kpin,0)coordinate(\kref\keast)--++(\kdelX,\kw/2)--++(-\klen,0)coordinate[pos=0.5](\kref\knorth)--++(-\kdelX,-\kw/2)coordinate(\kref\kwest)--++(\kdelX,-\kw/2)--++(\klen,0)coordinate[pos=0.5](\kref\ksouth)--++(-\kdelX,\kw/2);
}
%========================================
\newcommand{\koutright}[2][j0]{
\def\kref{#1}
\def\keast{east}
\def\kwest{west}
\def\knorth{north}
\def\ksouth{south}
\pgfmathsetmacro{\kpin}{0.30}
\pgfmathsetmacro{\kw}{0.20}
\pgfmathsetmacro{\klen}{0.50}
\pgfmathsetmacro{\kdelX}{0.10}
\draw(#2)--++(\kpin,0)coordinate(\kref\kwest)--++(-\kdelX,\kw/2)--++(\klen,0)coordinate[pos=0.5](\kref\knorth)--++(\kdelX,-\kw/2)coordinate(\kref\keast)--++(-\kdelX,-\kw/2)--++(-\klen,0)coordinate[pos=0.5](\kref\ksouth)--++(\kdelX,\kw/2);
}
%========================================
\newcommand{\koutup}[2][j0]{
\def\kref{#1}
\def\keast{east}
\def\kwest{west}
\def\knorth{north}
\def\ksouth{south}
\pgfmathsetmacro{\kpin}{0.30}
\pgfmathsetmacro{\kw}{0.20}
\pgfmathsetmacro{\klen}{0.50}
\pgfmathsetmacro{\kdelX}{0.10}
\draw(#2)--++(0,\kpin)coordinate(\kref\knorth)--++(\kw/2,-\kdelX)--++(0,\klen)coordinate[pos=0.5](\kref\keast)--++(-\kw/2,\kdelX)coordinate(\kref\knorth)--++(-\kw/2,-\kdelX)
--++(0,-\klen)coordinate[pos=0.5](\kref\kwest)--++(\kw/2,\kdelX);
}
%========================================
\newcommand{\koutdown}[2][j0]{
\def\kref{#1}
\def\keast{east}
\def\kwest{west}
\def\knorth{north}
\def\ksouth{south}
\pgfmathsetmacro{\kpin}{0.30}
\pgfmathsetmacro{\kw}{0.20}
\pgfmathsetmacro{\klen}{0.50}
\pgfmathsetmacro{\kdelX}{0.10}
\draw(#2)--++(0,-\kpin)coordinate(\kref\knorth)--++(-\kw/2,\kdelX)--++(0,-\klen)coordinate[pos=0.5](\kref\kwest)--++(\kw/2,-\kdelX)coordinate(\kref\ksouth)--++(\kw/2,\kdelX)
--++(0,\klen)coordinate[pos=0.5](\kref\keast)--++(-\kw/2,-\kdelX);
}
%=====================================
%=====================================
%========================================
\newcommand{\kinright}[2][j0]{
\def\kref{#1}
\def\keast{east}
\def\kwest{west}
\def\knorth{north}
\def\ksouth{south}
\pgfmathsetmacro{\kpin}{0.30}
\pgfmathsetmacro{\kw}{0.20}
\pgfmathsetmacro{\klen}{0.50}
\pgfmathsetmacro{\kdelX}{0.10}
\draw(#2)--++(\kpin,0)coordinate(\kref\kwest)--++(\kdelX,\kw/2)--++(\klen,0)coordinate[pos=0.5](\kref\knorth)--++(-\kdelX,-\kw/2)coordinate(aa)--++(\kdelX,-\kw/2)--++(-\klen,0)coordinate[pos=0.5](\kref\ksouth)--++(-\kdelX,\kw/2) 
(aa)++(\kdelX,0)coordinate(\kref\keast);
}
%========================================
\newcommand{\kinleft}[2][j0]{
\def\kref{#1}
\def\keast{east}
\def\kwest{west}
\def\knorth{north}
\def\ksouth{south}
\pgfmathsetmacro{\kpin}{0.30}
\pgfmathsetmacro{\kw}{0.20}
\pgfmathsetmacro{\klen}{0.50}
\pgfmathsetmacro{\kdelX}{0.10}
\draw(#2)--++(-\kpin,0)coordinate(\kref\keast)--++(-\kdelX,\kw/2)--++(-\klen,0)coordinate[pos=0.5](\kref\knorth)--++(\kdelX,-\kw/2)coordinate(aa)--++(-\kdelX,-\kw/2)--++(\klen,0)coordinate[pos=0.5](\kref\ksouth)--++(\kdelX,\kw/2) 
(aa)++(-\kdelX,0)coordinate(\kref\kwest);
}
%========================================
\newcommand{\kindown}[2][j0]{
\def\kref{#1}
\def\keast{east}
\def\kwest{west}
\def\knorth{north}
\def\ksouth{south}
\pgfmathsetmacro{\kpin}{0.30}
\pgfmathsetmacro{\kw}{0.20}
\pgfmathsetmacro{\klen}{0.50}
\pgfmathsetmacro{\kdelX}{0.10}
\draw(#2)--++(0,-\kpin)coordinate(\kref\knorth)--++(\kw/2,-\kdelX)--++(0,-\klen)
coordinate[pos=0.5](\kref\keast)--++(-\kw/2,\kdelX)coordinate(aa)--++(-\kw/2,-\kdelX)
--++(0,\klen)coordinate[pos=0.5](\kref\kwest)--++(\kw/2,\kdelX) (aa)++(0,-\kdelX)coordinate(\kref\ksouth);
}
%========================================
\newcommand{\kinup}[2][j0]{
\def\kref{#1}
\def\keast{east}
\def\kwest{west}
\def\knorth{north}
\def\ksouth{south}
\pgfmathsetmacro{\kpin}{0.30}
\pgfmathsetmacro{\kw}{0.20}
\pgfmathsetmacro{\klen}{0.50}
\pgfmathsetmacro{\kdelX}{0.10}
\draw(#2)--++(0,\kpin)coordinate(\kref\ksouth)--++(-\kw/2,\kdelX)--++(0,\klen)coordinate[pos=0.5](\kref\kwest)--++(\kw/2,-\kdelX)--++(\kw/2,\kdelX)coordinate(aa)--++(0,-\klen)coordinate[pos=0.5](\kref\keast)--++(-\kw/2,-\kdelX) (aa)(aa)++(0,\kdelX)coordinate(\kref\knorth);
}
%=====================================
%%%%%%%%%%%%%%%%%%%%%%%%%%%%%%%%
%=====================================
%=====================================
%========================================
%these are control signals, inputs only. 
%% \kinrightA[j1]{2,3}
%%\kinrightA[j1]{coordName}
\newcommand{\kinrightA}[2][j0]{
\def\kref{#1}
\def\keast{east}
\def\kwest{west}
\def\knorth{north}
\def\ksouth{south}
\pgfmathsetmacro{\kpin}{0.30}
\pgfmathsetmacro{\kw}{0.20}
\pgfmathsetmacro{\klen}{0.50}
\pgfmathsetmacro{\kdelX}{0.10}
\draw(#2)--++(\kpin,0)coordinate(\kref\kwest)--++(\kdelX,\kw/2)--++(\klen,0)coordinate[pos=0.5](\kref\knorth)--++(0,-\kw)coordinate[pos=0.5](\kref\keast)--++(-\klen,0)coordinate[pos=0.5](\kref\ksouth)--++(-\kdelX,\kw/2);
}
%========================================
\newcommand{\kinleftA}[2][j0]{
\def\kref{#1}
\def\keast{east}
\def\kwest{west}
\def\knorth{north}
\def\ksouth{south}
\pgfmathsetmacro{\kpin}{0.30}
\pgfmathsetmacro{\kw}{0.20}
\pgfmathsetmacro{\klen}{0.50}
\pgfmathsetmacro{\kdelX}{0.10}
\draw(#2)--++(-\kpin,0)coordinate(\kref\keast)--++(-\kdelX,\kw/2)--++(-\klen,0)coordinate[pos=0.5](\kref\knorth)--++(0,-\kw)coordinate[pos=0.5](\kref\kwest)--++(\klen,0)coordinate[pos=0.5](\kref\ksouth)--++(\kdelX,\kw/2);
}
%========================================
\newcommand{\kindownA}[2][j0]{
\def\kref{#1}
\def\keast{east}
\def\kwest{west}
\def\knorth{north}
\def\ksouth{south}
\pgfmathsetmacro{\kpin}{0.30}
\pgfmathsetmacro{\kw}{0.20}
\pgfmathsetmacro{\klen}{0.50}
\pgfmathsetmacro{\kdelX}{0.10}
\draw(#2)--++(0,-\kpin)coordinate(\kref\knorth)--++(\kw/2,-\kdelX)--++(0,-\klen)
coordinate[pos=0.5](\kref\keast)--++(-\kw,0)coordinate[pos=0.5](\kref\ksouth)
--++(0,\klen)coordinate[pos=0.5](\kref\kwest)--++(\kw/2,\kdelX);
}
%========================================
\newcommand{\kinupA}[2][j0]{
\def\kref{#1}
\def\keast{east}
\def\kwest{west}
\def\knorth{north}
\def\ksouth{south}
\pgfmathsetmacro{\kpin}{0.30}
\pgfmathsetmacro{\kw}{0.20}
\pgfmathsetmacro{\klen}{0.50}
\pgfmathsetmacro{\kdelX}{0.10}
\draw(#2)--++(0,\kpin)coordinate(\kref\ksouth)--++(-\kw/2,\kdelX)--++(0,\klen)coordinate[pos=0.5](\kref\kwest)--++(\kw,0)coordinate[pos=0.5](\kref\knorth)--++(0,-\klen)coordinate[pos=0.5](\kref\keast)--++(-\kw/2,-\kdelX);
}
%=====================================
%%%%%%%%%%%%%%%%%%%%%%%%%%%%%%%%
%+++++++++++++++++++++++++++++++++++++++++++++
%====================================
\begin{comment}
%draws NOTHING, only gives all pin related ANCHORS
% 

%   \kBOXa[u1]{xPins}{yPins}     %xPins and yPins are max pins expected along these edges

\newcommand{\kBOXa}[3][u0]{
\def\kref{#1}
\def\nnx{#2} 			%count of horizontal pins
\def\nny{#3}			%count of vertical pins
\pgfmathsetmacro{\nx}{\nnx-1}
\pgfmathsetmacro{\ny}{\nny-1}
\def\kshiftX{0}
\def\kshiftY{0}
\def\p{p}			%pin tip
\def\a{a}			%left pins are named \kref\a\p\n where \n=0 for lowest pin
\def\b{b}			%bottom pins are named \kref\b\p\n where \n=0 for left most pin
\def\c{c}			%right pins are named \kref\c\p\n where \n=0 for lowest pin
\def\d{d}			%top pins are named \kref\d\p\n where \n=0 for left most pin
\def\pb{pb}			%pin base
\def\pl{pl}			%pin label
\def\pn{pn}			%pin 'ocirc' indicating active LOW 
\def\pnum{pnum}		%pin number location; use \kref\a\pnum\n e.g. u0apnum3 means left 3rd from bottom
\def\px{x}			%pinAnchorX
\def\py{y}			%pinAnchorY
\def\pcu{pcu}			%clock edge top end; \kref\a\pcu\n
\def\pcm{pcm}			%clock edge mid
\def\pcd{pcd}			%clock edge bottom end
\pgfmathsetmacro{\klshift}{0.25}
\pgfmathsetmacro{\knshift}{0.07}
\pgfmathsetmacro{\knumshift}{0.25}
\pgfmathsetmacro{\kpin}{0.30}			%pin length
\pgfmathsetmacro{\kpsep}{0.40}			%pin to pin distance
\pgfmathsetmacro{\kulVs}{0.50}			%South edge clearance along vertical edge
\pgfmathsetmacro{\kulVn}{0.30}		%North edge clearance along vertical edge
\pgfmathsetmacro{\kulHe}{0.40}		%East edge clearance along horizontal side
\pgfmathsetmacro{\kulHw}{0.50}		%West edge clearance along horizontal side
\pgfmathsetmacro{\kdimX}{\kulHe+\kulHw+\nx*\kpsep}
\pgfmathsetmacro{\kdimY}{\kulVn+\kulVs+\ny*\kpsep}		%two spaces between 3 pins
\def\cdelX{0.15}
\def\cdelY{0.15}
%%\def\leftPins{0/R/R,2/S/S}
%%\def\rightPins{0/{\overline{Q}}/QN,2/Q/Q}
%\draw[thick](\kshiftX,\kshiftY) rectangle ++(\kdimX,\kdimY);
\draw(\kshiftX,\kshiftY)coordinate(\kref-south-west);
\draw(\kshiftX+\kdimX,\kshiftY)coordinate(\kref-south-east);
\draw(\kshiftX+\kdimX,\kdimY+\kshiftY)coordinate(\kref-north-east);
\draw(\kshiftX,\kdimY+\kshiftY)coordinate(\kref-north-west);
\draw(\kshiftX+\kdimX/2,\kshiftY)coordinate(\kref-south);
\draw(\kshiftX+\kdimX/2,\kdimY+\kshiftY)coordinate(\kref-north);
\draw(\kshiftX,\kdimY/2+\kshiftY)coordinate(\kref-west);
\draw(\kshiftX+\kdimX,\kdimY/2+\kshiftY)coordinate(\kref-east);
\draw(\kshiftX+\kdimX/2,\kdimY/2+\kshiftY)coordinate(\kref-center);
%%left pins
\foreach \n in {0,...,\ny}{\draw(\kshiftX,\kshiftY+\kulVs+\n*\kpsep)
coordinate(\kref\a\pb\n)+(\klshift,0)coordinate(\kref\a\pl\n)+(-\knshift,0)coordinate(\kref\a\pn\n)+(-\kpin,0)coordinate(\kref\a\p\n)+(-\knumshift,1.25ex)coordinate(\kref\a\pnum\n);}
%%right pins
\foreach \n in {0,...,\ny}{\draw(\kshiftX+\kdimX,\kshiftY+\kulVs+\n*\kpsep)
coordinate(\kref\c\pb\n)+(-\klshift,0)coordinate(\kref\c\pl\n)+(\knshift,0)coordinate(\kref\c\pn\n)+(\kpin,0)coordinate(\kref\c\p\n)+(\knumshift,1.25ex)coordinate(\kref\c\pnum\n);}
%%up pin
\foreach \n in {0,...,\nx}{\draw(\kshiftX+\kulHw+\n*\kpsep,\kdimY+\kshiftY)
coordinate(\kref\d\pb\n)+(0,-\klshift)coordinate(\kref\d\pl\n)+(0,\knshift)coordinate(\kref\d\pn\n)+(0,\kpin)coordinate(\kref\d\p\n)+(1.25ex,\knumshift)coordinate(\kref\d\pnum\n);}
%%down pin
\foreach \n in {0,...,\nx}{\draw(\kshiftX+\kulHw+\n*\kpsep,\kshiftY)
coordinate(\kref\b\pb\n)+(0,\klshift)coordinate(\kref\b\pl\n)+(0,-\knshift)coordinate(\kref\b\pn\n)+(0,-\kpin)coordinate(\kref\b\p\n)+(1.25ex,-\knumshift)coordinate(\kref\b\pnum\n);}
%%clock edges left pin
\foreach \n in {0,...,\ny}{\draw(\kshiftX,\kshiftY+\kulVs+\n*\kpsep)+(0,-\cdelY)
coordinate(\kref\a\pcd\n)+(\cdelX,0)coordinate(\kref\a\pcm\n)+(0,\cdelY)coordinate(\kref\a\pcu\n);}
%%clock edges right pin
\foreach \n in {0,...,\ny}{\draw(\kshiftX+\kdimX,\kshiftY+\kulVs+\n*\kpsep)
+(0,-\cdelY)coordinate(\kref\c\pcd\n)+(-\cdelX,0)coordinate(\kref\c\pcm\n)+(0,\cdelY)coordinate(\kref\c\pcu\n);}
%%clock edges up pin
\foreach \n in {0,...,\nx}{\draw(\kshiftX+\kulHw+\n*\kpsep,\kshiftY+\kdimY)+(-\cdelY,0)
coordinate(\kref\d\pcd\n)+(0,-\cdelX)coordinate(\kref\d\pcm\n)+(\cdelY,0)coordinate(\kref\d\pcu\n);}
%%clock edges down pin
\foreach \n in {0,...,\nx}{\draw(\kshiftX+\kulHw+\n*\kpsep,\kshiftY)+(-\cdelY,0)
coordinate(\kref\b\pcd\n)+(0,\cdelX)coordinate(\kref\b\pcm\n)+(\cdelY,0)coordinate(\kref\b\pcu\n);}
}
%%%%%%%%%%%%%%%%%%%%%%%%
%===================================
%should take input limits and strings for the four sides
\newcommand{\kBOXanc}{
\pgfmathsetmacro{\kpin}{0.30}
\pgfmathsetmacro{\kpsep}{0.40}			%pin to pin distance
\pgfmathsetmacro{\kulVs}{0.4}%{0.50}			%South edge clearance along vertical edge
\pgfmathsetmacro{\kulVn}{0.4}%{0.30}		%North edge clearance along vertical edge
\pgfmathsetmacro{\kulHe}{0.4}%{0.40}		%East edge clearance along horizontal side
\pgfmathsetmacro{\kulHw}{0.4}%{0.50}		%West edge clearance along horizontal side
%%left pins
\pgfmathsetmacro\spax{\kulHw+0*\kpsep}
\pgfmathsetmacro\spbx{\kulHw+1*\kpsep}
\pgfmathsetmacro\spcx{\kulHw+2*\kpsep}
\pgfmathsetmacro\spdx{\kulHw+3*\kpsep}
\pgfmathsetmacro\spex{\kulHw+4*\kpsep}
\pgfmathsetmacro\spfx{\kulHw+5*\kpsep}
\pgfmathsetmacro\spgx{\kulHw+6*\kpsep}
\pgfmathsetmacro\wpay{\kulVs+0*\kpsep}
\pgfmathsetmacro\wpby{\kulVs+1*\kpsep}
\pgfmathsetmacro\wpcy{\kulVs+2*\kpsep}
\pgfmathsetmacro\wpdy{\kulVs+3*\kpsep}
\pgfmathsetmacro\wpey{\kulVs+4*\kpsep}
\pgfmathsetmacro\wpfy{\kulVs+5*\kpsep}
\pgfmathsetmacro\wpgy{\kulVs+6*\kpsep}
\pgfmathsetmacro\wphy{\kulVs+7*\kpsep}
}
\end{comment}
%=============================
%draws general purpose BOX SKELETON ONLY
%labels, pins, and clock edge needs to be added separately
%can be anchored with pin at specific coordinate(x,y) as explained down
%%   \kBOX[u1]{xshift}{yshift}{xPins}{yPins}
%   to anchor pin2 at location (x,y) use
%	\kBOX[u1]{x-\pbx}{y-\pby}{xPins}{yPins}
%pin 1,2,3,4,5,6,u,d are in this context written \pba, ...\pbf,\pbup,\pbdown

\newcommand{\kBOX}[6][u0]{	%cannot use double quotes as the pin names allocated here are used when using the ICs
\def\kref{#1}
\def\kshiftX{#2}
\def\kshiftY{#3}
\def\nnx{#4}
\def\nny{#5}
\def\kBOXpin{#6}				%it is just \kpin but named it different variables here are global here
\pgfmathsetmacro{\nx}{\nnx-1}
\pgfmathsetmacro{\ny}{\nny-1}
\def\p{p}			%pin tip
\def\a{a}			%left pins
\def\b{b}			%bottom pins 
\def\c{c}			%right pins
\def\d{d}			%top pins
\def\pb{pb}			%pin base
\def\pl{pl}			%pin label
\def\pnum{pnum}		%pin number location; use \kref\a\pnum\n e.g. u0apnum3 means left 3rd from bottom
\def\pcu{pcu}			%clock edge, upper corner
\def\pcm{pcm}			%clock edge middle corner
\def\pcd{pcd}			%clock edge lower corner
\def\pn{pn}			%pin 'ocirc'
\pgfmathsetmacro{\klshift}{0.25}
\pgfmathsetmacro{\knshift}{0.07}
\pgfmathsetmacro{\knumshift}{0.25}
%\pgfmathsetmacro{\kpin}{0.30}
\pgfmathsetmacro{\kpsep}{0.40}			%pin to pin distance
\pgfmathsetmacro{\kul}{0.40}			%pin to edge distance	
\pgfmathsetmacro{\kdimX}{\kul+\kul+\nx*\kpsep}
\pgfmathsetmacro{\kdimY}{\kul+\kul+\ny*\kpsep}		%two spaces between 3 pins
\def\cdelX{0.15}
\def\cdelY{0.15}
%%\def\leftPins{0/R/R,2/S/S}
%%\def\rightPins{0/{\overline{Q}}/QN,2/Q/Q}
\draw[thick](\kshiftX,\kshiftY) rectangle ++(\kdimX,\kdimY);
\draw(\kshiftX,\kshiftY)coordinate(\kref-south-west);
\draw(\kshiftX+\kdimX,\kshiftY)coordinate(\kref-south-east);
\draw(\kshiftX+\kdimX,\kdimY+\kshiftY)coordinate(\kref-north-east);
\draw(\kshiftX,\kdimY+\kshiftY)coordinate(\kref-north-west);
\draw(\kshiftX+\kdimX/2,\kshiftY)coordinate(\kref-south);
\draw(\kshiftX+\kdimX/2,\kdimY+\kshiftY)coordinate(\kref-north);
\draw(\kshiftX,\kdimY/2+\kshiftY)coordinate(\kref-west);
\draw(\kshiftX+\kdimX,\kdimY/2+\kshiftY)coordinate(\kref-east);
\draw(\kshiftX+\kdimX/2,\kdimY/2+\kshiftY)coordinate(\kref-center);
%%left pins
\foreach \n in {0,...,\ny}{\draw(\kshiftX,\kshiftY+\kul+\n*\kpsep)
coordinate(\kref\a\pb\n)+(\klshift,0)coordinate(\kref\a\pl\n)+(-\knshift,0)coordinate(\kref\a\pn\n)+(-\kBOXpin,0)coordinate(\kref\a\p\n)+(-\knumshift,1.25ex)coordinate(\kref\a\pnum\n);}
%%right pins
\foreach \n in {0,...,\ny}{\draw(\kshiftX+\kdimX,\kshiftY+\kul+\n*\kpsep)
coordinate(\kref\c\pb\n)+(-\klshift,0)coordinate(\kref\c\pl\n)+(\knshift,0)coordinate(\kref\c\pn\n)+(\kBOXpin,0)coordinate(\kref\c\p\n)+(\knumshift,1.25ex)coordinate(\kref\c\pnum\n);}
%%up pin
\foreach \n in {0,...,\nx}{\draw(\kshiftX+\kul+\n*\kpsep,\kdimY+\kshiftY)
coordinate(\kref\d\pb\n)+(0,-\klshift)coordinate(\kref\d\pl\n)+(0,\knshift)coordinate(\kref\d\pn\n)+(0,\kBOXpin)coordinate(\kref\d\p\n)+(1.25ex,\knumshift)coordinate(\kref\d\pnum\n);}
%%down pin
\foreach \n in {0,...,\nx}{\draw(\kshiftX+\kul+\n*\kpsep,\kshiftY)
coordinate(\kref\b\pb\n)+(0,\klshift)coordinate(\kref\b\pl\n)+(0,-\knshift)coordinate(\kref\b\pn\n)+(0,-\kBOXpin)coordinate(\kref\b\p\n)+(1.25ex,-\knumshift)coordinate(\kref\b\pnum\n);}
%%clock edges left pin
\foreach \n in {0,...,\ny}{\draw(\kshiftX,\kshiftY+\kul+\n*\kpsep)+(0,-\cdelY)
coordinate(\kref\a\pcd\n)+(\cdelX,0)coordinate(\kref\a\pcm\n)+(0,\cdelY)coordinate(\kref\a\pcu\n);}
%%clock edges right pin
\foreach \n in {0,...,\ny}{\draw(\kshiftX+\kdimX,\kshiftY+\kul+\n*\kpsep)
+(0,-\cdelY)coordinate(\kref\c\pcd\n)+(-\cdelX,0)coordinate(\kref\c\pcm\n)+(0,\cdelY)coordinate(\kref\c\pcu\n);}
%%clock edges up pin
\foreach \n in {0,...,\nx}{\draw(\kshiftX+\kul+\n*\kpsep,\kshiftY+\kdimY)+(-\cdelY,0)
coordinate(\kref\d\pcd\n)+(0,-\cdelX)coordinate(\kref\d\pcm\n)+(\cdelY,0)coordinate(\kref\d\pcu\n);}
%%clock edges down pin
\foreach \n in {0,...,\nx}{\draw(\kshiftX+\kul+\n*\kpsep,\kshiftY)+(-\cdelY,0)
coordinate(\kref\b\pcd\n)+(0,\cdelX)coordinate(\kref\b\pcm\n)+(\cdelY,0)coordinate(\kref\b\pcu\n);}
}
%%%%%%%%%%%%%%%%%%%%%%%%%%%%%%%
\newcommand{\kmuxD}[3][u0]{
\def\kref{#1}
\kBOX{#2}{#3}{4}{3}{0.3}
\foreach \n/\m in{0/3,1/2,2/1,3/0}{\draw[thin](\kref\b\pb\n)--(\kref\b\p\n) (\kref\b\pl\n)node[]{\scriptsize{$\m$}};}
\foreach \n/\m in{1/{a_0},2/{a_1}}{\draw[thin](\kref\a\pb\n)--(\kref\a\p\n) (\kref\a\pl\n)node[]{\small{$\m$}};}
\foreach \n in{2}{\draw[thin](\kref\d\pb\n)--(\kref\d\p\n);}
}
%================================
%%%%%%%%%%%%%%%%%%%%%%%%
% OR gate  skeleton for use with  input BUS. Input can be accessed with say u0in and output with u0out
%\kBusOR[u1]{Xlocation}{Ylocation}
\newcommand{\kBusOR}[3][u0]{{
\def\kref{#1}
\def\kshiftX{#2}
\def\kshiftY{#3}
\def\kin{in}
\def\kout{out}
\def\p{p}			%pin tip
\def\pb{pb}			%pin base
\def\pl{pl}			%pin label
\def\pn{pn}			%pin 'ocirc'
\pgfmathsetmacro{\kxdim}{1}
\pgfmathsetmacro{\kydim}{1}
\draw[thick] (\kshiftX,\kshiftY) to [out=45,in=-45]coordinate[pos=0.5](\kref\kin) ++(0,\kydim) to [out=0,in=130] ++(\kxdim,-0.5*\kydim)coordinate(\kref\kout) to [out=-130,in=0]++(-\kxdim,-0.5*\kydim);}}
%%%%%%%%%%%%%%%%%%%%%%%%%%%%
\newcommand{\kBusBuffer}[3][u0]{{
\def\kref{#1}
\def\kshiftX{#2}
\def\kshiftY{#3}
\def\kin{in}
\def\kout{out}
\def\p{p}			%pin tip
\def\pb{pb}			%pin base
\def\pl{pl}			%pin label
\def\pn{pn}			%pin 'ocirc'
\def\pu{pu}
\def\pd{pd}
\pgfmathsetmacro{\klen}{1}
\draw[thick](\kshiftX,\kshiftY)coordinate(aa)++(0,-0.5*\klen)--++(30:\klen)
coordinate[pos=0.6](keda)--++(150:\klen)coordinate[pos=0.4](ked)--++(-90:\klen);
\draw[thin](ked)--++(\kpin,0)coordinate(\kref\pu)  (keda)--++(\kpin,0)coordinate(\kref\pd) 
(aa)--++(-\kpin,0)coordinate(\kref\kin);
\draw[thick]  (ked)++(0.07,0)node[ocirc,fill]{};
}}
%====================================
%%%%%%%%%%%%%%%%%%%%%%%%%%%%
%draws a cross at the location to show intact fuses in non volatile memory
\newcommand{\kCross}[1]{{
\pgfmathsetmacro{\kmv}{0.15}
 \draw[thin](#1)++(-0.5*\kmv,-0.5*\kmv)--++(\kmv,\kmv);
 \draw[thin](#1)++(-0.5*\kmv,0.5*\kmv)--++(\kmv,-\kmv);
}}
%%%%%%%%%%%%%%%%%%%%%%%%%%%%%%
%%%%%%%%%%%%%%%%%%%%%%%%%%%%%%%
% TTL used in SAP-1
%74107-A  JK-FF
%numbering code 0=z, 1=o, 2=t, 3=T, 4=f, 5=F, 6=s, 7=S, 8=e, 9=n
\newcommand{\kSfozSA}[3][u0]{
\def\kref{#1}
%\def\p{p}			%pin tip
%\def\pb{pb}			%pin base
%\def\pl{pl}			%pin label
%\def\pcu{pcu}			%pin edge, upper corner
%\def\pcm{pcm}			%pin edge middle corner
%\def\pcd{pcd}			%pin edge lower corner
%\def\pn{pn}			%pin 'ocirc'
\kFF[#1]{#2}{#3}
\foreach \n in {1,2,3,6,d}{\draw[thin](\kref\pb\n)--(\kref\p\n);}
\foreach \n/\lbl in {1/J,3/K,6/Q}{\draw[thin](\kref\pl\n)node[]{$\lbl$};}
\foreach \n in {2}{\draw[thin](\kref\pcd\n)--(\kref\pcm\n)--(\kref\pcu\n);}
\foreach \n in {2,d}{\draw[thin](\kref\pn\n)node[ocirc]{};}
\foreach \n/\a in {1/1,2/12,3/4,d/13,6/3}{\draw[thin](\kref\pnum\n)node[]{\scriptsize$\a$};}
}
%%%%%%%%%%%%%%%%%%%%%%%%%%%%%%
% TTL used in SAP-1
%74107-B   JK-FF
%numbering code 0=z, 1=o, 2=t, 3=T, 4=f, 5=F, 6=s, 7=S, 8=e, 9=n
\newcommand{\kSfozSB}[3][u0]{
\def\kref{#1}
%\def\p{p}			%pin tip
%\def\pb{pb}			%pin base
%\def\pl{pl}			%pin label
%\def\pcu{pcu}			%pin edge, upper corner
%\def\pcm{pcm}			%pin edge middle corner
%\def\pcd{pcd}			%pin edge lower corner
%\def\pn{pn}			%pin 'ocirc'
\kFF[#1]{#2}{#3}
\foreach \n in {1,2,3,6,d}{\draw[thin](\kref\pb\n)--(\kref\p\n);}
\foreach \n/\lbl in {1/J,3/K,6/Q}{\draw[thin](\kref\pl\n)node[]{$\lbl$};}
\foreach \n in {2}{\draw[thin](\kref\pcd\n)--(\kref\pcm\n)--(\kref\pcu\n);}
\foreach \n in {2,d}{\draw[thin](\kref\pn\n)node[ocirc]{};}
\foreach \n/\a in {1/8,2/9,3/11,d/10,6/5}{\draw[thin](\kref\pnum\n)node[]{\scriptsize$\a$};}
}
%==========================================
%==============================
%numbering code 0=z, 1=o, 2=t, 3=T, 4=f, 5=F, 6=s, 7=S, 8=e, 9=n
%74LS126
\newcommand{\kSfots}[3][u0]{	
\def\kref{#1}
\def\kshiftX{#2}
\def\kshiftY{#3}
\kBOX[#1]{#2}{#3}{4}{4}{0.5}
\foreach \n/\lbl in {0/13,1/10,2/4,3/1}\draw[thin](\kref\a\pb\n)--(\kref\a\p\n) (\kref\a\pnum\n)node[]{\scriptsize{$\lbl$}}; %left
\foreach \n/\lbl in {0/3,1/6,2/8,3/11}\draw[thin](\kref\b\pb\n)--(\kref\b\p\n) (\kref\b\pnum\n)node[]{\scriptsize{$\lbl$}};    %bottom
\foreach \n/\lbl in {0/7,3/14}\draw[thin](\kref\c\pb\n)--(\kref\c\p\n) (\kref\c\pnum\n)node[]{\scriptsize{$\lbl$}};		%right
\foreach \n/\lbl in {0/2,1/5,2/9,3/12}\draw[thin](\kref\d\pb\n)--(\kref\d\p\n) (\kref\d\pnum\n)node[]{\scriptsize{$\lbl$}};
%\foreach \n in {0}\draw[thin](\kref\a\pb\n)++(-\knshift,0)node[ocirc]{};
%\foreach \n in {0,1,2,3}\draw[thin](\kref\b\pb\n)++(0,-\knshift)node[ocirc]{};
%\foreach \n in {3,4}\draw[thin](\kref\c\pb\n)++(\knshift,0)node[ocirc]{};
%\foreach \n in {0}\draw[thin](\kref\d\pb\n)++(0,\knshift)node[ocirc]{};
}
%==============================
%numbering code 0=z, 1=o, 2=t, 3=T, 4=f, 5=F, 6=s, 7=S, 8=e, 9=n
%74LS126  rotatedAntiClockWise
\newcommand{\kSfotsACW}[3][u0]{
\def\kref{#1}
\def\kshiftX{#2}
\def\kshiftY{#3}
\kBOX[#1]{#2}{#3}{4}{4}{0.5}
\foreach \n/\lbl in {0/2,1/5,2/9,3/11}\draw[thin](\kref\a\pb\n)--(\kref\a\p\n) (\kref\a\pnum\n)node[]{\scriptsize{$\lbl$}}; %left
\foreach \n/\lbl in {0/1,1/4,2/10,3/13}\draw[thin](\kref\b\pb\n)--(\kref\b\p\n) (\kref\b\pnum\n)node[]{\scriptsize{$\lbl$}};   %bottom
\foreach \n/\lbl in {0/3,1/6,2/8,3/11}\draw[thin](\kref\c\pb\n)--(\kref\c\p\n) (\kref\c\pnum\n)node[]{\scriptsize{$\lbl$}};	%right
\foreach \n/\lbl in {0/7,3/14}\draw[thin](\kref\d\pb\n)--(\kref\d\p\n) (\kref\d\pnum\n)node[]{\scriptsize{$\lbl$}};
%\foreach \n in {0}\draw[thin](\kref\a\pb\n)++(-\knshift,0)node[ocirc]{};
%\foreach \n in {0,1,2,3}\draw[thin](\kref\b\pb\n)++(0,-\knshift)node[ocirc]{};
%\foreach \n in {3,4}\draw[thin](\kref\c\pb\n)++(\knshift,0)node[ocirc]{};
%\foreach \n in {0}\draw[thin](\kref\d\pb\n)++(0,\knshift)node[ocirc]{};
}
%============================================
%==============================
%numbering code 0=z, 1=o, 2=t, 3=T, 4=f, 5=F, 6=s, 7=S, 8=e, 9=n
%74LS126  rotatedClockWise
\newcommand{\kSfotsCW}[3][u0]{
\def\kref{#1}
\def\kshiftX{#2}
\def\kshiftY{#3}
\kBOX[#1]{#2}{#3}{4}{4}{0.5}
\foreach \n/\lbl in {0/11,1/8,2/6,3/3}\draw[thin](\kref\a\pb\n)--(\kref\a\p\n) (\kref\a\pnum\n)node[]{\scriptsize{$\lbl$}}; %left
\foreach \n/\lbl in {0/7,3/14}\draw[thin](\kref\b\pb\n)--(\kref\b\p\n) (\kref\b\pnum\n)node[]{\scriptsize{$\lbl$}};   %bottom
\foreach \n/\lbl in {0/12,1/9,2/5,3/2}\draw[thin](\kref\c\pb\n)--(\kref\c\p\n) (\kref\c\pnum\n)node[]{\scriptsize{$\lbl$}};	%right
\foreach \n/\lbl in {0/13,1/10,2/4,3/1}\draw[thin](\kref\d\pb\n)--(\kref\d\p\n) (\kref\d\pnum\n)node[]{\scriptsize{$\lbl$}};
%\foreach \n in {0}\draw[thin](\kref\a\pb\n)++(-\knshift,0)node[ocirc]{};
%\foreach \n in {0,1,2,3}\draw[thin](\kref\b\pb\n)++(0,-\knshift)node[ocirc]{};
%\foreach \n in {3,4}\draw[thin](\kref\c\pb\n)++(\knshift,0)node[ocirc]{};
%\foreach \n in {0}\draw[thin](\kref\d\pb\n)++(0,\knshift)node[ocirc]{};
}
%============================================
%%%%%%%%%%%%%%%%%%%%%%
%numbering code 0=z, 1=o, 2=t, 3=T, 4=f, 5=F, 6=s, 7=S, 8=e, 9=n
%74126-A
% buffer that gives both the Signal and its Complement is  \kBusBuffer
\newcommand{\kSfotsA}[3][u0]{
\def\kref{#1}
\def\kshiftX{#2}
\def\kshiftY{#3}
\def\pbu{pbu}
\def\pu{pu}
\def\kref{#1}
\kBuffer[#1]{#2}{#3}
\draw[thin](\kref\pbu)--(\kref\pu);
%\draw[thin](\kref\pnum\pin)node[]{$2$};
\foreach \n/\a in {pin/2,pout/3,u/1}{\draw[thin](\kref\pnum\n)node[]{\scriptsize$\a$};}
}
%%%%%%%%%%%%%%%%%%%%%%
%74126-B
% buffer that gives both the Signal and its Complement is  \kBusBuffer
\newcommand{\kSfotsB}[3][u0]{
\def\kref{#1}
\def\kshiftX{#2}
\def\kshiftY{#3}
\def\pbu{pbu}
\def\pu{pu}
\def\kref{#1}
\kBuffer[#1]{#2}{#3}
\draw[thin](\kref\pbu)--(\kref\pu);
%\draw[thin](\kref\pnum\pin)node[]{$2$};
\foreach \n/\a in {pin/5,pout/6,u/4}{\draw[thin](\kref\pnum\n)node[]{\scriptsize$\a$};}
}
%%%%%%%%%%%%%%%%%%%%%%
%74126-C
% buffer that gives both the Signal and its Complement is  \kBusBuffer
\newcommand{\kSfotsC}[3][u0]{
\def\kref{#1}
\def\kshiftX{#2}
\def\kshiftY{#3}
\def\pbu{pbu}
\def\pu{pu}
\def\kref{#1}
\kBuffer[#1]{#2}{#3}
\draw[thin](\kref\pbu)--(\kref\pu);
%\draw[thin](\kref\pnum\pin)node[]{$2$};
\foreach \n/\a in {pin/9,pout/8,u/10}{\draw[thin](\kref\pnum\n)node[]{\scriptsize$\a$};}
}
%%%%%%%%%%%%%%%%%%%%%%
%74126-D
% buffer that gives both the Signal and its Complement is  \kBusBuffer
\newcommand{\kSfotsD}[3][u0]{
\def\kref{#1}
\def\kshiftX{#2}
\def\kshiftY{#3}
\def\pbu{pbu}
\def\pnum{pnum}
\def\pu{pu}
\def\kref{#1}
\kBuffer[#1]{#2}{#3}
\draw[thin](\kref\pbu)--(\kref\pu);
%\draw[thin](\kref\pnum\pin)node[]{$2$};
\foreach \n/\a in {pin/12,pout/11,u/13}{\draw[thin](\kref\pnum\n)node[]{\scriptsize{$\a$}};}
}
%==============================
%numbering code 0=z, 1=o, 2=t, 3=T, 4=f, 5=F, 6=s, 7=S, 8=e, 9=n
%74LS189
\newcommand{\kSfoen}[3][u0]{
\def\kref{#1}
\def\kshiftX{#2}
\def\kshiftY{#3}
\def\knshift{0.07}
\kBOX[#1]{#2}{#3}{4}{5}{0.5}
\foreach \n/\lbl in {0/3,1/12,2/10,3/6,4/4}\draw[thin](\kref\a\pb\n)--(\kref\a\p\n) (\kref\a\pnum\n)node[]{\scriptsize{$\lbl$}};
\foreach \n/\lbl in {0/5,1/7,2/9,3/11}\draw[thin](\kref\b\pb\n)--(\kref\b\p\n) (\kref\b\pnum\n)node[]{\scriptsize{$\lbl$}};
\foreach \n/\lbl in {0/2,3/8,4/16}\draw[thin](\kref\c\pb\n)--(\kref\c\p\n) (\kref\c\pnum\n)node[]{\scriptsize{$\lbl$}};
\foreach \n/\lbl in {0/1,1/15,2/14,3/13}\draw[thin](\kref\d\pb\n)--(\kref\d\p\n) (\kref\d\pnum\n)node[]{\scriptsize{$\lbl$}};
\foreach \n in {0}\draw[thin](\kref\a\pb\n)++(-\knshift,0)node[ocirc]{};
\foreach \n in {0,1,2,3}\draw[thin](\kref\b\pb\n)++(0,-\knshift)node[ocirc]{};
\foreach \n in {0}\draw[thin](\kref\c\pb\n)++(\knshift,0)node[ocirc]{};
}
%==============================
%==============================
%numbering code 0=z, 1=o, 2=t, 3=T, 4=f, 5=F, 6=s, 7=S, 8=e, 9=n
%74LS219  Non-inverting 16*4 bit RAM
%pin 1 of uu7 can be accessed as uu7p1
\newcommand{\kSfton}[3][u0]{
\def\kref{#1}
\def\kshiftX{#2}
\def\kshiftY{#3}
\def\knshift{0.07}
\def\p{p}
\kBOX[#1]{#2}{#3}{4}{5}{0.5}
\foreach \n/\lbl in {0/3,4/12,3/10,2/6,1/4}\draw[thin](\kref\a\pb\n)--(\kref\a\p\n) (\kref\a\pnum\n)node[]{\scriptsize{$\lbl$}};
\foreach \n/\lbl in {3/5,2/7,1/9,0/11}\draw[thin](\kref\b\pb\n)--(\kref\b\p\n) (\kref\b\pnum\n)node[]{\scriptsize{$\lbl$}};
\foreach \n/\lbl in {0/2,3/8,4/16}\draw[thin](\kref\c\pb\n)--(\kref\c\p\n) (\kref\c\pnum\n)node[]{\scriptsize{$\lbl$}};
\foreach \n/\lbl in {3/1,2/15,1/14,0/13}\draw[thin](\kref\d\pb\n)--(\kref\d\p\n) (\kref\d\pnum\n)node[]{\scriptsize{$\lbl$}};
\foreach \n in {0}\draw[thin](\kref\a\pb\n)++(-\knshift,0)node[ocirc]{};
%\foreach \n in {0,1,2,3}\draw[thin](\kref\b\pb\n)++(0,-\knshift)node[ocirc]{};
\foreach \n in {0}\draw[thin](\kref\c\pb\n)++(\knshift,0)node[ocirc]{};
\foreach \n/\lbl in {0/3,1/4,2/6,3/10,4/12}\draw(\kref\a\p\n)coordinate(\kref\p\lbl);
\foreach \n/\lbl in {0/11,1/9,2/7,3/5}\draw(\kref\b\p\n)coordinate(\kref\p\lbl);
\foreach \n/\lbl in {0/2,3/8,4/16}\draw(\kref\c\p\n)coordinate(\kref\p\lbl);
\foreach \n/\lbl in {0/13,1/14,2/15,3/1}\draw(\kref\d\p\n)coordinate(\kref\p\lbl);
}
%==============================
%numbering code 0=z, 1=o, 2=t, 3=T, 4=f, 5=F, 6=s, 7=S, 8=e, 9=n
%74LS173
\newcommand{\kSfoSt}[3][u0]{
\def\kref{#1}
\def\kshiftX{#2}
\def\kshiftY{#3}
\kBOX[#1]{#2}{#3}{4}{5}{0.5}
\foreach \n/\lbl in {0/8,1/2,2/1,3/15,4/16}\draw[thin](\kref\a\pb\n)--(\kref\a\p\n) (\kref\a\pnum\n)node[]{\scriptsize{$\lbl$}}; %left
\foreach \n/\lbl in {0/3,1/4,2/5,3/6}\draw[thin](\kref\b\pb\n)--(\kref\b\p\n) (\kref\b\pnum\n)node[]{\scriptsize{$\lbl$}};       %bottom
\foreach \n/\lbl in {2/7,3/10,4/9}\draw[thin](\kref\c\pb\n)--(\kref\c\p\n) (\kref\c\pnum\n)node[]{\scriptsize{$\lbl$}};		%right
\foreach \n/\lbl in {0/14,1/13,2/12,3/11}\draw[thin](\kref\d\pb\n)--(\kref\d\p\n) (\kref\d\pnum\n)node[]{\scriptsize{$\lbl$}};
\foreach \n in {1,2}\draw[thin](\kref\a\pb\n)++(-\knshift,0)node[ocirc]{};
%\foreach \n in {0,1,2,3}\draw[thin](\kref\b\pb\n)++(0,-\knshift)node[ocirc]{};
\foreach \n in {3,4}\draw[thin](\kref\c\pb\n)++(\knshift,0)node[ocirc]{};
%\foreach \n in {0}\draw[thin](\kref\d\pb\n)++(0,\knshift)node[ocirc]{};
\def\n{2}
\draw[thin](\kref\c\pcd\n)--(\kref\c\pcm\n)--(\kref\c\pcu\n);
}
%==============================
%==============================
%numbering code 0=z, 1=o, 2=t, 3=T, 4=f, 5=F, 6=s, 7=S, 8=e, 9=n
%74LS173 rotated AntiClockWise
\newcommand{\kSfoStACW}[3][u0]{
\def\kref{#1}
\def\kshiftX{#2}
\def\kshiftY{#3}
\kBOX[#1]{#2}{#3}{5}{4}{0.5}
\foreach \n/\lbl in {0/14,1/13,2/12,3/11}\draw[thin](\kref\a\pb\n)--(\kref\a\p\n) (\kref\a\pnum\n)node[]{\scriptsize{$\lbl$}}; %left
\foreach \n/\lbl in {0/16,1/15,2/1,3/2,4/8}\draw[thin](\kref\b\pb\n)--(\kref\b\p\n) (\kref\b\pnum\n)node[]{\scriptsize{$\lbl$}}; %bot
\foreach \n/\lbl in {0/3,1/4,2/5,3/6}\draw[thin](\kref\c\pb\n)--(\kref\c\p\n) (\kref\c\pnum\n)node[]{\scriptsize{$\lbl$}};       %right
\foreach \n/\lbl in {0/9,1/10,2/7}\draw[thin](\kref\d\pb\n)--(\kref\d\p\n) (\kref\d\pnum\n)node[]{\scriptsize{$\lbl$}};		%top
\foreach \n in {2,3}\draw[thin](\kref\b\pb\n)++(0,-\knshift)node[ocirc]{};
%\foreach \n in {0,1,2,3}\draw[thin](\kref\b\pb\n)++(0,-\knshift)node[ocirc]{};
\foreach \n in {0,1}\draw[thin](\kref\d\pb\n)++(0,\knshift)node[ocirc]{};
%\foreach \n in {0}\draw[thin](\kref\d\pb\n)++(0,\knshift)node[ocirc]{};
\def\n{2}
\draw[thin](\kref\d\pcd\n)--(\kref\d\pcm\n)--(\kref\d\pcu\n);
}
%==================================
%==============================
%numbering code 0=z, 1=o, 2=t, 3=T, 4=f, 5=F, 6=s, 7=S, 8=e, 9=n
%74LS173
\newcommand{\kSfoStCW}[3][u0]{
\def\kref{#1}
\def\kshiftX{#2}
\def\kshiftY{#3}
\kBOX[#1]{#2}{#3}{5}{4}{0.5}
\foreach \n/\lbl in {0/6,1/5,2/4,3/3}\draw[thin](\kref\a\pb\n)--(\kref\a\p\n) (\kref\a\pnum\n)node[]{\scriptsize{$\lbl$}};  %left
\foreach \n/\lbl in {2/7,3/10,4/9}\draw[thin](\kref\b\pb\n)--(\kref\b\p\n) (\kref\b\pnum\n)node[]{\scriptsize{$\lbl$}};	%bot
\foreach \n/\lbl in {0/11,1/12,2/13,3/14}\draw[thin](\kref\c\pb\n)--(\kref\c\p\n) (\kref\c\pnum\n)node[]{\scriptsize{$\lbl$}}; %right
\foreach \n/\lbl in {0/15,1/8,2/2,3/1,4/16}\draw[thin](\kref\d\pb\n)--(\kref\d\p\n) (\kref\d\pnum\n)node[]{\scriptsize{$\lbl$}}; %top
%\foreach \n in {1,2}\draw[thin](\kref\a\pb\n)++(-\knshift,0)node[ocirc]{};
\foreach \n in {3,4}\draw[thin](\kref\b\pb\n)++(0,-\knshift)node[ocirc]{};
%\foreach \n in {3,3}\draw[thin](\kref\c\pb\n)++(\knshift,0)node[ocirc]{};
\foreach \n in {2,3}\draw[thin](\kref\d\pb\n)++(0,\knshift)node[ocirc]{};
\def\n{2}
\draw[thin](\kref\b\pcd\n)--(\kref\b\pcm\n)--(\kref\b\pcu\n);
}
%==============================
%==============================
%numbering code 0=z, 1=o, 2=t, 3=T, 4=f, 5=F, 6=s, 7=S, 8=e, 9=n
%74LS83
\newcommand{\kSfet}[3][u0]{
\def\kref{#1}
\def\kshiftX{#2}
\def\kshiftY{#3}
\kBOX[#1]{#2}{#3}{8}{3}{0.5}
\foreach \n/\lbl in {0/12,1/14,2/5}\draw[thin](\kref\a\pb\n)--(\kref\a\p\n) (\kref\a\pnum\n)node[]{\scriptsize{$\lbl$}}; %left
\foreach \n/\lbl in {3/15,4/2,5/6,6/9}\draw[thin](\kref\b\pb\n)--(\kref\b\p\n) (\kref\b\pnum\n)node[]{\scriptsize{$\lbl$}};       %bottom
\foreach \n/\lbl in {1/13}\draw[thin](\kref\c\pb\n)--(\kref\c\p\n) (\kref\c\pnum\n)node[]{\scriptsize{$\lbl$}};		%right
\foreach \n/\lbl in {0/1,1/3,2/8,3/10,4/16,5/4,6/7,7/11}\draw[thin](\kref\d\pb\n)--(\kref\d\p\n) (\kref\d\pnum\n)node[]{\scriptsize{$\lbl$}};  %upper pins
%\foreach \n in {1,2}\draw[thin](\kref\a\pb\n)++(-\knshift,0)node[ocirc]{};
%\foreach \n in {0,1,2,3}\draw[thin](\kref\b\pb\n)++(0,-\knshift)node[ocirc]{};
%\foreach \n in {3,4}\draw[thin](\kref\c\pb\n)++(\knshift,0)node[ocirc]{};
%\foreach \n in {0}\draw[thin](\kref\d\pb\n)++(0,\knshift)node[ocirc]{};
}
%==============================
%numbering code 0=z, 1=o, 2=t, 3=T, 4=f, 5=F, 6=s, 7=S, 8=e, 9=n
%74LS157
\newcommand{\kSfoFS}[3][u0]{
\def\kref{#1}
\def\kshiftX{#2}
\def\kshiftY{#3}
\kBOX[#1]{#2}{#3}{8}{5}{0.5}
\foreach \n/\lbl in {0/15,1/8,2/1,4/16}\draw[thin](\kref\a\pb\n)--(\kref\a\p\n) (\kref\a\pnum\n)node[]{\scriptsize{$\lbl$}}; %left
\foreach \n/\lbl in {4/12,5/9,6/7,7/4}\draw[thin](\kref\b\pb\n)--(\kref\b\p\n) (\kref\b\pnum\n)node[]{\scriptsize{$\lbl$}};    %bottom
%\foreach \n/\lbl in {0/5,2/3}\draw[thin](\kref\c\pb\n)--(\kref\c\p\n) (\kref\c\pnum\n)node[]{\scriptsize{$\lbl$}};		%right
\foreach \n/\lbl in {0/14,1/11,2/5,3/2,4/13,5/10,6/6,7/3}\draw[thin](\kref\d\pb\n)--(\kref\d\p\n) (\kref\d\pnum\n)node[]{\scriptsize{$\lbl$}};
\foreach \n in {0}\draw[thin](\kref\a\pb\n)++(-\knshift,0)node[ocirc]{};
%\foreach \n in {0,1,2,3}\draw[thin](\kref\b\pb\n)++(0,-\knshift)node[ocirc]{};
%\foreach \n in {3,4}\draw[thin](\kref\c\pb\n)++(\knshift,0)node[ocirc]{};
%\foreach \n in {0}\draw[thin](\kref\d\pb\n)++(0,\knshift)node[ocirc]{};
}
%==============================
%numbering code 0=z, 1=o, 2=t, 3=T, 4=f, 5=F, 6=s, 7=S, 8=e, 9=n
%NE555
\newcommand{\kFFF}[3][u0]{
\def\kref{#1}
\def\kshiftX{#2}
\def\kshiftY{#3}
\kBOX[#1]{#2}{#3}{3}{3}{0.5}
\foreach \n/\lbl in {0/6,2/7}\draw[thin](\kref\a\pb\n)--(\kref\a\p\n)coordinate(\kref\p\lbl) (\kref\a\pnum\n)node[]{\scriptsize{$\lbl$}}; %left
\foreach \n/\lbl in {0/2,2/1}\draw[thin](\kref\b\pb\n)--(\kref\b\p\n)coordinate(\kref\p\lbl) (\kref\b\pnum\n)node[]{\scriptsize{$\lbl$}};       %bottom
\foreach \n/\lbl in {0/5,2/3}\draw[thin](\kref\c\pb\n)--(\kref\c\p\n)coordinate(\kref\p\lbl) (\kref\c\pnum\n)node[]{\scriptsize{$\lbl$}};		%right
\foreach \n/\lbl in {0/8,2/4}\draw[thin](\kref\d\pb\n)--(\kref\d\p\n)coordinate(\kref\p\lbl) (\kref\d\pnum\n)node[]{\scriptsize{$\lbl$}};
%\foreach \n in {1,2}\draw[thin](\kref\a\pb\n)++(-\knshift,0)node[ocirc]{};
%\foreach \n in {0,1,2,3}\draw[thin](\kref\b\pb\n)++(0,-\knshift)node[ocirc]{};
%\foreach \n in {3,4}\draw[thin](\kref\c\pb\n)++(\knshift,0)node[ocirc]{};
%\foreach \n in {0}\draw[thin](\kref\d\pb\n)++(0,\knshift)node[ocirc]{};
}
%==========================
%==============================
%numbering code 0=z, 1=o, 2=t, 3=T, 4=f, 5=F, 6=s, 7=S, 8=e, 9=n
%74LS86
\newcommand{\kSfes}[3][u0]{
\def\kref{#1}
\def\kshiftX{#2}
\def\kshiftY{#3}
\kBOX[#1]{#2}{#3}{3}{8}{0.5}
\foreach \n/\lbl in {0/1,1/2,2/4,3/5,4/9,5/10,6/12,7/13} \draw[thin](\kref\a\pb\n)--(\kref\a\p\n) (\kref\a\pnum\n)node[]{\scriptsize{$\lbl$}}; %left
%\foreach \n/\lbl in {0/2,2/1}\draw[thin](\kref\b\pb\n)--(\kref\b\p\n) (\kref\b\pnum\n)node[]{\scriptsize{$\lbl$}};       %bottom
\foreach \n/\lbl in {2/3,3/6,4/8,5/9}\draw[thin](\kref\c\pb\n)--(\kref\c\p\n) (\kref\c\pnum\n)node[]{\scriptsize{$\lbl$}};		%right
%\foreach \n/\lbl in {0/8,2/4}\draw[thin](\kref\d\pb\n)--(\kref\d\p\n) (\kref\d\pnum\n)node[]{\scriptsize{$\lbl$}};
%\foreach \n in {1,2}\draw[thin](\kref\a\pb\n)++(-\knshift,0)node[ocirc]{};
%\foreach \n in {0,1,2,3}\draw[thin](\kref\b\pb\n)++(0,-\knshift)node[ocirc]{};
%\foreach \n in {3,4}\draw[thin](\kref\c\pb\n)++(\knshift,0)node[ocirc]{};
%\foreach \n in {0}\draw[thin](\kref\d\pb\n)++(0,\knshift)node[ocirc]{};
}
%==============================
%RotatedClockWise
% buffer that gives both the Signal and its Complement is  \kBusBuffer
\newcommand{\kBufferRCW}[2][u0]{
\def\kref{#1}
\def\kLoc{#2}
\def\p{p}			%pin tip
\def\pb{pb}			%pin base
\def\pl{pl}			%pin label
\def\pn{pn}			%pin 'ocirc'
\def\pnum{pnum}			%pin number  
\def\pin{pin}			%pin tip
\def\pout{pout}			%pin tip
\def\pu{u}			%pin tip
\def\pd{d}			%pin tip
\pgfmathsetmacro{\klshift}{0.25}
\pgfmathsetmacro{\knshift}{0.07}
\pgfmathsetmacro{\knumshift}{0.15}  %half of pin length
\pgfmathsetmacro{\kpin}{0.30}
\pgfmathsetmacro{\kpsep}{0.40}			%pin to pin distance
\pgfmathsetmacro{\kulV}{0.40}			%edge clearance along vertical edge
\pgfmathsetmacro{\kulH}{0.50}
\pgfmathsetmacro{\kdimX}{2*\kulH+0*\kpsep}
\pgfmathsetmacro{\kdimY}{\kdimX}		%two spaces between 3 pins
\draw[thin](\kLoc)coordinate(\kref\pin)--++(0,-\kpin)coordinate(\kref\pb\pin);
\draw(\kref\pb\pin)[thick]++(-\kdimY/2,0)coordinate(aa)--++(\kdimY,0)coordinate(bb)--++(-\kdimY/2,-\kdimX)coordinate(cc)coordinate(\kref\pb\pout)--++(-\kdimY/2,\kdimX);
\draw[thin](\kref\pb\pout)--++(0,-\kpin)coordinate(\kref\pout);
\draw($(bb)!0.5!(cc)$)coordinate(\kref\pb\pu)+(0.07,0)coordinate(\kref\pn\pu)+(\kpin,0)coordinate(\kref\pu)
+(\knumshift,-0.5*\kpin)coordinate(\kref\pnum\pu);
\draw($(aa)!0.5!(cc)$)coordinate(\kref\pb\pd)+(-0.07,0)coordinate(\kref\pn\pd)+(-\kpin,0)coordinate(\kref\pd)
+(\knumshift,-0.5*\kpin)coordinate(\kref\pnum\pd);
\draw(\kref\pb\pin)+(0,0.07)coordinate(\kref\pn\pin)+(0.5*\kpsep,\knumshift)coordinate(\kref\pnum\pin);
\draw(\kref\pb\pout)+(0,-0.07)coordinate(\kref\pn\pout)+(-0.5*\kpsep,-\knumshift)coordinate(\kref\pnum\pout);
}
%=====================================
%%%%%%%%%%%%%%%%%%%%%%
%numbering code 0=z, 1=o, 2=t, 3=T, 4=f, 5=F, 6=s, 7=S, 8=e, 9=n
%74126-A
% buffer that gives both the Signal and its Complement is  \kBusBuffer
\newcommand{\kSfotsRCWA}[2][u0]{
\def\kref{#1}
\def\kLoc{#2}
\def\ksec{A}
\def\pbu{pbu}
\def\pu{pu}
\def\p{p}			%pin tip
\def\kref{#1}
\kBufferRCW[#1]{#2}
\draw[thin](\kref\pbu)--(\kref\pu);
%\draw[thin](\kref\pnum\pin)node[]{$2$};
\foreach \n/\a in {pin/2,pout/3,u/1}{\draw[thin](\kref\pnum\n)node[]{\scriptsize$\a$};}
\foreach \n/\a in {pin/2,pout/3,u/1}{\draw[thin](\kref\n)coordinate(\kref\p\a);}
\path[thin](\kref\pb\pin)--(\kref\pb\pout)coordinate[pos=0.25](\kref\ksec-center);
}
%%%%%%%%%%%%%%%%%%%%%%
%%%%%%%%%%%%%%%%%%%%%%
%74126-B
% buffer that gives both the Signal and its Complement is  \kBusBuffer
\newcommand{\kSfotsRCWB}[2][u0]{
\def\kref{#1}
\def\kLoc{#2}
\def\ksec{B}
\def\pbu{pbu}
\def\pu{pu}
\def\p{p}			%pin tip
\def\kref{#1}
\kBufferRCW[#1]{#2}
\draw[thin](\kref\pbu)--(\kref\pu);
%\draw[thin](\kref\pnum\pin)node[]{$2$};
\foreach \n/\a in {pin/5,pout/6,u/4}{\draw[thin](\kref\pnum\n)node[]{\scriptsize$\a$};}
\foreach \n/\a in {pin/5,pout/6,u/4}{\draw[thin](\kref\n)coordinate(\kref\p\a);}
\path[thin](\kref\pb\pin)--(\kref\pb\pout)coordinate[pos=0.25](\kref\ksec-center);
}
%%%%%%%%%%%%%%%%%%%%%%
%74126-C
% buffer that gives both the Signal and its Complement is  \kBusBuffer
\newcommand{\kSfotsRCWC}[2][u0]{
\def\kref{#1}
\def\kLoc{#2}
\def\ksec{C}
\def\pbu{pbu}
\def\pu{pu}
\def\p{p}			%pin tip
\def\kref{#1}
\kBufferRCW[#1]{#2}
\draw[thin](\kref\pbu)--(\kref\pu);
%\draw[thin](\kref\pnum\pin)node[]{$2$};
\foreach \n/\a in {pin/9,pout/8,u/10}{\draw[thin](\kref\pnum\n)node[]{\scriptsize$\a$};}
\foreach \n/\a in {pin/9,pout/8,u/10}{\draw[thin](\kref\n)coordinate(\kref\p\a);}
\path[thin](\kref\pb\pin)--(\kref\pb\pout)coordinate[pos=0.25](\kref\ksec-center);
}
%%%%%%%%%%%%%%%%%%%%%%
%74126-D
% buffer that gives both the Signal and its Complement is  \kBusBuffer
\newcommand{\kSfotsRCWD}[2][u0]{
\def\kref{#1}
\def\kLoc{#2}
\def\ksec{D}
\def\pbu{pbu}
\def\pnum{pnum}
\def\pu{pu}
\def\p{p}			%pin tip
\def\kref{#1}
\kBufferRCW[#1]{#2}
\draw[thin](\kref\pbu)--(\kref\pu);
%\draw[thin](\kref\pnum\pin)node[]{$2$};
\foreach \n/\a in {pin/12,pout/11,u/13}{\draw[thin](\kref\pnum\n)node[]{\scriptsize{$\a$}};}
\foreach \n/\a in {pin/12,pout/11,u/13}{\draw[thin](\kref\n)coordinate(\kref\p\a);}
\path[thin](\kref\pb\pin)--(\kref\pb\pout)coordinate[pos=0.25](\kref\ksec-center);
}
%==============================


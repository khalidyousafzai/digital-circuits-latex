\begin{figure}
\centering
\begin{tikzpicture}
 %the flipflop def can be inside the node
\tikzset{flipflop SR/.style={flipflop, 
flipflop def={width=0.6,pin spacing=0.4,font=\normalsize,t1={\,S},t3={\, R},t6={Q\,} ,t4={\ctikztextnot{Q}\,}}}}
\tikzset{flipflop D/.style={flipflop, 
flipflop def={width=0.6,pin spacing=0.4,font=\normalsize,t1={\,D},t2=C,c2=1,n4=1,t4={\ctikztextnot{Q}\,} ,t6={Q\,} ,}}}
\node [flipflop D,label={north:u0}] (u0) {};
\node [flipflop SR,flipflop def={n4=1,font=\normalsize},right=of u0,label={[label distance=1cm]90:uu}] (u1) {}; 
\node [flipflop SR,flipflop def={n4=1,font=\normalsize},right=of u1,] (u2) {u2};
\end{tikzpicture}
\begin{tikzpicture}[node distance=2cm]
\tikzset{dot/.style = {
    shape  = circle,
    draw   = black,
    fill = black,
    minimum size = 0.2cm}}
\tikzset{squarenode/.style = {
        shape  = rectangle,
        draw   = black,
        minimum height = 2cm,
        minimum width  = 1cm}}
\pgfmathsetmacro{\a}{1}
\pgfmathsetmacro{\b}{2}
\pgfmathsetmacro{\c}{3}
\pgfmathsetmacro{\d}{4}
\pgfmathsetmacro{\e}{5}
\pgfmathsetmacro{\f}{6}
\node[squarenode](u\a){u\a};
\node[squarenode,right=of u\a](u\b){u\b};
\coordinate (ua\a) at ($(u\a.north west)!0.25!(u\a.south west)$);
\coordinate (ua\b) at ($(u\a.north west)!0.5!(u\a.south west)$);
\coordinate (ua\c) at ($(u\a.north west)!0.75!(u\a.south west)$);
\coordinate (ua\d) at ($(u\a.south east)!0.25!(u\a.north east)$);
\coordinate (ua\e) at ($(u\a.south east)!0.5!(u\a.north east)$);
\coordinate (ua\f) at ($(u\a.south east)!0.75!(u\a.north east)$);

\coordinate (ub\a) at ($(u\b.north west)!0.25!(u\b.south west)$);
\draw(ua\a)node[dot]{};
\draw(ua\b)node[dot]{};
\draw(ua\c)node[dot]{};
\draw(ua\d)node[dot]{};

\draw(ub\a)node[dot]{};
\end{tikzpicture}
\begin{circuitikz}
\ctikzset{multipoles/thickness=3}
\ctikzset{multipoles/dipchip/width=1,multipoles/dipchip/pin spacing=0.2}
\draw (0,0) node[dipchip, num pins=10, hide numbers, no topmark, external pins width=0](C){Block};
\node [right, font=\tiny] at (C.bpin 1) {RST};
\node [right, font=\tiny] at (C.bpin 2) {IN1};
\node [right, font=\tiny] at (C.bpin 4) {/IN2};
\node [left, font=\tiny] at (C.bpin 8) {OUT};
\draw (C.bpin 2) -- ++(-0.5,0) coordinate(extpin);
\node [ocirc, anchor=0](notin2) at (C.bpin 4) {};
\draw (notin2.180) -- (C.bpin 4 -| extpin);
\draw (C.bpin 8) to[short,-o] ++(0.5,0);
\draw (C.bpin 5) ++(0,0.1) -- ++(0.1,-0.1)node[right, font=\tiny]{CLK} -- ++(-0.1,-0.1);
\end{circuitikz}

\end{figure}


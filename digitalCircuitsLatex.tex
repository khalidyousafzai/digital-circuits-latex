\documentclass[leqno,b5paper]{book}
%\documentclass[leqno]{book} %the geometry package defines the paper size

% xelatex --shell-escape calculusAndAnalyticGeometry   %must run to avail gnuplot functionality. gnuplot must be installed.
\typeout{to avail gnuplot functionality must run   "xelatex --shell-escape calculusAndAnalyticGeometry"}

%===================
%testing tikz inside book
\usepackage[arrowmos,oldvoltagedirection]{circuitikz}
\usepackage{pgfplotstable}
\usepackage{pgfplots}

\usepackage{tikz-3dplot}
\usepackage{tikz-timing}[2014/10/29]
\usetikztiminglibrary[rising arrows]{clockarrows}
\usetikztiminglibrary[]{counters}
%\usetikztiminglibrary{advnodes}
\pgfplotsset{compat=newest,}
\usepgfplotslibrary{units}
%\pgfplotsset{compat=1.9}
\usepgfplotslibrary{polar}
\usepackage{ifdraft}
\usepackage{multicol}                                        %for getting multicolumn itemize, enumerate etc
\usepackage{enumerate}                   %for getting better auto numbering in enumerate


%pathmorphing gives the ripples like look
\usetikzlibrary{shapes,snakes,3d,shadings,fadings,intersections,calc,decorations.markings,
decorations.pathreplacing,external,shapes.misc,decorations.pathmorphing,patterns,circuits.logic.US,circuits.logic.IEC, positioning}

%\tikzexternalize[mode=list and make] %disable to generate figures
%\tikzexternaldisable  %enables figures after this command. put this is the tex file

\usepackage{scrextend}   % \begin{labeling}{}  \end{labeling}   just like Description
%for vertical spacing in tables use\Tstrut and \Bstrut  
\newcommand\Tstrut{\rule{0pt}{2.6ex}}       % Top strut
\newcommand\Bstrut{\rule[-1.2ex]{0pt}{0pt}} % Bottom strut
%\renewcommand{\arraystretch}{2} give this before tabular for vertical space 

\definecolor{lgray}{cmyk}{0,0,0,0.2}
\colorlet{llgray}{gray!10}
\definecolor{dgray}{cmyk}{0,0,0,0.7}
\definecolor{inactivePart}{cmyk}{0,0,0,0.2}		%used to dim the inactive circuits in digital book
%========================
%\usepackage[hidelinks]{hyperref}  %used in machines book but not here
\usepackage{./tex/khalidUrduBooks}                     %my sty file
\usepackage{amsbsy} %for bold Poynting, lessgtr symbol
\usepackage{mathrsfs}   %for Poynting symbol
\usepackage{IEEEtrantools}
\usepackage{multirow}   %for multiple row cells in a table column
\usepackage[misc]{ifsym} 

\newcommand{\tikzmark}[1]{%
  \tikz[overlay,remember picture] \node (#1) {};}

 
\newcommand*{\kStrokeOne}{|}                                      %tally marks (statistics)
\newcommand*{\kStrokeTwo}{|\!|}
\newcommand*{\kStrokeThree}{|\!|\!|}
\newcommand*{\kStrokeFour}{|\!|\!|\!|}
\newcommand*{\kStrokeFive}{\cancel\kStrokeFour}


%\renewcommand{\arraystretch}{2}   %space in tables and arrays


\sisetup{
math-micro=\textup{µ},text-micro=µ,math-ohm  =\upOmega,text-degree=°,
  math-degree=\textup{°}}   %with mathpazo this is needed else must not be here. now micro is smaller
\DeclareSIUnit \var {var}    %used in electric circuits volt-ampere-reactive

%\input longdiv.tex		%this is a package but doesnot load with usepackage. it writes pen and paper long division
\usepackage{polynom}		%this writes pen and paper long division for polynomials
\usepgfplotslibrary{fillbetween}
\usepackage[]{gnuplottex}
%\usepackage{xlop}		%excellent package;  automatically gives add, sub, mul, div. can be wrapped in align.
%============================
%%%%%%%%%%%%%%%%%%%%%%%%%%%%%%%%%%%%%%%%%
\usepackage{newfile}                                                                  %writing answers to the end of the book
\newwrite\tempfile
\immediate\openout\tempfile=answer.tex
\newcommand*\wf[1]{\immediate\write\tempfile{#1}}
%
%%the  \اصطلاح and \حاشیہب have been disabled  in "myUrduCommandsCalculus.tex" and instead 
%%the following is used. this retains their original definitions and writes them to a "technicalTerms.tex" text file too.
%\newwrite\tempfileTerms				 %collecting technical terms. uses newfile package
%\immediate\openout\tempfileTerms=technicalTerms.tex
%\newcommand*\wTechTerms[1]{\immediate\write\tempfileTerms{#1}}
%\newcommand*{\اصطلاح}[1]{\wTechTerms{\unexpanded{ur, #1 ,}}{\urduTechTermsfont{#1}}}
%\newcommand*{\حاشیہب}[1]{ \wTechTerms{\unexpanded{en, #1,}}{\raggedright{\footnote{\textenglish{#1}}}}}
%%%%%%%%%%%%%%%%%%%%%%%%%%%%%%%%%%%%%%%%%%%%
%============================                                                            
%=================================
%=================================

\newcommand{\krightharpoonup}[1]{\overset{\rightharpoonup}{\rule{0pt}{.9ex}\smash{#1}}}
\newcommand{\kleftharpoonup}[1]{\overset{\leftharpoonup}{\rule{0pt}{.9ex}\smash{#1}}}


\newcommand{\koverleftharp}[1]{\overharp{\leftharpoonup}{#1}{.7}}
\newcommand{\koverrightharp}[1]{\overharp{\rightharpoonup}{#1}{.7}}
\newcommand{\koverleftharpdown}[1]{\overharp{\leftharpoondown}{#1}{.9}}
\newcommand{\koverrightharpdown}[1]{\overharp{\rightharpoondown}{#1}{.9}}
\newcommand{\kunderleftharp}[1]{\overharp{\leftharpoonup}{#1}{-1}}
\newcommand{\kunderrightharp}[1]{\overharp{\rightharpoonup}{#1}{-1}}
\newcommand{\kunderleftharpdown}[1]{\overharp{\leftharpoondown}{#1}{-.8}}
\newcommand{\kunderrightharpdown}[1]{\overharp{\rightharpoondown}{#1}{-.8}}

\newlength{\argwd}  \newlength{\arght}%-Two variables
\newcommand{\overharp}[3]{%        -The command name
  \settowidth{\argwd}{#2}\settoheight{\arght}{#2}%
  %                                    -Set the variables
  \raisebox{#3\arght}{%                -Put the harp 6/10 of a line higher
    \makebox[0pt][r]{%                 -Put everything in a box ;           corrected by me for Flush Right
      \resizebox{\argwd}{.8\arght}{\!$#1$}% 
      %                                -Set harp to right length
    }%
  }%
#2}%                                   -Print the argument

                       %testing harpoons


%this file describes urdu commands for the commonly used english latex commands

%chapter, section etc
%  \newcommand*{newcommand}[arguments]{actual command}
\newcommand*{\باب}[1]{\chapter{#1}}                                      %defining commonly used commands
\newcommand*{\حصہ}[1]{\section{#1}}
\newcommand*{\جزوحصہ}[1]{\subsection{#1}}
\newcommand*{\جزوجزوحصہ}[1]{\subsubsection{#1}}

\newcommand*{\بابء}[1]{\chapter*{#1}}                                      %defining commonly used commands
\newcommand*{\حصہء}[1]{\section*{#1}}
\newcommand*{\جزوحصہء}[1]{\subsection*{#1}}
\newcommand*{\جزوجزوحصہء}[1]{\subsubsection*{#1}}


%english text in urdu mode
\newcommand*{\تحریر}[1]{\textenglish{#1}}	% english text in urduMode
%\newcommand*{\موٹا}[1]{\textbf{#1}}
%\newcommand*{\ترچھا}[1]{{\textit{#1}}}
\newcommand*{\موٹا}[1]{{\urduTechTermsfont{#1}}}
\newcommand*{\ترچھا}[1]{{\small{#1}}}

%%\newcommand*{\اصطلاح}[1]{{\color{red}{#1}}}   %colours spills to next word when there is index or footnote entry with the word
%%\newcommand{\اصطلاح}[1]{{\urdufontBig{#1}}}

\newcommand{\اصطلاح}[1]{{\urduTechTermsfont{#1}}}  %%%moved to main file
\newcommand{\اصطلاحء}[1]{{\small{#1}}} %highlighted where  chance of mixup with  regular urdu meaning


%end commands cannot be redefined and as such these two are not usable
\providecommand*{\ابتدا}[1]{\begin{#1}}
\providecommand*{\انتہا}[1]{\end{#1}}

%include and input directives for adding external files into the main document 
\newcommand*{\بشمول}[1]{\includeonly{#1}}
\newcommand*{\شامل}[1]{\include{#1}}
\newcommand*{\داخل}[1]{\input{#1}}

%to use extra latex packages
\newcommand*{\استعمال}[1]{\usepackage{#1}}

%footnotes and indexes
\newcommand*{\حاشیہب}[1]{{\raggedright{\footnote{\textenglish{#1}}}} }          %  moved to main file. footnote to the left hand side
\newcommand*{\حاشیہد}[1]{{\raggedleft{\footnote{#1}}}}
\newcommand*{\حاشیہط}[1]{\marginpar{#1}}

\newcommand*{\فرہنگ}[1]{\index{#1}}

%references and labels
\newcommand*{\شناخت}[1]{\label{#1}}
\newcommand*{\حوالہ}[1]{\textenglish{\ref{#1}}}  %corrects the ref number flipping
\newcommand*{\حوالہء}[1]{#1}			%dummy to enter the figure number directly while the figure is not ready
\newcommand*{\حوالہصفحہ}[1]{\pageref{#1}}

%counters
\newcommand*{\فاصلہ}{\vspace*{10mm}}
\newcommand*{\فاصلہء}{\quad}

%itemize, bullets and numbered items   
\newcommand*{\اشیاء}{itemize}                               %used in   \begin{itemize}
\newcommand*{\شے}[1]{\item {#1}}			%used in    \item, \description
%description
\newcommand*{\جزو}[1]{\item[#1]}                      %used in \begin{description}
%maths commands
%\newcommand*{\عددی}[1]{\: \ensuremath{#1} \:} % in-line math & inside math mode
%\newcommand*{\عددیء}[1]{\ensuremath{#1}}
\newcommand*{\عددی}[1]{\:\({#1}\)\:} % in-line math & inside math mode; {#1} ensures no splitting of inline
\newcommand*{\عددیء}[1]{\({#1}\)}
\newcommand*{\سمتیہ}[1]{\ensuremath{{\bf{#1}}}}
\newcommand*{\سمتیازیرنوشت}[2]{\ensuremath{{\boldsymbol{#1}}_{\textup{#2}}}}

\newcommand*{\ضرب}{\time}					%multiplication symbol
\newcommand*{\نکطہد}{\cdot}
\newcommand*{\نقطے}{\ensuremath{\cdots}}

\newcommand*{\زیرنوشت}[3]{\: \ensuremath{{#1_{#2 \textrm {#3}}}} \:}   %english+urdu subscript \زیرنوشت{V}{CE}{غیرافزائندہ}
\newcommand*{\سیدھازیرنوشت}[2]{\: \ensuremath{{#1_{\textup{#2}}}} \:} %RC

\newcommand*{\قریب}[1]{\mbox{#1}}  %dissallows splitting along two lines
\newcommand{\سن}[1]{؁\,\ensuremath{#1}}
\newcommand{\زور}[1]{\aemph{#1}} %overline urdu text to emphasize

\newcommand{\kQuote}[1]{“\ignorespaces#1\ignorespaces”}
\newcommand{\kquote}[1]{‘\ignorespaces#1\ignorespaces’}
\newcommand{\قولء}[1]{’\ignorespaces#1\ignorespaces‘}
\newcommand{\قول}[1]{”\ignorespaces#1\ignorespaces“}

%inline encloses a letter in a circle
\newcommand*{\دائرہبند}[1]{%
  \begin{tikzpicture}[baseline=(char.base)]
    \node[draw,circle,inner sep=1pt](char) {\(#1\)};
  \end{tikzpicture}}


\renewcommand{\indexname}{فرہنگ}        %does nothing here. must be placed within begin{urdufont} environment to be the last to take effect 
%===============================


%=======================================


%================
%my environments
%================

%environment for examples مثال
%\newcounter{examplecounter}[section]
%\renewcommand{\theexamplecounter}{\arabic{examplecounter}}
\newcounter{examplecounter}[chapter]
\renewcommand{\theexamplecounter}{\thechapter.\arabic{examplecounter}}

\newenvironment*{مثال}
{\par\noindent\ignorespaces    مثال \refstepcounter{examplecounter} \theexamplecounter :\quad}%
{\hfill\qedsymbol  \vspace{\baselineskip}\par}
%{\noindent\ignorespaces \vspace{\baselineskip} \hrule \vspace{\baselineskip}  مثال \refstepcounter{examplecounter} \theexamplecounter :}%
%{\par\noindent \hrule  \vspace{\baselineskip}}

%------
%practice problems environment مشق

%\newcounter{practicecounter}[section]                              %practice here means مشق
%\renewcommand{\thepracticecounter}{\arabic{practicecounter}}
\newcounter{practicecounter}[chapter]                              %practice here means مشق
\renewcommand{\thepracticecounter}{\thechapter.\arabic{practicecounter}}

\newenvironment*{مشق}
{\par\noindent\ignorespaces \vspace{\baselineskip} \hrule \vspace{\baselineskip} مشق \refstepcounter{practicecounter} \thepracticecounter :\quad}%
{\par\noindent \hrule  \vspace{\baselineskip}}
%---------

%end of chapter questions environment سوال

\newcounter{questioncounter}[chapter]                           %for reseting at every section. to be used only during writing stage
%\renewcommand{\thequestioncounter}{\arabic{questioncounter}}
%\newcounter{questioncounter}[chapter]						%when book is finished, use this instead of the above
\renewcommand{\thequestioncounter}{\thechapter.\arabic{questioncounter}}

\newenvironment*{سوال}				
{\noindent\ignorespaces  سوال \refstepcounter{questioncounter} \thequestioncounter :\quad}%
{\par\noindent\ignorespaces }
%--------------------

%defining a LAW   قانون

%\newcounter{lawcounter}[section]
%\renewcommand{\thelawcounter}{\arabic{lawcounter}}
\newcounter{lawcounter}[chapter]
\renewcommand{\thelawcounter}{\thechapter.\arabic{lawcounter}}

\newenvironment*{قانون}				
{\par\medskip \refstepcounter{lawcounter} \quad\nopagebreak}%
{\par\hfill\qedsymbol  \vspace{\baselineskip}\par }
%{\par\medskip }
%--------------------
%--------------------

%defining a THEOREM   مسئلہ

%\newcounter{kthcounter}[section]
%\renewcommand{\thekthcounter}{\arabic{kthcounter}}
\newcounter{kthcounter}[chapter]
\renewcommand{\thekthcounter}{\thechapter.\arabic{kthcounter}}

\newenvironment*{مسئلہ}				
{\par\noindent\ignorespaces  مسئلہ \refstepcounter{kthcounter} \thekthcounter :\quad}%
{\par\noindent }
%--------------------
%--------------------

%defining a Proof   ثبوت

%\newcounter{kprcounter}[chapter]
%\renewcommand{\thekprcounter}{\thechapter.\arabic{kprcounter}}

\newenvironment*{ثبوت}				
{\noindent\ignorespaces  ثبوت :\quad}%
{\par\hfill\qedsymbol  \vspace{\baselineskip}\par }
%{\par\noindent\qedsymbol  \vspace{\baselineskip} }
%--------------------
%--------------------
%++++++++++++++++++++++++++++++

%--------------------

%defining a COROLLARY   ضمنی نتیجہ


\newcounter{kcocounter}[chapter]
\renewcommand{\thekcocounter}{\thechapter.\arabic{kcocounter}}

\newenvironment*{ضمنی نتیجہ}				
{\par\noindent\ignorespaces  ضمنی نتیجہ \refstepcounter{kcocounter} \thekcocounter :\quad}%
{\par\noindent }
%--------------------
%--------------------

%defining a Proof   ثبوت ضمنی نتیجہ

%\newcounter{kprcocounter}[chapter]
%\renewcommand{\thekprcocounter}{\thechapter.\arabic{kprcocounter}}

\newenvironment*{ثبوت ضمنی نتیجہ}				
{\noindent\ignorespaces  ثبوت ضمنی نتیجہ :\quad}%
{\par\hfill\qedsymbol  \vspace{\baselineskip}\par }
%{\par\noindent\qedsymbol  \vspace{\baselineskip} }
%--------------------
%--------------------
%++++++++++++++++++++++++++++++++++++++++++++

%defining a Definition تعریف

%\newcounter{kdfcounter}[chapter]
%\renewcommand{\thedfrcounter}{\thechapter.\arabic{kdfcounter}}

\newenvironment*{تعریف}				
{\par\noindent\ignorespaces  تعریف :\quad}%
{\par\hfill\qedsymbol  \vspace{\baselineskip}\par }
%{\par\noindent }
%--------------------
%----------------------------
%defining a Definition تعریفات

%\newcounter{kdfcounter}[chapter]
%\renewcommand{\thedfrcounter}{\thechapter.\arabic{kdfcounter}}

\newenvironment*{تعریفات}				
{\par\noindent\ignorespaces  تعریفات :\quad}%
{\par\hfill\qedsymbol  \vspace{\baselineskip}\par }
%{\par\noindent }
%--------------------
%--------------------

%defining a Definition مفروضہ

%\newcounter{kAscounter}[chapter]
%\renewcommand{\theAsrcounter}{\thechapter.\arabic{kAscounter}}

\newenvironment*{مفروضہ}				
{\par\noindent\ignorespaces  مفروضہ \quad}%
{\par\hfill\qedsymbol  \vspace{\baselineskip}\par }
%{\par\noindent }
%--------------------
%--------------------

%defining a RULE   قاعدہ

%\newcounter{krucounter}[section]
%\renewcommand{\thekrucounter}{\arabic{krucounter}}
\newcounter{krucounter}[chapter]
\renewcommand{\thekrucounter}{\thechapter.\arabic{krucounter}}

\newenvironment*{قاعدہ}				
{\par\noindent\ignorespaces  قاعدہ \refstepcounter{krucounter} \thekrucounter :\quad}%
{\par\noindent }
%--------------------
%--------------------

%defining a Proof   ثبوت قاعدہ

%\newcounter{kprRcounter}[chapter]
%\renewcommand{\thekprRcounter}{\thechapter.\arabic{kprRcounter}}

\newenvironment*{ثبوت قاعدہ}				
{\noindent\ignorespaces  ثبوت قاعدہ :\quad}%
{\par\hfill\qedsymbol  \vspace{\baselineskip}\par }
%{\par\noindent\qedsymbol  \vspace{\baselineskip} }
%--------------------
%--------------------

%defining a TEST   پرکھ

%\newcounter{kttcounter}[section]
%\renewcommand{\thekttcounter}{\arabic{kttcounter}}
\newcounter{kttcounter}[chapter]
\renewcommand{\thekttcounter}{\thechapter.\arabic{kttcounter}}

\newenvironment*{پرکھ}				
{\par\noindent\ignorespaces  \refstepcounter{kttcounter} \quad \nopagebreak}%
{\par\hfill\qedsymbol  \vspace{\baselineskip}\par }
%--------------------

\newenvironment*{ثبوت پرکھ}				
{\noindent\ignorespaces  ثبوت پرکھ :\quad}%
{\par\hfill\qedsymbol  \vspace{\baselineskip}\par }
%{\par\noindent\qedsymbol  \vspace{\baselineskip} }

%---------

%questions environment جواب

%\newcounter{answercounter}[section]                                           %for reseting at every section. NOT NEEDED
%\renewcommand{\theanswercounter}{\arabic{answercounter}}
%\newcounter{answercounter}[chapter]						%when book is finished, use this instead of the above
%\renewcommand{\theanswercounter}{\thechapter.\arabic{answercounter}}

%\newenvironment*{جواب}				
%{\noindent\ignorespaces\wf{\thequestioncounter)\noexpand\quad}}%
%{\par\noindent\ignorespacesafterend}
%==================

\newenvironment*{سوالات}				
{\noindent\ignorespaces\wf{
\موٹا{حصہ} 
\thesection
\quad
\موٹا{صفحہ}
\thepage}
\wf{\unexpanded{\begin{description}\setlength{\parskip}{0pt} \setlength{\itemsep}{0pt plus 1pt}}}
}%
{\wf{\unexpanded{\end{description}}}\par\noindent\ignorespacesafterend}
%==================
\newenvironment*{جواب}  %the next is a better environment. it take less input		
{\noindent\ignorespaces\wf{\unexpanded{\item[}}\wf{\thequestioncounter)}\wf{\unexpanded{]}}}%
{\noindent\ignorespacesafterend}
%%==================
%%==================
%%this takes one input hence there is no need to write \wf{\unexpanded{...}} in every answer
%\newenvironment*{جواب}[1]			%put actual answer in {} examples {\(\sqrt{2}\)} 	
%{\noindent\ignorespaces\wf{\unexpanded{\item[}}\wf{\thequestioncounter)}\wf{\unexpanded{]}}\wf{\unexpanded{#1}}}%
%{\noindent\ignorespacesafterend}
%==================
%\newenvironment*{جوابء}				%i think is wrong and not used	
%{\noindent\ignorespaces\wf{\unexpanded{[}}\wf{\thequestioncounter)}\wf{\unexpanded{]}}}%
%{\par\noindent\ignorespacesafterend}
%==================

                  %turning latex into urdu
% Greek Letters for Urdu Latex usage

\newcommand*{\ایلفا}{\alpha}
\newcommand*{\بیٹا}{\beta}
\newcommand*{\گیما}{\gamma}
\newcommand*{\ڈیلٹا}{\delta}
\newcommand*{\ایپسلان}{\epsilon}
\newcommand*{\متغیرایپسلان}{\varepsilon}
\newcommand*{\زیٹا}{\zeta}
\newcommand*{\ایٹا}{\eta}
\newcommand*{\تھیٹا}{\theta}
\newcommand*{\متغیرتھیٹا}{\vartheta}
\newcommand*{\ایوٹا}{\iota}
\newcommand*{\کاپا}{\kappa}
\newcommand*{\لیمڈا}{\lambda}
\newcommand*{\میو}{\mu}
\newcommand*{\نیو}{\nu}
\newcommand*{\ژاے}{\xi}
\newcommand*{\پاے}{\pi}
\newcommand*{\متغیرپاے}{\varpi}
\newcommand*{\رھو}{\rho}
\newcommand*{\متغیررھو}{\varrho}
\newcommand*{\سگما}{\sigma}
\newcommand*{\متغیرسگما}{\varsigma}
\newcommand*{\ٹو}{\tau}
\newcommand*{\اپسیلان}{\upsilon}
\newcommand*{\فاے}{\phi}
\newcommand*{\متغیفاے}{\varphi}
\newcommand*{\چاے}{\chi}
\newcommand*{\ساے}{\psi}
\newcommand*{\اومیگا}{\omega}

\newcommand*{\بڑاگیما}{\Gamma}
\newcommand*{\بڑاڈیلٹا}{\Delta}
\newcommand*{\بڑاتھیٹا}{\Theta}
\newcommand*{\بڑالیمڈا}{\Lambda}
\newcommand*{\بڑاژاے}{\Xi}
\newcommand*{\بڑاپاے}{\Pi}
\newcommand*{\بڑاسگما}{\Sigma}
\newcommand*{\بڑاساے}{\Psi}
\newcommand*{\بڑااومیگا}{\Omega}



\newcommand*{\kvec}[1]{{\ensuremath{{\boldsymbol{#1}}}}}
\newcommand*{\kvecsub}[2]{{\ensuremath{{\boldsymbol{#1}}}_{\textup{#2}}}}

\newcommand*{\ax}{\ensuremath{{\boldsymbol{a}}_{\textup{x}}}}
\newcommand*{\ay}{\ensuremath{{\boldsymbol{a}}_{\textup{y}}}}
\newcommand*{\az}{\ensuremath{{\boldsymbol{a}}_{\textup{z}}}}
%
\newcommand*{\arho}{\ensuremath{{\boldsymbol{a}}_{\rho}}}
\newcommand*{\aphi}{\ensuremath{{\boldsymbol{a}}_{\phi}}}
%
\newcommand*{\ar}{\ensuremath{{\boldsymbol{a}}_{\textup{r}}}}
\newcommand*{\atheta}{\ensuremath{{\boldsymbol{a}}_{\theta}}}

\newcommand*{\aN}{\ensuremath{{\boldsymbol{a}}_N}}
\newcommand*{\aR}{\ensuremath{{\boldsymbol{a}}_{\textup{R}}}}
\newcommand*{\aL}{\ensuremath{{\boldsymbol{a}}_{\textup{L}}}}

\newcommand*{\au}{\ensuremath{{\boldsymbol{a}}_u}}
\newcommand*{\av}{\ensuremath{{\boldsymbol{a}}_v}}
\newcommand*{\aw}{\ensuremath{{\boldsymbol{a}}_w}}

\newcommand*{\ai}{\ensuremath{{\boldsymbol{i}}}}
\newcommand*{\aj}{\ensuremath{{\boldsymbol{j}}}}
\newcommand*{\ak}{\ensuremath{{\boldsymbol{k}}}}

\newcommand*{\uu}{\ensuremath{{\boldsymbol{u}}}}


\newcommand*{\Ex}{\ensuremath{{\boldsymbol{E}}_x}}
\newcommand*{\Ey}{\ensuremath{{\boldsymbol{E}}_y}}
\newcommand*{\Ez}{\ensuremath{{\boldsymbol{E}}_z}}
%
\newcommand*{\Erho}{\ensuremath{{\boldsymbol{E}}_{\rho}}}
\newcommand*{\Ephi}{\ensuremath{{\boldsymbol{E}}_{\phi}}}
%
\newcommand*{\Er}{\ensuremath{{\boldsymbol{E}}_r}}
\newcommand*{\Etheta}{\ensuremath{{\boldsymbol{E}}_{\theta}}}

\newcommand*{\TE}[1]{\ensuremath{\textup{TE}_{#1}}}
\newcommand*{\TM}[1]{\ensuremath{\textup{TM}_{#1}}}
\newcommand*{\TEM}{\ensuremath{\textup{TEM}}}
%===========================
\newcommand{\RightAngle}[4][5pt]{\draw[gray] ($#3!#1!#2$)--($ #3!2!($($#3!#1!#2$)!.5!($#3!#1!#4$)$) $) --($#3!#1!#4$) ;        }



\DeclareMathOperator{\sech}{sech}
\DeclareMathOperator{\csch}{csch}
\DeclareMathOperator{\cosec}{cosec}
\DeclareMathOperator{\arcsec}{arcsec}
\DeclareMathOperator{\arccot}{arcCot}
\DeclareMathOperator{\arccsc}{arcCsc}
\DeclareMathOperator{\arccosine}{arcCos}
\DeclareMathOperator{\arccosh}{arcCosh}
\DeclareMathOperator{\arcsinh}{arcsinh}
\DeclareMathOperator{\arctanh}{arctanh}
\DeclareMathOperator{\arcsech}{arcsech}
\DeclareMathOperator{\arccsch}{arcCsch}
\DeclareMathOperator{\arccoth}{arcCoth} 
\DeclareMathOperator{\erf}{erf} 
\DeclareMathOperator{\erfc}{erfc} 
%the following two Sine Integral symbols doesnot clash with SI units package
\DeclareMathOperator{\kSi}{Si} 
\DeclareMathOperator{\ksi}{si} 
\DeclareMathOperator{\kS}{S} 
%cosine integral, exponential integral, logrithmic integral
\DeclareMathOperator{\ci}{ci} 
\DeclareMathOperator{\kC}{C} 
\DeclareMathOperator{\Ei}{Ei} 
\DeclareMathOperator{\li}{li} 
%Fresnel Cosine and SineIntegrals and their Auxiliary integrals
\DeclareMathOperator{\FC}{C} 
\DeclareMathOperator{\FS}{S} 
\DeclareMathOperator{\FAC}{c}                  %complementary Fresnel integral
\DeclareMathOperator{\FAS}{s}                     %complementary Fresnel integral
\DeclareMathOperator{\gammaQ}{Q}                     %Incomplete gamma function
\DeclareMathOperator{\gammaP}{P}                     %Incomplete gamma function
%Hermite polynomials
\DeclareMathOperator{\He}{He} 
%Complex Natural Logarithm, Principal Value
\DeclareMathOperator{\Ln}{Ln} 
%Residue 
\DeclareMathOperator{\Res}{Res}
%Lagrange interpolation formula
\DeclareMathOperator{\Lagrange}{L}
%statistics UpperControlLimit and Lower Control Limit, Average Outgoing Quality
\DeclareMathOperator{\UCL}{UCL} 
\DeclareMathOperator{\LCL}{LCL} 
\DeclareMathOperator{\CL}{CL} 
\DeclareMathOperator{\OC}{OC} 
\DeclareMathOperator{\AOQ}{AOQ} 
%projection of a vector
\DeclareMathOperator{\proj}{proj} 
     %\sech, \csch, \arcsh, \arcs   hyperbolic and arc-secant etc

\pgfmathsetmacro{\x}{2}     %smallest resistor sizes
\pgfmathsetmacro{\y}{2}
\pgfmathsetmacro{\xx}{2.5}   %somewhat larger resistor leads. gives more space
\pgfmathsetmacro{\yy}{2.5}
\pgfmathsetmacro{\xxx}{3}   %still larger resistor leads. gives even more space
\pgfmathsetmacro{\yyy}{3}
\pgfmathsetmacro{\dx}{0.2}     %moving labels beyond resistor outline
\pgfmathsetmacro{\dy}{0.2}
\pgfmathsetmacro{\pin}{0.3}

\pgfmathsetmacro{\boxW}{0.5}   %width of box circuit
\pgfmathsetmacro{\boxH}{2.5}   %height of box circuit

%=============================
%complex numbers, squared voltages
\newcommand*{\bZ}{{\ensuremath{{\boldsymbol{Z}}}}}           %complex impedance
\newcommand*{\bY}{{\ensuremath{{\boldsymbol{Y}}}}}           %complex admittance
\newcommand*{\bZCC}{{\ensuremath{{\boldsymbol{Z}}^{*}}}}                                            %complex conjugate impedance
\newcommand*{\bYCC}{{\ensuremath{{\boldsymbol{Y}}^{*}}}}                                            %complex conjugate

\newcommand*{\bVrms}{{\ensuremath{\hat{V}_{\textup{rms}}}}}           %phasor voltage
\newcommand*{\bIrms}{{\ensuremath{\hat{I}_{\textup{rms}}}}}           %phasor current
\newcommand*{\Vrms}{{\ensuremath{V_{\textup{rms}}}}}       %rms voltage
\newcommand*{\Irms}{{\ensuremath{I_{\textup{rms}}}}}           %rms current
\newcommand*{\Arms}{{\ensuremath{A_{\textup{rms}}}}}           %rms amps
\newcommand*{\VrmsS}{{\ensuremath{V^2_{\textup{rms}}}}}       %rms squared
\newcommand*{\IrmsS}{{\ensuremath{I^2_{\textup{rms}}}}}           %rms squared
\newcommand*{\bVrmsCC}{{\ensuremath{\hat{V}^{*}_{\textup{rms}}}}}                  %conjugate phasor voltage
\newcommand*{\bIrmsCC}{{\ensuremath{\hat{I}^{*}_{\textup{rms}}}}}           %conjugate phasor current


\newcommand*{\kx}[1]{{\ensuremath{{\boldsymbol{#1}}}}}                  %complex quantity
\newcommand*{\bS}{{\ensuremath{{\boldsymbol{S}}}}}                         %complex power
\newcommand*{\bH}{{\ensuremath{{\boldsymbol{H}}}}}                       %network functions
\newcommand*{\bA}{{\ensuremath{{\boldsymbol{A}}}}}                        %voltage gain

\newcommand*{\pf}{{\ensuremath{{\textup{pf}}}}}
\newcommand*{\rms}{{\ensuremath{\textup{rms}}}}           %rms
\newcommand*{\BW}{{\ensuremath{{\textup{BW}}}}}   %bandwidth

\newcommand*{\Laplace}{\mathcal{L}}   %Laplace transform
\newcommand*{\Fourier}{\mathcal{F}}   %Fourier transform

\newcommand*{\kB}[1]{{\ensuremath{{\textup{#1}}}}}  %Laplace symbol general use. 
									%following were used too often so gave them specific symbols
\newcommand*{\bF}{{\ensuremath{{\textup{F}}}}}    %Fourier transform of 
\newcommand*{\bP}{{\ensuremath{{\textup{P}}}}}   %Laplace fraction
\newcommand*{\bQ}{{\ensuremath{{\textup{Q}}}}}  %Laplace fraction
\newcommand*{\bV}{{\ensuremath{{\textup{V}}}}}  %Laplace Voltage
\newcommand*{\bI}{{\ensuremath{{\textup{I}}}}}  %Laplace Current
%Matrices and Vectors
\newcommand*{\bM}[1]{{\ensuremath{{\boldsymbol{#1}}}}} 
 % resistor sizes and Laplace, Fourier, Complex, Phasor, etc  symbols
%draws left and right arrows where needed e.g.  
% \draw[->-=0.5] (0,0)--(3,0); draws arrow at the middle
\tikzset{->-/.style={decoration={markings, mark=at position #1 with {\arrow{latex}}},postaction={decorate}}}
\tikzset{-<-/.style={decoration={markings, mark=at position #1 with {\arrow{latex reversed}}},postaction={decorate}}}
\tikzset{osquare/.style={draw,solid,fill=white, rectangle, minimum size=4pt, inner sep=0pt, outer sep=0pt}}   %node[osquare,fill=black]{}

%this puts small orthogonal tick marks along a curve at selected points. a point at each side of the location has to be provided to %the \path[] section of the command as shown.
% 		\draw[smooth, domain=0:2]plot ({\x},{\x^2});
%		\foreach \t in {0.5,1,1.5}{\path[| mark=0.5] ({\t-0.1},{(\t-0.1)^2}) -- ({\t+0.1},{(\t+0.1)^2});}
\tikzset{| mark/.style={postaction=decorate,decoration={markings,
mark=at position #1 with {\draw[line cap=round,mark segment] (0,-2pt) -- (0,2pt);
}}},mark segment/.style={thick}}


%draws right angles \RightAngle{A}{B}{C}
\providecommand{\RightAngle}[4][5pt]{\draw[] ($#3!#1!#2$)--($ #3!2!($($#3!#1!#2$)!.5!($#3!#1!#4$)$) $) --($#3!#1!#4$) ;     }
%colours
\definecolor{lgray}{cmyk}{0,0,0,0.2}
\definecolor{dgray}{cmyk}{0,0,0,0.7}
%draws a cross just like ocirc, circ; usage \fill(0,2)circle(1.5py);\draw(0,2)node[kcross]{};
\tikzset{kcross/.style={cross out, draw, 
         minimum size=2*(2pt-\pgflinewidth), 
         inner sep=0pt, outer sep=0pt}}


%tikz, pgfplot TABLE
\pgfplotsset{select coords between index/.style 2 args={
    x filter/.code={
        \ifnum\coordindex<#1\def\pgfmathresult{}\fi
        \ifnum\coordindex>#2\def\pgfmathresult{}\fi
    }
}}
%
%boxed circuits
%=========================================
%\leftBox[K]{3,2}   draws a box with lower end at (3,2) and the terminals called Ka and Kb
\newcommand{\boxLeft}[2][p]{
\coordinate (a) at (#2);
\draw (a)++(-0.025,0.5) coordinate (b);
\draw (a)++(-0.04,1) coordinate (c);
\draw (a)++(-0.12,1.5) coordinate (d);
\draw (a)++(-0.2,2) coordinate (e);
\draw (a)++(-0.15,2.5) coordinate (f);
\draw (a)++(0.5,3) coordinate (g);

\draw (a)++(0.7,2.5)coordinate(h);
\draw (a)++(0.6,2)coordinate(i);
\draw (a)++(0.75,1.5)coordinate(j);
\draw (a)++(0.7,1)coordinate(k);
\draw (a)++(0.7,0.5)coordinate(l);
\draw (a)++(0.6,0)coordinate(m);
\draw plot [smooth cycle] coordinates {(a) (b) (c) (d) (e) (f) (g) (h) (i) (j) (k) (l) (m)};
\draw (h)coordinate(#1a);
\draw(l)coordinate(#1b);
}
%===================
%\rightBox[J]{3,2}   draws a box with lower end at (3,2) and the terminals called Ja and Jb
\newcommand{\boxRight}[2][p]{
\coordinate (aa) at (#2);
\draw (aa)++(0.025,0.5) coordinate(ba);
\draw (aa)++(0.04,1)coordinate(ca);
\draw (aa)++ (0.12,1.5)coordinate(da);
\draw (aa)++(0.13,2)coordinate(ea);
\draw (aa)++(0.1,2.5)coordinate(fa);
\draw (aa)++(-0.5,3)coordinate(ga);

\draw (aa)++(-0.8,2.5) coordinate(ha);
\draw (aa)++(-0.8,2) coordinate(ia);
\draw (aa)++ (-0.75,1.5) coordinate(ja);
\draw (aa)++(-0.7,1) coordinate(ka);
\draw (aa)++(-0.7,0.5) coordinate(la);
\draw (aa)++(-0.5,0) coordinate(ma);
\draw plot [smooth cycle] coordinates {(aa) (ba) (ca) (da) (ea) (fa) (ga) (ha) (ia) (ja) (ka) (la) (ma)};
\draw (ha)coordinate(#1a);
\draw(la)coordinate(#1b);
}
%===================
%writes text above matrix entries (outside the matrix bars)
\newcommand\bovermat[2]{%
  \makebox[0pt][r]{$\raisebox{16pt}[0pt][0pt]{\text{\RL{#1}}}$}#2}
\newcommand\covermat[2]{%
  \makebox[0pt][c]{$\raisebox{16pt}[0pt][0pt]{\text{\RL{#1}}}$}#2}
\newcommand\partialphantom{\vphantom{\frac{\partial e_{P,M}}{\partial w_{1,1}}}}

%=============================
%when a table is all math, instead of using $$ in each cell use the following.Text can be entered in a cell with \text{} 
%usage \begin{matrix}{C|L} ;not needed in array as array is $$ by default
\newcolumntype{L}{>{$}l<{$}}
\newcolumntype{C}{>{$}c<{$}}
\newcolumntype{R}{>{$}r<{$}}

%=============================
% this encircles the number. can be used inline, and inside tables too.
%the urdu equivalent is \دائرہبند  command
\newcommand*\encircle[1]{\tikz[baseline=(char.base)]{
            \node[shape=circle,draw,inner sep=2pt] (char) {#1};}}

\tikzset{flipflop SR/.style={flipflop,
flipflop def={width=1.0,pin spacing=0.4,font=\normalsize,t1={\,$S$},t3={\,$ R$},t6={$Q$\,} ,t4={$\overline{Q}$\,}}}}

\tikzset{flipflop D/.style={flipflop, 
flipflop def={width=1.0,pin spacing=0.4,font=\normalsize,t1={\,$D$},t2={\,$C$},c2=1,n4=1,
t4={$\overline{Q}$\,} ,t6={$Q$\,} ,}}}

\tikzset{flipflop adder/.style={flipflop, 
flipflop def={width=1.0,pin spacing=0.4,font=\normalsize,t1={\,$y$},t2={\,$z$},td={\normalsize $c_{\text{خ}}$} ,t6={$s$\,} , 
tu={\normalsize$c_{\text{د}}$}}}}

%=============================
%=============================
%   \kfulladder[u1]{xshift}{yshift}     %called with ref desig, used to provide pin anchor naming (such as u1S, u1Q, u1C)
							% and location (xshift,yshift) where the comp is to be placed
							% default ref des is "u0"
\newcommand{\kfulladder}[3][u0]{
\def\kshiftX{#2}
\def\kshiftY{#3}
\pgfmathsetmacro{\kpin}{0.50}
\pgfmathsetmacro{\kpsep}{0.60}			%pin to pin distance
\pgfmathsetmacro{\kulV}{0.50}			%edge clearance
\pgfmathsetmacro{\kdimY}{2*\kulV+0*\kpsep}
\pgfmathsetmacro{\kdimX}{2*\kulV+2*\kpsep}		%two spaces between 3 pins

\def\leftPins{1/y/y,2/z/z}
\def\rightPins{2/s/s}
\def\cdelX{0}
\def\cdelY{0}
\def\upClk{0/c_i/ci}
\def\downClk{0/c_o/co}
\draw[thick](\kshiftX,\kshiftY) rectangle ++(\kdimX,\kdimY);
\foreach \yLoc/\lbl/\anchor in \leftPins {\draw(\kshiftX+\kulV+\yLoc*\kpsep,\kshiftY+\kdimY)node[below]{$\lbl$}--++(0,\kpin,0)coordinate(#1\anchor);}
\foreach \yLoc/\lbl/\anchor in \rightPins{\draw(\kshiftX+\kulV+\yLoc*\kpsep,\kshiftY)node[above]{$\lbl$}--++(0,-\kpin,0)coordinate(#1\anchor);}
\foreach \yLoc/\lbl/\anchor in \upClk{\draw(\kshiftX+\kdimX,\kshiftY+\kulV+\yLoc*\kpsep)coordinate(cc)--++(\kpin,0)coordinate(#1\anchor)  (cc)++(0,-\cdelY)--++(-\cdelX,\cdelY)node[left]{$\lbl$}--++(\cdelX,\cdelY);}
\foreach \yLoc/\lbl/\anchor in \downClk{\draw(\kshiftX,\kshiftY+\kulV+\yLoc*\kpsep)coordinate(cc)--++(-\kpin,0)coordinate(#1\anchor)  (cc)++(0,-\cdelY)--++(\cdelX,\cdelY)node[right]{$\lbl$}--++(-\cdelX,\cdelY);}
}
%==================================
%%%%%%%%%%%%%%%%%%%%%%%%%%%%%
%%%%%%%%%%%%%%%%%%%%%%%%%%%%%%%
%=============================
%for FlipFlops
%gives pin tip's locations as (pax,pay), pins are pa,pb,...,pf, pup,pdown corresponding to p1,..,p6,pu,pd
\newcommand{\kpinAnchors}{
\def\pax{-\kpin}
\def\pay{\kulV+2*\kpsep}
\def\pbx{-\kpin}
\def\pby{\kulV+1*\kpsep}
\def\pcx{-\kpin}
\def\pcy{\kulV+0*\kpsep}
\def\pdx{\kdimX+\kpin}
\def\pdy{\kulV+0*\kpsep}
\def\pex{\kdimX+\kpin}
\def\pey{\kulV+1*\kpsep}
\def\pfx{\kdimX+\kpin}
\def\pfy{\kulV+2*\kpsep}
\def\pupx{\kdimX/2}
\def\pupy{\kpin+\kdimY}
\def\pdownx{\kdimX/2}
\def\pdowny{-\kpin}
}
%====================================
%puts labels of pins
%\kFFlabelPins[u1]{1/D,2/C,4/{\overline{Q}},6/Q}
\newcommand{\kFFlabelPins}[2][u0]{
\def\kref{#1}
\def\p{p}			%pin tip
\def\pb{pb}			%pin base
\def\pl{pl}			%pin label
\def\pn{pn}			%pin 'ocirc'
\def\px{x}			%pinAnchorX
\def\py{y}			%pinAnchorY
\foreach \n/\lbl in {#2}{\draw(\kref\pl\n)node[]{$\lbl$};}
}
%====================================
%====================================
%draws triangular edge of the clock. with the edge, clock should nt have label "C"
%\kFFclockEdge[u1]{2}
\newcommand{\kFFclockEdge}[2][u0]{
\def\kref{#1}
\def\cdelX{0.15}
\def\cdelY{0.15}
\def\p{p}			%pin tip
\def\pb{pb}			%pin base
\def\pl{pl}			%pin label
\def\pn{pn}			%pin 'ocirc'
\def\px{x}			%pinAnchorX
\def\py{y}			%pinAnchorY
\foreach \n in {#2}{\draw(\kref\pcd\n)--(\kref\pcm\n)--(\kref\pcu\n);}
}
%==================================
%draws small circle on the "Active Low Pins".    pins are counted like dip6 IC. package. 
%\kFFnegativePin[u1]{2,4,6} 
\newcommand{\kFFnegatePin}[2][u0]{
\def\kref{#1}
\def\p{p}			%pin tip
\def\pb{pb}			%pin base
\def\pl{pl}			%pin label
\def\pn{pn}			%pin 'ocirc'
\def\px{x}			%pinAnchorX
\def\py{y}			%pinAnchorY
\foreach \n in {#2} {\draw(\kref\pn\n)node[ocirc]{};}
}
%====================================
%draws NOTHING, only gives all pin related ANCHORS
% generic comp for extracting all pin anchors such as p1,p2,..,p6,pu,pd,   pb1,pb2,...pb6,pbu,pbd,   pa1,pa2,...,pa6,pau,pad,   %  pcu1,pcm1,pcd1,...pcu6,pcm6,pcd6

%   \kgenSRFF[u1]{xshift}{yshift}     %called with ref desig, used to provide pin anchor naming (such as u1S, u1Q, u1C)
							% and location (xshift,yshift) where the comp is to be placed
							% default ref des is "u0"
\newcommand{\kFFa}[1][u0]{
\def\kref{#1}
\def\kshiftX{0}
\def\kshiftY{0}
\def\p{p}			%pin tip
\def\pb{pb}			%pin base
\def\pl{pl}			%pin label
\def\pn{pn}			%pin 'ocirc'
\def\px{x}			%pinAnchorX
\def\py{y}			%pinAnchorY
\def\pcu{pcu}			%pin tip
\def\pcm{pcm}			%pin tip
\def\pcd{pcd}			%pin tip
\pgfmathsetmacro{\klshift}{0.25}
\pgfmathsetmacro{\knshift}{0.07}
\pgfmathsetmacro{\kpin}{0.30}
\pgfmathsetmacro{\kpsep}{0.40}			%pin to pin distance
\pgfmathsetmacro{\kulV}{0.40}			%edge clearance along vertical edges
\pgfmathsetmacro{\kulH}{0.50}
\pgfmathsetmacro{\kdimX}{2*\kulH+0*\kpsep}
\pgfmathsetmacro{\kdimY}{2*\kulV+2*\kpsep}		%two spaces between 3 pins
\def\cdelX{0.15}
\def\cdelY{0.15}
%%\def\leftPins{0/R/R,2/S/S}
%%\def\rightPins{0/{\overline{Q}}/QN,2/Q/Q}
%\draw[thick](\kshiftX,\kshiftY) rectangle ++(\kdimX,\kdimY);
\draw(\kshiftX,\kshiftY)coordinate(\kref-south-west);
\draw(\kshiftX+\kdimX,\kshiftY)coordinate(\kref-south-east);
\draw(\kshiftX+\kdimX,\kdimY+\kshiftY)coordinate(\kref-north-east);
\draw(\kshiftX,\kdimY+\kshiftY)coordinate(\kref-north-west);
\draw(\kshiftX+\kdimX/2,\kshiftY)coordinate(\kref-south);
\draw(\kshiftX+\kdimX/2,\kdimY+\kshiftY)coordinate(\kref-north);
\draw(\kshiftX,\kdimY/2+\kshiftY)coordinate(\kref-west);
\draw(\kshiftX+\kdimX,\kdimY/2+\kshiftY)coordinate(\kref-east);
\draw(\kshiftX+\kdimX/2,\kdimY/2+\kshiftY)coordinate(\kref-center);
%%left pins
\foreach \n/\m in {0/3,1/2,2/1}{\draw(\kshiftX,\kshiftY+\kulV+\n*\kpsep)
coordinate(\kref\pb\m)+(\klshift,0)coordinate(\kref\pl\m)+(-\knshift,0)coordinate(\kref\pn\m)+(-\kpin,0)coordinate(\kref\p\m);}
%%right pins
\foreach \n/\m in {0/4,1/5,2/6}{\draw(\kshiftX+\kdimX,\kshiftY+\kulV+\n*\kpsep)
coordinate(\kref\pb\m)+(-\klshift,0)coordinate(\kref\pl\m)+(\knshift,0)coordinate(\kref\pn\m)+(\kpin,0)coordinate(\kref\p\m);}
%%up pin
\foreach \n/\m in {0/u}{\draw(\kshiftX+\kulH+\n*\kpsep,\kdimY+\kshiftY)
coordinate(\kref\pb\m)+(0,-\klshift)coordinate(\kref\pl\m)+(0,\knshift)coordinate(\kref\pn\m)+(0,\kpin)coordinate(\kref\p\m);}
%%down pin
\foreach \n/\m in {0/d}{\draw(\kshiftX+\kulH+\n*\kpsep,\kshiftY)
coordinate(\kref\pb\m)+(0,\klshift)coordinate(\kref\pl\m)+(0,-\knshift)coordinate(\kref\pn\m)+(0,-\kpin)coordinate(\kref\p\m);}
%%clock edges left pin
\foreach \n/\m in {0/3,1/2,2/1}{\draw(\kshiftX,\kshiftY+\kulV+\n*\kpsep)+(0,-\cdelY)
coordinate(\kref\pcd\m)+(\cdelX,0)coordinate(\kref\pcm\m)+(0,\cdelY)coordinate(\kref\pcu\m);}
%%clock edges right pin
\foreach \n/\m in {0/4,1/5,2/6}{\draw(\kshiftX+\kdimX,\kshiftY+\kulV+\n*\kpsep)
+(0,-\cdelY)coordinate(\kref\pcd\m)+(-\cdelX,0)coordinate(\kref\pcm\m)+(0,\cdelY)coordinate(\kref\pcu\m);}
}
%%%%%%%%%%%%%%%%%%%%%%%%
%=============================
%draws general purpose FF's SKELETON ONLY
%labels, pins, clock edge needs to be added separately
%can be anchored with pin at specific coordinate(x,y) as explained down
%%   \kSRFF[u1]{xshift}{yshift}   
%   to anchor pin2 at location (x,y) use
%	\kSRFF[u1]{x-\pbx}{y-\pby}
%pin 1,2,3,4,5,6,u,d are in this context written \pba, ...\pbf,\pbup,\pbdown

\newcommand{\kFF}[3][u0]{
\def\kref{#1}
\def\kshiftX{#2}
\def\kshiftY{#3}

\def\p{p}			%pin tip
\def\pb{pb}			%pin base
\def\pl{pl}			%pin label
\def\pcu{pcu}			%clock edge, upper corner
\def\pcm{pcm}			%clock edge middle corner
\def\pcd{pcd}			%clock edge lower corner
\def\pn{pn}			%pin 'ocirc'
\pgfmathsetmacro{\klshift}{0.25}
\pgfmathsetmacro{\knshift}{0.07}
\pgfmathsetmacro{\kpin}{0.30}
\pgfmathsetmacro{\kpsep}{0.40}			%pin to pin distance
\pgfmathsetmacro{\kulV}{0.40}			%edge clearance along vertical edge
\pgfmathsetmacro{\kulH}{0.50}
\pgfmathsetmacro{\kdimX}{2*\kulH+0*\kpsep}
\pgfmathsetmacro{\kdimY}{2*\kulV+2*\kpsep}		%two spaces between 3 pins
\def\cdelX{0.15}
\def\cdelY{0.15}
%%\def\leftPins{0/R/R,2/S/S}
%%\def\rightPins{0/{\overline{Q}}/QN,2/Q/Q}
\draw[thick](\kshiftX,\kshiftY) rectangle ++(\kdimX,\kdimY);
\draw(\kshiftX,\kshiftY)coordinate(\kref-south-west);
\draw(\kshiftX+\kdimX,\kshiftY)coordinate(\kref-south-east);
\draw(\kshiftX+\kdimX,\kdimY+\kshiftY)coordinate(\kref-north-east);
\draw(\kshiftX,\kdimY+\kshiftY)coordinate(\kref-north-west);
\draw(\kshiftX+\kdimX/2,\kshiftY)coordinate(\kref-south);
\draw(\kshiftX+\kdimX/2,\kdimY+\kshiftY)coordinate(\kref-north);
\draw(\kshiftX,\kdimY/2+\kshiftY)coordinate(\kref-west);
\draw(\kshiftX+\kdimX,\kdimY/2+\kshiftY)coordinate(\kref-east);
\draw(\kshiftX+\kdimX/2,\kdimY/2+\kshiftY)coordinate(\kref-center);
%%left pins
\foreach \n/\m in {0/3,1/2,2/1}{\draw(\kshiftX,\kshiftY+\kulV+\n*\kpsep)
coordinate(\kref\pb\m)+(\klshift,0)coordinate(\kref\pl\m)+(-\knshift,0)coordinate(\kref\pn\m)+(-\kpin,0)coordinate(\kref\p\m);}
%%right pins
\foreach \n/\m in {0/4,1/5,2/6}{\draw(\kshiftX+\kdimX,\kshiftY+\kulV+\n*\kpsep)
coordinate(\kref\pb\m)+(-\klshift,0)coordinate(\kref\pl\m)+(\knshift,0)coordinate(\kref\pn\m)+(\kpin,0)coordinate(\kref\p\m);}
%%up pin
\foreach \n/\m in {0/u}{\draw(\kshiftX+\kulH+\n*\kpsep,\kdimY+\kshiftY)
coordinate(\kref\pb\m)+(0,-\klshift)coordinate(\kref\pl\m)+(0,\knshift)coordinate(\kref\pn\m)+(0,\kpin)coordinate(\kref\p\m);}
%%down pin
\foreach \n/\m in {0/d}{\draw(\kshiftX+\kulH+\n*\kpsep,\kshiftY)
coordinate(\kref\pb\m)+(0,\klshift)coordinate(\kref\pl\m)+(0,-\knshift)coordinate(\kref\pn\m)+(0,-\kpin)coordinate(\kref\p\m);}
%%clock edges left pin
\foreach \n/\m in {0/3,1/2,2/1}{\draw(\kshiftX,\kshiftY+\kulV+\n*\kpsep)+(0,-\cdelY)
coordinate(\kref\pcd\m)+(\cdelX,0)coordinate(\kref\pcm\m)+(0,\cdelY)coordinate(\kref\pcu\m);}
%%clock edges right pin
\foreach \n/\m in {0/4,1/5,2/6}{\draw(\kshiftX+\kdimX,\kshiftY+\kulV+\n*\kpsep)
+(0,-\cdelY)coordinate(\kref\pcd\m)+(-\cdelX,0)coordinate(\kref\pcm\m)+(0,\cdelY)coordinate(\kref\pcu\m);}
}
%%%%%%%%%%%%%%%%%%%%%%%%%%%%%%%
%   to anchor pin2 at location (x,y) use
%	\kgenSRFF[u1]{x-\pbx}{y-\pby}
%pin 1,2,3,4,5,6,u,d are in this context written \pba, ...\pbf,\pbup,\pbdown

\newcommand{\kSRFF}[3][u0]{
\kFF[#1]{#2}{#3}
\def\kref{#1}
%\def\p{p}			%pin tip
%\def\pb{pb}			%pin base
%\def\pl{pl}			%pin label
%\def\pcu{pcu}			%pin edge, upper corner
%\def\pcm{pcm}			%pin edge middle corner
%\def\pcd{pcd}			%pin edge lower corner
%\def\pn{pn}			%pin 'ocirc'
\foreach \n in {1,2,3,4,6}{\draw(\kref\pb\n)--(\kref\p\n);}
\foreach \n /\lbl in {1/S,2/C,3/R,4/{\overline{Q}},6/Q}{\draw(\kref\pl\n)node[]{$\lbl$};}
\foreach \n in {4}{\draw(\kref\pn\n)node[ocirc]{};}
}
%%%%%%%%%%%%%%%%%%%%%%%%%%%%%%%
\newcommand{\kDFF}[3][u0]{
\def\kref{#1}
%\def\p{p}			%pin tip
%\def\pb{pb}			%pin base
%\def\pl{pl}			%pin label
%\def\pcu{pcu}			%pin edge, upper corner
%\def\pcm{pcm}			%pin edge middle corner
%\def\pcd{pcd}			%pin edge lower corner
%\def\pn{pn}			%pin 'ocirc'
\kFF[#1]{#2}{#3}
\foreach \n in {1,2,4,6}{\draw(\kref\pb\n)--(\kref\p\n);}
\foreach \n/\lbl in {1/D,4/{\overline{Q}},6/Q}{\draw(\kref\pl\n)node[]{$\lbl$};}
\foreach \n in {2}{\draw(\kref\pcd\n)--(\kref\pcm\n)--(\kref\pcu\n);}
\foreach \n in {4}{\draw(\kref\pn\n)node[ocirc]{};}
}
%%%%%%%%%%%%%%%%%%%%%%%%%%%%%%%
%%%%%%%%%%%%%%%%%%%%%%%%%%%%%%%
\newcommand{\kDFFud}[3][u0]{
\def\kref{#1}
%\def\p{p}			%pin tip
%\def\pb{pb}			%pin base
%\def\pl{pl}			%pin label
%\def\pcu{pcu}			%pin edge, upper corner
%\def\pcm{pcm}			%pin edge middle corner
%\def\pcd{pcd}			%pin edge lower corner
%\def\pn{pn}			%pin 'ocirc'
\kFF[#1]{#2}{#3}
\foreach \n in {1,2,4,6,u,d}{\draw(\kref\pb\n)--(\kref\p\n);}
\foreach \n/\lbl in {1/D,4/{\overline{Q}},6/Q}{\draw(\kref\pl\n)node[]{$\lbl$};}
\foreach \n in {2}{\draw(\kref\pcd\n)--(\kref\pcm\n)--(\kref\pcu\n);}
\foreach \n in {4,u,d}{\draw(\kref\pn\n)node[ocirc]{};}
}
%%%%%%%%%%%%%%%%%%%%%%%%%%%%%%%
\newcommand{\kJKFF}[3][u0]{
\def\kref{#1}
%\def\p{p}			%pin tip
%\def\pb{pb}			%pin base
%\def\pl{pl}			%pin label
%\def\pcu{pcu}			%pin edge, upper corner
%\def\pcm{pcm}			%pin edge middle corner
%\def\pcd{pcd}			%pin edge lower corner
%\def\pn{pn}			%pin 'ocirc'
\kFF[#1]{#2}{#3}
\foreach \n in {1,2,3,4,6}{\draw(\kref\pb\n)--(\kref\p\n);}
\foreach \n/\lbl in {1/J,3/K,4/{\overline{Q}},6/Q}{\draw(\kref\pl\n)node[]{$\lbl$};}
\foreach \n in {2}{\draw(\kref\pcd\n)--(\kref\pcm\n)--(\kref\pcu\n);}
\foreach \n in {4}{\draw(\kref\pn\n)node[ocirc]{};}
}
%%%%%%%%%%%%%%%%%%%%%%
%%%%%%%%%%%%%%%%%%%%%%%%%%%%%%%
\newcommand{\kTFF}[3][u0]{
\def\kref{#1}
%\def\p{p}			%pin tip
%\def\pb{pb}			%pin base
%\def\pl{pl}			%pin label
%\def\pcu{pcu}			%pin edge, upper corner
%\def\pcm{pcm}			%pin edge middle corner
%\def\pcd{pcd}			%pin edge lower corner
%\def\pn{pn}			%pin 'ocirc'
\kFF[#1]{#2}{#3}
\foreach \n in {1,2,4,6}{\draw(\kref\pb\n)--(\kref\p\n);}
\foreach \n/\lbl in {1/T,4/{\overline{Q}},6/Q}{\draw(\kref\pl\n)node[]{$\lbl$};}
\foreach \n in {2}{\draw(\kref\pcd\n)--(\kref\pcm\n)--(\kref\pcu\n);}
\foreach \n in {4}{\draw(\kref\pn\n)node[ocirc]{};}
}
%%%%%%%%%%%%%%%%%%%%%%
%%%%%%%%%%%%%%%%%%%%%%%%%%%%%%%
\newcommand{\kTFFu}[3][u0]{
\def\kref{#1}
%\def\p{p}			%pin tip
%\def\pb{pb}			%pin base
%\def\pl{pl}			%pin label
%\def\pcu{pcu}			%pin edge, upper corner
%\def\pcm{pcm}			%pin edge middle corner
%\def\pcd{pcd}			%pin edge lower corner
%\def\pn{pn}			%pin 'ocirc'
\kFF[#1]{#2}{#3}
\foreach \n in {1,2,4,6,u}{\draw(\kref\pb\n)--(\kref\p\n);}
\foreach \n/\lbl in {1/T,4/{\overline{Q}},6/Q}{\draw(\kref\pl\n)node[]{$\lbl$};}
\foreach \n in {2}{\draw(\kref\pcd\n)--(\kref\pcm\n)--(\kref\pcu\n);}
\foreach \n in {4,u}{\draw(\kref\pn\n)node[ocirc]{};}
}
%%%%%%%%%%%%%%%%%%%%%%
%%%%%%%%%%%%%%%%%%%%%%%%%%%%%%%
\newcommand{\kTFFd}[3][u0]{
\def\kref{#1}
%\def\p{p}			%pin tip
%\def\pb{pb}			%pin base
%\def\pl{pl}			%pin label
%\def\pcu{pcu}			%pin edge, upper corner
%\def\pcm{pcm}			%pin edge middle corner
%\def\pcd{pcd}			%pin edge lower corner
%\def\pn{pn}			%pin 'ocirc'
\kFF[#1]{#2}{#3}
\foreach \n in {1,2,4,6,d}{\draw(\kref\pb\n)--(\kref\p\n);}
\foreach \n/\lbl in {1/T,4/{\overline{Q}},6/Q}{\draw(\kref\pl\n)node[]{$\lbl$};}
\foreach \n in {2}{\draw(\kref\pcd\n)--(\kref\pcm\n)--(\kref\pcu\n);}
\foreach \n in {4,d}{\draw(\kref\pn\n)node[ocirc]{};}
}
%%%%%%%%%%%%%%%%%%%%%%
%%%%%%%%%%%%%%%%%%%%%%%%%%%%%%
\newcommand{\kTFFud}[3][u0]{
\def\kref{#1}
%\def\p{p}			%pin tip
%\def\pb{pb}			%pin base
%\def\pl{pl}			%pin label
%\def\pcu{pcu}			%pin edge, upper corner
%\def\pcm{pcm}			%pin edge middle corner
%\def\pcd{pcd}			%pin edge lower corner
%\def\pn{pn}			%pin 'ocirc'
\kFF[#1]{#2}{#3}
\foreach \n in {1,2,4,6,u,d}{\draw(\kref\pb\n)--(\kref\p\n);}
\foreach \n/\lbl in {1/T,4/{\overline{Q}},6/Q}{\draw(\kref\pl\n)node[]{$\lbl$};}
\foreach \n in {2}{\draw(\kref\pcd\n)--(\kref\pcm\n)--(\kref\pcu\n);}
\foreach \n in {4,u,d}{\draw(\kref\pn\n)node[ocirc]{};}
}
%%%%%%%%%%%%%%%%%%%%%%
% buffer that gives both the Signal and its Complement is  \kBusBuffer
\newcommand{\kBuffer}[3][u0]{
\def\kref{#1}
\def\kshiftX{#2}
\def\kshiftY{#3}

\def\p{p}			%pin tip
\def\pb{pb}			%pin base
\def\pl{pl}			%pin label
\def\pn{pn}			%pin 'ocirc'
\def\pin{pin}			%pin tip
\def\pout{pout}			%pin tip
\def\pu{u}			%pin tip
\def\pd{d}			%pin tip
\pgfmathsetmacro{\klshift}{0.25}
\pgfmathsetmacro{\knshift}{0.07}
\pgfmathsetmacro{\kpin}{0.30}
\pgfmathsetmacro{\kpsep}{0.40}			%pin to pin distance
\pgfmathsetmacro{\kulV}{0.40}			%edge clearance along vertical edge
\pgfmathsetmacro{\kulH}{0.50}
\pgfmathsetmacro{\kdimX}{2*\kulH+0*\kpsep}
\pgfmathsetmacro{\kdimY}{\kdimX}		%two spaces between 3 pins
\draw(\kshiftX,\kshiftY)coordinate(\kref\pin)--++(\kpin,0)coordinate(\kref\pb\pin);
\draw(\kref\pb\pin)[thick]++(0,-\kdimY/2)coordinate(aa)--++(0,\kdimY)coordinate(bb)--++(\kdimX,-\kdimY/2)coordinate(cc)coordinate(\kref\pb\pout)--++(-\kdimX,-\kdimY/2);
\draw(\kref\pb\pout)--++(\kpin,0)coordinate(\kref\pout);
\draw($(bb)!0.5!(cc)$)coordinate(\kref\pb\pu)+(0,0.07)coordinate(\kref\pn\pu)+(0,\kpin)coordinate(\kref\pu);
\draw($(aa)!0.5!(cc)$)coordinate(\kref\pb\pd)+(0,-0.07)coordinate(\kref\pn\pd)+(0,-\kpin)coordinate(\kref\pd);
\draw(\kref\pb\pin)++(-0.07,0)coordinate(\kref\pn\pin);
\draw(\kref\pb\pout)++(0.07,0)coordinate(\kref\pn\pout);
}
%=====================================
%========================================
%input-output pins. if needs drawing at "coordinate(aa) at (2,3)" use the following command.
%%  \koutleft[j1]{aa}     
%%  \koutleft[j1]{2,3}
\newcommand{\koutleft}[2][j0]{
\def\kref{#1}
\def\keast{east}
\def\kwest{west}
\def\knorth{north}
\def\ksouth{south}
\pgfmathsetmacro{\kpin}{0.30}
\pgfmathsetmacro{\kw}{0.20}
\pgfmathsetmacro{\klen}{0.50}
\pgfmathsetmacro{\kdelX}{0.10}
\draw(#2)--++(-\kpin,0)coordinate(\kref\keast)--++(\kdelX,\kw/2)--++(-\klen,0)coordinate[pos=0.5](\kref\knorth)--++(-\kdelX,-\kw/2)coordinate(\kref\kwest)--++(\kdelX,-\kw/2)--++(\klen,0)coordinate[pos=0.5](\kref\ksouth)--++(-\kdelX,\kw/2);
}
%========================================
\newcommand{\koutright}[2][j0]{
\def\kref{#1}
\def\keast{east}
\def\kwest{west}
\def\knorth{north}
\def\ksouth{south}
\pgfmathsetmacro{\kpin}{0.30}
\pgfmathsetmacro{\kw}{0.20}
\pgfmathsetmacro{\klen}{0.50}
\pgfmathsetmacro{\kdelX}{0.10}
\draw(#2)--++(\kpin,0)coordinate(\kref\kwest)--++(-\kdelX,\kw/2)--++(\klen,0)coordinate[pos=0.5](\kref\knorth)--++(\kdelX,-\kw/2)coordinate(\kref\keast)--++(-\kdelX,-\kw/2)--++(-\klen,0)coordinate[pos=0.5](\kref\ksouth)--++(\kdelX,\kw/2);
}
%========================================
\newcommand{\koutup}[2][j0]{
\def\kref{#1}
\def\keast{east}
\def\kwest{west}
\def\knorth{north}
\def\ksouth{south}
\pgfmathsetmacro{\kpin}{0.30}
\pgfmathsetmacro{\kw}{0.20}
\pgfmathsetmacro{\klen}{0.50}
\pgfmathsetmacro{\kdelX}{0.10}
\draw(#2)--++(0,\kpin)coordinate(\kref\knorth)--++(\kw/2,-\kdelX)--++(0,\klen)coordinate[pos=0.5](\kref\keast)--++(-\kw/2,\kdelX)coordinate(\kref\knorth)--++(-\kw/2,-\kdelX)
--++(0,-\klen)coordinate[pos=0.5](\kref\kwest)--++(\kw/2,\kdelX);
}
%========================================
\newcommand{\koutdown}[2][j0]{
\def\kref{#1}
\def\keast{east}
\def\kwest{west}
\def\knorth{north}
\def\ksouth{south}
\pgfmathsetmacro{\kpin}{0.30}
\pgfmathsetmacro{\kw}{0.20}
\pgfmathsetmacro{\klen}{0.50}
\pgfmathsetmacro{\kdelX}{0.10}
\draw(#2)--++(0,-\kpin)coordinate(\kref\knorth)--++(-\kw/2,\kdelX)--++(0,-\klen)coordinate[pos=0.5](\kref\kwest)--++(\kw/2,-\kdelX)coordinate(\kref\ksouth)--++(\kw/2,\kdelX)
--++(0,\klen)coordinate[pos=0.5](\kref\keast)--++(-\kw/2,-\kdelX);
}
%=====================================
%=====================================
%========================================
\newcommand{\kinright}[2][j0]{
\def\kref{#1}
\def\keast{east}
\def\kwest{west}
\def\knorth{north}
\def\ksouth{south}
\pgfmathsetmacro{\kpin}{0.30}
\pgfmathsetmacro{\kw}{0.20}
\pgfmathsetmacro{\klen}{0.50}
\pgfmathsetmacro{\kdelX}{0.10}
\draw(#2)--++(\kpin,0)coordinate(\kref\kwest)--++(\kdelX,\kw/2)--++(\klen,0)coordinate[pos=0.5](\kref\knorth)--++(-\kdelX,-\kw/2)coordinate(aa)--++(\kdelX,-\kw/2)--++(-\klen,0)coordinate[pos=0.5](\kref\ksouth)--++(-\kdelX,\kw/2) 
(aa)++(\kdelX,0)coordinate(\kref\keast);
}
%========================================
\newcommand{\kinleft}[2][j0]{
\def\kref{#1}
\def\keast{east}
\def\kwest{west}
\def\knorth{north}
\def\ksouth{south}
\pgfmathsetmacro{\kpin}{0.30}
\pgfmathsetmacro{\kw}{0.20}
\pgfmathsetmacro{\klen}{0.50}
\pgfmathsetmacro{\kdelX}{0.10}
\draw(#2)--++(-\kpin,0)coordinate(\kref\keast)--++(-\kdelX,\kw/2)--++(-\klen,0)coordinate[pos=0.5](\kref\knorth)--++(\kdelX,-\kw/2)coordinate(aa)--++(-\kdelX,-\kw/2)--++(\klen,0)coordinate[pos=0.5](\kref\ksouth)--++(\kdelX,\kw/2) 
(aa)++(-\kdelX,0)coordinate(\kref\kwest);
}
%========================================
\newcommand{\kindown}[2][j0]{
\def\kref{#1}
\def\keast{east}
\def\kwest{west}
\def\knorth{north}
\def\ksouth{south}
\pgfmathsetmacro{\kpin}{0.30}
\pgfmathsetmacro{\kw}{0.20}
\pgfmathsetmacro{\klen}{0.50}
\pgfmathsetmacro{\kdelX}{0.10}
\draw(#2)--++(0,-\kpin)coordinate(\kref\knorth)--++(\kw/2,-\kdelX)--++(0,-\klen)
coordinate[pos=0.5](\kref\keast)--++(-\kw/2,\kdelX)coordinate(aa)--++(-\kw/2,-\kdelX)
--++(0,\klen)coordinate[pos=0.5](\kref\kwest)--++(\kw/2,\kdelX) (aa)++(0,-\kdelX)coordinate(\kref\ksouth);
}
%========================================
\newcommand{\kinup}[2][j0]{
\def\kref{#1}
\def\keast{east}
\def\kwest{west}
\def\knorth{north}
\def\ksouth{south}
\pgfmathsetmacro{\kpin}{0.30}
\pgfmathsetmacro{\kw}{0.20}
\pgfmathsetmacro{\klen}{0.50}
\pgfmathsetmacro{\kdelX}{0.10}
\draw(#2)--++(0,\kpin)coordinate(\kref\ksouth)--++(-\kw/2,\kdelX)--++(0,\klen)coordinate[pos=0.5](\kref\kwest)--++(\kw/2,-\kdelX)--++(\kw/2,\kdelX)coordinate(aa)--++(0,-\klen)coordinate[pos=0.5](\kref\keast)--++(-\kw/2,-\kdelX) (aa)(aa)++(0,\kdelX)coordinate(\kref\knorth);
}
%=====================================
%%%%%%%%%%%%%%%%%%%%%%%%%%%%%%%%
%=====================================
%=====================================
%========================================
%these are control signals, inputs only. 
%% \kinrightA[j1]{2,3}
%%\kinrightA[j1]{coordName}
\newcommand{\kinrightA}[2][j0]{
\def\kref{#1}
\def\keast{east}
\def\kwest{west}
\def\knorth{north}
\def\ksouth{south}
\pgfmathsetmacro{\kpin}{0.30}
\pgfmathsetmacro{\kw}{0.20}
\pgfmathsetmacro{\klen}{0.50}
\pgfmathsetmacro{\kdelX}{0.10}
\draw(#2)--++(\kpin,0)coordinate(\kref\kwest)--++(\kdelX,\kw/2)--++(\klen,0)coordinate[pos=0.5](\kref\knorth)--++(0,-\kw)coordinate[pos=0.5](\kref\keast)--++(-\klen,0)coordinate[pos=0.5](\kref\ksouth)--++(-\kdelX,\kw/2);
}
%========================================
\newcommand{\kinleftA}[2][j0]{
\def\kref{#1}
\def\keast{east}
\def\kwest{west}
\def\knorth{north}
\def\ksouth{south}
\pgfmathsetmacro{\kpin}{0.30}
\pgfmathsetmacro{\kw}{0.20}
\pgfmathsetmacro{\klen}{0.50}
\pgfmathsetmacro{\kdelX}{0.10}
\draw(#2)--++(-\kpin,0)coordinate(\kref\keast)--++(-\kdelX,\kw/2)--++(-\klen,0)coordinate[pos=0.5](\kref\knorth)--++(0,-\kw)coordinate[pos=0.5](\kref\kwest)--++(\klen,0)coordinate[pos=0.5](\kref\ksouth)--++(\kdelX,\kw/2);
}
%========================================
\newcommand{\kindownA}[2][j0]{
\def\kref{#1}
\def\keast{east}
\def\kwest{west}
\def\knorth{north}
\def\ksouth{south}
\pgfmathsetmacro{\kpin}{0.30}
\pgfmathsetmacro{\kw}{0.20}
\pgfmathsetmacro{\klen}{0.50}
\pgfmathsetmacro{\kdelX}{0.10}
\draw(#2)--++(0,-\kpin)coordinate(\kref\knorth)--++(\kw/2,-\kdelX)--++(0,-\klen)
coordinate[pos=0.5](\kref\keast)--++(-\kw,0)coordinate[pos=0.5](\kref\ksouth)
--++(0,\klen)coordinate[pos=0.5](\kref\kwest)--++(\kw/2,\kdelX);
}
%========================================
\newcommand{\kinupA}[2][j0]{
\def\kref{#1}
\def\keast{east}
\def\kwest{west}
\def\knorth{north}
\def\ksouth{south}
\pgfmathsetmacro{\kpin}{0.30}
\pgfmathsetmacro{\kw}{0.20}
\pgfmathsetmacro{\klen}{0.50}
\pgfmathsetmacro{\kdelX}{0.10}
\draw(#2)--++(0,\kpin)coordinate(\kref\ksouth)--++(-\kw/2,\kdelX)--++(0,\klen)coordinate[pos=0.5](\kref\kwest)--++(\kw,0)coordinate[pos=0.5](\kref\knorth)--++(0,-\klen)coordinate[pos=0.5](\kref\keast)--++(-\kw/2,-\kdelX);
}
%=====================================
%%%%%%%%%%%%%%%%%%%%%%%%%%%%%%%%
%+++++++++++++++++++++++++++++++++++++++++++++
%====================================
%draws NOTHING, only gives all pin related ANCHORS
% 

%   \kBOXa[u1]{xshift}{yshift}{x-pin}{y-pins}     %x-pins and y-pins are max pins expected along these edges

\newcommand{\kBOXa}[3][u0]{
\def\kref{#1}
\def\nnx{#2} 			%count of horizontal pins
\def\nny{#3}			%count of vertical pine
\pgfmathsetmacro{\nx}{\nnx-1}
\pgfmathsetmacro{\ny}{\nny-1}
\def\kshiftX{0}
\def\kshiftY{0}
\def\p{p}			%pin tip
\def\a{a}			%pin tip
\def\b{b}			%pin tip
\def\c{c}			%pin tip
\def\d{d}			%pin tip
\def\pb{pb}			%pin base
\def\pl{pl}			%pin label
\def\pn{pn}			%pin 'ocirc'
\def\px{x}			%pinAnchorX
\def\py{y}			%pinAnchorY
\def\pcu{pcu}			%clock edge top end
\def\pcm{pcm}			%clock edge mid
\def\pcd{pcd}			%clock edge bottom end
\pgfmathsetmacro{\klshift}{0.25}
\pgfmathsetmacro{\knshift}{0.07}
\pgfmathsetmacro{\kpin}{0.30}
\pgfmathsetmacro{\kpsep}{0.40}			%pin to pin distance
\pgfmathsetmacro{\kulVs}{0.50}			%South edge clearance along vertical edge
\pgfmathsetmacro{\kulVn}{0.30}		%North edge clearance along vertical edge
\pgfmathsetmacro{\kulHe}{0.40}		%East edge clearance along horizontal side
\pgfmathsetmacro{\kulHw}{0.50}		%West edge clearance along horizontal side
\pgfmathsetmacro{\kdimX}{\kulHe+\kulHw+\nx*\kpsep}
\pgfmathsetmacro{\kdimY}{\kulVn+\kulVs+\ny*\kpsep}		%two spaces between 3 pins
\def\cdelX{0.15}
\def\cdelY{0.15}
%%\def\leftPins{0/R/R,2/S/S}
%%\def\rightPins{0/{\overline{Q}}/QN,2/Q/Q}
%\draw[thick](\kshiftX,\kshiftY) rectangle ++(\kdimX,\kdimY);
\draw(\kshiftX,\kshiftY)coordinate(\kref-south-west);
\draw(\kshiftX+\kdimX,\kshiftY)coordinate(\kref-south-east);
\draw(\kshiftX+\kdimX,\kdimY+\kshiftY)coordinate(\kref-north-east);
\draw(\kshiftX,\kdimY+\kshiftY)coordinate(\kref-north-west);
\draw(\kshiftX+\kdimX/2,\kshiftY)coordinate(\kref-south);
\draw(\kshiftX+\kdimX/2,\kdimY+\kshiftY)coordinate(\kref-north);
\draw(\kshiftX,\kdimY/2+\kshiftY)coordinate(\kref-west);
\draw(\kshiftX+\kdimX,\kdimY/2+\kshiftY)coordinate(\kref-east);
\draw(\kshiftX+\kdimX/2,\kdimY/2+\kshiftY)coordinate(\kref-center);
%%left pins
\foreach \n in {0,...,\ny}{\draw(\kshiftX,\kshiftY+\kulVs+\n*\kpsep)
coordinate(\kref\a\pb\n)+(\klshift,0)coordinate(\kref\a\pl\n)+(-\knshift,0)coordinate(\kref\a\pn\n)+(-\kpin,0)coordinate(\kref\a\p\n);}
%%right pins
\foreach \n in {0,...,\ny}{\draw(\kshiftX+\kdimX,\kshiftY+\kulVs+\n*\kpsep)
coordinate(\kref\c\pb\n)+(-\klshift,0)coordinate(\kref\c\pl\n)+(\knshift,0)coordinate(\kref\c\pn\n)+(\kpin,0)coordinate(\kref\c\p\n);}
%%up pin
\foreach \n in {0,...,\nx}{\draw(\kshiftX+\kulHw+\n*\kpsep,\kdimY+\kshiftY)
coordinate(\kref\d\pb\n)+(0,-\klshift)coordinate(\kref\d\pl\n)+(0,\knshift)coordinate(\kref\d\pn\n)+(0,\kpin)coordinate(\kref\d\p\n);}
%%down pin
\foreach \n in {0,...,\nx}{\draw(\kshiftX+\kulHw+\n*\kpsep,\kshiftY)
coordinate(\kref\b\pb\n)+(0,\klshift)coordinate(\kref\b\pl\n)+(0,-\knshift)coordinate(\kref\b\pn\n)+(0,-\kpin)coordinate(\kref\b\p\n);}
%%clock edges left pin
\foreach \n in {0,...,\ny}{\draw(\kshiftX,\kshiftY+\kulVs+\n*\kpsep)+(0,-\cdelY)
coordinate(\kref\a\pcd\n)+(\cdelX,0)coordinate(\kref\a\pcm\n)+(0,\cdelY)coordinate(\kref\a\pcu\n);}
%%clock edges right pin
\foreach \n in {0,...,\ny}{\draw(\kshiftX+\kdimX,\kshiftY+\kulVs+\n*\kpsep)
+(0,-\cdelY)coordinate(\kref\c\pcd\n)+(-\cdelX,0)coordinate(\kref\c\pcm\n)+(0,\cdelY)coordinate(\kref\c\pcu\n);}
%%clock edges up pin
\foreach \n in {0,...,\nx}{\draw(\kshiftX+\kulHw+\n*\kpsep,\kshiftY+\kdimY)+(-\cdelY,0)
coordinate(\kref\d\pcd\n)+(0,-\cdelX)coordinate(\kref\d\pcm\n)+(\cdelY,0)coordinate(\kref\d\pcu\n);}
%%clock edges down pin
\foreach \n in {0,...,\nx}{\draw(\kshiftX+\kulHw+\n*\kpsep,\kshiftY)+(-\cdelY,0)
coordinate(\kref\b\pcd\n)+(0,\cdelX)coordinate(\kref\b\pcm\n)+(\cdelY,0)coordinate(\kref\b\pcu\n);}
}
%%%%%%%%%%%%%%%%%%%%%%%%
%===================================
%should take input limits and strings for the four sides
\newcommand{\kBOXanc}{
\pgfmathsetmacro{\kpin}{0.30}
\pgfmathsetmacro{\kpsep}{0.40}			%pin to pin distance
\pgfmathsetmacro{\kulVs}{0.50}			%South edge clearance along vertical edge
\pgfmathsetmacro{\kulVn}{0.30}		%North edge clearance along vertical edge
\pgfmathsetmacro{\kulHe}{0.40}		%East edge clearance along horizontal side
\pgfmathsetmacro{\kulHw}{0.50}		%West edge clearance along horizontal side
%%left pins
\pgfmathsetmacro\spax{\kulHw+0*\kpsep}
\pgfmathsetmacro\spbx{\kulHw+1*\kpsep}
\pgfmathsetmacro\spcx{\kulHw+2*\kpsep}
\pgfmathsetmacro\spdx{\kulHw+3*\kpsep}
\pgfmathsetmacro\spex{\kulHw+4*\kpsep}
\pgfmathsetmacro\spfx{\kulHw+5*\kpsep}
\pgfmathsetmacro\spgx{\kulHw+6*\kpsep}
\pgfmathsetmacro\wpay{\kulVs+0*\kpsep}
\pgfmathsetmacro\wpby{\kulVs+1*\kpsep}
\pgfmathsetmacro\wpcy{\kulVs+2*\kpsep}
\pgfmathsetmacro\wpdy{\kulVs+3*\kpsep}
\pgfmathsetmacro\wpey{\kulVs+4*\kpsep}
\pgfmathsetmacro\wpfy{\kulVs+5*\kpsep}
\pgfmathsetmacro\wpgy{\kulVs+6*\kpsep}
\pgfmathsetmacro\wphy{\kulVs+7*\kpsep}
}
%=============================
%draws general purpose BOX SKELETON ONLY
%labels, pins, and clock edge needs to be added separately
%can be anchored with pin at specific coordinate(x,y) as explained down
%%   \kBOX[u1]{xshift}{yshift}{x-pins}{y-pins}
%   to anchor pin2 at location (x,y) use
%	\kBOX[u1]{x-\pbx}{y-\pby}{x-pins}{y-pins}
%pin 1,2,3,4,5,6,u,d are in this context written \pba, ...\pbf,\pbup,\pbdown

\newcommand{\kBOX}[5][u0]{
\def\kref{#1}
\def\kshiftX{#2}
\def\kshiftY{#3}
\def\nnx{#4}
\def\nny{#5}
\pgfmathsetmacro{\nx}{\nnx-1}
\pgfmathsetmacro{\ny}{\nny-1}
\def\p{p}			%pin tip
\def\a{a}			%pin tip
\def\b{b}			%pin tip
\def\c{c}			%pin tip
\def\d{d}			%pin tip
\def\pb{pb}			%pin base
\def\pl{pl}			%pin label
\def\pcu{pcu}			%clock edge, upper corner
\def\pcm{pcm}			%clock edge middle corner
\def\pcd{pcd}			%clock edge lower corner
\def\pn{pn}			%pin 'ocirc'
\pgfmathsetmacro{\klshift}{0.25}
\pgfmathsetmacro{\knshift}{0.07}
\pgfmathsetmacro{\kpin}{0.30}
\pgfmathsetmacro{\kpsep}{0.40}			%pin to pin distance
\pgfmathsetmacro{\kulVs}{0.50}			%South edge clearance along vertical edge
\pgfmathsetmacro{\kulVn}{0.30}		%North edge clearance along vertical edge
\pgfmathsetmacro{\kulHe}{0.40}		%East edge clearance along horizontal side
\pgfmathsetmacro{\kulHw}{0.50}		%West edge clearance along horizontal side
\pgfmathsetmacro{\kdimX}{\kulHe+\kulHe+\nx*\kpsep}
\pgfmathsetmacro{\kdimY}{\kulVs+\kulVn+\ny*\kpsep}		%two spaces between 3 pins
\def\cdelX{0.15}
\def\cdelY{0.15}
%%\def\leftPins{0/R/R,2/S/S}
%%\def\rightPins{0/{\overline{Q}}/QN,2/Q/Q}
\draw[thick](\kshiftX,\kshiftY) rectangle ++(\kdimX,\kdimY);
\draw(\kshiftX,\kshiftY)coordinate(\kref-south-west);
\draw(\kshiftX+\kdimX,\kshiftY)coordinate(\kref-south-east);
\draw(\kshiftX+\kdimX,\kdimY+\kshiftY)coordinate(\kref-north-east);
\draw(\kshiftX,\kdimY+\kshiftY)coordinate(\kref-north-west);
\draw(\kshiftX+\kdimX/2,\kshiftY)coordinate(\kref-south);
\draw(\kshiftX+\kdimX/2,\kdimY+\kshiftY)coordinate(\kref-north);
\draw(\kshiftX,\kdimY/2+\kshiftY)coordinate(\kref-west);
\draw(\kshiftX+\kdimX,\kdimY/2+\kshiftY)coordinate(\kref-east);
\draw(\kshiftX+\kdimX/2,\kdimY/2+\kshiftY)coordinate(\kref-center);
%%left pins
\foreach \n in {0,...,\ny}{\draw(\kshiftX,\kshiftY+\kulVs+\n*\kpsep)
coordinate(\kref\a\pb\n)+(\klshift,0)coordinate(\kref\a\pl\n)+(-\knshift,0)coordinate(\kref\a\pn\n)+(-\kpin,0)coordinate(\kref\a\p\n);}
%%right pins
\foreach \n in {0,...,\ny}{\draw(\kshiftX+\kdimX,\kshiftY+\kulVs+\n*\kpsep)
coordinate(\kref\c\pb\n)+(-\klshift,0)coordinate(\kref\c\pl\n)+(\knshift,0)coordinate(\kref\c\pn\n)+(\kpin,0)coordinate(\kref\c\p\n);}
%%up pin
\foreach \n in {0,...,\nx}{\draw(\kshiftX+\kulHw+\n*\kpsep,\kdimY+\kshiftY)
coordinate(\kref\d\pb\n)+(0,-\klshift)coordinate(\kref\d\pl\n)+(0,\knshift)coordinate(\kref\d\pn\n)+(0,\kpin)coordinate(\kref\d\p\n);}
%%down pin
\foreach \n in {0,...,\nx}{\draw(\kshiftX+\kulHw+\n*\kpsep,\kshiftY)
coordinate(\kref\b\pb\n)+(0,\klshift)coordinate(\kref\b\pl\n)+(0,-\knshift)coordinate(\kref\b\pn\n)+(0,-\kpin)coordinate(\kref\b\p\n);}
%%clock edges left pin
\foreach \n in {0,...,\ny}{\draw(\kshiftX,\kshiftY+\kulVs+\n*\kpsep)+(0,-\cdelY)
coordinate(\kref\a\pcd\n)+(\cdelX,0)coordinate(\kref\a\pcm\n)+(0,\cdelY)coordinate(\kref\a\pcu\n);}
%%clock edges right pin
\foreach \n in {0,...,\ny}{\draw(\kshiftX+\kdimX,\kshiftY+\kulVs+\n*\kpsep)
+(0,-\cdelY)coordinate(\kref\c\pcd\n)+(-\cdelX,0)coordinate(\kref\c\pcm\n)+(0,\cdelY)coordinate(\kref\c\pcu\n);}
%%clock edges up pin
\foreach \n in {0,...,\nx}{\draw(\kshiftX+\kulHw+\n*\kpsep,\kshiftY+\kdimY)+(-\cdelY,0)
coordinate(\kref\d\pcd\n)+(0,-\cdelX)coordinate(\kref\d\pcm\n)+(\cdelY,0)coordinate(\kref\d\pcu\n);}
%%clock edges down pin
\foreach \n in {0,...,\nx}{\draw(\kshiftX+\kulHw+\n*\kpsep,\kshiftY)+(-\cdelY,0)
coordinate(\kref\b\pcd\n)+(0,\cdelX)coordinate(\kref\b\pcm\n)+(\cdelY,0)coordinate(\kref\b\pcu\n);}
}
%%%%%%%%%%%%%%%%%%%%%%%%%%%%%%%
\newcommand{\kmuxD}[3][u0]{
\def\kref{#1}
\kBOX[#1]{#2}{#3}{4}{2}
\foreach \n/\m in{0/3,1/2,2/1,3/0}{\draw(\kref\b\pb\n)--(\kref\b\p\n) (\kref\b\pl\n)node[]{$\m$};}
\foreach \n/\m in{0/{a_0},1/{a_1}}{\draw(\kref\a\pb\n)--(\kref\a\p\n) (\kref\a\pl\n)node[]{\small{$\m$}};}
\foreach \n in{2}{\draw(\kref\d\pb\n)--(\kref\d\p\n);}
}
%================================
%%%%%%%%%%%%%%%%%%%%%%%%
% OR gate  skeleton for use with  input BUS. Input can be accessed with say u0in and output with u0out
%\kBusOR[u1]{Xlocation}{Ylocation}
\newcommand{\kBusOR}[3][u0]{{
\def\kref{#1}
\def\kshiftX{#2}
\def\kshiftY{#3}
\def\kin{in}
\def\kout{out}
\def\p{p}			%pin tip
\def\pb{pb}			%pin base
\def\pl{pl}			%pin label
\def\pn{pn}			%pin 'ocirc'
\pgfmathsetmacro{\kxdim}{1}
\pgfmathsetmacro{\kydim}{1}
\draw[thick] (\kshiftX,\kshiftY) to [out=45,in=-45]coordinate[pos=0.5](\kref\kin) ++(0,\kydim) to [out=0,in=130] ++(\kxdim,-0.5*\kydim)coordinate(\kref\kout) to [out=-130,in=0]++(-\kxdim,-0.5*\kydim);}}
%%%%%%%%%%%%%%%%%%%%%%%%%%%%
\newcommand{\kBusBuffer}[3][u0]{{
\def\kref{#1}
\def\kshiftX{#2}
\def\kshiftY{#3}
\def\kin{in}
\def\kout{out}
\def\p{p}			%pin tip
\def\pb{pb}			%pin base
\def\pl{pl}			%pin label
\def\pn{pn}			%pin 'ocirc'
\def\pu{pu}
\def\pd{pd}
\pgfmathsetmacro{\klen}{1}
\draw[thick](\kshiftX,\kshiftY)coordinate(aa)++(0,-0.5*\klen)--++(30:\klen)
coordinate[pos=0.6](keda)--++(150:\klen)coordinate[pos=0.4](ked)--++(-90:\klen);
\draw[thin](ked)--++(\kpin,0)coordinate(\kref\pu)  (keda)--++(\kpin,0)coordinate(\kref\pd) 
(aa)--++(-\kpin,0)coordinate(\kref\kin);
\draw[thick]  (ked)++(0.07,0)node[ocirc,fill]{};
}}
%====================================
%%%%%%%%%%%%%%%%%%%%%%%%%%%%
%draws a cross at the location to show intact fuses in non volatile memory
\newcommand{\kCross}[1]{{
\pgfmathsetmacro{\kmv}{0.15}
 \draw(#1)++(-0.5*\kmv,-0.5*\kmv)--++(\kmv,\kmv);
 \draw(#1)++(-0.5*\kmv,0.5*\kmv)--++(\kmv,-\kmv);
}}

%=============================
%gives pin tip's locations as (pax,pay), pins are pa,pb,...,pf, pup,pdown corresponding to p1,..,p6,pu,pd
\newcommand{\kupinAnchors}{
\def\pay{-\kpin}
\def\pax{-\kulV-2*\kpsep}
\def\pby{-\kpin}
\def\pbx{-\kulV-1*\kpsep}
\def\pcy{-\kpin}
\def\pcx{-\kulV-0*\kpsep}
\def\pdy{\kdimX+\kpin}
\def\pdx{-\kulV-0*\kpsep}
\def\pey{\kdimX+\kpin}
\def\pex{-\kulV-1*\kpsep}
\def\pfy{\kdimX+\kpin}
\def\pfx{-\kulV-2*\kpsep}
\def\pupy{\kdimX/2}
\def\pupx{-\kpin-\kdimY}
\def\pdowny{\kdimX/2}
\def\pdownx{\kpin}
\def\udimFFx{\kdimY}
\def\udimFFy{\kdimX}
}
%====================================
%=============================
%draws general purpose FF's SKELETON ONLY
%labels, pins, clock edge needs to be added separately
%can be anchored with pin at specific coordinate(x,y) as explained down
%%   \kSRFF[u1]{xshift}{yshift}   
%   to anchor pin2 at location (x,y) use
%	\kSRFF[u1]{x-\pbx}{y-\pby}
%pin 1,2,3,4,5,6,u,d are in this context written \pba, ...\pbf,\pbup,\pbdown
% can place text at say "kref-north-east" where north-east is as viewed from the base of the FF
\newcommand{\kuFF}[3][u0]{
\def\kref{#1}
\def\kshiftX{#2}
\def\kshiftY{#3}

\def\p{p}			%pin tip
\def\pb{pb}			%pin base
\def\pl{pl}			%pin label
\def\pcu{pcu}			%clock edge, upper corner
\def\pcm{pcm}			%clock edge middle corner
\def\pcd{pcd}			%clock edge lower corner
\def\pn{pn}			%pin 'ocirc'
\pgfmathsetmacro{\klshift}{0.25}
\pgfmathsetmacro{\knshift}{0.07}
\pgfmathsetmacro{\kpin}{0.30}
\pgfmathsetmacro{\kpsep}{0.40}			%pin to pin distance
\pgfmathsetmacro{\kulV}{0.40}			%edge clearance along vertical edge
\pgfmathsetmacro{\kulH}{0.50}
\pgfmathsetmacro{\kdimX}{2*\kulH+0*\kpsep}
\pgfmathsetmacro{\kdimY}{2*\kulV+2*\kpsep}		%two spaces between 3 pins
\def\cdelX{0.15}
\def\cdelY{0.15}
%%\def\leftPins{0/R/R,2/S/S}
%%\def\rightPins{0/{\overline{Q}}/QN,2/Q/Q}
\draw[thick](\kshiftX,\kshiftY) rectangle ++(-\kdimY,\kdimX);
\draw(\kshiftX,\kshiftY)coordinate(\kref-south-west);
\draw(\kshiftX,\kshiftY+\kdimX)coordinate(\kref-south-east);
\draw(\kshiftX-\kdimY,\kdimX+\kshiftY)coordinate(\kref-north-east);
\draw(\kshiftX-\kdimY,\kshiftY)coordinate(\kref-north-west);
\draw(\kshiftX,\kdimX/2+\kshiftY)coordinate(\kref-south);
\draw(\kshiftX-\kdimY,\kdimX/2+\kshiftY)coordinate(\kref-north);
\draw(\kshiftX-\kdimY/2,\kshiftY)coordinate(\kref-west);
\draw(\kshiftX-\kdimY/2,\kdimX+\kshiftY)coordinate(\kref-east);
\draw(\kshiftX-\kdimY/2,\kdimX/2+\kshiftY)coordinate(\kref-center);
%%left pins
\foreach \n/\m in {0/3,1/2,2/1}{\draw(\kshiftX-\kulV-\n*\kpsep,\kshiftY)
coordinate(\kref\pb\m)+(0,\klshift)coordinate(\kref\pl\m)+(0,-\knshift)
coordinate(\kref\pn\m)+(0,-\kpin)coordinate(\kref\p\m);}
%%right pins
\foreach \n/\m in {0/4,1/5,2/6}{\draw(\kshiftX-\kulV-\n*\kpsep,\kdimX+\kshiftY)
coordinate(\kref\pb\m)+(0,-\klshift)coordinate(\kref\pl\m)+(0,\knshift)
coordinate(\kref\pn\m)+(0,\kpin)coordinate(\kref\p\m);}
%%up pin
\foreach \n/\m in {0/u}{\draw(\kshiftX-\kdimY,+\kulH+\n*\kpsep+\kshiftY)
coordinate(\kref\pb\m)+(\klshift,0)coordinate(\kref\pl\m)+(-\knshift,0)
coordinate(\kref\pn\m)+(-\kpin,0)coordinate(\kref\p\m);}
%%down pin
\foreach \n/\m in {0/d}{\draw(\kshiftX,\kulH+\n*\kpsep+\kshiftY)
coordinate(\kref\pb\m)+(-\klshift,0)coordinate(\kref\pl\m)+(\knshift,0)
coordinate(\kref\pn\m)+(\kpin,0)coordinate(\kref\p\m);}
%%clock edges left pin
\foreach \n/\m in {0/3,1/2,2/1}{\draw(\kshiftX-\kulV-\n*\kpsep,\kshiftY)+(\cdelY,0)
coordinate(\kref\pcd\m)+(0,\cdelX)coordinate(\kref\pcm\m)+(-\cdelY,0)coordinate(\kref\pcu\m);}
%%clock edges right pin
\foreach \n/\m in {0/4,1/5,2/6}{\draw(\kshiftX-\kulV-\n*\kpsep,\kdimX+\kshiftY)
+(\cdelY,0)coordinate(\kref\pcd\m)+(0,-\cdelX)coordinate(\kref\pcm\m)+(-\cdelY,0)coordinate(\kref\pcu\m);}
}
%%%%%%%%%%%%%%%%%%%%%%%%%%%%%%%

%   to anchor pin2 at location (x,y) use
%	\kgenSRFF[u1]{x-\pbx}{y-\pby}
%pin 1,2,3,4,5,6,u,d are in this context written \pba, ...\pbf,\pbup,\pbdown

\newcommand{\kuSRFF}[3][u0]{
\kuFF[#1]{#2}{#3}
\def\kref{#1}
%\def\p{p}			%pin tip
%\def\pb{pb}			%pin base
%\def\pl{pl}			%pin label
%\def\pcu{pcu}			%pin edge, upper corner
%\def\pcm{pcm}			%pin edge middle corner
%\def\pcd{pcd}			%pin edge lower corner
%\def\pn{pn}			%pin 'ocirc'
\foreach \n in {1,2,3,4,6}{\draw(\kref\pb\n)--(\kref\p\n);}
\foreach \n /\lbl in {1/S,2/C,3/R,4/{\overline{Q}},6/Q}{\draw(\kref\pl\n)node[]{$\lbl$};}
\foreach \n in {4}{\draw(\kref\pn\n)node[ocirc]{};}
}
%%%%%%%%%%%%%%%%%%%%%%%%%%%%%%%
\newcommand{\kuDFF}[3][u0]{
\def\kref{#1}
%\def\p{p}			%pin tip
%\def\pb{pb}			%pin base
%\def\pl{pl}			%pin label
%\def\pcu{pcu}			%pin edge, upper corner
%\def\pcm{pcm}			%pin edge middle corner
%\def\pcd{pcd}			%pin edge lower corner
%\def\pn{pn}			%pin 'ocirc'
\kuFF[#1]{#2}{#3}
\foreach \n in {1,2,4,6}{\draw(\kref\pb\n)--(\kref\p\n);}
\foreach \n/\lbl in {1/D,4/{\overline{Q}},6/Q}{\draw(\kref\pl\n)node[]{$\lbl$};}
\foreach \n in {2}{\draw(\kref\pcd\n)--(\kref\pcm\n)--(\kref\pcu\n);}
\foreach \n in {4}{\draw(\kref\pn\n)node[ocirc]{};}
}
%%%%%%%%%%%%%%%%%%%%%%%%%%%%%%%
%%%%%%%%%%%%%%%%%%%%%%%%%%%%%%%
\newcommand{\kuDFFud}[3][u0]{
\def\kref{#1}
%\def\p{p}			%pin tip
%\def\pb{pb}			%pin base
%\def\pl{pl}			%pin label
%\def\pcu{pcu}			%pin edge, upper corner
%\def\pcm{pcm}			%pin edge middle corner
%\def\pcd{pcd}			%pin edge lower corner
%\def\pn{pn}			%pin 'ocirc'
\kuFF[#1]{#2}{#3}
\foreach \n in {1,2,4,6,u,d}{\draw(\kref\pb\n)--(\kref\p\n);}
\foreach \n/\lbl in {1/D,4/{\overline{Q}},6/Q}{\draw(\kref\pl\n)node[]{$\lbl$};}
\foreach \n in {2}{\draw(\kref\pcd\n)--(\kref\pcm\n)--(\kref\pcu\n);}
\foreach \n in {4,u,d}{\draw(\kref\pn\n)node[ocirc]{};}
}
%%%%%%%%%%%%%%%%%%%%%%%%%%%%%%%
\newcommand{\kuJKFF}[3][u0]{
\def\kref{#1}
%\def\p{p}			%pin tip
%\def\pb{pb}			%pin base
%\def\pl{pl}			%pin label
%\def\pcu{pcu}			%pin edge, upper corner
%\def\pcm{pcm}			%pin edge middle corner
%\def\pcd{pcd}			%pin edge lower corner
%\def\pn{pn}			%pin 'ocirc'
\kuFF[#1]{#2}{#3}
\foreach \n in {1,2,3,4,6}{\draw(\kref\pb\n)--(\kref\p\n);}
\foreach \n/\lbl in {1/J,3/K,4/{\overline{Q}},6/Q}{\draw(\kref\pl\n)node[]{$\lbl$};}
\foreach \n in {2}{\draw(\kref\pcd\n)--(\kref\pcm\n)--(\kref\pcu\n);}
\foreach \n in {4}{\draw(\kref\pn\n)node[ocirc]{};}
}
%%%%%%%%%%%%%%%%%%%%%%
%%%%%%%%%%%%%%%%%%%%%%%%%%%%%%%
\newcommand{\kuTFF}[3][u0]{
\def\kref{#1}
%\def\p{p}			%pin tip
%\def\pb{pb}			%pin base
%\def\pl{pl}			%pin label
%\def\pcu{pcu}			%pin edge, upper corner
%\def\pcm{pcm}			%pin edge middle corner
%\def\pcd{pcd}			%pin edge lower corner
%\def\pn{pn}			%pin 'ocirc'
\kuFF[#1]{#2}{#3}
\foreach \n in {1,2,4,6}{\draw(\kref\pb\n)--(\kref\p\n);}
\foreach \n/\lbl in {1/T,4/{\overline{Q}},6/Q}{\draw(\kref\pl\n)node[]{$\lbl$};}
\foreach \n in {2}{\draw(\kref\pcd\n)--(\kref\pcm\n)--(\kref\pcu\n);}
\foreach \n in {4}{\draw(\kref\pn\n)node[ocirc]{};}
}
%%%%%%%%%%%%%%%%%%%%%
%%%%%%%%%%%%%%%%%%%%%%
\newcommand{\kuTFFud}[3][u0]{
\def\kref{#1}
%\def\p{p}			%pin tip
%\def\pb{pb}			%pin base
%\def\pl{pl}			%pin label
%\def\pcu{pcu}			%pin edge, upper corner
%\def\pcm{pcm}			%pin edge middle corner
%\def\pcd{pcd}			%pin edge lower corner
%\def\pn{pn}			%pin 'ocirc'
\kuFF[#1]{#2}{#3}
\foreach \n in {1,2,4,6,u,d}{\draw(\kref\pb\n)--(\kref\p\n);}
\foreach \n/\lbl in {1/T,4/{\overline{Q}},6/Q}{\draw(\kref\pl\n)node[]{$\lbl$};}
\foreach \n in {2}{\draw(\kref\pcd\n)--(\kref\pcm\n)--(\kref\pcu\n);}
\foreach \n in {4,u,d}{\draw(\kref\pn\n)node[ocirc]{};}
}
%%%%%%%%%%%%%%%%%%%%%
%%%%%%%%%%%%%%%%%%%%%%
\newcommand{\kuTFFd}[3][u0]{
\def\kref{#1}
%\def\p{p}			%pin tip
%\def\pb{pb}			%pin base
%\def\pl{pl}			%pin label
%\def\pcu{pcu}			%pin edge, upper corner
%\def\pcm{pcm}			%pin edge middle corner
%\def\pcd{pcd}			%pin edge lower corner
%\def\pn{pn}			%pin 'ocirc'
\kuFF[#1]{#2}{#3}
\foreach \n in {1,2,4,6,d}{\draw(\kref\pb\n)--(\kref\p\n);}
\foreach \n/\lbl in {1/T,4/{\overline{Q}},6/Q}{\draw(\kref\pl\n)node[]{$\lbl$};}
\foreach \n in {2}{\draw(\kref\pcd\n)--(\kref\pcm\n)--(\kref\pcu\n);}
\foreach \n in {4,d}{\draw(\kref\pn\n)node[ocirc]{};}
}
%%%%%%%%%%%%%%%%%%%%%
%%%%%%%%%%%%%%%%%%%%%%


\newcommand{\kuBuffer}[3][u0]{
\def\kref{#1}
\def\kshiftX{#2}
\def\kshiftY{#3}

\def\p{p}			%pin tip
\def\pb{pb}			%pin base
\def\pl{pl}			%pin label
\def\pn{pn}			%pin 'ocirc'
\def\pin{pin}			%pin tip
\def\pout{pout}			%pin tip
\def\pu{u}			%pin tip
\def\pd{d}			%pin tip
\pgfmathsetmacro{\klshift}{0.25}
\pgfmathsetmacro{\knshift}{0.07}
\pgfmathsetmacro{\kpin}{0.30}
\pgfmathsetmacro{\kpsep}{0.40}			%pin to pin distance
\pgfmathsetmacro{\kulV}{0.40}			%edge clearance along vertical edge
\pgfmathsetmacro{\kulH}{0.50}
\pgfmathsetmacro{\kdimX}{2*\kulH+0*\kpsep}
\pgfmathsetmacro{\kdimY}{\kdimX}		%two spaces between 3 pins
\draw(\kshiftX,\kshiftY)coordinate(\kref\pin)--++(0,\kpin)coordinate(\kref\pb\pin);
\draw(\kref\pb\pin)[thick]++(\kdimY/2,0)coordinate(aa)--++(-\kdimY,0)
coordinate(bb)--++(\kdimY/2,\kdimX)coordinate(cc)coordinate(\kref\pb\pout)--++(\kdimY/2,-\kdimX);
\draw(\kref\pb\pout)--++(0,\kpin)coordinate(\kref\pout);
\draw($(bb)!0.5!(cc)$)coordinate(\kref\pb\pu)+(-0.07,0)coordinate(\kref\pn\pu)+(-\kpin,0)coordinate(\kref\pu);
\draw($(aa)!0.5!(cc)$)coordinate(\kref\pb\pd)+(0.07,0)coordinate(\kref\pn\pd)+(\kpin,0)coordinate(\kref\pd);
\draw(\kref\pb\pin)++(0,-0.07)coordinate(\kref\pn\pin);
\draw(\kref\pb\pout)++(0,0.07)coordinate(\kref\pn\pout);
}

%=============================
%gives pin tip's locations as (pax,pay), pins are pa,pb,...,pf, pup,pdown corresponding to p1,..,p6,pu,pd
\newcommand{\kdpinAnchors}{
\def\pay{\kpin}
\def\pax{\kulV+2*\kpsep}
\def\pby{\kpin}
\def\pbx{\kulV+1*\kpsep}
\def\pcy{\kpin}
\def\pcx{\kulV+0*\kpsep}
\def\pdy{-\kdimX-\kpin}
\def\pdx{\kulV+0*\kpsep}
\def\pey{-\kdimX-\kpin}
\def\pex{\kulV+1*\kpsep}
\def\pfy{-\kdimX-\kpin}
\def\pfx{\kulV+2*\kpsep}
\def\pupy{-\kdimX/2}
\def\pupx{\kpin+\kdimY}
\def\pdowny{-\kdimX/2}
\def\pdownx{-\kpin}
}
%====================================
%=============================
%draws general purpose FF's SKELETON ONLY
%labels, pins, clock edge needs to be added separately
%can be anchored with pin at specific coordinate(x,y) as explained down
%%   \kSRFF[u1]{xshift}{yshift}   
%   to anchor pin2 at location (x,y) use
%	\kSRFF[u1]{x-\pbx}{y-\pby}
%pin 1,2,3,4,5,6,u,d are in this context written \pba, ...\pbf,\pbup,\pbdown

\newcommand{\kdFF}[3][u0]{
\def\kref{#1}
\def\kshiftX{#2}
\def\kshiftY{#3}

\def\p{p}			%pin tip
\def\pb{pb}			%pin base
\def\pl{pl}			%pin label
\def\pcu{pcu}			%clock edge, upper corner
\def\pcm{pcm}			%clock edge middle corner
\def\pcd{pcd}			%clock edge lower corner
\def\pn{pn}			%pin 'ocirc'
\pgfmathsetmacro{\klshift}{0.25}
\pgfmathsetmacro{\knshift}{0.07}
\pgfmathsetmacro{\kpin}{0.30}
\pgfmathsetmacro{\kpsep}{0.40}			%pin to pin distance
\pgfmathsetmacro{\kulV}{0.40}			%edge clearance along vertical edge
\pgfmathsetmacro{\kulH}{0.50}
\pgfmathsetmacro{\kdimX}{2*\kulH+0*\kpsep}
\pgfmathsetmacro{\kdimY}{2*\kulV+2*\kpsep}		%two spaces between 3 pins
\def\cdelX{0.15}
\def\cdelY{0.15}
%%\def\leftPins{0/R/R,2/S/S}
%%\def\rightPins{0/{\overline{Q}}/QN,2/Q/Q}
\draw[thick](\kshiftX,\kshiftY) rectangle ++(\kdimY,-\kdimX);
\draw(\kshiftX,\kshiftY)coordinate(\kref-south-west);
\draw(\kshiftX,\kshiftY-\kdimX)coordinate(\kref-south-east);
\draw(\kshiftX+\kdimY,-\kdimX+\kshiftY)coordinate(\kref-north-east);
\draw(\kshiftX+\kdimY,\kshiftY)coordinate(\kref-north-west);
\draw(\kshiftX,-\kdimX/2+\kshiftY)coordinate(\kref-south);
\draw(\kshiftX+\kdimY,-\kdimX/2+\kshiftY)coordinate(\kref-north);
\draw(\kshiftX+\kdimY/2,\kshiftY)coordinate(\kref-west);
\draw(\kshiftX+\kdimY/2,-\kdimX+\kshiftY)coordinate(\kref-east);
\draw(\kshiftX+\kdimY/2,-\kdimX/2+\kshiftY)coordinate(\kref-center);
%%left pins
\foreach \n/\m in {0/3,1/2,2/1}{\draw(\kshiftX+\kulV+\n*\kpsep,\kshiftY)
coordinate(\kref\pb\m)+(0,-\klshift)coordinate(\kref\pl\m)+(0,\knshift)
coordinate(\kref\pn\m)+(0,\kpin)coordinate(\kref\p\m);}
%%right pins
\foreach \n/\m in {0/4,1/5,2/6}{\draw(\kshiftX+\kulV+\n*\kpsep,-\kdimX+\kshiftY)
coordinate(\kref\pb\m)+(0,\klshift)coordinate(\kref\pl\m)+(0,-\knshift)
coordinate(\kref\pn\m)+(0,-\kpin)coordinate(\kref\p\m);}
%%up pin
\foreach \n/\m in {0/u}{\draw(\kshiftX+\kdimY,-\kulH-\n*\kpsep+\kshiftY)
coordinate(\kref\pb\m)+(-\klshift,0)coordinate(\kref\pl\m)+(\knshift,0)
coordinate(\kref\pn\m)+(\kpin,0)coordinate(\kref\p\m);}
%%down pin
\foreach \n/\m in {0/d}{\draw(\kshiftX,-\kulH-\n*\kpsep+\kshiftY)
coordinate(\kref\pb\m)+(\klshift,0)coordinate(\kref\pl\m)+(-\knshift,0)
coordinate(\kref\pn\m)+(-\kpin,0)coordinate(\kref\p\m);}
%%clock edges left pin
\foreach \n/\m in {0/3,1/2,2/1}{\draw(\kshiftX+\kulV+\n*\kpsep,\kshiftY)+(-\cdelY,0)
coordinate(\kref\pcd\m)+(0,-\cdelX)coordinate(\kref\pcm\m)+(\cdelY,0)coordinate(\kref\pcu\m);}
%%clock edges right pin
\foreach \n/\m in {0/4,1/5,2/6}{\draw(\kshiftX+\kulV+\n*\kpsep,-\kdimX+\kshiftY)
+(-\cdelY,0)coordinate(\kref\pcd\m)+(0,\cdelX)coordinate(\kref\pcm\m)+(\cdelY,0)coordinate(\kref\pcu\m);}
}
%%%%%%%%%%%%%%%%%%%%%%%%%%%%%%%

%   to anchor pin2 at location (x,y) use
%	\kgenSRFF[u1]{x-\pbx}{y-\pby}
%pin 1,2,3,4,5,6,u,d are in this context written \pba, ...\pbf,\pbup,\pbdown

\newcommand{\kdSRFF}[3][u0]{
\kdFF[#1]{#2}{#3}
\def\kref{#1}
%\def\p{p}			%pin tip
%\def\pb{pb}			%pin base
%\def\pl{pl}			%pin label
%\def\pcu{pcu}			%pin edge, upper corner
%\def\pcm{pcm}			%pin edge middle corner
%\def\pcd{pcd}			%pin edge lower corner
%\def\pn{pn}			%pin 'ocirc'
\foreach \n in {1,2,3,4,6}{\draw(\kref\pb\n)--(\kref\p\n);}
\foreach \n /\lbl in {1/S,2/C,3/R,4/{\overline{Q}},6/Q}{\draw(\kref\pl\n)node[]{$\lbl$};}
\foreach \n in {4}{\draw(\kref\pn\n)node[ocirc]{};}
}
%%%%%%%%%%%%%%%%%%%%%%%%%%%%%%%
\newcommand{\kdDFF}[3][u0]{
\def\kref{#1}
%\def\p{p}			%pin tip
%\def\pb{pb}			%pin base
%\def\pl{pl}			%pin label
%\def\pcu{pcu}			%pin edge, upper corner
%\def\pcm{pcm}			%pin edge middle corner
%\def\pcd{pcd}			%pin edge lower corner
%\def\pn{pn}			%pin 'ocirc'
\kdFF[#1]{#2}{#3}
\foreach \n in {1,2,4,6}{\draw(\kref\pb\n)--(\kref\p\n);}
\foreach \n/\lbl in {1/D,4/{\overline{Q}},6/Q}{\draw(\kref\pl\n)node[]{$\lbl$};}
\foreach \n in {2}{\draw(\kref\pcd\n)--(\kref\pcm\n)--(\kref\pcu\n);}
\foreach \n in {4}{\draw(\kref\pn\n)node[ocirc]{};}
}
%%%%%%%%%%%%%%%%%%%%%%%%%%%%%%%
%%%%%%%%%%%%%%%%%%%%%%%%%%%%%%%
\newcommand{\kdDFFud}[3][u0]{
\def\kref{#1}
%\def\p{p}			%pin tip
%\def\pb{pb}			%pin base
%\def\pl{pl}			%pin label
%\def\pcu{pcu}			%pin edge, upper corner
%\def\pcm{pcm}			%pin edge middle corner
%\def\pcd{pcd}			%pin edge lower corner
%\def\pn{pn}			%pin 'ocirc'
\kdFF[#1]{#2}{#3}
\foreach \n in {1,2,4,6,u,d}{\draw(\kref\pb\n)--(\kref\p\n);}
\foreach \n/\lbl in {1/D,4/{\overline{Q}},6/Q}{\draw(\kref\pl\n)node[]{$\lbl$};}
\foreach \n in {2}{\draw(\kref\pcd\n)--(\kref\pcm\n)--(\kref\pcu\n);}
\foreach \n in {4,u,d}{\draw(\kref\pn\n)node[ocirc]{};}
}
%%%%%%%%%%%%%%%%%%%%%%%%%%%%%%%
\newcommand{\kdJKFF}[3][u0]{
\def\kref{#1}
%\def\p{p}			%pin tip
%\def\pb{pb}			%pin base
%\def\pl{pl}			%pin label
%\def\pcu{pcu}			%pin edge, upper corner
%\def\pcm{pcm}			%pin edge middle corner
%\def\pcd{pcd}			%pin edge lower corner
%\def\pn{pn}			%pin 'ocirc'
\kdFF[#1]{#2}{#3}
\foreach \n in {1,2,3,4,6}{\draw(\kref\pb\n)--(\kref\p\n);}
\foreach \n/\lbl in {1/J,3/K,4/{\overline{Q}},6/Q}{\draw(\kref\pl\n)node[]{$\lbl$};}
\foreach \n in {2}{\draw(\kref\pcd\n)--(\kref\pcm\n)--(\kref\pcu\n);}
\foreach \n in {4}{\draw(\kref\pn\n)node[ocirc]{};}
}
%%%%%%%%%%%%%%%%%%%%%%
%%%%%%%%%%%%%%%%%%%%%%%%%%%%%%%
\newcommand{\kdTFF}[3][u0]{
\def\kref{#1}
%\def\p{p}			%pin tip
%\def\pb{pb}			%pin base
%\def\pl{pl}			%pin label
%\def\pcu{pcu}			%pin edge, upper corner
%\def\pcm{pcm}			%pin edge middle corner
%\def\pcd{pcd}			%pin edge lower corner
%\def\pn{pn}			%pin 'ocirc'
\kdFF[#1]{#2}{#3}
\foreach \n in {1,2,4,6}{\draw(\kref\pb\n)--(\kref\p\n);}
\foreach \n/\lbl in {1/T,4/{\overline{Q}},6/Q}{\draw(\kref\pl\n)node[]{$\lbl$};}
\foreach \n in {2}{\draw(\kref\pcd\n)--(\kref\pcm\n)--(\kref\pcu\n);}
\foreach \n in {4}{\draw(\kref\pn\n)node[ocirc]{};}
}
%%%%%%%%%%%%%%%%%%%%%
%%%%%%%%%%%%%%%%%%%%%%

\newcommand{\kdBuffer}[3][u0]{
\def\kref{#1}
\def\kshiftX{#2}
\def\kshiftY{#3}

\def\p{p}			%pin tip
\def\pb{pb}			%pin base
\def\pl{pl}			%pin label
\def\pn{pn}			%pin 'ocirc'
\def\pin{pin}			%pin tip
\def\pout{pout}			%pin tip
\def\pu{u}			%pin tip
\def\pd{d}			%pin tip
\pgfmathsetmacro{\klshift}{0.25}
\pgfmathsetmacro{\knshift}{0.07}
\pgfmathsetmacro{\kpin}{0.30}
\pgfmathsetmacro{\kpsep}{0.40}			%pin to pin distance
\pgfmathsetmacro{\kulV}{0.40}			%edge clearance along vertical edge
\pgfmathsetmacro{\kulH}{0.50}
\pgfmathsetmacro{\kdimX}{2*\kulH+0*\kpsep}
\pgfmathsetmacro{\kdimY}{\kdimX}		%two spaces between 3 pins
\draw(\kshiftX,\kshiftY)coordinate(\kref\pin)--++(0,-\kpin)coordinate(\kref\pb\pin);
\draw(\kref\pb\pin)[thick]++(-\kdimY/2,0)coordinate(aa)--++(\kdimY,0)
coordinate(bb)--++(-\kdimY/2,-\kdimX)coordinate(cc)coordinate(\kref\pb\pout)--++(-\kdimY/2,\kdimX);
\draw(\kref\pb\pout)--++(0,-\kpin)coordinate(\kref\pout);
\draw($(bb)!0.5!(cc)$)coordinate(\kref\pb\pu)+(0.07,0)coordinate(\kref\pn\pu)+(\kpin,0)coordinate(\kref\pu);
\draw($(aa)!0.5!(cc)$)coordinate(\kref\pb\pd)+(-0.07,0)coordinate(\kref\pn\pd)+(-\kpin,0)coordinate(\kref\pd);
\draw(\kref\pb\pin)++(0,0.07)coordinate(\kref\pn\pin);
\draw(\kref\pb\pout)++(0,-0.07)coordinate(\kref\pn\pout);
}

\newcommand*{\kop}[1]{#1}		%operand formatting
\newcommand*{\kopBinary}[1]{#1}  	%must ensure binary data grouped into 4's should retain LTR style and donot flip
\newcommand*{\زیرعمل}[1]{#1}

\newcommand*{\دفترالف}{\text{الف}}	%urdu names of A,B,C,D registers
\newcommand*{\دفترب}{\text{ب}}
\newcommand*{\دفترج}{\text{ج}}
\newcommand*{\دفترد}{\text{د}}
\newcommand*{\regA}{\text{الف}}	%english names of A,B,C,D registers
\newcommand*{\regB}{\text{ب}}
\newcommand*{\regC}{\text{ج}}
\newcommand*{\regD}{\text{د}}
\newcommand*{\regPC}{\text{برنامہ گنتکار}}  %گنت کار
 
 
 \newcommand*{\نقل}[1]{\text{\RL{نقل \, #1}}}
 \newcommand*{\جمع}[1]{\text{\RL{جمع \, #1}}}
 \newcommand*{\منفی}[1]{\text{\RL{منفی \, #1}}}
 \newcommand*{\برآمد}[1]{\text{\RL{برآمد \, #1}}}
 \newcommand*{\رک}{\text{\RL{رک}}}
 
\newcommand*{\ADD}[1]{\text{\RL{جمع \, #1}}}
\newcommand*{\ANA}[1]{\text{\RL{ضرب منطقی\, #1}}}		%ضرب منطقی
\newcommand*{\ANI}[1]{\text{\RL{ضرب  منطقی  متصل\, #1}}}	%ضرب منطقی  متصل
\newcommand*{\CALL}[1]{\text{\RL{طلبی\, #1}}}
\newcommand*{\CMA}{\text{\RL{متمم}}}
\newcommand*{\DCR}[1]{\text{\RL{گھٹا \, #1}}}
\newcommand*{\HLT}{\text{\RL{رک}}}
\newcommand*{\IN}[1]{\text{\RL{درآمد\, #1}}}
\newcommand*{\INR}[1]{\text{\RL{بڑھا\, #1}}}
\newcommand*{\JM}[1]{\text{\RL{شاخ منفی\, #1}}}		%
\newcommand*{\JMP}[1]{\text{\RL{شاخ \, #1}}}
%\newcommand*{\JMP}[1]{\text{\RL{شاخ \, #1}}}
\newcommand*{\JNZ}[1]{\text{\RL{شاخ غیر صفر\, #1}}}
\newcommand*{\JZ}[1]{\text{\RL{شاخ صفر\, #1}}}
\newcommand*{\LDA}[1]{\text{\RL{نقل \, #1}}}
\newcommand*{\MOV}[2]{\text{\RL{لاد\, #1 ، #2}}}
\newcommand*{\MVI}[2]{\text{\RL{متصل\, #1 ، #2}}}
\newcommand*{\NOP}{\text{\RL{فارغ}}}
\newcommand*{\ORA}[1]{\text{\RL{جمع منطقی\, #1}}}		%جمع منطقی
\newcommand*{\ORI}[1]{\text{\RL{جمع منطقی متصل\, #1}}}	%جمع  منطقی متصل
\newcommand*{\OUT}[1]{\text{\RL{برآمد\, #1}}}
\newcommand*{\RAL}{\text{\RL{گھوم بائیں}}}
\newcommand*{\RAR}{\text{\RL{گھوم دائیں}}}
\newcommand*{\RET}{\text{\RL{لوٹ}}}
\newcommand*{\STA}[1]{\text{\RL{ذخیرہ \, #1}}}
\newcommand*{\SUB}[1]{\text{\RL{منفی \, #1}}}
\newcommand*{\XRA}[1]{\text{\RL{بلا شرکت \, #1}}}
\newcommand*{\XRI}[1]{\text{\RL{بلا شرکت متصل \, #1}}}

 %short in-line versions without operands
\newcommand*{\sADD}{\text{\RL{جمع}}}
\newcommand*{\sANA}{\text{\RL{ضرب منطقی}}}		%ضرب منطقی
\newcommand*{\sANI}{\text{\RL{ضرب  منطقی  متصل}}}	%ضرب منطقی  متصل
\newcommand*{\sCALL}{\text{\RL{طلبی}}}
\newcommand*{\sCMA}{\text{\RL{متمم}}}
\newcommand*{\sDCR}{\text{\RL{گھٹا}}}
\newcommand*{\sHLT}{\text{\RL{\text{\RL{رک}}}}}
\newcommand*{\sIN}{\text{\RL{درآمد}}}
\newcommand*{\sINR}{\text{\RL{بڑھا}}}
\newcommand*{\sJM}{\text{\RL{شاخ منفی}}}		%
\newcommand*{\sJMP}{\text{\RL{شاخ}}}
\newcommand*{\sJNZ}{\text{\RL{شاخ غیر صفر}}}
\newcommand*{\sJZ}{\text{\RL{شاخ صفر}}}
\newcommand*{\sLDA}{\text{\RL{نقل}}}
\newcommand*{\sMOV}{\text{\RL{لاد}}}
\newcommand*{\sMVI}{\text{\RL{متصل}}}
\newcommand*{\sNOP}{\text{\RL{فارغ}}}
\newcommand*{\sORA}{\text{\RL{جمع منطقی}}}		%جمع منطقی
\newcommand*{\sORI}{\text{\RL{جمع منطقی متصل}}}	%جمع  منطقی متصل
\newcommand*{\sOUT}{\text{\RL{برآمد}}}
\newcommand*{\sRAL}{\text{\RL{گھوم بائیں}}}
\newcommand*{\sRAR}{\text{\RL{گھوم دائیں}}}
\newcommand*{\sRET}{\text{\RL{لوٹ}}}
\newcommand*{\sSTA}{\text{\RL{ذخیرہ}}}
\newcommand*{\sSUB}{\text{\RL{منفی}}}
\newcommand*{\sXRA}{\text{\RL{بلا شرکت}}}
\newcommand*{\sXRI}{\text{\RL{بلا شرکت متصل}}}

%for future use
\newcommand*{\sSHL}{\text{\RL{قدم بائیں}}}  % \newcommand*{\sSHL}{\text{\RL{بائیں}}}

\newcommand{\kANDbox}[5][u0]{	%cannot use double quotes as the pin names allocated here are used when using the ICs
\def\kref{#1}
\def\kshiftX{#2}
\def\kshiftY{#3}
\def\nny{#4}
\def\kpin{#5}			
\pgfmathsetmacro{\ny}{\nny-1}
\def\p{p}			%pin tip
\def\kin{in}			%left pins
\def\kout{out}		%bottom pins 
\def\pb{pb}			%pin base
\def\pl{pl}			%pin label
\def\pnum{pnum}		%pin number location; use \kref\a\pnum\n e.g. u0apnum3 means left 3rd from bottom
\def\pn{pn}			%pin 'ocirc'
\pgfmathsetmacro{\klshift}{0.25}
\pgfmathsetmacro{\knshift}{0.07}
\pgfmathsetmacro{\knumshift}{0.25}
%\pgfmathsetmacro{\kpin}{0.30}
\pgfmathsetmacro{\kpsep}{0.40}			%pin to pin distance
\pgfmathsetmacro{\kul}{0.40}			%pin to edge distance	
\pgfmathsetmacro{\kc}{0.40}
\pgfmathsetmacro{\kdimX}{0.40}
\pgfmathsetmacro{\kdimY}{0.5*\kul+0.5*\kul+\ny*\kpsep}		%two spaces between 3 pins

\draw[thin](\kshiftX,\kshiftY)++(-\kpin-0.5*\kc-0.5*\kdimX,0)coordinate(\kref-center);
\draw[thin](\kshiftX,\kshiftY)++(-\kpin-0.5*\kc-0.5*\kdimX,0)++(0,0.5*\kdimY)coordinate(\kref-north);
\draw[thin](\kshiftX,\kshiftY)++(-\kpin-0.5*\kc-0.5*\kdimX,0)++(0,-0.5*\kdimY)coordinate(\kref-south);
%
\draw[thick](\kshiftX,\kshiftY)coordinate(\kref\p\kout)++(-\kpin,0)coordinate(\kref\pb\kout)+(\knshift,0)coordinate(\kref\pn\kout)+(\knumshift,1.25ex)coordinate(\kref\pnum\kout)++(-\kc,-0.5*\kdimY)++(0,\kdimY)++(-\kdimX,0)++(0,-\kdimY)coordinate(AA)--++(\kdimX,0) to [out=0,in=-90]++(\kc,0.5*\kdimY) to [out=90,in=0]++(-\kc,0.5*\kdimY)--++(-\kdimX,0)--++(0,-\kdimY);
%

\foreach \n in {0,...,\ny} {\draw[thin](AA)++(0,0.5*\kul+\n*\kpsep)coordinate(\kref\pb\n)+(-\knumshift,1.25ex)coordinate(\kref\pnum\n)+(-\kpin,0)coordinate(\kref\kin\n)+(-\knshift,0)coordinate(\kref\pn\n);} 
}
%============================
\newcommand{\kORbox}[5][u0]{	%cannot use double quotes as the pin names allocated here are used when using the ICs
\def\kref{#1}
\def\kshiftX{#2}
\def\kshiftY{#3}
\def\nny{#4}
\def\kpin{#5}			
\pgfmathsetmacro{\ny}{\nny-1}
\def\p{p}			%pin tip
\def\kin{in}			%left pins
\def\kout{out}		%bottom pins 
\def\pb{pb}			%pin base
\def\pl{pl}			%pin label
\def\pnum{pnum}		%pin number location; use \kref\a\pnum\n e.g. u0apnum3 means left 3rd from bottom
\def\pn{pn}			%pin 'ocirc'
\pgfmathsetmacro{\klshift}{0.25}
\pgfmathsetmacro{\knshift}{0.07}
\pgfmathsetmacro{\knumshift}{0.25}
%\pgfmathsetmacro{\kpin}{0.30}
\pgfmathsetmacro{\kpsep}{0.40}			%pin to pin distance
\pgfmathsetmacro{\kul}{0.40}			%pin to edge distance	
%\pgfmathsetmacro{\kc}{0.40}
\pgfmathsetmacro{\kdimX}{0.8}
\pgfmathsetmacro{\kdimY}{0.5*\kul+0.5*\kul+\ny*\kpsep}		%two spaces between 3 pins

\draw[thin](\kshiftX,\kshiftY)++(-\kpin-0.5*\kdimX,0)coordinate(\kref-center);
\draw[thin](\kshiftX,\kshiftY)++(-\kpin-0.5*\kdimX,0)++(0,0.5*\kdimY)coordinate(\kref-north);
\draw[thin](\kshiftX,\kshiftY)++(-\kpin-0.5*\kdimX,0)++(0,-0.5*\kdimY)coordinate(\kref-south);
%
\draw[thin](\kshiftX,\kshiftY)coordinate(\kref\p\kout)++(-\kpin,0)coordinate(\kref\pb\kout)+(\knshift,0)coordinate(\kref\pn\kout)+(\knumshift,1.25ex)coordinate(\kref\pnum\kout)++(-\kdimX,-0.5*\kdimY)coordinate(AA);
%
\draw[thick](\kshiftX,\kshiftY)++(-\kpin,0) to [out=120,in=0]++(-\kdimX,0.5*\kdimY)--++(0,-\kdimY) to [out=0,in=-120]++(\kdimX,0.5*\kdimY);
%
\foreach \n in {0,...,\ny} {\draw[thin](AA)++(0,0.5*\kul+\n*\kpsep)coordinate(\kref\pb\n)+(-\knumshift,1.25ex)coordinate(\kref\pnum\n)+(-\kpin,0)coordinate(\kref\kin\n)+(-\knshift,0)coordinate(\kref\pn\n);} 
}
%============================
%==============================
\newcommand{\kNOTbox}[4][u0]{	%cannot use double quotes as the pin names allocated here are used when using the ICs
\def\kref{#1}
\def\kshiftX{#2}
\def\kshiftY{#3}
\def\kpin{#4}			
\def\p{p}			%pin tip
\def\kin{in}			%left pins
\def\kout{out}		%bottom pins 
\def\pb{pb}			%pin base
\def\pl{pl}			%pin label
\def\pnum{pnum}		%pin number location; use \kref\a\pnum\n e.g. u0apnum3 means left 3rd from bottom
\def\pn{pn}			%pin 'ocirc'
\def\pu{pu}			%control pin up
\def\pd{pd}			%control pin down
\pgfmathsetmacro{\klshift}{0.25}
\pgfmathsetmacro{\knshift}{0.07}
\pgfmathsetmacro{\knumshift}{0.25}
%\pgfmathsetmacro{\kpin}{0.30}
\pgfmathsetmacro{\kpsep}{0.40}			%pin to pin distance
\pgfmathsetmacro{\kul}{0.40}			%pin to edge distance	
%\pgfmathsetmacro{\kc}{0.40}
\pgfmathsetmacro{\kdimX}{0.7}
\pgfmathsetmacro{\kdimY}{0.7}		%two spaces between 3 pins

\draw[thin](\kshiftX,\kshiftY)++(-\kpin-0.5*\kdimX,0)coordinate(\kref-center);
\draw[thin](\kshiftX,\kshiftY)++(-\kpin-0.5*\kdimX,0)++(0,0.5*\kdimY)coordinate(\kref-north);
\draw[thin](\kshiftX,\kshiftY)++(-\kpin-0.5*\kdimX,0)++(0,-0.5*\kdimY)coordinate(\kref-south);
%
\draw[thin](\kshiftX,\kshiftY)coordinate(\kref\kout)++(-\kpin,0)coordinate(\kref\pb\kout)+(\knshift,0)coordinate(\kref\pn\kout)+(\knumshift,1.25ex)coordinate(\kref\pnum\kout)++(-\kdimX,0)coordinate(\kref\pb\kin)+(-\knshift,0)coordinate(\kref\pn\kin)+(-\knumshift,1.25ex)coordinate(\kref\pnum\kin)+(-\kpin,0)coordinate(\kref\kin);
%
\draw[thick](\kshiftX,\kshiftY)++(-\kpin,0)coordinate(kO)--++(-\kdimX,0.5*\kdimY)coordinate(kT)--++(0,-\kdimY)coordinate(kB)--++(\kdimX,0.5*\kdimY);
%
\draw[thin]($(kO)!0.6!(kT)$)coordinate(\kref\pb\pu)+(0,\knshift)coordinate(\kref\pn\pu)+(1.25ex,\knumshift)coordinate(\kref\pnum\pu)+(0,\kpin)coordinate(\kref\pu);
%
\draw[thin]($(kO)!0.6!(kB)$)coordinate(\kref\pb\pd)+(0,-\knshift)coordinate(\kref\pn\pd)+(1.25ex,-\knumshift) coordinate(\kref\pnum\pd)+(0,-\kpin)coordinate(\kref\pd);
}
%============================
%==============================
%numbering code 0=z, 1=o, 2=t, 3=T, 4=f, 5=F, 6=s, 7=S, 8=e, 9=n
%use u10Ap1, u10Ap12 etc to access pins where 1, 12 are IC pin numbers and A is the section number
%74LS00A quad NAND gates
\newcommand{\kSfzzA}[3][u0]{
\def\kref{#1}
\def\p{p}			%pin tip
\kANDbox[#1]{#2}{#3}{2}{0.4}
\foreach \n/\lbl in {0/1,1/2}\draw[thin](\kref\pb\n)--(\kref\kin\n)coordinate(\kref\p\lbl)  (\kref\pnum\n)node[]{\scriptsize{$\lbl$}}; 
\def\lbl{3}
\draw[thin](\kref\pnum\kout)node[]{\scriptsize{$3$}};
\draw[thin](\kref\pb\kout)--(\kref\p\kout)coordinate(\kref\p\lbl) (\kref\pn\kout)node[ocirc]{};
}
%==============================
%numbering code 0=z, 1=o, 2=t, 3=T, 4=f, 5=F, 6=s, 7=S, 8=e, 9=n
%use u10Bp1, u10Bp12 etc to access pins where 1, 12 are IC pin numbers and B is the section number
%74LS00B 2-inpput quad NAND gates
\newcommand{\kSfzzB}[3][u0]{	
\def\kref{#1}
\kANDbox[#1]{#2}{#3}{2}{0.4}
\foreach \n/\lbl in {0/4,1/5}\draw[thin](\kref\pb\n)--(\kref\kin\n)coordinate(\kref\p\lbl)  (\kref\pnum\n)node[]{\scriptsize{$\lbl$}}; 
\def\lbl{6}
\draw[thin](\kref\pnum\kout)node[]{\scriptsize{$6$}};
\draw[thin](\kref\pb\kout)--(\kref\p\kout)coordinate(\kref\p\lbl) (\kref\pn\kout)node[ocirc]{};
}
%==============================
%numbering code 0=z, 1=o, 2=t, 3=T, 4=f, 5=F, 6=s, 7=S, 8=e, 9=n
%74LS00C 2-input quad NAND gates
\newcommand{\kSfzzC}[3][u0]{	
\def\kref{#1}
\kANDbox[#1]{#2}{#3}{2}{0.4}
\foreach \n/\lbl in {0/9,1/10}\draw[thin](\kref\pb\n)--(\kref\kin\n)coordinate(\kref\p\lbl)  (\kref\pnum\n)node[]{\scriptsize{$\lbl$}}; 
\def\lbl{8}
\draw[thin](\kref\pnum\kout)node[]{\scriptsize{$8$}};
\draw[thin](\kref\pb\kout)--(\kref\p\kout)coordinate(\kref\p\lbl) (\kref\pn\kout)node[ocirc]{};
}
%==============================
%numbering code 0=z, 1=o, 2=t, 3=T, 4=f, 5=F, 6=s, 7=S, 8=e, 9=n
%74LS00D 2-input quad NAND gates
\newcommand{\kSfzzD}[3][u0]{	
\def\kref{#1}
\kANDbox[#1]{#2}{#3}{2}{0.4}
\foreach \n/\lbl in {0/12,1/13}\draw[thin](\kref\pb\n)--(\kref\kin\n)coordinate(\kref\p\lbl)  (\kref\pnum\n)node[]{\scriptsize{$\lbl$}}; 
\def\lbl{11}
\draw[thin](\kref\pnum\kout)node[]{\scriptsize{$11$}};
\draw[thin](\kref\pb\kout)--(\kref\p\kout)coordinate(\kref\p\lbl) (\kref\pn\kout)node[ocirc]{};
}
%==============================
%numbering code 0=z, 1=o, 2=t, 3=T, 4=f, 5=F, 6=s, 7=S, 8=e, 9=n
%use u10Ap1, u10Ap12 etc to access pins where 1, 12 are IC pin numbers and A is the section number
%74LS10A  3-input triple NAND gates
\newcommand{\kSfozA}[3][u0]{	
\def\kref{#1}
\kANDbox[#1]{#2}{#3}{3}{0.4}
\foreach \n/\lbl in {0/1,1/2,2/13}\draw[thin](\kref\pb\n)--(\kref\kin\n)coordinate(\kref\p\lbl)  (\kref\pnum\n)node[]{\scriptsize{$\lbl$}}; 
\def\lbl{12}
\draw[thin](\kref\pnum\kout)node[]{\scriptsize{$12$}};
\draw[thin](\kref\pb\kout)--(\kref\p\kout)coordinate(\kref\p\lbl) (\kref\pn\kout)node[ocirc]{};
}
%==============================
%numbering code 0=z, 1=o, 2=t, 3=T, 4=f, 5=F, 6=s, 7=S, 8=e, 9=n
%74LS10B  3-input triple NAND gates
\newcommand{\kSfozB}[3][u0]{	
\def\kref{#1}
\kANDbox[#1]{#2}{#3}{3}{0.4}
\foreach \n/\lbl in {0/3,1/4,2/5}\draw[thin](\kref\pb\n)--(\kref\kin\n)coordinate(\kref\p\lbl)  (\kref\pnum\n)node[]{\scriptsize{$\lbl$}}; 
\def\lbl{6}
\draw[thin](\kref\pnum\kout)node[]{\scriptsize{$6$}};
\draw[thin](\kref\pb\kout)--(\kref\p\kout)coordinate(\kref\p\lbl) (\kref\pn\kout)node[ocirc]{};
}
%==============================
%numbering code 0=z, 1=o, 2=t, 3=T, 4=f, 5=F, 6=s, 7=S, 8=e, 9=n
%74LS10C  3-input triple NAND gates
\newcommand{\kSfozC}[3][u0]{	
\def\kref{#1}
\kANDbox[#1]{#2}{#3}{3}{0.4}
\foreach \n/\lbl in {0/9,1/10,2/11}\draw[thin](\kref\pb\n)--(\kref\kin\n)coordinate(\kref\p\lbl)  (\kref\pnum\n)node[]{\scriptsize{$\lbl$}}; 
\def\lbl{8}
\draw[thin](\kref\pnum\kout)node[]{\scriptsize{$8$}};
\draw[thin](\kref\pb\kout)--(\kref\p\kout)coordinate(\kref\p\lbl) (\kref\pn\kout)node[ocirc]{};
}
%==============================
%==============================
%numbering code 0=z, 1=o, 2=t, 3=T, 4=f, 5=F, 6=s, 7=S, 8=e, 9=n
%use u10Ap1, u10Ap12 etc to access pins where 1, 12 are IC pin numbers and A is the section number
%74LS08A quad AND gates
\newcommand{\kSfzeA}[3][u0]{	
\def\kref{#1}
\kANDbox[#1]{#2}{#3}{2}{0.4}
\foreach \n/\lbl in {0/1,1/2}\draw[thin](\kref\pb\n)--(\kref\kin\n)coordinate(\kref\p\lbl)  (\kref\pnum\n)node[]{\scriptsize{$\lbl$}}; 
\def\lbl{3}
\draw[thin](\kref\pnum\kout)node[]{\scriptsize{$3$}};
\draw[thin](\kref\pb\kout)--(\kref\p\kout)coordinate(\kref\p\lbl);
}
%==============================
%numbering code 0=z, 1=o, 2=t, 3=T, 4=f, 5=F, 6=s, 7=S, 8=e, 9=n
%74LS08B 2-inpput quad AND gates
\newcommand{\kSfzeB}[3][u0]{	
\def\kref{#1}
\kANDbox[#1]{#2}{#3}{2}{0.4}
\foreach \n/\lbl in {0/4,1/5}\draw[thin](\kref\pb\n)--(\kref\kin\n)coordinate(\kref\p\lbl)  (\kref\pnum\n)node[]{\scriptsize{$\lbl$}}; 
\def\lbl{6}
\draw[thin](\kref\pnum\kout)node[]{\scriptsize{$6$}};
\draw[thin](\kref\pb\kout)--(\kref\p\kout)coordinate(\kref\p\lbl);
}
%==============================
%numbering code 0=z, 1=o, 2=t, 3=T, 4=f, 5=F, 6=s, 7=S, 8=e, 9=n
%74LS08C 2-input quad AND gates
\newcommand{\kSfzeC}[3][u0]{	
\def\kref{#1}
\kANDbox[#1]{#2}{#3}{2}{0.4}
\foreach \n/\lbl in {0/9,1/10}\draw[thin](\kref\pb\n)--(\kref\kin\n)coordinate(\kref\p\lbl)  (\kref\pnum\n)node[]{\scriptsize{$\lbl$}}; 
\def\lbl{8}
\draw[thin](\kref\pnum\kout)node[]{\scriptsize{$8$}};
\draw[thin](\kref\pb\kout)--(\kref\p\kout)coordinate(\kref\p\lbl);
}
%==============================
%numbering code 0=z, 1=o, 2=t, 3=T, 4=f, 5=F, 6=s, 7=S, 8=e, 9=n
%74LS08D 2-input quad AND gates
\newcommand{\kSfzeD}[3][u0]{	
\def\kref{#1}
\kANDbox[#1]{#2}{#3}{2}{0.4}
\foreach \n/\lbl in {0/12,1/13}\draw[thin](\kref\pb\n)--(\kref\kin\n)coordinate(\kref\p\lbl)  (\kref\pnum\n)node[]{\scriptsize{$\lbl$}}; 
\def\lbl{11}
\draw[thin](\kref\pnum\kout)node[]{\scriptsize{$11$}};
\draw[thin](\kref\pb\kout)--(\kref\p\kout)coordinate(\kref\p\lbl);
}
%==============================
%==============================
%numbering code 0=z, 1=o, 2=t, 3=T, 4=f, 5=F, 6=s, 7=S, 8=e, 9=n
%use u10Ap1, u10Ap12 etc to access pins where 1, 12 are IC pin numbers and A is the section number
%74LS11A  3-input triple AND gates
\newcommand{\kSfooA}[3][u0]{	
\def\kref{#1}
\kANDbox[#1]{#2}{#3}{3}{0.4}
\foreach \n/\lbl in {0/1,1/2,2/13}\draw[thin](\kref\pb\n)--(\kref\kin\n)coordinate(\kref\p\lbl)  (\kref\pnum\n)node[]{\scriptsize{$\lbl$}}; 
\draw[thin](\kref\pnum\kout)node[]{\scriptsize{$12$}};
\draw[thin](\kref\pb\kout)--(\kref\p\kout);
%
\def\lbl{12}
\draw(\kref\p\kout)coordinate(\kref\p\lbl);
 
}
%==============================
%numbering code 0=z, 1=o, 2=t, 3=T, 4=f, 5=F, 6=s, 7=S, 8=e, 9=n
%74LS11B  3-input triple AND gates
\newcommand{\kSfooB}[3][u0]{	
\def\kref{#1}
\kANDbox[#1]{#2}{#3}{3}{0.4}
\foreach \n/\lbl in {0/3,1/4,2/5}\draw[thin](\kref\pb\n)--(\kref\kin\n)coordinate(\kref\p\lbl)  (\kref\pnum\n)node[]{\scriptsize{$\lbl$}}; 
\draw[thin](\kref\pnum\kout)node[]{\scriptsize{$6$}};
\draw[thin](\kref\pb\kout)--(\kref\p\kout);
%
\def\lbl{6}
\draw(\kref\p\kout)coordinate(\kref\p\lbl);
}
%==============================
%numbering code 0=z, 1=o, 2=t, 3=T, 4=f, 5=F, 6=s, 7=S, 8=e, 9=n
%74LS11C  3-input triple AND gates
\newcommand{\kSfooC}[3][u0]{	
\def\kref{#1}
\kANDbox[#1]{#2}{#3}{3}{0.4}
\foreach \n/\lbl in {0/9,1/10,2/11}\draw[thin](\kref\pb\n)--(\kref\kin\n)coordinate(\kref\p\lbl)  (\kref\pnum\n)node[]{\scriptsize{$\lbl$}}; 
\draw[thin](\kref\pnum\kout)node[]{\scriptsize{$8$}};
\draw[thin](\kref\pb\kout)--(\kref\p\kout);
%
\def\lbl{8}
\draw(\kref\p\kout)coordinate(\kref\p\lbl);
}
%==============================
%numbering code 0=z, 1=o, 2=t, 3=T, 4=f, 5=F, 6=s, 7=S, 8=e, 9=n
%use u10Ap1, u10Ap2 etc to access pins where 1, 2 are IC pin numbers and A is the section number
%74LS04A   hex NOT gates
\newcommand{\kSfzfA}[3][u0]{	
\def\kref{#1}
\def\p{p}
\kNOTbox[#1]{#2}{#3}{0.4}
\draw[thin](\kref\pb\kin)--(\kref\kin)  (\kref\pb\kout)--(\kref\kout);
\def\lbl{1}
\draw[thin](\kref\pnum\kin)coordinate(\kref\p\lbl)node[]{\scriptsize{$1$}};
\def\lbl{2}
\draw[thin](\kref\pnum\kout)coordinate(\kref\p\lbl)node[]{\scriptsize{$2$}};
\draw[thin](\kref\pb\kout)--(\kref\kout) (\kref\pn\kout)node[ocirc]{};
}
%==============================
%numbering code 0=z, 1=o, 2=t, 3=T, 4=f, 5=F, 6=s, 7=S, 8=e, 9=n
%74LS04B   hex NOT gates
\newcommand{\kSfzfB}[3][u0]{	
\def\kref{#1}
\kNOTbox[#1]{#2}{#3}{0.4}
\draw[thin](\kref\pb\kin)--(\kref\kin)  (\kref\pb\kout)--(\kref\kout);
\def\p{p}
\def\lbl{3}
\draw[thin](\kref\pnum\kin)coordinate(\kref\p\lbl)node[]{\scriptsize{$3$}};
\def\lbl{4}
\draw[thin](\kref\pnum\kout)coordinate(\kref\p\lbl)node[]{\scriptsize{$4$}};
\draw[thin](\kref\pb\kout)--(\kref\kout) (\kref\pn\kout)node[ocirc]{};
}
%==============================
%numbering code 0=z, 1=o, 2=t, 3=T, 4=f, 5=F, 6=s, 7=S, 8=e, 9=n
%74LS04C   hex NOT gates
\newcommand{\kSfzfC}[3][u0]{	
\def\kref{#1}
\kNOTbox[#1]{#2}{#3}{0.4}
\draw[thin](\kref\pb\kin)--(\kref\kin)  (\kref\pb\kout)--(\kref\kout);
\def\p{p}
\def\lbl{5}
\draw[thin](\kref\pnum\kin)coordinate(\kref\p\lbl)node[]{\scriptsize{$5$}};
\def\lbl{6}
\draw[thin](\kref\pnum\kout)coordinate(\kref\p\lbl)node[]{\scriptsize{$6$}};
\draw[thin](\kref\pb\kout)--(\kref\kout) (\kref\pn\kout)node[ocirc]{};
}
%==============================
%numbering code 0=z, 1=o, 2=t, 3=T, 4=f, 5=F, 6=s, 7=S, 8=e, 9=n
%74LS04D   hex NOT gates
\newcommand{\kSfzfD}[3][u0]{	
\def\kref{#1}
\kNOTbox[#1]{#2}{#3}{0.4}
\draw[thin](\kref\pb\kin)--(\kref\kin)  (\kref\pb\kout)--(\kref\kout);
\def\p{p}
\def\lbl{9}
\draw[thin](\kref\pnum\kin)coordinate(\kref\p\lbl)node[]{\scriptsize{$9$}};
\def\lbl{8}
\draw[thin](\kref\pnum\kout)coordinate(\kref\p\lbl)node[]{\scriptsize{$8$}};
\draw[thin](\kref\pb\kout)--(\kref\kout) (\kref\pn\kout)node[ocirc]{};
}
%==============================
%numbering code 0=z, 1=o, 2=t, 3=T, 4=f, 5=F, 6=s, 7=S, 8=e, 9=n
%74LS04E   hex NOT gates
\newcommand{\kSfzfE}[3][u0]{	
\def\kref{#1}
\kNOTbox[#1]{#2}{#3}{0.4}
\draw[thin](\kref\pb\kin)--(\kref\kin)  (\kref\pb\kout)--(\kref\kout);
\def\p{p}
\def\lbl{11}
\draw[thin](\kref\pnum\kin)coordinate(\kref\p\lbl)node[]{\scriptsize{$11$}};
\def\lbl{10}
\draw[thin](\kref\pnum\kout)coordinate(\kref\p\lbl)node[]{\scriptsize{$10$}};
\draw[thin](\kref\pb\kout)--(\kref\kout) (\kref\pn\kout)node[ocirc]{};
}
%==============================
%numbering code 0=z, 1=o, 2=t, 3=T, 4=f, 5=F, 6=s, 7=S, 8=e, 9=n
%74LS04F   hex NOT gates
\newcommand{\kSfzfF}[3][u0]{	
\def\kref{#1}
\kNOTbox[#1]{#2}{#3}{0.4}
\draw[thin](\kref\pb\kin)--(\kref\kin)  (\kref\pb\kout)--(\kref\kout);
\def\p{p}
\def\lbl{13}
\draw[thin](\kref\pnum\kin)coordinate(\kref\p\lbl)node[]{\scriptsize{$13$}};
\def\lbl{12}
\draw[thin](\kref\pnum\kout)coordinate(\kref\p\lbl)node[]{\scriptsize{$12$}};
\draw[thin](\kref\pb\kout)--(\kref\kout) (\kref\pn\kout)node[ocirc]{};
}
%=======================
%==============================
%numbering code 0=z, 1=o, 2=t, 3=T, 4=f, 5=F, 6=s, 7=S, 8=e, 9=n
%use u10Ap1, u10Ap12 etc to access pins where 1, 12 are IC pin numbers and A is the section number
%74LS02A quad NOR gates
\newcommand{\kSfztA}[3][u0]{	
\def\kref{#1}
\kORbox[#1]{#2}{#3}{2}{0.4}
\foreach \n/\lbl in {0/2,1/3}\draw[thin](\kref\pb\n)--(\kref\kin\n)coordinate(\kref\p\lbl)  (\kref\pnum\n)node[]{\scriptsize{$\lbl$}}; 
\draw[thin](\kref\pnum\kout)node[]{\scriptsize{$1$}};
\draw[thin](\kref\pb\kout)--(\kref\p\kout) (\kref\pn\kout)node[ocirc]{};
%
\def\lbl{1}
\draw(\kref\p\kout)coordinate(\kref\p\lbl);
}
%==============================
%numbering code 0=z, 1=o, 2=t, 3=T, 4=f, 5=F, 6=s, 7=S, 8=e, 9=n
%74LS02B 2-inpput quad NOR gates
\newcommand{\kSfztB}[3][u0]{	
\def\kref{#1}
\kORbox[#1]{#2}{#3}{2}{0.4}
\foreach \n/\lbl in {0/5,1/6}\draw[thin](\kref\pb\n)--(\kref\kin\n)coordinate(\kref\p\lbl)  (\kref\pnum\n)node[]{\scriptsize{$\lbl$}}; 
\draw[thin](\kref\pnum\kout)node[]{\scriptsize{$4$}};
\draw[thin](\kref\pb\kout)--(\kref\p\kout) (\kref\pn\kout)node[ocirc]{};
%
\def\lbl{4}
\draw(\kref\p\kout)coordinate(\kref\p\lbl);
}
%==============================
%numbering code 0=z, 1=o, 2=t, 3=T, 4=f, 5=F, 6=s, 7=S, 8=e, 9=n
%74LS02C 2-input quad NOR gates
\newcommand{\kSfztC}[3][u0]{	
\def\kref{#1}
\kORbox[#1]{#2}{#3}{2}{0.4}
\foreach \n/\lbl in {0/8,1/9}\draw[thin](\kref\pb\n)--(\kref\kin\n)coordinate(\kref\p\lbl)  (\kref\pnum\n)node[]{\scriptsize{$\lbl$}}; 
\draw[thin](\kref\pnum\kout)node[]{\scriptsize{$10$}};
\draw[thin](\kref\pb\kout)--(\kref\p\kout) (\kref\pn\kout)node[ocirc]{};
%
\def\lbl{10}
\draw(\kref\p\kout)coordinate(\kref\p\lbl);
}
%==============================
%numbering code 0=z, 1=o, 2=t, 3=T, 4=f, 5=F, 6=s, 7=S, 8=e, 9=n
%74LS02D 2-input quad NOR gates
\newcommand{\kSfztD}[3][u0]{	
\def\kref{#1}
\kORbox[#1]{#2}{#3}{2}{0.4}
\foreach \n/\lbl in {0/11,1/12}\draw[thin](\kref\pb\n)--(\kref\kin\n)coordinate(\kref\p\lbl)  (\kref\pnum\n)node[]{\scriptsize{$\lbl$}}; 
\draw[thin](\kref\pnum\kout)node[]{\scriptsize{$13$}};
\draw[thin](\kref\pb\kout)--(\kref\p\kout) (\kref\pn\kout)node[ocirc]{};
%
\def\lbl{13}
\draw(\kref\p\kout)coordinate(\kref\p\lbl);
}
%==============================
%==============================
%numbering code 0=z, 1=o, 2=t, 3=T, 4=f, 5=F, 6=s, 7=S, 8=e, 9=n
%use u10Ap1, u10Ap12 etc to access pins where 1, 12 are IC pin numbers and A is the section number
%74LS32A quad OR gates
\newcommand{\kSfTtA}[3][u0]{	
\def\kref{#1}
\kORbox[#1]{#2}{#3}{2}{0.4}
\foreach \n/\lbl in {0/1,1/2}\draw[thin](\kref\pb\n)--(\kref\kin\n)coordinate(\kref\p\lbl)  (\kref\pnum\n)node[]{\scriptsize{$\lbl$}}; 
\draw[thin](\kref\pnum\kout)node[]{\scriptsize{$3$}};
\draw[thin](\kref\pb\kout)--(\kref\p\kout) (\kref\pn\kout);
%
\def\lbl{3}
\draw(\kref\p\kout)coordinate(\kref\p\lbl);
}
%==============================
%numbering code 0=z, 1=o, 2=t, 3=T, 4=f, 5=F, 6=s, 7=S, 8=e, 9=n
%74LS32B 2-inpput quad OR gates
\newcommand{\kSfTtB}[3][u0]{	
\def\kref{#1}
\kORbox[#1]{#2}{#3}{2}{0.4}
\foreach \n/\lbl in {0/4,1/5}\draw[thin](\kref\pb\n)--(\kref\kin\n)coordinate(\kref\p\lbl)  (\kref\pnum\n)node[]{\scriptsize{$\lbl$}}; 
\draw[thin](\kref\pnum\kout)node[]{\scriptsize{$6$}};
\draw[thin](\kref\pb\kout)--(\kref\p\kout) (\kref\pn\kout);
%
\def\lbl{6}
\draw(\kref\p\kout)coordinate(\kref\p\lbl);
}
%==============================
%numbering code 0=z, 1=o, 2=t, 3=T, 4=f, 5=F, 6=s, 7=S, 8=e, 9=n
%74LS32C 2-input quad OR gates
\newcommand{\kSfTtC}[3][u0]{	
\def\kref{#1}
\kORbox[#1]{#2}{#3}{2}{0.4}
\foreach \n/\lbl in {0/9,1/10}\draw[thin](\kref\pb\n)--(\kref\kin\n)coordinate(\kref\p\lbl)  (\kref\pnum\n)node[]{\scriptsize{$\lbl$}}; 
\draw[thin](\kref\pnum\kout)node[]{\scriptsize{$8$}};
\draw[thin](\kref\pb\kout)--(\kref\p\kout) (\kref\pn\kout);
%
\def\lbl{8}
\draw(\kref\p\kout)coordinate(\kref\p\lbl);
}
%==============================
%numbering code 0=z, 1=o, 2=t, 3=T, 4=f, 5=F, 6=s, 7=S, 8=e, 9=n
%74LS32D 2-input quad OR gates
\newcommand{\kSfTtD}[3][u0]{	
\def\kref{#1}
\kORbox[#1]{#2}{#3}{2}{0.4}
\foreach \n/\lbl in {0/12,1/13}\draw[thin](\kref\pb\n)--(\kref\kin\n)coordinate(\kref\p\lbl)  (\kref\pnum\n)node[]{\scriptsize{$\lbl$}}; 
\draw[thin](\kref\pnum\kout)node[]{\scriptsize{$11$}};
\draw[thin](\kref\pb\kout)--(\kref\p\kout) (\kref\pn\kout);
%
\def\lbl{11}
\draw(\kref\p\kout)coordinate(\kref\p\lbl);
}
%==============================





%%still decided to format everything at the very end
%try to make a jump table so that a single command can be built

\newcommand*{\ksubRB}{\ensuremath{R_{\textup{B}}}}   %transistor
\newcommand*{\ksubRC}{\ensuremath{R_{\textup{C}}}}
\newcommand*{\ksubRE}{\ensuremath{R_{\textup{E}}}}

\newcommand*{\ksubRG}{\ensuremath{R_{\textup{G}}}}	%mosfet
\newcommand*{\ksubRD}{\ensuremath{R_{\textup{D}}}}
\newcommand*{\ksubRS}{\ensuremath{R_{\textup{S}}}}

\newcommand*{\ksubCB}{\ensuremath{C_{\textup{B}}}}	%transistor
\newcommand*{\ksubCC}{\ensuremath{C_{\textup{C}}}}
\newcommand*{\ksubCE}{\ensuremath{C_{\textup{E}}}}

\newcommand*{\ksubCG}{\ensuremath{C_{\textup{G}}}}	%mosfet
\newcommand*{\ksubCD}{\ensuremath{C_{\textup{D}}}}
\newcommand*{\ksubCS}{\ensuremath{C_{\textup{S}}}}

\newcommand*{\ksubRCB}{\ensuremath{R_{\textup{CB}}}}    %resistor used with base capacitor    (GET RID OF SUCH USAGE)
\newcommand*{\ksubRCC}{\ensuremath{R_{\textup{CC}}}}   %resistor used with collector capacitor
\newcommand*{\ksubRCE}{\ensuremath{R_{\textup{CE}}}}   %resistor used with emitter capacitor

\newcommand*{\ksubVBE}{\ensuremath{V_{\textup{BE}}}}  %npnTransistor
\newcommand*{\ksubVBC}{\ensuremath{V_{\textup{BC}}}}
\newcommand*{\ksubVCE}{\ensuremath{V_{\textup{CE}}}}

\newcommand*{\ksubVEB}{\ensuremath{V_{\textup{EB}}}}  %pnpTransistor
\newcommand*{\ksubVCB}{\ensuremath{V_{\textup{CB}}}}
\newcommand*{\ksubVEC}{\ensuremath{V_{\textup{EC}}}}

\newcommand*{\ksubVGS}{\ensuremath{V_{\textup{GS}}}}  %nMosfet
\newcommand*{\ksubVGD}{\ensuremath{V_{\textup{GD}}}}
\newcommand*{\ksubVDS}{\ensuremath{V_{\textup{DS}}}}

\newcommand*{\ksubVSG}{\ensuremath{V_{\textup{SG}}}} 	%pMosfet
\newcommand*{\ksubVDG}{\ensuremath{V_{\textup{DG}}}}
\newcommand*{\ksubVSD}{\ensuremath{V_{\textup{SD}}}}

\newcommand*{\ksubsubVRE}{\ensuremath{V_{R_{\textup{E}}}}}
\newcommand*{\ksubsubVRC}{\ensuremath{V_{R_{\textup{C}}}}}
\newcommand*{\ksubsubVRB}{\ensuremath{V_{R_{\textup{B}}}}}

\newcommand*{\ksub}[2]{\ensuremath{#1_{\textup{#2}}}}     %R_1    or V_1   or C_1     where numbers are used

%voltage sources
\newcommand*{\ksubVCC}{\ensuremath{V_{\textup{CC}}}} %transistor
\newcommand*{\ksubVBB}{\ensuremath{V_{\textup{BB}}}}
\newcommand*{\ksubVEE}{\ensuremath{V_{\textup{EE}}}}

\newcommand*{\ksubVDD}{\ensuremath{V_{\textup{DD}}}}	%mosfet
\newcommand*{\ksubVGG}{\ensuremath{V_{\textup{GG}}}}
\newcommand*{\ksubVSS}{\ensuremath{V_{\textup{SS}}}}

\newcommand*{\ksubVS}{\ensuremath{V_{\textup{S}}}}
\newcommand*{\ksubVs}{\ensuremath{V_{\textup{s}}}}

%gain
\newcommand*{\ksubAv}{\ensuremath{A_{\textup{v}}}}		%gains
\newcommand*{\ksubAi}{\ensuremath{A_{\textup{i}}}}
\newcommand*{\ksubGm}{\ensuremath{G_{\textup{m}}}}
\newcommand*{\ksubRm}{\ensuremath{R_{\textup{m}}}}
         %these are all tested. to use at the very end when book is finished

\graphicspath{{./fig/figFrontPage/}{./fig/figCalculusLimits/}{./fig/figCalculusDerivatives/}{./fig/figCalculusIntegration/}{./fig/figCalculusApplicationsOfIntegrals}{./fig/figCalculusTechniquesOfIntegration/}}%paths to figures


%
%\includeonly{./tex/prefaceFirstBook}
%

%\includeonly{./tex/digitalCircuitsPreface,./tex/prefaceFirstBook,./tex/binarySystem,./tex/basicMaths,./tex/booleanAlgebra,./tex/karnaughMaps,./tex/combinationalLogicAndCircuits,./tex/synchronusSequentialLogic,./tex/registers,./tex/counters,./tex/memory,./tex/programmableLogicDevices,./tex/asynchronousCombinationalCircuits,./tex/simplestComputer}%
%
%\includeonly{./tex/simplestComputer}%
\includeonly{,./tex/schematicDiagramA}%
%\includeonly{./tex/simplestComputer,./tex/schematicDiagramA}%

\author{
خالد خان یوسفزئی\\
\\
\texttt{khalidyousafzai@hotmail.com}
}

%=========



\title{
عددی ادوار\\
{\small{تخلیق و تجزیہ}}
}
%\date{}                           %if absent gives date in arabic which is a rubbish

%%%%%%%%%%%%%%%%%%%%%%%%%%%%%%%%%%%%%%%%%
%%%%%%%%%%%%%%%%%%%%%%%%%%
%%included the following to correct the RTL issues with equation numbering after upgradation to ubuntu 18.04
%%might break bidi package as said on the internet
\makeatletter
\def\maketag@@@#1{\hbox{\m@th\normalfont\RTL{{\beginR #1\endR}}}}
\def\tagform@#1{\maketag@@@{(\ignorespaces{\beginR#1\endR}\unskip)}}
\makeatother
%%%%%%%%%%%%%%%%%%%%%%%%%%%%%%%%%%%
%%%%%%%%%%%%%%%%%%%%%%%%%%%%%%%%%%%

%\linenumbers
\makeindex

%==========
\begin{document}
\sloppy

\renewcommand*{\contentsname}{عنوان}    %this command has to be placed right here
%\renewcommand*{\proofname}{ثبوت}   %if placed before start of begin{urdufont}, it gets swept by the settings of the font environment
%\renewcommand*{\appendixname}{ضمیمہ}


\frontmatter                          %just added instead of \pagenumbering{roman}
%%\pagenumbering{roman}

\maketitle

\tableofcontents
\pagestyle{empty}
\newpage
\باب{دیباچہ}
یہ کتاب اس عزم سے لکھی گئی ہے کہ یہ ایک دن برقی انجنیئرنگ کی نصابی کتاب کے طور پر پڑھائی جائے گی۔امید کی جاتی ہے کہ اب بھی طلبہ و طالبات اس سے استفادہ حاصل کر سکیں گے۔
میں ڈاکٹر محمد اشرف عطا (ہلالِ امتیاز، ستارہِ امتیاز) کا خصوصی طور پر نہایت مشکور و ممنون ہوں جنہوں نے اپنے مصروفیات سے وقت نکال کر اس کتاب کو پڑھ کر نہ صرف درست کیا بلکہ بہت سارے تکنیکی اصطلاحات بھی فراہم کئے۔میں امید رکھتا ہوں کہ مجھے  آئندہ بھی ان کی مدد حاصل ہو گی۔

میں یہاں کامسیٹ کے طلبہ و طالبات کا بھی شکریہ ادا کرنا چاہتا ہوں جنہوں نے اس کتاب کو بار بار پڑھ کر غلطیوں کی نشاندہی کی۔

اس کتاب کے پڑھنے والوں سے گزارش کی جاتی ہے کہ وہ اس کتاب کو زیادہ سے زیادہ طلبہ و طالبات تک پہنچائیں اور اس میں غلطیوں کی نشاندہی میرے ای میل پتہ پر کریں۔

خالد خان یوسفزئی
5 فروری  \سن{2013}

\newpage
\باب{میری پہلی کتاب کا دیباچہ}
گزشتہ چند برسوں سے حکومتِ پاکستان اعلیٰ تعلیم کی طرف توجہ دے رہی ہے جس سے ملک کی تاریخ میں پہلی مرتبہ اعلیٰ تعلیمی اداروں میں تحقیق کا رجحان پیدا ہوا ہے۔امید کی جاتی ہے یہ سلسلہ جاری رہے گا۔

پاکستان میں  اعلیٰ  تعلیم کا نظام انگریزی زبان میں رائج ہے۔دنیا میں تحقیقی کام کا بیشتر حصہ انگریزی زبان میں ہی چھپتا ہے۔انگریزی زبان میں ہر موضوع پر لاتعداد کتابیں پائی جاتی ہیں جن سے طلبہ و طالبات استفادہ  کرتے ہیں۔

ہمارے ملک میں طلبہ و طالبات کی ایک بہت بڑی تعداد بنیادی تعلیم اردو زبان میں حاصل کرتی ہے۔ انگریزی زبان میں موجود مواد سے استفادہ  کرنا تو دور کی بات ،  ان کے لئے انگریزی زبان خود ایک رکاوٹ  ہے۔یہ طلبہ و طالبات ذہین ہونے کے باوجود آگے بڑھنے اور قوم و ملک کی بھر پور خدمت کرنے کے قابل نہیں رہتے۔ایسے طلبہ و طالبات کو اردو زبان میں نصاب کی اچھی کتابیں درکار ہیں۔ہم نے قومی سطح پر ایسا کرنے کی کوئی خاطر خواہ کوشش نہیں کی۔ 

میں برسوں تک اس صورت حال کی وجہ سے پریشانی کا شکار رہا۔کچھ کرنے کی نیت رکھنے کے باوجود کچھ نہ کر سکتا تھا۔میرے لئے اردو میں ایک صفحہ بھی لکھنا ناممکن تھا۔آخر کار ایک دن میں نے اپنی اس کمزوری کو کتاب نہ لکھنے کا جواز بنانے سے انکار کیا اور یوں یہ کتاب وجود میں آئی۔

یہ کتاب اردو زبان میں تعلیم حاصل کرنے والے طلبہ و طالبات کے لئے نہایت آسان اردو میں لکھی گئی ہے۔کوشش کی گئی ہے کہ اسکول کی سطح پر  مستعمل تکنیکی اصطلاحات استعمال کئے جائیں۔جہاں اصطلاحات موجود نہ تھی  وہاں روز مرہ  استعمال الفاظ چنے گئے۔تکنیکی  اصطلاحات  کی چنائی  یوں کی گئی ہے کہ ان کا استعمال دیگر مضامین میں بھی ہو۔

کتاب میں بین الاقوامی نظامِ اکائی استعمال کی گئی۔اہم متغیرات کی علامتیں وہی رکھی گئی جو موجودہ نظامِ تعلیم کی نصابی کتابوں میں رائج ہے۔یوں اردو میں لکھی اس کتاب اور انگریزی میں اسی مضمون پر لکھی کتاب پڑھنے والے طلبہ و طالبات کو ساتھ کام کرنے میں دشواری نہیں ہو گی۔ 

امید کی جاتی ہے یہ کتاب ایک دن خالصتاً اردو زبان میں انجنیئری نصاب کی کتاب کے طور پر  پڑھائی جائے گی۔اردو زبان میں برقی انجنیئری کی مکمل نصاب کی طرف یہ پہلا قدم ہے۔ 

کتاب کے پڑھنے والوں سے گزارش کی جاتی ہے کہ اسے زیادہ سے زیادہ طلبہ و طالبات تک پہنچانے میں مدد دیں  اور جہاں بھی   کتاب میں غلطی نظر آئے ، اس کی نشاندہی میری  برقیاتی پتہ پر کریں؛ میں ان کا نہایت شکر گزار ہوں گا۔

 کتاب میں تمام غلطیاں مجھ سے سر زد ہوئی ہیں جنہیں درست کرنے میں بہت لوگوں کا ہاتھ ہے۔میں ان سب کا شکریہ ادا کرتا ہوں۔ یہ سلسلہ ابھی جاری ہے اور مکمل ہونے پر ان حضرات کے تاثرات یہاں شامل کئے جائیں گے۔  

میں کامسیٹ یونیورسٹی اور ہائر ایجوکیشن کمیشن کا شکریہ ادا کرنا چاہتا ہوں جن کی وجہ سے ایسی سرگرمیاں ممکن ہوئیں۔	
\vspace{5mm}

{\raggedleft{
خالد خان یوسفزئی

28 اکتوبر \سن{2011}}}

%\newpage
%\include{./tex/cktSymbols}


\mainmatter                      %added this
%\renewcommand*{\chaptername}{باب}
%%\pagenumbering{arabic}   %instead of this

\pagestyle{headings}

\باب{ ثنائی نظام}

\حصہ{اعشاری نظامِ گنتی}

روز مرہ  زندگی میں اعشاری نظامِ گنتی  استعمال ہوتا ہے،  جو \عددی{0}  تا \عددی{9}  کے ہندسوں پر مبنی ہے۔کسی بھی گنتی کے نظام میں کل علامات کی تعداد کو اس نظام کی  اساس  کہتے ہیں۔اعشاری نظام   میں \عددی{0} تا \عددی{9}،  یعنی دس \عددی{10} علامات ہیں، یوں اعشاری نظام کی  اساس دس ہے اور اس  کو اساس  \عددی{10} کا نظام کہتے ہیں۔

	مساوات \حوالہ{مساوات_ثنائی_عدد}  میں \عددی{538.72}   کو  اعشاری نظام میں لکھتے ہوئے زیر نوشت  میں \عددی{10}   لکھا گیا ہے، جو  اس بات کی یاد دہانی کراتا ہے کہ یہ عدد  اساس دس کے نظام میں لکھا گیا ہے۔اس کتاب میں چونکہ کئی نظامِ گنتی استعمال ہوں گے،  لہٰذا جہاں متن سے واضح نہ ہو وہاں اعداد کے ساتھ ان کی  اساس زیر نوشت میں لکھی  جائے گا۔
\begin{align}\label{مساوات_ثنائی_عدد}
538.72_{10}
\end{align}
اس نظام میں اعشاریہ کی بائیں جانب پہلا ہندسہ اکائی وزن رکھتا ہے، دوسرا دہائی، تیسرا سینکڑا،  وغیرہ۔یوں مساوات  \حوالہ{مساوات_ثنائی_سینکڑا}  میں  دیے گئے ہندسوں میں \عددی{8}    کا 
مطلب \عددی{8\times 10^0=8\times 1=8_{10}}    ہے،  جبکہ  \عددی{3}     کا مطلب \عددی{3\times 10^1=30_{10}}     اور \عددی{5}    کا \عددی{5\times 10^2=500_{10}}     ہے۔اسی طرح اعشاریہ کے دائیں جانب پہلے  ہندسے   کا وزن ایک بٹہ دس ہے، دوسرے  ہندسے  کا ایک بٹہ سو،  اور تیسرے ہندسے کا ایک بٹہ ہزار،  وغیرہ۔یوں  اس عدد میں \عددی{7}   دراصل \عددی{7\times 10^{-1}=0.7_{10}}   جبکہ \عددی{2}   دراصل  \عددی{2\times 10^{-2}=0.02_{10}}  ہے۔
\begin{align}\label{مساوات_ثنائی_سینکڑا}
538.72_{10}=(5\times 10^2)+(3\times 10^1)+(8\times 10^0)+(7\times 10^{-1})+(2\times 10^{-2})
\end{align}

اس حقیقت  کو درج ذیل  عمومی روپ  میں   لکھ سکتے ہیں۔
\begin{multline}\label{مساوات_ثنائی_عمومی_روپ}
\cdots \,a_2\times 10^2+a_1\times 10^1+a_0\times 10^0+a_{-1}\times 10^{-1}+a_{-2}\times 10^{-2}\, \cdots\\
=(\cdots a_2a_1a_0\, .\, a_{-1}a_{-2}\cdots)_{10}
\end{multline}

	 عدد \عددی{538.72_{10}}  کو  \عددی{x}  لیتے   ہوئے، شکل  \حوالہء{1.1}   میں اس  کے مختلف ہندسوں کو پکارنے  کا طریقہ دکھایا گیا ہے،   جس کے تحت  \عددی{5} کو \عددی{x_2}  جبکہ \عددی{3}   کو \عددی{x_3} کہیں گے، وغیرہ وغیرہ۔

	اس طرح کسی بھی عدد میں بائیں جانب  ہندسے کا رتبہ  دائیں جانب  ہندسے  کے رتبہ سے بلند ہو گا۔مساوات \حوالہ{مساوات_ثنائی_عدد}  میں بلند تر رتبے  کا  ہندسہ \عددی{5} ہے ، جبکہ کم تر رتبے کا   ہندسہ \عددی{6 }ہے۔
	
	مساوات  \حوالہ{مساوات_ثنائی_ہندسوں}  میں سات کو تین مختلف طریقوں سے لکھا گیا ہے۔روز مرہ زندگی میں سات  پہلی طرز پر لکھا جاتا ہے۔یوں کاغذ پر لکھتے ہوئے کسی بھی عدد کے بائیں جانب صفر نہیں لکھے جاتے اور   عدد کے بائیں  جانب کاغذ کو خالی چھوڑا جاتا ہے۔یہاں یہ بات سمجھنا ضروری ہے کہ روز مرہ زندگی میں اعداد لکھتے وقت ان کی لمبائی یا ان میں کُل ہندسوں کی تعداد پہلے سے متعین نہیں کی جاتی۔کمپیوٹر میں چیزیں  کچھ  مختلف ہیں، جہاں صرف صفر \عددی{0} اور ایک  \عددی{1} کا وجود ممکن ہے۔کسی مقام پر اگر  \عددی{1} نہیں لکھا ہو تو اس پر \عددی{0}  لکھا ہو  گا۔یوں کسی بھی عدد کے بائیں جانب خالی جگہ کا کمپیوٹر میں کوئی مطلب نہیں۔یہاں  \عددی{0}   یا   \عددی{1} کا ہونا ضروری ہے۔کمپیوٹر میں ہر قسم کی معلومات لکھنے سے پہلے اس بات کا فیصلہ کیا جاتا ہے کہ اسے لکھنے کی خاطر کتنی جگہ  درکار ہو گی۔یوں اگر  عدد کو لکھنے کی خاطر تین ہندسوں کے لکھے جانے کے برابر جگہ مختص  کی گئی ہو تو اس تمام جگہ کو ہر صورت استعمال کرنا ہوگا ، مثلاً  سات کو   \عددی{7}    کی بجائے  \عددی{007} لکھنا ہو گا۔
\begin{gather}
\begin{aligned}\label{مساوات_ثنائی_ہندسوں}
7_{10} &\\
07_{10}&\\
007_{10}&
\end{aligned}
\end{gather}

	اعشاری نظام میں گنتی \عددی{0_{10}}    سے شروع ہوتی ہے اور بتدریج بڑھتے ہوئے   \عددی{9_{10}} تک پہنچتی ہے۔ اس دوران دہائی، سینکڑا، وغیرہ کے مقام پر صفر رہتا ہے اور انہیں عام طور نہیں لکھا جاتا۔گنتی   نو  تک پہنچنے کے بعد دہائی،  یعنی \عددی{10^1}،   وزن رکھنے والے مقام پر \عددی{0}   کی بجائے  \عددی{1} لکھا جاتا ہے اور اکائی،  یعنی \عددی{10^0}،     وزن رکھنے والے مقام پر دوبارہ  \عددی{0} تا \عددی{9}  گنتی کی جاتی  ہے۔ 
	
	اگر آپ کو اس پیراگراف کی سمجھ نہیں آئی تو اسے دوبارہ پڑھیں۔اس میں سادہ گنتی کی وضاحت کی گئی ہے۔ 
	
	اعشاری نظام میں اگر اعداد کو ایک ہندسے تک محدود کر دیا جائے تو اس میں \عددی{0_{10}}   سے \عددی{9_{10}}    تک گنتی ممکن ہو گی۔اگر اعداد کو دو ہندسوں تک محدود کر دیا جائے، یعنی اس میں زیادہ سے زیادہ دو ہندسے ہوں، تب \عددی{00_{10}}   سے \عددی{99_{10}}    تک گنتی ممکن ہو گی، اسی طرح  تین ہندسوں تک کے عدد استعمال کرنے سے \عددی{000_{10}}    سے \عددی{999_{10}}  تک گنتی کی جا سکتی ہے،  وغیرہ۔
	 


\حصہ{ہشتمی نظام گنتی}
	ہشتمی نظام  \عددی{0} تا \عددی{7}   ہندسوں پر مبنی ہے ۔اس  نظام میں آٹھ ہندسے ہیں لہٰذا  یہ اساس آٹھ   نظام ہے۔بالکل اعشاری نظام کی طرح، اس نظام میں اعداد لکھتے ہوئے اعشاریہ کے بائیں جانب پہلے ہندسے  کا وزن \عددی{8^0=1_{10}} ، دوسرے ہندسے کا \عددی{8^1=8_{10}}  ، تیسرے کا \عددی{8^2=64_{10}}،   وغیرہ، جبکہ اعشاریہ کے دائیں جانب پہلے  ہندسے کا وزن \عددی{8^{-1}=0.125_{10}}، دوسرے    کا   \عددی{8^{-2}=0.015625_{10}}  ہو گا، وغیرہ۔
\begin{gather}
\begin{aligned}\label{مساوات_ثنائی_آٹھ}
538.72_8&=[(5\times 8^2)+(3\times 8^1)+(8\times8^0)+(7\times 8^{-1})+(2\times 8^{-2})]_{10}\\
&=[(5\times 64)+(3\times 8)+(8\times 1)+(7\times 0.125)+(2\times 0.015625)]_{10}\\
&=[320+24+8+0.875+0.03125]_{10}\\
&=352.90625_{10}
\end{aligned}
\end{gather}

	 ہشتمی نظامِ گنتی کے لئے مساوات  \حوالہ{مساوات_ثنائی_عمومی_روپ}     درج ذیل روپ اختیار کرتی ہے۔
\begin{multline}
\cdots\, a_2\times 8^2+a_1\times 8^1+a_0\times 8^0+a_{-1}\times 8^{-1}+a_{-2}\times 8^{-2}\,\cdots\\
=(\cdots a_2a_1a_0\ . \,a_{-1}a_{-2}\cdots)_{8}
\end{multline}

	 ہشتمی نظام میں دیے گئے عدد کو اعشاری نظام میں تبدیل کرنا  مساوات  \حوالہ{مساوات_ثنائی_آٹھ}  میں دکھایا گیا ہے۔ہشتمی عدد کے زیر نوشت  میں \عددی{8}  اس بات کی یاد دہانی کراتا ہے کہ یہ عدد ہشتمی نظام میں لکھا گیا ہے۔ 
	 
	اس نظام میں گنتی \عددی{0}  سے شروع ہوتی ہے، \عددی{7}  تک پہنچنے کے  بعد \عددی{8^1}   وزن رکھنے والے مقام پر  \عددی{0}  کی بجائے   \عددی{1}  لکھا جاتا ہے اور  \عددی{8^0} وزن رکھنے والے مقام پر دوبارہ   \عددی{0}سے \عددی{7}    کی  گنتی شروع ہوتی ہے۔

\حصہ{ثنائی نظامِ گنتی} 
	مائکروکنٹرولر کی دنیا میں ثنائی نظامِ گنتی استعمال ہوتا ہے۔ثنائی نظام  دو ہندسوں،   \عددی{0} اور \عددی{1}،  پر مبنی ہے،  لہٰذا  یہ  اساس دو   کا نظام  ہے۔
	اس نظام میں گنتی \عددی{0} سے شروع ہوتی ہے، \عددی{1} تک پہنچنے کے بعد  \عددی{2^1}   وزن رکھنے  والی مقام پر \عددی{0} کی بجائے \عددی{1} لکھا جاتا ہے،  اور \عددی{2^0} وزن رکھنے والے مقام پر دوبارہ \عددی{0} سے \عددی{1}  گنتی شروع ہوتی ہے۔اس نظام میں گنتی کو  مساوات \حوالہ{مساوات_ثنائی_گنتی} میں دکھایا گیا ہے، جہاں زیر نوشت میں اساس لکھنے سے گریز کیا گیا ہے۔موازنہ کے لئے اعشاری گنتی بھی پیش کی  گئی ہے۔
\begin{gather}
\begin{aligned}\label{مساوات_ثنائی_گنتی}
0&=\phantom{000}0 &\quad \quad \quad  &16=10000\\
1&=\phantom{000}1 &\quad \quad \quad  &17=10001\\
2&=\phantom{00}10 &\quad \quad \quad  &18=10010\\
3&=\phantom{00}11 &\quad \quad \quad  &19=10011\\
4&=\phantom{0}100 &\quad \quad \quad  &20=10100\\
5&=\phantom{0}101 &\quad \quad \quad  &21=10101\\
6&=\phantom{0}110 &\quad \quad \quad  &22=10110\\
7&=\phantom{0}111 &\quad \quad \quad  &23=10111\\
8&=1000 &\quad \quad \quad  &24=11000\\
9&=1001 &\quad \quad \quad  &25=11001\\
10&=1010 &\quad \quad \quad  &26=11010\\
11&=1011 &\quad \quad \quad  &27=11011\\
12&=1100 &\quad \quad \quad  &28=11100\\
13&=1101 &\quad \quad \quad  &29=11101\\
14&=1110 &\quad \quad \quad  &30=11110\\
15&=1111 &\quad \quad \quad  &31=11111\\
\end{aligned}
\end{gather}
	اس نظام میں اعداد لکھتے ہوئے اعشاریہ کے بائیں جانب پہلے ہندسے کا وزن \عددی{2^0=1_{10}}   ہو گا،   دوسرے  ہندسے کا \عددی{2^1=2_{10}}   ، تیسرے کا \عددی{2^2=4_{10}}،    وغیرہ،  جبکہ اعشاریہ کے دائیں جانب پہلے ہندسے کا وزن \عددی{2^{-1}=0.5_{10}} ،دوسرے کا \عددی{2^{-2}=0.25_{10}} ہو گا۔
	
	 ثنائی نظامِ گنتی کے لئے ی مساوات  \حوالہ{مساوات_ثنائی_عمومی_روپ} درج ذیل روپ اختیار کرتی ہے۔
\begin{multline}\label{مساوات_ثنائی_عمومی_روپ_ثنائی}
\cdots\, b_2\times 2^2+b_1\times 2^1+b_0\times 2^0+b_{-1}\times 2^{-1}+b_{-2}\times 2^{-2}\,\cdots\\
=(\cdots b_2b_1b_0\, .\, b_{-1}b_{-2}\cdots)_{2}
\end{multline}
	مساوات  \حوالہ{مساوات_ثنائی_مثال}  میں ثنائی نظام میں دیے گئے عدد کو اعشاری نظام میں تبدیل کرنا دکھایا گیا ہے۔ ثنائی عدد کے زیر نوشت میں \عددی{2} اس بات کی یاد دہانی کراتا ہے کہ یہ عدد ثنائی نظام میں لکھا گیا ہے۔
\begin{gather} 
\begin{aligned}\label{مساوات_ثنائی_مثال}
1011.1_2&=[(1\times 2^3)+(0\times 2^2)+(1\times 2^1)+(1\times 2^0)+(1\times 2^{-1})]_{10}\\
&=[(1\times 8)+(0\times 4)+(1\times 2)+(1\times 1)+(1\times 0.5)]_{10}\\
&=[8+0+2+1+0.5]_{10}\\
&=11.5_{10}
\end{aligned}
\end{gather}
	
	 ثنائی عدد کے ہندسوں کو پکارنے کا طریقہ شکل \حوالہء{1.2}  میں  دکھایا گیا ہے۔  ثنائی عدد  کے  دائیں ترین  ہندسے کو کم تر رتبی    بِٹ    یا کم تر رتبی   ثنائی ہندسہ یا  بِٹ صفر یا  بِٹ   \عددی{b_0} کہیں گے؛  اس سے اگلے کو  بِٹ ایک یا بِٹ  \عددی{b_1}  اور اس سے اگلے کو بِٹ دو یا  بِٹ \عددی{b_2}،  وغیرہ؛ جبکہ بائیں  ترین  ہندسے کو بلند تر رتبی  ثنائی ہندسہ  یا بلند تر رتبی   بِٹ    یا (موجودہ مثال میں)     بِٹ سات یا  بِٹ \عددی{b_7}   کہیں گے۔


	اگر دیے گئے ثنائی  عدد  کے  اعشاریہ کے دائیں جانب کچھ نہ ہو،  تب  درج ذیل لکھا جا سکتا  ہے:
\begin{align}
1011_2=(2^3+2^1+2^0)_{10}=(8+2+1)_{10}=11_{10}
\end{align}
 جو  ہندسے  \عددی{1}   ہیں، ان کے وزن جمع کیے  جاتے ہیں۔

چار ہندسوں کا  ثنائی عدد \عددی{0000_2} تا  \عددی{1111_2}  گنتی کر سکتا ہے؛   اس سے بڑا عدد   لکھنے  کے لئے  چار سے زیادہ ہندسے درکار ہوں گے۔ مائکروکنٹرولر آٹھ ثنائی ہندسوں کے اعداد استعمال کرتا ہے جو   \عددی{00000000_2} تا \عددی{11111111_2}، یعنی \عددی{0_{10}} تا \عددی{255_{10}}     ظاہر کر سکتے ہیں۔

روز مرہ  زندگی میں اعشاری نظامِ گنتی استعمال کرتے ہوئے اعداد لکھتے  ہوئے  ان کی  بائیں جانب اضافی  صفر نہیں لکھے جاتے،  یعنی \عددی{27_{10}} کو \عددی{0027_{10}} نہیں لکھا جاتا۔کمپیوٹر کی دنیا میں اعداد عموماً آٹھ ہندسوں پر مبنی ثنائی عدد کی صورت میں لکھے جاتے ہیں؛ آٹھ سے کم ثنائی ہندسوں پر مبنی اعداد لکھتے  ہوئے،  بائیں جانب  اضافی صفر  لکھ کر انہیں    آٹھ ہندسوں کی صورت دی جاتی ہے۔ یوں \عددی{27_{10}} کو ہم  \عددی{101011_2}  کی بجائے  \عددی{00101011_2} لکھیں گے۔

\حصہ{اعشاری نظام سے ثنائی نظام میں تبادلہ}
اعشاری نظام میں دیے  گئے عدد کو ثنائی نظام میں لکھنے کی خاطر اس  عدد کو بار بار  \عددی{2}  سے تقسیم کریں،  حتٰی کہ یہ مزید تقسیم نہ ہو سکے۔ہر مرتبہ تقسیم کے بعد حاصل باقی لیں؛ پہلے حاصل باقی کو ثنائی عدد کے سب سے کم وزن  کے  مقام پر لکھیں؛ اگلے حاصل باقی کو اس سے دگنے وزن کے مقام پر لکھیں؛ اسی طرح آخری حاصل باقی کو  سب سے زیادہ وزن کے مقام پر لکھیں۔ یوں ثنائی عدد حاصل ہو گا۔	یہ طریقہ استعمال کرتے ہوئے \عددی{121_{10}}  کو ثنائی لکھائی میں لکھتے ہیں۔

\عددی{121} کو \عددی{2} سے تقسیم کرنے سے حاصل تقسیم \عددی{60} اور  باقی \عددی{1}   ملتا ہے۔\\
\عددی{60} کو \عددی{2} سے تقسیم کرنے سے حاصل تقسیم \عددی{30} اور  باقی \عددی{0}   ملتا ہے۔\\
\عددی{30} کو \عددی{2} سے تقسیم کرنے سے حاصل تقسیم \عددی{15} اور  باقی \عددی{0}   ملتا ہے۔\\
\عددی{15} کو \عددی{2} سے تقسیم کرنے سے حاصل تقسیم \عددی{7} اور  باقی \عددی{1}   ملتا ہے۔\\
\عددی{7} کو \عددی{2} سے تقسیم کرنے سے حاصل تقسیم \عددی{3} اور  باقی \عددی{1}   ملتا ہے۔\\
\عددی{3} کو \عددی{2} سے تقسیم کرنے سے حاصل تقسیم \عددی{1} اور  باقی \عددی{1}   ملتا ہے۔\\
\عددی{1} کو \عددی{2} سے تقسیم کرنے سے حاصل تقسیم \عددی{0} اور  باقی \عددی{1}   ملتا ہے۔


اب سب سے آخری  \قول{باقی} کو سب سے زیادہ وزن کے مقام پر اور سب سے پہلے  \قول{باقی}  کو سب سے کم وزن کے مقام پر لکھتے ہیں۔یوں \عددی{1111001_2} حاصل ہو گا،  لہٰذا 
 \begin{align*}
 121_{10}=1111001_2
 \end{align*}
 ہو گا جہاں سات ثنائی  ہندسے استعمال کیے گئے ہیں۔اپنی تسلی کے لئے  اس عدد کو واپس اعشاری نظام میں منتقل کرتے ہیں۔
\begin{align*}
1111001_2=2^6+2^5+2^4+2^3+2^0=64+32+16+8+1=121_{10}
\end{align*}
اس طریقہ کار  کی بہتر صورت پیش کرتے ہیں۔
\begin{otherlanguage}{english}
\begin{center}
\begin{tabular}{r|r|r}
2&121&\\
\hline
&60&1\\
\hline
&30&0\\
\hline
&15&0\\
\hline
&7&1\\
\hline
&3&1\\
\hline
&1&1\\
\hline
&0&1
\end{tabular}
\end{center}
\end{otherlanguage}
 عدد میں  اعشاریہ کے  بائیں جانب  حصہ کو حصہ صحیح  ،  جبکہ دائیں  حصہ کو حصہ مسکور    یا کسری کہتے ہیں۔
\begin{align*}
\overbrace{xxxxxx}^{\text{\RL{حصہ صحیح}}} \, . \, \underbrace{yyyyyy}_{\text{\RL{حصہ مسکور}}}
\end{align*}
یوں \عددی{121.6875} میں \عددی{121}    عدد صحیح اور \عددی{6875}   عدد مسکور  ہے۔

 عشری عدد     کے صحیح حصہ کو ثنائی نظام میں تبدیل کرنا آپ سیکھ چکے؛  حصہ مسکور  تبدیل کرنے   کا طریقہ  ذرہ   مختلف  ہے۔ آئیں یہ عمل سیکھیں۔
 
 
حصہ مسکور کو بار بار   \عددی{2} سے ضرب دیں۔اگر  حاصل ضرب کے  اعشاریہ کے بائیں جانب \عددی{1}  حاصل ہو تو اس  کو حاصل ضرب سے ہٹا کر  ثنائی عدد کے دائیں جانب منسلک کریں ورنہ  ثنائی عدد کے دائیں جانب \عددی{0}   منسلک کریں۔اس عمل کو ایک   مثال کی مدد  سے  سیکھتے ہیں۔ 
\begin{otherlanguage}{english}
\begin{center}
\begin{tabular}{R|L}
 &\text{\RL{ثنائی}}\\
\toprule
2\times 0.6875=1.375 & 0.1\\
2\times 0.3750=0.750&0.10\\
2\times 0.7500=1.500&0.101\\
2\times 0.5000=1.000&0.1011
\end{tabular}
\end{center}
\end{otherlanguage}
یوں \عددی{0.6875_{10}=0.1011_2} ہو گا ؛ آخر میں  دونوں حصوں  کو ملا کر ثنائی عدد حاصل کرتے ہیں۔
\begin{align*}
121.6875_{10}=111001.1011_2
\end{align*}

\حصہ{اساس سولہ کا   (سادس عشری)  نظام گنتی}
	اساس سولہ کے نظام میں اعداد کی  سولہ علامتیں ہیں۔ان میں پہلی دس علامتیں \عددی{0}  تا \عددی{9} ہیں،   جبکہ  باقی علامتیں،    بڑی لکھائی میں انگریزی حروف  تہجی کے پہلے چہ حروف یعنی  \عددی{A}، \عددی{B}، \عددی{C}، \عددی{D}، \عددی{E} اور \عددی{F}   ہیں۔  علامت  \عددی{A}  دس  \عددی{(10_{10})} کو ظاہر کرتی ہے،  یعنی \عددی{A=10_{10}} ہے ، جبکہ  \عددی{B} گیارہ کو  ، \عددی{B=11_{10}}،  اور اسی طرح چلتے ہوئے  \عددی{ F} پندرہ کو ظاہر کرتی ہے۔مساوات  \حوالہ{مساوات_ثنائی_تمام}  میں مختلف نظام  دیے   گئے ہیں۔ انہیں سمجھے  بغیر  آگے  ہرگز مت بڑھیں۔  
\begin{gather}
\begin{aligned}\label{مساوات_ثنائی_تمام}
&00_{10}=00_8=0000_2=0_{16}\\
&01_{10}=01_8=0001_2=1_{16}\\
&02_{10}=02_8=0010_2=2_{16}\\
&03_{10}=03_8=0011_2=3_{16}\\
&04_{10}=04_8=0100_2=4_{16}\\
&05_{10}=05_8=0101_2=5_{16}\\
&06_{10}=06_8=0110_2=6_{16}\\
&07_{10}=07_8=0111_2=7_{16}\\
&08_{10}=10_8=1000_2=8_{16}\\
&09_{10}=11_8=1001_2=9_{16}\\
&10_{10}=12_8=1010_2=A_{16}\\
&11_{10}=13_8=1011_2=B_{16}\\
&12_{10}=14_8=1100_2=C_{16}\\
&13_{10}=15_8=1101_2=D_{16}\\
&14_{10}=16_8=1110_2=E_{16}\\
&15_{10}=17_8=1111_2=F_{16}
\end{aligned}
\end{gather}
	اس نظام میں اشاریہ کی  بائیں جانب  پہلے ہندسے کا وزن    \عددی{16^0=1_{10}}،  دوسرے  کا   \عددی{16^1=16_{10}}،  اور تیسرے کا \عددی{16^2=256_{10}} ہو گا۔ 

مساوات  \حوالہ{مساوات_ثنائی_سادس}   میں سادس عشری یا اساس سولہ  نظام میں دیے گئے عدد کو اعشاری نظام میں تبدیل کرنا دکھایا گیا ہے۔ایسا کرتے ہوئے \عددی{A=10_{10}}  اور \عددی{C=12_{10}}  لئے گئے۔
\begin{gather}
\begin{aligned}\label{مساوات_ثنائی_سادس}
3AC.8_{16}&=(3\times 16^2)_{10}+(10\times 16^1)_{10}+(12\times 16^0)_{10}+(8\times 16^{-1})_{10}\\
&=(3\times 256)_{10}+(10\times 16)_{10}+(12\times 1)_{10}+(8\times 0.0625)_{10}\\
&=(768+160+12+0.5)_{10}\\
&=940.5_{10}
\end{aligned}
\end{gather}

	مساوات \حوالہ{مساوات_ثنائی_عمومی_روپ}    اساس سولہ کے لئے درج ذیل ہو گی۔
\begin{multline}\label{مساوات_ثنائی_عمومی_روپ_سادس}
\cdots\,a_2\times 16^2+a_1\times 16^1+a_0\times 16^0+a_{-1}\times 16^{-1}+a_{-2}\times 16^{-2}\,\cdots\\
=(\cdots a_2a_1a_0\, .\, a_{-1}a_{-2}\cdots)_{16}
\end{multline}
 

\حصہ{اساس دو کا اساس آٹھ میں تبادلہ}
	مساوات  \حوالہ{مساوات_ثنائی_تبدیلی_آٹھ}  میں  بائیں  ہاتھ      ثنائی عدد دیا گیا    ہے۔  اعشاریہ سے شروع کرتے ہوئے،  اعشاریہ کی دونوں جانب تین تین ہندسوں کے گروہ میں، اس ثنائی عدد کو  لکھیں۔اعشاریہ کی  بائیں جانب اگر آخر میں تین ہندسوں کا گروہ پورا نہ ہو تو بائیں اضافی  صفر  منسلک  کر کے  تین ہندسوں کا گروہ  پورا کریں؛ اسی طرح اعشاریہ کی  دائیں جانب اگر آخر میں تین ہندسوں کا گروہ پورا نہ ہو تو   دائیں جانب  اضافی صفر  منسلک  کر کے  تین ہندسوں کا گروہ پورا کریں۔اب مساوات  \حوالہ{مساوات_ثنائی_تمام}  کی مدد سے ان تین تین کے گروہ کی جگہ ان کا مساوی اساس آٹھ  ہندسہ لکھیں۔مساوات  \حوالہ{مساوات_ثنائی_تبدیلی_آٹھ}  میں یوں   دو مقامات  پر \عددی{100_2}  کی جگہ \عددی{4_8}  لکھا گیا،جبکہ   \عددی{101_2}   کی جگہ \عددی{5_8}،    اور \عددی{001_2}    کی جگہ \عددی{1_8}    لکھا گیا ہے۔اس طرح   اس  عدد کو  اساس آٹھ میں   منتقل کیا گیا۔یاد رہے،   اعشاریہ اپنی جگہ برقرار رکھتا ہے۔
	\begin{gather}
	\begin{aligned}\label{مساوات_ثنائی_تبدیلی_آٹھ}
	1101100.1_2&=(001\,\, 101 \,\,100 \,. \,100)_2\\
	&=(\phantom{0} 1\,\,\quad 5\,\,\quad 4\,\,\,\, . \quad 4)_8\\
	&=154.4_8
	\end{aligned}
	\end{gather}

	%?????KKKKK am here
\حصہ{اساس دو کا اساس سولہ میں تبادلہ}
	ثنائی عدد کو اساس سولہ میں لکھنے کی خاطر ثنائی عدد کو اعشاریہ سے شروع کرتے ہوئے اعشاریہ کی  دونوں جانب  چار چار ہندسوں کے گروہ میں لکھیں۔اگر اعشاریہ کی بائیں جانب آخر میں چار ہندسوں کا گروہ پورا نہ ہو تو عدد کی  بائیں جانب اضافی  صفر منسلک  کر کے  چار ہندسوں  کا گروہ  پورا کریں؛ اسی طرح  اگر اعشاریہ کی دائیں جانب آخر میں چار ہندسوں کا گروہ  پورا نہ ہوں تو دائیں جانب  اضافی صفر منسلک کر کے گروہ پورا کریں۔اب مساوات  \حوالہ{مساوات_ثنائی_تمام}   کی مدد سے ان چار چار کے گروہ کی جگہ ان کی  مساوی اساس سولہ کا ہندسہ لکھیں۔ یوں مساوات  \حوالہ{مساوات_ثنائی_سولہ_تبدیل}  میں  \عددی{1000_2} کی جگہ \عددی{8_{16}} لکھ کر،  \عددی{1100_2} کی جگہ \عددی{C_{16}}،  اور \عددی{0110_2}    کی جگہ \عددی{6_{16}}    لکھ کر   اساس سولہ میں مساوی   عدد حاصل کیا گیا۔ یاد رہے کہ اعشاریہ اپنی جگہ برقرار رکھتا ہے۔
\begin{gather}
\begin{aligned}\label{مساوات_ثنائی_سولہ_تبدیل}
1101100.1_2&=(0110  \,\,1100\,  . \, 1000)_2\\
&=(\phantom{0}\, 6\,\,\quad \,C\quad\, .\quad 8)_{16}\\
&=6C.8_{16}
\end{aligned}
\end{gather}
\حصہ{ اساس آٹھ اور اساس سولہ سے اساس دو میں تبادلہ}
	انہیں طریقوں کو الٹ استعمال کرتے ہوئے اساس آٹھ اور اساس سولہ کے اعداد با آسانی اساس دو میں لکھے جا سکتے ہیں۔مساوات  \حوالہ{مساوات_ثنائی_آٹھ_تبدیل_اساس}  میں اساس آٹھ:
 \begin{gather}
 \begin{aligned}\label{مساوات_ثنائی_آٹھ_تبدیل_اساس}
 372.5_8&=(3\quad\quad \,\, 7\quad\quad  2\quad  . \quad  5)_8\\
 &=(011\quad 111 \quad 010\,\,. \,\,\, 101)_2\\
 &=011111010.101_2
 \end{aligned}
 \end{gather}
 اور مساوات  \حوالہ{مساوات_ثنائی_سولہ_تبدیل_اساس} میں اساس سولہ کو ثنائی عدد کی شکل میں لکھنا دکھایا گیا ہے۔ 
\begin{gather}
 \begin{aligned}\label{مساوات_ثنائی_سولہ_تبدیل_اساس}
 9A2F.7_{16}&=(\quad 9 \quad \quad A\quad\quad  \,\, 2\quad \quad \, F\quad \quad . \quad \quad 7)_{16}\\
 &=(1001\quad 1010\quad 0010\quad 1111\quad . \quad 0111)_2\\
 &=(1001101000101111.0111)_2
 \end{aligned}
 \end{gather}
ہم نے دیکھا کہ ثنائی عدد کے ہندسوں  کو تین تین کے گروہ میں لکھنے سے اساس آٹھ اور چار چار کے گروہ میں لکھنے سے اساس سولہ  عدد حاصل کیا جا سکتا ہے۔ آئیں   درج بالا مساوات میں حاصل ثنائی عدد سے اساس آٹھ اور اساس سولہ  اعداد حاصل کریں۔
 \begin{align*}
 1001101000101111.0111_2&=(001\quad 001\quad 101\quad 000\quad 101\quad 111\,\, . \,\,011\quad 100)_2\\
 &=(\,\,1\quad \quad 1\quad \quad 5\quad \quad 0 \quad \quad 5\quad \quad 7\quad .\quad  3\quad \quad 4)_8\\
 &=115057.34_8\\
  1001101000101111.0111_2&=(1001\quad 1010\quad 0010\quad 1111\,\, . \,\,0111)_2\\
 &=(\,\,\,9\quad \quad \quad A\quad \quad 2\quad \quad \,\,F \quad  . \quad  7)_{16}\\
 &=9A2F.7_{16}
 \end{align*}
	مساوات  \حوالہ{مساوات_ثنائی_آٹھ_تبدیل_اساس}  اور  مساوات \حوالہ{مساوات_ثنائی_سولہ_تبدیل_اساس}  کی آخری لکیروں میں ثنائی اعداد کو دیکھتے ہوئے بہت جلد انسان اکتا جاتا ہے،  البتہ،  انہیں مساوات میں جہاں ثنائی اعداد کو گروہ کی شکل میں لکھا گیا ہے،  وہاں انہیں سمجھنا ممکن ہے۔ یہی وجہ ہے کہ ثنائی عدد کو  بالخصوص اور دیگر اعداد کو  بالعموم گروہوں کی شکل میں لکھا جاتا ہے۔
	
ایک ہندسے پر مبنی ثنائی عدد کو ثنائی ہندسہ یا بِٹ  کہتے ہیں؛    آٹھ ثنائی ہندسوں، ، یعنی آٹھ بِٹ،  کے گروہ  کو   ہشتمی ثنائی عدد یا  بائٹ کہتے ہیں۔بائٹ کو  عموماً  چار چار  ثنائی اعداد کے گروہ میں لکھا جاتا ہے۔یوں مساوات  \حوالہ{مساوات_ثنائی_سولہ_تبدیل_اساس}  میں دو بائٹ ہیں۔اسی مساوات کو الٹ چلاتے ہوئے یہ واضح ہے کہ ہشتمی ثنائی عدد کو چار چار ثنائی اعداد کے  گروہ میں لکھ کر انہیں جلد اساس سولہ میں لکھا جا سکتا ہے۔

\باب{بنیادی حساب}
	ثنائی نظام میں حساب بالکل اسی طرح کیا جاتا ہے جس طرح اعشاری نظام میں۔چند مثالوں کے مطالعہ سے  وضاحت ہو گی۔
	
	ثنائی نظام میں اعداد کا  مجموعہ  اعشاری نظام میں دو اعداد کے مجموعہ سے سمجھا جا سکتا ہے۔اعشاری نظام کی مندرجہ ذیل مثال پر غور کریں جس میں \عددی{37.5} اور \عددی{29.6}     جمع کیے گئے ہیں۔
\begin{align*}
\begin{split}
11\phantom{.1\,}&\\
37.5&\\
+29.6&\\
\noalign{\smallskip}\hline\noalign{\smallskip}
67.1&
\end{split}
\end{align*}
آپ نے دیکھا کہ    حاصل\عددی{(1)}  کو  (بائیں)   زیادہ  وزنی  مقام پر منتقل کیا   گیا۔یہی    ثنائی  جمع   میں کیا جائے گا۔ ثنائی نظام میں صرف دو ہندسے،  \عددی{0} اور \عددی{1}، پائے جاتے ہیں جن  کی چار ممکنہ  مجموعے درج ذیل ہیں۔
\begin{align*}
\begin{split}
1\phantom{1\,}&\\
1&\\
+1&\\
\noalign{\smallskip}\hline\noalign{\smallskip}
10&
\end{split}&
\begin{split}
&\\
&1\\
+&0\\
\noalign{\smallskip}\hline\noalign{\smallskip}
0&1
\end{split}&
\begin{split}
&\\
&0\\
+&1\\
\noalign{\smallskip}\hline\noalign{\smallskip}
0&1
\end{split}&
\begin{split}
&\\
&0\\
+&0\\
\noalign{\smallskip}\hline\noalign{\smallskip}
0&0
\end{split}
\end{align*}


پہلی تین جمع   میں حاصل  \عددی{0} جبکہ آخری میں حاصل \عددی{1}  ہے۔

آئیں،  زیادہ ثنائی ہندسوں کے اعداد کی  جمع   کی  مثالیں  دیکھیں؛ ان کی  اعشاری نظام میں  جمع  بھی دی گئی ہیں۔
\begin{align*}
\begin{split}
&1\\
&13\\
+&09\\
\noalign{\smallskip}\hline\noalign{\smallskip}
&22_{10}
\end{split}&
\begin{split}
1&\phantom{11}1\\
&1101\\
+&1001\\
\noalign{\smallskip}\hline\noalign{\smallskip}
1&0110_2
\end{split}&&
\begin{split}
&\\
&3\\
+&2\\
\noalign{\smallskip}\hline\noalign{\smallskip}
&5_{10}
\end{split}&
\begin{split}
1&\\
&11\\
+&10\\
\noalign{\smallskip}\hline\noalign{\smallskip}
1&01_2
\end{split}
\end{align*}
دائیں ہاتھ ثنائی \عددی{11} اور \عددی{10} جمع کر کے \عددی{101_2} حاصل کیا گیا جو اعشاری نظام میں  \عددی{3+2=5}ہو گا، جبکہ  بائیں ہاتھ ثنائی \عددی{1101} اور \عددی{1001} جمع کر کے \عددی{10110_2} حاصل کیا گیا   جو اعشاری نظام میں \عددی{13+9=22} کے مترادف ہے۔

آخر میں، کسری اعداد کی جمع  کی ایک مثال دیکھتے ہیں۔
\begin{align*}
\begin{split}
&1\\
&5.75\\
+&3.50\\
\noalign{\smallskip}\hline\noalign{\smallskip}
&9.25_{10}
\end{split}&
\begin{split}
&111\\
&101.11\\
+&\phantom{1}11.10\\
\noalign{\smallskip}\hline\noalign{\smallskip}
1&001.01_2
\end{split}
\end{align*}


\حصہ{ثنائی نظام میں  اعداد منفی کرنا}
دو  بِٹ (ثنائی   عدد)  منفی کرنے کے درج ذیل   چار ممکنات  پائے جاتے ہیں۔
\begin{align*}
0-0&=0\\
1-0&=1\\
1-1&=0\\
0-1&=1 \quad \text{\RL{\small{(ادھار ایک)}}}
\end{align*}
ی آخری مساوات میں صفر سے ایک اس صورت منفی کیا دکھایا گیا ہے جب  ادھار \عددی{1}   لینا ممکن ہو۔ایک اور مثال دیکھتے ہیں۔
\begin{align*}
\begin{split}
&6.25\\
-&5.50\\
\noalign{\smallskip}\hline\noalign{\smallskip}
&0.75_{10}
\end{split}&
\begin{split}
&110.01\\
-&101.1\\
\noalign{\smallskip}\hline\noalign{\smallskip}
&\phantom{11}0.11_2
\end{split}
\end{align*}
ثنائی منفی کی چند مثالیں حل کر کے اعشاری منفی سے ان کی تصدیق کریں۔ ایسا کرنے سے زیادہ   وضاحت ہو گی۔



\حصہ{اساسی تکملہ یا \عددی{r} کا تکملہ}
کسی بھی اساسی نظام میں،  ہندسہ  کو اساس ، \عددی{(r)}،  سے منفی کرنے سے ہندسے  کا اساسی   تکملہ (یا \عددی{r} کا تکملہ)  حاصل ہو گا۔یوں،   ہندسہ اور ہندسے  کے اساسی تکملہ کا مجموعہ اساس کے برابر ہو گا۔مثلاً، اعشاری نظام میں  \عددی{3}   کا اساسی تکملہ \عددی{10-3=7}  ،  جبکہ  \عددی{7}    کا اساسی تکملہ \عددی{3} اور ان دونوں کا مجموعہ \عددی{3+7=10}   اعشاری نظام کے اساس کے برابر ہے۔اسی طرح   \عددی{5}   کا    اساسی تکملہ  \عددی{5}، اور \عددی{9}   کا اساسی تکملہ \عددی{1}  ہو گا۔

درج بالا  مثالوں سے  واضح ہے کہ کسی بھی  ہندسہ (مثلاً \عددی{3})   کے اساسی تکملہ (یعنی  \عددی{7})   کا اساسی تکملہ    وہی ہندسہ  (یعنی \عددی{3})   ہو گا۔ 

اساسی تکملہ کے تصور کو ایک سے زائد ہندسوں پر مبنی عدد تک وسعت دیتے ہیں۔اساس \عددی{r}  کے اعدادی نظام میں عدد \عددی{N}،    جو  \عددی{n}   ہندسوں پر مبنی  ہو،  کے  اساسی تکملہ (یا   \عددی{r} کے تکملہ) سے مراد عدد  \عددی{r^n-N} ہو گا۔ 

اساس دس کے اساسی تکملہ کو عام طور  \عددی{10} کا تکملہ   کہتے ہیں۔اسی طرح اساس دو کے تکملہ کو   \عددی{2} کا تکملہ   کہتے ہیں۔ 

اعشاری نظام میں عدد   \عددی{10^n}      کے  سب سے  وزنی   ہندسے کی قیمت   \عددی{1} ہو گی، اور اس  کی  دائیں جانب  \عددی{0} قیمت کے  \عددی{n}  ہندسے ہوں گے۔
\begin{gather}
\begin{aligned}
10^2&=100_{10}\\
10^5&=100000_{10}\\
10^7&=10000000_{10}
\end{aligned}
\end{gather}
اعشاری نظام کی   اساس \عددی{r=10}   ہے۔اس نظام میں  عدد \عددی{N}،    جس میں \عددی{n}   ہندسے ہوں،  کے اساسی تکملہ (یعنی   \عددی{10} کے تکملہ)  سے مراد عدد \عددی{10^n-N} ہو گا۔یوں \عددی{N=5391} جس   میں چار ہندسے   \عددی{(n=4)} ہیں ، کا   \عددی{10} کا تکملہ    درج ذیل ہو گا۔
\begin{align}
(10^4-5391)_{10}=(10000-5391)_{10}=4609_{10}
\end{align}
اسی طرح  عدد \عددی{320753}   جس میں  \عددی{6} ہندسے ہیں کا اساسی تکملہ:
\begin{align}
(10^6-320753)_{10}=(1000000-320753)_{10}=679247_{10}
\end{align} 
اور \عددی{679247}   کا   \عددی{2} کا تکملہ   درج ذیل ہو گا۔ 
\begin{align}
(10^6-679247)_{10}=(1000000-679247)_{10}=320753_{10}
\end{align} 

ہر عدد \عددی{N}  کے اساسی تکملہ کا اساسی تکملہ وہی عدد  \عددی{N} ہو گا۔اس کا ثبوت کچھ یوں  ہے: عددی \عددی{N} کا اساسی تکملہ \عددی{r^n-N} اور عدد   \عددی{r^n-N} کا اساسی تکملہ \عددی{r^n-(r^n-N)} یعنی \عددی{N} ہو گا۔


ثنائی نظام کی   اساس  \عددی{2}  ہے لہٰذا \عددی{n}   ہندسوں پر مبنی ثنائی عدد \عددی{N}   کے \عددی{2} کا  تکملہ    (یعنی اساسی تکملہ)   \عددی{2^n-N}    ہو گا۔

ثنائی  نظام میں عدد   \عددی{10^n}      کے  سب سے  وزنی   ہندسے کی قیمت   \عددی{1} ہو گی، اور اس  کی  دائیں جانب  \عددی{0} قیمت کے  \عددی{n}  ہندسے ہوں گے۔
\begin{gather}
\begin{aligned}
2^2&=100_2\\
2^5&=100000_2\\
2^7&=10000000_2
\end{aligned}
\end{gather}
یوں \عددی{1011_2}     اور \عددی{10001_2}   کے   \عددی{2}  کے تکملہ   بالترتیب  درج ذیل ہوں گے۔
\begin{gather}
\begin{aligned}\label{مساوات_حساب_تکملہ_اساسی}
(2^4-1011)_2&=(10000-1011)_2=0101_2\\
(2^5-10001)_2&=(100000-10001)_2=01111_2
\end{aligned}
\end{gather} 


\حصہ{اساس منفی ایک  تکملہ یا  \عددی{(r-1)} کا تکملہ}
اساس \عددی{r}  کے نظام    میں،    عدد  \عددی{N}کے اساس منفی ایک (\عددی{r-1})  کے تکملہ  سے مراد    \عددی{r^n-1-N} ہے۔اعشاری نظام میں اساس منفی ایک کے   تکملہ کو عموماً \عددی{9} کا تکملہ (نو کا تکملہ)       اور ثنائی نظام  میں اسے    \عددی{1} کا تکملہ    (ایک کا تکملہ) کہتے ہیں۔
 
اعشاری نظام میں \عددی{376}   اور \عددی{7852}   کے  \عددی{9} کے تکملہ ،  بالترتیب   مندرجہ ذیل ہوں گے۔ 
\begin{gather}
\begin{aligned}
10^3-1-376&=1000-1-376\\
&=999-376\\
&=623_{10}\\
10^4-1-7852&=10000-1-7852\\
&=9999-7852\\
&=2147_{10}
\end{aligned}
\end{gather}
اعشاری نظام میں عدد  \عددی{10^n-1}  ،    \عددی{n}  ہندسوں پر مشتمل ہو گا،  جہاں ہر ہندسے کی قیمت   \عددی{9} ہو گی۔
\begin{gather}
\begin{aligned}
10^3-1&=1000-1=999_{10}\\
10^6-1&=1000000-1=999999_{10}\\
10^8-1&=100000000-1=99999999_{10}
\end{aligned}
\end{gather}
ثنائی  نظام میں عدد  \عددی{2^n-1}  ،    \عددی{n}  ہندسوں پر مشتمل ہو گا،  جہاں ہر ہندسے کی قیمت   \عددی{1} ہو گی۔
\begin{gather}
\begin{aligned}
2^3-1&=1000-1=111_{2}\\
2^5-1&=100000-1=11111_{2}\\
2^8-1&=100000000-1=11111111_{2}
\end{aligned}
\end{gather}
 ثنائی نظام میں \عددی{1001_2}  اور  \عددی{101110_2}    کے   \عددی{1} کے تکملہ، بالترتیب، درج ذیل ہوں گے- 
\begin{gather}
\begin{aligned}
2^4-1-1001&=1111-1001=0110_2\\
2^6-1-101110&=111111-101110=010001_2
\end{aligned}
\end{gather}
آپ دیکھ سکتے ہیں کہ ثنائی ہندسہ \عددی{0}   کا \قول{   ایک  کا تکملہ}،   ثنائی ہندسہ \عددی{1}   ہو گا  ، اور اسی طرح عدد  \عددی{1} کا    \قول{ایک کا تکملہ}،  ثنائی ہندسہ \عددی{0} ہو گا۔ ہم کہتے ہیں \عددی{0} کا متمم \عددی{1} اور \عددی{1} کا متمم \عددی{0} ہے۔

 ثنائی عدد    \عددی{N} کا اساس منفی ایک کا   تکملہ،   \عددی{\overline{N}}   سے ظاہر کیا جاتا ہے لہٰذا  درج ذیل لکھا جا سکتا ہے۔
\begin{gather}
\begin{aligned}
\overline{1}_2&=0_2\\
\overline{0}_2&=1_2\\
\overline{1001}_2&=0110_2\\
\overline{101110}_2&=010001_2
\end{aligned}
\end{gather}
ان دو مثالوں سے ایک اہم  حقیقت   واضح ہوتا ہے: ثنائی عدد  میں ہر ہندسے کا متمم لینے سے (یعنی  ہر  \عددی{0} کو \عددی{1}،  اور ہر  \عددی{1} کو \عددی{0} کرنے سے)  اس کا ایک کا  تکملہ   یا متمم  حاصل ہو گا ۔

\موٹا{ثنائی عدد کے  ہر بِٹ کا متمم لینے سے  عدد کا   \عددی{1} کا تکملہ  (یعنی متمم)  حاصل ہو گا۔}

  اساس \عددی{r} نظام میں  \عددی{r} کے  تکملہ سے مراد \عددی{r^n-N}   اور  \عددی{(r-1)} کے تکملہ    سے مراد \عددی{r^2-1-N} ہے،  لہٰذا  \عددی{(r-1)} کے تکملہ  کے ساتھ \عددی{1} جمع کر کے   \عددی{r} کا تکملہ  حاصل کیا جا سکتا ہے، یعنی عدد کے متمم کے ساتھ   \عددی{1} جمع کر کے  \عددی{2} کا تکملہ  حاصل ہو گا۔اس طرح اساسی تکملہ کا حصول عموماً   زیادہ آسان ثابت ہوتا ہے۔  مساوات \حوالہ{مساوات_حساب_تکملہ_اساسی} میں دیے گئے   اعداد کے    \عددی{2} کے تکملہ   ہم اس طریقہ سے حاصل کرتے ہیں۔
  
چونکہ \عددی{\overline{1011}=0100} ہے    لہٰذا   \عددی{1011} کا اساسی تکملہ  \عددی{0100+1=0101} ہو گا۔اسی طرح   \عددی{10001} کے متمم \عددی{01110}   کے ساتھ \عددی{1} جمع کرنے سے اس    کا اساسی تکملہ \عددی{01110+1=01111} حاصل  ہو گا۔

\حصہ{دو اعداد کی منفی بذریعہ اساسی تکملہ}\شناخت{حصہ_حساب_منفی_بذریعہ_اساسی_تکملہ}
قلم و کاغذ  کے ساتھ،  \عددی{M} سے \عددی{N}  منفی کرنا چھوٹی جماعتوں میں سکھایا جاتا ہے۔برقیات  میں تکملہ کی مدد سے دو اعداد منفی کیے جاتے ہیں، جہاں دونوں اعداد میں ہندسوں کی تعداد برابر ہونا لازم ہے ۔اساسی تکملہ  کی مدد سے  \عددی{M-N}   مندرجہ ذیل طریقہ کار سے حاصل کیا جاتا ہے۔
\begin{itemize}
    \item
     دونوں  اعداد میں ہندسوں کی تعداد   برابر  کرنے کی خاطر،  کم ہندسوں  والے   عدد کی بائیں جانب  (درکار تعداد کی) اضافی صفریں  چسپاں کریں۔ فرض کریں اب ہر عدد میں   \عددی{n}  ہندسے پائے جاتے ہیں۔ 
    \item
    \عددی{M} کے ساتھ \عددی{N}   کا اساسی تکملہ جمع کر کے  \موٹا{مجموعہ}   \عددی{M+r^n-N}  حاصل کریں۔
   \item
     \عددی{M}   کی قیمت \عددی{N}   کی قیمت سے زیادہ ہونے کی صورت میں، آخری  (بائیں)  ہندسے جمع کرنے سے  حاصل  \عددی{1} پیدا ہو  گا،  جس کی بنا  یہ \ترچھا{مجموعہ}  \عددی{n+1} ہندسوں پر مشتمل  ہو گا اور اس کا  بایاں ہندسہ      \عددی{1}   ہو گا۔اس  بائیں ہندسے    کو (یعنی حاصل \عددی{1} کو )  نظر انداز کریں؛ باقی   \عددی{n} ہندسوں پر مبنی عدد اصل جواب ہوگا۔
    \item
\عددی{M}کی قیمت  \عددی{N} کی قیمت سے کم  ہونے کی صورت میں،  آخری  (بائیں)  ہندسے جمع کرنے سے  حاصل  \عددی{1} پیدا  \موٹا{نہیں}  ہو  گا؛     \ترچھا{مجموعہ}   منفی عدد کو ظاہر کرے گا، اور \عددی{n} ہندسوں  پر مبنی ہو گا۔مجموعے کا اساسی تکملہ لے کر اس کی بائیں جانب  منفی علامت  منسلک  کر کے  جواب  حاصل ہو گا۔
\end{itemize}
ان  دونوں صورتوں کی وضاحت مثالوں  سے ہو  گی۔ 

\ابتدا{مثال}
 اعشاری اعداد  کا حاصل منفی  \عددی{7852-974} دس کے    تکملہ    کی مدد سے   دریافت  کریں۔ 
 
\ترچھا{جواب}:\quad 
یہاں بڑا عدد \عددی{7852}   چار ہندسوں پر مبنی ہے،  لہٰذا چھوٹا  عدد  \عددی{0974} لکھیں اور  \عددی{n=4}  لیں۔یوں \عددی{0974} کا اساسی  تکملہ \عددی{10000-0974=9026}  ہو گا، جس کو \عددی{7852} کے ساتھ جمع کرنے سے  \عددی{5} ہندسوں کا  \ترچھا{مجموعہ}  \عددی{9026+7852=16878} حاصل ہو گا۔چونکہ یہ عدد \عددی{5} ہندسوں پر مبنی ہے،  لہٰذا  بائیں ہندسے   کو نظر انداز کرتے ہوئے \عددی{6878} کو جواب تسلیم کرتے ہیں۔(ہم درحقیقت آخری ہندسوں کی جمع سے پیدا حاصل \عددی{1} کو رد کرتے ہیں۔ چونکہ یہ مجموعہ میں بائیں ترین مقام پر اترتا ہے لہٰذا مجموعہ کا بایاں ہندسہ رد کر کے جواب حاصل ہو گا۔)
\begin{center}
\begin{otherlanguage}{english}
\begin{tikzpicture}
%\draw[thick,black] (-1,-1) grid (1,1);
%\draw [thin, gray,step=0.1](-1,-1) grid (1,1);
\draw(0,0)node{$
\begin{tabular}{r}
$1\phantom{7852}$\\
$7852$\\
$+9026$\\
\hline
$16878$
\end{tabular}
$};
\draw(-0.4,0.7) to [out=180, in=45]++(-3,-0.5)node[below]{\begin{tabular}{r}
\RL{حاصل $1$  کو نظرانداز کر کے}\\
\RL{$6878$ کو جواب تسلیم کرتے ہیں}
 \end{tabular}
};
\end{tikzpicture}\quad\quad
\begin{tikzpicture}
\draw(0,0)node{$
\begin{tabular}{r}
$10000$\\
$-0974$\\
\hline
$9026$
\end{tabular}
$};
\end{tikzpicture}
\end{otherlanguage}
\end{center}
\انتہا{مثال}
\ابتدا{مثال}
دس کے تکملہ    کی مدد سے \عددی{974-7852}   حاصل کریں۔ 

\ترچھا{جواب}:\quad 
عدد \عددی{7852}   کے  اساسی تکملہ   \عددی{10000-7852=2148}   کا    \عددی{0974} کے ساتھ  مجموعہ     لیتے ہوئے: \عددی{0974+2148=3122}   آخری حاصل \عددی{1}  نہیں    پیدا ہوتا، لہٰذا یہ مجموعہ 
  \عددی{4}  ہندسوں پر مشتمل ہے؛  اس کے اساسی تکملہ \عددی{10000-3122=6878}   کے ساتھ منفی  علامت چسپاں کرتے ہوئے \عددی{-6878} کو جواب تسلیم کرتے ہیں۔
\begin{center}
\begin{otherlanguage}{english}
\begin{tikzpicture}
\draw(0,0.5)node[above,yshift=2em]{جواب}node[above,yshift=1em]{$-6878$};
\end{tikzpicture}\quad\quad
\begin{tikzpicture}
%\draw[thick,black] (-1,-1) grid (1,1);
%\draw [thin, gray,step=0.1](-1,-1) grid (1,1);
\draw(0,0)node{$
\begin{tabular}{r}
$10000$\\
$-3122$\\
\hline
$6878$
\end{tabular}
$};
\end{tikzpicture}\quad\quad
\begin{tikzpicture}
\draw(0,0)node{$
\begin{tabular}{r}
$0974$\\
$+2148$\\
\hline
$3122$
\end{tabular}
$};
\end{tikzpicture}\quad\quad
\begin{tikzpicture}
\draw(0,0)node{$
\begin{tabular}{r}
$10000$\\
$-7852$\\
\hline
$2148$
\end{tabular}
$};
\end{tikzpicture}
\end{otherlanguage}
\end{center}
\انتہا{مثال}

ثنائی اعداد بھی بالکل اسی طرح منفی کیے جاتے ہیں۔ان کی بھی دو مثالیں پیش کرتے ہیں۔
	

\ابتدا{مثال}
 اساسی تکملہ کی مدد سے مندرجہ ذیل حاصل کریں۔
 
(ا) \عددی{1011_2-11001_2} اور (ب) \عددی{11001_2-1011_2}

\ترچھا{جواب}:\quad
(ا) چونکہ \عددی{\overline{11001}=00110} ہے،   لہٰذا   دو کا تکملہ    \عددی{00110+1=00111} ہو گا۔ اس کو  دوسرے عدد \عددی{01011_2}  (جس کی بائیں جانب  اضافی \عددی{0} چسپاں کر کے   ہندسوں کی  تعداد  پوری کی گئی) کے ساتھ جمع کرتے ہیں۔
\begin{align*}
\begin{split}
01011&\\
+00111&\\
\noalign{\smallskip}\hline\noalign{\smallskip}
10010&
\end{split}
\end{align*}
بائیں آخری  ہندسوں کو جمع کرتے ہوئے  حاصل  \عددی{1}  پیدا نہیں ہوا،  لہٰذا اس کا   \عددی{2}  کا تکملہ  لینا ہوگا۔چونکہ  \عددی{\overline{10010}=01101} ہے لہٰذا اساسی تکملہ \عددی{01101+1=01110} ہو گا، جس کی  بائیں جانب منفی  علامت چسپاں کر کے  نتیجہ  \عددی{-01110_2} حاصل کرتے ہیں۔

\ترچھا{جواب}:\quad  (ب) یہاں ایک عدد پانچ ہندسوں پر مشتمل  ہے،  لہٰذا دوسرے عدد  میں بھی  پانچ ہندسے پورے  کیے  جائیں گے۔یوں \عددی{1011}   کو \عددی{01011} لکھ کر،  اس  کے   متمم 
 \عددی{\overline{01011}=10100} سے  عدد کا اساسی تکملہ \عددی{10100+1=10101} حاصل کر  کے،  دوسرے عدد کے ساتھ جمع کرتے ہیں۔
\begin{align*}
\begin{split}
1\phantom{11001}&\\
11001&\\
+10101&\\
\noalign{\smallskip}\hline\noalign{\smallskip}
101110&
\end{split}
\end{align*}
آخری ہندسے جمع کرتے ہوئے حاصل \عددی{1} پیدا ہوا جس کو نظرانداز کرکے  باقی مجموعہ،  \عددی{01110_2} ، کو نتیجہ  تسلیم کرتے  ہیں۔
\انتہا{مثال}


\حصہ{دو اعداد کی منفی بذریعہ اساس منفی ایک    کا تکملہ}\شناخت{حصہ_حساب_منفی_بذریعہ_اساس_منفی_ایک_تکملہ}
اساس منفی ایک تکملہ  کی مدد سے بھی   \عددی{M-N}    حاصل کیا جا سکتا ہے۔ اس کا  طریقہ کار  درج ذیل ہے جہاں    دونوں اعداد میں ہندسوں کی تعداد برابر ہونا لازم ہے ۔
\begin{itemize}
    \item
     دونوں  اعداد میں ہندسوں کی تعداد   برابر  کرنے کی خاطر،  کم ہندسوں  والے   عدد کی بائیں جانب  (درکار تعداد کی) اضافی صفریں  چسپاں کریں۔ فرض کریں اب ہر عدد میں   \عددی{n}  ہندسے پائے جاتے ہیں۔ 
    \item
    \عددی{M} کے ساتھ \عددی{N}   کا اساس منفی ایک کا   تکملہ جمع کر کے  \موٹا{مجموعہ}   \عددی{M+r^n-1-N}  حاصل کریں۔
   \item
     \عددی{M}   کی قیمت \عددی{N}   کی قیمت سے زیادہ ہونے کی صورت میں، آخری  (بائیں)  ہندسے جمع کرنے سے  حاصل  \عددی{1} پیدا ہو  گا،        جس کی بنا  یہ \ترچھا{مجموعہ}  \عددی{n+1} ہندسوں پر مشتمل  ہو گا اور اس کا  بایاں ہندسہ      \عددی{1}   ہو گا۔اس  بائیں ہندسے    کو (یعنی حاصل \عددی{1} کو )  نظر انداز کرنے کی بجائے،   مجموعہ سے خارج کر کے،  \عددی{1} وزن مختص کریں اور  \عددی{n} ہندسوں کے باقی مجموعہ کے ساتھ جمع کر کے جواب حاصل کریں۔ اس عمل کو واپسیں آخری حاصل ایک \عددی{(1)}  کہتے ہیں۔
    \item
\عددی{M}کی قیمت  \عددی{N} کی قیمت سے کم  ہونے کی صورت میں،  آخری  (بائیں)  ہندسے جمع کرنے سے  حاصل  \عددی{1} پیدا  \موٹا{نہیں}  ہو  گا؛     \ترچھا{مجموعہ}   منفی عدد کو ظاہر کرے گا، اور \عددی{n} ہندسوں  پر مبنی ہو گا۔مجموعے کا اساس منفی ایک کا   تکملہ لے کر اس کی بائیں جانب  منفی علامت  منسلک  کر کے  جواب  حاصل ہو گا۔
\end{itemize}
ان  دونوں صورتوں کی وضاحت مثالوں  سے ہو  گی۔ 

\ابتدا{مثال} 
نو کا  تکملہ    استعمال کرتے ہوئے    \عددی{974-7852} حاصل کریں۔ 

\ترچھا{جواب}:\quad
  عدد \عددی{974}   کے بائیں \عددی{0} چسپاں کر کے  اس میں  ہندسوں کی تعداد  پوری کریں اور  \عددی{7852} کے اساس منفی ایک کے  تکملہ \عددی{9999-7852=2147}کے ساتھ جمع کریں۔
  \begin{align*}
  \begin{split}
  2147&\\
  +0974&\\
  \noalign{\smallskip}\hline\noalign{\smallskip}
  3121&
  \end{split}
  \end{align*}

آخری  (بائیں) ہندسے جمع کرنے  سے  حاصل \عددی{1} پیدا نہیں ہوا، لہٰذا مجموعہ  چار ہندسوں پر مشتمل ہے۔ اس کے  اساس منفی ایک کے   تکملہ \عددی{9999-3121=6878} کے بائیں منفی علامت منسلک کر کے  جواب \عددی{-6878} حاصل کرتے ہیں۔
\انتہا{مثال}
\ابتدا{مثال}
نو کا  تکملہ    استعمال کرتے ہوئے \عددی{7852-974}    حاصل کریں۔ 

\ترچھا{جواب}\quad
چھوٹے عدد \عددی{974} میں ہندسوں کی تعداد پوری کر کے اس کے اساس منفی ایک کے     تکملہ     \عددی{9999-0974=9025} کو \عددی{7852} کے ساتھ جمع کرتے ہیں۔
 \begin{align*}
  \begin{split}
  1\phantom{7852}&\\
  7852&\\
  +9025&\\
  \noalign{\smallskip}\hline\noalign{\smallskip}
  16877&
  \end{split}
  \end{align*}
  آخری (بائیں) ہندسے جمع کرتے ہوئے حاصل \عددی{1} پیدا ہوا جس کی بنا یہ مجموعہ \عددی{5} ہندسوں پر مشتمل ہے۔ ہم اس حاصل \عددی{1} کو وزن \عددی{1} مختص کر کے باقی \عددی{4} ہندسوں پر مبنی مجموعہ \عددی{6877} کے ساتھ جمع کر کے جواب \عددی{6877+1=6878} حاصل کرتے ہیں۔
\انتہا{مثال}
اب ہم ثنائی اعداد کی مثال لیتے ہیں۔
\ابتدا{مثال}
 مندرجہ ذیل  کو  \عددی{1} کے تکملہ    کی مدد سے حل کریں۔

(ا) \عددی{101110_2-11011_2}،  (ب) \عددی{11011_2-101110_2} 

\ترچھا{حل}:\quad
 (ا)  منفی ہونے والے عدد میں ہندسوں کی تعداد پوری کر کے اس کا  متمم:
 \begin{align*}
 \overline{011011}=100100
 \end{align*}

دوسرے عدد کے ساتھ جمع کرتے ہیں۔
  \begin{align*}
  \begin{split}
  1\phantom{101110}&\\
 101110&\\
  +100100&\\
  \noalign{\smallskip}\hline\noalign{\smallskip}
  1010010&
  \end{split}
  \end{align*}
آخری حاصل \عددی{1} کو باقی عدد سے علیحدہ کر کے اسے \عددی{1} کا وزن مختص کر کے (یعنی اس کو    اکائی  تصور کر کے)،   دائیں چھ ہندسوں پر   مشتمل مجموعہ \عددی{010010} کے ساتھ جمع کرتے  ہوئے  جواب حاصل کرتے ہیں۔
\begin{align*}
\begin{split}
010010&\\
+1&\\
  \noalign{\smallskip}\hline\noalign{\smallskip}
010011&
\end{split}
\end{align*}
(ب)
متمم \عددی{\overline{101110}=010001} کو دوسرے عدد کے ساتھ جمع کرتے ہیں۔
\begin{align*}
\begin{split}
010001&\\
+011011&\\
  \noalign{\smallskip}\hline\noalign{\smallskip}
101100&
\end{split}
\end{align*}
چونکہ  آخری حاصل صفر ہے،  لہٰذا  مجموعے   کے متمم  \عددی{\overline{101100}=010011}   کے ساتھ منفی کی علامت چسپاں کر کے جواب \عددی{-010011_2} حاصل کرتے ہیں۔
 \انتہا{مثال}
 
 
\حصہ{مثبت اور منفی اعداد}
روز مرہ  زندگی میں مثبت اعداد لکھتے ہوئے انہیں بغیر کسی علامت کے،  یا  مثبت   علامت \عددی{(+)} کے ساتھ  لکھا جاتا ہے،  البتہ منفی اعداد کے ساتھ منفی علامت \عددی{(-)}  ضرور لکھی جاتی ہے۔یوں   درج ذیل اعداد درست  لکھے گئے ہیں۔
\begin{align*}
+3025, \quad \quad  3025, \quad \quad  -3025
\end{align*}
کسی بھی عدد کے مثبت  یا منفی ہونے کو اس عدد کی علامت کہتے ہیں۔ یوں،  وہ اعداد جو مثبت  علامت \عددی{(+)} یا منفی  علامت \عددی{(-)}   رکھتے ہوں   علامت دار  اعداد   کہلاتے ہیں،  اور جن  کی   علامت نہ ہو بے علامت اعداد کہلاتے ہیں۔ اعداد کو ان کی علامت   اور قدر  سے ظاہر کرنے  کو  علامت دار  قدر   اظہار کہتے ہیں۔ 

کمپیوٹر ثنائی اعداد ،    \عددی{0} اور \عددی{1}،    استعمال کرتا ہے،  اور  ہر   معلومات کو انہیں   سے ظاہر کرتا ہے۔روایتاً    مثبت علامت \عددی{(+)} کو \عددی{0} (صفر )     اور نفی علامت \عددی{(-)} کو  \عددی{1} (ایک)  سے ظاہر کیا جاتا ہے۔علامت عدد کی بائیں جانب لکھی جاتی ہے۔ یوں \عددی{+5_{10}} کو  چار  ثنائی ہندسوں سے ظاہر کرتے ہوئے، بایاں   ہندسہ مثبت علامت \عددی{(+)} کو جبکہ باقی تین   ہندسے \عددی{5} کو  ظاہر کریں  گے۔ اسی طرح \عددی{-5_{10}} کو آٹھ ثنائی   ہندسوں سے ظاہر کرتے ہوئے، بایاں ہندسہ منفی علامت \عددی{(-)} کو جبکہ باقی سات ہندسے  \عددی{5} کو ظاہر کریں گے۔  
\begin{align*}
\underbrace{0}_{+}\underbrace{1\,\,\,0\,\,\,1}_{\,\,5_{10}}\quad \quad \quad \underbrace{1}_{-}\underbrace{0\,\,\,0\,\,\,0\,\,\,0\,\,\,1\,\,\,0\,\,\,1}_{5_{10}}
\end{align*}


ایک دلچسپ  حقیقت پر غور کریں۔اگر ہم  \عددی{1101_2} میں بایاں  ہندسہ  علامت  تصور  کریں  تب یہ  \عددی{-5_{10}} کو ظاہر کرے گا، لیکن اگر ہم چاروں ہندسوں کو ایک عدد تصور کریں تب یہ \عددی{D_{16}} یا \عددی{13_{10}}   کو ظاہر کرتا ہے۔

یہ جاننا    ضروری ہے،  آیا ثنائی اعداد کا  بایاں   ہندسہ علامت کو ظاہر کرتا ہے یا یہ عدد کا حصہ ہے؛  یہ  فیصلہ  اعداد  استعمال کرنے والے پر  ہے۔کمپیوٹر استعمال کرتے وقت آپ  فیصلہ کرتے ہیں  کہ علامت دار یا بے علامت (غیر علامت دار)
 اعداد استعمال کریں گے۔ جدول  \حوالہ{جدول_حساب_علامت_دار_چار}  میں چار ثنائی ہندسوں پر مشتمل علامت دار   اعداد دکھائے گئے ہیں۔  آپ دیکھ سکتے ہیں کہ صفر کو دو مختلف طریقوں سے ظاہر کیا جا سکتا ہے، ان میں ایک مثبت اور دوسرا منفی ہے!
\begin{table}
\caption{چار ہندسوں کے علامت دار اعداد}
\label{جدول_حساب_علامت_دار_چار}
\centering
\begin{tabular}{CC}
\toprule
\text{ثنائی} & \text{علامت دار}\\
\midrule
0111_2 & +7_{10}\\
0110_2&+6_{10}\\
0101_2&+5_{10}\\
0100_2&+4_{10}\\[0.5em]
0011_2&+3_{10}\\
0010_2&+2_{10}\\
0001_2&+1_{10}\\
0000_2&+0_{10}\\[0.5em]
1000_2&-0_{10}\\
1001_2&-1_{10}\\
1010_2&-2_{10}\\
1011_2&-3_{10}\\[0.5em]
1100_2&-4_{10}\\
1101_2&-5_{10}\\
1110_2&-6_{10}\\
1111_2&-7_{10}\\
\bottomrule
\end{tabular}
\end{table}

اس جدول  میں  چار ثنائی ہندسوں سے اعداد لکھے گئے؛ کمپیوٹر میں اعداد، عموماً،  ایک بائٹ   استعمال کرتے ہوئے  لکھا جاتا ہے۔  ایک بائٹ   \عددی{8} ثنائی   ہندسوں کو کہتے ہیں۔   علامت دار  اعداد کو  بائٹ میں لکھتے  ہوئے،  دائیں سات  ہندسے  عدد کی قدر  جبکہ بایاں  آخری  ہندسہ اس کی علامت  ظاہر کرے گا۔
\begin{align*}
00000101_2&=+5_{10}\\
01111111_2&=+127_{10}\\
10000101_2&=-5_{10}\\
11111111_2&=-127_{10}\\
00000000_2&=+0_{10}\\
10000000_2&=-0_{10}
\end{align*}
ان اعداد میں بھی مثبت اور منفی صفر پایا گیا؛ روز مرہ  زندگی میں صفر کو ہم   مثبت  تصور کرتے ہیں۔

اتنا کچھ کہنے کے بعد آپ کو بتاتا چلوں کہ،  کمپیوٹر میں منفی  اعداد کو  علامت دار قدر اظہار  میں نہیں بلکہ علامت دار و  \عددی{1} کے  تکملہ   یا علامت دار و  \عددی{2} کے تکملہ  نظام  میں رکھا اور استعمال کیا جاتا ہے۔اگلے حصہ میں ان  نظام  پر غور ہو گا۔

\حصہ{علامت دار و  تکملہ نظام}
کمپیوٹر میں عددی  برقیات  کی  مدد سے اعداد جمع یا منفی کیے جاتے ہیں۔  یہ اعمال اساسی تکملہ یا اساس منفی ایک  کا تکملہ (حصہ \حوالہ{حصہ_حساب_منفی_بذریعہ_اساسی_تکملہ} اور  حصہ \حوالہ{حصہ_حساب_منفی_بذریعہ_اساس_منفی_ایک_تکملہ} دیکھیں)  استعمال کرتے ہوئے زیادہ خوش اسلوبی سے سرانجام دیے جاتے ہیں۔

 کمپیوٹر  چونکہ  ثنائی اعداد استعمال   کرتا ہے، لہٰذا اس میں   منفی اعداد     \عددی{1} کے تکملہ  یا \عددی{2} کے تکملہ     میں لکھے جاتے ہیں۔جدول \حوالہ{جدول_حساب_علامت_دار_تکملہ_ایک_دو}  میں    چار  ثنائی ہندسی (چار بِٹ)   علامت دار  اعداد کا    \عددی{1}  کا تکملہ  اور  \عددی{2} کا تکملہ   روپ پیش کیا گیا ہے۔
\begin{table}
\caption{علامت دار  ایک کا تکملہ  اور دو کا  تکملہ  اعداد}
\label{جدول_حساب_علامت_دار_تکملہ_ایک_دو}
\centering
\begin{tabular}{CCCC}
\toprule
\text{اعشاری عدد} & \text{علامت دار قدر} & \text{علامت دار ایک کا  تکملہ} & \text{علامت دار دو کا  تکملہ}\\
\midrule
+7&0111&0111&0111\\
+6&0110&0110&0110\\
+5&0101&0101&0101\\
+4&0100&0100&0100\\[0.5em]
+3&0011&0011&0011\\
+2&0010&0010&0010\\
+1&0001&0001&0001\\
+0&0000&0000&0000\\[0.5em]
-0&1000&1111&\text{نہیں پایا جاتا}\\
-1&1001&1110&1111\\
-2&1010&1101&1110\\
-3&1011&1100&1101\\[0.5em]
-4&1100&1011&1100\\
-5&1101&1010&1011\\
-6&1110&1001&1010\\
-7&1111&1000&1001\\
-8&\text{نہیں پایا جاتا}&\text{نہیں پایا جاتا}&1000\\
\bottomrule
\end{tabular}
\end{table}

جدول \حوالہ{جدول_حساب_علامت_دار_تکملہ_ایک_دو} سے آپ دیکھ سکتے ہیں کہ مثبت عدد،   ثنائی ہندسوں میں ایک ہی طریقہ سے لکھا جاتا ہے،  جبکہ منفی عدد  تین طریقوں سے لکھا   جا سکتا ہے۔یوں  تینوں  طریقوں میں مثبت عدد کو سادہ  ثنائی عدد لکھیں۔

	مثبت عدد \عددی{+x}   کی  علامت دار روپ  میں  علامتی  بِٹ     \عددی{0}  سے   \عددی{1}  کرنے سے   \عددی{-x}  کا علامت دار  روپ    حاصل ہو گا۔یوں \عددی{-5} کو  علامت دار روپ   میں لکھنے کی خاطر \عددی{+5}  کو علامت دار روپ  \عددی{0101_2}  میں لکھ کر علامتی بِٹ  \عددی{1} کرنے سے \عددی{-5} کی علامت دار روپ  \عددی{1101_2}  حاصل ہو گی۔
	
منفی عدد \عددی{-x}  کو  علامت دار ایک کے   تکملہ  روپ میں لکھنے کی خاطر \عددی{+x}  کو علامت دار  ثنائی عدد (یعنی سادہ ثنائی  روپ میں)   لکھ کر اس کا   \عددی{1} کا تکملہ  لیں۔یاد رہے کہ  \عددی{1} کا تکملہ  حاصل کرتے ہوئے  ثنائی عدد کے ہر ہندسہ  (بمع  علامتی بِٹ) کا متمم لینا ہو گا۔ یوں \عددی{-5} کو علامت دار ایک کے  تکملہ   روپ  میں لکھنے کی خاطر  \عددی{+5}  کو  \عددی{0101_2}   لکھ  کر متمم لیں   جو درکار روپ   \عددی{1010_2}  دے گا۔

منفی عدد \عددی{-x}  کو  علامت دار   دو کے تکملہ   روپ میں لکھنے کی خاطر \عددی{+x}  کو علامت دار  ثنائی عدد (یعنی سادہ ثنائی  روپ میں)   لکھ کر اس کا  \عددی{2} کا   تکملہ  لیں۔یاد رہے کہ  \عددی{2} کا تکملہ  حاصل کرتے ہوئے  ثنائی عدد کے ہر ہندسہ  (بمع  علامتی بِٹ) کا متمم لینا ہو گا۔ یوں \عددی{-5} کو علامت دار  دو کے تکملہ   روپ  میں لکھنے کی خاطر  \عددی{+5}  کو  \عددی{0101_2}   لکھ  کر دو کا   تکملہ  لیں   جو درکار روپ   \عددی{1011_2}   دے گا۔

 \باب{بوولین الجبرا}
بوولین الجبرا انگلستان کے ریاضی دان جارج بوولی کے نام سے جانا جاتا ہے، جنہوں نے اس الجبرا کو دریافت کیا۔بوولین الجبرا ذہنی سوچ یعنی منطق کو الجبرائی روپ میں لکھنے کی صلاحیت رکھتی ہے۔اس لئے حیرانی کی بات نہیں کہ کمپیوٹر اسی کو استعمال کرتا ہے۔

\حصہ{بوولین الجبرا کے بنیادی تصورات} \شناخت{حصہ_بوولین_الجبرا_بنیادی_تصورات}
عام الجبرا میں متغیرات استعمال کرتے ہوئے تصور کیا جاتا ہے کہ ان کی قیمت کچھ بھی ہو سکتی ہے۔مثلاً، تفاعل \عددی{z=f(x,y)}، جہاں \عددی{x} اور \عددی{y} آزاد متغیرات جبکہ \عددی{z} تابع متغیر ہے، میں متغیرات کی چند ممکنہ قیمتیں درج ذیل ہیں۔
\begin{center}
\begin{otherlanguage}{english}
\begin{tabular}{CC|C}
\toprule
x&y&z\\
\midrule
0&0&0\\
1&2&5\\
2&1&4\\
3&2&7\\
2&2&6\\
3&1&5\\
\bottomrule
\end{tabular}
\end{otherlanguage}
\end{center}
اس تفاعل جس کو ایک نا مکمل جدول کے روپ میں پیش کیا گیا ہے کا الجبرائی روپ درج ذیل ہے۔
 \begin{align*}
 z=x+2y
 \end{align*}
اس کے برعکس، بوولین الجبرا میں متغیرات کی صرف دو ممکنہ قیمتیں ہیں۔ ان دو قیمتوں کو عموماً \عددی{0} (صفر) اور \عددی{1} (ایک) سے ظاہر کیا جاتا ہے۔بوولین تفاعل کی چند مثالوں پر غور کرتے ہیں۔

\جزوحصہ{منطقی ضرب}
تصور کریں \عددی{X} اور \عددی{Y} آزاد بوولین متغیرات ہیں، جبکہ \عددی{Z} ان کا تابع بوولین متغیر \عددی{Z=f(X,Y)} ہے۔ چونکہ \عددی{X} بوولین متغیر ہے، لہٰذا اس کی ممکنہ قیمتیں صرف \عددی{0} اور \عددی{1} ہیں۔اسی طرح \عددی{Y} بھی بوولین متغیر ہے، لہٰذا اس کی قیمت بھی صرف \عددی{0} اور \عددی{1} ہو سکتی ہے۔تابع متغیر \عددی{Z}بھی بوولین متغیر ہے۔اس طرح اگرچہ اس کی قیمت \عددی{X} اور \عددی{Y} کی تابع ہے، اس کے باوجود \عددی{Z} کی قیمت صرف \عددی{0} یا \عددی{1} ہی ہو سکتا ہے۔ متغیرات \عددی{X} اور \عددی{Y} درج ذیل چار ممکنہ ترتیب میں پائے جا سکتے ہیں۔
\begin{center}
\begin{otherlanguage}{english}
\begin{tabular}{CC}
\toprule
X&Y\\
\midrule
0&0\\
0&1\\
1&0\\
1&1\\
\bottomrule
\end{tabular}
\end{otherlanguage}
\end{center}
ان چار ممکنہ صورتوں میں \عددی{Z} کی قیمت \عددی{0} یا \عددی{1} ہوگی۔

آئیں، جدول \حوالہ{جدول_بوولین_جمع} میں پیش کیے گئے منطقی تفاعل پر غور کرتے ہیں جس کی تمام ممکنہ قیمتیں اس جدول میں دی گئی ہیں۔
\begin{table}
\centering
\begin{otherlanguage}{english}
\begin{tabular}{CC|C}
\toprule
X&Y&Z\\
\midrule
0&0&0\\
0&1&0\\
1&0&0\\
1&1&1\\
\bottomrule
\end{tabular}
\end{otherlanguage}
\caption{دو متغیر منطقی ضرب}
\label{جدول_بوولین_جمع}
\end{table}
اس مثال میں تابع متغیر \عددی{Z} کی قیمت صرف اس وقت \عددی{1} ہے جب \عددی{X} اور \عددی{Y} دونوں کی قیمت \عددی{1} ہے۔یہی قیمتیں \عددی{X} اور \عددی{Y} کی سادہ ضرب \عددی{X\cdot Y} سے بھی حاصل ہوتی ہیں (ذیل دیکھیں)۔
\begin{align*}
0\cdot 0&=0\\
0\cdot 1&=0\\
1\cdot 0&=0\\
1\cdot 1&=1
\end{align*}
اسی کی بنا پر جدول \حوالہ{جدول_بوولین_جمع} میں پیش تفاعل (اور عمل) کو بوولین ضرب یا منطقی ضرب کہتے ہیں۔ بوولین ضرب کو آزاد متغیرات کے درمیان نقطہ \قول{\عددی{\cdot}} سے یا آزاد متغیرات کو قریب قریب لکھنے سے ظاہر کیا جاتا ہے۔یوں بوولین ضرب درج ذیل لکھا جائے گا۔
\begin{gather}
\begin{aligned}
Z&=X\cdot Y\\
Z&=XY&\text{\RL{\small{(بوولین ضرب)}}}
\end{aligned}
\end{gather}
منطقی ضرب کے تصور کو وسعت دے کر متعدد آزاد متغیرات کے لئے بیان کیا جا سکتا ہے۔ منطقی ضرب کی عمومی تعریف پیش کرتے ہیں۔

\ابتدا{تعریف}
منطقی ضرب اس صورت \عددی{1} دیگا جب تمام آزاد متغیرات کی قیمت \عددی{1} ہو۔
\انتہا{تعریف}

 جدول \حوالہ{شکل_بوولین_ضرب_دو_متغیر_گیٹ} کو مثال بناتے ہیں۔ اس طرح کے جدول میں آزاد متغیرات کی تمام ممکنات لکھنے (یعنی آزاد متغیرات کے خانے پر کرنے) کی خاطر مداخل \عددی{XY} کو ثنائی عدد کے ہندسے تصور کر کے، جدول کے مطلوبہ خانوں میں صفر \عددی{(00)} تا تین \عددی{(11)} گنتی لکھیں۔یوں پہلے صف میں \عددی{XY} کی جگہ \عددی{00}، دوسری صف میں \عددی{01}، تیسرے میں \عددی{10} اور آخری میں \عددی{11} لکھا جائے گا۔
 
تین آزاد متغیرات کے منطقی ضرب تفاعل \عددی{Z=ABC} کو جدول \حوالہ{جدول_بوولین_تین_متغیر_بوولین_ضرب} میں پیش کیا گیا ہے۔آپ دیکھ سکتے ہیں کہ جدول کے تین مداخل کے خانوں میں صفر \عددی{(000)} تا سات \عددی{(111)} گنتی لکھی گئی ہے (جو تین ہندسوں کے ثنائی اعداد ہیں )۔
\begin{table}
\centering
\begin{otherlanguage}{english}
\begin{tabular}{CCC|C}
\toprule
A&B&C&Z\\
\midrule
0&0&0&0\\
0&0&1&0\\
0&1&0&0\\
0&1&1&0\\
1&0&0&0\\
1&0&1&0\\
1&1&0&0\\
1&1&1&1\\
\bottomrule
\end{tabular}
\end{otherlanguage}
\caption{تین متغیر بوولین ضرب}
\label{جدول_بوولین_تین_متغیر_بوولین_ضرب}
\end{table}

\جزوحصہ{منطقی جمع}
دو آزاد متغیرات کے بوولین تفاعل کی ایک اور مثال لیتے ہیں جس کو جدول \حوالہ{جدول_بوولین_جمع_منطقی} میں پیش کیا گیا ہے۔
\begin{table}
\centering
\begin{minipage}[t]{0.45\textwidth}
\centering
\begin{otherlanguage}{english}
\begin{tabular}{CC|C}
\toprule
X&Y&Z\\
\midrule
0&0&0\\
0&1&1\\
1&0&1\\
1&1&1\\
\bottomrule
\end{tabular}
\end{otherlanguage}
\caption{دو متغیر منطقی جمع}
\label{جدول_بوولین_جمع_منطقی}
\end{minipage}\hfill
\begin{minipage}[t]{0.45\textwidth}
\centering
\begin{otherlanguage}{english}
\begin{tabular}{CC|C}
\toprule
X&Y&S\\
\midrule
0&0&0\\
0&1&1\\
1&0&1\\
1&1&2\\
\bottomrule
\end{tabular}
\end{otherlanguage}
\caption{دو ثنائی اعداد کا سادہ مجموعہ}
\label{جدول_بوولین_جمع_سادہ}
\end{minipage}
\end{table}
اب \عددی{Z} اس صورت \عددی{1} کے برابر ہے جب \عددی{X} یا \عددی{Y} یا دونوں کی قیمت \عددی{1} ہو۔اس بوولین عمل کو بوولین جمع یا منطقی جمع کہتے ہیں۔

آزاد متغیرات \عددی{X} اور \عددی{Y} کا (روز مرہ) سادہ الجبرائی مجموعہ \عددی{S=X+Y} جدول \حوالہ{جدول_بوولین_جمع_سادہ} میں پیش کیا گیا ہے۔

جدول \حوالہ{جدول_بوولین_جمع_منطقی} اور جدول \حوالہ{جدول_بوولین_جمع_سادہ} کے اولین تین نتائج ایک جیسے ہیں۔اس مشابہت کی بنا جدول \حوالہ{جدول_بوولین_جمع_منطقی} میں دیے گئے بوولین تفاعل کو بوولین جمع یا منطقی جمع کہتے ہیں اور اس بوولین تفاعل کو جمع کے نشان \قول{\عددی{+}} سے ہی ظاہر کیا جاتا ہے۔یوں جدول \حوالہ{جدول_بوولین_جمع_منطقی}میں پیش بوولین جمع تفاعل درج ذیل لکھا جائے گا۔
\begin{align}
Z&=X+Y&\text{\RL{\small{(بوولین جمع)}}}
\end{align}
یہ بوولین تفاعل کی مساوات ہے جس کو عام الجبرائی جمع ہرگز نہ سمجھا جائے۔بالخصوص، بوولین جمع کرتے وقت یاد رہے کہ \عددی{1+1=1}ہے۔

بوولین جمع کے تصور کو وسعت دے کر متعدد آزاد متغیرات کے لئے بیان کیا جا سکتا ہے۔ بوولین جمع کی عمومی تعریف درج ذیل ہے۔

\ابتدا{تعریف}
منطقی جمع اس صورت \عددی{1} دیگا جب آزاد متغیرات میں کم سے کم ایک متغیر کی قیمت \عددی{1} ہو۔
\انتہا{تعریف}

تین متغیر منطقی جمع تفاعل \عددی{Z=A+B+C} جدول \حوالہ{جدول_بوولین_تین_متغیر_جمع} میں پیش کیا گیا ہے۔
\begin{table}
\centering
\begin{minipage}[b]{0.45\textwidth}
\centering
\begin{otherlanguage}{english}
\begin{tabular}{CCC|C}
\toprule
A&B&C&Z\\
\midrule
0&0&0&0\\
0&0&1&1\\
0&1&0&1\\
0&1&1&1\\
1&0&0&1\\
1&0&1&1\\
1&1&0&1\\
1&1&1&1\\
\bottomrule
\end{tabular}
\end{otherlanguage}
\caption{تین متغیر منطقی جمع}
\label{جدول_بوولین_تین_متغیر_جمع}
\end{minipage}\hfill
\begin{minipage}[b]{0.45\textwidth}
\centering
\begin{otherlanguage}{english}
\begin{tabular}{C|C}
\toprule
X&Z\\
\midrule
0&1\\
1&0\\
\bottomrule
\end{tabular}
\end{otherlanguage}
\caption{منطقی نفی یا متمم}
\label{جدول_بوولین_نفی}
\end{minipage}
\end{table}
یاد رہے کہ تین آزاد متغیرات کے منطقی جمع کا الجبرائی جمع کے ساتھ کوئی تعلق نہیں۔یہاں جمع کی علامت بوولین جمع کو ظاہر کرتی ہے لہٰذا یہاں \عددی{1+1+1=1} ہو گا۔


\جزوحصہ{منطقی نفی}
بوولین تفاعل \عددی{ Z=f(X)} کی تیسری مثال لیتے ہیں جہاں آزاد متغیر \عددی{X} اور تابع متغیر \عددی{Z} کا تعلق جدول \حوالہ{جدول_بوولین_نفی} میں پیش کیا گیا ہے۔

 اس تفاعل کو بوولین نفی کہتے ہیں۔آپ دیکھ سکتے ہیں کہ درحقیقت، تابع متغیر \عددی{Z}، آزاد متغیر کا متمم ہے۔یوں بوولین نفی درج ذیل لکھا جا سکتا ہے۔

\begin{align}
Z&=\overline{X}&\text{\RL{\small{(بوولین نفی یا متمم)}}}
\end{align}
بوولین نفی صرف ایک آزاد متغیر کے لئے بیان کیا جا سکتا ہے، اور اس کی تعریف درج ذیل ہے۔

\ابتدا{تعریف}
بوولین نفی آزاد متغیر کا متمم دیتا ہے۔
\انتہا{تعریف}

\جزوحصہ{منطقی بلا شرکت جمع}
دو آزاد متغیرات کا ایسا بوولین تفاعل جدول \حوالہ{جدول_بوولین_دو_بلا_شرکت} میں دکھایا گیا ہے، جس کا تابع متغیر اس صورت \عددی{1} ہے جب صرف ایک آزاد متغیر \عددی{1} ہو۔ یہ دو متغیر بوولین بلا شرکت جمع ہے۔
\begin{table}
\centering
\begin{minipage}[b]{0.45\textwidth}
\centering
\begin{otherlanguage}{english}
\begin{tabular}{CC|C}
\toprule
A&B&Z\\
\midrule
0&0&0\\
0&1&1\\
1&0&1\\
1&1&0\\
\bottomrule
\end{tabular}
\end{otherlanguage}
\caption{دو متغیر منطقی بلا شرکت جمع}
\label{جدول_بوولین_دو_بلا_شرکت}
\end{minipage}\hfill
\begin{minipage}[b]{0.45\textwidth}
\centering
\begin{otherlanguage}{english}
\begin{tabular}{CCC|C}
\toprule
A&B&C&Z\\
\midrule
0&0&0&0\\
0&0&1&1\\
0&1&0&1\\
0&1&1&0\\
1&0&0&1\\
1&0&1&0\\
1&1&0&0\\
1&1&1&1\\
\bottomrule
\end{tabular}
\end{otherlanguage}
\caption{تین متغیر بوولین بلا شرکت جمع}
\label{جدول_بوولین_تین_متغیر_بلا_شرکت}
\end{minipage}
\end{table}
اس تصور کو متعدد آزاد متغیرات تک وسعت دے کر بیان کرتے ہیں۔

\ابتدا{تعریف}
طاق تعداد کے آزاد متغیرات \عددی{1} ہو نے کی صورت میں بوولین بلا شرکت کا تابع متغیر \عددی{1} ہو گا۔
\انتہا{تعریف}

تین آزاد متغیر بلا شرکت جمع تفاعل کو جدول \حوالہ{جدول_بوولین_تین_متغیر_بلا_شرکت} میں پیش کیا گیا ہے۔



دو اور تین آزاد متغیر بوولین بلا شرکت کی مساوات درج ذیل ہوں گی۔
\begin{gather}
\begin{aligned}
Z&=A\oplus B&\text{\RL{\small{(دو آزاد متغیر بلا شرکت جمع)}}}\\
Z&=A\oplus B\oplus C&\text{\RL{\small{(تین آزاد متغیر بلا شرکت جمع)}}}
\end{aligned}
\end{gather}

\جزوحصہ{منطقی ضد بلا شرکت جمع}
 بوولین بلا شرکت جمع تفاعل کا نفی (یعنی متمم) لینے سے بوولین ضد بلا شرکت جمع حاصل ہو گا، جو دو اور تین آزاد متغیرات کے لئے درج ذیل لکھا جاتا ہے۔
\begin{gather}
\begin{aligned}
Z&=\overline{A\oplus B}\\
Z&=\overline{A\oplus B\oplus C}&\text{\RL{\small{(تین متغیر منطقی ضد بلا شرکت جمع)}}}
\end{aligned}
\end{gather}
 جدول \حوالہ{جدول_بوولین_دو_بلا_شرکت} اور جدول \حوالہ{جدول_بوولین_تین_متغیر_بلا_شرکت} میں تابع متغیر نفی کرنے سے بالترتیب دو اور تین بوولین ضد بلا شرکت تفاعل حاصل ہوں گے جنہیں جدول \حوالہ{جدول_بوولین_دو_متمم_بلا_شرکت} اور جدول \حوالہ{جدول_بوولین_تین_متغیرمتمم_بلا_شرکت} میں پیش کیا گیا ہے۔
\begin{table}
\centering
\begin{minipage}[b]{0.45\textwidth}
\centering
\begin{otherlanguage}{english}
\begin{tabular}{CC|C}
\toprule
A&B&Z\\
\midrule
0&0&0\\
0&1&1\\
1&0&1\\
1&1&0\\
\bottomrule
\end{tabular}
\end{otherlanguage}
\caption{دو متغیر منطقی ضد بلا شرکت جمع}
\label{جدول_بوولین_دو_متمم_بلا_شرکت}
\end{minipage}\hfill
\begin{minipage}[b]{0.45\textwidth}
\centering
\begin{otherlanguage}{english}
\begin{tabular}{CCC|C}
\toprule
A&B&C&Z\\
\midrule
0&0&0&0\\
0&0&1&1\\
0&1&0&1\\
0&1&1&0\\
1&0&0&1\\
1&0&1&0\\
1&1&0&0\\
1&1&1&1\\
\bottomrule
\end{tabular}
\end{otherlanguage}
\caption{تین متغیر بوولین ضد بلا شرکت جمع}
\label{جدول_بوولین_تین_متغیرمتمم_بلا_شرکت}
\end{minipage}
\end{table}
\حصہ{برقی تاروں میں جوڑ کی وضاحت}
 شکل \حوالہ{شکل_بوولین_برقی_تار_جوڑ} پر غور کریں جس میں برقی تاروں کے بیچ جوڑ کی وضاحت کی گئی ہے۔
 
 جہاں ایک تار دوسری تار کے اوپر سے گزرتی ہو اور دونوں آپس میں جڑی ہوں، وہاں جوڑ کے مقام پر نقطے کا نشان لگایا جاتا ہے۔ایسی صورت میں انہیں ایک تار تصور کیا جائے۔
 
 جہاں تاریں آپس میں جڑی نہ ہوں وہاں انہیں بغیر نقطے کے نشان سے ایک دوسری کے اوپر سے گزرتا دکھایا جاتا ہے۔ نقطہ کے نشان کی غیر موجودگی میں ان تاروں کو دو علیحدہ اور بلا جوڑ تاریں سمجھا جائے۔

 تیسری صورت بھی شکل میں دکھائی گئی ہے جہاں غلط فہمی کا امکان نہیں پایا جاتا۔اس میں ایک تار کا سر دوسری تار پر ختم ہو تا ہے۔ ایسی صورت میں انہیں ایک تار تصور کیا جائے (یعنی یہ دونوں آپس میں جڑی ہیں) ۔
\begin{figure}
\centering
\begin{tikzpicture}
\draw(-1,0)--(1.5,0);
\draw(0,0)--++(0,-0.75)--++(1,0);
\draw(0,-1)node[above left]{\text{\RL{جوڑ}}};
\end{tikzpicture}\quad\quad
\begin{tikzpicture}
\draw(-1,0)--(1.5,0);
\draw(0,1)--(0,-1);
\draw(0,-1)node[above right]{\text{\RL{بلا جوڑ}}};
\end{tikzpicture}\quad \quad 
\begin{tikzpicture}
\draw(-1,0)--(1.5,0);
\draw(0,1)--(0,0)node[circ]{}--(0,-1);
\draw(0,-1)node[above right]{\text{\RL{جوڑ}}};
\end{tikzpicture}
\caption{تاروں کے بیچ برقی جوڑ۔}
\label{شکل_بوولین_برقی_تار_جوڑ}
\end{figure}

\حصہ{عددی گیٹ}
 بوولین الجبرا کے تین اہم ترین تفاعل پر حصہ \حوالہ{حصہ_بوولین_الجبرا_بنیادی_تصورات} میں غور کیا گیا۔یہ تفاعلات عددی برقیات میں کلیدی کردار ادا کرتے ہیں، جہاں انہیں عددی ادوار کی مدد سے جامہ پہنایا جاتا ہے۔یہ مخصوص عددی ادوار، عددی گیٹ کہلاتے ہیں۔
 
\جزوحصہ{ضرب گیٹ}
منطقی (بوولین) ضرب تفاعل کو ضرب گیٹ سے عملی جامع پہنایا جاتا ہے، جو شکل \حوالہ{شکل_بوولین_ضرب_گیٹ} میں دکھایا گیا ہے۔ آزاد متغیرات، \عددی{X} اور \عددی{Y}، ضرب گیٹ کی بائیں جانب ہیں جبکہ تابع متغیر، \عددی{Z}، دائیں جانب ہے۔ آزاد متغیرات کو مداخل جبکہ تابع متغیر کو مخارج کہتے ہیں۔دو متغیر ضرب گیٹ (دو مداخل ضرب گیٹ) کے دو مداخل اور ایک مخارج ہو گا۔ یہ گیٹ، ضرب تفاعل کے جدول کو مطمئن کرتا ہے۔
	
\begin{figure}
\centering
\begin{minipage}[b]{0.40\textwidth}
\centering
\begin{tikzpicture}
\draw(0,0)node[and port ,scale=1, number inputs=2](u1){};
\draw(u1.in 1)node[left]{$X$}node[xshift=-0.25cm, yshift=0.75cm]{مداخل};
\draw(u1.in 2)node[left]{$Y$};
\draw(u1.out)node[right]{$Z$}node[xshift=0.25cm,yshift=1cm]{مخارج};
\end{tikzpicture}
\caption{دو مداخل ضرب گیٹ۔}
\label{شکل_بوولین_ضرب_گیٹ}
\end{minipage}\hfill
\begin{minipage}[b]{0.55\textwidth}
\centering
\begin{tikzpicture}
\pgfmathsetmacro{\kts}{0.5}
\pgfmathsetmacro{\kys}{0.5}
\pgfmathsetmacro{\kysep}{0.75}
\draw(0,0)node[above left]{$X$}--++(\kts,0)--++(\kts,0)--++(\kts,0)--++(0,\kys)--++(4*\kts,0)--++(0,-\kys)--++(3*\kts,0)--++(0,\kys)--++(2*\kts,0);
\draw(-\kts/2,0)
\foreach \n in {0,0,0,1,1,1,1,0,0,0,1,1}{++(\kts,0)node[above]{\n}};
\draw(0,-\kysep)node[above left]{$Y$}--++(\kts,0)--++(0,\kys)--++(\kts,0)--++(0,-\kys)--++(2*\kts,0)--++(0,\kys)--++(1*\kts,0)--++(0,-\kys)--++(1*\kts,0)--++(0,\kys)--++(2*\kts,0)--++(0,-\kys)--++(1*\kts,0)--++(0,\kys)--++(2*\kts,0)--++(0,-\kys)--++(1*\kts,0);
\draw(-\kts/2,-\kysep)
\foreach \n in {0,1,0,0,1,0,1,1,0,1,1,0}{++(\kts,0)node[above]{\n}};
\draw(0,-2*\kysep)node[above left]{$Z$}--++(4*\kts,0)--++(0,\kys)--++(\kts,0)--++(0,-\kys)--++(1*\kts,0)--++(0,\kys)--++(1*\kts,0)--++(0,-\kys)--++(3*\kts,0)--++(0,\kys)--++(1*\kts,0)--++(0,-\kys)--++(1*\kts,0);
\draw(-\kts/2,-2*\kysep)
\foreach \n in {0,0,0,0,1,0,1,0,0,0,1,0}{++(\kts,0)node[above]{\n}};
\end{tikzpicture}
\caption{ضرب گیٹ کی کارکردگی۔}
\label{شکل_بوولین_ضرب_جدول_دو_دخول}
\end{minipage}
\end{figure}

شکل \حوالہ{شکل_بوولین_ضرب_جدول_دو_دخول} میں دو مداخل ضرب گیٹ کی کارکردگی ترسیم کی گئی ہے، جہاں \عددی{0} کو پست اور \عددی{1} کو بلند لکیر سے ظاہر کیا گیا ہے۔آپ دیکھ سکتے ہیں کہ مخارج صرف اور صرف اُس صورت بلند ہوتا ہے جب ضرب گیٹ کے تمام مداخل بلند ہوں۔ ہم \عددی{0} کو پست اور \عددی{1} کو بلند بھی پکارتے ہیں۔ اس شکل میں مداخل کو کسی خاص ترتیب سے تبدیل نہیں کیا گیا۔


ضرب گیٹ کو شکل \حوالہ{شکل_بوولین_گیٹ_بطور_سوئچ} میں بطور عددی گیٹ یا عددی سوئچ دکھایا گیا ہے جہاں ایک داخلی پنیا کو قابو پنیا کا نام دیا گیا ہے جبکہ دوسرے کو (اب بھی) مداخل کہا گیا ہے۔ضرب گیٹ کے جدول سے واضح ہے کہ جب تک قابو پنیا \عددی{0} ہو، خارجی پنیا \عددی{0} رہتا ہے۔اس صورت میں مداخل پر موجود مواد، خارجی پنیا تک نہیں پہنچ سکتا، یعنی اس پر \عددی{0} یا \عددی{1} کا مخارج پر کوئی اثر نہیں ہوتا؛ ہم کہتے ہیں قابو پنیا نے ضرب گیٹ کو معذور کر دیا ۔اس کے برعکس اگر قابو پنیا \عددی{1} ہو تب خارجی پنیا پر وہی کچھ ہوگا جو مداخل پر ہو گا؛ ہم کہتے ہیں ضرب گیٹ مجاز کر دیا گیا ہے۔قابو پنیا پر ایک یا صفر سے داخلی اشارہ (مواد) کو خارجی پنیا تک پہنچنا، ممکن یا ناممکن بنایا جا سکتا ہے۔یوں یہ ایک دروازے کی طرح کام کرتا ہے، جس کی بنا پر یہ گیٹ کہلاتا ہے۔قابو پنیا کو، معذور اور مجاز بنانے والا پنیا بھی کہتے ہیں۔شکل \حوالہ{شکل_بوولین_گیٹ_کارکردگی} میں ضرب گیٹ کی کارکردگی دکھائی گئی ہے۔آپ دیکھ سکتے ہیں کہ صرف مجاز صورت میں مواد مخارج تک پہنچ پاتا ہے؛ معذور صورت میں مخارج ہمیشہ پست رہے گا۔

\begin{figure}
\centering
\begin{tikzpicture}
\draw(0,-2)node[above right]{مداخل}--++(2,0)--++(0.5,0.5)coordinate[pos=0.5](kk){}++(0,-0.5)--++(1.5,0)node[above left]{مخارج};
\draw[dashed](kk)--++(0,-1)--++(-1,0)node[left]{قابو};
\end{tikzpicture}\hfill
\begin{tikzpicture}
\draw(0,0)node[and port ,scale=1, number inputs=2](u1){};
\draw(0,0)node[and port](and1){};
\draw(u1.in 1)node[left]{مداخل};
\draw(u1.in 2)--++(0,-0.5)node[below]{قابو};
\draw(u1.out)node[right]{مخارج};
\end{tikzpicture}
\caption{ضرب گیٹ بطور سوئچ یا ایک بِٹ گیٹ۔}
\label{شکل_بوولین_گیٹ_بطور_سوئچ}
\end{figure}
%
\begin{figure}
\centering
\begin{tikzpicture}
\pgfmathsetmacro{\kts}{0.5}
\pgfmathsetmacro{\kys}{0.5}
\pgfmathsetmacro{\kysep}{0.75}
\draw(0,0)node[above left]{قابو}--++(3*\kts,0)--++(0,\kys)--++(4*\kts,0)--++(0,-\kys)--++(5*\kts,0);
\draw(-\kts/2,0)
\foreach \n in {0,0,0,1,1,1,1,0,0,0,0,0}{++(\kts,0)node[above]{\n}};
\draw(0,-\kysep)node[above left]{مداخل}--++(\kts,0)--++(0,\kys)--++(\kts,0)--++(0,-\kys)--++(2*\kts,0)--++(0,\kys)--++(1*\kts,0)--++(0,-\kys)--++(1*\kts,0)--++(0,\kys)--++(2*\kts,0)--++(0,-\kys)--++(1*\kts,0)--++(0,\kys)--++(2*\kts,0)--++(0,-\kys)--++(1*\kts,0);
\draw(-\kts/2,-\kysep)
\foreach \n in {0,1,0,0,1,0,1,1,0,1,1,0}{++(\kts,0)node[above]{\n}};
\draw(0,-2*\kysep)node[above left]{مخارج}--++(4*\kts,0)--++(0,\kys)--++(\kts,0)--++(0,-\kys)--++(1*\kts,0)--++(0,\kys)--++(1*\kts,0)--++(0,-\kys)--++(5*\kts,0);
\draw(-\kts/2,-2*\kysep)
\foreach \n in {0,0,0,0,1,0,1,0,0,0,0,0}{++(\kts,0)node[above]{\n}};
\draw(1.5*\kts,0.75)node[]{معذور};
\draw(5*\kts,0.75)node[]{مجاز};
\draw(9.5*\kts,0.75)node[]{معذور};
\end{tikzpicture}
\caption{ضرب گیٹ کی کارکردگی۔}
\label{شکل_بوولین_گیٹ_کارکردگی}
\end{figure}


\جزوحصہ{جمع گیٹ}
منطقی جمع (بوولین جمع) تفاعل کو جمع گیٹ سے عملی جامع پہنایا جاتا ہے۔ دو مداخل جمع گیٹ شکل \حوالہ{شکل_بوولین_ضرب_دو_متغیر_گیٹ} میں دکھایا گیا ہے۔ یہ گیٹ، جمع تفاعل کے جدول کو مطمئن کرتا ہے۔

\begin{figure}
\centering
\begin{minipage}[b]{0.40\textwidth}
\centering
\begin{tikzpicture}
\draw(0,0)node[or port ,scale=1, number inputs=2](u1){};
\draw(u1.in 1)node[left]{$X$}node[xshift=-0.25cm, yshift=0.75cm]{مداخل};
\draw(u1.in 2)node[left]{$Y$};
\draw(u1.out)node[right]{$Z$}node[xshift=0.25cm,yshift=1cm]{مخارج};
\end{tikzpicture}
\caption{دو مداخل جمع گیٹ۔}
\label{شکل_بوولین_ضرب_دو_متغیر_گیٹ}
\end{minipage}\hfill
\begin{minipage}[b]{0.55\textwidth}
\centering
\begin{tikzpicture}
\pgfmathsetmacro{\kts}{0.5}
\pgfmathsetmacro{\kys}{0.5}
\pgfmathsetmacro{\kysep}{0.75}
\draw(0,0)node[above left]{$X$}--++(\kts,0)--++(\kts,0)--++(\kts,0)--++(0,\kys)--++(4*\kts,0)--++(0,-\kys)--++(3*\kts,0)--++(0,\kys)--++(2*\kts,0);
\draw(-\kts/2,0)
\foreach \n in {0,0,0,1,1,1,1,0,0,0,1,1}{++(\kts,0)node[above]{\n}};
\draw(0,-\kysep)node[above left]{$Y$}--++(\kts,0)--++(0,\kys)--++(\kts,0)--++(0,-\kys)--++(2*\kts,0)--++(0,\kys)--++(1*\kts,0)--++(0,-\kys)--++(1*\kts,0)--++(0,\kys)--++(2*\kts,0)--++(0,-\kys)--++(1*\kts,0)--++(0,\kys)--++(2*\kts,0)--++(0,-\kys)--++(1*\kts,0);
\draw(-\kts/2,-\kysep)
\foreach \n in {0,1,0,0,1,0,1,1,0,1,1,0}{++(\kts,0)node[above]{\n}};
\draw(0,-2*\kysep)node[above left]{$Z$}--++(1*\kts,0)--++(0,\kys)--++(\kts,0)--++(0,-\kys)--++(1*\kts,0)--++(0,\kys)--++(5*\kts,0)--++(0,-\kys)--++(1*\kts,0)--++(0,\kys)--++(2*\kts,0)--++(0,-\kys)--++(1*\kts,0);
\draw(-\kts/2,-2*\kysep)
\foreach \n in {0,1,0,1,1,1,1,1,0,1,1,0}{++(\kts,0)node[above]{\n}};
\end{tikzpicture}
\caption{جمع گیٹ کی کارکردگی۔}
\label{شکل_بوولین_جمع_جدول_دو_دخول}
\end{minipage}
\end{figure}

جمع گیٹ کی کارکردگی شکل \حوالہ{شکل_بوولین_جمع_جدول_دو_دخول} میں ترسیم کی گئی ہے۔آپ دیکھ سکتے ہیں، جمع گیٹ کا مخارج اُس صورت بلند ہوگا جب کوئی مداخل بلند ہو۔

جمع گیٹ میں اگر ایک پنیا کو قابو پنیا سمجھا جائے تو پست قابو، گیٹ کو مجاز بنا کر ، داخلی مواد کو مخارج تک پہنچنے کی اجازت دیتا ہے، جبکہ بلند قابو کی صورت میں مخارج لازماً بلند رہتا ہے۔

\جزوحصہ{نفی گیٹ}
نفی تفاعل کو نفی گیٹ سے عملی جامع پہنایا جاتا ہے، جس کی علامت شکل \حوالہ{شکل_بوولین_نفی_گیٹ} میں دکھائی گئی ہے، اور جو مواد کو مخارج تک پہنچنے سے روک نہ پانے کے باوجود (نفی) \قول{ گیٹ} کہلاتا ہے۔ اس کی کارکردگی شکل \حوالہ{شکل_بوولین_نفی_کارکردگی} میں ترسیم کی گئی ہے۔آپ دیکھ سکتے ہیں، نفی گیٹ کا مخارج اس کے مداخل کا اُلٹ ہو گا۔ یہ گیٹ، نفی تفاعل کے جدول کو مطمئن کرتا ہے۔

\begin{figure}
\centering
\begin{minipage}[b]{0.4\textwidth}
\centering
\begin{tikzpicture}
\draw(0,0)node[not port ,scale=1, number inputs=1](u1){};
\draw(u1.in)node[left]{مداخل};
\draw(u1.out)node[right]{مخارج};
\end{tikzpicture}
\caption{نفی گیٹ}
\label{شکل_بوولین_نفی_گیٹ}
\end{minipage}\hfill
\begin{minipage}[b]{0.6\textwidth}
\centering
\begin{tikzpicture}
\pgfmathsetmacro{\kts}{0.5}
\pgfmathsetmacro{\kys}{0.5}
\pgfmathsetmacro{\kysep}{0.5}
\draw(0,0)node[above left]{مداخل}--++(2*\kts,0)--++(0,\kys)--++(1*\kts,0)--++(0,-\kys)--++(1*\kts,0)--++(0,\kys)--++(3*\kts,0)--++(0,-\kys)--++(2*\kys,0)--++(0,\kys)--++(\kts,0)--++(0,-\kys)--++(\kts,0)--++(0,\kys)--++(\kts,0);
\draw(-\kts/2,0)
\foreach \n in {0,0,1,0,1,1,1,0,0,1,0,1}{++(\kts,0)node[above]{\n}};
\draw(0,-\kysep)node[below left]{مخارج}--++(2*\kts,0)--++(0,-\kys)--++(1*\kts,0)--++(0,\kys)--++(1*\kts,0)--++(0,-\kys)--++(3*\kts,0)--++(0,\kys)--++(2*\kys,0)--++(0,-\kys)--++(\kts,0)--++(0,\kys)--++(\kts,0)--++(0,-\kys)--++(\kts,0);
\draw(-\kts/2,-\kysep-\kys)
\foreach \n in {1,1,0,1,0,0,0,1,1,0,1,0}{++(\kts,0)node[above]{\n}};
\end{tikzpicture}
\caption{نفی گیٹ کی کارکردگی۔}
\label{شکل_بوولین_نفی_کارکردگی}
\end{minipage}
\end{figure}

نفی تفاعل ایک آزاد اور ایک تابع متغیر رکھتا ہے، لہٰذا نفی گیٹ کا ایک مداخل اور ایک مخارج ہو گا۔

\جزوحصہ{متعدد مداخل گیٹ}
 ضرب گیٹ اور جمع گیٹ کے متعدد مداخل ہو سکتے ہیں (تاہم، ان کا مخارج ایک ہو گا)۔ شکل \حوالہ{شکل_بوولین_تین_ضرب_گیٹ} میں تین مداخل ضرب گیٹ اور جدول، اور شکل \حوالہ{شکل_بوولین_تین_جمع} میں تین مداخل جمع گیٹ اور جدول دکھائے گئے ہیں، جہاں \عددی{A}، \عددی{B}، اور \عددی{C} مداخل جبکہ \عددی{Z} مخارج ہے۔ ضرب گیٹ کا مخارج اس صورت بلند ہو گا جب تمام مداخل بلند ہوں، جبکہ جمع گیٹ کا مخارج اس صورت بلند ہو گا جب کوئی بھی مداخل بلند ہو۔
 
\begin{figure}
\centering
\begin{minipage}{0.45\textwidth}
\centering
\begin{tikzpicture}
\draw(0,0)node[and port ,scale=1, number inputs=3](u1){};
\end{tikzpicture}%
\begin{otherlanguage}{english}
\begin{tabular}{CCC|C}
\toprule
A&B&C&Z\\
\midrule
0&0&0&0\\
0&0&1&0\\
0&1&0&0\\
0&1&1&0\\
1&0&0&0\\
1&0&1&0\\
1&1&0&0\\
1&1&1&1\\
\bottomrule
\end{tabular}
\end{otherlanguage} 
\caption{تین مداخل ضرب گیٹ۔}
\label{شکل_بوولین_تین_ضرب_گیٹ}
\end{minipage}\hfill
\begin{minipage}{0.45\textwidth}
\centering
\begin{tikzpicture}%
\draw(0,0)node[or port ,scale=1, number inputs=3](u1){};
\end{tikzpicture}
\begin{otherlanguage}{english}
\begin{tabular}{CCC|C}
\toprule
A&B&C&Z\\
\midrule
0&0&0&0\\
0&0&1&1\\
0&1&0&1\\
0&1&1&1\\
1&0&0&1\\
1&0&1&1\\
1&1&0&1\\
1&1&1&1\\
\bottomrule
\end{tabular}
\end{otherlanguage}
\caption{تین مداخل جمع گیٹ۔}
\label{شکل_بوولین_تین_جمع}
\end{minipage}
\end{figure}


شکل \حوالہ{شکل_بوولین_دو_ضرب_سے_تین} میں دو ضرب گیٹ یوں جوڑے گئے ہیں کہ ایک کا مخارج دوسرے کے مداخل سے جڑا ہے۔ساتھ ہی اس دور کا بوولین جدول دیا گیا ہے۔ پہلے جدول استعمال کیے بغیر اس دور کو سمجھنے کی کوشش کرتے ہیں۔ مخارج \عددی{Z} اس صورت بلند ہو گا جب دائیں گیٹ کے مداخل \عددی{C} اور \عددی{D} دونوں بلند ہوں لیکن \عددی{D} بلند ہونے کے لئے ضروری ہے کہ بائیں گیٹ کے مداخل \عددی{A} اور \عددی{B} دونوں بلند ہوں۔ یوں \عددی{A}، \عددی{B} اور \عددی{C} بلند ہونے کی صورت میں مخارج \عددی{Z} بلند ہو گا؛ یہی تین مداخل ضرب گیٹ کی خاصیت ہے۔


آئیں اب جدول کو سمجھتے ہیں۔ تین مداخل \عددی{ABC} کے خانوں کو تین ہندسوں کے ثنائی اعداد \عددی{000} تا \عددی{111} سے پُر کریں۔ اس کے بعد بائیں ضرب گیٹ کے مخارج \عددی{D} کے خانے پُر کریں۔ یاد رہے کہ یہ صرف \عددی{A} اور \عددی{B} پر منحصر ہے اور صرف اس صورت بلند ہو گا جب یہ دونوں بلند ہوں، جو آخری دو صفوں میں ہو گا۔ اس کے بعد دائیں ضرب گیٹ کے مخارج \عددی{Z} کے خانے پُر کریں۔ یہ صرف \عددی{C} اور \عددی{D} پر منحصر ہے، اور بلند صرف اس صورت ہو گا جب یہ دونوں بلند ہوں۔

ان نتائج کا جدول \حوالہ{شکل_بوولین_تین_ضرب_گیٹ} میں پیش تین مداخل ضرب گیٹ کے جدول کے ساتھ کریں۔آپ دیکھ سکتے ہیں کہ شکل \حوالہ{شکل_بوولین_دو_ضرب_سے_تین} میں دونوں ضرب گیٹ مل کر تین مداخل ضرب گیٹ کا کردار ادا کرتے ہیں۔ یوں دو داخلی ضرب گیٹوں کی مدد سے زیادہ مداخل کا ضرب گیٹ حاصل کیا جا سکتا ہے۔

 شکل \حوالہ{شکل_بوولین_دو_جمع_سے_تین} میں دو مداخل جمع گیٹوں سے تین مداخل جمع گیٹ کا حصول دکھایا گیا ہے۔ یہاں \عددی{Z} صرف اس صورت پست ہو گا جب دائیں گیٹ کے دونوں مداخل، \عددی{C} اور \عددی{D}، پست ہوں لیکن \عددی{D} صرف اس صورت پست ہو سکتا ہے جب بائیں گیٹ کے مداخل، \عددی{A} اور \عددی{B}، پست ہوں۔ یوں \عددی{Z} صرف اس صورت پست ہو گا جب \عددی{A}، \عددی{B}، اور \عددی{C} پست ہوں، جو تین مداخل جمع گیٹ کی خاصیت ہے۔

\begin{figure}
\centering
\begin{otherlanguage}{english}
\begin{tikzpicture}
\draw(0,0)node[and port ,scale=1, number inputs=2](u1){};
\draw(2,-1)node[and port ,scale=1, number inputs=2](u2){};
\draw(u1.in 1)--++(-0.5,0)node[left]{A} (u1.in 2)--++(-0.5,0)node[left]{B} (u1.out)node[above]{D}-| (u2.in 1) (u2.in 2)--++(-2.5,0)node[left]{C} (u2.out)node[right]{Z};
\end{tikzpicture}\quad \quad 
\begin{tikzpicture}
\draw(0,0)node[]{
\begin{tabular}{CCC|C|C}
\toprule
A&B&C&D&Z\\
\midrule
0&0&0&0&0\\
0&0&1&0&0\\
0&1&0&0&0\\
0&1&1&0&0\\
1&0&0&0&0\\
1&0&1&0&0\\
1&1&0&1&0\\
1&1&1&1&1\\
\bottomrule
\end{tabular}
};
\end{tikzpicture}
\end{otherlanguage}
\caption{دو مداخل ضرب گیٹ سے تین مداخل ضرب گیٹ کا حصول۔}
\label{شکل_بوولین_دو_ضرب_سے_تین}
\end{figure}
%
\begin{figure}
\centering
\begin{otherlanguage}{english}
\begin{tikzpicture}
\draw(0,0)node[or port ,scale=1, number inputs=2](u1){}; 
\draw(2,-1)node[or port ,scale=1, number inputs=2](u2){};
\draw(u1.in 1)--++(-0.5,0)node[left]{A} (u1.in 2)--++(-0.5,0)node[left]{B} (u1.out)node[above]{D}-| (u2.in 1) (u2.in 2)--++(-2.5,0)node[left]{C} (u2.out)node[right]{Z};
\end{tikzpicture}\quad \quad %
\begin{tikzpicture}
\draw(0,0)node[]{
\begin{tabular}{CCC|C|C}
\toprule
A&B&C&D&Z\\
\midrule
0&0&0&0&0\\
0&0&1&0&1\\
0&1&0&1&1\\
0&1&1&1&1\\
1&0&0&1&1\\
1&0&1&1&1\\
1&1&0&1&1\\
1&1&1&1&1\\
\bottomrule
\end{tabular}
};
\end{tikzpicture}
\end{otherlanguage}
\caption{دو مداخل جمع گیٹ سے تین مداخل جمع گیٹ کا حصول۔}
\label{شکل_بوولین_دو_جمع_سے_تین}
\end{figure}
%

\begin{figure}
\centering
\begin{subfigure}{1\textwidth}
\centering
\begin{tikzpicture}
\pgfmathsetmacro{\kxs}{3}
\pgfmathsetmacro{\kys}{1.25}
\draw(0,0)node[or port ,scale=1, number inputs=2](u1){u1} (0,-\kys)node[or port ,scale=1, number inputs=2](u2){u2} (\kxs,-\kys/2)node[or port ,scale=1, number inputs=2](u3){u3};
\draw(u1.in 1)node[left]{$A$} (u1.in 2)node[left]{$B$} (u2.in 1)node[left]{$C$} (u2.in 2)node[left]{$D$};
\draw(u1.out)node[above right]{$A+B$}-|(u3.in 1) (u2.out)node[below right]{$C+D$}-|(u3.in 2) (u3.out)node[right]{$(A+B)+(C+D)$};
\end{tikzpicture}
\caption{}
\end{subfigure}
\begin{subfigure}{1\textwidth}
\centering
\begin{tikzpicture}
\pgfmathsetmacro{\kxs}{3}
\pgfmathsetmacro{\kys}{1.25}
\draw(0,0)node[or port ,scale=1, number inputs=2](u4){u4} (0,-\kys)node[or port ,scale=1, number inputs=2](u5){u5} (\kxs,-\kys/2)node[and port ,scale=1, number inputs=2](u6){u6};
\draw(u4.in 1)node[left]{$A$} (u4.in 2)node[left]{$B$} (u5.in 1)node[left]{$C$} (u5.in 2)node[left]{$D$};
\draw(u4.out)node[above right]{$A+B$}-|(u6.in 1) (u5.out)node[below right]{$C+D$}-|(u6.in 2) (u6.out)node[right]{$(A+B)(C+D)$};
\end{tikzpicture}
\caption{}
\end{subfigure}
\caption{جمع اور ضرب گیٹ کے ادوار۔}
\label{شکل_بوولین_جمع_ضرب_ادوار}
\end{figure}
 جمع گیٹ اور ضرب گیٹ پر مبنی، شکل \حوالہ{شکل_بوولین_جمع_ضرب_ادوار}میں دکھائے گئے ادوار کو مثال بنا کر، عددی ادوار حل کرنا سیکھتے ہیں۔
 
 شکل \حوالہ{شکل_بوولین_جمع_ضرب_ادوار}-الف سے آغاز کرتے ہیں جہاں گیٹوں کو \عددی{u1}، \عددی{u2}، اور \عددی{u3} کے نام دیے گئے ہیں۔ جمع گیٹ \عددی{u1} اور \عددی{u2} کے خارجی پنیے، جمع گیٹ \عددی{u3} کے داخلی پنیوں سے جڑے ہیں۔ چونکہ \عددی{u1} کا مخارج \عددی{A+B} اور \عددی{u2} کا مخارج \عددی{C+D} دیگا، لہٰذا \عددی{u3} کا مخارج \عددی{(A+B)+(C+D)} یعنی \عددی{A+B+C+D} دیگا۔ 


آئیں ا ب شکل \حوالہ{شکل_بوولین_جمع_ضرب_ادوار}-ب حل ہیں۔ یہاں \عددی{u4} اور \عددی{u5} کے مخارج بالترتیب \عددی{A+B} اور \عددی{C+D} دیں گے۔ چونکہ \عددی{u6} ضرب گیٹ ہے، لہٰذا اس کا مخارج \عددی{(A+B)(C+D)} دیگا۔
\begin{figure}
\centering
\begin{subfigure}{1\textwidth}
\centering
\begin{tikzpicture}
\pgfmathsetmacro{\kxs}{3}
\pgfmathsetmacro{\kys}{1}
\draw(0,0)node[or port ,scale=1, number inputs=2](u1){u1} (0,-1.5*\kys)node[or port ,scale=1, number inputs=2](u2){u2} (\kxs,-\kys/2)node[and port ,scale=1, number inputs=2](u3){u3} (\kxs,-2*\kys)node[not port ,scale=1, number inputs=1, label={[xshift=-0.5em,yshift=-0.5em]u4}](u4){};
\draw(u1.in 1)node[left]{$A$} (u1.in 2)node[left]{$B$} (u2.in 1)node[left]{$C$} (u2.in 2)node[left]{$D$};
\draw(u1.out)node[above right]{$A+B$}-|(u3.in 1) (u2.out)node[below right]{$C+D$}-|coordinate(ktt)(u3.in 2) (u3.out)node[right]{$(A+B)(C+D)$};
\draw(ktt)|-(u4.in) (u4.out)node[right]{$\overline{C+D}$};
\end{tikzpicture}
\caption{}
\end{subfigure}
\begin{subfigure}{1\textwidth}
\centering
\begin{tikzpicture}%[circuit logic US,minimum height=1cm] 
\pgfmathsetmacro{\kxs}{3}
\pgfmathsetmacro{\kys}{1.5}
\draw (0,0) node[or port ,scale=1, number inputs=3](u5){u5};
\draw (0,-\kys) node[or port ,scale=1, number inputs=2](u6){u6};
\draw (1*\kxs,-\kys/2) node[and port ,scale=1, number inputs=2](u7){u7};
\draw(u5.in 1)node[left]{$E$} (u5.in 2)node[left]{$F$} (u5.in 3)node[left]{$G$} 
(u6.in 1)node[left]{$H$} (u6.in 2)node[left]{$I$} (u7.out)node[right]{$(E+F+G)(H+I)$};
\draw(u5.out)node[above right]{$E+F+G$}-|(u7.in 1);
\draw(u6.out)node[below right]{$H+I$}-|(u7.in 2);
\end{tikzpicture}
\caption{}
\end{subfigure}
\caption{گیٹوں کا دوسرا دور۔}
\label{شکل_بوولین_جمع_ضرب_دوسرا_ادوار}
\end{figure}

شکل \حوالہ{شکل_بوولین_جمع_ضرب_دوسرا_ادوار}-الف میں \عددی{u2} کا مخارج \عددی{u3} کے مداخل اور \عددی{u4} کے مداخل کے ساتھ جڑا ہے۔ گیٹ \عددی{u1} اور \عددی{u2} کے مخارج بالترتیب \عددی{A+B} اور \عددی{C+D} ہیں۔ گیٹ \عددی{u3} کا مخارج \عددی{(A+B)(C+D)} اور \عددی{u4} کا مخارج \عددی{\overline{C+D}} ہو گا۔

 آپ شکل \حوالہ{شکل_بوولین_جمع_ضرب_دوسرا_ادوار}-ب کا حل، شکل کو دیکھ کر سمجھ سکتے ہیں۔ 


\جزوحصہ{ضرب متمم گیٹ اور جمع متمم گیٹ}
شکل \حوالہ{شکل_بوولین_ضرب_متمم}-الف میں تین مداخل ضرب گیٹ کا مخارج \عددی{ABC} ہو گا، جو نفی گیٹ کا مداخل ہے، لہٰذا نفی گیٹ کا مخارج \عددی{Z=\overline{ABC}} ہوگا۔ ضرب گیٹ کے مخارج کا متمم اتنی اہمیت رکھتا ہے کہ اس کے لئے علیحدہ گیٹ بنایا گیا ہے، جسے ضرب متمم گیٹ (یا ضد ضرب گیٹ) کہتے ہیں اور جو شکل-ب میں (تین مداخل کے لئے) دکھایا گیا ہے۔ ضرب گیٹ کے جدول کا متمم لینے سے ضرب متمم گیٹ کا جدول حاصل ہو گا جو اسی شکل میں پیش کیا گیا ہے۔

 دو مداخل ضرب متمم گیٹ کی مساوات درج ذیل ہو گی، جہاں \عددی{X} اور \عددی{Y} مداخل جبکہ \عددی{Z} مخارج ہے۔
\begin{align}
Z&=\overline{XY}=\overline{X}+\overline{Y}&\text{\RL{\small{(ضرب متمم)}}}
\end{align}


\begin{figure}
\centering
\begin{minipage}{0.8\textwidth}
\centering
\begin{subfigure}{1\textwidth}
\centering
\begin{tikzpicture}%[circuit logic US,minimum height=1cm] 
\pgfmathsetmacro{\kxs}{1.75}
\pgfmathsetmacro{\kys}{1.5}
\draw(0,0)node[and port ,scale=1, number inputs=3](u1){} (\kxs,0) node[not port, scale=1, number inputs=1](u2){};
\draw(u1.in 1)node[left]{$A$} (u1.in 2)node[left]{$B$} (u1.in 3)node[left]{$C$}
 (u2.out)node[above right]{$Z=\overline{ABC}$};
\draw(u1.out)node[above right,xshift=-0.5em]{$ABC$}--(u2.in);
\end{tikzpicture}
\caption{}
\end{subfigure}
\begin{subfigure}{1\textwidth}
\centering
\begin{otherlanguage}{english}
\begin{tikzpicture}
\pgfmathsetmacro{\kxs}{3}
\pgfmathsetmacro{\kys}{1}
\draw(0,0)node[nand port ,scale=1, number inputs=3](u1){};
\draw(u1.in 1)node[left]{$A$} (u1.in 2)node[left]{$B$} (u1.in 3)node[left]{$C$}
 (u1.out)node[right]{$Z=\overline{ABC}$};
\end{tikzpicture}
\end{otherlanguage}
\caption{}
\end{subfigure}
\end{minipage}\hfill
\begin{minipage}{0.2\textwidth}
\centering
\begin{otherlanguage}{english}
\begin{tabular}{CCC|C}
\toprule
A&B&C&Z\\
\midrule
0&0&0&1\\
0&0&1&1\\
0&1&0&1\\
0&1&1&1\\
1&0&0&1\\
1&0&1&1\\
1&1&0&1\\
1&1&1&0\\
\bottomrule
\end{tabular}
\end{otherlanguage}
\end{minipage}
\caption{ضرب متمم گیٹ یا ضد ضرب گیٹ۔}
\label{شکل_بوولین_ضرب_متمم}
\end{figure}
%
\begin{figure}
\centering
\begin{minipage}{0.8\textwidth}
\centering
\begin{subfigure}{1\textwidth}
\centering
\begin{tikzpicture}%[circuit logic US,minimum height=1cm] 
\pgfmathsetmacro{\kxs}{2}
\pgfmathsetmacro{\kys}{1.5}
\draw(0,0)node[or port ,scale=1, number inputs=3](u1){} (\kxs,0) node[not port, scale=1, number inputs=1](u2){};
\draw(u1.in 1)node[left]{$A$} (u1.in 2)node[left]{$B$} (u1.in 3)node[left]{$C$}
 (u2.out)node[above right,font=\small,xshift=-0.5em]{$Z=\overline{A+B+C}$};
\draw(u1.out)node[above right,xshift=-0.5em,font=\small,xshift=-0.5em]{$A+B+C$}--(u2.in);
\end{tikzpicture}
\caption{}
\end{subfigure}
\begin{subfigure}{1\textwidth}
\centering
\begin{tikzpicture}
\pgfmathsetmacro{\kxs}{3}
\pgfmathsetmacro{\kys}{1}
\draw(0,0)node[nor port ,scale=1, number inputs=3](u1){};
\draw(u1.in 1)node[left]{$A$} (u1.in 2)node[left]{$B$} (u1.in 3)node[left]{$C$}
 (u1.out)node[right]{$Z=\overline{A+B+C}$};
\end{tikzpicture}
\caption{}
\end{subfigure}
\end{minipage}\hfill
\begin{minipage}{0.2\textwidth}
\centering
\begin{otherlanguage}{english}
\begin{tabular}{CCC|C}
\toprule
A&B&C&Z\\
\midrule
0&0&0&1\\
0&0&1&0\\
0&1&0&0\\
0&1&1&0\\
1&0&0&0\\
1&0&1&0\\
1&1&0&0\\
1&1&1&0\\
\bottomrule
\end{tabular}
\end{otherlanguage}
\end{minipage}
\caption{جمع متمم گیٹ یا ضد جمع گیٹ۔}
\label{شکل_بوولین_جمع_متمم}
\end{figure}


شکل \حوالہ{شکل_بوولین_جمع_متمم}-الف میں تین مداخل جمع گیٹ کا مخارج \عددی{A+B+C} ہو گا، جو نفی گیٹ کا مداخل ہے، لہٰذا نفی گیٹ کا مخارج \عددی{Z=\overline{A+B+C}} ہوگا۔ جمع گیٹ کے مخارج کا متمم اتنی اہمیت رکھتا ہے کہ اس کے لئے علیحدہ گیٹ بنایا گیا ہے، جسے جمع متمم گیٹ (یا ضد جمع گیٹ) کہتے ہیں اور جو شکل-ب میں (تین مداخل کے لئے) دکھایا گیا ہے۔ جمع گیٹ کے جدول کا متمم لینے سے جمع متمم گیٹ کا جدول حاصل ہو گا جو اسی شکل میں پیش کیا گیا ہے۔

 دو مداخل جمع متمم گیٹ کی مساوات درج ذیل ہو گی، جہاں \عددی{X} اور \عددی{Y} مداخل جبکہ \عددی{Z} مخارج ہے۔
\begin{align}
Z&=\overline{X+Y}=\overline{X}\cdot\overline{Y}&\text{\RL{\small{(جمع متمم)}}}
\end{align}



\begin{figure}
\centering
\begin{tikzpicture}
\draw(0,0)node[nor port, scale=1, number inputs=2](u1){};
\draw(u1.in 1)--(u1.in 2)coordinate[pos=0.5](kk);
\draw(kk)--++(-0.5,0)node[left]{$X$};
\draw(u1.out)node[right]{$Z=\overline{X+ X}=\overline{X}$};
\end{tikzpicture}\quad\quad\quad
\begin{tikzpicture}
\draw(0,0)node[nand port, scale=1, number inputs=2](u1){};
\draw(u1.in 1)--(u1.in 2)coordinate[pos=0.5](kk);
\draw(kk)--++(-0.5,0)node[left]{$X$};
\draw(u1.out)node[right]{$Z=\overline{X\cdot X}=\overline{X}$};
\end{tikzpicture}
\caption{ضرب متمم اور جمع متمم گیٹ سے نفی گیٹ کا حصول۔}
\label{شکل_بوولین_نفی_حصول}
\end{figure}

شکل \حوالہ{شکل_بوولین_نفی_حصول} میں ضرب متمم اور جمع متمم گیٹ سے نفی گیٹ کا حصول دکھایا گیا ہے۔ ضرب متمم کے دونوں مداخل کو آپس میں جوڑا گیا ہے، لہٰذا دونوں مداخل پر \عددی{X} ہو گا۔یوں مخارج \عددی{Z=\overline{X\cdot X}} یعنی \عددی{Z=\overline{X}} ہو گا؛ یہاں اس حقیقت کو استعمال کیا گیا ہے کہ اگر \عددی{X=0} ہو تب \عددی{X\cdot X} بھی \عددی{0} ہو گا، اور اگر \عددی{X=1} ہو تب \عددی{X\cdot X} بھی \عددی{1} ہو گا، لہٰذا \عددی{X\cdot X=X} لکھا جا سکتا ہے۔ نفی گیٹ کا مخارج بھی یہی (\عددی{Z=\overline{X}}) دیگا، لہٰذا ضرب گیٹ کے دونوں مداخل آپس میں جوڑنے سے نفی گیٹ کی کارکردگی حاصل ہو گی۔ اسی طرح (تسلی کر لیں کہ) جمع گیٹ کے مداخل آپس میں جوڑنے سے بھی نفی گیٹ حاصل ہو گا۔

 شکل \حوالہ{شکل_بوولین_متمم_سے_جمع_ضرب}-الف میں تین جمع متمم گیٹ یوں جوڑے گئے ہیں کہ \عددی{Z=XY} حاصل ہو، جو ضرب گیٹ کی کارکردگی ہے۔یوں جمع متمم گیٹوں سے ضرب گیٹ حاصل ہو گا۔

 شکل \حوالہ{شکل_بوولین_متمم_سے_جمع_ضرب}-ب میں جمع گیٹ کا حصول دکھایا گیا ہے۔ اس کا مخارج \عددی{Z=X+Y} ہے۔
 
 شکل \حوالہ{شکل_بوولین_ضرب_متمم_سے_جمع_ضرب} میں ضرب متمم گیٹ سے (ا) جمع گیٹ اور (ب) ضرب گیٹ کا حصول دکھایا گیا ہے۔ 


\begin{figure}
\centering
\begin{subfigure}{1\textwidth}
\centering
\begin{tikzpicture}
\pgfmathsetmacro{\kxs}{4}
\pgfmathsetmacro{\kys}{1.5}
\draw(0,0)node[nor port ,scale=1, number inputs=2](u1){};
\draw(0,-\kys)node[nor port ,scale=1, number inputs=2](u2){};
\draw(\kxs,-\kys/2)node[nor port ,scale=1, number inputs=2](u3){};
\draw(u1.in 1)--(u1.in 2)coordinate[pos=0.5](kXX) (kXX)--++(-0.5,0)node[left]{$X$};
\draw(u2.in 1)--(u2.in 2)coordinate[pos=0.5](kYY) (kYY)--++(-0.5,0)node[left]{$Y$};
\draw(u1.out)node[above]{$\overline{X}$}-|(u3.in 1);
\draw(u2.out)node[above]{$\overline{Y}$}-|(u3.in 2);
\draw(u3.out)node[right]{$Z=\overline{\overline{X}+\overline{Y}}=\overline{\overline{X}}\cdot\overline{\overline{Y}}=XY$};
\end{tikzpicture}
\caption{}
\end{subfigure}
\begin{subfigure}{1\textwidth}
\centering
\begin{tikzpicture}
\pgfmathsetmacro{\kxs}{4}
\pgfmathsetmacro{\kys}{1.5}
\draw(0,0)node[nor port ,scale=1, number inputs=2](u1){};
\draw(\kxs,0)node[nor port ,scale=1, number inputs=2](u2){};
\draw(u2.in 1)--(u2.in 2)coordinate[pos=0.5](kk);
\draw(u1.out)node[above right]{$\overline{X+Y}$}--(kk);
\draw(u2.out)node[right]{$Z=\overline{\overline{X+Y}}=X+Y$};
\draw(u1.in 1)node[left]{$X$} (u1.in 2)node[left]{$Y$};
\end{tikzpicture}
\caption{}
\end{subfigure}
\caption{جمع متمم سے (ا) ضرب گیٹ اور (ب) جمع گیٹ کا حصول۔}
\label{شکل_بوولین_متمم_سے_جمع_ضرب}
\end{figure}
%
\begin{figure}
\centering
\begin{subfigure}{1\textwidth}
\centering
\begin{tikzpicture}
\pgfmathsetmacro{\kxs}{4}
\pgfmathsetmacro{\kys}{1.5}
\draw(0,0)node[nand port ,scale=1, number inputs=2](u1){};
\draw(0,-\kys)node[nand port ,scale=1, number inputs=2](u2){};
\draw(\kxs,-\kys/2)node[nand port ,scale=1, number inputs=2](u3){};
\draw(u1.in 1)--(u1.in 2)coordinate[pos=0.5](kXX) (kXX)--++(-0.5,0)node[left]{$X$};
\draw(u2.in 1)--(u2.in 2)coordinate[pos=0.5](kYY) (kYY)--++(-0.5,0)node[left]{$Y$};
\draw(u1.out)node[above]{$\overline{X}$}-|(u3.in 1);
\draw(u2.out)node[above]{$\overline{Y}$}-|(u3.in 2);
\draw(u3.out)node[right]{$Z=\overline{\overline{X}\cdot\overline{Y}}=\overline{\overline{X}}+\overline{\overline{Y}}=X+Y$};
\end{tikzpicture}
\caption{}
\end{subfigure}
\begin{subfigure}{1\textwidth}
\centering
\begin{tikzpicture}
\pgfmathsetmacro{\kxs}{4}
\pgfmathsetmacro{\kys}{1.5}
\draw(0,0)node[nand port ,scale=1, number inputs=2](u1){};
\draw(\kxs,0)node[nand port ,scale=1, number inputs=2](u2){};
\draw(u2.in 1)--(u2.in 2)coordinate[pos=0.5](kk);
\draw(u1.out)node[above right]{$\overline{X\cdot Y}$}--(kk);
\draw(u2.out)node[right]{$Z=\overline{\overline{X\cdot Y}}=XY$};
\draw(u1.in 1)node[left]{$X$} (u1.in 2)node[left]{$Y$};
\end{tikzpicture}
\caption{}
\end{subfigure}
\caption{ضرب متمم سے (ا) جمع گیٹ اور (ب) ضرب گیٹ کا حصول۔}
\label{شکل_بوولین_ضرب_متمم_سے_جمع_ضرب}
\end{figure}


\جزوحصہ{بلا شرکت جمع گیٹ اور بلا شرکت جمع متمم گیٹ}
بلا شرکت جمع تفاعل کو بلا شرکت جمع گیٹ سے حاصل کیا جاتا ہے جس کا جدول اور علامت، شکل \حوالہ{شکل_بوولین_بلاشرکت_متمم}-الف میں پیش کیے گئے ہیں۔اسی طرح بلا شرکت جمع متمم (یا ضد بلا شرکت جمع) تفاعل کو بلا شرکت جمع متمم گیٹ (یعنی ضد بلا شرکت جمع گیٹ) کی مدد سے حاصل کیا جاتا ہے جس کا جدول اور علامت، شکل-ب میں پیش کیے گئے ہیں۔

بلا شرکت جمع گیٹ کے مخارج کے ساتھ نفی گیٹ منسلک کرنے سے بلا شرکت جمع متمم گیٹ حاصل کیا جا سکتا ہے۔بلا شرکت جمع گیٹ کی کارکردگی شکل \حوالہ{شکل_بوولین_بلاشرکت_کارکردگی} میں دکھائی گئی ہے، جہاں \عددی{X} اور \عددی{Y} مداخل جبکہ \عددی{Z} مخارج ہے۔
	
تین مداخل بلا شرکت جمع گیٹ کا مخارج حاصل کرنے کے لئے اس کے کسی دو مداخل کا بلا شرکت جمع حاصل کریں اور حاصل جواب کا تیسرے مداخل کے ساتھ بلا شرکت جمع لیں۔ یہی بلا شرکت جمع ہو گا۔متعدد مداخل بلا شرکت جمع گیٹ کا مخارج اُس صورت بلند ہوگا جب بلند مداخل کی تعداد طاق ہو۔

 \begin{figure}
 \centering
 \begin{subfigure}{0.45\textwidth}
 \centering
 \begin{tikzpicture}
 \draw(0,0)node[xor port, scale=1, number inputs=2](){};
 \end{tikzpicture}%
 \begin{otherlanguage}{english}
\begin{tabular}{CCC|C}
\toprule
A&B&C&Z\\
\midrule
0&0&0&0\\
0&0&1&1\\
0&1&0&1\\
0&1&1&0\\
1&0&0&1\\
1&0&1&0\\
1&1&0&0\\
1&1&1&1\\
\bottomrule
\end{tabular}
\end{otherlanguage}
 \caption{}
 \end{subfigure}\hfill
 \begin{subfigure}{0.45\textwidth}
 \centering
 \begin{tikzpicture}
 \draw(0,0)node[xnor port, scale=1, number inputs=2](){};
 \end{tikzpicture}%
 \begin{otherlanguage}{english}
\begin{tabular}{CCC|C}
\toprule
A&B&C&Z\\
\midrule
0&0&0&1\\
0&0&1&0\\
0&1&0&0\\
0&1&1&1\\
1&0&0&0\\
1&0&1&1\\
1&1&0&1\\
1&1&1&0\\
\bottomrule
\end{tabular}
\end{otherlanguage}
 \caption{}
 \end{subfigure}
 \caption{(ا) بلا شرکت جمع گیٹ اور (ب) بلا شرکت جمع متمم گیٹ۔}
 \label{شکل_بوولین_بلاشرکت_متمم}
 \end{figure}
%
\begin{figure}
\centering
\begin{tikzpicture}
\pgfmathsetmacro{\kts}{0.5}
\pgfmathsetmacro{\kys}{0.5}
\pgfmathsetmacro{\kysep}{0.75}
\draw(0,0)node[above left]{$X$}--++(3*\kts,0)--++(0,\kys)--++(5*\kts,0)--++(0,-\kys)--++(4*\kts,0)--++(0,\kys)--++(2*\kts,0);
\draw(0,-\kysep)node[above left]{$Y$}--++(1*\kts,0)--++(0,\kys)--++(1*\kts,0)--++(0,-\kys)--++(2*\kts,0)--++(0,\kys)--++(\kts,0)
--++(0,-\kys)--++(\kts,0)--++(0,\kys)--++(3*\kts,0)--++(0,-\kys)--++(\kts,0)--++(0,\kys)--++(3*\kts,0)--++(0,-\kys)--++(\kts,0);
\draw(0,-2*\kysep)node[above left]{$Z$}--++(\kts,0)--++(0,\kys)--++(\kts,0)--++(0,-\kys)--++(\kts,0)--++(0,\kys)--++(\kts,0)--++(0,-\kys)--++(\kts,0)--++(0,\kys)--++(\kts,0)--++(0,-\kys)--++(2*\kts,0)--++(0,\kys)--++(\kts,0)--++(0,-\kys)--++(\kts,0)--++(0,\kys)--++(2*\kts,0)--++(0,-\kys)--++(\kts,0)--++(0,\kys)--++(\kts,0);
\end{tikzpicture}
\caption{بلا شرکت جمع گیٹ کی کارکردگی۔}
\label{شکل_بوولین_بلاشرکت_کارکردگی}
\end{figure}

آپ سے گزارش ہے کہ مذکورہ بالا تفاعلات اور گیٹوں کو اچھی طرح سمجھیں اور ذہن نشین کریں۔



\حصہ{گیٹوں کے برقی خواص} 
گیٹ (کا مخارج) اس صورت بلند تصور کیا جاتا ہے جب اس ( کے مخارج پنیا) کا خارجی دباو ایک مخصوص قیمت یا اس سے زیادہ ہو۔یہ قیمت بلند خارجی برقی دباو \عددی{V_{OH}} کہلاتی ہے۔بلند صورت میں گیٹ مخارج پنیے پر ایک مخصوص قیمت تک برقی رو خارج (مہیا) کر سکتا ہے ، جو گیٹ کا بلند خارجی برقی رو \عددی{I_{OH}} کہلاتا ہے۔

گیٹ (کا مخارج) اس صورت پست تصور کیا جاتا ہے جب اس (کے مخارج پنیا) کا خارجی دباو ایک مخصوص قیمت یا اس سے کم ہو۔یہ قیمت پست خارجی برقی دباو \عددی{V_{OL}} کہلاتی ہے۔پست گیٹ، مخارج پنیے پر ایک مخصوص قیمت تک برقی رو جذب کر سکتا ہے ، جو گیٹ کا پست خارجی برقی رو \عددی{I_{OL}} کہلاتا ہے۔

	
گیٹ ایک مخصوص قیمت اور اس سے زیادہ داخلی برقی دباو کو بلند تصور کرتا ہے۔اس برقی دباو کو بلند داخلی برقی دباو \عددی{V_{IH}} کہتے ہیں۔گیٹ کے کسی ایک مداخل کو بلند کرنے کی خاطر درکار برقی رو کو بلند داخلی برقی رو \عددی{I_{IH}} کہتے ہیں۔

گیٹ ایک مخصوص قیمت اور اس سے کم داخلی برقی دباو کو پست تصور کرتا ہے۔اس قیمت کو پست داخلی برقی دباو \عددی{V_{IL}} کہتے ہیں۔گیٹ کے کسی ایک مداخل کو پست کرنے کی خاطر درکار برقی رو کو پست داخلی برقی رو \عددی{I_{IL}} کہتے ہیں۔
	
 گیٹوں کو آپس میں برقی تاروں سے جوڑا جاتا ہے۔کبھی کبھار ان تاروں میں، جائے استعمال پر پائے جانے والے تغیر پذیر برقی و مقناطیسی میدان کی وجہ سے، غیر ضروری اور ناپسندیدہ برقی دباو پیدا ہوتا ہے جسے برقی شور کہتے ہیں۔ایک گیٹ کے پست خارجی برقی دباو کے ساتھ یہ شور جمع ہو کر اگلے گیٹ کے پست داخلی برقی دباو سے تجاوز کر سکتا ہے۔اسی طرح برقی شور بلند خارجی برقی دباو سے نفی ہو کر بلند داخلی برقی دباو سے کم ہو سکتا ہے۔ان دونوں صورتوں میں اگلا گیٹ غیر متوقع نتائج دیگا۔

 بلند خارجی برقی دباو کی قیمت ، بلند داخلی برقی دباو کی قیمت سے زیادہ ہوتی ہے۔ان کے فرق کو بلند شور گنجائش \عددی{V_{NH}} کہتے ہیں (شکل \حوالہ{شکل_بوولین_گنجائش_شور} دیکھیں)۔
\begin{align}
V_{NH}=V_{OH}-V_{IH}
\end{align}
 پست خارجی برقی دباو کی قیمت، پست داخلی برقی دباو کی قیمت سے کم ہوتی ہے۔ان کے فرق کو پست شور گنجائش \عددی{V_{NL}} کہتے ہیں۔
\begin{align}
V_{NL}=V_{IL}-V_{OL}
\end{align}


\begin{figure}
\centering
\begin{tikzpicture}
\pgfmathsetmacro{\kkx}{2}
\pgfmathsetmacro{\kky}{3}
\pgfmathsetmacro{\kkOL}{0.65}
\pgfmathsetmacro{\kkOH}{2.35}
\pgfmathsetmacro{\kTOL}{\kkOL/2}
\pgfmathsetmacro{\kTOH}{(\kky+\kkOH)/2}
\pgfmathsetmacro{\kkIL}{0.85}
\pgfmathsetmacro{\kkIH}{2}
\pgfmathsetmacro{\kTIL}{\kkIL/2}
\pgfmathsetmacro{\kTIH}{(\kky+\kkIH)/2}
\pgfmathsetmacro{\kkxsep}{4}
\draw(0,0)node[left]{$0$} rectangle ++(\kkx,\kky);
\draw(0,\kkOL)node[left]{$V_{OL}$}--++(\kkx,0);
\draw(0,\kkOH)node[left]{$V_{OH}$}--++(\kkx,0);
\draw(\kkxsep,0) rectangle ++(\kkx,\kky);
\draw(\kkxsep,\kkIL)--++(\kkx,0)node[right]{$V_{IL}$};
\draw(\kkxsep,\kkIH)--++(\kkx,0)node[right]{$V_{IH}$};
\draw[dashed] (\kkx,\kkOL)--++(1.5,0)coordinate[pos=0.75](kkL);
\draw[dashed] (\kkx,\kkOH)--++(1.5,0)coordinate[pos=0.6](kkH);
\draw[dashed] (\kkxsep,\kkIL)--++(-1.5,0);
\draw[dashed] (\kkxsep,\kkIH)--++(-1.5,0);
\draw[<-] (kkL)--++(0,-0.5)node[below]{$V_{NL}$};
\draw[<-]($(kkL)+(0,\kkIL-\kkOL)$)--++(0,0.25);
\draw[<-] (kkH)--++(0,0.5)node[above]{$V_{NH}$};
\draw[<-]($(kkH)+(0,\kkIH-\kkOH)$)--++(0,-0.25);
\draw(\kkx/2,\kTOL)node[]{\RL{خارجی پست}};
\draw(\kkx/2,\kTOH)node[]{\RL{خارجی بلند}};
\draw(\kkxsep+\kkx/2,\kTIL)node[]{\RL{داخلی پست}};
\draw(\kkxsep+\kkx/2,\kTIH)node[]{\RL{داخلی بلند}};
\draw(0,\kky)node[left]{$V_{DD}$};
\draw(\kkx+\kkxsep,0)node[right]{$0$};
\draw(\kkx+\kkxsep,\kky)node[right]{$V_{DD}$};
\end{tikzpicture}
\caption{شور کی گنجائش کا تخمینہ۔}
\label{شکل_بوولین_گنجائش_شور}
\end{figure}


 شکل \حوالہ{شکل_بوولین_گنجائش_شور} میں \عددی{V_{DD}} گیٹ کو مہیا کردہ برقی دباو ہے جسے اس کتاب میں مثبت پانچ وولٹ \عددی{(\SI{5}{\volt})} تصور کیا گیا ہے جبکہ \عددی{0}سے مراد صفر وولٹ برقی دباو (یعنی برقی زمین) ہے۔
 
 
پست داخلی برقی دباو اور بلند داخل برقی دباو کے بیچ سعت (\عددی{V_{IL}} تا \عددی{V_{IH}}) معنی نہیں رکھتا اور غیر متوقع صورت پیدا کر سکتا ہے، لہٰذا عددی اشارات اس خطہ کو استعمال نہیں کرتے۔ 
 گیٹ اپنے مخارج کو تب تک بلند رکھ سکتا ہے جب تک یہ (اپنی) بلند خارجی برقی رو حد یا اس سے کم برقی رو مہیا کرتا ہو۔اسی طرح گیٹ اپنے مخارج تب تک پست رکھ سکتا ہے جب تک گیٹ (اپنی) پست خارجی برقی رو حد یا اس سے کم رو جذب کرے۔ ایسے مقام پر جہاں گیٹ ان حدود کے اندر نہ رہ سکے، ایسا توانا گیٹ نسب کیا جائے گا جو زیادہ برقی رو خارج یا (اور) جذب کر سکے۔یہ توانا گیٹ، مستحکم کار کہلاتا ہے، جس پر اب غور کرتے ہیں۔
 	
\جزوحصہ{مستحکم کار}
جیسا ا ذکر ہو، ا مستحکم کار وہ توانا گیٹ ہے جو زیادہ برقی رو خارج اور جذب کر سکتا ہے۔اسے عموماً اس مقام پر نسب کیا جاتا ہے جہاں درکار برقی رو عام گیٹ کے برقی رو کی حدود سے تجاوز کرتا ہو۔عموماً مستحکم کار مجاز و معذور ہونے کی صلاحیت بھی رکھتا ہے۔ 

\begin{figure}
\centering
\begin{subfigure}{0.45\textwidth}
\centering
\begin{tikzpicture}
\pgfmathsetmacro{\klen}{1};
\pgfmathsetmacro{\kpin}{0.5};
\draw[thick](0,0)--++(30:\klen)coordinate(ktip)--++(150:\klen)coordinate[pos=0.4](ked)--++(-90:\klen);
\draw[thick](ked)--++(0,0.5)--++(-1,0)node[left]{$\text{مجاز}$};
\draw[thick](ktip)--++(\kpin,0) (0,\klen/2)--++(-\kpin,0);
\end{tikzpicture}
\caption{بلند عمل پیرا غیر متمم مستحکم کار}
\end{subfigure}\hfill
\begin{subfigure}{0.45\textwidth}
\centering
\begin{tikzpicture}
\pgfmathsetmacro{\klen}{1};
\pgfmathsetmacro{\kpin}{0.5};
\draw[thick](0,0)--++(30:\klen)coordinate(ktip)--++(150:\klen)coordinate[pos=0.4](ked)--++(-90:\klen);
\draw[thick](ked)--++(0,0.5)--++(-1,0)node[left]{$\text{مجاز}$};
\draw[thick](ktip)++(0.07,0)node[ocirc]{}++(0.07,0)--++(\kpin,0) (0,\klen/2)--++(-\kpin,0);
\end{tikzpicture}
\caption{بلند عمل پیرا متمم مستحکم کار}
\end{subfigure}
\begin{subfigure}{0.45\textwidth}
\centering
\begin{tikzpicture}
\pgfmathsetmacro{\klen}{1};
\pgfmathsetmacro{\kpin}{0.5};
\draw[thick](0,0)--++(30:\klen)coordinate(ktip)--++(150:\klen)coordinate[pos=0.4](ked)--++(-90:\klen);
\draw[thick](ked)++(0,0.07)node[ocirc,fill]{}++(0,0.07)--++(0,0.5)--++(-1,0)node[left]{$\overline{\text{مجاز}}$};
\draw[thick](ktip)--++(\kpin,0) (0,\klen/2)--++(-\kpin,0);
\end{tikzpicture}
\caption{پست عمل پیرا غیر متمم مستحکم کار}
\end{subfigure}\hfill
\begin{subfigure}{0.45\textwidth}
\centering
\begin{tikzpicture}
\pgfmathsetmacro{\klen}{1};
\pgfmathsetmacro{\kpin}{0.5};
\draw[thick](0,0)--++(30:\klen)coordinate(ktip)--++(150:\klen)coordinate[pos=0.4](ked)--++(-90:\klen);
\draw[thick](ked)++(0,0.07)node[ocirc,fill]{}++(0,0.07)--++(0,0.5)--++(-1,0)node[left]{$\overline{\text{مجاز}}$};
\draw[thick](ktip)++(0.07,0)node[ocirc,fill]{}++(0.07,0)--++(\kpin,0) (0,\klen/2)--++(-\kpin,0);
\end{tikzpicture}
\caption{پست عمل پیرا متمم مستحکم کار}
\end{subfigure}
\caption{مجاز و معذور صلاحیت کے مستحکم کار۔}
\label{شکل_بوولین_مستحکم_کار_اقسام}
\end{figure}
مستحکم کار کی مختلف اقسام کی علامتیں شکل\حوالہ{شکل_بوولین_مستحکم_کار_اقسام} میں دکھائی گئی ہیں۔مجاز کردہ مستحکم کار، داخلی مواد کو خارج کرتا ہے جبکہ معذور کردہ مستحکم کار منقطع سوئچ کی طرح دونوں اطراف کے ادوار منقطع کرتا ہے۔معذور مستحکم کار \قول{زیادہ رکاوٹی حال} اختیار کرتے ہوئے نہ \عددی{0} اور نہ \عددی{1} خارج کرتا ہے۔


مجاز و معذور صلاحیت کے مستحکم کار بطور برقی سوئچ کام کرتے ہیں۔ شکل \حوالہ{شکل_بوولین_مستحکم_کار_اقسام}- ا اور ب کے مستحکم کار کو منقطع کرنے کی خاطر \قول{مجاز} کو پست کیا جائے گا، جبکہ اسے بلند کرنے سے مستحکم کار مجاز ہو کر مداخل کے مواد کو مخارج تک پہنچائے گا۔ شکل-ج اور د میں مستحکم کار کے مخارج کو مداخل سے منقطع کرنے کی خاطر \عددی{\overline{\text{مجاز}}} برقی اشارہ کو بلند کیا جائے گا، جبکہ انہیں جوڑنے کی خاطر اس برقی اشارے کو پست کیا جائے گا۔ مزید، شکل ب اور د میں مخارج پر داخلی اشارے کا متمم حاصل ہو گا۔انہیں وجوہات کی بنا پر شکل \حوالہ{شکل_بوولین_مستحکم_کار_اقسام}- ا کا دور\اصطلاح{بلند عمل پیرا غیر متمم مستحکم کار}\فرہنگ{مستحکم کار!بلند عمل پیرا غیر متمم}\حاشیہب{active high non inverting buffer}\فرہنگ{buffer!active high non inverting} ، شکل-ب \اصطلاح{بلند عمل پیرا متمم مستحکم کار}\فرہنگ{مستحکم کار!بلند عمل پیرا متمم}\حاشیہب{active high inverting buffer}\فرہنگ{buffer!active high,inverting}، شکل-ج \اصطلاح{پست عمل پیرا غیر متمم مستحکم کار}\فرہنگ{مستحکم کار!پست عمل پیرا غیر متمم }\حاشیہب{active low non inverting buffer}\فرہنگ{buffer!active low non inverting}، اور شکل-د \اصطلاح{پست عمل پیرا متمم مستحکم کار}\فرہنگ{مستحکم کار!پست عمل پیرا متمم}\حاشیہب{active low inverting buffer}\فرہنگ{buffer!active low,inverting} کہلاتے ہیں۔
 

\begin{figure}
\centering
\begin{subfigure}{0.45\textwidth}
\centering
\begin{tikzpicture}
\pgfmathsetmacro{\klen}{1};
\pgfmathsetmacro{\klenn}{0.75};
\pgfmathsetmacro{\kpin}{0.5};
\draw[thick](0,0)--++(30:\klen)coordinate(ktip)--++(150:\klen)coordinate[pos=0.4](ked)--++(-90:\klen);
\draw[thick](ked)--++(0,0.5)--++(-1,0)node[left]{$\text{مجاز}$};
\draw[thick](ktip)--++(\kpin,0) ++(0,-\klenn/2)--++(30:\klenn)coordinate(ktipa)--++(150:\klenn)--++(-90:\klenn);
\draw[thick](ktipa)++(0.07,0)node[ocirc]{}++(0.07,0)--++(\kpin,0);
\draw [thick](0,\klen/2)--++(-\kpin,0);
\end{tikzpicture}
\caption{}
\end{subfigure}\hfill
\begin{subfigure}{0.45\textwidth}
\centering
\begin{tikzpicture}
\pgfmathsetmacro{\klen}{1};
\pgfmathsetmacro{\klenn}{0.75};
\pgfmathsetmacro{\kpin}{0.5};
\draw[thick](0,0)--++(30:\klen)coordinate(ktip)--++(150:\klen)coordinate[pos=0.4](ked)--++(-90:\klen);
\draw[thick](ktip)--++(\kpin,0) (0,\klen/2)--++(-\kpin,0);
\draw[thick](ked)--++(0,0.25)++(0,0.07)node[ocirc]{}++(0,0.07)--++(60:\klenn)--++(180:\klenn)coordinate[pos=0.5](kttop)--++(-60:\klenn);
\draw[thick](kttop)--++(0,0.25)--++(-1,0)node[left]{$\overline{\text{مجاز}}$};
\end{tikzpicture}
\caption{}
\end{subfigure}
\caption{نفی گیٹ استعمال کرنے سے دیگر مستحکم کار حاصل کیے جاتے ہیں۔}
\label{شکل_بوولین_مستحکم_کار_متمم_قابو_حصول}
\end{figure}

شکل \حوالہ{شکل_بوولین_مستحکم_کار_اقسام}-الف کے مستحکم کار کے مخارج کو نفی گیٹ سے منسلک کر کے شکل-ب کا مستحکم کار حاصل ہو گا (شکل \حوالہ{شکل_بوولین_مستحکم_کار_متمم_قابو_حصول}-الف دیکھیں) جس کا مخارج داخلی اشارے کا متمم ہو گا۔ اسی طرح شکل \حوالہ{شکل_بوولین_مستحکم_کار_اقسام}-الف کے قابو اشارہ (مجاز) سے پہلے نفی گیٹ نسب کرنے سے شکل-ج حاصل ہو گا (شکل \حوالہ{شکل_بوولین_مستحکم_کار_متمم_قابو_حصول}-ب دیکھیں)۔ شکل \حوالہ{شکل_بوولین_مستحکم_کار_اقسام}-الف کے قابو اشارہ (مجاز) سے پہلے اور مخارج کے بعد نفی گیٹ نسب کرنے سے شکل-د حاصل ہو گا۔


بلند عمل پیرا غیر متمم مستحکم کار (شکل \حوالہ{شکل_بوولین_مستحکم_کار_اقسام}-الف) کی کارکردگی جدول \حوالہ{جدول_بوولین_مجاز_مستحکم_کار}-الف میں پیش کی گئی ہے۔ غیر مجاز مستحکم کار کا مخارج \قول{بلند رکاوٹی حال} میں ہو گا۔ جدول-الف کی اولین دو صف اس صورت کو ظاہر کرتی ہیں؛ چونکہ غیر مجاز حال میں مداخل کی قیمت نتائج پر اثر انداز نہیں ہوتی، انہیں جدول میں \عددی{x} سے ظاہر کیا جاتا ہے (جدول-ب دیکھیں)؛ جہاں \عددی{x} \قول{غیر دلچسپ} قیمتوں کو ظاہر کرتا ہے (جن کا \عددی{0} یا \عددی{1} ہونے کا کوئی اثر نہیں پایا جاتا)۔

\begin{table}
\centering
\caption{بلند عمل پیرا غیر متمم مستحکم کار کی کارکردگی۔}
\label{جدول_بوولین_مجاز_مستحکم_کار}
\begin{subtable}{0.45\textwidth}
\caption{}
\centering
\begin{otherlanguage}{english}
\begin{tabular}{CCC}
\toprule
\text{مجاز}&\text{مداخل}&\text{مخارج}\\
\midrule
0&0&\text{\RL{بلند رکاوٹی حال}}\\
0&1&\text{\RL{بلند رکاوٹی حال}}\\
1&0&0\\
1&1&1\\
\bottomrule
\end{tabular}
\end{otherlanguage}
\end{subtable}\hfill
\begin{subtable}{0.45\textwidth}
\caption{}
\centering
\begin{otherlanguage}{english}
\begin{tabular}{CCC}
\toprule
\text{مجاز}&\text{مداخل}&\text{مخارج}\\
\midrule
0&x&\text{\RL{بلند رکاوٹی حال}}\\
1&0&0\\
1&1&1\\
\bottomrule
\end{tabular}
\end{otherlanguage}
\end{subtable}
\end{table}

جدول سے آپ دیکھ سکتے ہیں کہ \قول{مجاز} کو پست \عددی{(0)} کرنے سے مستحکم کار بلند رکاوٹی حال اختیار کر کے، مخارج سے جڑے ادوار پر کسی قسم کا اثر نہیں رکھتا۔مجاز بلند \عددی{(1)} کرنے سے مخارج پر وہی مواد خارج ہو گا جو مداخل پر مہیا کیا جائے۔


مستحکم کار داخلی جانب سے خارجی جانب مواد منتقل کرتا ہے۔ جہاں دو ادوار کے مابین دونوں جانب مواد کی ترسیل درکار ہو، وہاں دو مستحکم کار آپس میں متوازی اُلٹ جوڑے جاتے ہیں، شکل \حوالہ{شکل_بوولین_دو_طرفہ_مستحکم_کار}-الف دیکھیں۔اس کو دو طرفہ مستحکم کار کہتے ہیں۔شکل-ب میں اس کی علامت پیش کی گئی ہے۔ بلند \قول{مجاز} کی صورت میں \عددی{u1} مجاز اور \عددی{u2} معذور ہو گا لہٰذا مواد بائیں سے دائیں منتقل ہو گا، جبکہ پست \قول{مجاز} کی صورت میں \عددی{u2} مجاز اور \عددی{u1} معذور ہو گا لہٰذا مواد دائیں سے بائیں منتقل ہو گا۔

اسی طرح متمم دو طرفہ مستحکم کار بھی بنایا جاتا ہے، جو مواد کا متمم خارج کرے گا۔

 مستحکم کار اور متمم مستحکم کار کے مداخل آپس میں جوڑنے سے ان کے مخارج پر تضاد حال حاصل کیے جا سکتے ہیں؛ شکل \حوالہ{شکل_بوولین_اشارہ_اور_متمم}-الف دیکھیں۔شکل -ب میں اس کی علامت پیش کی گئی ہے۔
 
 \begin{figure}
 \centering
 \begin{subfigure}{0.45\textwidth}
\centering
 \begin{tikzpicture}
\pgfmathsetmacro{\klen}{1};
\pgfmathsetmacro{\kpin}{0.5};
\pgfmathsetmacro{\kpina}{0.75};
\pgfmathsetmacro{\kysep}{1};
\pgfmathsetmacro{\kxsep}{1.25};
\draw[thick](0,0)--++(30:\klen)coordinate(ktip)--++(150:\klen)coordinate[pos=0.4](ked)--++(-90:\klen);
\draw[thick](0,-\kysep)coordinate(klout)--++(-30:\klen)coordinate[pos=0.4](keda)--++(90:\klen)--++(210:\klen);
\draw[thick,name path=kkk](ked)--++(0,0.75)coordinate(kthere)coordinate[pos=0.5](kcon)node[above]{$\text{مجاز}$};
\draw[thick](ktip)--++(\kpin,0)--++(0,-\kysep-\klen/2)--++(-\kpin,0) (0,\klen/2)--++(-\kpin,0)--++(0,-\kysep-\klen/2)--++(\kpin,0);
\draw[thick](ktip)++(\kpin,0)++(0,-\kysep/2-\klen/4)--++(\kpina,0);
\draw[thick](0,\klen/2)++(-\kpin,0)++(0,-\kysep/2-\klen/4)--++(-\kpina,0);
\draw[thick](keda)++(0,-0.07)node[ocirc]{}++(0,-0.07)--++(0,-\kpin)--++(\kxsep,0)--++(0,\kysep+1.5*\kpin+\klen)coordinate(khere);
\draw[thick](khere)--($(ked)!(khere)!(kthere)$);
\draw(0,0)++(0,\klen/2)node[xshift=0.75em]{$u1$};
\draw(0,-\kysep)node[xshift=1.5em]{$u2$};
\end{tikzpicture}
\caption{}
\end{subfigure}\hfill
\begin{subfigure}{0.45\textwidth}
 \centering
 \begin{tikzpicture}
\pgfmathsetmacro{\klen}{1};
\pgfmathsetmacro{\kpin}{0.5};
\pgfmathsetmacro{\kpina}{0.75};
\pgfmathsetmacro{\kysep}{0.3};
\pgfmathsetmacro{\kxsep}{1.25};
\draw[thick](0,0)--++(30:\klen)coordinate(ktip)--++(150:\klen)coordinate[pos=0.4](ked)--++(-90:\klen);
\draw[thick](0,-\kysep)coordinate(klout)--++(-30:\klen)coordinate[pos=0.4](keda)--++(90:\klen)--++(210:\klen);
\draw[thick,name path=kkk](ked)--++(0,0.75)coordinate[pos=0.5](kcon)node[above]{$\text{مجاز}$};
\draw[thick](ktip)--++(\kpin,0)--++(0,-\kysep-\klen/2)--++(-\kpin,0) (0,\klen/2)--++(-\kpin,0)--++(0,-\kysep-\klen/2)--++(\kpin,0);
\draw[thick](ktip)++(\kpin,0)++(0,-\kysep/2-\klen/4)--++(\kpina,0);
\draw[thick](0,\klen/2)++(-\kpin,0)++(0,-\kysep/2-\klen/4)--++(-\kpina,0);
\end{tikzpicture}
\caption{}
\end{subfigure}
\caption{دو طرفہ مستحکم کار۔}
\label{شکل_بوولین_دو_طرفہ_مستحکم_کار}
 \end{figure}
%
\begin{figure}
 \centering
 \begin{subfigure}{0.45\textwidth}
\centering
 \begin{tikzpicture}
\pgfmathsetmacro{\klen}{1};
\pgfmathsetmacro{\kpin}{0.5};
\pgfmathsetmacro{\kpina}{0.75};
\pgfmathsetmacro{\kysep}{1.5};
\pgfmathsetmacro{\kxsep}{1.25};
\draw[thick](0,0)--++(30:\klen)coordinate(ktip)--++(150:\klen)coordinate[pos=0.4](ked)--++(-90:\klen);
\draw[thick](0,-\kysep)--++(30:\klen)coordinate(ktipa)--++(150:\klen)coordinate[pos=0.4](ked)--++(-90:\klen);
\draw[thick](ktip)++(0.07,0)node[ocirc,fill]{}++(0.07,0)--++(\kpin,0)node[right]{$\overline{X}$};
\draw[thick](ktipa)--++(\kpin+0.14,0)node[right]{$X$};
\draw[thick](0,\klen/2)--++(-\kpin,0)--++(0,-\kysep)--++(\kpin,0);
\draw[thick](0,\klen/2)++(-\kpin,0)++(0,-\kysep/2)--++(-\kpin,0)node[left]{$X$};
\end{tikzpicture}
\caption{}
\end{subfigure}\hfill
\begin{subfigure}{0.45\textwidth}
 \centering
 \begin{tikzpicture}
\pgfmathsetmacro{\klen}{1};
\pgfmathsetmacro{\kpin}{0.5};
\pgfmathsetmacro{\kpina}{0.75};
\pgfmathsetmacro{\kysep}{0.3};
\pgfmathsetmacro{\kxsep}{1.25};
\draw[thick](0,0)--++(30:\klen)coordinate[pos=0.6](keda)--++(150:\klen)coordinate[pos=0.4](ked)--++(-90:\klen);
\draw[thick](ked)++(0,0.07)node[ocirc,fill]{}++(0.07,0)--++(\kpin,0)node[right]{$\overline{X}$};
\draw[thick](keda)--++(\kpin+0.07,0)node[right]{$X$};
\draw[thick](0,\klen/2)--++(-\kpin,0)node[left]{$X$};
\end{tikzpicture}
\caption{}
\end{subfigure}
\caption{اشارہ اور اشارے کا متمم دیتا مستحکم کار۔}
\label{شکل_بوولین_اشارہ_اور_متمم}
 \end{figure}



\جزوحصہ{مخلوط ادوار}
عام دستیاب ضرب متمم گیٹ شکل \حوالہ{شکل_بوولین_دستیاب_ضرب_متمم} میں دکھایا گیا ہے۔برقیاتی ادوار، عموماً، اسی طرح ڈبی میں بند دستیاب ہوں گے جنہیں مخلوط دور کہتے ہیں۔مخلوط ادوار پر مخلوط دور کا اعدادی نام مثلاً \عددی{7400} درج ہو گا؛ اس عدد کے ہندسوں کے بیچ یا اطراف پر حروف بھی ہوں گے جو اضافی معلومات فراہم کرتے ہیں۔ ساتھ ہی ڈبی پر دوسرا عدد مخلوط دور تیار کرنے کی تاریخ دے گا۔مثلاً یہاں دوسرے عدد کے مطابق یہ مخلوط دور سن \عددی{1976} کے پینتالیسویں \عددی{(45)} ہفتے میں کارخانے میں تیار کیا گیا۔جیسا شکل میں دکھایا گیا ہے، اس مخلوط دور میں چار ضرب متمم گیٹ موجود ہیں۔

 ڈبی پر \قول{کٹ} کے نشان سے گھڑی مخالف رخ پنیے گننے جاتے ہیں۔ گیٹ کی علامت میں پنیے پر لکھا عدد ڈبی میں اس پنیے کا مقام دیتا ہے۔یوں گیٹ کے خارجی پنیے پر \عددی{6} اس پنیے کا ڈبی میں مقام دیتا ہے۔گیٹ کا خاکہ بناتے وقت اس کے قریب مخلوط دور کا نام (یا نمبر جو یہاں \عددی{7400} ہے) بھی لکھا جاتا ہے۔
\begin{figure}
\centering
\begin{tikzpicture}
\draw(0,0)node[nand port, scale=1, number inputs=2](u2){};
\draw(0,0)node[shift={(-0.75,-0.75)}]{$7400$};
\draw(u2.in 1)node[above,font=\tiny]{$4$} (u2.in 2)node[above,font=\tiny]{$5$} (u2.out)node[above,font=\tiny]{$6$};
\end{tikzpicture}\quad\quad\quad
\begin{tikzpicture}[scale=0.5]
\pgfmathsetmacro{\kxlen}{6}
\pgfmathsetmacro{\kylen}{4}
\pgfmathsetmacro{\kxsep}{3}
\pgfmathsetmacro{\kysep}{1.5}
\pgfmathsetmacro{\kscale}{0.5}
\pgfmathsetmacro{\kpin}{0.5}
\pgfmathsetmacro{\kpina}{1.25}
\pgfmathsetmacro{\kpinb}{0.5}
\pgfmathsetmacro{\kpinc}{\kpina/2+\kpinb/2}
\draw(0,0)node[nand port,scale=0.5](u1){} (\kxsep,0)node[nand port,scale=0.5](u2){} (\kxsep/3,\kysep)node[nand port,scale=0.5](u3){} (\kxsep+\kxsep/3,\kysep)node[nand port,scale=0.5](u4){};
\path(-2,-1)coordinate(kll)--++(7,0)coordinate(krr);
\draw(u1.in 1)--++(-\kpin,0)coordinate(aa);
\draw(u1.in 2)--++(0,-\kpinb)coordinate(bb);
\draw(aa)--($(kll)!(aa)!(krr)$)coordinate(dd);
\draw(u1.out)--($(kll)!(u1.out)!(krr)$)coordinate(ee);
\draw($(dd)!0.5!(ee)$)coordinate(ff)++(0,1)coordinate(gg);
\draw(bb)--($(ff)!(bb)!(gg)$)--(ff);
\draw(dd)node[yshift=-0.3cm]{\small{1}}++(-0.1,0) rectangle++(0.2,-0.1);
\draw(ff)++(-0.1,0) rectangle++(0.2,-0.1);
\draw(ee)++(-0.1,0) rectangle++(0.2,-0.1);
%
\draw(u2.in 1)--++(-\kpin,0)coordinate(aa);
\draw(u2.in 2)--++(0,-\kpinb)coordinate(bb);
\draw(aa)--($(kll)!(aa)!(krr)$)coordinate(dd);
\draw(u2.out)--($(kll)!(u2.out)!(krr)$)coordinate(ee);
\draw($(dd)!0.5!(ee)$)coordinate(ff)++(0,1)coordinate(gg);
\draw(bb)--($(ff)!(bb)!(gg)$)--(ff);
\draw(dd)++(-0.1,0) rectangle++(0.2,-0.1);
\draw(ff)++(-0.1,0) rectangle++(0.2,-0.1);
\draw(ee)++(-0.1,0) rectangle++(0.2,-0.1);
\draw(ee)++(\kxsep/3,0)node[yshift=-0.3cm]{\small{7}}++(-0.1,0) rectangle++(0.2,-0.1);
%
\path(-2,\kysep+1)coordinate(kll)--++(7,0)coordinate(krr);
%
\draw(u3.in 2)--++(-\kpin,0)coordinate(aa);
\draw(u3.in 1)--++(0,\kpinb)coordinate(bb);
\draw(aa)--($(kll)!(aa)!(krr)$)coordinate(dd);
\draw(u3.out)--($(kll)!(u3.out)!(krr)$)coordinate(ee);
\draw($(dd)!0.5!(ee)$)coordinate(ff)++(0,1)coordinate(gg);
\draw(bb)--($(ff)!(bb)!(gg)$)--(ff);
\draw(dd)++(-0.1,0) rectangle++(0.2,0.1);
\draw(ff)++(-0.1,0) rectangle++(0.2,0.1);
\draw(ee)++(-0.1,0) rectangle++(0.2,0.1);
\draw(dd)++(-\kxsep/3,0)node[yshift=0.3cm]{\small{14}};
\draw(dd)++(-\kxsep/3,0)++(-0.1,0) rectangle++(0.2,0.1);
%
\draw(u4.in 2)--++(-\kpin,0)coordinate(aa);
\draw(u4.in 1)--++(0,\kpinb)coordinate(bb);
\draw(aa)--($(kll)!(aa)!(krr)$)coordinate(dd);
\draw(u4.out)--($(kll)!(u4.out)!(krr)$)coordinate(ee);
\draw($(dd)!0.5!(ee)$)coordinate(ff)++(0,1)coordinate(gg);
\draw(bb)--($(ff)!(bb)!(gg)$)--(ff);
\draw(dd)++(-0.1,0) rectangle++(0.2,0.1);
\draw(ff)++(-0.1,0) rectangle++(0.2,0.1);
\draw(ee)node[yshift=0.3cm]{\small{8}}++(-0.1,0) rectangle++(0.2,0.1);
\draw(-2.25,-1) rectangle (4.5,\kysep+1);
\path(-2.25,-1)--(-2.25,\kysep+1)coordinate[pos=0.5](kmid);
\draw(kmid)++(0,-0.2)--++(0.2,0.2)--++(-0.2,0.2);
\draw(kmid)++(-0.1,0) to [out=130,in=0]++(-0.5,0.25)node[left]{کٹ};
\end{tikzpicture}
\caption{مخلوط دور \عددی{7400}}
\label{شکل_بوولین_دستیاب_ضرب_متمم}
\end{figure}

 چند مخلوط ادوار درج ذیل ہیں۔
\begin{center}
 \begin{tabular}{rrc}
 \toprule
 نام&گیٹ&ڈبی میں گیٹوں کی تعداد\\
 \midrule
7400& دو مداخل ضرب متمم&4\\
7402& دو مداخل جمع متمم &4\\
7404& نفی&6\\
7406& متمم مستحکم کار &6\\
7408&دو مداخل ضرب&4\\
\bottomrule
 \end{tabular}
 \end{center}

\ابتدا{مشق}
انٹرنیٹ سے مندرجہ بالا تمام مخلوط ادوار کے \اصطلاح{معلوماتی صفحات}\فرہنگ{معلوماتی صفحات}\حاشیہب{datasheet}\فرہنگ{datasheet} حاصل کریں اور ان میں علیحدہ علیحدہ گیٹوں کے مقام دریافت کریں۔	معلوماتی صفحات میں بکثرت مواد موجود ہو گا جنہیں دیکھ کر پریشان مت ہوں۔
\انتہا{مشق}


آپ نے کئی مخلوط ادوار جدول \حوالہ{شکل_بوولین_دستیاب_ضرب_متمم} میں دیکھے جن کے نمبر \عددی{74}سے شروع ہوئے۔دراصل \عددی{74xx}\شناخت{بوولین_مخلوط_ادوار_سلسلہ} مخلوط ادوار کا ایک سلسلہ ہے جس میں جیسے جیسے نئے ادوار بنائے گئے، انہیں شامل کیا گیا۔ان اعداد \عددی{(74xx)} کا از خود کوئی مطلب نہیں۔اسی طرح کا دوسرا سلسلہ \عددی{40xx} پکارا جاتا ہے، جس میں تمام مخلوط ادوار کے نمبر \عددی{40} سے شروع ہوتے ہیں۔

مخلوط ادوار سے کارکردگی حاصل کرنے کے لئے ان کو برقی دباو مہیا کرنا لازم ہے۔ سلسلہ \عددی{7400} کے تمام مخلوط ادوار مثبت یک سمتی پانچ وولٹ \عددی{(\SI{5}{\volt})} پر کام کرتے ہیں۔شکل \حوالہ{شکل_بوولین_دستیاب_ضرب_متمم} میں دکھائے گئے مخلوط دور کو یک سمتی برقی دباو پنیا سات \عددی{(7)} اور چودہ \عددی{(14)} پر مہیا کیا جائے گا، جہاں پنیا \عددی{14} مثبت ہو گا۔جن دو پنیوں پر مخلوط دور کو برقی طاقت مہیا کی جاتی ہے، انہیں \اصطلاح{طاقتی پنیے} کہتے ہیں۔

\ابتدا{مشق}
انٹرنیٹ سے سلسلہ \عددی{40xx} میں دستیاب چار مداخل ضرب گیٹ مخلوط دور کا نمبر دریافت کریں۔اس مخلوط دور کو کتنا برقی دباو درکار ہو گا؟
\انتہا{مشق}

\حصہ{بوولین تفاعل کا تخمینہ}
منطقی ضرب، جمع، نفی تفاعل کے جدول آپ نے دیکھے۔منطقی تفاعل کے جدول کو اس کتاب میں منطقی جدول کہا جائے گا۔منطقی تفاعل کا تخمینہ لگانے میں منطقی جدول نہایت کارآمد ثابت ہو گا۔
بوولین تفاعل کا تخمینہ لگاتے وقت (اس کے) آزاد بوولین متغیرات کی تمام ممکنہ قیمتوں کو ترتیب وار لکھ کر تفاعل حل کیا جائے گا۔


	
\جزوحصہ{بوولین تفاعل کا تخمینہ}
بوولین تفاعل کا تخمینہ لگانے کی خاطر ہم بوولین تفاعل \عددی{Z=A+B\overline{C}} کو مثال لیتے ہیں۔اس تفاعل کے تین آزاد متغیرات ہیں، لہٰذا تین ہندسوں کے تمام ثنائی اعداد لکھ کر آزاد متغیرات کی تمام ممکنہ ترتیب کا جدول لکھتے ہیں۔
\begin{center}
\begin{otherlanguage}{english}
\begin{tabular}{CCC}
\toprule
A&B&C\\
\midrule
0&0&0\\
0&0&1\\
0&1&0\\
0&1&1\\
1&0&0\\
1&0&1\\
1&1&0\\
1&1&1\\
\bottomrule
\end{tabular}
\end{otherlanguage}
\end{center}
تفاعل میں \عددی{C} کی بجائے \عددی{\overline{C}} استعمال ہوا ہے، لہٰذا جدول میں \عددی{\overline{C}} خانہ شامل کرتے ہیں۔ پہلی صف میں \عددی{ABC=000} ہے؛ یوں \عددی{C} کی قیمت \عددی{0} لہٰذا \عددی{\overline{C}} کی قیمت \عددی{1} ہو گی، جس کو نئی قطار میں بطور پہلا جزو درج کرتے ہیں۔ یاد رہے کہ \عددی{C} اور \عددی{\overline{C}} ایک ہی متغیرہ کے دو پہلو ہیں، لہٰذا متغیرات کی تعداد تین رہے گی۔
\begin{center}
\begin{otherlanguage}{english}
\begin{tabular}{CCC|C}
\toprule
A&B&C&\overline{C}\\
\midrule
0&0&0&1\\
0&0&1&0\\
0&1&0&1\\
0&1&1&0\\
1&0&0&1\\
1&0&1&0\\
1&1&0&1\\
1&1&1&0\\
\bottomrule
\end{tabular}
\end{otherlanguage}
\end{center}
تفاعل کی قیمت حاصل کرنے کی خاطر \عددی{B} اور \عددی{\overline{C}} کا منطقی ضرب \عددی{B\overline{C}} درکار ہے، لہٰذا صف در صف \عددی{B} اور \عددی{\overline{C}} کی (مطابقتی قیمتوں کی) منطقی ضرب لے کر نئی قطار میں (مطابقتی صف میں) درج کرتے ہیں۔
\begin{center}
\begin{otherlanguage}{english}
\begin{tabular}{CCC|CC}
\toprule
A&B&C&\overline{C}&B\overline{C}\\
\midrule
0&0&0&1&0\\
0&0&1&0&0\\
0&1&0&1&1\\
0&1&1&0&0\\
1&0&0&1&0\\
1&0&1&0&0\\
1&1&0&1&1\\
1&1&1&0&0\\
\bottomrule
\end{tabular}
\end{otherlanguage}
\end{center}
اب بوولین تفاعل \عددی{A+B\overline{C}} کی قیمت حاصل کرتے ہیں۔جدول میں ایک نیا خانہ شامل کرتے ہیں، جس میں \عددی{A} اور \عددی{B\overline{C}} کا منطقی جمع درج کیا جائے گا ۔
\begin{center}
\begin{otherlanguage}{english}
\begin{tabular}{CCC|CC|C}
\toprule
A&B&C&\overline{C}&B\overline{C}&A+B\overline{C}\\
\midrule
0&0&0&1&0&0\\
0&0&1&0&0&0\\
0&1&0&1&1&1\\
0&1&1&0&0&0\\
1&0&0&1&0&1\\
1&0&1&0&0&1\\
1&1&0&1&1&1\\
1&1&1&0&0&1\\
\bottomrule
\end{tabular}
\end{otherlanguage}
\end{center}

اس جدول میں دایاں خانہ (قطار) دیے گئے بوولین تفاعل کی قیمت دیتا ہے۔یہ آزاد متغیرات کی تین ممکنہ قیمتوں کے لئے \عددی{0}اور باقی تمام کے لئے \عددی{1} کے برابر ہے۔ اس تفاعل کا منطقی گیٹوں کے ذریعہ حصول شکل \حوالہ{شکل_بوولین_تفاعل_گیٹ_کے_ذریعہ} میں دکھایا گیا ہے۔

درج بالا جدول میں کسی بھی صف میں \عددی{A}، \عددی{B}، اور \عددی{C} کی قیمتیں اس دور ( شکل \حوالہ{شکل_بوولین_تفاعل_گیٹ_کے_ذریعہ}) کو مہیا کرنے سے دور، اسی صف میں دی گئی، تفاعل کی قیمت دے گا۔ یوں پہلی صف میں \عددی{A=0}، \عددی{B=0}، اور \عددی{C=0} کے لئے دور \عددی{Z=0} دے گا۔ تیسری صف میں \عددی{A=0}، \عددی{B=1}، اور \عددی{C=0} ہیں جن کے لئے، عین جدول کے مطابق، \عددی{Z=1} حاصل ہو گا۔
\begin{figure}
\centering
\begin{tikzpicture}
\pgfmathsetmacro{\kxsep}{2.75};
\pgfmathsetmacro{\kysep}{0.5};
\path(-1.75,-0.5)coordinate(kkl)--++(0,1)coordinate(kkh);
\draw(0,0)node[not port ,scale=1, number inputs=1](u1){};
\draw(\kxsep,\kysep)node[and port,scale=1, number inputs=2](u2){};
\draw(2*\kxsep,2*\kysep)node[or port,scale=1, number inputs=2](u3){};
\draw(u1.in)--($(kkl)!(u1.in)!(kkh)$)node[left]{$C$};
\draw(u2.in 1)--($(kkl)!(u2.in 1)!(kkh)$)node[left]{$B$};
\draw(u3.in 1)--($(kkl)!(u3.in 1)!(kkh)$)node[left]{$A$};
\draw(u1.out)node[above right]{$\overline{C}$}-|(u2.in 2);
\draw(u2.out)node[above right]{$B\overline{C}$}-|(u3.in 2);
\draw(u3.out)node[right]{$Z=A+B\overline{C}$};
\end{tikzpicture}
\caption{تفاعل \عددی{A+B\overline{C}} کو عددی دور۔}
\label{شکل_بوولین_تفاعل_گیٹ_کے_ذریعہ}
\end{figure}


\حصہ{قوسین میں بند بوولین تفاعل}
روز مرہ الجبرا کی طرح بوولین الجبرا میں بھی قوسین میں بند تفاعل پہلے حل کئے جاتے ہیں۔

\ابتدا{مثال}
تفاعل \عددی{\overline{A}+B(\overline{B}+A)} حل کریں۔

\ترچھا{حل:}\quad 
 تفاعل میں دو آزاد متغیرات ہیں لہٰذا دو ہندسوں پر مبنی ثنائی گنتی لکھ کر آزاد متغیرات کی تمام ترتیب حاصل ہوں گی۔
\begin{center}
\begin{otherlanguage}{english}
\begin{tabular}{CC}
\toprule
A&B\\
\midrule
0&0\\
0&1\\
1&0\\
1&1\\
\bottomrule
\end{tabular}
\end{otherlanguage}
\end{center}
تفاعل میں دونوں متغیرات کے متمم استعمال ہوئے ہیں لہٰذا جدول میں ان کے خانے بناتے ہیں۔
\begin{center}
\begin{otherlanguage}{english}
\begin{tabular}{CC|CC}
\toprule
A&B&\overline{A}&\overline{B}\\
\midrule
0&0&1&1\\
0&1&1&0\\
1&0&0&1\\
1&1&0&0\\
\bottomrule
\end{tabular}
\end{otherlanguage}
\end{center}
اب قوسین میں بند حصہ \عددی{(\overline{B}+A)} کا خانہ بناتے ہیں۔
\begin{center}
\begin{otherlanguage}{english}
\begin{tabular}{CC|CCC}
\toprule
A&B&\overline{A}&\overline{B}&(\overline{B}+A)\\
\midrule
0&0&1&1&1\\
0&1&1&0&0\\
1&0&0&1&1\\
1&1&0&0&1\\
\bottomrule
\end{tabular}
\end{otherlanguage}
\end{center}
اس کے ساتھ \عددی{B(\overline{B}+A)} کا خانہ بناتے ہیں۔یہ خانہ جدول میں دیے \عددی{(\overline{B}+A)} اور \عددی{B} کے مطابقتی اجزاء کی منطقی ضرب سے حاصل ہو گا۔
\begin{center}
\begin{otherlanguage}{english}
\begin{tabular}{CC|CCCC}
\toprule
A&B&\overline{A}&\overline{B}&(\overline{B}+A)&B(\overline{B}+A)\\
\midrule
0&0&1&1&1&0\\
0&1&1&0&0&0\\
1&0&0&1&1&0\\
1&1&0&0&1&1\\
\bottomrule
\end{tabular}
\end{otherlanguage}
\end{center}


اب ہم مکمل بوولین تفاعل کی قیمت حاصل کر سکتے ہیں۔تفاعل \عددی{\overline{A}+B(\overline{B}+A)} حاصل کرنے کی خاطر \عددی{B(\overline{B}+A)} اور \عددی{\overline{A}} کا منطقی جمع حاصل کرنا ہو گا۔
\begin{center}
\begin{otherlanguage}{english}
\begin{tabular}{CC|CCCC|C}
\toprule
A&B&\overline{A}&\overline{B}&(\overline{B}+A)&B(\overline{B}+A)&\overline{A}+B(\overline{B}+A)\\
\midrule
0&0&1&1&1&0&1\\
0&1&1&0&0&0&1\\
1&0&0&1&1&0&0\\
1&1&0&0&1&1&1\\
\bottomrule
\end{tabular}
\end{otherlanguage}
\end{center}
\انتہا{مثال}

\حصہ{بوولین الجبرا کے بنیادی قوانین}
بوولین الجبرا کے پانچ بنیادی قوانین مندرجہ ذیل ہیں۔
\begin{enumerate}[1]
\item
 اگر \عددی{X\ne 0} ہو تب \عددی{X=1} ہو گا، اور
\item
 اگر \عددی{X\ne 1} ہو تب \عددی{X=0} ہو گا۔
\item
 منطقی جمع
\begin{align*}
0+0&=0\\
0+1&=1\\
1+0&=1\\
1+1&=1
\end{align*}	 
\item
 منطقی ضرب
\begin{align*}
0\cdot 0&=0\\
0\cdot 1&=0\\
1\cdot 0&=0\\
1\cdot 1&=1
\end{align*}	
\item
 منطقی نفی
\begin{align*}
\overline{0}&=1\\
\overline{1}&=0
\end{align*}	
\end{enumerate}

اگرچہ یہ پانچ قوانین نہایت سادہ معلوم ہوتے ہیں، ان سے مکمل بوولین الجبرا اخذ کیا جا سکتا ہے۔بوولین الجبرا کے چند قوانین جدول \حوالہ{جدول_بوولین_دو_پہلو_تفاعل} - الف اور ب میں پیش کیے گئے ہیں۔یہ تمام درج بالا پانچ بنیادی قوانین سے اخذ کیے جا سکتے ہیں۔
\begin{table}
\caption{بوولین الجبرا کے چند بنیادی قوانین۔}
\label{جدول_بوولین_دو_پہلو_تفاعل}
\centering
\small
\begin{subtable}{0.45\textwidth}
\caption{پہلا پہلو۔}
%\label{جدول_بوولین_پہلا_پہلو}
\centering
\begin{otherlanguage}{english}
\begin{tabular}{R|L}
\toprule
\text{\RL{شِق}}& \text{\RL{مساوات}}\\
\midrule
1&0 \cdot X=0\\
2&1\cdot X=X\\
3&X\cdot \overline{X}=0\\
4&X\cdot X=X\\
5&X\cdot Y=Y\cdot X\\
6&(X\cdot Y)\cdot Z=X\cdot(Y\cdot Z)\\
7&X+XY=X\\
8&X(X+Y)=X\\
9&(X+Y)(X+Z)=X+YZ\\
10&X+\overline{X}Y=X+Y\\
11&XY+YZ+\overline{Y}Z=XY+Z\\
12&X(Y+Z)=XY+XZ\\
13&\overline{\overline{X}}=X\\
\bottomrule
\end{tabular}
\end{otherlanguage}
\end{subtable}\hfill
\begin{subtable}{0.45\textwidth}
\caption{دوسرا پہلو۔}
%\label{جدول_بوولین_دوسرا_پہلو}
\centering
\begin{otherlanguage}{english}
\begin{tabular}{R|L}
\toprule
\text{\RL{شِق}}& \text{\RL{مساوات}}\\
\midrule
1&1 + X=1\\
2&0+ X=X\\
3&X+ \overline{X}=1\\
4&X+ X=X\\
5&X+ Y=Y+ X\\
6&(X+ Y)+ Z=X+(Y+ Z)\\
7&X(X+Y)=X\\
8&X+XY=X\\
9&XY+XZ=X(Y+Z)\\
10&X(\overline{X}+Y)=XY\\
11&(X+Y)(Y+Z)(\overline{Y}+Z)=(X+Y)Z\\
12&X+YZ=(X+Y)(X+Z)\\
13&\overline{\overline{X}}=X\\
\bottomrule
\end{tabular}
\end{otherlanguage}
\end{subtable}
\end{table}

بوولین مساوات ثابت کرنے کا ایک اہم طریقہ بوولین جدول سے اخذ کرنے کا طریقہ کہلاتا ہے۔آئیں، درج بالا میں سے چند قوانین اس طریقہ سے حاصل کریں۔	


\ابتدا{مثال} 
جدول \حوالہ{جدول_بوولین_دو_پہلو_تفاعل}-الف کی شِق \عددی{1} کو بوولین جدول کی مدد سے ثابت کریں۔

\ترچھا{حل:}\quad 
 اس شِق کے بائیں ہاتھ ، \عددی{X}واحد متغیرہ ہے۔اس کے بوولین جدول میں دو اندراج \عددی{0} اور \عددی{1} ہوں گے، جو ایک ہندسی ثنائی عدد کی تمام ممکنہ قیمتیں ہیں۔
 \begin{center}
 \begin{otherlanguage}{english}
 \begin{tabular}{C}
 \toprule
 X\\
 \midrule
 0\\
 1\\
 \bottomrule
 \end{tabular}
 \end{otherlanguage}
 \end{center} 

اس میں \عددی{0\cdot X} کا خانہ شامل کرتے ہیں، جس میں \عددی{0\cdot 0=0} اور \عددی{0\cdot 1=0} درج ہوں گے۔
 \begin{center}
 \begin{otherlanguage}{english}
 \begin{tabular}{C|C}
 \toprule
 X&0\cdot X\\
 \midrule
 0&0\\
 1&0\\
 \bottomrule
 \end{tabular}
 \end{otherlanguage}
 \end{center} 
اس جدول کی دائیں قطار کہتی ہے کہ \عددی{0\cdot X} ہمیشہ \عددی{0} ہو گا۔ ہم یہی ثابت کرنا چاہتے تھے۔
\انتہا{مثال}

اس طرح کے سوال، جن میں ایک متغیرہ \عددی{X} کو مستقل عدد \عددی{C} سے منطقی ضرب دینا ہو،کی قدم با قدم ترکیب دیکھتے ہیں۔متغیرہ \عددی{X}کے تمام ممکنہ قیمتوں کے جدول میں مستقل \عددی{C} کی قطار شامل کریں۔موجودہ مثال میں مستقل \عددی{0} ہے، لہٰذا \عددی{C} کی قطار میں تمام اندراج کی قیمت \عددی{0} ہو گی۔
 \begin{center}
 \begin{otherlanguage}{english}
 \begin{tabular}{CC}
 \toprule
 C&X\\
 \midrule
 0&0\\
 0&1\\
 \bottomrule
 \end{tabular}
 \end{otherlanguage}
 \end{center}
اب \عددی{0\cdot X} کی قطار شامل کریں۔
 \begin{center}
 \begin{otherlanguage}{english}
 \begin{tabular}{CC|C}
 \toprule
 C&X&C\cdot X\\
 \midrule
 0&0&0\\
 0&1&0\\
 \bottomrule
 \end{tabular}
 \end{otherlanguage}
 \end{center}
ہم دیکھتے ہیں کہ \عددی{C\cdot X} ہمیشہ \عددی{0} ہے، لہٰذا \عددی{0\cdot X=0} ہو گا۔

\ابتدا{مثال}
جدول \حوالہ{جدول_بوولین_دو_پہلو_تفاعل} -الف کی شِق \عددی{2} کو بوولین جدول سے ثابت کریں۔

 \ترچھا{حل:}\quad 
 اس شِق کے بائیں ہاتھ \عددی{X} واحد متغیرہ، جبکہ \عددی{1} مستقل ہے۔متغیرہ کا بوولین جدول لکھتے ہیں؛ ساتھ ہی مستقل \عددی{1} کی قطار بھی شامل کرتے ہیں، جس کے تمام اندراج کی قیمت \عددی{1} ہو گی۔آخر میں \عددی{1\cdot X} کی قطار شامل کرتے ہیں۔
 \begin{center}
 \begin{otherlanguage}{english}
 \begin{tabular}{CC|C}
 \toprule
 1&X&1 \cdot X\\
 \midrule
 1&0&0\\
 1&1&1\\
 \bottomrule
 \end{tabular}\quad \quad\quad
 \begin{tabular}{CC}
 \toprule
 1&X\\
 \midrule
 1&0\\
 1&1\\
 \bottomrule
 \end{tabular}
 \end{otherlanguage}
 \end{center}
ہم دیکھتے ہیں کہ \عددی{1\cdot X} اور \عددی{X} کی مطابقتی قیمتیں ہمیشہ ایک جیسی ہیں، لہٰذا ثابت ہوا کہ \عددی{1\cdot X=X} ہو گا۔
\انتہا{مثال}


\ابتدا{مثال}
 \عددی{X\cdot \overline{X}=0} ثابت کریں۔
 \ترچھا{حل:}\quad
 \begin{center}
 \begin{otherlanguage}{english}
 \begin{tabular}{CC|C}
 \toprule
 X&\overline{X}&X\cdot \overline{X}\\
 \midrule
 0&1&0\\
 1&0&0\\
 \bottomrule
 \end{tabular}
 \end{otherlanguage}
 \end{center}
\انتہا{مثال}
\ابتدا{مثال}
 ثابت کرتے ہیں کہ \عددی{X\cdot X=X} ہے۔اگر \عددی{X=0} ہو تب \عددی{X\cdot X=0\cdot 0=0} ہو گا جو \عددی{X} کے برابر ہے۔ اسی طرح \عددی{X=1} کی صورت میں \عددی{X\cdot X=1\cdot 1=1} ہو گا جو \عددی{X} کے برابر ہے۔ ہن نے دیکھا کہ \عددی{X} کی تمام قیمتوں کے لئے یہ فقرہ درست ہے۔
\انتہا{مثال}
\ابتدا{مثال}	 
فقرہ \عددی{\overline{\overline{X}}=X} ثابت کریں۔
\ترچھا{حل:}
 \begin{center}
 \begin{otherlanguage}{english}
 \begin{tabular}{CC|C}
 \toprule
 X&\overline{X}&\overline{\overline{X}}\\
 \midrule
 0&1&0\\
 1&0&1\\
 \bottomrule
 \end{tabular}
 \end{otherlanguage}
 \end{center}
\انتہا{مثال}
\ابتدا{مثال} 	
ثابت کریں کہ \عددی{(0+X=X)} ہو گا۔
\ترچھا{حل:}
 \begin{center}
 \begin{otherlanguage}{english}
 \begin{tabular}{CC|C}
 \toprule
 0&X&0+X\\
 \midrule
 0&0&0\\
 0&1&1\\
 \bottomrule
 \end{tabular}
 \end{otherlanguage}
 \end{center}
 دائیں دو قطار ایک جیسے ہیں لہٰذا ثبوت پورا ہوا۔
\انتہا{مثال}
\ابتدا{مثال}	 
\عددی{(1+X=1)} ثابت کریں۔
\ترچھا{حل:}
 \begin{center}
 \begin{otherlanguage}{english}
 \begin{tabular}{CC|C}
 \toprule
 1&X&1+X\\
 \midrule
 1&0&1\\
 1&1&1\\
 \bottomrule
 \end{tabular}
 \end{otherlanguage}
 \end{center}
 دائیں دو قطار ایک جیسے ہیں لہٰذا ثبوت پورا ہوتا ہے۔
\انتہا{مثال}
\ابتدا{مثال}
فقرہ \عددی{X+Y=Y+X} ثابت کریں۔
\ترچھا{حل:}
 \begin{center}
 \begin{otherlanguage}{english}
 \begin{tabular}{CC|CC}
 \toprule
 X&Y&X+Y&Y+X\\
 \midrule
 0&0&0&0\\
 0&1&1&1\\
 1&0&1&1\\
 1&1&1&1\\
 \bottomrule
 \end{tabular}
 \end{otherlanguage}
 \end{center}
 دائیں دو قطار ایک جیسے ہیں لہٰذا ثبوت پورا ہوتا ہے۔
\انتہا{مثال}
\ابتدا{مثال}
 ثابت کریں کہ \عددی{X(Y+Z)=XY+XZ} ہو گا۔
 \ترچھا{حل:}
 \begin{center}
 \begin{otherlanguage}{english}
 \begin{tabular}{CCC|CCC|CC}
 \toprule
 X&Y&Z&Y+Z&XY&XZ&X(Y+Z)&XY+XZ\\
 \midrule
 0&0&0&0&0&0&0&0\\
 0&0&1&1&0&0&0&0\\
 0&1&0&1&0&0&0&0\\
 0&1&1&1&0&0&0&0\\
 1&0&0&0&0&0&0&0\\
 1&0&1&1&0&1&1&1\\
 1&1&0&1&1&0&1&1\\
 1&1&1&1&1&1&1&1\\
 \bottomrule
 \end{tabular}
 \end{otherlanguage}
 \end{center}
 دائیں دو قطار ایک جیسے ہیں لہٰذا ثبوت پورا ہوا۔
\انتہا{مثال}
\ابتدا{مثال} 
ثابت کریں \عددی{X+XY=X} ہو گا۔

\ترچھا{حل:}\quad
اس کو بوولین جدول کے بجائے بوولین الجبرا کی مدد سے حل کرتے ہیں۔ ہم مساوات کے بائیں ہاتھ کو \عددی{XZ+XY} لکھ سکتے ہیں جہاں \عددی{Z=1} ہو گا۔ یوں جدول \حوالہ{جدول_بوولین_دو_پہلو_تفاعل}-الف کی شِق \عددی{12} کے تحت درج ذیل ہو گا، جہاں \عددی{Z} کی قیمت \عددی{1} لی گئی ہے۔
\begin{align*}
X+XY=X(1+Y)
\end{align*}
جدول \حوالہ{جدول_بوولین_دو_پہلو_تفاعل}-ب کی شِق \عددی{1} کے تحت \عددی{1+Y=1} ہو گا، لہٰذا درج ذیل لکھا جا سکتا ہے
\begin{align*}
X+XY=X(1+Y)=X\cdot 1=X
\end{align*}
جہاں آخری قدم پر جدول \حوالہ{جدول_بوولین_دو_پہلو_تفاعل}-الف کی شِق \عددی{2} استعمال کی گئی۔
\انتہا{مثال}



جدول \حوالہ{جدول_بوولین_دو_پہلو_تفاعل}-الف کی شِق \عددی{ 5} کو متعدد متغیرات تک وسعت دی جا سکتی ہے۔ تین متغیرات کے لئے درج ذیل ہوں گے۔
\begin{align*}
ABC&=BAC\\
&=BCA\\
&=CBA\\
&=CAB
\end{align*}
 
اس طرح جدول \حوالہ{جدول_بوولین_دو_پہلو_تفاعل}-ب کی شِق \عددی{5} کو بھی دو سے زیادہ متغیرات کے لئے وسعت دی جا سکتی ہے۔ تین متغیرات کے لئے، یہ شِق درج ذیل صورتیں اختیار کرتی ہے۔

\begin{align*}
A+B+C&=B+A+C\\
&=B+C+A\\
&=C+B+A\\
&=C+A+B
\end{align*}
\حصہ{ڈی مارگن کے کلیات}
دو نہایت اہم قوانین جنہیں ڈی مارگن کے کلیات (یا ڈی مارگن کے مسائل) کہتے ہیں مندرجہ ذیل ہیں۔ 
\begin{gather}
 \begin{aligned}
 \overline{X+Y}&=\overline{X}\cdot\overline{Y}\\
 \overline{X\cdot Y}&=\overline{X}+\overline{Y}
 \end{aligned}
 \end{gather}
 ان دو مسائل کو بوولین جدول کی مدد سے ثابت کرتے ہیں۔ڈی مارگن کے پہلے مسئلہ \عددی{ \overline{X+Y}=\overline{X}\cdot\overline{Y}} کا ثبوت درج ذیل ہے۔ 
 \begin{center}
 \begin{otherlanguage}{english}
 \begin{tabular}{CC|CCC|CC}
 \toprule
 X&Y&\overline{X}&\overline{Y}&X+Y&\overline{X+Y}&\overline{X}\cdot\overline{Y}\\
 \midrule
 0&0&1&1&0&1&1\\
 0&1&1&0&1&0&0\\
 1&0&0&1&1&0&0\\
 1&1&0&0&1&0&0\\
 \bottomrule
 \end{tabular}
 \end{otherlanguage}
 \end{center}
 ڈی مارگن کے دوسرے مسئلہ \عددی{ \overline{X\cdot Y}=\overline{X}+\overline{Y}} کا ثبوت درج ذیل ہے۔
 \begin{center}
 \begin{otherlanguage}{english}
 \begin{tabular}{CC|CCC|CC}
 \toprule
 X&Y&\overline{X}&\overline{Y}&X\cdot Y&\overline{X\cdot Y}&\overline{X}+\overline{Y}\\
 \midrule
 0&0&1&1&0&1&1\\
 0&1&1&0&0&1&1\\
 1&0&0&1&0&1&1\\
 1&1&0&0&1&0&0\\
 \bottomrule
 \end{tabular}
 \end{otherlanguage}
 \end{center}

ڈی مارگن کے مسائل منطقی جمع کو منطقی ضرب میں اور منطقی ضرب کو منطقی جمع میں تبدیل کرتے ہیں، اور بوولین تفاعل حل کرنے میں مددگار ثابت ہوتے ہیں۔ 
	
 مثال کے طور پر، جدول \حوالہ{جدول_بوولین_دو_پہلو_تفاعل}-الف کی پہلی شِق \عددی{0\cdot X=0} کا متمم لیتے ہیں۔
\begin{align*}
\overline{0\cdot X}=\overline{0}
\end{align*}
بائیں ہاتھ ڈی مارگن کا دوسرا مسئلہ لاگو کرتے ہیں۔
\begin{align*}
\overline{0}+\overline{X}=\overline{0}
\end{align*}

مزید، چونکہ \عددی{0} کا متمم \عددی{1} ہے، یعنی \عددی{\overline{0}=1} ہو گا، لہٰذا درج ذیل لکھا جا سکتا ہے۔
\begin{align*}
1+\overline{X}=1
\end{align*}
اس مساوات میں \عددی{\overline{X}} کو بوولین متغیرہ \عددی{Z} تصور کیا جا سکتا ہے۔ہوں درج ذیل حاصل ہو گا۔
\begin{align*}
1+Z=1
\end{align*}
اس کا جدول \حوالہ{جدول_بوولین_دو_پہلو_تفاعل}-ب کی شِق \عددی{1} سے موازنہ کریں۔متغیرہ کے نام مختلف ہونے کے علاوہ دونوں یکساں ہیں۔

ڈی مارگن مسائل کی مدد سے ہم نے دیکھا کہ 
\begin{align*}
0\cdot X=0
\end{align*}
اور
\begin{align*}
1+X=1
\end{align*}
در حقیقت ایک ہی تفاعل کے دو پہلو ہیں۔
\begin{align*}
(0\cdot X=0) & \Leftrightarrow (1+X=1)&(\text{\RL{\small{مماثلہ}}})
\end{align*}


اس مسئلہ کو ڈی مارگن کے پہلے مسئلہ کی مدد سے بھی دیکھا جا سکتا ہے۔ایسا کرنے کی خاطر ہم بوولین تفاعل \عددی{1+X=1} کے دونوں اطراف کا متمم لیتے ہیں۔
\begin{align*}
\overline{1+X}=\overline{1}
\end{align*}
بائیں ہاتھ ڈی مارگن کا پہلا مسئلہ لاگو کرتے ہیں۔
\begin{align*}
\overline{1}\cdot \overline{X}=\overline{1}
\end{align*}
اب \عددی{\overline{1}} کی جگہ \عددی{0} ڈالتے ہیں۔
\begin{align*}
0\cdot \overline{X}=0
\end{align*}
یہ مساوات کسی بھی متغیرہ \عددی{\overline{X}} کے لئے درست ہے۔اس متغیرہ کو ہم \عددی{Z}بھی پکار سکتے ہیں۔ ایسا کرنے سے درج ذیل حاصل ہو گا۔
\begin{align*}
0\cdot Z=0
\end{align*}
ہم دیکھتے ہیں کہ یہ بالکل \عددی{0\cdot X=0} کی طرح ہے۔فرق صرف متغیرہ کے نام کا ہے۔لہٰذا ثابت ہوا کہ \عددی{1+X=1} اور \عددی{0\cdot X=0} ایک ہی تفاعل کے دو پہلو ہیں۔

\ابتدا{مثال}
ثابت کریں کہ \عددی{1\cdot X=X} اور \عددی{0+X=X} ایک ہی تفاعل کی دو شکلیں ہیں۔

\ترچھا{حل:}\quad
\عددی{1\cdot X=X}	کے دونوں اطراف کا متمم لیتے ہیں۔
\begin{align*}
\overline{1\cdot X}=\overline{X}
\end{align*}
بائیں ہاتھ ڈی مارگن کا دوسرا قانون لاگو کرتے ہیں
\begin{align*}
\overline{1}+\overline{X}=\overline{X}
\end{align*}
اور \عددی{\overline{1}} کی جگہ \عددی{0} پُر کرتے ہیں۔
\begin{align*}
0+\overline{X}=\overline{X}
\end{align*}
متغیرہ \عددی{\overline{X}} کو نیے نام \عددی{Z} سے پکارتے ہیں۔
\begin{align*}
0+Z=Z
\end{align*}
یہ مساوات کہتی ہے کہ صفر جمع ایک بوولین متغیرہ اس متغیرہ کے برابر ہو گا۔ یوں ثابت ہوا کہ \عددی{1\cdot X=X} اور \عددی{0+X=X} مماثلہ ہیں۔

آپ اسی مثال کو پچھلی مثال کی طرح اُلٹ رخ میں ثابت کریں۔
\انتہا{مثال}

\ابتدا{مثال}
 بوولین تفاعل \عددی{(X\cdot Y)\cdot Z=X\cdot(Y\cdot Z)} کا مماثلہ ڈی مارگن کے قانون لاگو کر کے حاصل کریں۔

\ترچھا{حل:}\quad
دئے گئے تفاعل کے دونوں اطراف کا متمم لیتے ہیں۔
\begin{align*}
\overline{(X\cdot Y)\cdot Z}=\overline{X\cdot(Y\cdot Z)}
\end{align*}
دونوں اطراف ڈی مارگن کا دوسرا قانون لاگو کرتے ہیں۔
\begin{align*}
(\overline{X\cdot Y})+\overline{Z}=\overline{X}+(\overline{Y\cdot Z})
\end{align*}
ڈی مارگن کا قانون استعمال کرتے وقت قوسین میں بند حصہ کو ایک متغیرہ تصور کیا گیا۔دونوں اطراف قوسین میں بند تفاعل پر دوبارہ ڈی مارگن کا دوسرا قانون لاگو کرتے ہیں۔
\begin{align*}
(\overline{X}+\overline{ Y})+\overline{Z}=\overline{X}+(\overline{Y}+\overline{Z})
\end{align*}
یہاں تینوں متغیرات کے متمم لکھے گئے ہیں۔ہم انہیں تین نئے ناموں سے پکار سکتے ہیں، مثلاً، \عددی{\overline{X}}کو \عددی{A} پکارتے ہیں، \عددی{\overline{Y}} کو \عددی{B} اور \عددی{\overline{Z}} کو \عددی{C}، لہٰذا درج ذیل لکھا جائے گا، جو متغیرات کے نام مختلف ہونے کے علاوہ، جدول \حوالہ{جدول_بوولین_دو_پہلو_تفاعل}-ب کی شِق \عددی{6} ہے۔
\begin{align*}
(A+B)+C=A+(B+C)
\end{align*}
\انتہا{مثال}


\حصہ{جڑواں بوولین تفاعل}
گزشتہ حصہ میں دیکھا گیا کہ بوولین تفاعل کے دو پہلو ہوتے ہیں۔یوں کسی بوولین تفاعل کو ثابت کرتے ہی اس کا جڑواں تفاعل فوراً لکھا جا سکتا ہے۔جدول \حوالہ{جدول_بوولین_دو_پہلو_تفاعل}-الف اور ب میں اس طرح کے جڑواں بوولین تفاعل پیش کیے گئے ہیں۔ان جدول میں آخری شِق کے علاوہ ہر شِق ایک تفاعل کے دو پہلو پیش کرتا ہے۔ مثلاً، جدول-الف کی شِق \عددی{7} کا دوسرا پہلو جدول-ب کی شِق \عددی{7} دے گا۔

\حصہ{ارکان ضرب کے مجموعہ کی ترکیب}\شناخت{حصہ_بوولین_ارکان_ضرب_کا_مجموعہ}
منطقی مسئلہ کو بوولین تفاعل کی صورت میں لکھنا مندرجہ ذیل مثال سے با آسانی سمجھا جا سکتا ہے۔

فرض کریں، ایک تفاعل جس کے آزاد متغیرات \عددی{A} اور \عددی{B}، جبکہ تابع متغیرہ \عددی{C} ہے، اس صورت بلند ہوتا ہے جب \عددی{A=0} اور \عددی{B=1} ہو، یا جب \عددی{A=1} اور \عددی{B=1} ہو۔

ان معلومات کو جدول \حوالہ{جدول_بوولین_مثال_تفاعل} میں پیش کیا گیا ہے۔
\begin{table}
\caption{تفاعل کا جدول (برائے حصہ \حوالہ{حصہ_بوولین_ارکان_ضرب_کا_مجموعہ})}
\label{جدول_بوولین_مثال_تفاعل}
\centering
\begin{otherlanguage}{english}
\begin{tabular}{CC|C}
\toprule
A&B&C\\
\midrule
0&0&0\\
0&1&1\\
1&0&0\\
1&1&1\\
\bottomrule
\end{tabular}
\end{otherlanguage}
\end{table}
جدول میں \قول{ارکان ضرب} کی قطار شامل کریں۔اس قطار کے ہر خانے میں اسی صف کے آزاد متغیرہ پست ہونے کی صورت میں متغیرہ کا متمم اور بلند صورت میں متغیرہ بذات خود درج کیا جائے گا۔ اس عمل کو سمجھنے کی خاطر، جدول کی پہلی صف پر توجہ رکھیں۔ یہاں \عددی{A=0} اور \عددی{B=0} ہے، لہٰذا پہلی صف میں رکن ضرب \عددی{\overline{A}\,\overline{B}} ہو گا۔ دوسری صف میں \عددی{A=0} اور \عددی{B=1} ہیں، لہٰذا دوسری صف میں \عددی{\overline{A}B} درج ہو گا۔

\begin{center}
\begin{otherlanguage}{english}
\begin{tabular}{CC|C|C}
\toprule
A&B&C&\text{\RL{ارکان ضرب}}\\
\midrule
0&0&0&\overline{A}\,\overline{B}\\
0&1&1&\overline{A}B\\
1&0&0&A\overline{B}\\
1&1&1&AB\\
\bottomrule
\end{tabular}
\end{otherlanguage}
\end{center}

\موٹا{ تفاعل کے جدول کے ان تمام \قول{ ارکان ضرب} کا مجموعہ لیں جن کی صف میں تابع متغیرہ \عددی{C} کی قیمت \عددی{1} ہو۔یہ مجموعہ تابع متغیرہ کے برابر ہوگا۔} اس طرح تفاعل لکھنے کو ارکان ضرب کے مجموعہ کی ترکیب کہتے ہیں۔(اس کو مجموعہ ارکان ضرب بھی پکار سکتے ہیں۔) 

یوں درج ذیل لکھا جائے گا۔

\begin{align}\label{مساوات_بوولین_ارکان_ضرب_مجموعہ}
C&=\overline{A}B+AB&(\text{\RL{\small{ارکان ضرب کا مجموعہ}}})
\end{align}


\begin{figure}
\centering
\begin{tikzpicture}
\pgfmathsetmacro{\kxstep}{2.5}
\pgfmathsetmacro{\kystep}{1.5}
\draw(0,0)node[and port,scale=1, number inputs=2](u1){} (0,\kystep)node[and port,scale=1, number inputs=2](u2){} (\kxstep,\kystep/2)node[or port,scale=1, number inputs=2](u3){};
\draw(u1.in 1)node[left]{$A$} (u1.in 2)node[left]{$B$} (u2.in 1)node[left]{$\overline{A}$} (u2.in 2)node[left]{$B$}
(u3.out)node[right]{$C=\overline{A}B+AB$};
\draw(u1.out)node[above right]{$AB$}-|(u3.in 2) (u2.out)node[above right]{$\overline{A}B$}-|(u3.in 1);
\end{tikzpicture}
\caption{ارکان ضرب کے مجموعہ (مساوات \حوالہ{مساوات_بوولین_ارکان_ضرب_مجموعہ}) کا منطقی دور۔}
\label{شکل_بوولین_حاصل_تفاعل_کا_دور}
\end{figure}
مساوات \حوالہ{مساوات_بوولین_ارکان_ضرب_مجموعہ} میں حاصل تفاعل کا منطقی دور شکل \حوالہ{شکل_بوولین_حاصل_تفاعل_کا_دور} میں دکھایا گیا ہے۔

ارکان ضرب کے مجموعہ سے حاصل مساوات ہر صورت ضرب گیٹوں کی ایک قطار (یا صف) اور ایک جمع گیٹ سے حاصل کی جا سکتی ہے (جہاں فرض کیا جاتا ہے کہ، آزاد متغیرات کے ساتھ ان کے متمم بھی میسر ہیں)۔ایسا دور \اصطلاح{ضرب و جمع}\فرہنگ{ضرب و جمع }\حاشیہب{AND-OR}\فرہنگ{AND-OR} کہلائے گا۔

 مساوات \حوالہ{مساوات_بوولین_ارکان_ضرب_مجموعہ} اور شکل \حوالہ{شکل_بوولین_حاصل_تفاعل_کا_دور} کی درستگی کی تصدیق بوولین جدول سے کرتے ہیں (جدول میں موازنے کے لئے \عددی{C} کا خانہ بھی پیش کیا گیا ہے)۔
 \begin{center}
\begin{otherlanguage}{english}
\begin{tabular}{CC|C|CCC|C}
\toprule
A&B&C&\overline{A}&\overline{A}B&AB&\overline{A}B+AB\\
\midrule
0&0&0&1&0&0&0\\
0&1&1&1&1&0&1\\
1&0&0&0&0&0&0\\
1&1&1&0&0&1&1\\
\bottomrule
\end{tabular}
\end{otherlanguage}
\end{center}
اس جدول کا دایاں قطار \عددی{C} کے برابر ہے۔


مساوات \حوالہ{مساوات_بوولین_ارکان_ضرب_مجموعہ} لکھنے کا دوسرا انداز جو نہایت مقبول ہے سمجھنے کی خاطر تفاعل کے جدول میں \قول{ارکان ضرب} کے علاوہ ایک نئی قطار \عددی{(m)} شامل کرتے ہیں۔
\begin{center}
\begin{otherlanguage}{english}
\begin{tabular}{CC|C|CC}
\toprule
A&B&C&\text{\RL{ارکان ضرب}}&m\\
\midrule
0&0&0&\overline{A}\,\overline{B}&m_0\\
0&1&1&\overline{A}B&m_1\\
1&0&0&A\overline{B}&m_2\\
1&1&1&AB&m_3\\
\bottomrule
\end{tabular}
\end{otherlanguage}
\end{center}
 نئی قطار میں \عددی{m} ارکان ضرب کو ظاہر کرتا ہے، لہٰذا تفاعل \عددی{C} کی مساوات لکھتے ہوئے \عددی{\overline{A}B} کی بجائے \عددی{m_1} اور \عددی{AB}کی بجائے \عددی{m_3} لکھتے ہیں۔یوں مساوات \حوالہ{مساوات_بوولین_ارکان_ضرب_مجموعہ}سے درج ذیل لکھا جا سکتا ہے۔
 \begin{gather}
\begin{aligned}
C&=\overline{A}B+AB\\
&=m_1+m_3\\
&=\sum (m_1,m_3)\\
&=\sum (1,3)
\end{aligned}
\end{gather}

	
ارکان ضرب روایتاً (چھوٹی لکھائی میں) \عددی{m_x} لکھے جاتے ہیں، جہاں زیر نوشت \عددی{x} جدول میں مطابقتی صف کے آزاد متغیرات کو ثنائی عدد (کے ہندسے) سمجھ کر، برابر کا اعشاری عدد لیا جاتا ہے۔

\ابتدا{مثال}
درج ذیل بوولین جدول سے بوولین تفاعل کی مساوات حاصل کریں۔
\begin{center}
\begin{otherlanguage}{english}
\begin{tabular}{CCC|C}
\toprule
A&B&C&Z\\
\midrule
0&0&0&1\\
0&0&1&0\\
0&1&0&1\\
0&1&1&1\\
1&0&0&0\\
1&0&1&0\\
1&1&0&1\\
1&1&1&1\\
\bottomrule
\end{tabular}
\end{otherlanguage}
\end{center}
\ترچھا{حل:}\quad
 جدول میں \عددی{Z} تابع متغیرہ ہے۔ جدول کی دائیں جانب ارکان ضرب کی قطار شامل کرتے ہیں۔
\begin{center}
\begin{otherlanguage}{english}
\begin{tabular}{CCC|C|CC}
\toprule
A&B&C&Z&\text{\RL{ارکان ضرب}}&m\\
\midrule
0&0&0&1&\overline{A}\,\overline{B}\,\overline{C}&m_0\\
0&0&1&0&\overline{A}\,\overline{B}\, C&m_1\\
0&1&0&1&\overline{A}\, B\,\overline{C}&m_2\\
0&1&1&1&\overline{A}\,B\,C&m_3\\
1&0&0&0&A\,\overline{B}\,\overline{C}&m_4\\
1&0&1&0&A\,\overline{B}\, C&m_5\\
1&1&0&1&A\, B\,\overline{C}&m_6\\
1&1&1&1&A\,B\,C&m_7\\
\bottomrule
\end{tabular}
\end{otherlanguage}
\end{center}


اُن ارکان ضرب کا مجموعہ لیتے ہیں جن کی صف میں تابع متغیرہ کی قیمت \عددی{1} ہے۔
\begin{align*}
Z=\overline{A}\,\overline{B}\,\overline{C}+\overline{A}\,B\,\overline{C}+\overline{A}\,B\,C+A\,B\,\overline{C}+A\,B\,C
\end{align*}
یہ دیے گئے تفاعل کی مساوات ہے جس کو درج ذیل بھی لکھا جا سکتا ہے۔
\begin{align*}
Z=\sum(m_0,m_2,m_3,m_6,m_7)
\end{align*}
جدول \حوالہ{جدول_بوولین_دو_پہلو_تفاعل} میں دیے گئے قوانین استعمال کرتے ہوئے مساوات کی سادہ صورت حاصل کرتے ہیں۔
\begin{align*}
Z&=\overline{A}\,\overline{B}\,\overline{C}+\overline{A}\,B\,\overline{C}+\overline{A}\,B\,C+A\,B\,\overline{C}+A\,B\,C\\
&=\overline{A}(\overline{B}+B)\overline{C}+\overline{A}BC+AB(\overline{C}+C)\\
&=\overline{A}(1)\overline{C}+\overline{A}BC+AB(1)\\
&=\overline{A}(\overline{C}+BC)+AB\\
&=\overline{A}(\overline{C}+B)+AB\\
&=\overline{A}\,\overline{C}+\overline{A}B+AB\\
&=\overline{A}\,\overline{C}+(\overline{A}+A)B\\
&=\overline{A}\,\overline{C}+B
\end{align*}
یہ دیے گئے بوولین جدول کی سادہ ترین مساوات ہے۔اس کا بوولین جدول لکھ کر آپ ثابت کر سکتے ہیں کہ یہ اصل تفاعل ہی ہے۔
\انتہا{مثال}


\حصہ{ارکان جمع کی ضرب کی ترکیب}
گزشتہ حصہ میں بوولین جدول سے تفاعل کا مساواتی روپ حاصل کیا گیا، جہاں ان صفوں کے ارکان ضرب کا مجموعہ لیا گیا جن میں تابع متغیرات کی قیمت \عددی{1} تھی۔آئیں اب \قول{ ارکان جمع} لکھنا اور ان سے تفاعل کی مساوات حاصل کرنا سیکھیں۔


 حصہ\حوالہ{حصہ_بوولین_ارکان_ضرب_کا_مجموعہ} میں مستمل جدول \حوالہ{جدول_بوولین_مثال_تفاعل} کو مثال بناتے ہوئے اس میں ارکان ضرب کی بجائے ارکان جمع کی قطار شامل کرتے ہیں۔ارکان جمع لکھتے ہوئے، مطابقتی آزاد متغیرہ پست ہونے کی صورت میں متغیرہ بذات خود اور بلند صورت میں متغیرہ کا متمم جمع کیا جاتا ہے۔ اس عمل کو سمجھنے کی خاطر، جدول کی پہلی صف پر توجہ رکھیں۔ یہاں \عددی{A=0} اور \عددی{B=0} ہے، لہٰذا پہلی صف میں رکن جمع \عددی{A+B} ہو گا۔ دوسری صف میں \عددی{A=0} اور \عددی{B=1} ہیں، لہٰذا دوسری صف 
میں \عددی{A+\overline{B}} درج ہو گا۔
\begin{center}
\begin{otherlanguage}{english}
\begin{tabular}{CC|C|C}
\toprule
A&B&C&\text{\RL{ارکان جمع}}\\
\midrule
0&0&0&A+B\\
0&1&1&A+\overline{B}\\
1&0&0&\overline{A}+B\\
1&1&1&\overline{A}+\overline{B}\\
\bottomrule
\end{tabular}
\end{otherlanguage}
\end{center}

\موٹا{تفاعل کے جدول کے ان تمام \قول{ ارکان جمع} کا حاصل ضرب لیں جن کی صف میں تفاعل کے تابع متغیرہ \عددی{C} کی قیمت \عددی{0} ہو۔} یہ حاصل ضرب تابع متغیرہ کے برابر ہوگا۔ اس طرح تفاعل لکھنے کو ارکان جمع کی ضرب کی ترکیب کہتے ہیں (اس کو ضرب بعد از جمع بھی پکار سکتے ہیں)۔

یوں درج ذیل لکھا جائے گا۔
\begin{align}\label{مساوات_بوولین_ارکان_جمع_ضرب}
C&=(A+B)(\overline{A}+B)(\text{\RL{\small{ارکان جمع کی ضرب}}})
\end{align}
 ارکان جمع کی ضرب سے حاصل مساوات کو ہر صورت جمع گیٹوں کی ایک قطار (یا صف) اور ایک ضرب گیٹ سے حاصل کیا جا سکتا ہے (جہاں فرض کیا جاتا ہے کہ، آزاد متغیرات کے ساتھ ان کے متمم بھی میسر ہیں)۔ یوں بنائے گئے دور کو \اصطلاح{جمع و ضرب}\فرہنگ{جمع و ضرب}\حاشیہب{OR-AND}\فرہنگ{OR-AND} کہتے ہیں۔ 

مساوات \حوالہ{مساوات_بوولین_ارکان_جمع_ضرب} میں حاصل دور شکل \حوالہ{شکل_بوولین_ارکان_جمع_کی_ضرب_مثال} میں پیش کیا گیا ہے۔
\begin{figure}
\centering
\begin{tikzpicture}
\pgfmathsetmacro{\kxstep}{3}
\pgfmathsetmacro{\kystep}{1.5}
\draw(0,0)node[or port,scale=1, number inputs=2](u1){} (0,\kystep)node[or port,scale=1, number inputs=2](u2){} (\kxstep,\kystep/2)node[and port,scale=1, number inputs=2](u3){};
\draw(u1.in 1)node[left]{$A$} (u1.in 2)node[left]{$B$} (u2.in 1)node[left]{$\overline{A}$} (u2.in 2)node[left]{$B$}
(u3.out)node[right]{$C=(\overline{A}+B)(A+B)$};
\draw(u1.out)node[above right]{$A+B$}-|(u3.in 2) (u2.out)node[above right]{$\overline{A}+B$}-|(u3.in 1);
\end{tikzpicture}
\caption{ارکان جمع کی ضرب سے حاصل دور (مساوات \حوالہ{مساوات_بوولین_ارکان_جمع_ضرب})۔}
\label{شکل_بوولین_ارکان_جمع_کی_ضرب_مثال}
\end{figure}

مساوات \حوالہ{مساوات_بوولین_ارکان_جمع_ضرب} لکھنے کا دوسرا انداز جو نہایت مقبول ہے سمجھنے کی خاطر تفاعل کے جدول میں \قول{ارکان جمع} کے علاوہ، بڑی لکھائی میں ایک نئی قطار \عددی{(M)} شامل کرتے ہیں، جو ارکان جمع کو ظاہر کرتا ہے۔
\begin{center}
\begin{otherlanguage}{english}
\begin{tabular}{CC|C|CC}
\toprule
A&B&C&\text{\RL{ارکان جمع}}&M\\
\midrule
0&0&0&\overline{A}\,\overline{B}&M_0\\
0&1&1&\overline{A}B&M_1\\
1&0&0&A\overline{B}&M_2\\
1&1&1&AB&M_3\\
\bottomrule
\end{tabular}
\end{otherlanguage}
\end{center}
یوں مساوات \حوالہ{مساوات_بوولین_ارکان_جمع_ضرب} درج ذیل روپ اختیار کرتی ہے۔
\begin{align}
C&=(A+B)(\overline{A}+B)=M_0 M_2=\prod (M_0, M_2)=\prod (0,2)
\end{align}


\ابتدا{مثال}
 ڈی مارگن کے کلیات استعمال کرتے ہوئے مجموعہ ارکان ضرب سے ارکان جمع کی ضرب کی ترکیب حاصل کریں۔
 
\ترچھا{حل:}\quad
ہم حصہ \حوالہ{حصہ_بوولین_ارکان_ضرب_کا_مجموعہ} میں مستعمل جدول \حوالہ{جدول_بوولین_مثال_تفاعل} کو مثال بنا کر اس میں \عددی{\overline{C}} اور ارکان ضرب کی قطاریں شامل کرتے ہیں۔
\begin{center}
\begin{otherlanguage}{english}
\begin{tabular}{CC|CC|C}
\toprule
A&B&C&\overline{C}&\text{\RL{ارکان ضرب}}\\
\midrule
0&0&0&1&\overline{A}\,\overline{B}\\
0&1&1&0&\overline{A}B\\
1&0&0&1&A\overline{B}\\
1&1&1&0&AB\\
\bottomrule
\end{tabular}
\end{otherlanguage}
\end{center}
 ہم \عددی{\overline{C}} کے لئے ارکان ضرب کا مجموعہ لکھ کر (یعنی ان ارکان ضرب کا مجموعہ جن کے صف میں \عددی{\overline{C}} کی قیمت \عددی{1} ہو):
\begin{align*}
\overline{C}=\overline{A}\,\overline{B}+A\,\overline{B}
\end{align*}
دونوں اطراف کا متمم لے کر \عددی{C} کی مساوات حاصل کرتے ہیں۔
\begin{align*}
\overline{\overline{C}}=C=\overline{\overline{A}\,\overline{B}+A\,\overline{B}}
\end{align*}
ڈی مارگن کلیات بار بار استعمال کرتے ہوئے درج ذیل حاصل کیا جا سکتا ہے۔
\begin{gather}
\begin{aligned}
C&=\overline{\overline{A}\,\overline{B}+A\,\overline{B}}\\
&=(\overline{\overline{A}\,\overline{B}})(\overline{A\,\overline{B}})\\
&=(\overline{\overline{A}}+\overline{\overline{B}})(\overline{A}+\overline{\overline{B}})\\
&=(A+B)(\overline{A}+B)
\end{aligned}
\end{gather}
اس نتیجے کا مساوات \حوالہ{مساوات_بوولین_ارکان_جمع_ضرب} کے ساتھ موازنہ کریں۔پس ثابت ہوا کہ مجموعہ ارکان ضرب سے ارکان جمع کی ضرب حاصل کی جا سکتی ہے۔
\انتہا{مثال}
\ابتدا{مثال} \شناخت{مثال_بوولین_دونوں_انداز_ادوار}
درج ذیل بوولین جدول سے (ا) ارکان جمع کی ضرب، (ب) ارکان ضرب کا مجموعہ لے کر تفاعل کی مساوات حاصل کریں۔دونوں نتائج کے ادوار دکھائیں۔
\begin{center}
\begin{otherlanguage}{english}
\begin{tabular}{CCC|C}
\toprule
A&B&C&Z\\
\midrule
0&0&0&0\\
0&0&1&1\\
0&1&0&1\\
0&1&1&0\\
1&0&0&0\\
1&0&1&1\\
1&1&0&1\\
1&1&1&1\\
\bottomrule
\end{tabular}
\end{otherlanguage}
\end{center}

\ترچھا{حل:}\quad
 جدول میں ارکان جمع اور ارکان ضرب کی قطاریں شامل کرتے ہیں۔
\begin{center}
\begin{otherlanguage}{english}
\begin{tabular}{CCC|C|CC}
\toprule
A&B&C&Z&\text{\RL{ارکان جمع}}&\text{\RL{ارکان ضرب}}\\
\midrule
0&0&0&0&A+B+C&\overline{A}\,\overline{B}\,\overline{C}\\
0&0&1&1&A+B+\overline{C}&\overline{A}\,\overline{B}\,C\\
0&1&0&1&A+\overline{B}+C&\overline{A}\,B\,\overline{C}\\
0&1&1&0&A+\overline{B}+\overline{C}&\overline{A}\,B\,C\\
1&0&0&0&\overline{A}+B+C&A\,\overline{B}\,\overline{C}\\
1&0&1&1&\overline{A}+B+\overline{C}&A\,\overline{B}\,C\\
1&1&0&1&\overline{A}+\overline{B}+C&A\,B\,\overline{C}\\
1&1&1&1&\overline{A}+\overline{B}+\overline{C}&A\,B\,C\\
\bottomrule
\end{tabular}
\end{otherlanguage}
\end{center}
(ا) جن صفوں میں تابع متغیرہ \عددی{Z} کی قیمت \عددی{0} ہے ان صفوں کے ارکان جمع کی ضرب مطلوبہ نتیجہ ہو گا۔
\begin{align}\label{مساوات_بوولین_مثال_جمع_کی_ضرب}
Z=(A+B+C)(A+\overline{B}+\overline{C})(\overline{A}+B+C)
\end{align}
اس کو درج ذیل بھی لکھ سکتے ہیں۔
\begin{align*}
Z=M_0 M_3 M_4=\prod (M_0, M_3, M_4)
\end{align*}
مساوات \حوالہ{مساوات_بوولین_مثال_جمع_کی_ضرب} میں حاصل نتیجہ کا جمع و ضرب دور شکل \حوالہ{شکل_بوولین_مثال_جمع_کی_ضرب} میں پیش کیا گیا ہے۔
\begin{figure}
\centering
\begin{tikzpicture}
\pgfmathsetmacro{\kxstep}{4}
\pgfmathsetmacro{\kystep}{1.5}
\draw(0,0)node[or port, scale=1, number inputs=3](u1){} (0,\kystep)node[or port, scale=1, number inputs=3](u2){}
(0,2*\kystep)node[or port, scale=1, number inputs=3](u3){} 
(\kxstep,\kystep)node[and port, scale=1, number inputs=3](u4){};
\draw(u1.in 1)node[left]{$\overline{A}$} (u1.in 2)node[left]{$B$} (u1.in 3)node[left]{$C$} (u1.out)node[above right]{$\overline{A}+B+C$}-|(u4.in 3);
\draw(u2.in 1)node[left]{$A$} (u2.in 2)node[left]{$\overline{B}$} (u2.in 3)node[left]{$\overline{C}$} (u2.out)node[above right]{$A+\overline{B}+\overline{C}$}-|(u4.in 2);
\draw(u3.in 1)node[left]{$A$} (u3.in 2)node[left]{$B$} (u3.in 3)node[left]{$C$} (u3.out)node[above right]{$A+B+C$}-|(u4.in 1);
\draw(u4.out)node[right]{\small$Z=(A+B+C)(A+\overline{B}+\overline{C})(\overline{A}+B+C)$};
\end{tikzpicture}
\caption{جمع و ضرب دور (مساوات \حوالہ{مساوات_بوولین_مثال_جمع_کی_ضرب})۔}
\label{شکل_بوولین_مثال_جمع_کی_ضرب}
\end{figure}
(ب) جدول کے ارکان ضرب کا مجموعہ لے کر ضرب و جمع دور حاصل کرتے ہیں۔
\begin{align}\label{مساوات_بوولین_مثال_ضرب_جمع}
Z=\overline{A}\,\overline{B}\,C+\overline{A}B\overline{C}+A\overline{B}C+AB\overline{C}+ABC
\end{align}
اس دور کو شکل \حوالہ{شکل_بوولین_مثال_ضرب_کی_جمع} میں پیش کیا گیا ہے۔
\begin{figure}
\centering
\begin{tikzpicture}
\pgfmathsetmacro{\kxstep}{4}
\pgfmathsetmacro{\kystep}{1.5}
\draw(0,0)node[and port, scale=1, number inputs=3](u1){} (0,\kystep)node[and port, scale=1, number inputs=3](u2){}
(0,2*\kystep)node[and port, scale=1, number inputs=3](u3){} (0,3*\kystep)node[and port, scale=1, number inputs=3](u4){}
(0,4*\kystep)node[and port, scale=1, number inputs=3](u5){}
 (\kxstep,2*\kystep)node[or port, scale=1, number inputs=5](u6){};
\draw(u1.in 1)node[left]{$A$} (u1.in 2)node[left]{$B$} (u1.in 3)node[left]{$C$}
(u6.in 5)|- (u1.out)node[above right]{$ABC$};
 \draw(u2.in 1)node[left]{$A$} (u2.in 2)node[left]{$B$} (u2.in 3)node[left]{$\overline{C}$}
 (u6.in 4)--++(-0.5,0)|-(u2.out)node[above right]{$AB\overline{C}$};
 \draw(u3.in 1)node[left]{$A$} (u3.in 2)node[left]{$\overline{B}$} (u3.in 3)node[left]{$C$}
 (u6.in 3)|-(u3.out)node[above right]{$A\overline{B}C$};
 \draw(u4.in 1)node[left]{$\overline{A}$} (u4.in 2)node[left]{$B$} (u4.in 3)node[left]{$\overline{C}$}
 (u6.in 2)--++(-0.5,0)|-(u4.out)node[above right]{$\overline{A}B\overline{C}$};
 \draw(u5.in 1)node[left]{$\overline{A}$} (u5.in 2)node[left]{$\overline{B}$} (u5.in 3)node[left]{$C$}
 (u6.in 1)|-(u5.out)node[above right]{$\overline{A}\,\overline{B}C$};
 \draw(u6.out)node[right]{\small $Z=\overline{A}\,\overline{B}C+\overline{A}B\overline{C} + A\overline{B}C+AB\overline{C} +ABC$};
\end{tikzpicture}
\caption{ ضرب و جمع دور (مساوات \حوالہ{مساوات_بوولین_مثال_ضرب_جمع})۔}
\label{شکل_بوولین_مثال_ضرب_کی_جمع}
\end{figure}
 \انتہا{مثال}


اس مثال میں ایک ہی تفاعل کے دو ادوار، شکل \حوالہ{شکل_بوولین_مثال_جمع_کی_ضرب}اور شکل\حوالہ{شکل_بوولین_مثال_ضرب_کی_جمع} پیش کیے گئے۔پہلے دور میں تین جمع اور ایک ضرب گیٹ استعمال ہوا، جبکہ دوسرے میں پانچ ضرب اور ایک جمع گیٹ استعمال ہوا۔ (جیسا ہم ذکر کر چکے ہیں، ارکان جمع کی ضرب سے حاصل دور جمع گیٹوں کی قطار اور ایک ضرب گیٹ سے بنے گا۔ ارکان ضرب کے مجموعہ سے حاصل دور ضرب گیٹوں کی قطار اور ایک جمع گیٹ سے حاصل ہو گا۔) یوں اس تفاعل کو  ضرب بعد از جمع سے حاصل کرنے میں کم منطقی گیٹ استعمال ہوئے۔یاد رہے کہ  ضرب بعد از جمع اور مجموعہ ارکان ضرب منطقی طور پر ایک ہیں۔

\حصہ{مجموعہ ارکان ضرب اور  ضرب بعد از جمع کے مابین تبادلہ}
ہم نے مثال \حوالہ{مثال_بوولین_دونوں_انداز_ادوار} میں تفاعل کی مساوات، مجموعہ ارکان ضرب اور  ضرب بعد از جمع کی  شکل میں حاصل کی، جنہیں یہاں دوبارہ پیش کرتے ہیں۔
\begin{align*}
Z&=m_1+m_2+m_5+m_6+m_7=\sum(1,2,5,6,7)\\
Z&=M_0 M_3 M_4=\prod (0,3,4)
\end{align*}
مجموعہ ارکان ضرب میں پہلا، دوسرا،پانچواں، چھٹا اور ساتواں رکن ضرب استعمال ہوا جبکہ صفرواں، تیسرا اور چوتھا رکن غیر مستعمل رہے ۔ ضرب بعد از جمع میں پہلا، دوسرا،پانچواں،چھٹا اور ساتواں رکن جمع غیر مستعمل، جبکہ صفرواں، تیسرا اور چوتھا رکن استعمال ہوا۔یہ ایک عمومی حقیقت ہے جسے استعمال کر کے تفاعل کی مساوات کو ایک روپ سے دوسرے روپ میں تبدیل کیا جاتا ہے۔ارکان جمع سے ارکان ضرب یا ارکان ضرب سے ارکان جمع کے روپ میں مساوات حاصل کرتے ہوئے پہلے روپ میں غیر مستعمل ارکان، دوسرے روپ میں استعمال ہوں گے۔ 

\حصہ{ضرب و جمع دور سے متمم ضرب و متمم ضرب دور کا حصول}
کسی بھی بوولین تفاعل کو مجموعہ ارکان ضرب کی صورت میں بیان کیا جا سکتا ہے، جس کو ضرب گیٹوں کی قطار اور ایک جمع گیٹ سے حاصل کیا جا سکتا ہے۔شکل \حوالہ{شکل_بوولین_متمم_ضرب_و_متمم_ضرب}-الف میں تفاعل \عددی{AB+CD} کا مجموعہ ارکان ضرب دور دکھایا گیا ہے۔ جمع گیٹ \عددی{u3} کی جگہ شکل \حوالہ{شکل_بوولین_ضرب_متمم_سے_جمع_ضرب}-الف کا مساوی دور نصب کرتے ہوئے شکل-ب حاصل ہو گا (جہاں \عددی{u3} کی جگہ \عددی{u4}، \عددی{u5} اور \عددی{u6} استعمال کیے گئے)۔شکل \حوالہ{شکل_بوولین_نفی_حصول} میں متمم ضرب گیٹ بطور نفی گیٹ دکھایا گیا ہے۔یوں ضرب گیٹ (مثلاً \عددی{u1}) اور نفی گیٹ (مثلاً \عددی{u4} جس کو نفی گیٹ تصور کرتے ہیں) کی جگہ (شکل \حوالہ{شکل_بوولین_ضرب_متمم} دیکھیں) متمم ضرب گیٹ (مثلاً \عددی{u7}) استعمال کرتے ہوئے شکل-ج حاصل ہو گا، جو صرف متمم ضرب گیٹوں پر مشتمل ہے؛ یہ \اصطلاح{متمم ضرب و متمم ضرب}\فرہنگ{متمم ضرب و متمم ضرب}\حاشیہب{NAND-NAND}\فرہنگ{NAND-NAND} دور کہلاتا ہے۔ 

\begin{figure}
\centering
\begin{subfigure}{0.4\textwidth}
\centering
\begin{tikzpicture}
\pgfmathsetmacro{\kxstep}{2}
\pgfmathsetmacro{\kystep}{1.5}
\draw(0,0)node[and port,scale=1,number inputs=2](u1){$u1$} (0,-\kystep)node[and port,scale=1,number inputs=2](u2){$u2$}(\kxstep,-\kystep/2)node[or port,scale=1,number inputs=2](u3){$u3$};
\draw(u1.in 1)node[left]{$A$} (u1.in 2)node[left]{$B$};
\draw(u2.in 1)node[left]{$C$} (u2.in 2)node[left]{$D$};
\draw(u1.out)-|(u3.in 1) (u2.out)-|(u3.in 2);
\end{tikzpicture}
\caption{}
\end{subfigure}%
\begin{subfigure}{0.6\textwidth}
\centering
\begin{tikzpicture}
\pgfmathsetmacro{\kxstep}{2}
\pgfmathsetmacro{\kystep}{1.5}
\draw(0,0)node[and port,scale=1,number inputs=2](u1){$u1$} (0,-\kystep)node[and port,scale=1,number inputs=2](u2){$u2$}(\kxstep,0)node[nand port,scale=1,number inputs=2](u4){$u4$} (\kxstep,-\kystep)node[nand port,scale=1,number inputs=2](u5){$u5$} (2*\kxstep,-\kystep/2)node[nand port,scale=1,number inputs=2](u6){$u6$};
\draw(u1.in 1)node[left]{$A$} (u1.in 2)node[left]{$B$};
\draw(u2.in 1)node[left]{$C$} (u2.in 2)node[left]{$D$};
\draw(u4.out)-|(u6.in 1) (u5.out)-|(u6.in 2);
\draw(u4.in 1)--(u4.in 2)coordinate[pos=0.5](kina);
\draw(u5.in 1)--(u5.in 2)coordinate[pos=0.5](kinb);
\draw(u1.out)--(kina);
\draw(u2.out)--(kinb);
\end{tikzpicture}
\caption{}
\end{subfigure}
\begin{subfigure}{1\textwidth}
\centering
\begin{tikzpicture}
\pgfmathsetmacro{\kxstep}{2}
\pgfmathsetmacro{\kystep}{1.5}
\draw(0,0)node[nand port,scale=1,number inputs=2](u7){$u7$} (0,-\kystep)node[nand port,scale=1,number inputs=2](u8){$u8$}(\kxstep,-\kystep/2)node[nand port,scale=1,number inputs=2](u6){$u6$};
\draw(u7.in 1)node[left]{$A$} (u7.in 2)node[left]{$B$};
\draw(u8.in 1)node[left]{$C$} (u8.in 2)node[left]{$D$};
\draw(u7.out)-|(u6.in 1) (u8.out)-|(u6.in 2);
\end{tikzpicture}
\caption{}
\end{subfigure}
\caption{ارکان ضرب کے مجموعہ سے متمم ضرب و متمم ضرب دور کا حصول۔}
\label{شکل_بوولین_متمم_ضرب_و_متمم_ضرب}
\end{figure}

آپ نے دیکھا کہ شکل \حوالہ{شکل_بوولین_متمم_ضرب_و_متمم_ضرب}-الف کے ضرب و جمع دور میں تمام گیٹ تبدیل کر کے متمم ضرب گیٹ نسب کرنے سے شکل-ج کا متمم ضرب و متمم ضرب دور حاصل ہو گا۔ یہ ایک اہم اور عمومی مشاہدہ ہے۔ یاد رہے کہ مجموعہ ارکان ضرب کے ضرب و جمع دور میں ضرب گیٹوں کی قطار اور ایک جمع گیٹ ہو گا۔


\موٹا{ضرب و جمع دور کی شکل و صورت تبدیل کیے بغیر تمام گیٹوں کی جگہ متمم ضرب گیٹ نسب کرنے سے متمم ضرب و متمم ضرب دور حاصل ہو گا۔} 

 سلیکان کی فی مربع سنٹی میٹر پتری پر بہت بڑی تعداد میں گیٹ بنائے جا سکتے ہیں اور یہ تعداد دن با دن بڑھتی چلی جا رہی ہے۔ سلیکان کی پتری پر ایک ہی قسم کے گیٹ نسبتاً زیادہ آسانی اور بہتر بنائے جا سکتے ہیں۔یوں کسی بھی تفاعل کو ضرب و جمع کی بجائے متمم ضرب و متمم ضرب دور سے حاصل کرنا زیادہ سود مند ثابت ہو گا۔اسی وجہ سے وسیع پیمانہ کی مخلوط برقیات میں متمم ضرب گیٹ نہایت مقبول ہیں۔


\ابتدا{مثال}\شناخت{مثال_بوولین_متمم_ضرب_حصول}
 مندرجہ ذیل تفاعل کا متمم ضرب و متمم ضرب دور حاصل کریں۔
\begin{center}
\begin{otherlanguage}{english}
\begin{tabular}{CC|C}
\toprule
A&B&Z\\
\midrule
0&0&1\\
0&1&0\\
1&0&1\\
1&1&1\\
\bottomrule
\end{tabular}
\end{otherlanguage}
\end{center}

\ترچھا{حل:}\quad
 تفاعل کا مجموعہ ارکان ضرب لکھنے کی غرض سے جدول میں ارکان ضرب کی قطار شامل کرتے ہیں۔
\begin{center}
\begin{otherlanguage}{english}
\begin{tabular}{CC|C|C}
\toprule
A&B&Z&\text{\RL{ارکان ضرب}}\\
\midrule
0&0&1&\overline{A}\, \overline{B}\\
0&1&0&\overline{A}\, B\\
1&0&1&A\,\overline{B}\\
1&1&1&A\, B\\
\bottomrule
\end{tabular}
\end{otherlanguage}
\end{center}
یوں \عددی{Z=\overline{A}\,\overline{B}+A\overline{B}+AB} ہو گا، جو شکل \حوالہ{شکل_بوولین_مثال_متمم_ضرب_حصول}-الف میں پیش ہے۔ تمام گیٹوں کی جگہ متمم ضرب گیٹ نصب کرنے سے متمم ضرب و متمم ضرب دور حاصل ہو گا جو شکل-ب میں پیش ہے۔
\begin{figure}
\centering
\begin{subfigure}{0.45\textwidth}
\centering
\begin{tikzpicture}
\pgfmathsetmacro{\kxstep}{2}
\pgfmathsetmacro{\kystep}{1.5}
\draw(0,0)node[and port,scale=1,number inputs=2](u1){} 
(0,-\kystep)node[and port,scale=1,number inputs=2](u2){}
(0,-2*\kystep)node[and port,scale=1,number inputs=2](u3){}
(\kxstep,-\kystep)node[or port,scale=1,number inputs=3](u4){};
\draw(u1.in 1)node[left]{$\overline{A}$} (u1.in 2)node[left]{$\overline{B}$};
\draw(u2.in 1)node[left]{$A$} (u2.in 2)node[left]{$\overline{B}$};
\draw(u3.in 1)node[left]{$A$} (u3.in 2)node[left]{$B$};
\draw(u1.out)-|(u4.in 1) (u2.out)-|(u4.in 2) (u3.out)-|(u4.in 3) (u4.out)node[right]{$Z$};
\end{tikzpicture}
\caption{}
\end{subfigure}\hfill
\begin{subfigure}{0.45\textwidth}
\centering
\begin{tikzpicture}
\pgfmathsetmacro{\kxstep}{2}
\pgfmathsetmacro{\kystep}{1.5}
\draw(0,0)node[nand port,scale=1,number inputs=2](u1){} 
(0,-\kystep)node[nand port,scale=1,number inputs=2](u2){}
(0,-2*\kystep)node[nand port,scale=1,number inputs=2](u3){}
(\kxstep,-\kystep)node[nand port,scale=1,number inputs=3](u4){};
\draw(u1.in 1)node[left]{$\overline{A}$} (u1.in 2)node[left]{$\overline{B}$};
\draw(u2.in 1)node[left]{$A$} (u2.in 2)node[left]{$\overline{B}$};
\draw(u3.in 1)node[left]{$A$} (u3.in 2)node[left]{$B$};
\draw(u1.out)-|(u4.in 1) (u2.out)-|(u4.in 2) (u3.out)-|(u4.in 3) (u4.out)node[right]{$Z$};
\end{tikzpicture}
\caption{}
\end{subfigure}
\caption{ضرب و جمع سے متمم ضرب و متمم ضرب (مثال \حوالہ{مثال_بوولین_متمم_ضرب_حصول})۔}
\label{شکل_بوولین_مثال_متمم_ضرب_حصول}
\end{figure}
\انتہا{مثال}



\حصہ{جمع و ضرب دور سے متمم جمع و متمم جمع دور کا حصول}
 تفاعل کے ارکان جمع کی ضرب سے حاصل جمع و ضرب دور میں تمام گیٹوں کی جگہ متمم جمع گیٹ نصب کرنے سے تفاعل کا متمم جمع و متمم جمع دور حاصل ہو گا۔
 
\begin{figure}
\centering
\begin{subfigure}{0.4\textwidth}
\centering
\begin{tikzpicture}
\pgfmathsetmacro{\kxstep}{2}
\pgfmathsetmacro{\kystep}{1.5}
\draw(0,0)node[or port,scale=1,number inputs=2](u1){$u1$} (0,-\kystep)node[or port,scale=1,number inputs=2](u2){$u2$}(\kxstep,-\kystep/2)node[and port,scale=1,number inputs=2](u3){$u3$};
\draw(u1.out)-|(u3.in 1) (u2.out)-|(u3.in 2);
\end{tikzpicture}
\caption{}
\end{subfigure}%
\begin{subfigure}{0.6\textwidth}
\centering
\begin{tikzpicture}
\pgfmathsetmacro{\kxstep}{2}
\pgfmathsetmacro{\kystep}{1.5}
\draw(0,0)node[or port,scale=1,number inputs=2](u1){$u1$} (0,-\kystep)node[or port,scale=1,number inputs=2](u2){$u2$}(\kxstep,0)node[nor port,scale=1,number inputs=2](u4){$u4$} (\kxstep,-\kystep)node[nor port,scale=1,number inputs=2](u5){$u5$} (2*\kxstep,-\kystep/2)node[nor port,scale=1,number inputs=2](u6){$u6$};
\draw(u4.out)-|(u6.in 1) (u5.out)-|(u6.in 2);
\draw(u4.in 1)--(u4.in 2)coordinate[pos=0.5](kina);
\draw(u5.in 1)--(u5.in 2)coordinate[pos=0.5](kinb);
\draw(u1.out)--(kina);
\draw(u2.out)--(kinb);
\end{tikzpicture}
\caption{}
\end{subfigure}
\begin{subfigure}{1\textwidth}
\centering
\begin{tikzpicture}
\pgfmathsetmacro{\kxstep}{2}
\pgfmathsetmacro{\kystep}{1.5}
\draw(0,0)node[nor port,scale=1,number inputs=2](u7){$u7$} (0,-\kystep)node[nor port,scale=1,number inputs=2](u8){$u8$}(\kxstep,-\kystep/2)node[nor port,scale=1,number inputs=2](u6){$u6$};
\draw(u7.out)-|(u6.in 1) (u8.out)-|(u6.in 2);
\end{tikzpicture}
\caption{}
\end{subfigure}
\caption{جمع و ضرب سے متمم جمع و متمم جمع۔}
\label{شکل_بوولین_مثال_متمم_جمع_حصول}
\end{figure}
شکل \حوالہ{شکل_بوولین_مثال_متمم_جمع_حصول} میں جمع و ضرب دور سے قدم با قدم متمم جمع و متمم جمع دور کا حصول دکھایا گیا ہے۔پہلی قدم میں، شکل-الف کے ضرب گیٹ \عددی{u3} کی جگہ (شکل \حوالہ{شکل_بوولین_متمم_سے_جمع_ضرب}-الف کی مدد سے) مساوی جمع متمم گیٹ \عددی{u4}، \عددی{u5}، \عددی{u6} نسب کیے گئے۔ اس کے بعد (شکل \حوالہ{شکل_بوولین_نفی_حصول} کی مدد سے) \عددی{u4}، اور \عددی{u5} کو نفی گیٹ مان کر، \عددی{u1} اور \عددی{u4} جوڑی کی جگہ متمم جمع \عددی{u7} جبکہ ، \عددی{u2} اور \عددی{u5} جوڑی کی جگہ متمم جمع \عددی{u8} نسب کیا گیا۔ یوں شکل \حوالہ{شکل_بوولین_مثال_متمم_جمع_حصول}-ج کا متمم جمع و متمم جمع دور حاصل کیا گیا۔

شکل \حوالہ{شکل_بوولین_مثال_متمم_جمع_حصول}-الف کے جمع و ضرب دور کی شکل و صورت تبدیل کیے بغیر تمام گیٹ کی جگہ متمم جمع نسب کرنے سے شکل-ج حاصل ہو گا۔ یہ ایک اہم اور عمومی مشاہدہ ہے۔ یاد رہے کہ ضرب ارکان مجموعہ سے حاصل جمع و ضرب دور میں جمع گیٹوں کی قطار اور ایک ضرب گیٹ ہو گا۔

\موٹا{جمع و ضرب دور کی شکل و صورت تبدیل کیے بغیر تمام گیٹوں کی جگہ متمم جمع گیٹ نسب کرنے سے متمم جمع و متمم جمع دور حاصل ہو گا۔}


\حصہ{علامتی روپ یا رموز}
عموماً زبانوں میں الفاظ یا معلومات کی لکھائی اس زبان کے حروف تہجی میں کی جاتی ہے۔حروف تہجی کو سلسلہ وار اس طرح جوڑا جاتا ہے کہ ان کی آوازیں مل کر لفظ کی آواز پیدا کریں، مگر چینی زبان مختلف ہے۔چینی زبان ایک علامتی زبان ہے جس میں ہر لفظ کی اپنی علامت یا \اصطلاح{رمز}\فرہنگ{رمز}\حاشیہب{code}\فرہنگ{code} ہے۔حروف تہجی پر مبنی لکھائی، یہ حروف سیکھنے کے بعد، کوئی بھی پڑھ سکتا ہے، جبکہ رمزی لکھائی میں کسی بھی رمز کا استعمال اس وقت ممکن ہو گا جب تمام لوگ اس رمز پر متفق ہوں ۔کمپیوٹر اس لحاظ سے چینی زبان سے مشابہت رکھتا ہے، اور معلومات کو رمزی روپ میں رکھتا ہے۔ 

قلم و کاغذ سے انسان کسی بھی شکل کی لکیر بنا کر اسے ایک علامت یا رمز تصور کر سکتا ہے۔کمپیوٹر کی دنیا میں ایسا کرنا ممکن نہیں۔کمپیوٹر صرف \عددی{0} اور \عددی{1} جانتا ہے، لہٰذا اس میں رموز بھی \عددی{0} اور \عددی{1} مختلف ترتیب سے جوڑ کر بنائے جاتے ہیں۔مثلاً، تین بِٹ استعمال کر کے جدول \حوالہ{جدول_بوولین_تین_بٹ_رموز} میں پیش رموز ممکن ہوں گے۔
\begin{table}
\caption{تین بِٹ رموز۔}
\label{جدول_بوولین_تین_بٹ_رموز}
\centering
\begin{otherlanguage}{english}
\begin{tabular}{C}
\toprule
\text{\RL{تین بِٹ رموز}}\\
\midrule
000\\
001\\
010\\
011\\
100\\
101\\
110\\
111\\
\bottomrule
\end{tabular}
\end{otherlanguage}
\end{table}
یوں تین بِٹ استعمال کر کے آٹھ رموز تشکیل دیے جا سکتے ہیں، جنہیں آٹھ مختلف اشیاء یا معلومات کی پہچان کے لئے استعمال کیا جا سکتا ہے۔تین بِٹ استعمال کرتے ہوئے، اس سے زیادہ رموز ممکن نہیں۔ آٹھ بِٹ میں \عددی{2^8=256} رمز ممکن ہیں۔

\جزوحصہ{ایسکی رموز اور عالمی رموز} 
ابتدا میں، کمپیوٹر استعمال کی خاطر لاطینی حروف تہجی اور اعشاری گنتی کے رمز طے کیے گئے۔ایک بائٹ پر مبنی رموز جو نہایت مقبول ہوئے، \اصطلاح{ایسکی رموز}\فرہنگ{رمز!ایسکی}\حاشیہب{ascii codes}\فرہنگ{codes!ascii} کہلاتے ہیں۔ لاطینی حروف تہجی اور اعشاری ہندسوں کے رموز جدول \حوالہ{جدول_بوولین_ایسکی_رموز} میں پیش کیے گئے ہیں۔
\begin{table}
\caption{ایسکی رموز۔}
\label{جدول_بوولین_ایسکی_رموز}
\centering
\begin{tabular}{CC}
\toprule
\text{\RL{ایسکی رمز}}&\text{\RL{لاطینی حرف یا ہندسہ}}\\
\midrule
01000001_2 & A\\
01000010_2&B\\
01000011_2 &C\\
01000100_2&D\\
\vdots&\vdots\\
01011000_2&X\\
01011001_2&Y\\
01011010_2&Z\\
\midrule
01100001_2&a\\
01100010_2&b\\
01100011_2&c\\
\vdots&\vdots\\
01111010_2&z\\
\midrule
00110000_2&0_{10}\\
00110001_2&1_{10}\\
00110010_2&2_{10}\\
\vdots&\vdots\\
00111000_2&8_{10}\\
00111001_2&9_{10}\\
\bottomrule
\end{tabular}
\end{table}
ایسکی رموز میں بڑے حرف \عددی{A} کو \عددی{01000001_2} یعنی \عددی{41_{16}} اور صفر کو \عددی{00110000_2} (\عددی{30_{16}}) کے رموز مختص کیے گئے۔یوں، اس نظام کو استعمال کرتے ہوئے کمپیوٹر \عددی{A} کو \عددی{01000001_2} سے، اور صفر کو \عددی{00110000_2} سے ظاہر کرے گا۔یاد رہے کہ، اس طرح کے نظام میں جدول دیکھ کر رمز کی معنی اخذ کی جا ئے گی۔
%???KKK

ایک بائٹ میں \عددی{00000000_2} سے \عددی{11111111_2} تک \عددی{256_{10}} مختلف رموز ہو گے، جو ایک محدود تعداد ہے۔ جیسے جیسے دنیا کی مختلف زبان بولنے والوں کے ہاں کمپیوٹر کا استعمال رائج ہوا، ایسکی رموز کے (محدود) رمز کم پڑ گئے۔موجودہ دور میں \اصطلاح{عالمی رموز}\فرہنگ{رمز!عالمی}\حاشیہب{uni code}\فرہنگ{code!uni} رائج ہے، جس میں دنیا کی تمام زبانوں (بشمول اردو، پشتو، بلوچی، سندی، وغیرہ) کے حروف تہجی کے رموز موجود ہیں۔اس نظام میں ہر رمز چار بائٹ کا ہے۔یہ کتاب عالمی رموز میں تشکیل دی گئی ہے۔اس نظام میں ریاضیات اور سائنس کے دیگر مضامین میں درکار علامتیں بھی ڈھالی جا سکتی ہیں۔امید یہی ہے کہ یہ نظام آنے والے زمانے میں درکار ضروریات پوری کرے گا۔

\جزوحصہ{اعشاری اعداد کے ثنائی رموز}
کمپیوٹر کی مادری زبان ثنائی ہے، جبکہ انسان اعشاری نظام استعمال کرتا ہے۔اعشاری گنتی کے کئی رموز زیر استعمال ہیں، جن میں سے ایک \اصطلاح{ثنائی مرموز اعشاریہ}\فرہنگ{ثنائی مرموز اعشاریہ}\حاشیہب{binary coded decimal (BCD)}\فرہنگ{binary coded decimal (BCD)} ہے۔اعشاری گنتی کے کل دس رموز ہیں۔جدول \حوالہ{جدول_بوولین_تین_بٹ_رموز} میں تین بِٹ رمز دکھائے گئے جو کُل آٹھ ہیں۔انہیں استعمال کرتے ہوئے اعشاری گنتی کے دس ہندسوں کو ظاہر نہیں کیا جا سکتا۔اس کے برعکس چار بِٹ کل سولہ رمز دیں گے، جنہیں اعشاری گنتی کے دس ہندسوں کے رموز کے طور پر استعمال کیا جا سکتا ہے۔جدول \حوالہ{جدول_بوولین_چار_بٹ_رموز} میں چار بِٹ پر مبنی ابتدائی دس علامتیں استعمال کرتے ہوئے اعشاری گنتی کے ہندسوں کے رموز پیش کیے گئے ہیں۔آخری چھ علامتیں زیر استعمال نہیں۔ یہ \اصطلاح{ثنائی مرموز اعشاریہ} کہلاتے ہیں۔
\begin{table}
\caption{اعشاری اعداد کے چار بِٹ ثنائی رموز۔}
\label{جدول_بوولین_چار_بٹ_رموز}
\centering
\begin{otherlanguage}{english}
\begin{tabular}{CC}
\toprule
\text{\RL{ثنائی مرموز اعشاریہ}}&\text{\RL{اعشاری اعداد}}\\
\midrule
0000&0\\
0001&1\\
0010&2\\
0011&3\\
0100&4\\
0101&5\\
0110&6\\
0111&7\\
1000&8\\
1001&9\\
\bottomrule
\end{tabular}
\end{otherlanguage}
\end{table}


\جزوحصہ{گرے رموز}
اس نظام میں اعشاری ہندسوں کے رمز یوں رکھے گئے کہ کسی بھی دو متواتر اعشاری ہندسوں کے رمز میں صرف ایک بِٹ کا فرق ہو۔ جدول \حوالہ{جدول_بوولین_چار_بٹ_گرے_رموز} چار بِٹ گرے رموز پیش کرتا ہے۔

طبیعی متغیرات کو عددی روپ میں، عموماً، گرے رموز میں لکھا جاتا ہے۔اس کی افادیت ایک مثال سے سمجھتے ہیں۔

تصور کریں کہ ایک بڑھتے ہوئے فاصلے کو چار بِٹ کے عام ثنائی نظام میں ناپا جاتا ہے۔ یوں \عددی{0111_2} کے بعد \عددی{1000_2} آئے گا۔اب تصور کریں کسی وجہ سے، اس چار بِٹ ثنائی عدد کا بلند رتبی بِٹ نسبتاً جلدی \عددی{0} سے \عددی{1} میں تبدیل ہوتا ہو۔یوں ایک لمحہ کے لئے \عددی{0111_2} کے بعد \عددی{1111_2} پڑھا جائے گا، جس کے بعد اصل عدد \عددی{1000_2}آ جائے گا۔آپ دیکھ سکتے ہیں کہ ایک لمحے کے لئے فاصلہ غلط پڑھا جائے گا، جس سے مسائل کھڑے ہو سکتے ہیں۔اس کے برعکس اگر گرے رمز استعمال کیا جائے تب \عددی{0100} کے بعد \عددی{1100} پڑھا جائے گا جو درست قیمت ہے۔
\begin{table}
\caption{اعشاری اعداد کے چار بِٹ گرے رموز۔}
\label{جدول_بوولین_چار_بٹ_گرے_رموز}
\centering
\begin{otherlanguage}{english}
\begin{tabular}{CC}
\toprule
\text{\RL{چار بِٹ گرے رموز}}&\text{\RL{اعشاری اعداد}}\\
\midrule
0000&0\\
0001&1\\
0011&2\\
0010&3\\
0110&4\\
0111&5\\
0101&6\\
0100&7\\
1100&8\\
1101&9\\
1111&10\\
1110&11\\
1010&12\\
1011&13\\
1001&14\\
1000&15\\
\bottomrule
\end{tabular}
\end{otherlanguage}
\end{table}



\باب{ کارناف نقشہ جات} 
بوولین جدول سے کسی بھی تفاعل کی مساوات بذریعہ مجموعہ ارکان ضرب یا  ضرب بعد از جمع حاصل کر کے اسے گیٹوں کی مدد سے جامہ پہنایا جا سکتا ہے۔عموماً، اس مساوات میں گیٹوں کی تعداد اور فی گیٹ مداخل کی تعداد کم کی جا سکتی ہے۔ کم مداخل کے، کم تعداد گیٹ استعمال کرنے سے عددی دور پر کم لاگت آئے گی۔ تفاعل کی سادہ صورت بوولین منطق سے حاصل کی جا سکتی ہے، البتہ ایک نہایت عمدہ اور سادہ طریقہ کار جسے کارناف نقشہ جات کی ترکیب کہتے ہیں، استعمال کیا جاتا ہے۔اس باب میں اس ترکیب پر غور کیا جائے گا۔یہ ترکیب چار اور چار سے کم آزاد متغیرات کے تفاعل کی سادہ صورت حاصل کرنے میں نہایت آسان ثابت ہو گا۔

\حصہ{کارناف نقشے کا بنیادی خاکہ}
دو آزاد متغیر تفاعل \عددی{F(x,y)} کے بوولین جدول میں چار مختلف ارکان ضرب ہوں گے، جنہیں جدول \حوالہ{جدول_کارناف_دو_آزاد} میں پیش کیا گیا ہے۔ اس کے کارناف نقشے میں چار خانے ہوں گے، جہاں ایک خانہ ایک رکن ضرب کو ظاہر کرتا ہے۔ کارناف نقشے میں ان چار خانوں کی ترتیب، شکل \حوالہ{شکل_کارناف_بنیادی_صورت}-الف میں دکھائی گئی ہے،جہاں بالائی صف میں \عددی{x=0} جبکہ نچلی صف میں \عددی{x=1} ہے؛ یہ قیمتیں صفوں کے بائیں طرف، خانوں سے باہر، لکھی گئی ہیں۔اسی طرح بائیں قطار میں \عددی{y=0}جبکہ دائیں قطار میں \عددی{y=1}ہے؛ یہ قیمتیں خانوں سے باہر، قطاروں کے اوپر جانب لکھی گئی ہیں ۔یوں بالائی صف اور دائیں قطار کے مشترکہ خانے میں \عددی{x=0} اور \عددی{y=1} ہے۔اس خانے کے آزاد متغیرات کی ثنائی قیمتوں کو اکٹھے \عددی{01} لکھیں۔ یہ خانہ رکن ضرب \عددی{\overline{x}y} کو ظاہر کرتا ہے، لہٰذا اس خانے میں \عددی{\overline{x}y} (شکل-الف) یا \عددی{m_1} (شکل-ب) لکھا جائے گا۔باقی خانوں میں اسی طرح اندراج کیے جاتے ہیں۔شکل \حوالہ{شکل_کارناف_چار_متغیر_بنیادی_صورت} میں اسی طرز پر چار آزاد متغیر تفاعل کارناف نقشے میں خانہ \عددی{m_{11}} کی نشاندہی کی گئی ہے۔

 \begin{table}
 \caption{دو متغیر ارکان ضرب۔}
 \label{جدول_کارناف_دو_آزاد}
 \centering
\begin{otherlanguage}{english}
\begin{tabular}{CC|CC}
\toprule
x&y&&\\
\midrule
0&0&\overline{x}\,\overline{y}&m_0\\
0&1&\overline{x}y&m_1\\
1&0&x\overline{y}&m_2\\
1&1&xy&m_3\\
\bottomrule
\end{tabular}
\end{otherlanguage}
\end{table}
%
\begin{figure}
\centering
\begin{subfigure}{0.6\textwidth}
\centering
\begin{tikzpicture}
\pgfmathsetmacro{\kxstep}{1}
\pgfmathsetmacro{\kystep}{1}
\pgfmathsetmacro{\kpin}{0.75}
\draw[xstep=\kxstep,ystep=\kystep](0,0) grid (2*\kxstep,-2*\kystep);
\draw(0,0)--++(135:\kpin)node[pos=0.75,above right]{$y$}node[pos=0.75,below left]{$x$};
\draw(\kxstep/2,0)node[above]{$0$} ++(\kxstep,0)node[above]{$1$};
\draw(0,-\kystep/2)node[left]{$0$} ++(0,-\kystep)node[left]{$1$};
\draw(\kxstep/2,-\kystep/2)node[]{$\overline{x}\,\overline{y}$} ++(\kxstep,0)node[]{$\overline{x}y$};
\draw(\kxstep/2,-\kystep-\kystep/2)node[]{$x\overline{y}$} ++(\kxstep,0)node[]{$xy$};
\draw[gray](\kxstep,-\kystep/2) circle (1.1cm and 0.4cm);
\draw[](2*\kxstep+0.2,-\kystep/2) to [out=35,in=180]++(0.5,0.2)node[right]{\text{\RL{اس صف میں \عددی{x=0} ہے}}};
\draw[gray](1.5*\kxstep,-\kystep) circle (0.4cm and 1.05cm);
\draw[](1.5*\kxstep,-2*\kystep-0.05) to [out=-35,in=180]++(0.5,-0.2)node[right]{\text{\RL{اس قطار میں \عددی{y=1} ہے}}};
\end{tikzpicture}
\caption{}
\end{subfigure}\hfill
\begin{subfigure}{0.4\textwidth}
\centering
\begin{tikzpicture}
\pgfmathsetmacro{\kxstep}{1}
\pgfmathsetmacro{\kystep}{1}
\pgfmathsetmacro{\kpin}{0.75}
\draw[xstep=\kxstep,ystep=\kystep](0,0) grid (2*\kxstep,-2*\kystep);
\draw(0,0)--++(135:\kpin)node[pos=0.75,above right]{$y$}node[pos=0.75,below left]{$x$};
\draw(\kxstep/2,-\kystep/2)node[]{$m_0$} ++(\kxstep,0)node[]{$m_1$};
\draw(\kxstep/2,-\kystep-\kystep/2)node[]{$m_2$} ++(\kxstep,0)node[]{$m_3$};
\end{tikzpicture}
\caption{}
\end{subfigure}
\caption{دا آزاد متغیر کارناف نقشے کی بنیادی صورت۔}
\label{شکل_کارناف_بنیادی_صورت}
\end{figure}

تین آزاد متغیر تفاعل \عددی{F(x,y,z)} کے آٹھ ارکان ضرب ہوں گے۔ انہیں شکل \حوالہ{شکل_کارناف_تین_متغیر_بنیادی_صورت} کے کارناف نقشہ میں دکھایا گیا ہے۔اس شکل میں دو صف اور چار قطار ہیں۔صفوں کا تعین \عددی{x}کی قیمت، جبکہ قطاروں کا تعین \عددی{yz}کی قیمت کرتی ہے۔ان قیمتوں کو (ثنائی گنتی کے روپ میں نہیں بلکہ)گرے رمز میں لکھا جاتا ہے۔ یوں، بائیں ہاتھ سے شروع کر کے، پہلی قطار میں \عددی{yz} کی قیمت \عددی{00}، دوسری میں \عددی{01}،تیسری میں \عددی{11} جبکہ آخری قطار میں \عددی{10} ہو گی۔

\begin{figure}
\centering
\begin{tikzpicture}
\pgfmathsetmacro{\kxstep}{1}
\pgfmathsetmacro{\kystep}{1}
\pgfmathsetmacro{\kpin}{0.75}
\draw[xstep=\kxstep,ystep=\kystep](0,0) grid (4*\kxstep,-2*\kystep);
\draw(0,0)--++(135:\kpin)node[pos=0.75,above right]{$yz$}node[pos=0.75,below left]{$x$};
\foreach \kx/\xlb in {0/{00},1/{01},2/{11},3/{10}}{\draw(\kx*\kxstep+\kxstep/2,0)node[above]{$\xlb$};}
\foreach \ky/\ylb in {0/0,1/1}{\draw(0,-\ky*\kystep-\kystep/2)node[left]{$\ylb$};}
\foreach \kx/\xlb in {0/{m_0},1/{m_1},2/{m_3},3/{m_2}}{\draw(\kx*\kxstep+\kxstep/2,-\kystep/2)node[]{$\xlb$};}
\foreach \kx/\xlb in {0/{m_4},1/{m_5},2/{m_7},3/{m_6}}{\draw(\kx*\kxstep+\kxstep/2,-1.5*\kystep)node[]{$\xlb$};}
\draw(4*\kxstep,0.25) to [out=0,in=180]++(0.5,0.25)node[right]{\text{\RL{گرے رمز}}};
\end{tikzpicture}
\caption{تین متغیر کارناف نقشے کی بنیادی صورت۔}
\label{شکل_کارناف_تین_متغیر_بنیادی_صورت}
\end{figure}

چار آزاد متغیر تفاعل \عددی{F(w,x,y,z)} کے سولہ ارکان ضرب ہوں گے، جنہیں چار صف اور چار قطار کے کارناف کے نقشے میں سمویا جا سکتا ہے۔شکل \حوالہ{شکل_کارناف_چار_متغیر_بنیادی_صورت} میں ایسا کارناف نقشہ دکھایا گیا ہے۔یہاں صفوں کا تعین \عددی{wx} کی قیمت، جبکہ قطاروں کا تعین \عددی{yz} کی قیمت کرتی ہیں۔ ان قیمتوں کو گرے رمز میں لکھ کر خانوں کی پہچان کی جاتی ہے۔

\begin{figure}
\centering
\begin{tikzpicture}
\pgfmathsetmacro{\kxstep}{1}
\pgfmathsetmacro{\kystep}{1}
\pgfmathsetmacro{\kpin}{0.75}
\draw[xstep=\kxstep,ystep=\kystep](0,0) grid (4*\kxstep,-4*\kystep);
\draw(0,0)--++(135:\kpin)node[pos=0.75,above right]{$yz$}node[pos=0.75,below left]{$wx$};
\foreach \kx/\xlb in {0/{00},1/{01},2/{11},3/{10}}{\draw(\kx*\kxstep+\kxstep/2,0)node[above]{$\xlb$};}
\foreach \ky/\ylb in {0/{00},1/{01},2/{11},3/{10}}{\draw(0,-\ky*\kystep-\kystep/2)node[left]{$\ylb$};}
\foreach \kx/\xlb in {0/{m_0},1/{m_1},2/{m_3},3/{m_2}}{\draw(\kx*\kxstep+\kxstep/2,-\kystep/2)node[]{$\xlb$};}
\foreach \kx/\xlb in {0/{m_4},1/{m_5},2/{m_7},3/{m_6}}{\draw(\kx*\kxstep+\kxstep/2,-1.5*\kystep)node[]{$\xlb$};}
\foreach \kx/\xlb in {0/{m_{12}},1/{m_{13}},2/{m_{15}},3/{m_{14}}}{\draw(\kx*\kxstep+\kxstep/2,-2.5*\kystep)node[]{$\xlb$};}
\foreach \kx/\xlb in {0/{m_8},1/{m_9},2/{m_{11}},3/{m_{10}}}{\draw(\kx*\kxstep+\kxstep/2,-3.5*\kystep)node[]{$\xlb$};}
\end{tikzpicture}
\caption{چار متغیر کارناف نقشے کی بنیادی صورت۔}
\label{شکل_کارناف_چار_متغیر_بنیادی_صورت}
\end{figure}

اب تک آپ پر واضح ہو چکا ہو گا کہ کارناف نقشے بناتے ہوئے صفوں اور قطاروں کو گرے رمز میں میں رکھا جاتا ہے۔چار سے زیادہ متغیرات کے کارناف نقشوں کا استعمال نسبتاً پیچیدہ ہوتا ہے، لہٰذا ان سے تفاعل کا سادہ روپ عموماً کمپیوٹر کی مدد سے حاصل کیا جاتا ہے۔ 


\حصہ{کارناف نقشے کی بھرائی}
 بوولین جدول سے کارناف نقشے کی بھرائی نہایت آسان اور سیدھا عمل ہے۔بوولین جدول کی جن صفوں میں تفاعل کی قیمت \عددی{1} ہو، ان کے مطابقتی (کارناف نقشہ کے) خانوں میں \عددی{1} پُر کریں؛ باقی خانوں میں \عددی{0} پُر کریں۔شکل \حوالہ{شکل_کارناف_دو_بھرائی}-الف میں دو آزاد متغیر تفاعل \عددی{F=\sum (m_0,m_1)} کے لئے یہ عمل دکھایا گیا ہے۔شکل-ج میں تفاعل کا کارناف کا نقشہ پُر کیا ہوا دکھایا گیا ہے۔ تفاعل کو مجموعہ ارکان ضرب کے روپ میں لکھنے سے کارناف نقشہ میں پُر کئے جانے والے خانوں کی نشاندہی ہوتی ہے۔
 
\begin{figure}
\centering
\begin{subfigure}{1\textwidth}
\centering
\begin{otherlanguage}{english}
\begin{tabular}{CC|C|C}
\toprule
x&y&F&\text{\RL{ارکان ضرب}}\\
\midrule
0&0&1&m_0\\
0&1&1&m_1\\
1&0&0&m_2\\
1&1&0&m_3\\
\bottomrule
\end{tabular}\quad\quad
\begin{tikzpicture}
$F=\sum(m_0,m_1)$
\end{tikzpicture}
\end{otherlanguage}
\caption{}
\end{subfigure}\hfill
\begin{subfigure}{0.45\textwidth}
\centering
\begin{tikzpicture}
\pgfmathsetmacro{\kxstep}{1}
\pgfmathsetmacro{\kystep}{1}
\pgfmathsetmacro{\kpin}{0.75}
\draw[xstep=\kxstep,ystep=\kystep](0,0) grid (2*\kxstep,-2*\kystep);
\draw(0,0)--++(135:\kpin)node[pos=0.75,above right]{$y$}node[pos=0.75,below left]{$x$};
\foreach \kx/\xlb in {0/{0},1/{1}}{\draw(\kx*\kxstep+\kxstep/2,0)node[above]{$\xlb$};}
\foreach \ky/\ylb in {0/0,1/1}{\draw(0,-\ky*\kystep-\kystep/2)node[left]{$\ylb$};}
\foreach \kx/\xlb in {0/{m_0},1/{m_1}}{\draw(\kx*\kxstep+\kxstep/2,-\kystep/2)node[]{$\xlb$};}
\foreach \kx/\xlb in {0/{m_2},1/{m_3}}{\draw(\kx*\kxstep+\kxstep/2,-1.5*\kystep)node[]{$\xlb$};}
\end{tikzpicture}
\caption{}
\end{subfigure}\hfill
\begin{subfigure}{0.45\textwidth}
\centering
\begin{tikzpicture}
\pgfmathsetmacro{\kxstep}{1}
\pgfmathsetmacro{\kystep}{1}
\pgfmathsetmacro{\kpin}{0.75}
\draw[xstep=\kxstep,ystep=\kystep](0,0) grid (2*\kxstep,-2*\kystep);
\draw(0,0)--++(135:\kpin)node[pos=0.75,above right]{$y$}node[pos=0.75,below left]{$x$};
\foreach \kx/\xlb in {0/{0},1/{1}}{\draw(\kx*\kxstep+\kxstep/2,0)node[above]{$\xlb$};}
\foreach \ky/\ylb in {0/0,1/1}{\draw(0,-\ky*\kystep-\kystep/2)node[left]{$\ylb$};}
\foreach \kx/\xlb in {0/{1},1/{1}}{\draw(\kx*\kxstep+\kxstep/2,-\kystep/2)node[]{$\xlb$};}
\foreach \kx/\xlb in {0/{0},1/{0}}{\draw(\kx*\kxstep+\kxstep/2,-1.5*\kystep)node[]{$\xlb$};}
\end{tikzpicture}
\caption{}
\end{subfigure}
\caption{دو متغیر تفاعل کارناف نقشے کی بھرائی۔}
\label{شکل_کارناف_دو_بھرائی}
\end{figure}

تین آزاد متغیر تفاعل \عددی{F=\sum (m_3,m_5,m_6,m_7)} کی مثال شکل \حوالہ{شکل_کارناف_تین_بھرائی} میں پیش کی گئی ہیں۔

\begin{figure}
\centering
\begin{subfigure}{1\textwidth}
\centering
\begin{otherlanguage}{english}
\begin{tabular}{CCC|C|C}
\toprule
x&y&z&F&\text{\RL{ارکان ضرب}}\\
\midrule
0&0&0&0&m_0\\
0&0&1&0&m_1\\
0&1&0&0&m_2\\
0&1&1&1&m_3\\
1&0&0&0&m_4\\
1&0&1&1&m_5\\
1&1&0&1&m_6\\
1&1&1&1&m_7\\
\bottomrule
\end{tabular}\quad\quad
\begin{tikzpicture}
$F=\sum(m_3, m_5, m_6, m_7)$
\end{tikzpicture}
\end{otherlanguage}
\caption{}
\end{subfigure}
\begin{subfigure}{0.45\textwidth}
\centering
\begin{tikzpicture}
\pgfmathsetmacro{\kxstep}{1}
\pgfmathsetmacro{\kystep}{1}
\pgfmathsetmacro{\kpin}{0.75}
\draw[xstep=\kxstep,ystep=\kystep](0,0) grid (4*\kxstep,-2*\kystep);
\draw(0,0)--++(135:\kpin)node[pos=0.75,above right]{$yz$}node[pos=0.75,below left]{$x$};
\foreach \kx/\xlb in {0/{00},1/{01},2/{11},3/{10}}{\draw(\kx*\kxstep+\kxstep/2,0)node[above]{$\xlb$};}
\foreach \ky/\ylb in {0/0,1/1}{\draw(0,-\ky*\kystep-\kystep/2)node[left]{$\ylb$};}
\foreach \kx/\xlb in {0/{m_0},1/{m_1},2/{m_3},3/{m_2}}{\draw(\kx*\kxstep+\kxstep/2,-\kystep/2)node[]{$\xlb$};}
\foreach \kx/\xlb in {0/{m_4},1/{m_5},2/{m_7},3/{m_6}}{\draw(\kx*\kxstep+\kxstep/2,-1.5*\kystep)node[]{$\xlb$};}
\end{tikzpicture}
\caption{}
\end{subfigure}\hfill
\begin{subfigure}{0.45\textwidth}
\centering
\begin{tikzpicture}
\pgfmathsetmacro{\kxstep}{1}
\pgfmathsetmacro{\kystep}{1}
\pgfmathsetmacro{\kpin}{0.75}
\draw[xstep=\kxstep,ystep=\kystep](0,0) grid (4*\kxstep,-2*\kystep);
\draw(0,0)--++(135:\kpin)node[pos=0.75,above right]{$yz$}node[pos=0.75,below left]{$x$};
\foreach \kx/\xlb in {0/{00},1/{01},2/{11},3/{10}}{\draw(\kx*\kxstep+\kxstep/2,0)node[above]{$\xlb$};}
\foreach \ky/\ylb in {0/0,1/1}{\draw(0,-\ky*\kystep-\kystep/2)node[left]{$\ylb$};}
\foreach \kx/\xlb in {0/{0},1/{0},2/{1},3/{0}}{\draw(\kx*\kxstep+\kxstep/2,-\kystep/2)node[]{$\xlb$};}
\foreach \kx/\xlb in {0/{0},1/{1},2/{1},3/{1}}{\draw(\kx*\kxstep+\kxstep/2,-1.5*\kystep)node[]{$\xlb$};}
\end{tikzpicture}
\caption{}
\end{subfigure}
\caption{تین متغیر کارناف نقشے کی بھرائی۔}
\label{شکل_کارناف_تین_بھرائی}
\end{figure}



\حصہ{کارناف نقشے سے تفاعل کی سادہ مساوات کا حصول}
کارناف نقشے میں قریبی خانوں سے مراد ایسے \عددی{2^n} خانے ہیں جنہیں مربع یا مستطیل میں گھیرا جا سکے؛ یہاں \عددی{n} کی قیمت \عددی{1}، \عددی{2}، \عددی{3}، وغیرہ ہو سکتی ہے۔یوں \عددی{2}، \عددی{4}، \عددی{8}،وغیرہ ، ایسے خانے جنہیں مربع یا مستطیل میں گھیرا جا سکے قریبی خانے کہلائیں گے۔کوئی بھی خانہ (یا خانے) ایک سے زیادہ مربع یا مستطیل کا حصہ بن سکتا ہے (سکتے ہیں)۔

قریبی خانوں میں تفاعل کی قیمت \عددی{1} ہونے کی صورت میں، ان خانوں کے ارکان ضرب کا مجموعہ بوولین قوانین سے حل کر کے سادہ ترین رکن ضرب حاصل کیا جا سکتا ہے۔یہ رکن ان قریبی خانوں کے ارکان ضرب میں مشترک حصے پر مشتمل ہو گا۔

دو قریبی بلند خانوں (جن میں تفاعل کی قیمت \عددی{1} ہو گی، کے ارکان ضرب کے مجموعہ ) سے حاصل، سادہ ترین رکن ضرب میں آزاد متغیرات کی تعداد، تفاعل میں آزاد متغیرات کی تعداد سے ایک کم ہو گی۔اسی طرح، چار بلند قریبی خانوں سے حاصل، سادہ ترین رکن ضرب میں آزاد متغیرات کی تعداد، تفاعل میں آزاد متغیرات کی تعداد سے دو کم ہو گی۔آٹھ قریبی بلند خانوں سے حاصل، سادہ ترین رکن ضرب میں آزاد متغیرات کی تعداد، تفاعل میں آزاد متغیرات کی تعداد سے چار کم ہو گی۔
	
قریبی خانے گھیرتے وقت یہ کوشش ہونی چاہئے کہ بڑے سے بڑا مربع یا مستطیل بنے۔ایسا کرنے سے سادہ ترین رکن ضرب حاصل ہو گا۔عموماً، قریبی خانوں کو ایک سے زیادہ طریقوں سے گھیرا جا سکتا ہے، جن سے تفاعل کی مختلف سادہ صورتیں حاصل ہوں گی۔
	
اب ہم چند مثالوں کی مدد سے اس طریقہ کار کو سیکھتے ہیں۔
	

\جزوحصہ{دو آزاد متغیر تفاعل}
دو متغیر تفاعل کے کارناف نقشہ میں \عددی{m_0} اور \عددی{m_1} قریبی خانے ہوں گے۔اسی طرح \عددی{m_0} اور \عددی{m_2} بھی قریبی خانے ہوں گے، جبکہ \عددی{m_1} اور \عددی{m_2} قریبی خانے نہیں ہوں گے۔

شکل \حوالہ{شکل_کارناف_قریبی_سادہ} میں دو متغیر تفاعل اور اس کا کارناف نقشہ دیا گیا ہے۔ کارناف نقشے میں خانوں سے اوپر، متغیر \عددی{y} کی ممکن قیمتوں \عددی{0} اور \عددی{1} کی بجائے بالترتیب \عددی{\overline{y}} اور \عددی{y} لکھا گیا ہے (یعنی \عددی{1} کی جگہ متغیر لکھا گیا ہے جبکہ \عددی{0} کی جگہ متغیر لکھ کر اس پر لکیر لگائی گئی ہے جو پست متغیر کو ظاہر کرتا ہے)۔ اسی طرح خانوں کے بائیں جانب \عددی{\overline{x}} اور \عددی{x} لکھا گیا ہے۔

 کارناف نقشے کے دو قریبی خانوں میں تفاعل کی قیمت \عددی{1} ہے، جنہیں نقطہ دار مستطیل میں گھیرا گیا ہے۔شکل-د میں ان خانوں کے ارکان ضرب کے مجموعے کو بوولین قوانین سے حل کر کے سادہ رکن حاصل کیا گیا۔آپ دیکھ سکتے ہیں کہ ان خانوں کے ارکان ضرب کے مجموعے سے ایک متغیر رکن حاصل ہوتا ہے؛ یعنی دو متغیر تفاعل کی صورت میں دو خانوں سے ایک متغیر رکن حاصل ہوا۔

یہی مساوات، شکل-ج کے کارناف نقشے میں نقطہ دار مستطیل میں گھیرے ، دو قریبی خانوں کو دیکھ کر لکھی جا سکتی ہے۔نقطہ دار مستطیل میں گھیرے دو قریبی خانوں کے ارکان ضرب \عددی{\overline{x}\,\overline{y}} اور \عددی{\overline{x}y} ہیں۔ ان ارکان ضرب میں \عددی{\overline{x}} مشترک ہے، جبکہ ایک رکن میں \عددی{\overline{y}} اور دوسرے میں \عددی{y} ہے۔ یوں، نقطہ دار مستطیل میں گھیرے ارکان ضرب میں وہ حصہ جو مشترک ہو مطلوبہ سادہ رکن ہو گا۔ ( غیر مشترک حصہ رد کرنا، شکل-د میں \عددی{\overline{y}+y=1} کے مترادف ہے۔) چونکہ ان خانوں کے علاوہ تمام خانوں میں \عددی{0} ہے لہٰذا یہی رکن تفاعل کی مساوات \عددی{(F=\overline{x})} ہو گی۔

\begin{figure}
\centering
\begin{subfigure}{0.25\textwidth}
\centering
\begin{otherlanguage}{english}
\begin{tabular}{CC|C|C}
\toprule
x&y&F&\\
\midrule
0&0&1&m_0\\
0&1&1&m_1\\
1&0&0&m_2\\
1&1&0&m_3\\
\bottomrule
\end{tabular}
\end{otherlanguage}
\caption{}
\end{subfigure}\hfill
\begin{subfigure}{0.25\textwidth}
\centering
\begin{tikzpicture}
\pgfmathsetmacro{\kxstep}{1}
\pgfmathsetmacro{\kystep}{1}
\pgfmathsetmacro{\kpin}{0.75}
\pgfmathsetmacro{\kmv}{0.15}
\draw[xstep=\kxstep,ystep=\kystep](0,0) grid (2*\kxstep,-2*\kystep);
%\draw(0,0)--++(135:\kpin)node[pos=0.75,above right]{$y$}node[pos=0.75,below left]{$x$};
\foreach \kx/\xlb in {0/{\overline{y}},1/{y}}{\draw(\kx*\kxstep+\kxstep/2,0)node[above]{$\xlb$};}
\foreach \ky/\ylb in {0/{\overline{x}},1/{x}}{\draw(0,-\ky*\kystep-\kystep/2)node[left]{$\ylb$};}
\foreach \kx/\xlb in {0/{1},1/{1}}{\draw(\kx*\kxstep+\kxstep/2,-\kystep/2)node[]{$\xlb$};}
\foreach \kx/\xlb in {0/{0},1/{0}}{\draw(\kx*\kxstep+\kxstep/2,-1.5*\kystep)node[]{$\xlb$};}
\draw[gray,dashed] ($(0,0)+(\kmv,-\kmv)$) rectangle ($(2*\kxstep,-\kystep)+(-\kmv,\kmv)$);
\end{tikzpicture}
\caption{}
\end{subfigure}\hfill
\begin{subfigure}{0.25\textwidth}
\centering
\begin{tikzpicture}
\pgfmathsetmacro{\kxstep}{1}
\pgfmathsetmacro{\kystep}{1}
\pgfmathsetmacro{\kpin}{0.75}
\pgfmathsetmacro{\kmv}{0.15}
\draw[xstep=\kxstep,ystep=\kystep](0,0) grid (2*\kxstep,-2*\kystep);
%\draw(0,0)--++(135:\kpin)node[pos=0.75,above right]{$yz$}node[pos=0.75,below left]{$x$};
\foreach \kx/\xlb in {0/{\overline{y}},1/{y}}{\draw(\kx*\kxstep+\kxstep/2,0)node[above]{$\xlb$};}
\foreach \ky/\ylb in {0/{\overline{x}},1/{x}}{\draw(0,-\ky*\kystep-\kystep/2)node[left]{$\ylb$};}
\foreach \kx/\xlb in {0/{\overline{x}\,\overline{y}},1/{\overline{x}y}}{\draw(\kx*\kxstep+\kxstep/2,-\kystep/2)node[]{$\xlb$};}
%\foreach \kx/\xlb in {0/{0},1/{0}}{\draw(\kx*\kxstep+\kxstep/2,-1.5*\kystep)node[]{$\xlb$};}
\draw[gray,dashed] ($(0,0)+(\kmv,-\kmv)$) rectangle ($(2*\kxstep,-\kystep)+(-\kmv,\kmv)$);
\end{tikzpicture}
\caption{}
\end{subfigure}\hfill
\begin{subfigure}{0.25\textwidth}
\centering
\begin{tikzpicture}
\draw(0,0)node[]{$\begin{aligned}
F&=\overline{x}\,\overline{y}+\overline{x}y\\
&=\overline{x}(\overline{y}+y)\\
&=\overline{x}(1)\\
&=\overline{x}
\end{aligned}$};
\end{tikzpicture}
\caption{}
\end{subfigure}
\caption{قریبی بلند خانوں سے سادہ رکن ضرب کا حصول۔}
\label{شکل_کارناف_قریبی_سادہ}
\end{figure}
	


شکل \حوالہ{شکل_کارناف_قریبی_پہلا_مزید} میں ایک تفاعل کا جدول دیا گیا ہے جس قریبی خانوں کے ارکان ضرب \عددی{\overline{x}\,\overline{y}} اور \عددی{x\overline{y}} میں \عددی{\overline{y}} مشترک ہے۔چونکہ باقی خانوں میں \عددی{0} ہے، لہٰذا اس تفاعل کی سادہ مساوات \عددی{F=\overline{y}} ہو گی۔

\begin{figure}
\centering
\begin{subfigure}{0.30\textwidth}
\centering
\begin{tikzpicture}
\pgfmathsetmacro{\kxstep}{1}
\pgfmathsetmacro{\kystep}{1}
\pgfmathsetmacro{\kpin}{0.75}
\pgfmathsetmacro{\kmv}{0.15}
\draw[xstep=\kxstep,ystep=\kystep](0,0) grid (2*\kxstep,-2*\kystep);
%\draw(0,0)--++(135:\kpin)node[pos=0.75,above right]{$y$}node[pos=0.75,below left]{$x$};
\foreach \kx/\xlb in {0/{\overline{y}},1/{y}}{\draw(\kx*\kxstep+\kxstep/2,0)node[above]{$\xlb$};}
\foreach \ky/\ylb in {0/{\overline{x}},1/{x}}{\draw(0,-\ky*\kystep-\kystep/2)node[left]{$\ylb$};}
\foreach \kx/\xlb in {0/{1},1/{0}}{\draw(\kx*\kxstep+\kxstep/2,-\kystep/2)node[]{$\xlb$};}
\foreach \kx/\xlb in {0/{1},1/{0}}{\draw(\kx*\kxstep+\kxstep/2,-1.5*\kystep)node[]{$\xlb$};}
\draw[gray,dashed] ($(0,0)+(\kmv,-\kmv)$) rectangle ($(\kxstep,-2*\kystep)+(-\kmv,\kmv)$);
\end{tikzpicture}
\caption{}
\end{subfigure}\hfill
\begin{subfigure}{0.30\textwidth}
\centering
\begin{tikzpicture}
\pgfmathsetmacro{\kxstep}{1}
\pgfmathsetmacro{\kystep}{1}
\pgfmathsetmacro{\kpin}{0.75}
\pgfmathsetmacro{\kmv}{0.15}
\draw[xstep=\kxstep,ystep=\kystep](0,0) grid (2*\kxstep,-2*\kystep);
%\draw(0,0)--++(135:\kpin)node[pos=0.75,above right]{$yz$}node[pos=0.75,below left]{$x$};
\foreach \kx/\xlb in {0/{\overline{y}},1/{y}}{\draw(\kx*\kxstep+\kxstep/2,0)node[above]{$\xlb$};}
\foreach \ky/\ylb in {0/{\overline{x}},1/{x}}{\draw(0,-\ky*\kystep-\kystep/2)node[left]{$\ylb$};}
\foreach \kx/\xlb in {0/{\overline{x}\,\overline{y}}}{\draw(\kx*\kxstep+\kxstep/2,-\kystep/2)node[]{$\xlb$};}
\foreach \kx/\xlb in {0/{x\overline{y}}}{\draw(\kx*\kxstep+\kxstep/2,-1.5*\kystep)node[]{$\xlb$};}
\draw[gray,dashed] ($(0,0)+(\kmv,-\kmv)$) rectangle ($(\kxstep,-2*\kystep)+(-\kmv,\kmv)$);
\end{tikzpicture}
\caption{}
\end{subfigure}\hfill
\begin{subfigure}{0.30\textwidth}
\centering
\begin{tikzpicture}
\draw(0,0)node[]{$\begin{aligned}
F&=\overline{x}\,\overline{y}+x\overline{y}\\
&=(\overline{x}+x)\overline{y}\\
&=(1)\overline{y}\\
&=\overline{y}
\end{aligned}$};
\end{tikzpicture}
\caption{}
\end{subfigure}
\caption{قریبی بلند خانوں سے سادہ رکن ضرب کا حصول۔}
\label{شکل_کارناف_قریبی_پہلا_مزید}
\end{figure}

شکل \حوالہ{شکل_کارناف_قریبی_دوسرا_مزید} کے تفاعل کے ارکان ضرب \عددی{x\overline{y}} اور \عددی{xy} میں \عددی{x} مشترک ہے (شکل-ج دیکھیں)۔چونکہ باقی خانوں میں تفاعل کی قیمت \عددی{0} ہے لہٰذا تفاعل کے ارکان ضرب کا مجموعہ اسی رکن کے برابر ہو گا۔ یوں اس کی مساوات \عددی{F=x} ہو گی۔

\begin{figure}
\centering
\begin{subfigure}{0.30\textwidth}
\centering
\begin{tikzpicture}
\pgfmathsetmacro{\kxstep}{1}
\pgfmathsetmacro{\kystep}{1}
\pgfmathsetmacro{\kpin}{0.75}
\pgfmathsetmacro{\kmv}{0.15}
\draw[xstep=\kxstep,ystep=\kystep](0,0) grid (2*\kxstep,-2*\kystep);
%\draw(0,0)--++(135:\kpin)node[pos=0.75,above right]{$y$}node[pos=0.75,below left]{$x$};
\foreach \kx/\xlb in {0/{\overline{y}},1/{y}}{\draw(\kx*\kxstep+\kxstep/2,0)node[above]{$\xlb$};}
\foreach \ky/\ylb in {0/{\overline{x}},1/{x}}{\draw(0,-\ky*\kystep-\kystep/2)node[left]{$\ylb$};}
\foreach \kx/\xlb in {0/{0},1/{0}}{\draw(\kx*\kxstep+\kxstep/2,-\kystep/2)node[]{$\xlb$};}
\foreach \kx/\xlb in {0/{1},1/{1}}{\draw(\kx*\kxstep+\kxstep/2,-1.5*\kystep)node[]{$\xlb$};}
\draw[gray,dashed] ($(0,-\kystep)+(\kmv,-\kmv)$) rectangle ($(2*\kxstep,-2*\kystep)+(-\kmv,\kmv)$);
\end{tikzpicture}
\caption{}
\end{subfigure}\hfill
\begin{subfigure}{0.30\textwidth}
\centering
\begin{tikzpicture}
\pgfmathsetmacro{\kxstep}{1}
\pgfmathsetmacro{\kystep}{1}
\pgfmathsetmacro{\kpin}{0.75}
\pgfmathsetmacro{\kmv}{0.15}
\draw[xstep=\kxstep,ystep=\kystep](0,0) grid (2*\kxstep,-2*\kystep);
%\draw(0,0)--++(135:\kpin)node[pos=0.75,above right]{$yz$}node[pos=0.75,below left]{$x$};
\foreach \kx/\xlb in {0/{\overline{y}},1/{y}}{\draw(\kx*\kxstep+\kxstep/2,0)node[above]{$\xlb$};}
\foreach \ky/\ylb in {0/{\overline{x}},1/{x}}{\draw(0,-\ky*\kystep-\kystep/2)node[left]{$\ylb$};}
%\foreach \kx/\xlb in {0/{\overline{x}\,\overline{y}}}{\draw(\kx*\kxstep+\kxstep/2,-\kystep/2)node[]{$\xlb$};}
\foreach \kx/\xlb in {0/{x\overline{y}},1/{xy}}{\draw(\kx*\kxstep+\kxstep/2,-1.5*\kystep)node[]{$\xlb$};}
\draw[gray,dashed] ($(0,-\kystep)+(\kmv,-\kmv)$) rectangle ($(2*\kxstep,-2*\kystep)+(-\kmv,\kmv)$);
\end{tikzpicture}
\caption{}
\end{subfigure}\hfill
\begin{subfigure}{0.30\textwidth}
\centering
\begin{tikzpicture}
%\draw[step=1,thick](-2,-2) grid (2,1);
%\draw[step=0.1,gray](-2,-2) grid (2,1);
\draw(0,0) node[]{$x\overline{y}, \quad xy$};
\draw[gray](-0.5,-0.2)--++(0,-0.1)--(0.5,-0.5) (0.3,-0.2)--++(0,-0.1)--(0.5,-0.5)node[black,below,xshift=-2em]{\text{\RL{لکھائی میں \عددی{x} مشترک ہے}}};
\draw[gray](-0.3,0.2)--++(0,0.1)--(-1,0.5) (0.5,0.2)--++(0,0.1)--(-1,0.5)node[black,above]{\text{\RL{یہ دو مختلف ہیں}}};
\end{tikzpicture}
\caption{}
\end{subfigure}
\caption{قریبی بلند خانوں سے سادہ رکن ضرب کا حصول۔}
\label{شکل_کارناف_قریبی_دوسرا_مزید}
\end{figure}



شکل \حوالہ{شکل_کارناف_قریبی_تیسرا_مزید} میں ایک ہی خانے کو دو قریبی خانوں کے ساتھ باری باری جوڑتے ہوئے سادہ مساوات \عددی{(F=\overline{x}+\overline{y})} حاصل کرنا دکھایا گیا ہے۔آئیں اس مساوات کو بوولین منطق کی مدد سے حاصل کریں۔مساوات کو ارکان ضرب کا مجموعہ لکھ کر اس کی سادہ روپ اخذ کرتے ہیں:
\begin{align*}
F&=x\overline{y}+\overline{x}\,\overline{y}+\overline{x}y\\
&=x\overline{y}+\overline{x}\,\overline{y}+\overline{x}\,\overline{y}+\overline{x}y\\
&=(x+\overline{x})\overline{y}+\overline{x}(\overline{y}+y)\\
&=(1)\overline{y}+\overline{x}(1)\\
&=\overline{y}+\overline{x}
\end{align*}

 جہاں، دوسرے قدم پر جدول \حوالہ{جدول_بوولین_دو_پہلو_تفاعل}-ب کی شِق \عددی{4} (صفحہ \حوالہصفحہ{جدول_بوولین_دو_پہلو_تفاعل}) استعمال کرتے
 ہوئے \عددی{\overline{x}\,\overline{y}=\overline{x}\,\overline{y}+\overline{x}\,\overline{y}} لکھا گیا۔

\begin{figure}
\centering
\begin{subfigure}{0.30\textwidth}
\centering
\begin{tikzpicture}
\pgfmathsetmacro{\kxstep}{1}
\pgfmathsetmacro{\kystep}{1}
\pgfmathsetmacro{\kpin}{0.75}
\pgfmathsetmacro{\kmv}{0.15}
\draw[xstep=\kxstep,ystep=\kystep](0,0) grid (2*\kxstep,-2*\kystep);
%\draw(0,0)--++(135:\kpin)node[pos=0.75,above right]{$y$}node[pos=0.75,below left]{$x$};
\foreach \kx/\xlb in {0/{\overline{y}},1/{y}}{\draw(\kx*\kxstep+\kxstep/2,0)node[above]{$\xlb$};}
\foreach \ky/\ylb in {0/{\overline{x}},1/{x}}{\draw(0,-\ky*\kystep-\kystep/2)node[left]{$\ylb$};}
\foreach \kx/\xlb in {0/{1},1/{1}}{\draw(\kx*\kxstep+\kxstep/2,-\kystep/2)node[]{$\xlb$};}
\foreach \kx/\xlb in {0/{1},1/{0}}{\draw(\kx*\kxstep+\kxstep/2,-1.5*\kystep)node[]{$\xlb$};}
%\draw[gray,dashed] ($(0,-\kystep)+(\kmv,-\kmv)$) rectangle ($(2*\kxstep,-2*\kystep)+(-\kmv,\kmv)$);
\end{tikzpicture}
\end{subfigure}\hfill
\begin{subfigure}{0.30\textwidth}
\centering
\begin{tikzpicture}
\pgfmathsetmacro{\kxstep}{1}
\pgfmathsetmacro{\kystep}{1}
\pgfmathsetmacro{\kpin}{0.75}
\pgfmathsetmacro{\kmv}{0.15}
\pgfmathsetmacro{\kmva}{0.1}
\draw[xstep=\kxstep,ystep=\kystep](0,0) grid (2*\kxstep,-2*\kystep);
%\draw(0,0)--++(135:\kpin)node[pos=0.75,above right]{$yz$}node[pos=0.75,below left]{$x$};
\foreach \kx/\xlb in {0/{\overline{y}},1/{y}}{\draw(\kx*\kxstep+\kxstep/2,0)node[above]{$\xlb$};}
\foreach \ky/\ylb in {0/{\overline{x}},1/{x}}{\draw(0,-\ky*\kystep-\kystep/2)node[left]{$\ylb$};}
\foreach \kx/\xlb in {0/{\overline{x}\,\overline{y}},1/{\overline{x}y}}{\draw(\kx*\kxstep+\kxstep/2,-\kystep/2)node[]{$\xlb$};}
\foreach \kx/\xlb in {0/{x\overline{y}}}{\draw(\kx*\kxstep+\kxstep/2,-1.5*\kystep)node[]{$\xlb$};}
\draw[gray,dashed] ($(0,0)+(\kmv,-\kmv)$) rectangle ($(2*\kxstep,-1*\kystep)+(-\kmv,\kmv)$);
\draw[gray,dashed] ($(0,0)+(\kmva,-\kmva)$) rectangle ($(1*\kxstep,-2*\kystep)+(-\kmva,\kmva)$);
\end{tikzpicture}
\end{subfigure}\hfill
\begin{subfigure}{0.40\textwidth}
\centering
\begin{tikzpicture}
%\draw[step=1,thick](-2,-2) grid (2,1);
%\draw[step=0.1,gray](-2,-2) grid (2,1);
\draw(0,0) node[]{\text{\RL{
\عددی{\overline{x}\,\overline{y}} اور \عددی{\overline{x}y} لکھنے میں \عددی{\overline{x}} مشترک ہے،
}}};
\draw(0,-0.75) node[]{\text{\RL{
\عددی{\overline{x}\,\overline{y}} اور \عددی{x\overline{y}} لکھنے میں \عددی{\overline{y}} مشترک ہے،
}}};
\draw(0,-1.5) node[]{\text{\RL{
لہٰذا مساوات \عددی{F=\overline{x}+\overline{y}} ہو گی۔
}}};
\end{tikzpicture}
\end{subfigure}
\caption{قریبی بلند خانوں سے سادہ رکن کا حصول۔}
\label{شکل_کارناف_قریبی_تیسرا_مزید}
\end{figure}

شکل \حوالہ{شکل_کارناف_قریبی_چوتھا_مزید} میں چار قریبی خانے ایک مستطیل میں گھیرے جا سکتے ہیں۔ ایسی صورت میں تفاعل ہمیشہ بلند \عددی{(1)} رہے گا لہٰذا اس کی مساوات \عددی{F=1} ہو گی۔
\begin{figure}
\centering
\begin{subfigure}{0.40\textwidth}
\centering
\begin{tikzpicture}
\pgfmathsetmacro{\kxstep}{1}
\pgfmathsetmacro{\kystep}{1}
\pgfmathsetmacro{\kpin}{0.75}
\pgfmathsetmacro{\kmv}{0.15}
\draw[xstep=\kxstep,ystep=\kystep](0,0) grid (2*\kxstep,-2*\kystep);
%\draw(0,0)--++(135:\kpin)node[pos=0.75,above right]{$y$}node[pos=0.75,below left]{$x$};
\foreach \kx/\xlb in {0/{\overline{y}},1/{y}}{\draw(\kx*\kxstep+\kxstep/2,0)node[above]{$\xlb$};}
\foreach \ky/\ylb in {0/{\overline{x}},1/{x}}{\draw(0,-\ky*\kystep-\kystep/2)node[left]{$\ylb$};}
\foreach \kx/\xlb in {0/{1},1/{1}}{\draw(\kx*\kxstep+\kxstep/2,-\kystep/2)node[]{$\xlb$};}
\foreach \kx/\xlb in {0/{1},1/{1}}{\draw(\kx*\kxstep+\kxstep/2,-1.5*\kystep)node[]{$\xlb$};}
%\draw[gray,dashed] ($(0,-\kystep)+(\kmv,-\kmv)$) rectangle ($(2*\kxstep,-2*\kystep)+(-\kmv,\kmv)$);
\end{tikzpicture}
\end{subfigure}\hfill
\begin{subfigure}{0.40\textwidth}
\centering
\begin{tikzpicture}
\pgfmathsetmacro{\kxstep}{1}
\pgfmathsetmacro{\kystep}{1}
\pgfmathsetmacro{\kpin}{0.75}
\pgfmathsetmacro{\kmv}{0.15}
\pgfmathsetmacro{\kmva}{0.1}
\draw[xstep=\kxstep,ystep=\kystep](0,0) grid (2*\kxstep,-2*\kystep);
%\draw(0,0)--++(135:\kpin)node[pos=0.75,above right]{$yz$}node[pos=0.75,below left]{$x$};
\foreach \kx/\xlb in {0/{\overline{y}},1/{y}}{\draw(\kx*\kxstep+\kxstep/2,0)node[above]{$\xlb$};}
\foreach \ky/\ylb in {0/{\overline{x}},1/{x}}{\draw(0,-\ky*\kystep-\kystep/2)node[left]{$\ylb$};}
\foreach \kx/\xlb in {0/{\overline{x}\,\overline{y}},1/{\overline{x}y}}{\draw(\kx*\kxstep+\kxstep/2,-\kystep/2)node[]{$\xlb$};}
\foreach \kx/\xlb in {0/{x\overline{y}},1/{xy}}{\draw(\kx*\kxstep+\kxstep/2,-1.5*\kystep)node[]{$\xlb$};}
\draw[gray,dashed] ($(0,0)+(\kmv,-\kmv)$) rectangle ($(2*\kxstep,-2*\kystep)+(-\kmv,\kmv)$);
\end{tikzpicture}
\end{subfigure}\hfill
\begin{subfigure}{0.20\textwidth}
\centering
\begin{tikzpicture}
%\draw[step=1,thick](-2,-2) grid (2,1);
%\draw[step=0.1,gray](-2,-2) grid (2,1);
\draw(0,0) node[]{$F=1$};
\end{tikzpicture}
\end{subfigure}
\caption{چار قریبی خانوں سے سادہ رکن \عددی{1} حاصل ہو گا۔}
\label{شکل_کارناف_قریبی_چوتھا_مزید}
\end{figure}

شکل \حوالہ{شکل_کارناف_قریبی_پانچ_مزید} میں قریبی خانے نہیں پائے جاتے، لہٰذا ارکان ضرب کے مجموعہ کو مزید سادہ نہیں بنایا جا سکتا۔ جب بھی کوئی خانہ کسی مستطیل میں شامل نہ ہو، اس کا رکن ضرب جوں کا توں مجموعہ (اور مساوات) میں رہے گا۔ 

\begin{figure}
\centering
\begin{subfigure}{0.30\textwidth}
\centering
\begin{tikzpicture}
\pgfmathsetmacro{\kxstep}{1}
\pgfmathsetmacro{\kystep}{1}
\pgfmathsetmacro{\kpin}{0.75}
\pgfmathsetmacro{\kmv}{0.15}
\draw[xstep=\kxstep,ystep=\kystep](0,0) grid (2*\kxstep,-2*\kystep);
%\draw(0,0)--++(135:\kpin)node[pos=0.75,above right]{$y$}node[pos=0.75,below left]{$x$};
\foreach \kx/\xlb in {0/{\overline{y}},1/{y}}{\draw(\kx*\kxstep+\kxstep/2,0)node[above]{$\xlb$};}
\foreach \ky/\ylb in {0/{\overline{x}},1/{x}}{\draw(0,-\ky*\kystep-\kystep/2)node[left]{$\ylb$};}
\foreach \kx/\xlb in {0/{0},1/{1}}{\draw(\kx*\kxstep+\kxstep/2,-\kystep/2)node[]{$\xlb$};}
\foreach \kx/\xlb in {0/{1},1/{0}}{\draw(\kx*\kxstep+\kxstep/2,-1.5*\kystep)node[]{$\xlb$};}
%\draw[gray,dashed] ($(0,-\kystep)+(\kmv,-\kmv)$) rectangle ($(2*\kxstep,-2*\kystep)+(-\kmv,\kmv)$);
\end{tikzpicture}
\end{subfigure}\hfill
\begin{subfigure}{0.30\textwidth}
\centering
\begin{tikzpicture}
\pgfmathsetmacro{\kxstep}{1}
\pgfmathsetmacro{\kystep}{1}
\pgfmathsetmacro{\kpin}{0.75}
\pgfmathsetmacro{\kmv}{0.15}
\pgfmathsetmacro{\kmva}{0.1}
\draw[xstep=\kxstep,ystep=\kystep](0,0) grid (2*\kxstep,-2*\kystep);
%\draw(0,0)--++(135:\kpin)node[pos=0.75,above right]{$yz$}node[pos=0.75,below left]{$x$};
\foreach \kx/\xlb in {0/{\overline{y}},1/{y}}{\draw(\kx*\kxstep+\kxstep/2,0)node[above]{$\xlb$};}
\foreach \ky/\ylb in {0/{\overline{x}},1/{x}}{\draw(0,-\ky*\kystep-\kystep/2)node[left]{$\ylb$};}
\foreach \kx/\xlb in {1/{\overline{x}y}}{\draw(\kx*\kxstep+\kxstep/2,-\kystep/2)node[]{$\xlb$};}
\foreach \kx/\xlb in {0/{x\overline{y}}}{\draw(\kx*\kxstep+\kxstep/2,-1.5*\kystep)node[]{$\xlb$};}
%\draw[gray,dashed] ($(0,0)+(\kmv,-\kmv)$) rectangle ($(2*\kxstep,-2*\kystep)+(-\kmv,\kmv)$);
\end{tikzpicture}
\end{subfigure}\hfill
\begin{subfigure}{0.30\textwidth}
\centering
\begin{tikzpicture}
%\draw[step=1,thick](-2,-2) grid (2,1);
%\draw[step=0.1,gray](-2,-2) grid (2,1);
\draw(0,0) node[]{$F=x\overline{y}+\overline{x}y$};
\end{tikzpicture}
\end{subfigure}
\caption{قریبی خانے نہیں پائے جاتے۔}
\label{شکل_کارناف_قریبی_پانچ_مزید}
\end{figure}



\ابتدا{مشق}
ارکان ضرب کے مجموعہ کی سادہ صورت بوولین قوانین سے حاصل کر کے ثابت کریں کہ شکل \حوالہ{شکل_کارناف_قریبی_چوتھا_مزید} میں تفاعل کی سادہ مساوات \عددی{F=1} ہے۔
\انتہا{مشق}
\ابتدا{مشق}
 رکن ضرب نہ ہونے کی صورت میں ثابت کریں کہ تفاعل کی مساوات \عددی{F=0} ہو گی۔
\انتہا{مشق}

شکل \حوالہ{شکل_کارناف_قریبی_پانچ_مزید} میں ایسا تفاعل دیا گیا ہے جس کے خانے کسی مربع یا مستطیل میں نہیں گھیرے جا سکتے۔ایسے تفاعل کی مساوات کو سادہ نہیں بنایا جا سکتا۔


\جزوحصہ{تین متغیر تفاعل}
تین متغیر تفاعل اور اس کا کارناف نقشہ شکل \حوالہ{شکل_کارناف_قریبی_تین_متغیر} میں دکھایا گیا ہے۔کارناف نقشے میں دو قریبی خانوں کو گھیرنے والے تین مستطیل بنائے گئے ہیں۔ یاد رہے، مستطیل یوں بنانا لازمی ہے کہ اس میں \عددی{2^n} خانے سموئے جائیں، جہاں \عددی{n} عدد صحیح ہے۔ یوں تین خانوں کو گھیرنے کی اجازت نہیں۔
\begin{figure}
\centering
\begin{subfigure}{0.45\textwidth}
\centering
\begin{tikzpicture}
\pgfmathsetmacro{\kxstep}{1}
\pgfmathsetmacro{\kystep}{1}
\pgfmathsetmacro{\kpin}{0.75}
\pgfmathsetmacro{\kmv}{0.15}
\pgfmathsetmacro{\kmva}{0.10}
\draw[xstep=\kxstep,ystep=\kystep](0,0) grid (4*\kxstep,-2*\kystep);
%\draw(0,0)--++(135:\kpin)node[pos=0.75,above right]{$y$}node[pos=0.75,below left]{$x$};
\foreach \kx/\xlb in {0/{\overline{y}\,\overline{z}},1/{\overline{y}z},2/{yz},3/{y\overline{z}}}{\draw(\kx*\kxstep+\kxstep/2,0)node[above]{$\xlb$};}
\foreach \ky/\ylb in {0/{\overline{x}},1/{x}}{\draw(0,-\ky*\kystep-\kystep/2)node[left]{$\ylb$};}
\foreach \kx/\xlb in {0/{0},1/{0},2/1,3/0}{\draw(\kx*\kxstep+\kxstep/2,-\kystep/2)node[]{$\xlb$};}
\foreach \kx/\xlb in {0/{0},1/{1},2/1,3/1}{\draw(\kx*\kxstep+\kxstep/2,-1.5*\kystep)node[]{$\xlb$};}
\draw[gray,dashed] ($(\kxstep,-\kystep)+(\kmv,-\kmv)$) rectangle ($(3*\kxstep,-2*\kystep)+(-\kmv,\kmv)$);
\draw[gray,dashed] ($(2*\kxstep,-\kystep)+(\kmva,\kmva)$) rectangle ($(4*\kxstep,-2*\kystep)+(\kmva,-\kmva)$);
\draw[gray,dashed] ($(2*\kxstep,0)+(-\kmv,-\kmv)$) rectangle ($(3*\kxstep,-2*\kystep)+(\kmv,-2*\kmv)$);
\draw(4*\kxstep,-2*\kystep)++(0.1,\kystep/3) to [out=0,in=-135]++(0.5,0.5)node[right]{$xy$};
\draw(2.5*\kxstep,-2*\kystep-2*\kmv) to [out=-90,in=180]++(0.5,-0.5)node[right]{$yz$};
\draw(\kxstep+\kmv,-1.5*\kystep) to [out=180,in=45]++(-1,-1.25)node[left]{$xz$};
\end{tikzpicture}
\end{subfigure}\hfill
\begin{subfigure}{0.45\textwidth}
\centering
\begin{tikzpicture}
%\draw[step=1,thick](-2,-2) grid (2,1);
%\draw[step=0.1,gray](-2,-2) grid (2,1);
\draw(0,1)node[]{$F=\sum(m_3, m_5, m_6, m_7)$};
\draw(0,0) node[]{$F(x,y,z)=xy+yz+xz$};
\end{tikzpicture}
\end{subfigure}
\caption{تین متغیر تفاعل کے کارناف نقشے سے سادہ مساوات کا حصول۔}
\label{شکل_کارناف_قریبی_تین_متغیر}
\end{figure}


 درمیانی مستطیل \عددی{m_3} اور \عددی{m_7} گھیرتا ہے۔ ان خانوں کے ارکان ضرب میں \عددی{x} کی قیمت تبدیل ہوتی ہے، جبکہ \عددی{yz} دونوں میں مشترک ہے۔یوں ان کا سادہ رکن \عددی{yz} ہو گا۔باقی دو مستطیل سے \عددی{xy} اور \عددی{xz} حاصل ہو گا۔ یوں تفاعل کی سادہ مساوات ان کا مجموعہ \عددی{(F=xy+yz+xz)} ہو گا۔ اس مساوات کو ارکان ضرب کے مجموعہ سے بوولین قوانین کی مدد سے حاصل کر سکتے ہیں (جو آپ کو اگلی مشق میں کرنا ہو گا)۔
 \begin{gather}
 \begin{aligned}\label{مساوات_بوولین_تین_سادہ}
 F(x,y,z)&=\sum(m_3,m_5,m_6,m_7)\\
 &=\overline{x}yz+x\overline{y}z+xyz+xy\overline{z}& \text{\small\RL{(تفصیلی ارکان ضرب کا مجموعہ)}}&\\
 &=xy+yz+xz&\text{\small\RL{(سادہ ارکان ضرب کا مجموعہ)}}&
 \end{aligned}
 \end{gather}
اس مساوات کی دوسری لکیر میں، ارکان ضرب تمام آزاد متغیرات پر مشتمل ہیں۔اس طرح کے رکن ضرب کو تفصیلی رکن ضرب کہتے ہیں۔مساوات کی تیسری لکیر کے ارکان ضرب میں، آزاد متغیرات کی تعداد کم ہے۔ اس طرح کے رکن ضرب کو سادہ رکن ضرب کہتے ہیں۔اس کتاب میں، عموماً، دونوں اقسام رکن ضرب پکارے جائیں گے۔امید کی جاتی ہے، متن سے مطلوبہ مطلب واضح ہو گا؛جہاں ایسا نہ ہو، وہاں انہیں مکمل نام سے پکارا جائے گا۔ 





\ابتدا{مشق}
بوولین الجبرا استعمال کر کے مساوات \حوالہ{مساوات_بوولین_تین_سادہ} کی دوسری لکیر سے تیسری لکیر حاصل کریں۔ ساتھ ہی تسلی کر لیں کہ آپ شکل \حوالہ{شکل_کارناف_قریبی_تین_متغیر} کے کارناف نقشے سے سادہ ارکان ضرب حاصل کرنا جانتے ہیں۔
\انتہا{مشق}



شکل \حوالہ{شکل_کارناف_کی_تین_متغیر} میں تین متغیر کارناف نقشہ پیش کیا گیا ہے۔ نقشے میں \عددی{m_0=\overline{x}\,\overline{y}\,\overline{z}} اور
 \عددی{m_2=\overline{x}y\overline{z}} کا مجموعہ حاصل کرتے ہیں۔
 \begin{align*}
 m_0+m_2&=\overline{x}\,\overline{y}\,\overline{z}+\overline{x}y\overline{z}\\
 &=\overline{x}\,\overline{z}(\overline{y}+y)\\
 &=\overline{x}\,\overline{z}
 \end{align*}


\begin{figure}
\centering
\begin{subfigure}{0.50\textwidth}
\centering
\begin{tikzpicture}
\pgfmathsetmacro{\kxstep}{1}
\pgfmathsetmacro{\kystep}{1}
\pgfmathsetmacro{\kpin}{0.75}
\pgfmathsetmacro{\kmv}{0.15}
\pgfmathsetmacro{\kmva}{0.10}
\draw[xstep=\kxstep,ystep=\kystep](0,0) grid (4*\kxstep,-2*\kystep);
%\draw(0,0)--++(135:\kpin)node[pos=0.75,above right]{$y$}node[pos=0.75,below left]{$x$};
\foreach \kx/\xlb in {0/{\overline{y}\,\overline{z}},1/{\overline{y}z},2/{yz},3/{y\overline{z}}}{\draw(\kx*\kxstep+\kxstep/2,0)node[above]{$\xlb$};}
\foreach \ky/\ylb in {0/{\overline{x}},1/{x}}{\draw(0,-\ky*\kystep-\kystep/2)node[left]{$\ylb$};}
\foreach \kx/\xlb in {0/{1},1/{0},2/0,3/1}{\draw(\kx*\kxstep+\kxstep/2,-\kystep/2)node[]{$\xlb$};}
\foreach \kx/\xlb in {0/{0},1/{1},2/1,3/0}{\draw(\kx*\kxstep+\kxstep/2,-1.5*\kystep)node[]{$\xlb$};}
\draw[gray,dashed] ($(\kxstep,-\kystep)+(\kmv,-\kmv)$) rectangle ($(3*\kxstep,-2*\kystep)+(-\kmv,\kmv)$);
%\draw[gray,dashed] ($(2*\kxstep,-\kystep)+(\kmva,\kmva)$) rectangle ($(4*\kxstep,-2*\kystep)+(\kmva,-\kmva)$);
%\draw[gray,dashed] ($(2*\kxstep,0)+(-\kmv,-\kmv)$) rectangle ($(3*\kxstep,-2*\kystep)+(\kmv,-2*\kmv)$);
%\draw(4*\kxstep,-2*\kystep)++(0.1,\kystep/3) to [out=0,in=-135]++(0.5,0.5)node[right]{$xy$};
%\draw(2.5*\kxstep,-2*\kystep-2*\kmv) to [out=-90,in=180]++(0.5,-0.5)node[right]{$yz$};
\draw(\kxstep+\kmv,-1.5*\kystep) to [out=180,in=45]++(-1,-1.25)node[left]{$xz$};
\draw[gray,dashed] ($(0*\kxstep,0)+(-\kmv,-\kmv)$) --++ (1*\kxstep,0*\kystep)--++(0,-\kystep+2*\kmv)--++(-\kxstep,0);
\draw[gray,dashed] ($(4*\kxstep,0)+(\kmv,-\kmv)$) --++ (-1*\kxstep,0*\kystep)--++(0,-\kystep+2*\kmv)--++(\kxstep,0)coordinate(kend);
\draw(kend)++(0.05,0)--++(0.5,-0.5)node[right]{$\overline{x}\,\overline{z}$};
\end{tikzpicture}
\end{subfigure}\hfill
\begin{subfigure}{0.40\textwidth}
\centering
\begin{tikzpicture}
%\draw[step=1,thick](-2,-2) grid (2,1);
%\draw[step=0.1,gray](-2,-2) grid (2,1);
\draw(0,1)node[]{$F=\sum(m_0, m_2, m_5, m_7)$};
\draw(0,0) node[]{$F(x,y,z)=\overline{x}\,\overline{z}+xz$};
\end{tikzpicture}
\end{subfigure}
\caption{کارناف نقشے کے اطراف آپس میں ملائیں۔}
\label{شکل_کارناف_کی_تین_متغیر}
\end{figure}

ان تین متغیر ارکان ضرب کے مجموعے سے دو متغیر رکن ضرب حاصل ہوا۔یوں \عددی{m_0} اور \عددی{m_2} خانوں کو قریبی خانے تصور کرنا ہو گا۔ آئیں اس پر تفصیل سے گفتگو کریں۔

کارناف نقشے کے بایاں اور دایاں قطار کے خانوں کو قریبی تصور کریں۔ تصور میں اس کاغذ کو، جس پر کارناف نقشہ بنا ہو، یوں گول کریں کہ کاغذ کا بایاں اور دایاں کنارہ آپس مل جائیں۔ اب پہلی اور آخری قطار کے خانے قریبی ہوں گے۔ اسی طرح، دو سے زیادہ صفوں کی صورت میں، نچلی اور بالائی صف کے خانے قریبی ہوں گے۔ تصور میں کاغذ کو یوں لپیٹیں کہ اس کا نچلا کنارہ بالائی کنارے سے جا ملے۔ یوں ان صفوں کے خانوں کو قریبی تصور کیا جا سکتا ہے۔ 
	
شکل \حوالہ{شکل_کارناف_کی_تین_متغیر} میں \عددی{m_0} اور \عددی{m_2} کو مستطیل میں گھیرا دکھایا گیا ہے۔ (تصور کریں کہ لپیٹے گئے کاغذ پر ان خانوں کو مستطیل میں گھیرنے کے بعد، کاغذ کو دوبارہ سیدھا کیا گیا ہے؛ یوں مستطیل دو ٹکڑوں میں نظر آئے گا۔) ان خانوں میں \عددی{\overline{x}\,\overline{z}} مشترک ہے، جو ہمارے توقع کے عین مطابق ہے۔ خانہ \عددی{m_5} اور \عددی{m_7} میں \عددی{xz} مشترک ہے۔ یوں تفاعل کی سادہ مساوات ان سادہ ارکان کا مجموعہ \عددی{F=\overline{x}\,\overline{z}+xz} ہو گا۔


شکل \حوالہ{شکل_کارناف_چار_قریبی} میں تین متغیر کارناف نقشہ دیا گیا ہے، جس میں چار قریبی خانوں کے دو مربعے بنائے گئے ہیں۔آپ کارناف نقشے کو دیکھ کر تفاعل کی سادہ مساوات لکھ سکتے ہیں۔ (اگر آپ ایسا نہیں کر سکتے، تیار ہو جائیں! اگلی مشق میں یہی کہنے کو کہا گیا ہے۔)

\begin{figure}
\centering
\begin{subfigure}{0.55\textwidth}
\centering
\begin{tikzpicture}
\pgfmathsetmacro{\kxstep}{1}
\pgfmathsetmacro{\kystep}{1}
\pgfmathsetmacro{\kpin}{0.75}
\pgfmathsetmacro{\kmv}{0.1}
\pgfmathsetmacro{\kmva}{0.20}
\draw[xstep=\kxstep,ystep=\kystep](0,0) grid (4*\kxstep,-2*\kystep);
%\draw(0,0)--++(135:\kpin)node[pos=0.75,above right]{$y$}node[pos=0.75,below left]{$x$};
\foreach \kx/\xlb in {0/{\overline{y}\,\overline{z}},1/{\overline{y}z},2/{yz},3/{y\overline{z}}}{\draw(\kx*\kxstep+\kxstep/2,0)node[above]{$\xlb$};}
\foreach \ky/\ylb in {0/{\overline{x}},1/{x}}{\draw(0,-\ky*\kystep-\kystep/2)node[left]{$\ylb$};}
\foreach \kx/\xlb in {0/1,1/0,2/1,3/1}{\draw(\kx*\kxstep+\kxstep/2,-\kystep/2)node[]{$\xlb$};}
\foreach \kx/\xlb in {0/1,1/0,2/1,3/1}{\draw(\kx*\kxstep+\kxstep/2,-1.5*\kystep)node[]{$\xlb$};}
\draw[gray,dashed] ($(2*\kxstep,-0*\kystep)+(\kmva,-\kmva)$) rectangle ($(4*\kxstep,-2*\kystep)+(-\kmva,\kmva)$);
%\draw[gray,dashed] ($(2*\kxstep,-\kystep)+(\kmva,\kmva)$) rectangle ($(4*\kxstep,-2*\kystep)+(\kmva,-\kmva)$);
%\draw[gray,dashed] ($(2*\kxstep,0)+(-\kmv,-\kmv)$) rectangle ($(3*\kxstep,-2*\kystep)+(\kmv,-2*\kmv)$);
%\draw(4*\kxstep,-2*\kystep)++(0.1,\kystep/3) to [out=0,in=-135]++(0.5,0.5)node[right]{$xy$};
%\draw(2.5*\kxstep,-2*\kystep-2*\kmv) to [out=-90,in=180]++(0.5,-0.5)node[right]{$yz$};
\draw(2*\kxstep+\kmva,-1.5*\kystep) to [out=180,in=45]++(-1,-1)node[left]{$y$};
\draw[gray,dashed] ($(0*\kxstep,0)+(-\kmv,-\kmv)$) --++ (1*\kxstep,0*\kystep)--++(0,-2*\kystep+2*\kmv)--++(-\kxstep,0);
\draw[gray,dashed] ($(4*\kxstep,0)+(\kmv,-\kmv)$) --++ (-1*\kxstep,0*\kystep)--++(0,-2*\kystep+2*\kmv)--++(\kxstep,0)coordinate(kend);
\draw(kend)++(0.05,0)--++(0.5,-0.5)node[right]{$\overline{z}$};
\end{tikzpicture}
\end{subfigure}\hfill
\begin{subfigure}{0.35\textwidth}
\centering
\begin{tikzpicture}
%\draw[step=1,thick](-2,-2) grid (2,1);
%\draw[step=0.1,gray](-2,-2) grid (2,1);
\draw(0,0)node[]{
\begin{otherlanguage}{english}
\begin{tabular}{CCC}
\multicolumn{3}{c}{\text{\RL{چار کونے}}}\\[0.5em]
\overline{x}&\overline{y}&\overline{z}\\
x&\overline{y}&\overline{z}\\
\overline{x}& y &\overline{z}\\
x&y&\overline{z}
\end{tabular}
\end{otherlanguage}
};
\draw[dashed](0.3,-1.25)rectangle ++(0.6,1.9);
%\draw(0,-1.5) node[]{$F(x,y,z)=y+\overline{z}$};
\end{tikzpicture}
\end{subfigure}
\caption{چار قریبی خانے۔}
\label{شکل_کارناف_چار_قریبی}
\end{figure}

\ابتدا{مشق}
 شکل \حوالہ{شکل_کارناف_چار_قریبی} میں دئے تفاعل کی سادہ مساوات کارناف نقشے سے حاصل کریں۔اسی مساوات کو بوولین الجبرا کی مدد سے حاصل کریں۔شکل میں چار کونوں کا مشترک حصہ \عددی{(\overline{z})} دکھایا گیا ہے۔ 
\انتہا{مشق}


\جزوحصہ{چار متغیر تفاعل}
چار آزاد متغیر تفاعل کے سولہ ارکان ضرب ہوں گے۔اس کے کارناف نقشے میں قریبی خانوں کو پہچانے کی خاطر نقشے کو ایسی سطح پر بنا ہوا تصور کریں کہ نقشے کی دایاں قطار نقشے کی بائیں قطار سے جڑا ہو۔اسی طرح نقشے کی بالائی صف اور نچلی صف سے آپس میں جڑے ہوں۔یوں \عددی{m_4} خانہ \عددی{m_6} خانے سے جڑتا ہے، اور \عددی{m_1} خانہ \عددی{m_9} خانے سے جڑتا ہے۔

اس نقشے میں دو، چار، آٹھ اور سولہ قریبی خانے بنانا ممکن ہے۔دو قریبی خانوں کے ارکان ضرب کا مجموعہ ایک رکن ضرب دے گا، جس میں تین متغیرات ہوں گے۔چار قریبی خانوں کے ارکان ضرب کا مجموعہ ایک رکن ضرب دے گا، جس میں دو آزاد متغیرات ہوں گے۔آٹھ قریبی خانوں کے ارکان ضرب کا مجموعہ ایک رکن ضرب دے گا، جس میں ایک متغیر ہو گا، جبکہ سولہ قریبی خانوں کے ارکان ضرب کا مجموعہ \عددی{1} کے برابر ہو گا۔

چار متغیر کارناف نقشوں کی چند مثالیں دیکھتے ہیں۔

\ابتدا{مثال}\شناخت{مثال_بوولین_چار_پہلا}
 درج ذیل تفاعل کی سادہ مساوات شکل \حوالہ{شکل_کارناف_مثال_چار_پہلا_متغیر} میں پیش کی گئی ہے۔	
\begin{align*}
F(w,x,y,z)=\sum(m_0, m_2,m_4,m_6,m_8,m_{10}, m_{12},m_{13},m_{14}, m_{15})
\end{align*}
%
\begin{figure}
\centering
\begin{subfigure}{0.60\textwidth}
\centering
\begin{tikzpicture}
\pgfmathsetmacro{\kxstep}{1}
\pgfmathsetmacro{\kystep}{1}
\pgfmathsetmacro{\kpin}{0.75}
\pgfmathsetmacro{\kmv}{0.15}
\pgfmathsetmacro{\kmva}{0.10}
\draw[xstep=\kxstep,ystep=\kystep](0,0) grid (4*\kxstep,-4*\kystep);
%\draw(0,0)--++(135:\kpin)node[pos=0.75,above right]{$y$}node[pos=0.75,below left]{$x$};
\foreach \kx/\xlb in {0/{\overline{y}\,\overline{z}},1/{\overline{y}z},2/{yz},3/{y\overline{z}}}{\draw(\kx*\kxstep+\kxstep/2,0)node[above]{$\xlb$};}
\foreach \ky/\ylb in {0/{\overline{w}\,\overline{x}},1/{\overline{w}x},2/{wx},3/{w\overline{x}}}{\draw(0,-\ky*\kystep-\kystep/2)node[left]{$\ylb$};}
\foreach \kx/\xlb in {0/{1},3/1}{\draw(\kx*\kxstep+\kxstep/2,-\kystep/2)node[]{$\xlb$};}
\foreach \kx/\xlb in {0/{1},3/1}{\draw(\kx*\kxstep+\kxstep/2,-1.5*\kystep)node[]{$\xlb$};}
\foreach \kx/\xlb in {0/{1},1/1,2/1,3/1}{\draw(\kx*\kxstep+\kxstep/2,-2.5*\kystep)node[]{$\xlb$};}
\foreach \kx/\xlb in {0/{1},3/1}{\draw(\kx*\kxstep+\kxstep/2,-3.5*\kystep)node[]{$\xlb$};}
\draw[gray,dashed] ($(0,-2*\kystep)+(\kmv,-\kmv)$) rectangle ($(4*\kxstep,-3*\kystep)+(-\kmv,\kmv)$);
%\draw[gray,dashed] ($(2*\kxstep,-\kystep)+(\kmva,\kmva)$) rectangle ($(4*\kxstep,-2*\kystep)+(\kmva,-\kmva)$);
%\draw[gray,dashed] ($(2*\kxstep,0)+(-\kmv,-\kmv)$) rectangle ($(3*\kxstep,-2*\kystep)+(\kmv,-2*\kmv)$);
%\draw(4*\kxstep,-2*\kystep)++(0.1,\kystep/3) to [out=0,in=-135]++(0.5,0.5)node[right]{$xy$};
%\draw(2.5*\kxstep,-2*\kystep-2*\kmv) to [out=-90,in=180]++(0.5,-0.5)node[right]{$yz$};
\draw($(4*\kxstep,-2.5*\kystep)+(-\kmv,0)$) to [out=0,in=180]++(0.5,0.5)node[right]{$wx$};
\draw[gray,dashed] ($(0*\kxstep,0)+(-\kmv,-\kmv)$) --++ (1*\kxstep,0*\kystep)--++(0,-4*\kystep+2*\kmv)--++(-\kxstep,0);
\draw[gray,dashed] ($(4*\kxstep,0)+(\kmv,-\kmv)$) --++ (-1*\kxstep,0*\kystep)--++(0,-4*\kystep+2*\kmv)--++(\kxstep+2*\kmv,0)coordinate(kend);
\draw(kend)++(-0.05,0)--++(0.5,-0.5)node[right]{$\overline{z}$};
\end{tikzpicture}
\end{subfigure}\hfill
\begin{subfigure}{0.30\textwidth}
\centering
\begin{tikzpicture}
%\draw[step=1,thick](-2,-2) grid (2,1);
%\draw[step=0.1,gray](-2,-2) grid (2,1);
%\draw(0,1)node[]{$F=\sum(m_0, m_2, m_5, m_7)$};
\draw(0,0) node[]{$F(w,x,y,z)=wx+\overline{z}$};
\end{tikzpicture}
\end{subfigure}
\caption{چار متغیر نقشہ ( برائے مثال \حوالہ{مثال_بوولین_چار_پہلا})}
\label{شکل_کارناف_مثال_چار_پہلا_متغیر}
\end{figure}
\انتہا{مثال}
\ابتدا{مثال}\شناخت{مثال_بوولین_چار_دوسرا}
 درج ذیل تفاعلات کی سادہ مساوات حاصل کریں۔
\begin{align*}
F(w,x,y,z)&=\sum(m_0,m_5,m_7,m_{10},m_{11},m_{13},m_{15})\\
F(w,x,y,z)&=\sum(m_0,m_2,m_8,m_{10})
\end{align*}
%
\begin{figure}
\centering
\begin{subfigure}{0.45\textwidth}
\centering
\begin{tikzpicture}
\pgfmathsetmacro{\kxstep}{1}
\pgfmathsetmacro{\kystep}{1}
\pgfmathsetmacro{\kpin}{0.75}
\pgfmathsetmacro{\kmv}{0.15}
\pgfmathsetmacro{\kmva}{0.10}
\draw[xstep=\kxstep,ystep=\kystep](0,0) grid (4*\kxstep,-4*\kystep);
%\draw(0,0)--++(135:\kpin)node[pos=0.75,above right]{$y$}node[pos=0.75,below left]{$x$};
\foreach \kx/\xlb in {0/{\overline{y}\,\overline{z}},1/{\overline{y}z},2/{yz},3/{y\overline{z}}}{\draw(\kx*\kxstep+\kxstep/2,0)node[above]{$\xlb$};}
\foreach \ky/\ylb in {0/{\overline{w}\,\overline{x}},1/{\overline{w}x},2/{wx},3/{w\overline{x}}}{\draw(0,-\ky*\kystep-\kystep/2)node[left]{$\ylb$};}
\foreach \kx/\xlb in {0/{1}}{\draw(\kx*\kxstep+\kxstep/2,-\kystep/2)node[]{$\xlb$};}
\foreach \kx/\xlb in {1/{1},2/1}{\draw(\kx*\kxstep+\kxstep/2,-1.5*\kystep)node[]{$\xlb$};}
\foreach \kx/\xlb in {1/{1},2/1}{\draw(\kx*\kxstep+\kxstep/2,-2.5*\kystep)node[]{$\xlb$};}
\foreach \kx/\xlb in {2/1,3/1}{\draw(\kx*\kxstep+\kxstep/2,-3.5*\kystep)node[]{$\xlb$};}
\draw[gray,dashed] ($(1*\kxstep,-1*\kystep)+(\kmv,-\kmv)$) rectangle ($(3*\kxstep,-3*\kystep)+(-\kmv,\kmv)$);
\draw[gray,dashed] ($(2*\kxstep,-3*\kystep)+(\kmv,-\kmv)$) rectangle ($(4*\kxstep,-4*\kystep)+(-\kmv,\kmv)$);
\draw(4*\kxstep-\kmv,-3.5*\kystep) to [out=0,in=180]++(0.5,0.5)node[right]{$w\overline{x}x$};
\draw(3*\kxstep-\kmv,-1.5*\kystep) to [out=0,in=180]++(1.52,0.4)node[right]{$xz$};
\draw(0.25*\kxstep,-0.25*\kystep) to [out=180,in=0]++(-0.5,0.25)node[left]{$\overline{w}\,\overline{x}\,\overline{y}\,\overline{z}x$};
\path(0,-4*\kystep)--++(0,-0.5); %gives space between the map and its equation in the next tikzpic
\end{tikzpicture}
\begin{tikzpicture}
\draw(0,0)node[]{$F(w,x,y,z)=\overline{w}\,\overline{x}\,\overline{y}\,\overline{z}+xz+w\overline{x}y$};
\end{tikzpicture}
\caption{}
\end{subfigure}\hfill
\begin{subfigure}{0.45\textwidth}
\centering
\begin{tikzpicture}
\pgfmathsetmacro{\kxstep}{1}
\pgfmathsetmacro{\kystep}{1}
\pgfmathsetmacro{\kpin}{0.75}
\pgfmathsetmacro{\kmv}{0.15}
\pgfmathsetmacro{\kmva}{0.10}
\draw[xstep=\kxstep,ystep=\kystep](0,0) grid (4*\kxstep,-4*\kystep);
%\draw(0,0)--++(135:\kpin)node[pos=0.75,above right]{$y$}node[pos=0.75,below left]{$x$};
\foreach \kx/\xlb in {0/{\overline{y}\,\overline{z}},1/{\overline{y}z},2/{yz},3/{y\overline{z}}}{\draw(\kx*\kxstep+\kxstep/2,0)node[above]{$\xlb$};}
\foreach \ky/\ylb in {0/{\overline{w}\,\overline{x}},1/{\overline{w}x},2/{wx},3/{w\overline{x}}}{\draw(0,-\ky*\kystep-\kystep/2)node[left]{$\ylb$};}
\foreach \kx/\xlb in {0/{1},3/1}{\draw(\kx*\kxstep+\kxstep/2,-\kystep/2)node[]{$\xlb$};}
%\foreach \kx/\xlb in {1/{1},2/1}{\draw(\kx*\kxstep+\kxstep/2,-1.5*\kystep)node[]{$\xlb$};}
%\foreach \kx/\xlb in {1/{1},2/1}{\draw(\kx*\kxstep+\kxstep/2,-2.5*\kystep)node[]{$\xlb$};}
\foreach \kx/\xlb in {0/1,3/1}{\draw(\kx*\kxstep+\kxstep/2,-3.5*\kystep)node[]{$\xlb$};}
%\draw[gray,dashed] ($(1*\kxstep,-1*\kystep)+(\kmv,-\kmv)$) rectangle ($(3*\kxstep,-3*\kystep)+(-\kmv,\kmv)$);
%\draw[gray,dashed] ($(2*\kxstep,-3*\kystep)+(\kmv,-\kmv)$) rectangle ($(4*\kxstep,-4*\kystep)+(-\kmv,\kmv)$);
\draw[gray,dashed] ($(0,-1*\kystep)+(-\kmv,\kmv)$)--++(1*\kxstep,0)--++(0,\kystep+\kmv);
\draw[gray,dashed] ($(0,-3*\kystep)+(-\kmv,-\kmv)$)--++(1*\kxstep,0)--++(0,-\kystep-\kmv);
\draw[gray,dashed] ($(3*\kxstep,-0*\kystep)+(\kmv,\kmv)$)--++(0,-1*\kystep)--++(\kxstep+\kmv,0);
\draw[gray,dashed] ($(3*\kxstep,-4*\kystep)+(\kmv,-\kmv)$)--++(0,1*\kystep)--++(\kxstep+\kmv,0);
\path(0,-4*\kystep)--++(0,-0.5); %gives space between the map and its equation in the next tikzpic
\end{tikzpicture}
\begin{tikzpicture}
\draw(0,0)node[]{$F(w,x,y,z)=\overline{x}\,\overline{z}$};
\end{tikzpicture}
\caption{}
\end{subfigure}
\caption{چار متغیر نقشہ ( برائے مثال \حوالہ{مثال_بوولین_چار_دوسرا})}
\label{شکل_کارناف_مثال_چار_دوسرا_متغیر}
\end{figure}

\ترچھا{حل:}\quad
پہلا تفاعل شکل \حوالہ{شکل_کارناف_مثال_چار_دوسرا_متغیر}-الف میں دکھایا گیا ہے، جہاں چار قریبی خانے سادہ رکن ضرب \عددی{(xz)}، جبکہ دو قریبی خانے \عددی{w\overline{x}y} رکن ضرب دیں گے، اور ایک خانہ جو کسی کے قریب نہیں پایا جاتا رکن \عددی{\overline{w}\,\overline{x}\,\overline{y}\,\overline{z}} دے گا۔یوں تفاعل کی سادہ مساوات ان ارکان کا مجموعہ \عددی{F=\overline{w}\,\overline{x}\,\overline{y}\,\overline{z}+xz+w\overline{x}y} ہو گا۔


دوسرا تفاعل شکل \حوالہ{شکل_کارناف_مثال_چار_دوسرا_متغیر}-ب میں پیش کیا گیا ہے، جہاں چار کونوں کو قریبی تصور کریں، جو رکن \عددی{\overline{x}\,\overline{z}} دیں 
گے۔یہی اس تفاعل کی سادہ مساوات \عددی{F=\overline{x}\,\overline{z}} ہے۔
\انتہا{مثال}
\ابتدا{مشق}
شکل \حوالہ{شکل_کارناف_مثال_چار_دوسرا_متغیر}-ب کے چار خانوں کے ارکان ضرب کے مجموعہ کا سادہ روپ، بوولین قوانین کی مدد سے حاصل کر کے ثابت کریں کہ یہ قریبی خانے ہیں۔
\انتہا{مشق}
\ابتدا{مثال}\شناخت{مثال_بوولین_تین_بلاشرکت}
تین آزاد متغیرات کے بلا شرکت گیٹ کا کارناف نقشہ حاصل کریں۔

\begin{figure}
\centering
\begin{tikzpicture}
\pgfmathsetmacro{\kxstep}{1}
\pgfmathsetmacro{\kystep}{1}
\pgfmathsetmacro{\kpin}{0.75}
\pgfmathsetmacro{\kmv}{0.15}
\pgfmathsetmacro{\kmva}{0.10}
\draw[xstep=\kxstep,ystep=\kystep](0,0) grid (4*\kxstep,-2*\kystep);
%\draw(0,0)--++(135:\kpin)node[pos=0.75,above right]{$y$}node[pos=0.75,below left]{$x$};
\foreach \kx/\xlb in {0/{\overline{y}\,\overline{z}},1/{\overline{y}z},2/{yz},3/{y\overline{z}}}{\draw(\kx*\kxstep+\kxstep/2,0)node[above]{$\xlb$};}
\foreach \ky/\ylb in {0/{\overline{x}},1/x}{\draw(0,-\ky*\kystep-\kystep/2)node[left]{$\ylb$};}
\foreach \kx/\xlb in {1/{1},3/1}{\draw(\kx*\kxstep+\kxstep/2,-\kystep/2)node[]{$\xlb$};}
\foreach \kx/\xlb in {0/{1},2/1}{\draw(\kx*\kxstep+\kxstep/2,-1.5*\kystep)node[]{$\xlb$};}
%\foreach \kx/\xlb in {1/{1},2/1}{\draw(\kx*\kxstep+\kxstep/2,-2.5*\kystep)node[]{$\xlb$};}
%\foreach \kx/\xlb in {0/1,3/1}{\draw(\kx*\kxstep+\kxstep/2,-3.5*\kystep)node[]{$\xlb$};}
%\draw[gray,dashed] ($(1*\kxstep,-1*\kystep)+(\kmv,-\kmv)$) rectangle ($(3*\kxstep,-3*\kystep)+(-\kmv,\kmv)$);
%\draw[gray,dashed] ($(2*\kxstep,-3*\kystep)+(\kmv,-\kmv)$) rectangle ($(4*\kxstep,-4*\kystep)+(-\kmv,\kmv)$);
%\draw[gray,dashed] ($(0,-1*\kystep)+(-\kmv,\kmv)$)--++(1*\kxstep,0)--++(0,\kystep+\kmv);
%\draw[gray,dashed] ($(0,-3*\kystep)+(-\kmv,-\kmv)$)--++(1*\kxstep,0)--++(0,-\kystep-\kmv);
\end{tikzpicture}\quad 
\begin{tikzpicture}
\draw(0,0)node[]{$F(x,y,z)=x\oplus y \oplus z$};
\end{tikzpicture}
\caption{تین متغیر بلا شرکت گیٹ کا نقشہ (برائے مثال \حوالہ{مثال_بوولین_تین_بلاشرکت})}
\label{شکل_بوولین_تین_بلاشرکت}
\end{figure}
\ترچھا{حل:}\quad
شکل \حوالہ{شکل_بوولین_تین_بلاشرکت} میں نقشہ پیش ہے۔اس میں قریب خانے نہیں پائے جاتے، لہٰذا اس کی مساوات مزید سادہ نہیں بنائی جا سکتی۔ 
\انتہا{مثال}


\جزوحصہ{سادہ مساوات سے تفاعل کے ارکان ضرب کا حصول}
کسی بھی تفاعل کی سادہ مساوات کا حصول بذریعہ کارناف نقشہ آپ نے دیکھا۔اس حصے میں اس طریقہ کار کو اُلٹ چلا کر تفاعل کی سادہ مساوات سے ارکان ضرب کا مجموعہ حاصل کیا جائے گا۔یہ ترکیب مثال سے بہتر سمجھ آئی گی۔

\ابتدا{مثال}\شناخت{مثال_بوولین_الٹ_رخ}
درج ذیل سادہ مساوات سے تفاعل کے ارکان ضرب کا مجموعہ دریافت کریں۔
\begin{align*}
F(x,y,z)=y+\overline{x}\,\overline{z}
\end{align*}
\ترچھا{حل:}\quad
شکل \حوالہ{شکل_بوولین_الٹ_رخ} میں سادہ مساوات سے کارناف نقشہ حاصل کیا گیا، جس سے مجموعہ ارکان ضرب لکھا گیا ۔
\begin{figure}
\centering
\begin{tikzpicture}
\pgfmathsetmacro{\kxstep}{1}
\pgfmathsetmacro{\kystep}{1}
\pgfmathsetmacro{\kpin}{0.75}
\pgfmathsetmacro{\kmv}{0.1}
\pgfmathsetmacro{\kmva}{0.20}
\draw[xstep=\kxstep,ystep=\kystep](0,0) grid (4*\kxstep,-2*\kystep);
%\draw(0,0)--++(135:\kpin)node[pos=0.75,above right]{$y$}node[pos=0.75,below left]{$x$};
\foreach \kx/\xlb in {0/{\overline{y}\,\overline{z}},1/{\overline{y}z},2/{yz},3/{y\overline{z}}}{\draw(\kx*\kxstep+\kxstep/2,0)node[above]{$\xlb$};}
\foreach \ky/\ylb in {0/{\overline{x}},1/x}{\draw(0,-\ky*\kystep-\kystep/2)node[left]{$\ylb$};}
\foreach \kx/\xlb in {0/{1},2/1,3/1}{\draw(\kx*\kxstep+\kxstep/2,-\kystep/2)node[]{$\xlb$};}
\foreach \kx/\xlb in {2/{1},3/1}{\draw(\kx*\kxstep+\kxstep/2,-1.5*\kystep)node[]{$\xlb$};}
%\foreach \kx/\xlb in {1/{1},2/1}{\draw(\kx*\kxstep+\kxstep/2,-2.5*\kystep)node[]{$\xlb$};}
%\foreach \kx/\xlb in {0/1,3/1}{\draw(\kx*\kxstep+\kxstep/2,-3.5*\kystep)node[]{$\xlb$};}
\draw[gray,dashed] ($(2*\kxstep,-0*\kystep)+(\kmv,-\kmv)$) rectangle ($(4*\kxstep,-2*\kystep)+(-\kmv,\kmv)$);
\draw[gray,dashed] ($(0*\kxstep,-0*\kystep)+(-\kmva,-\kmva)$)--++(\kxstep,0)--++(0,-\kystep+2*\kmva)--++(-\kxstep,0);
\draw[gray,dashed] ($(4*\kxstep,-0*\kystep)+(\kmva,-\kmva)$)--++(-\kxstep,0)--++(0,-\kystep+2*\kmva)
--++(\kxstep,0)coordinate(kend);
%\draw[gray,dashed] ($(0,-1*\kystep)+(-\kmv,\kmv)$)--++(1*\kxstep,0)--++(0,\kystep+\kmv);
%\draw[gray,dashed] ($(0,-3*\kystep)+(-\kmv,-\kmv)$)--++(1*\kxstep,0)--++(0,-\kystep-\kmv);
\draw[](4*\kxstep-\kmv,-1.5*\kystep) to [out=0,in=180]++(0.5,-0.25)node[right]{$y$};
\draw[](kend) to [out=0,in=180]++(0.5,0.25)node[right]{$\overline{x}\,\overline{z}$};
\end{tikzpicture}\quad 
\begin{tikzpicture}
\draw(0,0)node[]{$F(x,y,z)=\sum (m_0,m_2,m_3,m_6,m_7)$};
\end{tikzpicture}
\caption{سادہ مساوات سے ارکان ضرب کے مجموعہ کا حصول (مثال \حوالہ{مثال_بوولین_الٹ_رخ})۔}
\label{شکل_بوولین_الٹ_رخ}
\end{figure}
\انتہا{مثال}

%???KKK
\حصہ{ ضرب بعد از جمع کی شکل میں سادہ مساوات}
 کارناف نقشے کے ان خانوں میں \عددی{1} پُر کیا جاتا ہے جن میں تفاعل کے بوولین جدول میں ارکان ضرب کی قیمت \عددی{1} ہو۔تفاعل کے متمم کے بوولین جدول میں جہاں پہلے \عددی{0} تھا اب وہاں \عددی{1} ہو گا۔ اس جدول کے کارناف نقشے سے ارکان ضرب کے مجموعے کی مساوات، تفاعل کے متمم کی سادہ مساوات ہو گی۔یہ مساوات مجموعہ ارکان ضرب کے روپ میں ہو گی، جس کا متمم لے کر اصل تفاعل کی ( ضرب بعد از جمع کی شکل  میں) سادہ مساوات حاصل ہو گی۔ایک مثال سے اس بات کی وضاحت کرتے ہیں۔

\ابتدا{مثال}\شناخت{مثال_بوولین_متمم_حصول}
مندرجہ ذیل تفاعل کی  مجموعہ ارکان ضرب اور  ضرب بعد از جمع  شکل میں سادہ مساوات حاصل کریں۔
\begin{align*}
F(x,y,z)=\sum(m_2,m_3,m_4,m_5)
\end{align*}
\ترچھا{حل:}\quad
شکل \حوالہ{شکل_بوولین_متمم_سے_حصول}-الف میں تفاعل اور اس کے متمم کا جدول پیش کیا گیا ہے۔ ،شکل-ب میں تفاعل کی مساوات، ارکان ضرب کے مجموعہ کی صورت میں دی گئی ہے۔ شکل-ج میں دی گئی مساوات، تفاعل کے متمم کی ہے، جس کا متمم لے کر (اور بوولین کلیات استعمال کر کے) تفاعل کے ارکان جمع کی ضرب کی (درج ذیل) سادہ مساوات حاصل ہو گی۔
\begin{align*}
F=\overline{\overline{F}}&=\overline{\overline{x}\,\overline{y}+xy}\\
&=(\overline{\overline{x}\,\overline{y}})(\overline{xy})\\
&=(\overline{\overline{x}}+\overline{\overline{y}})(\overline{x}+\overline{y})\\
&=(x+y)(\overline{x}+\overline{y})
\end{align*}
%
\begin{figure}
\centering
\begin{subfigure}{0.45\textwidth}
\centering
\begin{otherlanguage}{english}
\begin{tabular}{CCC|CC}
\toprule
x&y&z&F&\overline{F}\\
\midrule
0&0&0&0&1\\
0&0&1&0&1\\
0&1&0&1&0\\
0&1&1&1&0\\
1&0&0&1&0\\
1&0&1&1&0\\
1&1&0&0&1\\
1&1&1&0&1\\
\bottomrule
\end{tabular}
\end{otherlanguage}
\caption{}
\end{subfigure}%
\begin{subfigure}{0.45\textwidth}
\centering
\begin{tikzpicture}
\pgfmathsetmacro{\kxstep}{1}
\pgfmathsetmacro{\kystep}{1}
\pgfmathsetmacro{\kpin}{0.75}
\pgfmathsetmacro{\kmv}{0.15}
\pgfmathsetmacro{\kmva}{0.10}
\draw[xstep=\kxstep,ystep=\kystep](0,0) grid (4*\kxstep,-2*\kystep);
%\draw(0,0)--++(135:\kpin)node[pos=0.75,above right]{$y$}node[pos=0.75,below left]{$x$};
\foreach \kx/\xlb in {0/{\overline{y}\,\overline{z}},1/{\overline{y}z},2/{yz},3/{y\overline{z}}}{\draw(\kx*\kxstep+\kxstep/2,0)node[above]{$\xlb$};}
\foreach \ky/\ylb in {0/{\overline{x}},1/{x}}{\draw(0,-\ky*\kystep-\kystep/2)node[left]{$\ylb$};}
\foreach \kx/\xlb in {0/{0},1/0,2/1,3/1}{\draw(\kx*\kxstep+\kxstep/2,-\kystep/2)node[]{$\xlb$};}
\foreach \kx/\xlb in {0/{1},1/1,2/0,3/0}{\draw(\kx*\kxstep+\kxstep/2,-1.5*\kystep)node[]{$\xlb$};}
%\foreach \kx/\xlb in {1/{1},2/1}{\draw(\kx*\kxstep+\kxstep/2,-2.5*\kystep)node[]{$\xlb$};}
%\foreach \kx/\xlb in {0/1,3/1}{\draw(\kx*\kxstep+\kxstep/2,-3.5*\kystep)node[]{$\xlb$};}
\draw[gray,dashed] ($(2*\kxstep,-0*\kystep)+(\kmv,-\kmv)$) rectangle ($(4*\kxstep,-1*\kystep)+(-\kmv,\kmv)$);
\draw[gray,dashed] ($(0*\kxstep,-1*\kystep)+(\kmv,-\kmv)$) rectangle ($(2*\kxstep,-2*\kystep)+(-\kmv,\kmv)$);
%\draw[gray,dashed] ($(0,-1*\kystep)+(-\kmv,\kmv)$)--++(1*\kxstep,0)--++(0,\kystep+\kmv);
%\draw[gray,dashed] ($(0,-3*\kystep)+(-\kmv,-\kmv)$)--++(1*\kxstep,0)--++(0,-\kystep-\kmv);
%\draw[gray,dashed] ($(3*\kxstep,-0*\kystep)+(\kmv,\kmv)$)--++(0,-1*\kystep)--++(\kxstep+\kmv,0);
%\draw[gray,dashed] ($(3*\kxstep,-4*\kystep)+(\kmv,-\kmv)$)--++(0,1*\kystep)--++(\kxstep+\kmv,0);
\draw(3*\kxstep,-2*\kystep-0.5)node[right]{\text{\RL{(ب)}}}node[left]{$F=\overline{x}y+x\overline{y}$};
\path(0,-3*\kystep)node[]{};
\end{tikzpicture}
\begin{tikzpicture}
\pgfmathsetmacro{\kxstep}{1}
\pgfmathsetmacro{\kystep}{1}
\pgfmathsetmacro{\kpin}{0.75}
\pgfmathsetmacro{\kmv}{0.15}
\pgfmathsetmacro{\kmva}{0.10}
\draw[xstep=\kxstep,ystep=\kystep](0,0) grid (4*\kxstep,-2*\kystep);
%\draw(0,0)--++(135:\kpin)node[pos=0.75,above right]{$y$}node[pos=0.75,below left]{$x$};
\foreach \kx/\xlb in {0/{\overline{y}\,\overline{z}},1/{\overline{y}z},2/{yz},3/{y\overline{z}}}{\draw(\kx*\kxstep+\kxstep/2,0)node[above]{$\xlb$};}
\foreach \ky/\ylb in {0/{\overline{x}},1/{x}}{\draw(0,-\ky*\kystep-\kystep/2)node[left]{$\ylb$};}
\foreach \kx/\xlb in {0/{1},1/1,2/0,3/0}{\draw(\kx*\kxstep+\kxstep/2,-\kystep/2)node[]{$\xlb$};}
\foreach \kx/\xlb in {0/{0},1/0,2/1,3/1}{\draw(\kx*\kxstep+\kxstep/2,-1.5*\kystep)node[]{$\xlb$};}
%\foreach \kx/\xlb in {1/{1},2/1}{\draw(\kx*\kxstep+\kxstep/2,-2.5*\kystep)node[]{$\xlb$};}
%\foreach \kx/\xlb in {0/1,3/1}{\draw(\kx*\kxstep+\kxstep/2,-3.5*\kystep)node[]{$\xlb$};}
\draw[gray,dashed] ($(2*\kxstep,-1*\kystep)+(\kmv,-\kmv)$) rectangle ($(4*\kxstep,-2*\kystep)+(-\kmv,\kmv)$);
\draw[gray,dashed] ($(0*\kxstep,-0*\kystep)+(\kmv,-\kmv)$) rectangle ($(2*\kxstep,-1*\kystep)+(-\kmv,\kmv)$);
%\draw[gray,dashed] ($(0,-1*\kystep)+(-\kmv,\kmv)$)--++(1*\kxstep,0)--++(0,\kystep+\kmv);
%\draw[gray,dashed] ($(0,-3*\kystep)+(-\kmv,-\kmv)$)--++(1*\kxstep,0)--++(0,-\kystep-\kmv);
%\draw[gray,dashed] ($(3*\kxstep,-0*\kystep)+(\kmv,\kmv)$)--++(0,-1*\kystep)--++(\kxstep+\kmv,0);
%\draw[gray,dashed] ($(3*\kxstep,-4*\kystep)+(\kmv,-\kmv)$)--++(0,1*\kystep)--++(\kxstep+\kmv,0);
\draw(3*\kxstep,-2*\kystep-0.5)node[right]{\text{\RL{(ج)}}}node[left]{$\overline{F}=\overline{x}\,\overline{y}+x y$};
\end{tikzpicture}
\end{subfigure}
\caption{مجموعہ ارکان ضرب اور  ضرب بعد از جمع کی شکل  میں سادہ مساوات (مثال \حوالہ{مثال_بوولین_متمم_حصول})۔}
\label{شکل_بوولین_متمم_سے_حصول}
\end{figure}
\انتہا{مثال}

\حصہ{غیر دلچسپ حال}
 ہم نے اب تک جتنے تفاعل دیکھے، ان میں مداخل کی تمام صورتوں کے مطابقتی مخارج دستیاب اور ضروری تھے۔بعض اوقات مداخل کی چند قیمتیں ممکن نہیں ہوں گی یا ان کے مطابقتی مخارج استعمال نہیں ہوں گے۔مداخل کے ان قیمتوں کو غیر دلچسپ حال کہتے ہیں۔ 

تفاعل کی سادہ مساوات حاصل کرتے وقت، کارناف نقشے کے غیر دلچسپ حال خانوں میں \عددی{0} یا \عددی{1} کی بجائے \عددی{d} درج کیا جاتا ہے۔قریبی خانے گھیرتے وقت اگر کسی غیر ضروری خانے میں \عددی{1} تصور کرنے سے زیادہ سادہ مساوات حاصل ہو تو اس خانے میں \عددی{1} تصور کیا جاتا ہے، اور اگر اس میں \عددی{0} تصور کرنے سے زیادہ سادہ مساوات حاصل ہوتی ہے تو اس میں \عددی{0} تصور کیا جاتا ہے۔


\ابتدا{مثال}\شناخت{مثال_بوولین_غیر_دلچسپ_پہلا}
درج ذیل تفاعل کی سادہ مساوات، مجموعہ ارکان ضرب اور  ضرب بعد از جمع کے روپ میں حاصل کریں۔
\begin{align*}
F(x,y)&=\sum (m_0,m_3)\\
d(x,y)&=\sum (m_2)
\end{align*}
%
\begin{figure}
\centering
\begin{subfigure}{0.33\textwidth}
\centering
\begin{otherlanguage}{english}
\begin{tabular}{CC|CC}
\toprule
x&y&F&\overline{F}\\
\midrule
0&0&0&1\\
0&1&1&0\\
1&0&d&d\\
1&1&1&0\\
\bottomrule
\end{tabular}
\end{otherlanguage}
\caption{}
\end{subfigure}\hfill
\begin{subfigure}{0.33\textwidth}
\centering
\begin{tikzpicture}
\pgfmathsetmacro{\kxstep}{1}
\pgfmathsetmacro{\kystep}{1}
\pgfmathsetmacro{\kpin}{0.75}
\pgfmathsetmacro{\kmv}{0.1}
\pgfmathsetmacro{\kmva}{0.20}
\draw[xstep=\kxstep,ystep=\kystep](0,0) grid (2*\kxstep,-2*\kystep);
%\draw(0,0)--++(135:\kpin)node[pos=0.75,above right]{$y$}node[pos=0.75,below left]{$x$};
\foreach \kx/\xlb in {0/{\overline{y}},1/y}{\draw(\kx*\kxstep+\kxstep/2,0)node[above]{$\xlb$};}
\foreach \ky/\ylb in {0/{\overline{x}},1/{x}}{\draw(0,-\ky*\kystep-\kystep/2)node[left]{$\ylb$};}
\foreach \kx/\xlb in {0/{1},1/0}{\draw(\kx*\kxstep+\kxstep/2,-\kystep/2)node[]{$\xlb$};}
\foreach \kx/\xlb in {0/{d},1/1}{\draw(\kx*\kxstep+\kxstep/2,-1.5*\kystep)node[]{$\xlb$};}
%\foreach \kx/\xlb in {1/{1},2/1}{\draw(\kx*\kxstep+\kxstep/2,-2.5*\kystep)node[]{$\xlb$};}
%\foreach \kx/\xlb in {0/1,3/1}{\draw(\kx*\kxstep+\kxstep/2,-3.5*\kystep)node[]{$\xlb$};}
\draw[gray,dashed] ($(0*\kxstep,-0*\kystep)+(\kmv,-\kmv)$) rectangle ($(1*\kxstep,-2*\kystep)+(-\kmv,\kmv)$);
\draw[gray,dashed] ($(0*\kxstep,-1*\kystep)+(\kmva,-\kmva)$) rectangle ($(2*\kxstep,-2*\kystep)+(-\kmva,\kmva)$);
%\draw[gray,dashed] ($(0,-1*\kystep)+(-\kmv,\kmv)$)--++(1*\kxstep,0)--++(0,\kystep+\kmv);
%\draw[gray,dashed] ($(0,-3*\kystep)+(-\kmv,-\kmv)$)--++(1*\kxstep,0)--++(0,-\kystep-\kmv);
%\draw[gray,dashed] ($(3*\kxstep,-0*\kystep)+(\kmv,\kmv)$)--++(0,-1*\kystep)--++(\kxstep+\kmv,0);
%\draw[gray,dashed] ($(3*\kxstep,-4*\kystep)+(\kmv,-\kmv)$)--++(0,1*\kystep)--++(\kxstep+\kmv,0);
\draw(1*\kxstep,-2*\kystep-0.5)node[]{$F=\overline{y}+x$};
\end{tikzpicture}
\caption{}
\end{subfigure}\hfill
\begin{subfigure}{0.33\textwidth}
\centering
\begin{tikzpicture}
\pgfmathsetmacro{\kxstep}{1}
\pgfmathsetmacro{\kystep}{1}
\pgfmathsetmacro{\kpin}{0.75}
\pgfmathsetmacro{\kmv}{0.15}
\pgfmathsetmacro{\kmva}{0.10}
\draw[xstep=\kxstep,ystep=\kystep](0,0) grid (2*\kxstep,-2*\kystep);
%\draw(0,0)--++(135:\kpin)node[pos=0.75,above right]{$y$}node[pos=0.75,below left]{$x$};
\foreach \kx/\xlb in {0/{\overline{y}},1/y}{\draw(\kx*\kxstep+\kxstep/2,0)node[above]{$\xlb$};}
\foreach \ky/\ylb in {0/{\overline{x}},1/{x}}{\draw(0,-\ky*\kystep-\kystep/2)node[left]{$\ylb$};}
\foreach \kx/\xlb in {0/{1},1/0}{\draw(\kx*\kxstep+\kxstep/2,-\kystep/2)node[]{$\xlb$};}
\foreach \kx/\xlb in {0/{d},1/1}{\draw(\kx*\kxstep+\kxstep/2,-1.5*\kystep)node[]{$\xlb$};}
%\foreach \kx/\xlb in {1/{1},2/1}{\draw(\kx*\kxstep+\kxstep/2,-2.5*\kystep)node[]{$\xlb$};}
%\foreach \kx/\xlb in {0/1,3/1}{\draw(\kx*\kxstep+\kxstep/2,-3.5*\kystep)node[]{$\xlb$};}
\draw[gray,dashed] ($(1*\kxstep,-0*\kystep)+(\kmv,-\kmv)$) rectangle ($(2*\kxstep,-1*\kystep)+(-\kmv,\kmv)$);
%\draw[gray,dashed] ($(0*\kxstep,-0*\kystep)+(\kmv,-\kmv)$) rectangle ($(2*\kxstep,-1*\kystep)+(-\kmv,\kmv)$);
%\draw[gray,dashed] ($(0,-1*\kystep)+(-\kmv,\kmv)$)--++(1*\kxstep,0)--++(0,\kystep+\kmv);
%\draw[gray,dashed] ($(0,-3*\kystep)+(-\kmv,-\kmv)$)--++(1*\kxstep,0)--++(0,-\kystep-\kmv);
%\draw[gray,dashed] ($(3*\kxstep,-0*\kystep)+(\kmv,\kmv)$)--++(0,-1*\kystep)--++(\kxstep+\kmv,0);
%\draw[gray,dashed] ($(3*\kxstep,-4*\kystep)+(\kmv,-\kmv)$)--++(0,1*\kystep)--++(\kxstep+\kmv,0);
\draw(1*\kxstep,-2*\kystep-0.5)node[]{$F=x+\overline{y}$};
\end{tikzpicture}
\caption{}
\end{subfigure}
\caption{غیر دلچسپ حال (مثال \حوالہ{مثال_بوولین_غیر_دلچسپ_پہلا})۔}
\label{شکل_بوولین_غیر_دلچسپ_پہلا}
\end{figure}
\ترچھا{حل:}\quad
 تفاعل کا ایک حال غیر دلچسپ ہے۔ شکل \حوالہ{شکل_بوولین_غیر_دلچسپ_پہلا} میں تفاعل کا بوولین جدول اور کارناف نقشے دکھائے گئے ہیں۔مجموعہ ارکان ضرب کے روپ میں سادہ مساوات حاصل کرتے وقت غیر دلچسپ خانے کی قیمت \عددی{1} تصور کرنے سے (زیادہ) سادہ مساوات حاصل ہو گی (شکل-ب)۔  ضرب بعد از جمع کے روپ میں بھی غیر دلچسپ خانے کی قیمت \عددی{1} تصور کرنے سے (زیادہ) سادہ مساوات حاصل ہو گی (شکل-ج)۔
\انتہا{مثال}
\ابتدا{مثال}\شناخت{مثال_بوولین_غیر_دلچسپ_دوسرا}
درج ذیل تفاعل کی سادہ مساوات حاصل کریں۔	
\begin{align*}
F(w,x,y,z)&=\sum(m_0,m_2,m_8,m_9,m_{12},m_{13},m_{15})\\
d(w,x,y,z)&=\sum(m_1,m_3,m_{11})
\end{align*}
%
\begin{figure}
\centering
\begin{tikzpicture}
\pgfmathsetmacro{\kxstep}{1}
\pgfmathsetmacro{\kystep}{1}
\pgfmathsetmacro{\kpin}{0.75}
\pgfmathsetmacro{\kmv}{0.15}
\pgfmathsetmacro{\kmva}{0.10}
\draw[xstep=\kxstep,ystep=\kystep](0,0) grid (4*\kxstep,-4*\kystep);
%\draw(0,0)--++(135:\kpin)node[pos=0.75,above right]{$y$}node[pos=0.75,below left]{$x$};
\foreach \kx/\xlb in {0/{\overline{y}\,\overline{z}},1/{\overline{y}z},2/{yz},3/{y\overline{z}}}{\draw(\kx*\kxstep+\kxstep/2,0)node[above]{$\xlb$};}
\foreach \ky/\ylb in {0/{\overline{w}\,\overline{x}},1/{\overline{w}x},2/{wx},3/{w\overline{x}}}{\draw(0,-\ky*\kystep-\kystep/2)node[left]{$\ylb$};}
\foreach \kx/\xlb in {0/{1},1/d,2/d,3/1}{\draw(\kx*\kxstep+\kxstep/2,-\kystep/2)node[]{$\xlb$};}
\foreach \kx/\xlb in {1/{d}}{\draw(\kx*\kxstep+\kxstep/2,-1.5*\kystep)node[]{$\xlb$};}
\foreach \kx/\xlb in {0/{1},1/1}{\draw(\kx*\kxstep+\kxstep/2,-2.5*\kystep)node[]{$\xlb$};}
\foreach \kx/\xlb in {0/1,1/1,2/d}{\draw(\kx*\kxstep+\kxstep/2,-3.5*\kystep)node[]{$\xlb$};}
\draw[gray,dashed] ($(0*\kxstep,-0*\kystep)+(\kmv,-\kmv)$) rectangle ($(4*\kxstep,-1*\kystep)+(-\kmv,\kmv)$);
\draw[gray,dashed] ($(0*\kxstep,-2*\kystep)+(\kmv,-\kmv)$) rectangle ($(2*\kxstep,-4*\kystep)+(-\kmv,\kmv)$);
%\draw[gray,dashed] ($(0,-1*\kystep)+(-\kmv,\kmv)$)--++(1*\kxstep,0)--++(0,\kystep+\kmv);
%\draw[gray,dashed] ($(0,-3*\kystep)+(-\kmv,-\kmv)$)--++(1*\kxstep,0)--++(0,-\kystep-\kmv);
%\draw[gray,dashed] ($(3*\kxstep,-0*\kystep)+(\kmv,\kmv)$)--++(0,-1*\kystep)--++(\kxstep+\kmv,0);
%\draw[gray,dashed] ($(3*\kxstep,-4*\kystep)+(\kmv,-\kmv)$)--++(0,1*\kystep)--++(\kxstep+\kmv,0);
\draw(4*\kxstep-\kmv,-0.5*\kystep) to [out=0,in=180]++(0.5,-0.25)node[right]{$\overline{w}\,\overline{x}$};
\draw(1.5*\kxstep-\kmv,-4*\kystep+\kmv) to [out=-90,in=180]++(0.5,-0.5)node[right]{$w\overline{y}$};
\draw(4*\kxstep+0.5,-2*\kystep)node[right]{$F(w,x,y,z)=w\overline{y}+\overline{w}\,\overline{x}$};
\end{tikzpicture}
\caption{غیر دلچسپ حالات (مثال \حوالہ{مثال_بوولین_غیر_دلچسپ_دوسرا})۔}
\label{شکل_بوولین_غیر_دلچسپ_دوسرا}
\end{figure}
\ترچھا{حل:}\quad
شکل \حوالہ{شکل_بوولین_غیر_دلچسپ_دوسرا} میں کارناف نقشہ پیش کیا گیا ہے۔ سادہ مساوات کے حصول میں (بالائی صف کے) دو غیر دلچسپ خانوں کی قیمت \عددی{1} ، جبکہ باقی دو غیر دلچسپ خانوں کی قیمت \عددی{0} تصور کی گئی۔کارناف نقشے میں \عددی{0} کو نظر پوش کیا گیا ہے۔ تفاعل کی مساوات شکل میں دی گئی ہے۔
\انتہا{مثال}

%=============================
\حصہء{سوالات}
\ابتدا{سوال}
جدول صداقت  میں  چار داخلی متغیرات ہیں۔ ابتدائی آٹھ مخارج \عددی{0} اور آخری آٹھ \عددی{1} ہیں۔ اس کا کارناف نقشہ کھینچیں۔

جواب:\quad
\begin{center}
\centering
\begin{tikzpicture}
\pgfmathsetmacro{\kxstep}{1}
\pgfmathsetmacro{\kystep}{1}
\pgfmathsetmacro{\kpin}{0.75}
\draw[xstep=\kxstep,ystep=\kystep](0,0) grid (4*\kxstep,-4*\kystep);
\draw(0,0)--++(135:\kpin)node[pos=0.75,above right]{$CD$}node[pos=0.75,below left]{$AB$};
\foreach \kx/\xlb in {0/{00},1/{01},2/{11},3/{10}}{\draw(\kx*\kxstep+\kxstep/2,0)node[above]{$\xlb$};}
\foreach \ky/\ylb in {0/{00},1/{01},2/{11},3/{10}}{\draw(0,-\ky*\kystep-\kystep/2)node[left]{$\ylb$};}
\foreach \kx/\xlb in {0/0,1/0,2/0,3/0}{\draw(\kx*\kxstep+\kxstep/2,-\kystep/2)node[]{$\xlb$};}
\foreach \kx/\xlb in {0/0,1/0,2/0,3/0}{\draw(\kx*\kxstep+\kxstep/2,-1.5*\kystep)node[]{$\xlb$};}
\foreach \kx/\xlb in {0/1,1/1,2/1,3/1}{\draw(\kx*\kxstep+\kxstep/2,-2.5*\kystep)node[]{$\xlb$};}
\foreach \kx/\xlb in {0/1,1/1,2/1,3/1}{\draw(\kx*\kxstep+\kxstep/2,-3.5*\kystep)node[]{$\xlb$};}
\end{tikzpicture}
\end{center}

\انتہا{سوال}
%---------------------
\ابتدا{سوال}
جدول \حوالہ{جدول_کارناف_سوالات_الف}-الف  کے  \عددی{Y_3} کا کارناف نقشہ  بنائیں۔ اس سے سادہ ترین مساوات حاصل کر کے عددی دور تخلیق دیں۔
\begin{table}
\centering
\caption{تفاعلات کے  جدول }
\label{جدول_کارناف_سوالات_الف}
\begin{subtable}{0.45\textwidth}
\caption{}
\centering
\begin{otherlanguage}{english}
\begin{tabular}{CCCC|CCCC}
\toprule
A&B&C&D&Y_3&Y_2&Y_1&Y_0\\
\midrule
0&0&0&0&1&0&1&0\\
0&0&0&1&0&1&0&1\\
0&0&1&0&0&1&1&1\\
0&0&1&1&1&0&0&1\\
0&1&0&0&0&0&1&1\\
0&1&0&1&1&0&0&0\\
0&1&1&0&1&1&1&0\\
0&1&1&1&1&1&1&1\\
1&0&0&0&0&0&0&0\\
1&0&0&1&0&0&0&1\\
1&0&1&0&1&0&1&1\\
1&0&1&1&0&1&0&0\\
1&1&0&0&0&1&1&0\\
1&1&0&1&1&0&1&0\\
1&1&1&0&1&1&0&0\\
1&1&1&1&1&1&0&1\\
\bottomrule
\end{tabular}
\end{otherlanguage}
\end{subtable}\hfill
\begin{subtable}{0.45\textwidth}
\caption{}
\centering
\begin{otherlanguage}{english}
\begin{tabular}{CCCC|CCCC}
\toprule
A&B&C&D&Y_3&Y_2&Y_1&Y_0\\
\midrule
0&0&0&0&1&0&1&0\\
0&0&0&1&0&1&0&1\\
0&0&1&0&0&1&1&1\\
0&0&1&1&1&0&0&1\\
0&1&0&0&0&0&1&1\\
0&1&0&1&1&0&0&0\\
0&1&1&0&1&1&1&0\\
0&1&1&1&1&1&1&1\\
1&0&0&0&0&0&0&0\\
1&0&0&1&0&0&0&1\\
1&0&1&0&x&x&x&x\\
1&0&1&1&x&x&x&x\\
1&1&0&0&x&x&x&x\\
1&1&0&1&x&x&x&x\\
1&1&1&0&x&x&x&x\\
1&1&1&1&x&x&x&x\\
\bottomrule
\end{tabular}
\end{otherlanguage}
\end{subtable}
\end{table}
\انتہا{سوال}
%-----------------------
\ابتدا{سوال}
جدول \حوالہ{جدول_کارناف_سوالات_الف}-الف  کے  \عددی{Y_2}   مخارج کا کارناف نقشہ  بنا کر  سادہ ترین عددی دور تخلیق دیں۔


جواب:\quad

\begin{minipage}{0.45\textwidth}
\centering
\begin{tikzpicture}
\pgfmathsetmacro{\kxstep}{1}
\pgfmathsetmacro{\kystep}{1}
\pgfmathsetmacro{\kpin}{0.75}
\pgfmathsetmacro{\kmv}{0.15}
\pgfmathsetmacro{\kmva}{0.10}
\draw[xstep=\kxstep,ystep=\kystep](0,0) grid (4*\kxstep,-4*\kystep);
\draw(0,0)--++(135:\kpin)node[pos=0.75,above right]{$CD$}node[pos=0.75,below left]{$AB$};
\foreach \kx/\xlb in {0/{00},1/{01},2/{11},3/{10}}{\draw(\kx*\kxstep+\kxstep/2,0)node[above]{$\xlb$};}
\foreach \ky/\ylb in {0/{00},1/{01},2/{11},3/{10}}{\draw(0,-\ky*\kystep-\kystep/2)node[left]{$\ylb$};}
\foreach \kx/\xlb in {0/0,1/1,2/0,3/1}{\draw(\kx*\kxstep+\kxstep/2,-\kystep/2)node[]{$\xlb$};}
\foreach \kx/\xlb in {0/0,1/0,2/1,3/1}{\draw(\kx*\kxstep+\kxstep/2,-1.5*\kystep)node[]{$\xlb$};}
\foreach \kx/\xlb in {0/1,1/0,2/1,3/1}{\draw(\kx*\kxstep+\kxstep/2,-2.5*\kystep)node[]{$\xlb$};}
\foreach \kx/\xlb in {0/0,1/0,2/1,3/0}{\draw(\kx*\kxstep+\kxstep/2,-3.5*\kystep)node[]{$\xlb$};}
\draw[gray,dashed] ($(1*\kxstep,-0*\kystep)+(\kmv,-\kmv)$) rectangle ($(2*\kxstep,-1*\kystep)+(-\kmv,\kmv)$);
\draw[gray,dashed] ($(3*\kxstep,-0*\kystep)+(\kmv,-\kmv)$) rectangle ($(4*\kxstep,-2*\kystep)+(-\kmv,\kmv)$);
\draw[gray,dashed] ($(2*\kxstep,-1*\kystep)+(\kmva,-\kmva)$) rectangle ($(4*\kxstep,-3*\kystep)+(-\kmva,\kmva)$);
\draw[gray,dashed] ($(2*\kxstep,-2*\kystep)+(\kmv,-\kmv)$) rectangle ($(3*\kxstep,-4*\kystep)+(-\kmv,\kmv)$);
\draw[gray,dashed] ($(0*\kxstep,-2*\kystep)+(-\kmv,-\kmv)$) --++(1*\kxstep,-0*\kystep)--++(0,-\kystep+2.5*\kmv)--++(-\kxstep,0);
\draw[gray,dashed] ($(4*\kxstep,-2*\kystep)+(\kmv,-\kmv)$) --++(-1*\kxstep,-0*\kystep)--++(0,-\kystep+2.5*\kmv)--++(\kxstep,0);
\end{tikzpicture}
\end{minipage}\hfill
\begin{minipage}{0.55\textwidth}
\centering
\begin{tikzpicture}
\pgfmathsetmacro{\kxstep}{1}
\pgfmathsetmacro{\kystep}{1}
\pgfmathsetmacro{\kpin}{0.75}
\pgfmathsetmacro{\ksepX}{0.25}
\pgfmathsetmacro{\ksepY}{1}
\pgfmathsetmacro{\kW}{0.3}
\draw(0,0)node[or  port,number inputs=5,anchor=out](u1){};
\draw(u1.out)node[right]{$Y$};
\draw[thin](u1.in 1)--++(0,2*\ksepY)--++(-\ksepX,0)node[and port,anchor=out](u2){};
\draw[thin](u1.in 2)--++(-\ksepX,0)--++(0,\ksepY)node[and port,anchor=out,number inputs=3](u3){};
\draw[thin](u1.in 3)--++(-\ksepX,0)node[and port,anchor=out,number inputs=3](u4){};
\draw[thin](u1.in 4)--++(-\ksepX,0)--++(0,-\ksepY)node[and port,anchor=out,number inputs=3](u5){};
\draw[thin](u1.in 5)--++(0,-2*\ksepY)--++(-\ksepX,0)node[and port,anchor=out,number inputs=4](u6){};
\foreach \n/\lbl in {0/{\overline{D}},1/{D},2/{\overline{C}},3/{C},4/{\overline{B}},5/{B},6/{\overline{A}},7/{A}}\draw[thin](u2.in 1)++(-\ksepX-\n*\kW,\kW)node[above]{$\lbl$}coordinate(w\n)coordinate(kA)--($(kA|-u6.in 4)+(0,-\kW)$);
\draw(u2.in 1)coordinate(kA)--(kA-| w5);
\draw(u2.in 2)coordinate(kA)--(kA-| w3);
\draw(u3.in 1)coordinate(kA)--(kA-| w6);
\draw(u3.in 2)coordinate(kA)--(kA-| w3);
\draw(u3.in 3)coordinate(kA)--(kA-| w0);
\draw(u4.in 1)coordinate(kA)--(kA-| w7);
\draw(u4.in 2)coordinate(kA)--(kA-| w3);
\draw(u4.in 3)coordinate(kA)--(kA-| w1);
\draw(u5.in 1)coordinate(kA)--(kA-| w7);
\draw(u5.in 2)coordinate(kA)--(kA-| w5);
\draw(u5.in 3)coordinate(kA)--(kA-| w0);
\draw(u6.in 1)coordinate(kA)--(kA-| w6);
\draw(u6.in 2)coordinate(kA)--(kA-| w4);
\draw(u6.in 3)coordinate(kA)--(kA-| w2);
\draw(u6.in 4)coordinate(kA)--(kA-| w1);
\end{tikzpicture}
\end{minipage}
\انتہا{سوال}
%-------------------------------
\ابتدا{سوال}
جدول \حوالہ{جدول_کارناف_سوالات_الف}-الف  کے  \عددی{Y_1}   مخارج کا کارناف نقشہ  بنا کر  سادہ ترین عددی دور تخلیق دیں۔
\انتہا{سوال}
%-------------------------
\ابتدا{سوال}
جدول \حوالہ{جدول_کارناف_سوالات_الف}-الف  کے  \عددی{Y_0}   مخارج کا کارناف نقشہ  بنا کر  سادہ ترین عددی دور تخلیق دیں۔

جواب:

\begin{minipage}{0.45\textwidth}
\centering
\begin{tikzpicture}
\pgfmathsetmacro{\kxstep}{1}
\pgfmathsetmacro{\kystep}{1}
\pgfmathsetmacro{\kpin}{0.75}
\pgfmathsetmacro{\kmv}{0.15}
\pgfmathsetmacro{\kmva}{0.10}
\draw[xstep=\kxstep,ystep=\kystep](0,0) grid (4*\kxstep,-4*\kystep);
\draw(0,0)--++(135:\kpin)node[pos=0.75,above right]{$CD$}node[pos=0.75,below left]{$AB$};
\foreach \kx/\xlb in {0/{00},1/{01},2/{11},3/{10}}{\draw(\kx*\kxstep+\kxstep/2,0)node[above]{$\xlb$};}
\foreach \ky/\ylb in {0/{00},1/{01},2/{11},3/{10}}{\draw(0,-\ky*\kystep-\kystep/2)node[left]{$\ylb$};}
\foreach \kx/\xlb in {0/0,1/1,2/1,3/1}{\draw(\kx*\kxstep+\kxstep/2,-\kystep/2)node[]{$\xlb$};}
\foreach \kx/\xlb in {0/1,1/0,2/1,3/0}{\draw(\kx*\kxstep+\kxstep/2,-1.5*\kystep)node[]{$\xlb$};}
\foreach \kx/\xlb in {0/0,1/0,2/1,3/0}{\draw(\kx*\kxstep+\kxstep/2,-2.5*\kystep)node[]{$\xlb$};}
\foreach \kx/\xlb in {0/0,1/1,2/0,3/1}{\draw(\kx*\kxstep+\kxstep/2,-3.5*\kystep)node[]{$\xlb$};}
\draw[gray,dashed] ($(0*\kxstep,-1*\kystep)+(\kmv,-\kmv)$) rectangle ($(1*\kxstep,-2*\kystep)+(-\kmv,\kmv)$);
\draw[gray,dashed] ($(2*\kxstep,-1*\kystep)+(\kmv,-\kmv)$) rectangle ($(3*\kxstep,-3*\kystep)+(-\kmv,\kmv)$);
\draw[gray,dashed] ($(2*\kxstep,-0*\kystep)+(\kmva,-\kmva)$) rectangle ($(4*\kxstep,-1*\kystep)+(-\kmva,\kmva)$);
\draw[gray,dashed] ($(1*\kxstep,-0*\kystep)+(\kmv,\kmv)$) --++(0*\kxstep,-1*\kystep) --++(1*\kxstep-2*\kmv,-0*\kystep+0*\kmv) --++(-0*\kxstep,1*\kystep);
\draw[gray,dashed] ($(1*\kxstep,-4*\kystep)+(\kmv,-\kmv)$) --++(0*\kxstep,1*\kystep)--++(1*\kxstep-2*\kmv,-0*\kystep+0*\kmv)--++(-0*\kxstep,-1*\kystep);
\draw[gray,dashed] ($(3*\kxstep,-0*\kystep)+(\kmv,\kmv)$) --++(0*\kxstep,-1*\kystep+\kmv) --++(1*\kxstep-2.5*\kmv,-0*\kystep+0*\kmv) --++(-0*\kxstep,1*\kystep);
\draw[gray,dashed] ($(3*\kxstep,-4*\kystep)+(\kmv,-\kmv)$) --++(0*\kxstep,1*\kystep)--++(1*\kxstep-2*\kmv,-0*\kystep+0*\kmv)--++(-0*\kxstep,-1*\kystep);
%\draw[gray,dashed] ($(4*\kxstep,-2*\kystep)+(\kmv,-\kmv)$) --++(-1*\kxstep,-0*\kystep)--++(0,-\kystep+2.5*\kmv)--++(\kxstep,0);
\end{tikzpicture}
\end{minipage}\hfill
\begin{minipage}{0.55\textwidth}
\centering
\begin{tikzpicture}
\pgfmathsetmacro{\kxstep}{1}
\pgfmathsetmacro{\kystep}{1}
\pgfmathsetmacro{\kpin}{0.75}
\pgfmathsetmacro{\ksepX}{0.25}
\pgfmathsetmacro{\ksepY}{1}
\pgfmathsetmacro{\kW}{0.3}
\draw(0,0)node[or  port,number inputs=5,anchor=out](u1){};
\draw(u1.out)node[right]{$Y$};
\draw[thin](u1.in 1)--++(0,2*\ksepY)--++(-\ksepX,0)node[and port,anchor=out,number inputs=4](u2){};
\draw[thin](u1.in 2)--++(-\ksepX,0)--++(0,\ksepY)node[and port,anchor=out,number inputs=3](u3){};
\draw[thin](u1.in 3)--++(-\ksepX,0)node[and port,anchor=out,number inputs=3](u4){};
\draw[thin](u1.in 4)--++(-\ksepX,0)--++(0,-\ksepY)node[and port,anchor=out,number inputs=3](u5){};
\draw[thin](u1.in 5)--++(0,-2*\ksepY)--++(-\ksepX,0)node[and port,anchor=out,number inputs=3](u6){};
\foreach \n/\lbl in {0/{\overline{D}},1/{D},2/{\overline{C}},3/{C},4/{\overline{B}},5/{B},6/{\overline{A}},7/{A}}\draw[thin](u2.in 1)++(-\ksepX-\n*\kW,\kW)node[above]{$\lbl$}coordinate(w\n)coordinate(kA)--($(kA|-u6.in 3)+(0,-\kW)$);
\draw(u2.in 1)coordinate(kA)--(kA-| w6);
\draw(u2.in 2)coordinate(kA)--(kA-| w5);
\draw(u2.in 3)coordinate(kA)--(kA-| w2);
\draw(u2.in 4)coordinate(kA)--(kA-| w0);
\draw(u3.in 1)coordinate(kA)--(kA-| w5);
\draw(u3.in 2)coordinate(kA)--(kA-| w3);
\draw(u3.in 3)coordinate(kA)--(kA-| w1);
\draw(u4.in 1)coordinate(kA)--(kA-| w6);
\draw(u4.in 2)coordinate(kA)--(kA-| w4);
\draw(u4.in 3)coordinate(kA)--(kA-| w3);
\draw(u5.in 1)coordinate(kA)--(kA-| w4);
\draw(u5.in 2)coordinate(kA)--(kA-| w2);
\draw(u5.in 3)coordinate(kA)--(kA-| w1);
\draw(u6.in 1)coordinate(kA)--(kA-| w4);
\draw(u6.in 2)coordinate(kA)--(kA-| w3);
\draw(u6.in 3)coordinate(kA)--(kA-| w0);
\end{tikzpicture}
\end{minipage}
\انتہا{سوال}
%-------------------------
\ابتدا{سوال}
جدول \حوالہ{جدول_کارناف_سوالات_الف}-ب  کے  \عددی{Y_3}   مخارج کا کارناف نقشہ  بنا کر  سادہ ترین عددی دور تخلیق دیں۔
\انتہا{سوال}
%-------------------------
\ابتدا{سوال}
جدول \حوالہ{جدول_کارناف_سوالات_الف}-ب کے  \عددی{Y_2}   مخارج کا کارناف نقشہ  بنا کر  سادہ ترین عددی دور تخلیق دیں۔

جواب:


\begin{minipage}{0.45\textwidth}
\centering
\begin{tikzpicture}
\pgfmathsetmacro{\kxstep}{1}
\pgfmathsetmacro{\kystep}{1}
\pgfmathsetmacro{\kpin}{0.75}
\pgfmathsetmacro{\kmv}{0.15}
\pgfmathsetmacro{\kmva}{0.10}
\draw[xstep=\kxstep,ystep=\kystep](0,0) grid (4*\kxstep,-4*\kystep);
\draw(0,0)--++(135:\kpin)node[pos=0.75,above right]{$CD$}node[pos=0.75,below left]{$AB$};
\foreach \kx/\xlb in {0/{00},1/{01},2/{11},3/{10}}{\draw(\kx*\kxstep+\kxstep/2,0)node[above]{$\xlb$};}
\foreach \ky/\ylb in {0/{00},1/{01},2/{11},3/{10}}{\draw(0,-\ky*\kystep-\kystep/2)node[left]{$\ylb$};}
\foreach \kx/\xlb in {0/0,1/1,2/0,3/1}{\draw(\kx*\kxstep+\kxstep/2,-\kystep/2)node[]{$\xlb$};}
\foreach \kx/\xlb in {0/0,1/0,2/1,3/1}{\draw(\kx*\kxstep+\kxstep/2,-1.5*\kystep)node[]{$\xlb$};}
\foreach \kx/\xlb in {0/x,1/x,2/x,3/x}{\draw(\kx*\kxstep+\kxstep/2,-2.5*\kystep)node[]{$\xlb$};}
\foreach \kx/\xlb in {0/0,1/0,2/x,3/x}{\draw(\kx*\kxstep+\kxstep/2,-3.5*\kystep)node[]{$\xlb$};}
\draw[gray,dashed] ($(1*\kxstep,-0*\kystep)+(\kmv,-\kmv)$) rectangle ($(2*\kxstep,-1*\kystep)+(-\kmv,\kmv)$);
\draw[gray,dashed] ($(2*\kxstep,-1*\kystep)+(\kmv,-\kmv)$) rectangle ($(4*\kxstep,-3*\kystep)+(-1.5*\kmv,\kmv)$);
\draw[gray,dashed] ($(3*\kxstep,-0*\kystep)+(\kmva,-\kmva)$) rectangle ($(4*\kxstep,-4*\kystep)+(-\kmva,\kmva)$);
%\draw[gray,dashed] ($(4*\kxstep,-2*\kystep)+(\kmv,-\kmv)$) --++(-1*\kxstep,-0*\kystep)--++(0,-\kystep+2.5*\kmv)--++(\kxstep,0);
\end{tikzpicture}
\end{minipage}\hfill
\begin{minipage}{0.55\textwidth}
\centering
\begin{tikzpicture}
\pgfmathsetmacro{\kxstep}{1}
\pgfmathsetmacro{\kystep}{1}
\pgfmathsetmacro{\kpin}{0.75}
\pgfmathsetmacro{\ksepX}{0.25}
\pgfmathsetmacro{\ksepY}{1}
\pgfmathsetmacro{\kW}{0.3}
\draw(0,0)node[or  port,number inputs=3,anchor=out](u1){};
\draw(u1.out)node[right]{$Y$};
\draw[thin](u1.in 1)--++(0,1*\ksepY)--++(-\ksepX,0)node[and port,anchor=out,number inputs=4](u2){};
\draw[thin](u1.in 2)--++(-\ksepX,0)node[and port,anchor=out,number inputs=2](u3){};
\draw[thin](u1.in 3)--++(0,-1*\ksepY)--++(-\ksepX,0)node[and port,anchor=out,number inputs=2](u4){};
\foreach \n/\lbl in {0/{\overline{D}},1/{D},2/{\overline{C}},3/{C},4/{\overline{B}},5/{B},6/{\overline{A}},7/{A}}\draw[thin](u2.in 1)++(-\ksepX-\n*\kW,\kW)node[above]{$\lbl$}coordinate(w\n)coordinate(kA)--($(kA|-u4.in 2)+(0,-\kW)$);
\draw(u2.in 1)coordinate(kA)--(kA-| w6);
\draw(u2.in 2)coordinate(kA)--(kA-| w4);
\draw(u2.in 3)coordinate(kA)--(kA-| w2);
\draw(u2.in 4)coordinate(kA)--(kA-| w1);
\draw(u3.in 1)coordinate(kA)--(kA-| w5);
\draw(u3.in 2)coordinate(kA)--(kA-| w3);
\draw(u4.in 1)coordinate(kA)--(kA-| w3);
\draw(u4.in 2)coordinate(kA)--(kA-| w0);
\end{tikzpicture}
\end{minipage}
\انتہا{سوال}
%-------------------------
\ابتدا{سوال}
جدول \حوالہ{جدول_کارناف_سوالات_الف}-ب کے  \عددی{Y_1}   مخارج کا کارناف نقشہ  بنا کر  سادہ ترین عددی دور تخلیق دیں۔
\انتہا{سوال}
%-------------------------
\ابتدا{سوال}
جدول \حوالہ{جدول_کارناف_سوالات_الف}-ب کے  \عددی{Y_0}   مخارج کا کارناف نقشہ  بنا کر  سادہ ترین عددی دور تخلیق دیں۔

جواب:

\begin{minipage}{0.45\textwidth}
\centering
\begin{tikzpicture}
\pgfmathsetmacro{\kxstep}{1}
\pgfmathsetmacro{\kystep}{1}
\pgfmathsetmacro{\kpin}{0.75}
\pgfmathsetmacro{\kmv}{0.15}
\pgfmathsetmacro{\kmva}{0.10}
\draw[xstep=\kxstep,ystep=\kystep](0,0) grid (4*\kxstep,-4*\kystep);
\draw(0,0)--++(135:\kpin)node[pos=0.75,above right]{$CD$}node[pos=0.75,below left]{$AB$};
\foreach \kx/\xlb in {0/{00},1/{01},2/{11},3/{10}}{\draw(\kx*\kxstep+\kxstep/2,0)node[above]{$\xlb$};}
\foreach \ky/\ylb in {0/{00},1/{01},2/{11},3/{10}}{\draw(0,-\ky*\kystep-\kystep/2)node[left]{$\ylb$};}
\foreach \kx/\xlb in {0/0,1/1,2/1,3/1}{\draw(\kx*\kxstep+\kxstep/2,-\kystep/2)node[]{$\xlb$};}
\foreach \kx/\xlb in {0/1,1/0,2/1,3/0}{\draw(\kx*\kxstep+\kxstep/2,-1.5*\kystep)node[]{$\xlb$};}
\foreach \kx/\xlb in {0/x,1/x,2/x,3/x}{\draw(\kx*\kxstep+\kxstep/2,-2.5*\kystep)node[]{$\xlb$};}
\foreach \kx/\xlb in {0/0,1/1,2/x,3/x}{\draw(\kx*\kxstep+\kxstep/2,-3.5*\kystep)node[]{$\xlb$};}
\draw[gray,dashed] ($(0*\kxstep,-1*\kystep)+(\kmv,-\kmv)$) rectangle ($(1*\kxstep,-3*\kystep)+(-\kmv,\kmv)$);
\draw[gray,dashed] ($(2*\kxstep,-0*\kystep)+(\kmv,-\kmv)$) rectangle ($(3*\kxstep,-4*\kystep)+(-\kmv,\kmv)$);
\draw[gray,dashed] ($(1*\kxstep,-0*\kystep)+(\kmv,\kmv)$) --++(0*\kxstep,-1*\kystep) --++(2*\kxstep-2.5*\kmv,-0*\kystep+0*\kmv) --++(-0*\kxstep,1*\kystep);
\draw[gray,dashed] ($(1*\kxstep,-4*\kystep)+(\kmv,-1.5*\kmv)$) --++(0*\kxstep,1*\kystep) --++(2*\kxstep-2.5*\kmv,-0*\kystep+0*\kmv) --++(-0*\kxstep,-1*\kystep);
\draw[gray,dashed] ($(2*\kxstep,-0*\kystep)+(1.5*\kmv,1.5*\kmv)$) --++(0*\kxstep,-1*\kystep-2.5*\kmv) --++(2*\kxstep-2.5*\kmv,-0*\kystep+0*\kmv) --++(-0*\kxstep,1*\kystep+2.5*\kmv);
\draw[gray,dashed] ($(2*\kxstep,-4*\kystep)+(1.5*\kmv,-1.5*\kmv)$) --++(0*\kxstep,1*\kystep+1*\kmv) --++(2*\kxstep-2.5*\kmv,-0*\kystep+0*\kmv) --++(-0*\kxstep,-1*\kystep-1*\kmv);
\end{tikzpicture}
\end{minipage}\hfill
\begin{minipage}{0.55\textwidth}
\centering
\begin{tikzpicture}
\pgfmathsetmacro{\kxstep}{1}
\pgfmathsetmacro{\kystep}{1}
\pgfmathsetmacro{\kpin}{0.75}
\pgfmathsetmacro{\ksepX}{0.25}
\pgfmathsetmacro{\ksepY}{1}
\pgfmathsetmacro{\kW}{0.3}
\draw(0,0)node[or  port,number inputs=4,anchor=out](u1){};
\draw(u1.out)node[right]{$Y$};
\draw[thin](u1.in 1)--++(0,1.5*\ksepY)--++(-\ksepX,0)node[and port,anchor=out,number inputs=3](u2){};
\draw[thin](u1.in 2)--++(-\ksepX,0)--++(0,0.5*\ksepY)node[and port,anchor=out,number inputs=2](u3){};
\draw[thin](u1.in 3)--++(-\ksepX,0)--++(0,-0.5*\ksepY)node[and port,anchor=out,number inputs=2](u4){};
\draw[thin](u1.in 4)--++(0,-1.5*\ksepY)--++(-\ksepX,0)node[and port,anchor=out,number inputs=2](u5){};
\foreach \n/\lbl in {0/{\overline{D}},1/{D},2/{\overline{C}},3/{C},4/{\overline{B}},5/{B},6/{\overline{A}},7/{A}}\draw[thin](u2.in 1)++(-\ksepX-\n*\kW,\kW)node[above]{$\lbl$}coordinate(w\n)coordinate(kA)--($(kA|-u5.in 2)+(0,-\kW)$);
\draw(u2.in 1)coordinate(kA)--(kA-| w5);
\draw(u2.in 2)coordinate(kA)--(kA-| w2);
\draw(u2.in 3)coordinate(kA)--(kA-| w0);
\draw(u3.in 1)coordinate(kA)--(kA-| w4);
\draw(u3.in 2)coordinate(kA)--(kA-| w3);
\draw(u4.in 1)coordinate(kA)--(kA-| w4);
\draw(u4.in 2)coordinate(kA)--(kA-| w1);
\draw(u5.in 1)coordinate(kA)--(kA-| w3);
\draw(u5.in 2)coordinate(kA)--(kA-| w1);
\end{tikzpicture}
\end{minipage}
\انتہا{سوال}
%-------------------------

\باب{ترکیبی منطق اور ترکیبی ادوار} 
\اصطلاح{ترکیبی منطق}\فرہنگ{ترکیبی منطق}\حاشیہب{combinational logic}\فرہنگ{combinational logic} سے مراد وہ منطق ہے جس میں مخارج موجودہ مداخل پر منحصر ہو؛یعنی، 
 کسی بھی لمحہ پر تفاعل کا مخارج، اُسی لمحہ کے مداخل پر منحصر ہو گا۔ایسے تفاعل کو ترکیبی ادوار سے جامہ عمل پہنایا جاتا ہے، جو ثنائی گیٹ سے حاصل کئے جاتے ہیں۔اس باب میں ترکیبی ادوار پر غور کیا جائے گا۔

اس کے برعکس،\اصطلاح{ ترتیبی منطق}\فرہنگ{ترتیبی منطق}\حاشیہب{sequential logic}\فرہنگ{sequential logic} سے مراد وہ منطق ہے جس میں مخارج موجودہ اور ماضی مداخل پر منحصر ہو؛ یعنی، کسی بھی لمحہ پر تفاعل کا مخارج، گزرے اور موجودہ مداخل پر منحصر ہو گا۔ترتیبی منطق کو ترتیبی ادوار سے جامہ عمل پہنایا جاتا ہے، جن پر اگلے باب میں غور کیا جائے گا۔

کسی بھی ترکیبی دور کو شکل \حوالہ{شکل_ترکیبی_ڈبہ_دور} کی \اصطلاح{ڈبہ شکل}\فرہنگ{ڈبہ شکل}\حاشیہب{box diagram}\فرہنگ{box diagram} سے ظاہر کیا جا سکتا ہے، جہاں مداخل ثنائی ہندسوں (مداخل بِٹ) کو بائیں جبکہ مخارج ثنائی ہندسوں کو دائیں ہاتھ رکھا جاتا ہے۔
\begin{figure}
\centering
\begin{tikzpicture}
\pgfmathsetmacro{\kxs}{1.5}
\pgfmathsetmacro{\kys}{2.25}
\pgfmathsetmacro{\kpin}{0.5}
\pgfmathsetmacro{\kxmv}{0.25}
\pgfmathsetmacro{\kymv}{0.25}
\draw(0,0) rectangle ++(\kxs,\kys);
\draw(\kxs/2,\kys/2)node[yshift=1em]{\text{\RL{ترکیبی دور}}};
\foreach \n in {1,2,3,6}{\draw[stealth-] (0,\n*\kys/7)--++(-\kpin,0);}
\foreach \n in {1,2,3,6}{\draw[-stealth] (\kxs,\n*\kys/7)--++(\kpin,0);}
\draw(0,\kys)node[above left]{\text{\RL{داخلی ثنائی ہندسے}}} (\kxs,\kys)node[above right]{\text{\RL{خارجی ثنائی ہندسے}}};
%\draw(-\kpin,\kys/2)++(-0.25,0)node[rotate=90]{\text{\RL{داخلی ثنائی ہندسے}}};
%\draw(\kxs+\kpin,\kys/2)++(0.25,0)node[rotate=90]{\text{\RL{خارجی ثنائی ہندسے}}};
\end{tikzpicture}
\caption{ترکیبی دور کی ڈبہ شکل۔}
\label{شکل_ترکیبی_ڈبہ_دور}
\end{figure}
\حصہ{ثنائی جمع کار اور ثنائی منفی کار}
دو اعداد کو جمع یا منفی کرنا بنیادی حساب کا حصہ ہے۔آئیں دو بِٹ جمع کرنے والے دور پر غور کریں۔

\جزوحصہ{نصف جمع کار}
 ایک بٹ کی قیمت صرف \عددی{0} یا \عددی{1} ہو سکتی ہے، لہٰذا دو بٹ جمع کرتے ہوئے درج ذیل چار (ثنائی) صورتیں پیدا ہوں گی۔ (اس باب میں ثنائی ہندسے اور اعداد استعمال ہوں گے؛ زیر نوشت \عددی{2} لکھ کر وضاحت نہیں کی جائے گی۔)
\begin{align*}
0+0&=0\\
0+1&=1\\
1+0&=1\\
1+1&=10
\end{align*}
اس مساوات میں دو بٹ جمع کئے گئے، لہٰذا مداخل کی تعداد دو ہو گی۔ مساوات میں اگرچہ پہلے تین جوابات ایک بٹ ہیں، لیکن آخری جواب دو بٹ ہے۔یوں، تمام صورتوں سے نپٹنے کی خاطر، جوابات دو بٹ تصور کیے جائیں گے، اور ذیل لکھنا بہتر ہو گا:
\begin{align*}
0+0&=00\\
0+1&=01\\
1+0&=01\\
1+1&=10
\end{align*}
جس سے واضح ہے کہ جواب دو بٹ ہیں۔ یوں، دو بٹ جمع کرنے والے دور کے دو مداخل اور دو مخارج ہوں گے۔

مداخل کو \عددی{y} اور \عددی{z}، جبکہ مخارج کو \عددی{s} اور \عددی{c} لکھ کر درج بالا مساوات کو جدول \حوالہ{شکل_ترکیبی_دو_بٹ_جمع_الف} میں پیش کیا گیا ہے، جس سے تفاعلات \عددی{c} اور \عددی{s} کی مساوات، مجموعہ ارکان ضرب کے روپ میں حاصل کرتے ہیں۔
\begin{table}
\caption{دو بِٹ جمع}
\label{شکل_ترکیبی_دو_بٹ_جمع_الف}
\centering
\begin{otherlanguage}{english}
\begin{tabular}{CC|CC}
\toprule
y&z&c&s\\
\midrule
0&0&0&0\\
0&1&0&1\\
1&0&0&1\\
1&1&1&0\\
\bottomrule
\end{tabular}
\end{otherlanguage}
\end{table}
%
\begin{gather}
\begin{aligned}\label{مساوات_ترکیبی_نصف_جمع_کار}
c&=yz\\
s&=\overline{y}z+y\overline{z}
\end{aligned}
\end{gather}
اِن تفاعلات کے (دو مختلف اقسام کے) ادوار شکل \حوالہ{شکل_ترکیبی_جمع_دو_بٹ} میں پیش کیے گئے ہیں، جو \اصطلاح{نصف جمع کار}\فرہنگ{جمع کار!نصف}\حاشیہب{half adder}\فرہنگ{adder!half} کہلاتے ہیں۔اس نام کی وضاحت اگلے حصہ میں ہو گی۔
\begin{figure}
\centering
\begin{subfigure}{0.45\textwidth}
\centering
\begin{tikzpicture}
\pgfmathsetmacro{\kxsep}{2};
\pgfmathsetmacro{\kysep}{1.5};
\draw(0,0)node[xnor port ,scale=1, number inputs=2](u1){};
 \draw(0,-1.5*\kysep)node[and port ,scale=1, number inputs=2](u2){};
\draw[](u1.in 1)node[left]{$y$} (u1.in 2)node[left]{$z$} (u1.out)node[right]{$s$};
 \draw(u2.in 1)node[left]{$y$} (u2.in 2)node[left]{$z$} (u2.out)node[right]{$c$};
\end{tikzpicture}
\caption{}
\end{subfigure}\hfill
\begin{subfigure}{0.45\textwidth}
\centering
\begin{tikzpicture}
\pgfmathsetmacro{\kxsep}{2};
\pgfmathsetmacro{\kysep}{1.5};
\draw(0,0)node[and port ,scale=1, number inputs=2](u1){};
\draw(0,-\kysep)node[and port,scale=1, number inputs=2](u2){};
\draw(\kxsep,-\kysep/2)node[or port,scale=1, number inputs=2](u3){};
\draw(u1.out)-|(u3.in 1) (u2.out)-|(u3.in 2);
\draw[](u1.in 1)node[left]{$y$} (u1.in 2)node[left]{$\overline{z}$}
 (u2.in 1)node[left]{$\overline{y}$} (u2.in 2)node[left]{$z$} (u3.out)node[right]{$s$};
 \draw(0,-2*\kysep)node[and port ,scale=1, number inputs=2](u4){};
 \draw(u4.in 1)node[left]{$y$} (u4.in 2)node[left]{$z$};
 \path(u3.out)--++(0,-1*\kysep)coordinate(kR);
 \draw(u4.out) --($(u3.out)!(u4.out)!(kR)$)node[right]{$c$};
\end{tikzpicture}
\caption{}
\end{subfigure}
\caption{نصف جمع کار}
\label{شکل_ترکیبی_جمع_دو_بٹ}
\end{figure}


\جزوحصہ{مکمل جمع کار}
 آئیں، ایک سے زیادہ بٹ ثنائی اعداد \عددی{y=111_2} اور \عددی{z=11_2} کے مجموعے کا حصول دیکھتے ہیں۔
\begin{align*}
\begin{split}
11\phantom{1\,}&\\
111&\\
+\phantom{1}11&\\
\noalign{\smallskip}\hline\noalign{\smallskip}
1010&
\end{split}
\end{align*}
 پہلے قدم پر کم تر رتبی بِٹ \عددی{y_0} اور \عددی{z_0} کو نصف جمع کار حل کر سکتا ہے، لیکن اگلے قدم پر بِٹ \عددی{y_1} اور \عددی{z_1} جمع کرتے ہوئے گزشتہ قدم کا \اصطلاح{حاصل}\فرہنگ{حاصل}\حاشیہب{carry}\فرہنگ{carry} \عددی{(1)} بھی جمع کرنا ہو گا ۔

 ظاہر ہوا، دو اعداد جمع کرنے کی خاطر ایسا دور درکار ہو گا جو تین بٹ جمع کر سکے۔ آئیں ایسا دور دیکھتے ہیں۔

 اس دور کے مداخل \عددی{x}، \عددی{y} اور \عددی{z} جبکہ مخارج \عددی{c} اور \عددی{s} لیتے ہوئے (جہاں \عددی{x} پچھلے قدم کا حاصل ہو گا) جدول \حوالہ{جدول_ترکیبی_مکمل_جمع_کار} لکھتے ہیں۔
 
\begin{table}
\caption{مکمل جمع کار}
\label{جدول_ترکیبی_مکمل_جمع_کار}
\centering
\begin{otherlanguage}{english}
\begin{tabular}{CCC|CC}
\toprule
x&y&z&c&s\\
\midrule
0&0&0&0&0\\
0&0&1&0&1\\
0&1&0&0&1\\
0&1&1&1&0\\
1&0&0&0&1\\
1&0&1&1&0\\
1&1&0&1&0\\
1&1&1&1&1\\
\bottomrule
\end{tabular}
\end{otherlanguage}
\end{table}

 جدول سے \عددی{c} اور \عددی{s} کے تفاعلات کی مساوات ، مجموعہ ارکان ضرب کے روپ میں حاصل کرتے ہیں۔یاد رہے جدول میں تین آزاد اور دو تابع متغیرات ہیں۔ ایک تابع متغیرہ کی مساوات حاصل کرتے وقت دوسرے تابع متغیرہ کو نظر انداز کریں۔یوں \عددی{c} کی مساوات حاصل کرتے وقت تین مداخل \عددی{x}، \عددی{y}، اور \عددی{z} پر نظر رکھتے ہوئے \عددی{c} کے ارکان ضرب کا مجموعہ لیں۔
 شکل \حوالہ{شکل_ترکیبی_مکمل_جمع} میں کارناف نقشوں سے ان تفاعلات کی (درج ذیل) سادہ مساوات حاصل کی گئی ہیں۔
 \begin{gather}
\begin{aligned}\label{مساوات_ترکیبی_مکمل_جمع_کار_کارناف}
c&=xz+xy+yz\\
s&=x\oplus y \oplus z
\end{aligned}
\end{gather}
 %
\begin{figure}
\centering
\begin{subfigure}{0.45\textwidth}
\centering
\begin{tikzpicture}
\pgfmathsetmacro{\kxstep}{1}
\pgfmathsetmacro{\kystep}{1}
\pgfmathsetmacro{\kpin}{0.75}
\pgfmathsetmacro{\kmv}{0.15}
\pgfmathsetmacro{\kmva}{0.10}
\draw[xstep=\kxstep,ystep=\kystep](0,0) grid (4*\kxstep,-2*\kystep);
\draw(0,0)--++(135:\kpin)node[pos=0.75,above right]{$yz$}node[pos=0.75,below left]{$x$};
\foreach \kx/\xlb in {0/{00},1/{01},2/{11},3/{10}}{\draw(\kx*\kxstep+\kxstep/2,0)node[above]{$\xlb$};}
\foreach \ky/\ylb in {0/{0},1/{1}}{\draw(0,-\ky*\kystep-\kystep/2)node[left]{$\ylb$};}
\foreach \kx/\xlb in {2/{1}}{\draw(\kx*\kxstep+\kxstep/2,-\kystep/2)node[]{$\xlb$};}
\foreach \kx/\xlb in {1/1,2/1,3/1}{\draw(\kx*\kxstep+\kxstep/2,-1.5*\kystep)node[]{$\xlb$};}
%\foreach \kx/\xlb in {0/{1},1/1}{\draw(\kx*\kxstep+\kxstep/2,-2.5*\kystep)node[]{$\xlb$};}
%\foreach \kx/\xlb in {0/1,1/1,2/d}{\draw(\kx*\kxstep+\kxstep/2,-3.5*\kystep)node[]{$\xlb$};}
\draw[gray,dashed] ($(2*\kxstep,-0*\kystep)+(\kmv,-\kmv)$) rectangle ($(3*\kxstep,-2*\kystep)+(-\kmv,\kmv)$);
\draw[gray,dashed] ($(1*\kxstep,-1*\kystep)+(\kmva,-\kmva)$) rectangle ($(3*\kxstep,-2*\kystep)+(-\kmva,\kmva)$);
\draw[gray,dashed] ($(2*\kxstep,-1*\kystep)+(-\kmv,\kmv)$) rectangle ($(4*\kxstep,-2*\kystep)+(-\kmv,-\kmv)$);
%\draw[gray,dashed] ($(0,-3*\kystep)+(-\kmv,-\kmv)$)--++(1*\kxstep,0)--++(0,-\kystep-\kmv);
%\draw[gray,dashed] ($(3*\kxstep,-0*\kystep)+(\kmv,\kmv)$)--++(0,-1*\kystep)--++(\kxstep+\kmv,0);
%\draw[gray,dashed] ($(3*\kxstep,-4*\kystep)+(\kmv,-\kmv)$)--++(0,1*\kystep)--++(\kxstep+\kmv,0);
%\draw(4*\kxstep-\kmv,-0.5*\kystep) to [out=0,in=180]++(0.5,-0.25)node[right]{$\overline{w}\,\overline{x}$};
%\draw(1.5*\kxstep-\kmv,-4*\kystep+\kmv) to [out=-90,in=180]++(0.5,-0.5)node[right]{$w\overline{y}$};
\draw(2*\kxstep,-2*\kystep-0.5)node[]{$c=xz+xy+yz$};
\end{tikzpicture}
\end{subfigure}\hfill
\begin{subfigure}{0.45\textwidth}
\centering
\begin{tikzpicture}
\pgfmathsetmacro{\kxstep}{1}
\pgfmathsetmacro{\kystep}{1}
\pgfmathsetmacro{\kpin}{0.75}
\pgfmathsetmacro{\kmv}{0.15}
\pgfmathsetmacro{\kmva}{0.10}
\draw[xstep=\kxstep,ystep=\kystep](0,0) grid (4*\kxstep,-2*\kystep);
\draw(0,0)--++(135:\kpin)node[pos=0.75,above right]{$yz$}node[pos=0.75,below left]{$x$};
\foreach \kx/\xlb in {0/{00},1/{01},2/{11},3/{10}}{\draw(\kx*\kxstep+\kxstep/2,0)node[above]{$\xlb$};}
\foreach \ky/\ylb in {0/{0},1/{1}}{\draw(0,-\ky*\kystep-\kystep/2)node[left]{$\ylb$};}
\foreach \kx/\xlb in {1/1,3/{1}}{\draw(\kx*\kxstep+\kxstep/2,-\kystep/2)node[]{$\xlb$};}
\foreach \kx/\xlb in {0/1,2/1}{\draw(\kx*\kxstep+\kxstep/2,-1.5*\kystep)node[]{$\xlb$};}
%\foreach \kx/\xlb in {0/{1},1/1}{\draw(\kx*\kxstep+\kxstep/2,-2.5*\kystep)node[]{$\xlb$};}
%\foreach \kx/\xlb in {0/1,1/1,2/d}{\draw(\kx*\kxstep+\kxstep/2,-3.5*\kystep)node[]{$\xlb$};}
%\draw[gray,dashed] ($(2*\kxstep,-0*\kystep)+(\kmv,-\kmv)$) rectangle ($(3*\kxstep,-2*\kystep)+(-\kmv,\kmv)$);
%\draw[gray,dashed] ($(1*\kxstep,-1*\kystep)+(\kmva,-\kmva)$) rectangle ($(3*\kxstep,-2*\kystep)+(-\kmva,\kmva)$);
%\draw[gray,dashed] ($(2*\kxstep,-1*\kystep)+(-\kmv,\kmv)$) rectangle ($(4*\kxstep,-2*\kystep)+(-\kmv,-\kmv)$);
%\draw[gray,dashed] ($(0,-3*\kystep)+(-\kmv,-\kmv)$)--++(1*\kxstep,0)--++(0,-\kystep-\kmv);
%\draw[gray,dashed] ($(3*\kxstep,-0*\kystep)+(\kmv,\kmv)$)--++(0,-1*\kystep)--++(\kxstep+\kmv,0);
%\draw[gray,dashed] ($(3*\kxstep,-4*\kystep)+(\kmv,-\kmv)$)--++(0,1*\kystep)--++(\kxstep+\kmv,0);
%\draw(4*\kxstep-\kmv,-0.5*\kystep) to [out=0,in=180]++(0.5,-0.25)node[right]{$\overline{w}\,\overline{x}$};
%\draw(1.5*\kxstep-\kmv,-4*\kystep+\kmv) to [out=-90,in=180]++(0.5,-0.5)node[right]{$w\overline{y}$};
\draw(2*\kxstep,-2*\kystep-0.5)node[]{$s=x\oplus y\oplus z$};
\end{tikzpicture}
\end{subfigure}
\caption{مکمل جمع کار}
\label{شکل_ترکیبی_مکمل_جمع}
\end{figure}

کارناف نقشہ استعمال کیے بغیر جدول \حوالہ{جدول_ترکیبی_مکمل_جمع_کار} سے ان تفاعلات کی مساوات، مجموعہ ارکان ضرب کے روپ میں لکھتے ہیں۔
\begin{gather}
\begin{aligned}\label{مساوات_ترکیبی_مکمل_جمع_کار_پہلا}
c&=\overline{x}yz+x\overline{y}z+xy\overline{z}+xyz\\
s&=\overline{x}\,\overline{y}z+\overline{x}y\overline{z}+x\overline{y}\,\overline{z}+xyz
\end{aligned}
\end{gather}
انہیں شکل \حوالہ{شکل_ترکیبی_مکمل_جمع_کار_پہلا} میں عملی جامہ پہنایا گیا ہے۔
%
\begin{figure}
\centering
\begin{subfigure}{0.45\textwidth}
\centering
\begin{tikzpicture}
\pgfmathsetmacro{\kxsep}{2};
\pgfmathsetmacro{\kysep}{1.5};
\pgfmathsetmacro{\kpin}{0.25};
\draw(0,0)node[and port ,scale=1, number inputs=3](u1){};
\draw(0,-1*\kysep)node[and port ,scale=1, number inputs=3](u2){};
\draw(0,-2*\kysep)node[and port ,scale=1, number inputs=3](u3){};
\draw(0,-3*\kysep)node[and port ,scale=1, number inputs=3](u4){};
 \draw(\kxsep,-1.5*\kysep)node[or port ,scale=1, number inputs=4](u5){};
 \draw(u1.out)-|(u5.in 1) (u4.out)-|(u5.in 4) (u5.in 2)--++(-\kpin,0)-|(u2.out) (u5.in 3)--++(-\kpin,0)-|(u3.out);
\draw[](u1.in 1)node[left]{$\overline{x}$} (u1.in 2)node[left]{$\overline{y}$} (u1.in 3)node[left]{$z$};
\draw[](u2.in 1)node[left]{$\overline{x}$} (u2.in 2)node[left]{$y$} (u2.in 3)node[left]{$\overline{z}$};
\draw[](u3.in 1)node[left]{$x$} (u3.in 2)node[left]{$\overline{y}$} (u3.in 3)node[left]{$\overline{z}$};
\draw[](u4.in 1)node[left]{$x$} (u4.in 2)node[left]{$y$} (u4.in 3)node[left]{$z$};
 \draw(u5.out)node[right]{$s$};
\end{tikzpicture}
\end{subfigure}\hfill
\begin{subfigure}{0.45\textwidth}
\centering
\begin{tikzpicture}
\pgfmathsetmacro{\kxsep}{2};
\pgfmathsetmacro{\kysep}{1.5};
\pgfmathsetmacro{\kpin}{0.25};
\draw(0,0)node[and port ,scale=1, number inputs=3](u1){};
\draw(0,-1*\kysep)node[and port ,scale=1, number inputs=3](u2){};
\draw(0,-2*\kysep)node[and port ,scale=1, number inputs=3](u3){};
\draw(0,-3*\kysep)node[and port ,scale=1, number inputs=3](u4){};
 \draw(\kxsep,-1.5*\kysep)node[or port ,scale=1, number inputs=4](u5){};
 \draw(u1.out)-|(u5.in 1) (u4.out)-|(u5.in 4) (u5.in 2)--++(-\kpin,0)-|(u2.out) (u5.in 3)--++(-\kpin,0)-|(u3.out);
\draw[](u1.in 1)node[left]{$\overline{x}$} (u1.in 2)node[left]{$y$} (u1.in 3)node[left]{$z$};
\draw[](u2.in 1)node[left]{$x$} (u2.in 2)node[left]{$\overline{y}$} (u2.in 3)node[left]{$z$};
\draw[](u3.in 1)node[left]{$x$} (u3.in 2)node[left]{$y$} (u3.in 3)node[left]{$\overline{z}$};
\draw[](u4.in 1)node[left]{$x$} (u4.in 2)node[left]{$y$} (u4.in 3)node[left]{$z$};
 \draw(u5.out)node[right]{$c$};
\end{tikzpicture}
\end{subfigure}
\caption{مکمل جمع کار (مساوات \حوالہ{مساوات_ترکیبی_مکمل_جمع_کار_پہلا})}
\label{شکل_ترکیبی_مکمل_جمع_کار_پہلا}
\end{figure}

درج بالا پہلی مساوات کے درمیانے دو اجزاء کا مجموعہ \عددی{x(\overline{y}z+y\overline{z})} جبکہ باقی اجزاء کا \عددی{(\overline{x}+x)yz} ہے، لہٰذا \عددی{c} کے لئے درج ذیل لکھا جا سکتا ہے۔
\begin{align*}
c&=(\overline{x}+x)yz+x(\overline{y}z+y\overline{z})\\
&=yz+x(y\oplus z)
\end{align*}
اس کو مساوات \حوالہ{مساوات_ترکیبی_مکمل_جمع_کار_کارناف} میں پیش \عددی{s} کے ساتھ اکٹھا لکھتے ہیں۔
\begin{gather}
\begin{aligned}\label{مساوات_ترکیبی_مکمل_جمع_کار_دوسرا}
c&=yz+x(y\oplus z)\\
s&=x\oplus y \oplus z
\end{aligned}
\begin{cases}
\begin{minipage}{0.2\textwidth}
مکمل جمع کار کی\\
بہتر مساوات
\end{minipage}
\end{cases}
\end{gather}
ان تفاعلات کو شکل \حوالہ{شکل_ترکیبی_مکمل_جمع_کار_دوسرا} میں پیش کیا گیا ہے، جو شکل \حوالہ{شکل_ترکیبی_مکمل_جمع_کار_پہلا} سے بہتر (چھوٹا)ہے۔
\begin{figure}
\centering
\begin{tikzpicture}
\pgfmathsetmacro{\kxsep}{2};
\pgfmathsetmacro{\kysep}{1.5};
\pgfmathsetmacro{\kpin}{1};
\draw(0,0)node[xor port ,scale=1, number inputs=2](u1){};
\draw(0,-1*\kysep)node[and port ,scale=1, number inputs=2](u2){};
\draw(2*\kxsep,0.5*\kysep)node[xor port ,scale=1, number inputs=2](u3){};
\draw(2*\kxsep,-0.5*\kysep)node[and port ,scale=1, number inputs=2](u4){};
 %\draw(3*\kxsep,-0.75*\kysep-0.1)node[or port ,scale=1, number inputs=2](u5){};
 \draw(u1.in 1)--++(-\kpin,0)coordinate(kla)coordinate[pos=0.4](kua)node[left]{$y$};
\draw(u1.in 2)--++(-\kpin,0)coordinate(klb)coordinate[pos=0.6](kuaa)node[left]{$z$};
\draw(kua)|-(u2.in 1) (kuaa)|-(u2.in 2); 
\draw(u3.in 1)--($(kla)!(u3.in 1)!(klb)$)coordinate(kll)node[left]{$x$};
\draw(u1.out)-|(u3.in 2);
\draw(u3.in 2)-|(u4.in 1);
\draw(u4.in 2)--++(-\kpin,0)coordinate(kcc)--($(u3.in 1)!(kcc)!(kll)$);
\draw(u2.out)--++(2.25*\kxsep,0)node[or port, scale=1,number inputs=2, anchor=in 2](u5){};
\draw(u4.out)-|(u5.in 1);
\draw(u5.out)node[right]{$c$};
\path(u5.out)--++(0,\kpin)coordinate(krr);
\draw(u3.out)--($(u5.out)!(u3.out)!(krr)$)node[right]{$s$};
\end{tikzpicture}
\caption{مکمل جمع کار کا بہتر دور (مساوات \حوالہ{مساوات_ترکیبی_مکمل_جمع_کار_دوسرا})}
\label{شکل_ترکیبی_مکمل_جمع_کار_دوسرا}
\end{figure}


مساوات \حوالہ{مساوات_ترکیبی_مکمل_جمع_کار_دوسرا} میں دیے \عددی{s} سے ارکان ضرب کا مجموعہ حاصل کرتے ہیں۔
\begin{align*}
s&=x\oplus (y\oplus z)\\
&=x\oplus (y\overline{z}+\overline{y}z)\\
&=x(\overline{y\overline{z}+\overline{y}z})+\overline{x}(y\overline{z}+\overline{y}z)\\
&=x(\overline{y\overline{z}})(\overline{\overline{y}z})+\overline{x}(y\overline{z}+\overline{y}z)\\
&=x(\overline{y}+z)(y+\overline{z})+\overline{x}(y\overline{z}+\overline{y}z)\\
&=x(yz+\overline{y}\,\overline{z})+\overline{x}(y\overline{z}+\overline{y}z)\\
&=xyz+x\overline{y}\,\overline{z}+\overline{x}y\overline{z}+\overline{x}\,\overline{y}z
\end{align*}
شکل \حوالہ{شکل_ترکیبی_مکمل_جمع_کار_دوسرا} \اصطلاح{ مکمل جمع کار}\فرہنگ{جمع کار!مکمل}\حاشیہب{full adder}\فرہنگ{adder!full} کہلاتا ہے ، لہٰذا شکل \حوالہ{شکل_ترکیبی_جمع_دو_بٹ} کو \اصطلاح{نصف جمع کار}\فرہنگ{جمع کار!نصف}\حاشیہب{half adder}\فرہنگ{adder!half} کہیں گے۔

جدول \حوالہ{جدول_ترکیبی_مکمل_جمع_کار} میں \عددی{y} اور \عددی{z} ثنائی ہندسوں کے ساتھ گزشتہ قدم کا حاصل \عددی{x} جمع کیا گیا۔ شکل \حوالہ{شکل_ترکیبی_مکمل_جمع_کار_ڈبہ} میں نصف جمع کار اور مکمل جمع کار کی علامت پیش ہیں۔ مکمل جمع کار میں گزشتہ قدم سے \اصطلاح{داخلی حاصل}\فرہنگ{حاصل!داخلی}\حاشیہب{carry in}\فرہنگ{carry!in} کو \عددی{c_{\text{د}}} جبکہ اس قدم کے \اصطلاح{خارجی حاصل}\فرہنگ{حاصل!خارجی}\حاشیہب{carry out}\فرہنگ{carry!out} کو \عددی{c_{\text{خ}}}سے ظاہر کیا گیا۔

\begin{figure}
\centering
\begin{tikzpicture}[
 fulladder/.style={draw, minimum height=1.5cm, minimum width=3cm,thick,
 label={[anchor=west]left:$c_\text{خ}$},
 label={[anchor=south]below:$s$},
 label={[anchor=east]right:$c_\text{د}$},
 label={[anchor=north]65:$z\vphantom{y}$},
 label={[anchor=north]115:$y\vphantom{z}$},
 },
 halfadder/.style={draw, minimum height=1.5cm, minimum width=3cm,thick,
 label={[anchor=west]left:$c$},
 label={[anchor=south]below:$s$},
 label={[anchor=north]65:$z\vphantom{y}$},
 label={[anchor=north]115:$y\vphantom{z}$},
 }]
\pgfmathsetmacro{\kpin}{0.5}
\node[fulladder] (a) {\text{\RL{مکمل جمع کار}}};
\draw[stealth-] (a.115) --++(90:\kpin) node [above] {};
\draw[stealth-] (a.65) --++(90:\kpin) node [above] {};
\draw[stealth-] (a.east) --++(\kpin,0) node [right] {};
\draw[-stealth] (a.west) --++(-\kpin,0) node [left] {};
\draw[-stealth] (a.south) --++(0,-\kpin) node [left] {};
 \node[halfadder, right=3cm of a] (b) {\text{\RL{نصف جمع کار}}};
\draw[stealth-] (b.115) --++(90:\kpin) node [above] {};
\draw[stealth-] (b.65) --++(90:\kpin) node [above] {};
\draw[-stealth] (b.west) --++(-\kpin,0) node [left] {};
\draw[-stealth] (b.south) --++(0,-\kpin) node [left] {};
\end{tikzpicture}
\caption{نصف جمع کار اور مکمل جمع کار کی علامتیں۔}
\label{شکل_ترکیبی_مکمل_جمع_کار_ڈبہ}
\end{figure}


آئیں \عددی{y=111_2} اور \عددی{z=11_2} کا مجموعہ مکمل جمع کار کی مدد سے حاصل کریں۔سب سے پہلے دونوں اعداد کو تین ثنائی ہندسوں میں لکھیں ، لہٰذا \عددی{z=011_2} ہو گا۔شکل \حوالہ{شکل_ترکیبی_تین_درجہ_جمع_کار} میں مطلوبہ تین درجی، تین بِٹ جمع کار پیش کیا گیا ہے، جہاں مکمل جمع کار کو مختصراً \قول{جمع کار} کہا گیا ہے ۔ ثنائی عدد \عددی{y=111=y_2y_1y_0} اور \عددی{z=011=z_2z_1z_0} ہیں۔یوں کم رتبی بِٹ کے مکمل جمع کار کو دونوں اعداد کے کم رتبی ہندسے، \عددی{y_0=1} اور \عددی{z_0=1}، فراہم کیے جائیں گے، اور ساتھ ہی چونکہ پہلے قدم میں کوئی \قول{داخلی حاصل} نہیں ہو گا لہٰذا داخلی حاصل \عددی{c_0=0} فراہم کیا جائے گا۔ اگلے قدم میں جمع کار کو \عددی{y_1=1} اور \عددی{z=1} کے ساتھ پہلے قدم کا حاصل \عددی{c_1} بطور داخلی حاصل، فراہم کیا جائے گا، جبکہ آخری جمع کار کو \عددی{y_2=1} اور \عددی{z_2=0} کے ساتھ گزشتہ قدم کا حاصل \عددی{c_2} فراہم کیا جائے گا۔تین بِٹ جمع کار، ان اعداد کا مجموعہ \عددی{c_3s_2s_1s_0=1010_2} دے گا۔
\begin{align*}
\begin{split}
111\phantom{1\,}&\\
\phantom{1}111&\\
+\phantom{1}011&\\
\noalign{\smallskip}\hline\noalign{\smallskip}
1010&
\end{split}
\end{align*}
%
\begin{figure}
\centering
\begin{tikzpicture}[
 fulladder/.style={draw, minimum height=1.5cm, minimum width=2cm,thick,
 label={[anchor=west]left:$c_\text{خ}$},
 label={[anchor=south]below:$s$},
 label={[anchor=east]right:$c_\text{د}$},
 label={[anchor=north]65:$z\vphantom{y}$},
 label={[anchor=north]115:$y\vphantom{z}$},
 }]
\pgfmathsetmacro{\kpin}{0.5}
\node[fulladder] (u0) {\text{\RL{جمع کار}}};
\node[fulladder, left=1cm of u0] (u1) {\text{\RL{جمع کار}}};
\node[fulladder, left=1cm of u1] (u2) {\text{\RL{جمع کار}}};
%
\draw[stealth-] (u0.115) --++(90:\kpin) node[above]{$1$}node [above,yshift=1.5em] {$y_0$};
\draw[stealth-] (u0.65) --++(90:\kpin) node[above]{$1$}node [above,yshift=1.5em] {$z_0$};
\draw[stealth-] (u1.115) --++(90:\kpin) node[above]{$1$}node [above,yshift=1.5em] {$y_1$};
\draw[stealth-] (u1.65) --++(90:\kpin) node[above]{$1$}node [above,yshift=1.5em] {$z_1$};
\draw[stealth-] (u2.115) --++(90:\kpin) node[above]{$1$}node [above,yshift=1.5em] {$y_2$};
\draw[stealth-] (u2.65) --++(90:\kpin) node[above]{$0$}node [above,yshift=1.5em] {$z_2$};
\draw[stealth-] (u0.east) --++(\kpin,0) node [right] {0}node[right,yshift=-1em]{$c_0$};
\draw[-stealth] (u0.south) --++(0,-\kpin)coordinate(ka) node[below]{$0$}node [below,yshift=-1.5em] {$s_0$};
\draw[-stealth] (u1.south) --++(0,-\kpin)coordinate(kb) node[below]{$1$}node [below,yshift=-1.5em] {$s_1$};
\draw[-stealth] (u2.south) --++(0,-\kpin) node[below]{$0$}node [below,yshift=-1.5em] {$s_2$};
\draw[-stealth](u0)--(u1)node[pos=0.5,above]{$1$}node[pos=0.5,below]{$c_1$};
\draw[-stealth](u1)--(u2)node[pos=0.5,above]{$1$}node[pos=0.5,below]{$c_2$};
\draw[-stealth](u2.west)--++(-\kpin,0)coordinate(kc)--($(ka)!(kc)!(kb)$)node[below]{$1$}node[below,yshift=-1.5em]{$c_3$};
\draw(u0.south)node[yshift=-1.75cm]{\text{\RL{درجہ صفر}}};
\draw(u1.south)node[yshift=-1.75cm]{\text{\RL{درجہ ایک}}};
\draw(u2.south)node[yshift=-1.75cm]{\text{\RL{درجہ دو}}};
\end{tikzpicture}
\caption{تین درجی، تین بِٹ جمع کار}
\label{شکل_ترکیبی_تین_درجہ_جمع_کار}
\end{figure}

شکل \حوالہ{شکل_ترکیبی_تین_درجہ_جمع_کار} میں چونکہ درجہ صفر کا داخلی حاصل ہمیشہ \عددی{0} ہو گا لہٰذا یہاں مکمل جمع کار کی بجائے نصف جمع کار بھی استعمال کیا جا سکتا تھا۔ ایسا کرتے ہوئے \عددی{c_0} فراہم کرنے کی ضرورت نہیں ہو گی۔

 زیادہ بِٹ اعداد کے مجموعہ کے لئے شکل \حوالہ{شکل_ترکیبی_تین_درجہ_جمع_کار} میں بائیں جانب مزید مکمل جمع کار کا اضافہ کیا جائے گا۔ یوں \عددی{8} بِٹ (یعنی ایک بائٹ) اعداد کا مجموعہ آٹھ درجی جمع کار دے گا، جو \عددی{8} مکمل جمع کار پر مشتمل ہو گا، جبکہ \عددی{64} بِٹ اعداد کے مجموعہ کے لئے \عددی{64} مکمل جمع کار پر مشتمل \عددی{64} بِٹ جمع کار درکار ہو گا۔


\ابتدا{مشق}
مخلوط دور 74283 چار بِٹ مکمل جمع کار ہے (صفحہ \حوالہصفحہ{بوولین_مخلوط_ادوار_سلسلہ} پر مخلوط ادوار کے سلسلہ \عددی{74xxx} کے بارے میں دوبارہ پڑھیں)۔اس کے معلوماتی صفحات انٹرنیٹ\حاشیہد{انٹرنیٹ میں \تحریر{74283 datasheet} تلاش کریں۔} سے حاصل کریں۔ اس مخلوط دور کو استعمال کرتے ہوئے \عددی{8} بٹ کے دو ثنائی اعداد جمع کریں۔
\انتہا{مشق}

\جزوحصہ{منفی کار}
 ثنائی اعداد کو کمپیوٹر دو کے تکملہ کی مدد سے منفی کرتا ہے۔دو کا تکملہ استعمال کرتے ہوئے ثنائی اعداد منفی کرنے کے عمل پر دوبارہ نظر ڈالتے ہیں۔یاد رہے، بلند تر رتبی بِٹ کی جمع سے پیدا، آخری حاصل ضائع کیا جاتا ہے،جبکہ اس کی غیر موجودگی میں نتیجے کا دو کا تکملہ لیا جاتا ہے۔

 ثنائی عدد کے اساس منفی ایک تکملہ (یا متمم) کے ساتھ \عددی{1} جمع کرنے سے عدد کا اساسی تکملہ حاصل ہو گا۔ عدد کا متمم حاصل کرنے کی خاطر عدد کے ہر بِٹ کا متمم لیا جاتا ہے۔بِٹ کا متمم بذریعہ نفی گیٹ لیا جا سکتا ہے۔
	
تین بِٹ ثنائی اعداد \عددی{y} اور \عددی{z} سے \عددی{(y-z)} حاصل کرنے کے لئے \عددی{z} کے متمم کے ساتھ \عددی{1} اور \عددی{y} جمع کرنا ہو گا۔
شکل \حوالہ{شکل_ترکیبی_تین_درجہ_منفی_کار} میں اس عمل کو عملی جامہ پہنایا گیا ہے، جہاں نفی گیٹ استعمال کر کے \عددی{z} کا متمم (یا ایک کا تکملہ) حاصل کیا گیا، اور ساتھ \عددی{1} جمع کرنے کی خاطر درجہ صفر کو داخلی حاصل \عددی{1} فراہم کیا گیا۔

\begin{figure}
\centering
\begin{tikzpicture}[
 fulladder/.style={draw, minimum height=1.5cm, minimum width=2.25cm,thick,
 label={[anchor=west]left:$c_\text{خ}$},
 label={[anchor=south]below:$s$},
 label={[anchor=east]right:$c_\text{د}$},
 label={[anchor=north]65:$z\vphantom{y}$},
 label={[anchor=north]115:$y\vphantom{z}$},
 }]
\pgfmathsetmacro{\kpin}{0.5}
\pgfmathsetmacro{\kpina}{0.15}
\node[fulladder] (u0) {\text{\RL{جمع کار}}};
\node[fulladder, left=1cm of u0] (u1) {\text{\RL{جمع کار}}};
\node[fulladder, left=1cm of u1] (u2) {\text{\RL{جمع کار}}};
%
\draw[stealth-] (u0.65) --++(90:\kpina) coordinate(knota);
\draw (knota)node[not port,scale=0.8,rotate=-90,anchor=out](u4){} (u4.in)node[above]{$z_0$};
\draw[stealth-] (u1.65) --++(90:\kpina)coordinate(knotb);
\draw (knotb)node[not port,scale=0.8,rotate=-90,anchor=out](u5){} (u5.in)node[above]{$z_1$};
\draw[stealth-] (u2.65) --++(90:\kpina)coordinate(knotc);
\draw (knotc)node[not port,scale=0.8,rotate=-90,anchor=out](u6){} (u6.in)node[above]{$z_2$};
\draw[stealth-] (u0.115) --($(u4.in)!(u0.115)!(u5.in)$) node[above]{$y_0$};
\draw[stealth-] (u1.115) --($(u4.in)!(u1.115)!(u5.in)$) node[above]{$y_1$};
\draw[stealth-] (u2.115) --($(u4.in)!(u2.115)!(u5.in)$) node[above]{$y_2$};
\draw[stealth-] (u0.east) --++(\kpin,0) node [right] {1};
\draw[-stealth] (u0.south) --++(0,-\kpin)coordinate(ka) node[below]{$s_0$};
\draw[-stealth] (u1.south) --++(0,-\kpin)coordinate(kb) node[below]{$s_1$};
\draw[-stealth] (u2.south) --++(0,-\kpin) node[below]{$s_2$};
\draw[-stealth](u0)--(u1);
\draw[-stealth](u1)--(u2);
\draw[-stealth](u2.west)--++(-\kpin,0);
\end{tikzpicture}
\caption{تین درجی، تین بِٹ منفی کار}
\label{شکل_ترکیبی_تین_درجہ_منفی_کار}
\end{figure}

شکل \حوالہ{شکل_ترکیبی_تین_درجہ_جمع_کار} اور شکل \حوالہ{شکل_ترکیبی_تین_درجہ_منفی_کار} دونوں میں مکمل جمع کار استعمال ہوئے۔شکل \حوالہ{شکل_ترکیبی_تین_درجہ_جمع_کار} کے ساتھ نفی گیٹ منسلک کر کے اور داخلی حاصل \عددی{c_0} کو \عددی{0} کی بجائے \عددی{1} رکھنے سے شکل \حوالہ{شکل_ترکیبی_تین_درجہ_منفی_کار} حاصل ہو گا۔ جمع اور منفی اعمال ایک ہی دور سے بھی حاصل کیے جا سکتے ہیں۔ایسا دور جسے جمع و منفی کار کہتے ہیں شکل \حوالہ{شکل_ترکیبی_تین_درجہ_جمع_و_منفی_کار} میں پیش ہے۔
\begin{figure}
\centering
\begin{tikzpicture}[
 fulladder/.style={draw, minimum height=1.5cm, minimum width=2cm,thick,
 label={[anchor=west]left:$c_\text{خ}$},
 label={[anchor=south]below:$s$},
 label={[anchor=east]right:$c_\text{د}$},
 label={[anchor=north]65:$z\vphantom{y}$},
 label={[anchor=north]115:$y\vphantom{z}$},
 }]
\pgfmathsetmacro{\kpin}{0.5}
\pgfmathsetmacro{\kpina}{0.25}
\node[fulladder] (u0) {\text{\RL{جمع کار}}};
\node[fulladder, left=1cm of u0] (u1) {\text{\RL{جمع کار}}};
\node[fulladder, left=1cm of u1] (u2) {\text{\RL{جمع کار}}};
%
\draw[stealth-] (u0.65) --++(90:\kpina) coordinate(knota);
\draw (knota)node[xor port,scale=0.75,number inputs=2,rotate=-90,anchor=out](u4){}
 (u4.in 1)--++(0,\kpin)coordinate(kaa)node[above]{$z_0$};
\draw[stealth-] (u1.65) --++(90:\kpina)coordinate(knotb);
\draw (knotb)node[xor port,scale=0.75,number inputs=2,rotate=-90,anchor=out](u5){}
 (u5.in 1)--++(0,\kpin)coordinate(kbb)node[above]{$z_1$};
\draw[stealth-] (u2.65) --++(90:\kpina)coordinate(knotc);
\draw (knotc)node[xor port,scale=0.75,number inputs=2,rotate=-90,anchor=out](u6){} 
(u6.in 1)--++(0,\kpin)node[above]{$z_2$};
\draw[stealth-] (u0.115) --($(kaa)!(u0.115)!(kbb)$) node[above]{$y_0$};
\draw[stealth-] (u1.115) --($(kaa)!(u1.115)!(kbb)$) node[above]{$y_1$};
\draw[stealth-] (u2.115) --($(kaa)!(u2.115)!(kbb)$) node[above]{$y_2$};
\draw[stealth-] (u0.east) --++(\kpin,0)node[below]{$c_0$}coordinate(kff)--($(u4.in 2)!(kff)!(u5.in 2)$)--(u4.in 2);
\draw[-stealth] (u0.south) --++(0,-\kpin)coordinate(ka) node[below]{$s_0$};
\draw[-stealth] (u1.south) --++(0,-\kpin)coordinate(kb) node[below]{$s_1$};
\draw[-stealth] (u2.south) --++(0,-\kpin) node[below]{$s_2$};
\draw[-stealth](u0)--(u1)node[pos=0.5,below]{$c_1$};
\draw[-stealth](u1)--(u2)node[pos=0.5,below]{$c_2$};
\draw[-stealth](u2.west)--++(-\kpin,0)coordinate(kcc)node[below]{$c_3$};
\path(kcc)--++(0,\kpin)coordinate(kdd);
\draw(u4.in 2)--(u5.in 2)--(u6.in 2)--($(kcc)!(u6.in 2)!(kdd)$)node[left]{\RL{{$\overline{\text{جمع}}$}/منفی}};
\end{tikzpicture}
\caption{تین بِٹ جمع و منفی کار}
\label{شکل_ترکیبی_تین_درجہ_جمع_و_منفی_کار}
\end{figure}

اس شکل میں بلا شرکت جمع گیٹ استعمال کیا گیا، اور قابو اشارہ {{$\overline{\text{جمع}}$}/منفی} کا اضافہ کیا گیا۔اس قابو اشارہ کی کارکردگی پر غور کرتے ہیں۔جب {{$\overline{\text{جمع}}$}/منفی} اشارہ پست \عددی{(0)}ہو بلا شرکت جمع گیٹ عدد \عددی{z} جوں کا توں مکمل جمع کار تک پہنچائے گا، اور ساتھ ہی \عددی{c_0=0}ہو گا؛لہٰذا یہ دور تین بِٹ جمع کار کی حیثیت سے کام کرے گا۔

اس کے برعکس،{{$\overline{\text{جمع}}$}/منفی} اشارہ بلند \عددی{(1)}ہو بلا شرکت جمع گیٹ عدد \عددی{z} کا متمم \عددی{\overline{z}} مکمل جمع کار تک پہنچائے گا، اور ساتھ ہی \عددی{c_0=1}ہو گا؛لہٰذا یہ دور تین بِٹ منفی کار کی حیثیت سے کام کرے گا۔
	
قابو اشارہ کے نام میں \قول{منفی} اور \قول{\عددیء{\overline{جمع}}} لکھ کر یہ واضح کی گیا ہے کہ اشارہ بلند ہونے کی صورت میں منفی کار اور پست ہونے کی صورت میں جمع کار حاصل ہو گا۔

	
آٹھ بِٹ جمع و منفی کار کو ایک بائٹ جمع و منفی کار کہتے ہیں۔ شکل \حوالہ{شکل_ترکیبی_دو_بائٹ_جمع_و_منفی_کار} میں ایک بائٹ اور دو بائٹ جمع و منفی کار دکھائے گئے ہیں۔اس کے بائیں جانب مزید درجات جوڑ کر متعدد بائٹ کا دور بنایا جا سکتا ہے۔ یہاں \عددی{Y_0} پہلے بائٹ (یعنی بِٹ \عددی{y_0} تا \عددی{y_7}) کو، \عددی{Y_1} اگلے بائٹ (یعنی بِٹ \عددی{y_8} تا \عددی{y_{14}}) کو ظاہر کرتا ہے، جبکہ \عددی{C_2} سے مراد دوسرے بائٹ کی جمع کا خارجی حاصل ہے۔

\begin{figure}
\centering
\begin{subfigure}{1\textwidth}
\centering
\begin{tikzpicture}
\pgfmathsetmacro{\kul}{0.75}
\pgfmathsetmacro{\kpsep}{0.15}
\pgfmathsetmacro{\kum}{0.75}
\pgfmathsetmacro{\kur}{0.5}
\pgfmathsetmacro{\kus}{\kul+7*\kpsep+\kum}
\pgfmathsetmacro{\kll}{\kus}
\pgfmathsetmacro{\kxdim}{\kul+14*\kpsep+\kum+\kur}
\pgfmathsetmacro{\kydim}{1.25}
\pgfmathsetmacro{\kpin}{0.5}
\pgfmathsetmacro{\kpina}{1}
\pgfmathsetmacro{\kvspace}{6}
\pgfmathsetmacro{\khspace}{\kxdim+0.75}
\pgfmathsetmacro{\kvcon}{1.25}
\draw[thick](0,0) rectangle ++(\kxdim,\kydim)node[pos=0.5]{\text{\RL{آٹھ بِٹ جمع و منفی کار}}};
\foreach \x in {0,1,2,3,4,5,6,7}{\draw[stealth-](\kul+\x*\kpsep,\kydim)--++(0,\kpin);}
\foreach \x in {0,1,2,3,4,5,6,7}{\draw[stealth-](\kus+\x*\kpsep,\kydim)--++(0,\kpin);}
\foreach \x in {0,1,2,3,4,5,6,7}{\draw[-stealth](\kll+\x*\kpsep,0)--++(0,-\kpin);}
\draw (\kul+3.5*\kpsep,\kydim+0.75)node[]{$Y$};
\draw (\kus+3.5*\kpsep,\kydim+0.75)node[]{$Z$};
\draw (\kll+3.5*\kpsep,-0.75)node[]{$S$};
\draw[-stealth](-\kpin,\kydim+\kvcon)node[left]{\RL{{$\overline{\text{جمع}}$}/منفی}}
--++(1.5*\kpin,0)coordinate(ucon)--++(0,-\kvcon);
\draw[-stealth](ucon)--++(\kxdim,0)--++(0,-\kydim/2-\kvcon)node[right]{$c_0$}--++(-0.5*\kpin,0);
\draw[-stealth](0,\kydim/2)--++(-\kpin,0)node[left]{$c_8$};
\end{tikzpicture}
\end{subfigure}
\begin{subfigure}{1\textwidth}
\centering
\begin{tikzpicture}
\pgfmathsetmacro{\kul}{0.75}
\pgfmathsetmacro{\kpsep}{0.15}
\pgfmathsetmacro{\kum}{0.75}
\pgfmathsetmacro{\kur}{0.5}
\pgfmathsetmacro{\kus}{\kul+7*\kpsep+\kum}
\pgfmathsetmacro{\kll}{\kus}
\pgfmathsetmacro{\kxdim}{\kul+14*\kpsep+\kum+\kur}
\pgfmathsetmacro{\kydim}{1.25}
\pgfmathsetmacro{\kpin}{0.5}
\pgfmathsetmacro{\kpina}{1}
\pgfmathsetmacro{\kvspace}{6}
\pgfmathsetmacro{\khspace}{\kxdim+0.75}
\pgfmathsetmacro{\kvcon}{1.25}
\draw[thick](0,0) rectangle ++(\kxdim,\kydim)node[pos=0.5]{\text{\RL{آٹھ بِٹ جمع و منفی کار}}};
\foreach \x in {0,1,2,3,4,5,6,7}{\draw[stealth-](\kul+\x*\kpsep,\kydim)--++(0,\kpin);}
\foreach \x in {0,1,2,3,4,5,6,7}{\draw[stealth-](\kus+\x*\kpsep,\kydim)--++(0,\kpin);}
\foreach \x in {0,1,2,3,4,5,6,7}{\draw[-stealth](\kll+\x*\kpsep,0)--++(0,-\kpin);}
\draw (\kul+3.5*\kpsep,\kydim+0.75)node[]{$Y_1$};
\draw (\kus+3.5*\kpsep,\kydim+0.75)node[]{$Z_1$};
\draw (\kll+3.5*\kpsep,-0.75)node[]{$S_1$};
%
\draw[thick](\khspace,0) rectangle ++(\kxdim,\kydim)node[pos=0.5]{\text{\RL{آٹھ بِٹ جمع و منفی کار}}};
\foreach \x in {0,1,2,3,4,5,6,7}{\draw[stealth-](\khspace+\kul+\x*\kpsep,\kydim)--++(0,\kpin);}
\foreach \x in {0,1,2,3,4,5,6,7}{\draw[stealth-](\khspace+\kus+\x*\kpsep,\kydim)--++(0,\kpin);}
\foreach \x in {0,1,2,3,4,5,6,7}{\draw[-stealth](\khspace+\kll+\x*\kpsep,0)--++(0,-\kpin);}
\draw (\khspace+\kul+3.5*\kpsep,\kydim+0.75)node[]{$Y_0$};
\draw (\khspace+\kus+3.5*\kpsep,\kydim+0.75)node[]{$Z_0$};
\draw (\khspace+\kll+3.5*\kpsep,-0.75)node[]{$S_0$};
\draw[-stealth](-\kpin,\kydim+\kvcon)node[left]{\RL{{$\overline{\text{جمع}}$}/منفی}}
--++(1.5*\kpin,0)coordinate(kcon)--++(0,-\kvcon);
\draw[-stealth](kcon)--++(\khspace,0)coordinate(kconr)--++(0,-\kvcon);
\draw(kconr)--++(\kxdim+0.5*\kpin,0)--++(0,-\kydim/2-\kvcon)node[below]{$C_0$}--++(-\kpin,0);
\draw[-stealth](0,\kydim/2)--++(-\kpin,0)node[below]{$C_2$};
\draw[stealth-](\kxdim,\kydim/2)--(\khspace,\kydim/2)node[pos=0.5,below]{$C_1$};
\end{tikzpicture}
\end{subfigure}
\caption{ایک اور دو بائٹ جمع و منفی کار}
\label{شکل_ترکیبی_دو_بائٹ_جمع_و_منفی_کار}
\end{figure}


\جزوحصہ{ اعشاری جمع کار}
جیسا پہلے ذکر ہوا، اعشاری اعداد کو \اصطلاح{ثنائی مرموز اعشاریہ}\فرہنگ{ثنائی مرموز اعشاریہ}\حاشیہب{binary coded decimal (BCD)}\فرہنگ{binary coded decimal (BCD)} سے ظاہر کیا جاتا ہے۔ ایسا مکمل جمع کار بناتے ہیں جو دو اعشاری ہندسوں \عددی{M}، \عددی{N} اور داخلی حاصل \عددی{c_d} کو جمع کرتا ہو۔چونکہ اعشاری ہندسے \عددی{0} تا \عددی{9} ، جبکہ داخلی حاصل \عددی{0} یا \عددی{1} ہو سکتا ہے، لہٰذا اس جمع کار کے جواب \عددی{(M+N+c_d)} کی قیمت \عددی{(0+0+0=0)} تا \عددی{(9+9+1=19)} ہو گی، جنہیں اعشاری، ثنائی مرموز اعشاریہ اور ثنائی روپ میں جدول \حوالہ{جدول_ترکیبی_ثنائی_مرموز_اعشاریہ} میں پیش کیا گیا ہے۔

\begin{table}
\caption{اعشاری جمع کار کے مطلوبہ جواب}
\label{جدول_ترکیبی_ثنائی_مرموز_اعشاریہ}
\centering
\begin{otherlanguage}{english}
\begin{tabular}{CCCCC|CCCCC|R}
\toprule
\multicolumn{5}{c|}{\text{\RL{ثنائی}}}&\multicolumn{5}{c|}{\text{\RL{ثنائی مرموز اعشاریہ}}}&\text{\RL{اعشاری}}\\
b_4&b_3&b_2&b_1&b_0&c&d_3&d_2&d_1&d_0&\\
\midrule
0&0&0&0&0&0&0&0&0&0&0\\
0&0&0&0&1&0&0&0&0&1&1\\
0&0&0&1&0&0&0&0&1&0&2\\
0&0&0&1&1&0&0&0&1&1&3\\
0&0&1&0&0&0&0&1&0&0&4\\
0&0&1&0&1&0&0&1&0&1&5\\
0&0&1&1&0&0&0&1&1&0&6\\
0&0&1&1&1&0&0&1&1&1&7\\
0&1&0&0&0&0&1&0&0&0&8\\
0&1&0&0&1&0&1&0&0&1&9\\
\midrule
0&1&0&1&0&1&0&0&0&0&10\\
0&1&0&1&1&1&0&0&0&1&11\\
0&1&1&0&0&1&0&0&1&0&12\\
0&1&1&0&1&1&0&0&1&1&13\\
0&1&1&1&0&1&0&1&0&0&14\\
0&1&1&1&1&1&0&1&0&1&15\\
1&0&0&0&0&1&0&1&1&0&16\\
1&0&0&0&1&1&0&1&1&1&17\\
1&0&0&1&0&1&1&0&0&0&18\\
1&0&0&1&1&1&1&0&0&1&19\\
\bottomrule
\end{tabular}
\end{otherlanguage}
\end{table}

 جدول میں، چار بِٹ ثنائی روپ میں خارجی حاصل کو \عددی{b_4} ، جبکہ ثنائی مرموز اعشاریہ میں خارجی حاصل کو \عددی{c} سے ظاہر کیا گیا ہے۔ان طریقوں میں \عددی{0} تا \عددی{9} جوابات ایک جیسے، جبکہ \عددی{10} تا \عددی{19} ایک دوسرے سے مختلف لکھے جاتے ہیں۔یوں اگر چار بِٹ ثنائی جمع کار استعمال ہو اور جواب \عددی{0} تا \عددی{9} ہو تب یہی جواب بطور ثنائی مرموز اعشاریہ جواب قابل قبول ہو گا، البتہ \عددی{9} سے بڑے ثنائی جواب کو ثنائی مرموز اعشاریہ جواب تسلیم نہیں کیا جا سکتا۔آئیں دیکھتے ہیں ایسی صورت میں کیا کیا جا سکتا ہے۔

یہاں ایک دلچسپ حقیقت پر غور کرتے ہیں۔ نا قابل قبول ثنائی جواب کے ساتھ \عددی{0110_2} ثنائی طور جمع کرنے سے درست ثنائی مرموز اعشاریہ جواب حاصل ہو گا۔مثلاً،\عددی{01010_2} کے ساتھ \عددی{0110_2} جمع کرنے سے \عددی{10000_2} حاصل ہو گا، جو ثنائی مرموز اعشاریہ میں درست جواب ہے۔ یوں \عددی{0} تا \عددی{9} ثنائی جوابات کو جوں کا توں، جبکہ ان سے بڑے جوابات کے ساتھ \عددی{0110_2} ثنائی طور جمع کر کے ثنائی مرموز اعشاریہ جواب حاصل کیے جا سکتے ہیں۔

	
جدول سے واضح ہے کہ جب ثنائی جمع کار کے جواب میں خارجی حاصل \عددی{b_4} بلند ہو، اس جواب کو ثنائی مرموز اعشاریہ جواب تسلیم نہیں کیا جا سکتا؛ اس کے علاوہ جب \عددی{b_3} کے ساتھ \عددی{b_2} یا \عددی{b_1} بھی بلند ہو تب بھی جواب کو ثنائی مرموز اعشاریہ تسلیم نہیں کیا جا سکتا۔ ان حقائق کو درج ذیل بوولین مساوات بیان کرتی ہے، جہاں ناقابل قبول جواب کی صورت میں \عددی{G} بلند ہو گا۔
 \begin{align}
 G=b_4+b_3b_2+b_3b_1
 \end{align}
اس حقیقت کو استعمال کرتے ہوئے ثنائی جمع کار کی مدد سے ثنائی مرموز اعشاریہ جمع کار کا حصول شکل \حوالہ{شکل_ترکیبی_اعشاری_جمع_کار} میں دکھایا گیا ہے۔
\begin{figure}
\centering
\begin{tikzpicture}
\pgfmathsetmacro{\kul}{0.5}
\pgfmathsetmacro{\kpsep}{0.40}
\pgfmathsetmacro{\kum}{0.75}
\pgfmathsetmacro{\kur}{0.5}
\pgfmathsetmacro{\kll}{1.5}
\pgfmathsetmacro{\kus}{\kul+3*\kpsep+\kum}
\pgfmathsetmacro{\kxdim}{\kul+6*\kpsep+\kum+\kur}
\pgfmathsetmacro{\kydim}{1.25}
\pgfmathsetmacro{\kpin}{0.5}
\pgfmathsetmacro{\kpina}{1}
\pgfmathsetmacro{\kvspace}{5}
\draw[thick](0,0) rectangle ++(\kxdim,\kydim)node[pos=0.5,yshift=0.5em]{\text{\RL{چار بِٹ ثنائی جمع کار}}};
\foreach \x in {0,1,2,3}{\draw[](\kul+\x*\kpsep,\kydim)--++(0,\kpin);}
\foreach \x in {0,1,2,3}{\draw[](\kus+\x*\kpsep,\kydim)--++(0,\kpin);}
\draw(\kxdim,\kydim/2)--++(\kpin,0)node[right]{$c_d$};
\draw (\kul+1.5*\kpsep,\kydim+0.75)node[]{$M$};
\draw (\kus+1.5*\kpsep,\kydim+0.75)node[]{$N$};
\draw[thick](0,-\kvspace) rectangle ++(\kxdim,\kydim)node[pos=0.5,yshift=0.5em]{\text{\RL{چار بِٹ ثنائی جمع کار}}};
\foreach \x/\d in {0/{d_3},1/{d_2},2/{d_1},3/{d_0}}{\draw[](\kus+\x*\kpsep,-\kvspace)--++(0,-\kpin)node[below]{$\d$};}
\foreach \x/\b in {0/{b_3},1/{b_2},2/{b_1},3/{b_0}}{\draw[](\kus+\x*\kpsep,-\kvspace+\kydim)--++(0,\kvspace-\kydim)node[above]{$\b$};}
\draw(\kxdim,-\kvspace+\kydim/2)--++(\kpin,0)node[right]{$0$};
\draw(0,-\kvspace+\kydim/2)--++(-\kpin,0)--++(0,-\kydim/2-\kpin)node[below]{$c$};
%
\draw(\kus-2*\kpsep,-0.50)node[and port, scale=1,number inputs=2,rotate=180,anchor=in 2](u1){};
\draw(\kus-2*\kpsep,-1.75)node[and port, scale=1,number inputs=2,rotate=180, anchor=in 2](u2){};
\draw(u1.out)++(-2,0)node[or port, scale=1,number inputs=3,rotate=180](u3){};
\draw(0,\kydim/2)node[above left]{$b_4$}-|(u3.in 3) (u1.out)--(u3.in 2) (u2.out)-|(u3.in 1);
\draw(u1.in 1)--++(2*\kpsep,0) (u1.in 2)--++(3*\kpsep,0) (u2.in 1)--++(2*\kpsep,0) (u2.in 2)--++(4*\kpsep,0);
%
\foreach \x in {0,3}{\draw[](\kul+\x*\kpsep,-\kvspace+\kydim)--++(0,\kpin);}
\foreach \x in {1,2}{\draw[](\kul+\x*\kpsep,-\kvspace+\kydim)--++(0,\kpina);}
\draw(\kul+3*\kpsep,-\kvspace+\kydim)++(0,\kpin)--++(-2,0)node[left]{$0$};
\draw(\kul+2*\kpsep,-\kvspace+\kydim)++(0,\kpina)--++(-2,0)-|(u3.out)node[above left]{$G$};
\end{tikzpicture}
\caption{ثنائی مرموز اعشاریہ روپ میں اعشاری جمع کار}
\label{شکل_ترکیبی_اعشاری_جمع_کار}
\end{figure}
 اگر \عددی{G} پست ہو تب نچلا جمع کار بالائی جمع کار کے جواب کے ساتھ \عددی{0} جمع کر کے اسی جواب کو خارج کرتا ہے، جبکہ \عددی{G} بلند ہونے کی صورت میں ساتھ \عددی{0110_2} جمع کر کے درست ثنائی مرموز اعشاریہ خارج کرتا ہے۔


\حصہ{ثنائی ضرب کار}
ثنائی ضرب بالکل اعشاری ضرب کی طرح کی جاتی ہے۔دو بِٹ ثنائی اعداد \عددی{y} اور \عددی{z} کو قلم و کاغذ کی طرز پر ضرب کرتے ہیں۔
\begin{align*}
\begin{array}{cccc}
&& z_1&z_0\\
&& y_1&y_0\\
\cline{3-4}
&& y_0z_1&y_0z_0\\
&y_1z_1&y_1z_0&\\
\cline{1-4}
m_3&m_2&m_1&m_0
\end{array}
\end{align*}
اس مساوات سے حاصل دو بِٹ ثنائی ضرب کار شکل \حوالہ{شکل_ترکیبی_دو_بٹ_ضرب_کار} میں پیش ہے۔زیادہ بِٹ کے ضرب کار بھی اسی طرح تشکیل دیے جاتے ہیں۔

 درج بالا قلم و کاغذ کی طرز پر ضرب میں کمتر بِٹ \عددی{m_0=y_0z_0} ہے جو شکل میں جمع گیٹ \عددی{u_4} دیتا ہے۔ اگلا بِٹ \عددی{m_1} ہے جو \عددی{y_0z_1} اور \عددی{y_1z_0} کو جمع کر کے حاصل ہو گا۔ جمع گیٹ \عددی{u_3} ہمیں \عددی{y_0z_1} جبکہ \عددی{u_2} ہمیں \عددی{y_1z_0} دیتا ہے، جنہیں دایاں نصف جمع کار \عددی{u_6} آپس میں جمع کر کے \عددی{m_1} اور حاصل (اگر موجود ہو) دیتا ہے۔ اس حاصل کو \عددی{y_1z_1} (جو گیٹ \عددی{u_1} سے ملتا ہے) کے ساتھ بایاں نصف جمع کار \عددی{u_5} ملا کر \عددی{m_2} اور حاصل \عددی{m_3} دیگا۔ 
\begin{figure}
\centering
\begin{tikzpicture}
\pgfmathsetmacro{\kul}{1}
\pgfmathsetmacro{\kpsep}{0.15}
\pgfmathsetmacro{\kum}{1.5}
\pgfmathsetmacro{\kur}{\kul}
\pgfmathsetmacro{\kus}{\kul+\kum}
\pgfmathsetmacro{\kll}{\kus}
\pgfmathsetmacro{\kxdim}{\kul+\kum+\kur}
\pgfmathsetmacro{\kydim}{1.25}
\pgfmathsetmacro{\kpin}{0.5}
\pgfmathsetmacro{\kpina}{1.5}
\pgfmathsetmacro{\kvspace}{6}
\pgfmathsetmacro{\khspace}{\kxdim+1.5}
\pgfmathsetmacro{\kcona}{3}
\pgfmathsetmacro{\kconb}{2.5}
\pgfmathsetmacro{\kconc}{2}
\draw[thick](0,0) rectangle ++(\kxdim,\kydim)node [pos=0.5,below]{$u_5$}node[pos=0.5,above]{\text{\RL{نصف جمع کار}}};
\draw[thick](\khspace,0) rectangle ++(\kxdim,\kydim)node [pos=0.5,below]{$u_6$}node[pos=0.5,above]{\text{\RL{نصف جمع کار}}};
\draw(\kul,\kydim)node[and port,scale=1,number inputs=2,rotate=-90,anchor=out](u1){\rotatebox{90}{$u_1$}};
\draw(\kul+\khspace,\kydim)node[and port,scale=1,number inputs=2,rotate=-90,anchor=out](u2){\rotatebox{90}{$u_2$}};
\draw(\kus+\khspace,\kydim)node[and port,scale=1,number inputs=2,rotate=-90,anchor=out](u3){\rotatebox{90}{$u_3$}};
\draw(\khspace+\kxdim+\kpin,\kydim)node[and port,scale=1,number inputs=2,rotate=-90,anchor=out](u4){\rotatebox{90}{$u_4$}};
\draw(-\kpin,\kydim+\kconb)node[left]{$y_1$}-|(u2.in 2);
\draw(\khspace+\kxdim+\kpina,\kydim+\kconb)node[right]{$z_0$}-|(u2.in 1);
\path(-\kpina,\kydim+\kconb)--++(\kpin,0)coordinate(aa);
\draw(u1.in 2)--($(-\kpina,\kydim+\kconb)!(u1.in 2)!(aa)$);
\draw(\khspace+\kxdim+\kpina,\kydim+\kconc)node[right]{$z_1$}-|(u1.in 1);
\path(\khspace+\kxdim+\kpina,\kydim+\kconc)--++(-\kpin,0)coordinate(bb);
\draw(u3.in 1)--($(\khspace+\kxdim+\kpina,\kydim+\kconc)!(u3.in 1)!(bb)$);
\path(\khspace+\kxdim+\kpina,\kydim+\kconb)--++(-\kpin,0)coordinate(cc);
\draw(u4.in 1)--($(\khspace+\kxdim+\kpina,\kydim+\kconb)!(u4.in 1)!(cc)$);
\draw(-\kpin,\kydim+\kcona)node[left]{$y_0$}-|(u4.in 2);
\path(-\kpin,\kydim+\kcona)--++(-\kpin,0)coordinate(dd);
\draw(u3.in 2)--($(-\kpin,\kydim+\kcona)!(u3.in 2)!(dd)$);
\draw(0,\kydim/2)--++(-\kpin,0)--++(0,-\kydim/2-\kpin)coordinate(lll)node[below]{$m_3$} (\kus,0)--++(0,-\kpin)coordinate(llr)node[below]{$m_2$};
\draw(\khspace+\kus,0)--++(0,-\kpin)node[below]{$m_1$};
\draw(u4.out)--($(lll)!(u4.out)!(llr)$)node[below]{$m_0$};
\draw(\khspace,\kydim/2)--++(-\khspace/2+\kxdim/2,0)--++(0,\kydim/2+\kpin)-|(\kus,\kydim);
\end{tikzpicture}
\caption{دو بِٹ ثنائی ضرب کار}
\label{شکل_ترکیبی_دو_بٹ_ضرب_کار}
\end{figure}

\ابتدا{مشق}
ثنائی اعداد \عددی{11_2} اور \عددی{10_0} جمع کرنے کے قدم شکل \حوالہ{شکل_ترکیبی_دو_بٹ_ضرب_کار} کے دور میں کرتے ہوئے دکھائیں۔ 
\انتہا{مشق}
\ابتدا{مشق}
انٹرنیٹ سے \عددی{74284} مخلوط دور کے معلوماتی صفحات حاصل کریں۔یہ مخلوط دور کیا کام سرانجام دیتا ہے؟
\انتہا{مشق}

 
\حصہ{شناخت کار}
دو بِٹ چار علامتوں \عددی{(2^2)} کو ظاہر کر سکتا ہے، جبکہ \عددی{n} بِٹ \عددی{2^n} علامتوں کو ظاہر کر سکتا ہے۔ایسا دور جو \عددی{n} مداخل کو دیکھ \عددی{2^n} منفرد مخارج میں سے ایک چُن سکے \اصطلاح{شناخت کار}\فرہنگ{شناخت کار}\حاشیہب{decoder}\فرہنگ{decoder} کہلاتا ہے۔اگر شناخت کار کے \عددی{n}مداخل کے تمام ترتیب زیر استعمال نہ لائے گئے ہوں، تب اس کے مخارج \عددی{2^n} سے کم ہوں گے۔
\begin{figure}
\centering
\begin{subfigure}{0.4\textwidth}
\centering
\begin{tikzpicture}
\pgfmathsetmacro{\kpin}{0.5}
\pgfmathsetmacro{\kpsep}{0.40}
\pgfmathsetmacro{\kul}{0.50}
\pgfmathsetmacro{\kxdim}{1.5}
\pgfmathsetmacro{\kydim}{2*\kul+3*\kpsep}
\draw[thick](0,0) rectangle ++(\kxdim,\kydim);
\foreach \y/\a in {1/{a_0},2/{a_1}}{\draw(0,\kul+\y*\kpsep)node[right]{$\a$}--++(-\kpin,0);}
\foreach \y/\d in {0/{d_0},1/{d_1},2/{d_2},3/{d_3}}{\draw(\kxdim,\kul+\y*\kpsep)node[left]{$\d$}--++(\kpin,0);}
\end{tikzpicture}
\end{subfigure}\hfill
\begin{subfigure}{0.6\textwidth}
\centering
\begin{otherlanguage}{english}
\begin{tabular}{CC|CCCC}
\toprule
\multicolumn{2}{c|}{\text{\RL{داخلی بِٹ}}}&\multicolumn{4}{c}{\text{\RL{خارجی بِٹ}}}\\
a_0&a_0&d_3&d_2&d_1&d_0\\
\midrule
0&0&0&0&0&1\\
0&1&0&0&1&0\\
1&0&0&1&0&0\\
1&1&1&0&0&0\\
\bottomrule
\end{tabular}
\end{otherlanguage}
\end{subfigure}
\caption{دو سے چار شناخت کار}
\label{شکل_ترکیبی_دو_چار_شناخت_کار}
\end{figure}
شکل \حوالہ{شکل_ترکیبی_دو_چار_شناخت_کار} میں دو سے چار شناخت کار کی علامت اور کارکردگی کا جدول پیش ہیں۔داخلی بِٹوں کی ہر منفرد ترتیب، خارجی بِٹوں میں سے ایک منفرد بِٹ منتخب کرتی ہے۔یہاں چنی گئی بِٹ بلند کی گئی ہے، شناخت کار یوں بھی تشکیل دی جا سکتی ہے کہ منتخب بِٹ پست ہو۔

 مداخل \عددی{00} (جدول کی پہلی صف) کرنے سے چار مخارج میں سے ایک، یعنی \عددی{d_0} کی شناخت ہوتی ہے۔اسی طرح \عددی{01} مخارج \عددی{d_1} کی، \عددی{10} مخارج \عددی{d_2} کی ، اور \عددی{11}مخارج \عددی{d_3} کی شناخت کرتے ہیں۔
 
 اگر \عددی{d} چار مختلف جگہیں، مثلاً، چار گلیاں، یا چار مکان، تصور کی جائیں، تب \عددی{a} ان کا پتہ ہو گا، جس کے ذریعہ ان تک پہنچنا ممکن ہو گا۔ اسی مشابہت سے \عددی{a} کو \اصطلاح{پتہ کے بِٹ} یا \اصطلاح{پتہ بِٹ}\فرہنگ{پتہ بِٹ}\حاشیہب{address bits}\فرہنگ{address bits} یا صرف \اصطلاح{پتہ}\فرہنگ{پتہ}\حاشیہب{address}\فرہنگ{address} کہتے ہیں۔عددی برقیات میں اس طرح جگہ تعین کرنے والے \قول{پتہ کے بِٹوں} کا استعمال عام ہے اور انہیں، عموماً، \عددی{a}سے ظاہر کیا جاتا ہے۔

کسی بھی پتہ کو اعشاری روپ میں لکھیں؛ یہی مقام منتخب ہو گا۔ یوں \عددی{101_2} پتہ مقام \عددی{5_{10}} یعنی \عددی{d_5} منتخب کرے گا۔



شکل \حوالہ{شکل_ترکیبی_دو_چار_شناخت_کار} میں دیے جدول کو مخارج کے لئے حل کر کے درج ذیل حاصل ہوں گے۔
\begin{align*}
d_0&=\overline{a}_1\overline{a}_0\\
d_1&=\overline{a}_1a_0\\
d_2&=a_1\overline{a}_0\\
d_3&=a_1a_0
\end{align*}

شکل \حوالہ{شکل_ترکیبی_دو_با_چار_شناخت_کار} میں ان مساوات سے حاصل دو با چار \عددی{(2\times 4)} \اصطلاح{ شناخت کار پیش}\فرہنگ{شناخت کار}\حاشیہب{decoder}\فرہنگ{decoder} ہے، جس کے داخلی بِٹ کی تعداد دو \عددی{(2)}، جبکہ خارجی بِٹ کی تعداد چار \عددی{(4)} ہے۔ 
\begin{figure}
\centering
\begin{tikzpicture}
\pgfmathsetmacro{\kpin}{0.5}
\pgfmathsetmacro{\kxsep}{4}
\pgfmathsetmacro{\kysep}{1.25}
\pgfmathsetmacro{\kysepa}{1.5}
\pgfmathsetmacro{\kxa}{0.5}
\pgfmathsetmacro{\kxb}{0.75}
\pgfmathsetmacro{\kxc}{1}
\pgfmathsetmacro{\kxd}{1.25}
\draw(0,\kysepa) node[not port,scale=1](u1){};
\draw(0,0)node[not port, scale=1](u2){};
\draw(\kxsep,-\kysep)node[and port, scale=1,number inputs=2](u3){};
\draw(\kxsep,-2*\kysep)node[and port, scale=1,number inputs=2](u4){};
\draw(\kxsep,-3*\kysep)node[and port, scale=1,number inputs=2](u5){};
\draw(\kxsep,-4*\kysep)node[and port, scale=1,number inputs=2](u6){};
\draw(u6.in 1)--++(-\kxa,0)coordinate(aa) |-coordinate(bb)(u1.out)node[above]{$\overline{a}_1$};
\draw($(aa)!(u5.in 1)!(bb)$)--(u5.in 1);
\draw(u1.in)--++(0,-\kysepa/2)coordinate(cc) (u1.in)--++(-\kpin,0)node[left]{$a_1$}; 
\draw(u4.in 1)--++(-\kxb,0) |-(cc);
\draw(u3.in 1)--++(-\kxb,0);
\draw(u6.in 2)--++(-\kxc,0) |-(u2.out)node[above]{$\overline{a}_0$} (u4.in 2)--++(-\kxc,0);
\draw(u3.in 2)-|coordinate(ff)(u2.in)--++(-\kpin,0)node[left]{$a_0$};
\draw(u5.in 2)--++(-\kxd,0)coordinate(gg)--($(u3.in 2)!(gg)!(ff)$);
\draw(u6.out)node[right]{$d_0$};
\draw(u5.out)node[right]{$d_1$};
\draw(u4.out)node[right]{$d_2$};
\draw(u3.out)node[right]{$d_3$};
\end{tikzpicture}
\caption{دو با چار شناخت کار}
\label{شکل_ترکیبی_دو_با_چار_شناخت_کار}
\end{figure}

شکل \حوالہ{شکل_ترکیبی_دو_با_چار_شناخت_کار} میں پیش شناخت کار کے تمام ضرب گیٹوں کے ساتھ اضافی قابو مداخل جوڑ کر مجاز و معذور صلاحیت کا \عددی{(2\times 4)} شناخت کار حاصل ہو گا، جو شکل \حوالہ{شکل_ترکیبی_مجازومعذور_شناخت_کار} میں پیش ہے۔ شناخت کار، بلند قابو اشارہ \عددی{(e)} کی صورت میں، شناخت کرنے کا مجاذ ہو گا، پست اشارے کی صورت میں شناخت کار معذور ہو گا اور اس کے تمام مخارج پست ہوں گے۔ شکل-ب میں اس کی علامت پیش کی گئی ہے، جہاں قابو اشارہ کو مختصراً \قول{مجاز} کہا گیا ہے۔

\begin{figure}
\centering
\begin{subfigure}{0.6\textwidth}
\centering
\begin{tikzpicture}
\pgfmathsetmacro{\kpin}{0.5}
\pgfmathsetmacro{\kxsep}{4}
\pgfmathsetmacro{\kysep}{1.25}
\pgfmathsetmacro{\kysepa}{1.5}
\pgfmathsetmacro{\kxa}{0.5}
\pgfmathsetmacro{\kxb}{0.75}
\pgfmathsetmacro{\kxc}{1}
\pgfmathsetmacro{\kxd}{1.25}
\draw(0,\kysepa) node[not port,scale=1](u1){};
\draw(0,0)node[not port, scale=1](u2){};
\draw(\kxsep,-\kysep)node[and port, scale=1,number inputs=3](u3){};
\draw(\kxsep,-2*\kysep)node[and port, scale=1,number inputs=3](u4){};
\draw(\kxsep,-3*\kysep)node[and port, scale=1,number inputs=3](u5){};
\draw(\kxsep,-4*\kysep)node[and port, scale=1,number inputs=3](u6){};
\draw(u6.in 1)--++(-\kxa,0)coordinate(aa) |-coordinate(bb)(u1.out)node[above]{$\overline{a}_1$};
\draw($(aa)!(u5.in 1)!(bb)$)--(u5.in 1);
\draw(u1.in)--++(0,-\kysepa/2)coordinate(cc) (u1.in)--++(-\kpin,0)node[left]{$a_1$}; 
\draw(u4.in 1)--++(-\kxb,0) |-(cc);
\draw(u3.in 1)--++(-\kxb,0);
\draw(u6.in 2)--++(-\kxc,0) |-(u2.out)node[above]{$\overline{a}_0$} (u4.in 2)--++(-\kxc,0);
\draw(u3.in 2)-|coordinate(ff)(u2.in)--++(-\kpin,0)node[left]{$a_0$};
\draw(u5.in 2)--++(-\kxd,0)coordinate(gg)--($(u3.in 2)!(gg)!(ff)$);
\draw(u6.out)node[right]{$d_0$};
\draw(u5.out)node[right]{$d_1$};
\draw(u4.out)node[right]{$d_2$};
\draw(u3.out)node[right]{$d_3$};
\draw(u3.in 3)--(u6.in 3)--++(-\kxsep+\kpin,0)node[below]{$e$}node[above]{$\text{مجاز}/\overline{\text{معذور}}$};
\end{tikzpicture}
\caption{}
\end{subfigure}\hfill
\begin{subfigure}{0.40\textwidth}
\centering
\begin{tikzpicture}
\pgfmathsetmacro{\kpin}{0.5}
\pgfmathsetmacro{\kpsep}{0.40}
\pgfmathsetmacro{\kul}{0.50}
\pgfmathsetmacro{\kxdim}{1.5}
\pgfmathsetmacro{\kydim}{2*\kul+3*\kpsep}
\draw[thick](0,0) rectangle ++(\kxdim,\kydim);
\foreach \y/\a in {0/e,2/{a_0},3/{a_1}}{\draw(0,\kul+\y*\kpsep)node[right]{$\a$}--++(-\kpin,0);}
\foreach \y/\d in {0/{d_0},1/{d_1},2/{d_2},3/{d_3}}{\draw(\kxdim,\kul+\y*\kpsep)node[left]{$\d$}--++(\kpin,0);}
\draw(0,\kul)node[above left]{مجاز};
\end{tikzpicture}
\caption{}
\end{subfigure}
\caption{مجاز و معذور صلاحیت کا دو با چار شناخت کار}
\label{شکل_ترکیبی_مجازومعذور_شناخت_کار}
\end{figure}


\begin{table}
\caption{مجاز و معذور صلاحیت کا شناخت کار}
\label{جدول_ترکیبی_مجازومعذار_شناخت_کار}
\centering
\begin{subtable}{0.45\textwidth}
\caption{}
\centering
\begin{otherlanguage}{english}
\begin{tabular}{CCC|CCCC}
\toprule
e&a_1&a_0&d_3&d_2&d_1&d_0\\
\midrule
0&0&0&0&0&0&0\\
0&0&1&0&0&0&0\\
0&1&0&0&0&0&0\\
0&1&1&0&0&0&0\\
1&0&0&0&0&0&1\\
1&0&1&0&0&1&0\\
1&1&0&0&1&0&0\\
1&1&1&1&0&0&0\\
\bottomrule
\end{tabular}
\end{otherlanguage}
\end{subtable}\hfill
\begin{subtable}{0.45\textwidth}
\caption{}
\centering
\begin{otherlanguage}{english}
\begin{tabular}{CCC|CCCC}
\toprule
e&a_1&a_0&d_3&d_2&d_1&d_0\\
\midrule
0&x&x&0&0&0&0\\
1&0&0&0&0&0&1\\
1&0&1&0&0&1&0\\
1&1&0&0&1&0&0\\
1&1&1&1&0&0&0\\
\bottomrule
\end{tabular}
\end{otherlanguage}
\end{subtable}
\end{table}

 جدول \حوالہ{جدول_ترکیبی_مجازومعذار_شناخت_کار}-الف میں مجاز و معذور صلاحیت کے شناخت کار کی کارکردگی پیش کی گئی ہے۔اس جدول کو مختصراً جدول-ب کی صورت میں پیش کیا جاتا ہے، جہاں پہلی صف میں قابو اشارہ پست \عددی{(e=0)} ہے لہٰذا \عددی{a_0} اور \عددی{a_1} کی قیمتیں اہمیت نہیں رکھتی؛ یوں پہلی صف میں \عددی{a_0} اور \عددی{a_1} کی قیمت \عددی{x} لکھی جاتی ہے۔ 


\begin{table}
\caption{بلند عمل پیرا، تین با آٹھ شناخت کار}
\label{جدول_ترکیبی_تین_با_آٹھ_شناخت_کار}
\centering
\begin{otherlanguage}{english}
\begin{tabular}{CCC|CCCCCCCC}
\toprule
a_2&a_1&a_0&d_7&d_6&d_5&d_4&d_3&d_2&d_1&d_0\\
\midrule
0&0&0&0&0&0&0&0&0&0&1\\
0&0&1&0&0&0&0&0&0&1&0\\
0&1&0&0&0&0&0&0&1&0&0\\
0&1&1&0&0&0&0&1&0&0&0\\
1&0&0&0&0&0&1&0&0&0&0\\
1&0&1&0&0&1&0&0&0&0&0\\
1&1&0&0&1&0&0&0&0&0&0\\
1&1&1&1&0&0&0&0&0&0&0\\
\bottomrule
\end{tabular}
\end{otherlanguage}
\end{table}

تین با آٹھ \عددی{(3\times 8)} شناخت کار کا دور حاصل کرنے کی خاطر، تین مداخل کا ایسا جدول لکھتے ہیں جس میں مداخل کی ہر ترتیب ایک منفرد مخارج
 منتخب کرے ( جدول \حوالہ{جدول_ترکیبی_تین_با_آٹھ_شناخت_کار} دیکھیں)۔چونکہ چنُا گیا مخارج بلند ہو گا، لہٰذا ایسا شناخت کار، \اصطلاح{ بلند عمل پیرا}\فرہنگ{عمل پیرا!بلند}\حاشیہب{active high}\فرہنگ{active!high} کہلاتا ہے۔مخارج تفاعلات کی مساوات، مجموعہ ارکان ضرب کی صورت میں حاصل کرتے ہیں۔
\begin{align*}
d_0&=\overline{a}_2 \overline{a}_1 \overline{a}_0\\
d_1&=\overline{a}_2 \overline{a}_1 a_0\\
d_2&=\overline{a}_2 a_1 \overline{a}_0\\
d_3&=\overline{a}_2 a_1 a_0\\
d_4&=a_2\, \overline{a}_1 \overline{a}_0\\
d_5&=a_2\, \overline{a}_1 a_0\\
d_6&=a_2\, a_1\, \overline{a}_0\\
d_7&=a_2\, a_1\, a_0
\end{align*}
ان تفاعلات سے حاصل، بلند عمل پیرا، تین با آٹھ \عددی{(3\times 8)} شناخت کار شکل \حوالہ{شکل_ترکیبی_تین_با_آٹھ_شناخت_کار} میں پیش ہے۔
\begin{figure}
\centering
\begin{tikzpicture}
\pgfmathsetmacro{\kxs}{1.25}
\pgfmathsetmacro{\kxsa}{1.5}
\pgfmathsetmacro{\kstep}{0.25}
\pgfmathsetmacro{\kbase}{0.5}
\pgfmathsetmacro{\kys}{\kbase+5*\kstep}
\pgfmathsetmacro{\kpin}{0.5}
\pgfmathsetmacro{\kpina}{\kxsa/2}
\draw(0*\kxs,0)node[and port,scale=1,number inputs=3,rotate=-90,anchor=in 1](u7){};
\draw(1*\kxs,0)node[and port,scale=1,number inputs=3,rotate=-90,anchor=in 1](u6){};
\draw(2*\kxs,0)node[and port,scale=1,number inputs=3,rotate=-90,anchor=in 1](u5){};
\draw(3*\kxs,0)node[and port,scale=1,number inputs=3,rotate=-90,anchor=in 1](u4){};
\draw(4*\kxs,0)node[and port,scale=1,number inputs=3,rotate=-90,anchor=in 1](u3){};
\draw(5*\kxs,0)node[and port,scale=1,number inputs=3,rotate=-90,anchor=in 1](u2){};
\draw(6*\kxs,0)node[and port,scale=1,number inputs=3,rotate=-90,anchor=in 1](u1){};
\draw(7*\kxs,0)node[and port,scale=1,number inputs=3,rotate=-90,anchor=in 1](u0){};
\draw(0.5,\kys)node[not port,scale=0.9,rotate=-90,anchor=out](u8){};
\draw(0.5-\kxsa,\kys)node[not port,scale=0.9,rotate=-90,anchor=out](u9){};
\draw(0.5-2*\kxsa,\kys)node[not port,scale=0.9,rotate=-90,anchor=out](u10){};
\draw(u8.in)--++(0,\kpin)node[above]{$a_0$};
\draw(u9.in)--++(0,\kpin)node[above]{$a_1$};
\draw(u10.in)--++(0,\kpin)node[above]{$a_2$};
\draw(u0.out)node[below]{$d_0$} (u1.out)node[below]{$d_1$} (u2.out)node[below]{$d_2$} 
(u3.out)node[below]{$d_3$} (u4.out)node[below]{$d_4$} (u5.out)node[below]{$d_5$}
(u6.out)node[below]{$d_6$} (u7.out)node[below]{$d_7$};
\draw(u8.in)--++(-\kpina,0)coordinate(kin8);
\draw(u9.in)--++(-\kpina,0)coordinate(kin9);
\draw(u10.in)--++(-\kpina,0)coordinate(kin10);
\draw(u0.in 1)++(\kpin/2,\kbase+0*\kstep)-|(kin10);
\draw(u0.in 1)++(\kpin/2,\kbase+1*\kstep)-|(u10.out);
\draw(u0.in 1)++(\kpin/2,\kbase+2*\kstep)-|(kin9);
\draw(u0.in 1)++(\kpin/2,\kbase+3*\kstep)-|(u9.out);
\draw(u0.in 1)++(\kpin/2,\kbase+4*\kstep)-|(kin8);
\draw(u0.in 1)++(\kpin/2,\kbase+5*\kstep)-|(u8.out);
\draw(u0.in 1)--++(0,\kbase+5*\kstep) (u0.in 2)--++(0,\kbase+3*\kstep) (u0.in 3)--++(0,\kbase+1*\kstep);
\draw(u1.in 1)--++(0,\kbase+4*\kstep) (u1.in 2)--++(0,\kbase+3*\kstep) (u1.in 3)--++(0,\kbase+1*\kstep);
\draw(u2.in 1)--++(0,\kbase+5*\kstep) (u2.in 2)--++(0,\kbase+2*\kstep) (u2.in 3)--++(0,\kbase+1*\kstep);
\draw(u3.in 1)--++(0,\kbase+4*\kstep) (u3.in 2)--++(0,\kbase+2*\kstep) (u3.in 3)--++(0,\kbase+1*\kstep);
\draw(u4.in 1)--++(0,\kbase+5*\kstep) (u4.in 2)--++(0,\kbase+3*\kstep) (u4.in 3)--++(0,\kbase+0*\kstep);
\draw(u5.in 1)--++(0,\kbase+4*\kstep) (u5.in 2)--++(0,\kbase+3*\kstep) (u5.in 3)--++(0,\kbase+0*\kstep);
\draw(u6.in 1)--++(0,\kbase+5*\kstep) (u6.in 2)--++(0,\kbase+2*\kstep) (u6.in 3)--++(0,\kbase+0*\kstep);
\draw(u7.in 1)--++(0,\kbase+4*\kstep) (u7.in 2)--++(0,\kbase+2*\kstep) (u7.in 3)--++(0,\kbase+0*\kstep);
\end{tikzpicture}
\caption{بلند عمل پیرا، تین با آٹھ \عددی{(3\times 8)} شناخت کار}
\label{شکل_ترکیبی_تین_با_آٹھ_شناخت_کار}
\end{figure}


اس میں مجاز مداخل کا اضافہ کرنے سے مجاز و معذور صلاحیت، بلند عمل پیرا، تین با آٹھ شناخت کار حاصل ہو گا جو شکل \حوالہ{شکل_ترکیبی_تین_با_آٹھ_مجاز_شناخت_کار} میں پیش ہے۔ مجاز بلند ہونے کی صورت میں شناخت کار کام کرے گا، جبکہ پست مجاز کی صورت میں تمام مخارج پست رہیں گے؛ ہم کہتے ہیں یہ \اصطلاح{بلند مجاز}\فرہنگ{مجاز!بلند}\حاشیہب{active high}\فرہنگ{active!high} شناخت کار ہے۔ جدول \حوالہ{جدول_ترکیبی_مجاز_تین_با_آٹھ_شناخت_کار} میں اس کی کارکردگی پیش کی گئی ہے۔ پہلی صف میں \عددی{e} پست ہے، لہٰذا، شناخت کار معذور ہو گا، اور اس کے تین مداخل \عددی{a_0}، \عددی{a_1}، اور \عددی{a_2} کی قیمتیں اہمیت نہیں رکھتی؛ اسی لئے انہیں \عددی{x} لکھا گیا ہے جو \عددی{0} یا \عددی{1} ہو سکتا ہے۔ یہ (پہلی) صف درحقیقت، \عددی{a_2a_1a_0} کی آٹھ \عددی{(8)} قیمتوں، \عددی{000_2} تا \عددی{111_2}، لہٰذا، آٹھ صفوں کو ظاہر کرتی ہے۔
\begin{figure}
\centering
\begin{tikzpicture}
\pgfmathsetmacro{\kxs}{1.25}
\pgfmathsetmacro{\kxsa}{1.5}
\pgfmathsetmacro{\kstep}{0.25}
\pgfmathsetmacro{\kbase}{1}
\pgfmathsetmacro{\kys}{\kbase+5*\kstep}
\pgfmathsetmacro{\kpin}{0.5}
\pgfmathsetmacro{\kpina}{\kxsa/2}
\draw(0*\kxs,0)node[and port,scale=1,number inputs=4,rotate=-90,anchor=in 1](u7){};
\draw(1*\kxs,0)node[and port,scale=1,number inputs=4,rotate=-90,anchor=in 1](u6){};
\draw(2*\kxs,0)node[and port,scale=1,number inputs=4,rotate=-90,anchor=in 1](u5){};
\draw(3*\kxs,0)node[and port,scale=1,number inputs=4,rotate=-90,anchor=in 1](u4){};
\draw(4*\kxs,0)node[and port,scale=1,number inputs=4,rotate=-90,anchor=in 1](u3){};
\draw(5*\kxs,0)node[and port,scale=1,number inputs=4,rotate=-90,anchor=in 1](u2){};
\draw(6*\kxs,0)node[and port,scale=1,number inputs=4,rotate=-90,anchor=in 1](u1){};
\draw(7*\kxs,0)node[and port,scale=1,number inputs=4,rotate=-90,anchor=in 1](u0){};
\draw(0.5,\kys)node[not port,scale=1,rotate=-90,anchor=out](u8){};
\draw(0.5-\kxsa,\kys)node[not port,scale=1,rotate=-90,anchor=out](u9){};
\draw(0.5-2*\kxsa,\kys)node[not port,scale=1,rotate=-90,anchor=out](u10){};
\draw(u8.in)--++(0,\kpin)node[above]{$a_0$};
\draw(u9.in)--++(0,\kpin)node[above]{$a_1$};
\draw(u10.in)--++(0,\kpin)node[above]{$a_2$};
\draw(u0.out)node[below]{$d_0$} (u1.out)node[below]{$d_1$} (u2.out)node[below]{$d_2$} 
(u3.out)node[below]{$d_3$} (u4.out)node[below]{$d_4$} (u5.out)node[below]{$d_5$}
(u6.out)node[below]{$d_6$} (u7.out)node[below]{$d_7$};
\draw(u8.in)--++(-\kpina,0)coordinate(kin8);
\draw(u9.in)--++(-\kpina,0)coordinate(kin9);
\draw(u10.in)--++(-\kpina,0)coordinate(kin10);
\draw(u0.in 1)++(\kpin/2,\kbase+0*\kstep)-|(kin10);
\draw(u0.in 1)++(\kpin/2,\kbase-2*\kstep)-|coordinate(ken)(u7.in 4) (ken)--++(-4*\kpin,0)node[left]{$e$}node[below,xshift=0.5em,yshift=-0.25em]{مجاز};
\draw(u0.in 1)++(\kpin/2,\kbase+1*\kstep)-|(u10.out);
\draw(u0.in 1)++(\kpin/2,\kbase+2*\kstep)-|(kin9);
\draw(u0.in 1)++(\kpin/2,\kbase+3*\kstep)-|(u9.out);
\draw(u0.in 1)++(\kpin/2,\kbase+4*\kstep)-|(kin8);
\draw(u0.in 1)++(\kpin/2,\kbase+5*\kstep)-|(u8.out);
\draw(u0.in 1)--++(0,\kbase+5*\kstep) (u0.in 2)--++(0,\kbase+3*\kstep) (u0.in 3)--++(0,\kbase+1*\kstep);
\draw(u1.in 1)--++(0,\kbase+4*\kstep) (u1.in 2)--++(0,\kbase+3*\kstep) (u1.in 3)--++(0,\kbase+1*\kstep);
\draw(u2.in 1)--++(0,\kbase+5*\kstep) (u2.in 2)--++(0,\kbase+2*\kstep) (u2.in 3)--++(0,\kbase+1*\kstep);
\draw(u3.in 1)--++(0,\kbase+4*\kstep) (u3.in 2)--++(0,\kbase+2*\kstep) (u3.in 3)--++(0,\kbase+1*\kstep);
\draw(u4.in 1)--++(0,\kbase+5*\kstep) (u4.in 2)--++(0,\kbase+3*\kstep) (u4.in 3)--++(0,\kbase+0*\kstep);
\draw(u5.in 1)--++(0,\kbase+4*\kstep) (u5.in 2)--++(0,\kbase+3*\kstep) (u5.in 3)--++(0,\kbase+0*\kstep);
\draw(u6.in 1)--++(0,\kbase+5*\kstep) (u6.in 2)--++(0,\kbase+2*\kstep) (u6.in 3)--++(0,\kbase+0*\kstep);
\draw(u7.in 1)--++(0,\kbase+4*\kstep) (u7.in 2)--++(0,\kbase+2*\kstep) (u7.in 3)--++(0,\kbase+0*\kstep);
\draw(u0.in 4)--++(0,\kbase-2*\kstep) (u1.in 4)--++(0,\kbase-2*\kstep) (u2.in 4)--++(0,\kbase-2*\kstep)
(u3.in 4)--++(0,\kbase-2*\kstep) (u4.in 4)--++(0,\kbase-2*\kstep) 
(u5.in 4)--++(0,\kbase-2*\kstep) (u6.in 4)--++(0,\kbase-2*\kstep);
\end{tikzpicture}
\caption{بلند مجاز، بلند عمل پیرا، تین با آٹھ شناخت کار}
\label{شکل_ترکیبی_تین_با_آٹھ_مجاز_شناخت_کار}
\end{figure}
%
\begin{table}
\caption{بلند مجاز، بلند عمل پیرا، تین با آٹھ شناخت کار}
\label{جدول_ترکیبی_مجاز_تین_با_آٹھ_شناخت_کار}
\centering
\begin{otherlanguage}{english}
\begin{tabular}{CCCC|CCCCCCCC}
\toprule
e&a_2&a_1&a_0&d_7&d_6&d_5&d_4&d_3&d_2&d_1&d_0\\
\midrule
0&x&x&x&0&0&0&0&0&0&0&0\\
1&0&0&0&0&0&0&0&0&0&0&1\\
1&0&0&1&0&0&0&0&0&0&1&0\\
1&0&1&0&0&0&0&0&0&1&0&0\\
1&0&1&1&0&0&0&0&1&0&0&0\\
1&1&0&0&0&0&0&1&0&0&0&0\\
1&1&0&1&0&0&1&0&0&0&0&0\\
1&1&1&0&0&1&0&0&0&0&0&0\\
1&1&1&1&1&0&0&0&0&0&0&0\\
\bottomrule
\end{tabular}
\end{otherlanguage}
\end{table}
\ابتدا{مشق}
شکل \حوالہ{شکل_ترکیبی_تین_با_آٹھ_مجاز_شناخت_کار} میں دایاں جمع گیٹ کا مخارج کیا ہے؟ باقی مخارج بھی شکل سے حاصل کریں۔ کیا یہ جدول \حوالہ{جدول_ترکیبی_تین_با_آٹھ_شناخت_کار} پر پورا اترتے ہیں؟
\انتہا{مشق}

بعض اوقات، ایسے شناخت کار کی ضرورت پیش آتی ہے جس کا چنا گیا مخارج پست ہو۔ایسا شناخت کار،\اصطلاح{پست عمل پیرا}\فرہنگ{عمل پیرا!پست}\حاشیہب{active low}\فرہنگ{active!low} کہلاتا ہے۔ جدول \حوالہ{جدول_ترکیبی_مجاز__پست_عمل_پیرا_تین_با_آٹھ_شناخت_کار} میں ایسا پست عمل پیرا، تین با آٹھ شناخت کار پیش ہے، جو قابو اشارہ \عددی{\overline{\text{مجاز}}} پست ہونے کی صورت میں کام کرتا ہے؛ ہم کہتے ہیں یہ \اصطلاح{پست مجاز}\فرہنگ{مجاز!پست}\حاشیہب{active low}\فرہنگ{active!low} ہے۔ روایتاً، پست عمل پیرا مخارج کو \عددی{\overline{y}} سے ظاہر کیا جاتا ہے، جہاں بِٹ پر \قول{لکیر} اس بات کی یاد دہانی کراتی ہے کہ چنا گیا مخارج پست ہو گا۔ قابو اشارہ پر بھی \قول{لکیر} کھینچی گئی ہے \عددی{(\overline{e})} جو اس حقیقت کو ظاہر کرتی ہے کہ شناخت کار اس صورت کام کرے گا جب قابو اشارہ پست کیا جائے۔ شکل \حوالہ{شکل_ترکیبی_تین_با_آٹھ_پست_عمل_پیرا_شناخت_کار} میں اس کا دور پیش ہے، جو 
شکل \حوالہ{شکل_ترکیبی_تین_با_آٹھ_مجاز_شناخت_کار} میں ضرب گیٹ کی جگہ متمم ضرب گیٹ ڈالنے سے، اور قابو اشارہ کے ساتھ نفی گیٹ منسلک کرنے سے حاصل ہو گا۔
 
\begin{table}
\caption{پست مجاز، پست عمل پیرا، تین با آٹھ شناخت کار}
\label{جدول_ترکیبی_مجاز__پست_عمل_پیرا_تین_با_آٹھ_شناخت_کار}
\centering
\begin{otherlanguage}{english}
\begin{tabular}{CCCC|CCCCCCCC}
\toprule
\overline{e}&a_2&a_1&a_0&\overline{y}_7&\overline{y}_6&\overline{y}_5&\overline{y}_4&\overline{y}_3&
\overline{y}_2&\overline{y}_1&\overline{y}_0\\
\midrule
1&x&x&x&1&1&1&1&1&1&1&1\\
0&0&0&0&1&1&1&1&1&1&1&0\\
0&0&0&1&1&1&1&1&1&1&0&1\\
0&0&1&0&1&1&1&1&1&0&1&1\\
0&0&1&1&1&1&1&1&0&1&1&1\\
0&1&0&0&1&1&1&0&1&1&1&1\\
0&1&0&1&1&1&0&1&1&1&1&1\\
0&1&1&0&1&0&1&1&1&1&1&1\\
0&1&1&1&0&1&1&1&1&1&1&1\\
\bottomrule
\end{tabular}
\end{otherlanguage}
\end{table}
%
\begin{figure}
\centering
\begin{tikzpicture}
\pgfmathsetmacro{\kxs}{1.25}
\pgfmathsetmacro{\kxsa}{1.5}
\pgfmathsetmacro{\kstep}{0.25}
\pgfmathsetmacro{\kbase}{1}
\pgfmathsetmacro{\kys}{\kbase+5*\kstep}
\pgfmathsetmacro{\kpin}{0.5}
\pgfmathsetmacro{\kpina}{\kxsa/2}
\pgfmathsetmacro{\ke}{0.5*(\kbase-2.5*\kstep)}
\draw(0*\kxs,0)node[nand port,scale=1,number inputs=4,rotate=-90,anchor=in 1](u7){};
\draw(1*\kxs,0)node[nand port,scale=1,number inputs=4,rotate=-90,anchor=in 1](u6){};
\draw(2*\kxs,0)node[nand port,scale=1,number inputs=4,rotate=-90,anchor=in 1](u5){};
\draw(3*\kxs,0)node[nand port,scale=1,number inputs=4,rotate=-90,anchor=in 1](u4){};
\draw(4*\kxs,0)node[nand port,scale=1,number inputs=4,rotate=-90,anchor=in 1](u3){};
\draw(5*\kxs,0)node[nand port,scale=1,number inputs=4,rotate=-90,anchor=in 1](u2){};
\draw(6*\kxs,0)node[nand port,scale=1,number inputs=4,rotate=-90,anchor=in 1](u1){};
\draw(7*\kxs,0)node[nand port,scale=1,number inputs=4,rotate=-90,anchor=in 1](u0){};
\draw(0.5,\kys)node[not port,scale=0.8,rotate=-90,anchor=out](u8){};
\draw(0.5-\kxsa,\kys)node[not port,scale=0.8,rotate=-90,anchor=out](u9){};
\draw(0.5-2*\kxsa,\kys)node[not port,scale=0.8,rotate=-90,anchor=out](u10){};
\draw(u8.in)--++(0,\kpin)node[above]{$a_0$};
\draw(u9.in)--++(0,\kpin)node[above]{$a_1$};
\draw(u10.in)--++(0,\kpin)node[above]{$a_2$};
\draw(u0.out)node[below]{$d_0$} (u1.out)node[below]{$d_1$} (u2.out)node[below]{$d_2$} 
(u3.out)node[below]{$d_3$} (u4.out)node[below]{$d_4$} (u5.out)node[below]{$d_5$}
(u6.out)node[below]{$d_6$} (u7.out)node[below]{$d_7$};
\draw(u8.in)--++(-\kpina,0)coordinate(kin8);
\draw(u9.in)--++(-\kpina,0)coordinate(kin9);
\draw(u10.in)--++(-\kpina,0)coordinate(kin10);
\draw(u0.in 1)++(\kpin/2,\kbase+0*\kstep)-|(kin10);
\draw(u0.in 1)++(\kpin/2,\ke)-|coordinate(ken)(u7.in 4) (ken)--++(-0.25*\kpin,0)
node[not port,scale=0.8,anchor=out](u11){} (u11.in)node[above]{$\overline{e}$}--++(-\kpin,0)node[left]{$\overline{\text{مجاز}}$};
\draw(u0.in 1)++(\kpin/2,\kbase+1*\kstep)-|(u10.out);
\draw(u0.in 1)++(\kpin/2,\kbase+2*\kstep)-|(kin9);
\draw(u0.in 1)++(\kpin/2,\kbase+3*\kstep)-|(u9.out);
\draw(u0.in 1)++(\kpin/2,\kbase+4*\kstep)-|(kin8);
\draw(u0.in 1)++(\kpin/2,\kbase+5*\kstep)-|(u8.out);
\draw(u0.in 1)--++(0,\kbase+5*\kstep) (u0.in 2)--++(0,\kbase+3*\kstep) (u0.in 3)--++(0,\kbase+1*\kstep);
\draw(u1.in 1)--++(0,\kbase+4*\kstep) (u1.in 2)--++(0,\kbase+3*\kstep) (u1.in 3)--++(0,\kbase+1*\kstep);
\draw(u2.in 1)--++(0,\kbase+5*\kstep) (u2.in 2)--++(0,\kbase+2*\kstep) (u2.in 3)--++(0,\kbase+1*\kstep);
\draw(u3.in 1)--++(0,\kbase+4*\kstep) (u3.in 2)--++(0,\kbase+2*\kstep) (u3.in 3)--++(0,\kbase+1*\kstep);
\draw(u4.in 1)--++(0,\kbase+5*\kstep) (u4.in 2)--++(0,\kbase+3*\kstep) (u4.in 3)--++(0,\kbase+0*\kstep);
\draw(u5.in 1)--++(0,\kbase+4*\kstep) (u5.in 2)--++(0,\kbase+3*\kstep) (u5.in 3)--++(0,\kbase+0*\kstep);
\draw(u6.in 1)--++(0,\kbase+5*\kstep) (u6.in 2)--++(0,\kbase+2*\kstep) (u6.in 3)--++(0,\kbase+0*\kstep);
\draw(u7.in 1)--++(0,\kbase+4*\kstep) (u7.in 2)--++(0,\kbase+2*\kstep) (u7.in 3)--++(0,\kbase+0*\kstep);
%
\draw(u0.in 4)--++(0,\ke) (u1.in 4)--++(0,\ke) (u2.in 4)--++(0,\ke) (u3.in 4)--++(0,\ke) (u4.in 4)--++(0,\ke) 
(u5.in 4)--++(0,\ke) (u6.in 4)--++(0,\ke);
\end{tikzpicture}
\caption{پست مجاز، پست عمل پیرا، تین با آٹھ شناخت کار}
\label{شکل_ترکیبی_تین_با_آٹھ_پست_عمل_پیرا_شناخت_کار}
\end{figure}

شکل \حوالہ{شکل_ترکیبی_بلند_پست_علامت_شناخت_کار} میں تین با آٹھ شناخت کار کی علامتیں پیش ہیں۔شکل -الف میں بلند مجاز، بلند عمل پیرا، شکل-ب میں پست مجاز، بلند عمل پیرا اور شکل-ج میں پست مجاز، پست عمل پیرا روپ دکھائے گئے ہیں۔ ان علامتوں میں خارجی پنیوں پر گول دائرہ اس بات کی یقین دہانی کراتا ہے کہ منتخب ہونے کی صورت میں یہ بِٹ پست ہو گی۔ اسی طرح قابو بِٹ پر گول دائرہ یاد دہانی کراتا ہے کہ شناخت کار صرف اس صورت مجاز ہو گا جب یہ اشارہ پست ہو۔
\begin{figure}
\centering
\begin{subfigure}{0.30\textwidth}
\centering
\begin{tikzpicture}
\pgfmathsetmacro{\kpin}{0.5}
\pgfmathsetmacro{\kpsep}{0.40}
\pgfmathsetmacro{\kul}{0.50}
\pgfmathsetmacro{\kxdim}{1.5}
\pgfmathsetmacro{\kydim}{2*\kul+7*\kpsep}
\draw[thick](0,0) rectangle ++(\kxdim,\kydim);
\foreach \y/\a in {0/e,4/{a_0},5/{a_1},6/{a_2}}{\draw(0,\kul+\y*\kpsep)node[right]{$\a$}--++(-\kpin,0);}
\foreach \y in {0,1,2,3,4,5,6,7}{\draw(\kxdim,\kul+\y*\kpsep)node[left]{$d_{\y}$}--++(\kpin,0);}
\draw(0,\kul)node[above left]{مجاز};
\end{tikzpicture}
\caption{بلند مجاز،بلند عمل پیرا}
\end{subfigure}\hfill
\begin{subfigure}{0.30\textwidth}
\centering
\begin{tikzpicture}
\pgfmathsetmacro{\kpin}{0.5}
\pgfmathsetmacro{\kpsep}{0.40}
\pgfmathsetmacro{\kul}{0.50}
\pgfmathsetmacro{\kxdim}{1.5}
\pgfmathsetmacro{\kydim}{2*\kul+7*\kpsep}
\draw[thick](0,0) rectangle ++(\kxdim,\kydim);
\foreach \y/\a in {4/{a_0},5/{a_1},6/{a_2}}{\draw(0,\kul+\y*\kpsep)node[right]{$\a$}--++(-\kpin,0);}
\foreach \y/\a in {0/e}{\draw(0,\kul+\y*\kpsep)node[right]{$\overline{\a}$}++(-0.07,0)node[ocirc]{}++(-0.07,0)--++(-\kpin+0.14,0);}
\foreach \y in {0,1,2,3,4,5,6,7}{\draw(\kxdim,\kul+\y*\kpsep)node[left]{$d_{\y}$}--++(\kpin,0);}
\draw(0,\kul)node[above left]{مجاز};
\end{tikzpicture}
\caption{پست مجاز، بلند عمل پیرا}
\end{subfigure}\hfill
\begin{subfigure}{0.30\textwidth}
\centering
\begin{tikzpicture}
\pgfmathsetmacro{\kpin}{0.5}
\pgfmathsetmacro{\kpsep}{0.40}
\pgfmathsetmacro{\kul}{0.50}
\pgfmathsetmacro{\kxdim}{1.5}
\pgfmathsetmacro{\kydim}{2*\kul+7*\kpsep}
\draw[thick](0,0) rectangle ++(\kxdim,\kydim);
\foreach \y/\a in {4/{a_0},5/{a_1},6/{a_2}}{\draw(0,\kul+\y*\kpsep)node[right]{$\a$}--++(-\kpin,0);}
\foreach \y/\a in {0/e}{\draw(0,\kul+\y*\kpsep)node[right]{$\overline{\a}$}++(-0.07,0)node[ocirc]{}++(-0.07,0)--++(-\kpin+0.14,0);}
\foreach \y in {0,1,2,3,4,5,6,7}{\draw(\kxdim,\kul+\y*\kpsep)node[left]{$\overline{y}_{\y}$}++(0.07,0)node[ocirc]{}++(0.07,0)--++(\kpin-0.14,0);}
\draw(0,\kul)node[above left]{$\overline{\text{مجاز}}$};
\end{tikzpicture}
\caption{پست مجاز، پست عمل پیرا}
\end{subfigure}
\caption{تین با آٹھ شناخت کار کی مختلف اقسام کی علامتیں۔}
\label{شکل_ترکیبی_بلند_پست_علامت_شناخت_کار}
\end{figure}


\ابتدا{مشق}
انٹرنیٹ سے \عددی{3\times 8} پست عمل پیرا شناخت کار کے مخلوط دور \عددی{74138} کے معلوماتی صفحات حاصل کریں۔اس مخلوط دور کا \قول{دورانیہ رد عمل} کتنا ہے؟
\انتہا{مشق}

\حصہ{شناخت کار کی مدد سے تفاعل کا حصول}\شناخت{حصہ_ترکیبی_منطق_شناخت_کار_سے_تفاعل_حصول}
ہر تفاعل کی مساوات، ارکان ضرب کے مجموعہ کے روپ میں حاصل کی جا سکتی ہے۔چونکہ شناخت کار تمام ممکنہ ارکان ضرب فراہم کرتا ہے، لہٰذا اس کے ساتھ جمع گیٹ جوڑ کر تفاعل کو عملی جامہ پہنایا جا سکتا ہے۔ی طریقہ کار ایک مثال کی مدد سے سیکھتے ہیں۔


\ابتدا{مثال}\شناخت{مثال_ترکیبی_شناخت_کار_سے_مکمل_جمع_کار}
 مکمل جمع کار کو شناخت کار کی مدد سے ارکان ضرب استعمال کرتے ہوئے حاصل کریں۔

\ترچھا{حل:}\quad
 مکمل جمع کار کی کارکردگی جدول \حوالہ{جدول_ترکیبی_مثال_ترکیبی_شناخت_کار_سے_مکمل_جمع_کار} میں پیش ہے، جہاں بِٹ \عددی{x_0} اور \عددی{y_0} کے ساتھ داخلی حاصل \عددی{c_0} جمع ہو کر \عددی{s_0} اور خارجی حاصل \عددی{c_1} پیدا ہو گا۔
\begin{table}
\caption{مکمل جمع کار کی کارکردگی (برائے مثال \حوالہ{جدول_ترکیبی_مثال_ترکیبی_شناخت_کار_سے_مکمل_جمع_کار})}
\label{جدول_ترکیبی_مثال_ترکیبی_شناخت_کار_سے_مکمل_جمع_کار}
\centering
\begin{otherlanguage}{english}
\begin{tabular}{CCC|CC}
\toprule
x_0&y_0&c_0&c_1&s_0\\
\midrule
0&0&0&0&0\\
0&0&1&0&1\\
0&1&0&0&1\\
0&1&1&1&0\\
1&0&0&0&1\\
1&0&1&1&0\\
1&1&0&1&0\\
1&1&1&1&1\\
\bottomrule
\end{tabular}
\end{otherlanguage}
\end{table}

اس جدول سے درج ذیل مساوات حاصل ہوتی ہیں۔
\begin{gather}
\begin{aligned}\label{مساوات_ترکیبی_مکمل_ارکان_ضرب}
c_1&=\overline{x}_0y_0c_0+x_0\overline{y}_0c_0+x_0y_0\overline{c}_0+x_0y_0c_0\\
s_0&=\overline{x}_0\,\overline{y}_0c_0+\overline{x}_0y_0\overline{c}_0+
x_0\overline{y}_0\,\overline{c}_0+x_0y_0c_0
\end{aligned}
\end{gather}
تین سے آٹھ شناخت کار جدول \حوالہ{جدول_ترکیبی_شناخت_کار_ارکان_ضرب} میں پیش ہے، جہاں خارجی بِٹ کو مطابقتی ارکان ضرب لکھا گیا ہے۔
\begin{table}
\caption{تین با آٹھ شناخت کار ارکان ضرب دیتا ہے (برائے مثال \حوالہ{مثال_ترکیبی_شناخت_کار_سے_مکمل_جمع_کار})}
\label{جدول_ترکیبی_شناخت_کار_ارکان_ضرب}
\centering
\begin{otherlanguage}{english}
\begin{tabular}{CCC|CCCCCCCC}
\toprule
x_0&y_0&c_0&m_7&m_6&m_5&m_4&m_3&m_2&m_1&m_0\\
\midrule
0&0&0&0&0&0&0&0&0&0&1\\
0&0&1&0&0&0&0&0&0&1&0\\
0&1&0&0&0&0&0&0&1&0&0\\
0&1&1&0&0&0&0&1&0&0&0\\
1&0&0&0&0&0&1&0&0&0&0\\
1&0&1&0&0&1&0&0&0&0&0\\
1&1&0&0&1&0&0&0&0&0&0\\
1&1&1&1&0&0&0&0&0&0&0\\
\bottomrule
\end{tabular}
\end{otherlanguage}
\end{table}
یوں درج ذیل ہوں گے۔
\begin{gather}
\begin{aligned}\label{مساوات_ترکیبی_شناخت_ارکان_ضرب}
m_7&=x_0y_0c_0\\
m_6&=x_0y_0\overline{c}_0\\
m_5&=x_0\overline{y}_0c_0\\
m_4&=x_0\overline{y}_0\overline{c}_0\\
m_3&=\overline{x}_0y_0c_0\\
m_2&=\overline{x}_0y_0\overline{c}_0\\
m_1&=\overline{x}_0\overline{y}_0c_0\\
m_0&=\overline{x}_0\overline{y}_0\overline{c}_0
\end{aligned}
\end{gather}
مساوات \حوالہ{مساوات_ترکیبی_شناخت_ارکان_ضرب} کو دیکھتے ہوئے مساوات \حوالہ{مساوات_ترکیبی_مکمل_ارکان_ضرب} درج ذیل لکھی
 جا سکتی ہیں، جن سے مکمل جمع کار کا شکل \حوالہ{شکل_ترکیبی_شناخت_کار_سے_مکمل_ضرب}حاصل ہو گا۔
\begin{gather}
\begin{aligned}
c_1&=m_3+m_5+m_6+m_7=\sum (m_3,m_5,m_6,m_7)\\
s_0&=m_1+m_2+m_4+m_7=\sum (m_1,m_2,m_4,m_7)
\end{aligned}
\end{gather}
%
\begin{figure}
\centering
\begin{tikzpicture}
\pgfmathsetmacro{\kpin}{0.5}
\pgfmathsetmacro{\kstep}{0.25}
\pgfmathsetmacro{\kbase}{1}
\pgfmathsetmacro{\kysep}{\kbase+5*\kstep+2*\kpin}
\pgfmathsetmacro{\kpsep}{0.40}
\pgfmathsetmacro{\kul}{0.50}
\pgfmathsetmacro{\kxdim}{1.5}
\pgfmathsetmacro{\kydim}{2*\kul+7*\kpsep}
\draw[thick](0,0) rectangle ++(\kxdim,\kydim);
\foreach \y/\a/\l in {4/{a_0}/{c_0},5/{a_1}/{y_0},6/{a_2}/{x_0}}{\draw(0,\kul+\y*\kpsep)node[right]{$\a$}--++(-\kpin,0)node[left]{$\l$};}
\draw(\kxdim,\kul+7*\kpsep)node[left]{$m_7$}--++(\kysep,0)coordinate[pos=0.75](kkk)node[or port,scale=1,number inputs=4,anchor=in 1](u1){};
\draw(\kxdim,\kul+6*\kpsep)node[left]{$m_6$}--++(5*\kstep,0)|-(u1.in 2);
\draw(\kxdim,\kul+5*\kpsep)node[left]{$m_5$}--++(6*\kstep,0)|-(u1.in 3);
\draw(\kxdim,\kul+3*\kpsep)node[left]{$m_3$}--++(7*\kstep,0)|-(u1.in 4);
\draw(\kxdim,\kul+1*\kpsep)node[left]{$m_1$}--++(\kysep,0)node[or port,scale=1,number inputs=4,anchor=in 4](u2){};
\draw(\kxdim,\kul+4*\kpsep)node[left]{$m_4$}--++(5*\kstep,0)|-(u2.in 2);
\draw(\kxdim,\kul+2*\kpsep)node[left]{$m_2$}--++(4*\kstep,0)|-(u2.in 3);
\draw(kkk)|-(u2.in 1);
\draw(\kxdim,\kul+0*\kpsep)node[left]{$m_0$};
\draw(u1.out)node[right]{$c_1$};
\draw(u2.out)node[right]{$s_0$};
\end{tikzpicture}
\caption{شناخت کار کی مدد سے مکمل جمع کار کا حصول}
\label{شکل_ترکیبی_شناخت_کار_سے_مکمل_ضرب}
\end{figure}

یہ تمام عمل نہایت آسان بنایا جا سکتا ہے اگر جدول \حوالہ{جدول_ترکیبی_مثال_ترکیبی_شناخت_کار_سے_مکمل_جمع_کار} میں ارکان ضرب کا خانہ بنا یا جائے (جدول \حوالہ{جدول_ترکیبی_مثال_آسان} دیکھیں)۔
\begin{table}
\caption{مکمل جمع کار کے ارکان ضرب (برائے مثال \حوالہ{مثال_ترکیبی_شناخت_کار_سے_مکمل_جمع_کار})}
\label{جدول_ترکیبی_مثال_آسان}
\centering
\begin{otherlanguage}{english}
\begin{tabular}{CCC|CC|C}
\toprule
x_0&y_0&c_0&c_1&s_0&m\\
\midrule
0&0&0&0&0&m_0\\
0&0&1&0&1&m_1\\
0&1&0&0&1&m_2\\
0&1&1&1&0&m_3\\
1&0&0&0&1&m_4\\
1&0&1&1&0&m_5\\
1&1&0&1&0&m_6\\
1&1&1&1&1&m_7\\
\bottomrule
\end{tabular}
\end{otherlanguage}
\end{table}
اس طرز پر جدول لکھ کر تفاعل کی مساوات، ارکان ضرب کے روپ میں حاصل کی جا سکتی ہے۔اس جدول کو دیکھ کر مطلوبہ جواب فوراً لکھا جا سکتا ہے۔
\begin{align*}
c_1&=\sum (m_3,m_5,m_6,m_7)\\
s_0&=\sum (m_1,m_2,m_4,m_7)
\end{align*}
\انتہا{مثال}


\حصہ{داخلی منتخب کار اور خارجی منتخب کار}
 ایسا دور جو اکلوتے مداخل پر مہیا ثنائی مواد کو \عددی{2^n} مخارج میں کسی بھی ایک پر بھیج سکے \اصطلاح{خارجی منتخب کار}\فرہنگ{منتخب کار!خارجی}\حاشیہب{demultiplexer}\فرہنگ{demultiplexer} کہلاتا ہے۔ مطلوبہ مخارج کی نشاندہی \عددی{n} بِٹ پتہ کرتا ہے۔ 

 ایسا دور جو \عددی{2^n} مداخل میں کسی بھی ایک پر مہیا ثنائی مواد کو اکلوتے مخارج پر بھیج سکے \اصطلاح{داخلی منتخب کار}\فرہنگ{منتخب کار!داخلی}\حاشیہب{multiplexer}\فرہنگ{multiplexer} کہلاتا ہے۔ مطلوبہ مداخل کی نشاندہی \عددی{n} بِٹ پتہ کرتا ہے۔

	
\جزوحصہ{خارجی منتخب کار}
شکل \حوالہ{شکل_ترکیبی_ایک_سے_چار_خارجی_منتخب_کار_تصور} میں خارجی منتخب کار کا تصور پیش کیا گیا ہے، جہاں مداخل \عددی{e} پر آمد ثنائی مواد کو، پیچی سوئچ کے ذریعہ، چار مختلف خارجی راستوں بھیجا جا سکتا ہے۔

\begin{figure}
\centering
\begin{tikzpicture}
\pgfmathsetmacro{\kpin}{1}
\pgfmathsetmacro{\kang}{30}
\draw[thick] (0,0)node[left]{$e$}--++(\kpin,0)--++(1.5*\kang:\kpin);
\foreach \a in {0,1,2,3}{\draw[thick](\kpin,0)++(-1.5*\kang+\a*\kang:\kpin)--++(\kpin,0)node[right]{$d_{\a}$};}
\draw[stealth-stealth] ([shift={(-2.25*\kang:\kpin/2)}]\kpin,0) arc (-2.25*\kang:2.25*\kang:\kpin/2);
\draw(-1.5*\kpin,-0.5)node[]{\text{\RL{داخلی مواد}}};
\draw(4.5*\kpin,-0.5)node[]{\text{\RL{خارجی مواد}}};
\end{tikzpicture}
\caption{ایک سے چار خارجی منتخب کار کا تصور۔}
\label{شکل_ترکیبی_ایک_سے_چار_خارجی_منتخب_کار_تصور}
\end{figure}

مجاز و معذور صلاحیت کا شناخت کار بھی یہ کام سرانجام دے سکتا ہے۔یہ دیکھنے کی خاطر جدول \حوالہ{جدول_ترکیبی_مجازومعذار_شناخت_کار} کو یہاں دوبارہ پیش کرتے ہیں۔
\begin{center}
\begin{otherlanguage}{english}
\begin{tabular}{CCC|CCCC}
\toprule
e&a_1&a_0&d_3&d_2&d_1&d_0\\
\midrule
0&0&0&0&0&0&0\\
0&0&1&0&0&0&0\\
0&1&0&0&0&0&0\\
0&1&1&0&0&0&0\\
1&0&0&0&0&0&1\\
1&0&1&0&0&1&0\\
1&1&0&0&1&0&0\\
1&1&1&1&0&0&0\\
\bottomrule
\end{tabular}
\end{otherlanguage}
\end{center}


جدول میں \عددی{a_1a_0} کو دو بِٹ پتہ، \عددی{e} کو داخلی مواد، اور \عددی{d_0} تا \عددی{d_3} کو چار مخارج راستے تصور کریں۔ جدول کی پہلی اور پانچویں صف پر نظر رکھیں، جہاں \عددی{a_1a_0}دو بِٹ پتہ \عددی{00} ہے، جو مخارج \عددی{d_0} منتخب کرے گا۔ پہلی صف میں داخلی مواد \عددی{0} جبکہ پانچویں صف میں \عددی{1} ہے۔ مخارج \عددی{d_0} کی مطابقتی قیمتیں یہی ہیں۔پہلی صف میں \عددی{d_0} کی قیمت \عددی{0} جبکہ پانچویں صف میں اس کی قیمت \عددی{1} ہے۔ غیر منتخب مخارج پست رہیں گے۔

باقی تین پتے \عددی{01}، \عددی{10}، اور \عددی{11} بالترتیب \عددی{d_1}، \عددی{d_2}، اور \عددی{d_3} منتخب کرتے ہیں۔ تسلی کر لیں کہ منتخب مخارج پر وہی مواد ہے جو مداخل \عددی{e} پر ہے۔

اس جدول میں صفوں کی ترتیب نو کر کے شکل \حوالہ{جدول_ترکیبی_ترتیب_نو_خارجی_منتخب_کار} میں پیش جدول کی صورت میں لکھا جا سکتا ہے، جو اس کی کارکردگی بطور خارجی منتخب کار واضح کرتا ہے۔اس شکل میں \عددی{(1\times 4)} منتخب کار کی علامت بھی پیش ہے۔
\begin{figure}
\centering
\begin{subfigure}{0.6\textwidth}
\centering
\begin{otherlanguage}{english}
\begin{tabular}{C|CC|CCCC}
\toprule
e&a_1&a_0&d_3&d_2&d_1&d_0\\
\midrule
0&0&0&0&0&0&0\\
1&0&0&0&0&0&1\\
\midrule
0&0&1&0&0&0&0\\
1&0&1&0&0&1&0\\
\midrule
0&1&0&0&0&0&0\\
1&1&0&0&1&0&0\\
\midrule
0&1&1&0&0&0&0\\
1&1&1&1&0&0&0\\
\bottomrule
\end{tabular}
\end{otherlanguage}
%\end{table}
\end{subfigure}\hfill
\begin{subfigure}{0.4\textwidth}
\centering
\begin{tikzpicture}
\pgfmathsetmacro{\kpin}{0.5}
\pgfmathsetmacro{\kpsep}{0.40}
\pgfmathsetmacro{\kul}{0.50}
\pgfmathsetmacro{\kxdim}{2*\kul+1*\kpsep}
\pgfmathsetmacro{\kydim}{2.5*\kul+3*\kpsep}
\draw[thick](0,0) rectangle ++(\kxdim,\kydim);
\foreach \y/\a in {2/{x}}{\draw(0,1.5*\kul+\y*\kpsep)node[right]{$e$}--++(-\kpin,0);}
\foreach \y in {0,1,2,3}{\draw(\kxdim,1.5*\kul+\y*\kpsep)node[left]{$d_{\y}$}--++(\kpin,0);}
\draw(\kxdim-\kul-0*\kpsep,0)node[above]{$a_0$}--++(0,-\kpin);
\draw(\kxdim-\kul-1*\kpsep,0)node[above]{$a_1$}--++(0,-\kpin);
\end{tikzpicture}
\end{subfigure}
\caption{ایک سے چار \عددی{(1\times 4)} خارجی منتخب کار}
\label{جدول_ترکیبی_ترتیب_نو_خارجی_منتخب_کار}
\end{figure}



\جزوحصہ{داخلی منتخب کار}
شکل \حوالہ{شکل_ترکیبی_چار_سے_ایک_داخلی_منتخب_کار_تصور} میں داخلی منتخب کار کا تصور پیش کیا گیا ہے، جہاں پیچی سوئچ کے ذریعہ \عددی{d_0} تا \عددی{d_3} میں سے ایک کا مواد مخارج منتقل کیا جا سکتا ہے۔
\begin{figure}
\centering
\begin{tikzpicture}
\pgfmathsetmacro{\kpin}{1}
\pgfmathsetmacro{\kang}{30}
\draw[thick] (0,0)node[right]{$D$}--++(-\kpin,0)--++(180-1.5*\kang:\kpin);
\foreach \a in {0,1,2,3}{\draw[thick](-\kpin,0)++(180+1.5*\kang-\a*\kang:\kpin)--++(-\kpin,0)node[left]{$d_{\a}$};}
\draw[stealth-stealth] ([shift={(180-2.25*\kang:\kpin/2)}]-\kpin,0) arc (180-2.25*\kang:180+2.25*\kang:\kpin/2);
\draw(-4.5*\kpin,-0.5)node[]{\text{\RL{داخلی مواد}}};
\draw(1.5*\kpin,-0.5)node[]{\text{\RL{خارجی مواد}}};
\end{tikzpicture}
\caption{چار سے ایک داخلی منتخب کار کا تصور۔}
\label{شکل_ترکیبی_چار_سے_ایک_داخلی_منتخب_کار_تصور}
\end{figure}

داخلی منتخب کار کو شناخت کار کی مدد سے شکل \حوالہ{شکل_ترکیبی_چار_سے_ایک_داخلی_منتخب_کار} میں حاصل کیا گیا ہے؛ شکل-ب میں اس کی علامت پیش ہے۔ یہاں مجاز و معذور صلاحیت کا شناخت کار استعمال کرکے مجاز و معذور صلاحیت کا داخلی منتخب کار حاصل کیا گیا۔ ایسا شناخت کار جس میں قابو اشارہ نہ ہو، استعمال کرتے ہوئے حاصل داخلی منتخب کار میں بھی مجاز و معذور قابو اشارہ نہیں ہو گا۔

مجاز کردہ شناخت کار \عددی{00} پتہ کی صورت میں \عددی{y_0} بلند کرے گا، جبکہ \عددی{y_1}، \عددی{y_2} اور \عددی{y_3} پست رہیں گے۔یوں دائیں تین ضرب گیٹ پست رہیں گے، جبکہ بایاں گیٹ \عددی{d_0} خارج کرے گا۔ یوں جمع گیٹ بھی \عددی{d_0} خارج کرے گا۔قابو اشارہ \عددی{e} پست کرنے سے داخلی شناخت کار معذور ہو گا اور \عددی{0} خارج کرے گا۔

تسلی کر لیں کہ مجاز حال میں ، پتہ کے دو بِٹ \عددی{a_0} اور \عددی{a_1}، چار مداخل \عددی{d_0} تا \عددی{d_1}، میں سے ایک کو منتخب کر کے خارج کرتا ہے۔
\begin{figure}
\centering
\begin{subfigure}{0.65\textwidth}
\centering
\begin{tikzpicture}
\pgfmathsetmacro{\kpin}{0.50}
\pgfmathsetmacro{\kbase}{2.5}
\pgfmathsetmacro{\kxsep}{1.25}
\pgfmathsetmacro{\kysep}{0}
\pgfmathsetmacro{\kpsep}{0.40}
\pgfmathsetmacro{\kul}{0.50}
\pgfmathsetmacro{\kxdim}{1.5}
\pgfmathsetmacro{\kydim}{2*\kul+3*\kpsep}
\pgfmathsetmacro{\kpina}{\kydim+\kysep}
\pgfmathsetmacro{\kor}{1.75}
\draw[thick](0,0) rectangle ++(\kxdim,\kydim);
\foreach \y/\a in {0/{e},2/{a_0},3/{a_1}}{\draw(0,\kul+\y*\kpsep)node[right]{$\a$}--++(-\kpin,0)node[left]{$\a$};}
\foreach \y in {0,1,2,3}{\draw(\kxdim,\kul+\y*\kpsep)node[left]{$y_{\y}$};}
\draw(\kbase+0*\kxsep,-\kysep)node[and port,scale=1,number inputs=2,rotate=-90,anchor=in 2](u0){};
\draw(\kbase+1*\kxsep,-\kysep)node[and port,scale=1,number inputs=2,rotate=-90,anchor=in 2](u1){};
\draw(\kbase+2*\kxsep,-\kysep)node[and port,scale=1,number inputs=2,rotate=-90,anchor=in 2](u2){};
\draw(\kbase+3*\kxsep,-\kysep)node[and port,scale=1,number inputs=2,rotate=-90,anchor=in 2](u3){};
\draw($(u1.out)!0.5!(u2.out)$)++(0,-\kor)node[or port,scale=1,number inputs=4,rotate=-90](u4){};
\draw(\kxdim,\kul+0*\kpsep)-|(u0.in 2);
\draw(\kxdim,\kul+1*\kpsep)-|(u1.in 2);
\draw(\kxdim,\kul+2*\kpsep)-|(u2.in 2);
\draw(\kxdim,\kul+3*\kpsep)-|(u3.in 2);
\draw(u0.out)|-(u4.in 4);
\draw(u1.out)-|(u4.in 3);
\draw(u2.out)-|(u4.in 2);
\draw(u3.out)|-(u4.in 1);
\draw(u4.out)node[below]{$D$};
\draw(u0.in 1)--++(0,\kpina)node[above]{$d_0$};
\draw(u1.in 1)--++(0,\kpina)node[above]{$d_1$};
\draw(u2.in 1)--++(0,\kpina)node[above]{$d_2$};
\draw(u3.in 1)--++(0,\kpina)node[above]{$d_3$};
\end{tikzpicture}
\caption{}
\end{subfigure}\hfill
\begin{subfigure}{0.35\textwidth}
\centering
\begin{tikzpicture}
\pgfmathsetmacro{\kpin}{0.50}
\pgfmathsetmacro{\kbase}{2.5}
\pgfmathsetmacro{\kxsep}{1.25}
\pgfmathsetmacro{\kysep}{0}
\pgfmathsetmacro{\kpsep}{0.40}
\pgfmathsetmacro{\kul}{0.50}
\pgfmathsetmacro{\kxdim}{2*\kul+1*\kpsep}
\pgfmathsetmacro{\kydim}{2.5*\kul+3*\kpsep}
\pgfmathsetmacro{\kpina}{\kydim+\kysep}
\pgfmathsetmacro{\kor}{1.75}
\draw[thick](0,0) rectangle ++(\kxdim,\kydim);
\foreach \y in {0,1,2,3}{\draw(0,1.5*\kul+\y*\kpsep)node[right]{$d_{\y}$}--++(-\kpin,0);}
\foreach \y in {2}{\draw(\kxdim,\kul+\y*\kpsep)node[left]{$D$}--++(\kpin,0);}
\draw(\kxdim-\kul-0*\kpsep,0)node[above]{$a_0$}--++(0,-\kpin);
\draw(\kxdim-\kul-1*\kpsep,0)node[above]{$a_1$}--++(0,-\kpin);
\draw(\kxdim/2,\kydim)node[below]{$e$}--++(0,\kpin);
\end{tikzpicture}
\caption{}
\end{subfigure}
\caption{چار سے ایک \عددی{(4\times 1)} داخلی منتخب کار۔}
\label{شکل_ترکیبی_چار_سے_ایک_داخلی_منتخب_کار}
\end{figure}

\ابتدا{مشق}
انٹرنیٹ سے \عددی{74153} کے معلوماتی صفحات حاصل کریں۔ یہ مخلوط دور کیا کام سرانجام دیتا ہے؟
\انتہا{مشق}


\جزوحصہ{داخلی منتخب کار سے تفاعل کا حصول}
 شناخت کار کے ساتھ جمع گیٹ جوڑ کر مجموعہ ارکان ضرب کے روپ میں تفاعل کا حصول آپ دیکھ چکے۔ داخلی منتخب کار میں شناخت کار اور جمع گیٹ دونوں موجود ہیں (شکل \حوالہ{شکل_ترکیبی_چار_سے_ایک_داخلی_منتخب_کار} دیکھیں)۔یوں \عددی{n} پتہ بِٹ کا \عددی{2^n\times 1} داخلی منتخب کار سے \عددی{n} آزاد متغیر تفاعل حاصل کیا جا سکتا ہے۔اس عمل کو ایک مثال کی مدد سے سمجھتے ہیں۔
 
\ابتدا{مثال}\شناخت{مثال_ترکیبی_داخلی_منتخب_کار_سے_تفاعل_الف}
 درج ذیل تفاعل \عددی{8\times 1} داخلی منتخب کار سے حاصل کریں۔
 \begin{align*}
 F(x,y,z)=\sum (m_3,m_4,m_6,m_7)
 \end{align*}

\ترچھا{حل:}\quad
اس تفاعل کا جدول شکل \حوالہ{شکل_ترکیبی_داخلی_منتخب_کار_سے_تفاعل_الف} میں پیش ہے۔
\begin{figure}
\centering
\begin{subfigure}{0.40\textwidth}
\centering
\begin{otherlanguage}{english}
\begin{tabular}{CCC|CC}
\toprule
x&y&z&F&m\\
\midrule
0&0&0&0&m_0\\
0&0&1&0&m_1\\
0&1&0&0&m_2\\
0&1&1&1&m_3\\
1&0&0&1&m_4\\
1&0&1&0&m_5\\
1&1&0&1&m_6\\
1&1&1&1&m_7\\
\bottomrule
\end{tabular}
\end{otherlanguage}
\end{subfigure}\hfill
\begin{subfigure}{0.60\textwidth}
\centering
\begin{tikzpicture}
\pgfmathsetmacro{\kpin}{0.50}
\pgfmathsetmacro{\kpina}{0.75}
\pgfmathsetmacro{\kbase}{2.5}
\pgfmathsetmacro{\kxsep}{1.25}
\pgfmathsetmacro{\kysep}{0}
\pgfmathsetmacro{\kpsep}{0.40}
\pgfmathsetmacro{\kul}{0.50}
\pgfmathsetmacro{\kxdim}{2*\kul+2*\kpsep}
\pgfmathsetmacro{\kydim}{2.5*\kul+7*\kpsep}
\pgfmathsetmacro{\kor}{1.75}
\draw[thick](0,0) rectangle ++(\kxdim,\kydim);
\foreach \y in {0,1,2,5}{\draw(0,1.5*\kul+\y*\kpsep)node[right]{$d_{\y}$}--++(-\kpin,0);}
\foreach \y in {3,4,6,7}{\draw(0,1.5*\kul+\y*\kpsep)node[right]{$d_{\y}$}--++(-\kpina,0);}
\foreach \y in {6}{\draw(\kxdim,1.5*\kul+\y*\kpsep)node[left]{$D$}--++(\kpin,0)node[right]{$F=\sum(m_3,m_4,m_6,m_7)$};}
\draw(\kxdim-\kul-0*\kpsep,0)node[above]{$a_0$}--++(0,-\kpin)node[below]{$z$};
\draw(\kxdim-\kul-1*\kpsep,0)node[above]{$a_1$}--++(0,-\kpin)node[below]{$y$};
\draw(\kxdim-\kul-2*\kpsep,0)node[above]{$a_2$}--++(0,-\kpin)node[below]{$x$};
\draw(\kxdim/2,\kydim)node[below]{$e$}--++(0,\kpin)node[left]{$1$};
\draw(-\kpin,1.5*\kul+5*\kpsep)--(-\kpin,1.5*\kul+0*\kpsep)--++(-\kpina,0)node[left]{$0$};
\draw(-\kpina,1.5*\kul+3*\kpsep)--(-\kpina,1.5*\kul+7*\kpsep)--++(-\kpin,0)node[left]{$1$};
\draw(\kxdim-1.5*\kul,1.5*\kul+3.5*\kpsep)node[rotate=90]{\RL{داخلی منتخب کار}};
\end{tikzpicture}
\end{subfigure}
\caption{داخلی منتخب کار سے تفاعل کا حصول (برائے مثال \حوالہ{مثال_ترکیبی_داخلی_منتخب_کار_سے_تفاعل_الف})}
\label{شکل_ترکیبی_داخلی_منتخب_کار_سے_تفاعل_الف}
\end{figure}
 تفاعل کے تین آزاد متغیرات \عددی{xyz} کو \عددی{8\times 1} داخلی منتخب کار کے تین پتہ بِٹ تصور کر کے، داخلی منتخب کار کے آٹھ مداخل \عددی{d_0} تا \عددی{d_7} میں سے \عددی{d_3}، \عددی{d_4}، \عددی{d_6}، اور \عددی{d_7} کو بلند، جبکہ باقی کو پست رکھ کر تفاعل حاصل ہو گا، جو شکل \حوالہ{شکل_ترکیبی_داخلی_منتخب_کار_سے_تفاعل_الف} میں پیش ہے۔داخلی منتخب کار کو مجاز \عددی{(e=1)} رکھا گیا ہے۔

یوں پتہ \عددی{000}، \عددی{001}، \عددی{010}،اور \عددی{101} کی صورت میں داخلی منتخب کار بالترتیب \عددی{d_0}، \عددی{d_1}، \عددی{d_2}، اور \عددی{d_5} پر فراہم مواد خارج کرے گا؛ ان تمام کو پست رکھ کر درکار تفاعل کی پست صورت حاصل ہو گی۔اسی طرح پتہ \عددی{011}، \عددی{100}، \عددی{110}، اور \عددی{11} کی صورت میں بالترتیب \عددی{d_3}، \عددی{d_4}، \عددی{d_6}، اور \عددی{d_7} کے مواد خارج ہوں گے؛انہیں بلند رکھ کر تفاعل کی بلند صورت حاصل ہو گی۔کسی ایک لمحہ پر پتہ صرف ایک قیمت رکھ سکتا ہے۔
\انتہا{مثال}

 \عددی{n} آزاد متغیر تفاعل ، \عددی{(n-1)}پتہ بِٹ کے داخلی منتخب کار سے بھی حاصل کیا جا سکتا ہے۔یہاں کوئی بھی \عددی{(n-1)} متغیرات بطور داخلی منتخب کار کے پتہ استعمال ہوں گے، جبکہ ایک متغیر بطور مداخل استعمال ہو گا۔ ایک مثال کی مدد سے ایسا کرنا سیکھتے ہیں۔


\ابتدا{مثال}\شناخت{مثال_ترکیبی_داخلی_منتخب_کار_سے_تفاعل_ب}
 درج بالا مثال میں دیا گیا تفاعل \عددی{ F(x,y,z)=\sum (m_3,m_4,m_6,m_7)} دو پتہ بِٹ کے \عددی{4\times 1} داخلی منتخب کار سے حاصل کریں۔

\ترچھا{حل:}\quad
شکل \حوالہ{شکل_ترکیبی_داخلی_منتخب_کار_سے_تفاعل_ب} میں تفاعل کا جدول ایک نئے انداز میں لکھا گیا ہے۔
\begin{figure}
\centering
\begin{subfigure}{0.40\textwidth}
\centering
\begin{otherlanguage}{english}
\begin{tabular}{CC|CCC}
\toprule
x&y&z&F&\\
\midrule
0&0&0&0&\multirow{2}{*}{$F=0$}\\
0&0&1&0&\\
\midrule
0&1&0&0&\multirow{2}{*}{$F=z$}\\
0&1&1&1&\\
\midrule
1&0&0&1&\multirow{2}{*}{$F=\overline{z}$}\\
1&0&1&0&\\
\midrule
1&1&0&1&\multirow{2}{*}{$F=1$}\\
1&1&1&1&\\
\bottomrule
\end{tabular}
\end{otherlanguage}
\end{subfigure}\hfill
\begin{subfigure}{0.60\textwidth}
\centering
\begin{tikzpicture}
\pgfmathsetmacro{\kpin}{0.50}
\pgfmathsetmacro{\kpina}{2}
\pgfmathsetmacro{\kbase}{2.5}
\pgfmathsetmacro{\kxsep}{1.25}
\pgfmathsetmacro{\kysep}{0}
\pgfmathsetmacro{\kpsep}{0.40}
%\pgfmathsetmacro{\kpsepa}{1.2}
\pgfmathsetmacro{\kul}{0.50}
\pgfmathsetmacro{\kxdim}{2*\kul+2*\kpsep}
\pgfmathsetmacro{\kydim}{2.5*\kul+5*\kpsep}
\pgfmathsetmacro{\kor}{1.75}
\draw[thick](0,0) rectangle ++(\kxdim,\kydim);
\foreach \y/\d in {0/0,1/1,3/2,5/3}{\draw(0,1.5*\kul+\y*\kpsep)node[right]{$d_{\d}$};}
%\foreach \y in {3,4,6,7}{\draw(0,1.5*\kul+\y*\kpsep)node[right]{$d_{\y}$}--++(-\kpina,0);}
\foreach \y in {2}{\draw(\kxdim,1.5*\kul+\y*\kpsep)node[left]{$D$}--++(\kpin,0)node[right]{$F(x,y,z)$};}
\draw(\kxdim-\kul-0*\kpsep,0)node[above]{$a_0$}--++(0,-\kpin)node[below]{$y$};
\draw(\kxdim-\kul-1*\kpsep,0)node[above]{$a_1$}--++(0,-\kpin)node[below]{$x$};
%\draw(\kxdim-\kul-2*\kpsep,0)node[above]{$a_2$}--++(0,-\kpin)node[below]{$x$};
%\draw(\kxdim/2,\kydim)node[below]{$e$}--++(0,\kpin)node[left]{$1$};
\draw(\kxdim-1.5*\kul,1.5*\kul+3*\kpsep)node[rotate=90]{\RL{داخلی منتخب کار}};
\draw(0,1.5*\kul+3*\kpsep)node[not port,scale=1,anchor=out](u1){};
\draw(u1.in)--++(-\kpin,0)node[left]{$z$}coordinate(aaa)++(0,\kpin)coordinate(bbb);
\draw(0,1.5*\kul+0*\kpsep)--($(aaa)!(0,1.5*\kul+0*\kpsep)!(bbb)$)node[left]{$0$};
\draw(0,1.5*\kul+5*\kpsep)--($(aaa)!(0,1.5*\kul+5*\kpsep)!(bbb)$)node[left]{$1$};
\draw(0,1.5*\kul+1*\kpsep)-|(u1.in);
\end{tikzpicture}
\end{subfigure}
\caption{داخلی منتخب کار سے تفاعل کا حصول (برائے مثال \حوالہ{مثال_ترکیبی_داخلی_منتخب_کار_سے_تفاعل_ب})}
\label{شکل_ترکیبی_داخلی_منتخب_کار_سے_تفاعل_ب}
\end{figure}
 آزاد متغیرات \عددی{xy} کے دائیں کھڑی لکیر کھینچ گئی، اور \عددی{xy} کی قیمت کے مطابق جدول کے چار حصے کیے گئے ۔پہلے (بالائی) حصہ میں (جہاں \عددی{xy=00}ہے) تفاعل \عددی{F} کی قیمت بدستور \عددی{0} ہے، لہٰذا اس حصہ کے اضافی قطار میں \عددی{F=0} لکھا گیا۔ دوسرے حصے ( \عددی{xy=01}) کی دونوں صفوں میں \عددی{z} کی قیمت اور تفاعل \عددی{F} کی قیمت برابر ہیں، لہٰذا یہاں \عددی{F=z} لکھا گیا۔ تیسرے حصے (\عددی{xy=10}) میں \عددی{z} اور \عددی{F} کی قیمتیں آپ میں متمم ہیں، لہٰذا یہاں \عددی{F=\overline{z}} لکھا گیا ہے۔آخری حصہ ( \عددی{xy=11}) میں تفاعل بدستور بلند ہے، لہٰذا یہاں \عددی{F=1} لکھا گیا۔


شکل \حوالہ{شکل_ترکیبی_داخلی_منتخب_کار_سے_تفاعل_ب} میں اس جدول سے حاصل دور دکھایا گیا ہے، جہاں (مجاز و معذور صلاحیت نہ رکھنے والا) \عددی{4\times 1} داخلی منتخب کار استعمال کیا گیا۔پتہ \عددی{xy=00} کی صورت میں داخلی منتخب کار مداخل \عددی{d_0} کا مواد خارج کرے گا۔یوں \عددی{d_0} پر \عددی{0} مہیا کر کے اس صورت میں تفاعل کی درست قیمت حاصل کی گئی۔اسی طرح \عددی{xy=01} کی صورت میں \عددی{d_1} کا مواد خارج کیا جائے گا، لہٰذا یہاں متغیر \عددی{z} فراہم کر کے تفاعل کی درست قیمت حاصل کی گئی۔اسی طرح \عددی{xy=10} کی صورت میں \عددی{d_2} کا مواد مخارج کیا جائے گا، لہٰذا یہاں \عددی{\overline{z}} فراہم کر کے تفاعل کی درست قیمت حاصل کی گئی، اور آخر میں \عددی{xy=11} کی صورت میں تفاعل بدستور بلند رہتا ہے، لہٰذا \عددی{d_3} پر \عددی{1} مہیا کیا گیا۔
\انتہا{مثال}


\حصہ{متوازی ثنائی ضرب کار}
حسابی اعمال میں ضرب کا کردار کلیدی ہے۔ثنائی اعداد کی ضرب کا عمل بالکل اعشاری اعداد کی ضرب کی طرح ہے۔دو بِٹ ثنائی اعداد \عددی{a} اور \عددی{b} کی ضرب درج ذیل ہے، جہاں ان ثنائی اعداد کو \عددی{a_1a_0} اور \عددی{b_1b_0} لکھا گیا ہے ۔

\begin{align*}
\begin{array}{cccc}
&& b_1&b_0\\
&& a_1&a_0\\
\cline{3-4}
&& a_0b_1&a_0b_0\\
&a_1b_1&a_1b_0&\\
\cline{1-4}
p_3&p_2&p_1&p_0
\end{array}
\end{align*}
یہاں درج ذیل ہوں گے، جنہیں ثنائی جمع کار کی مساوات \حوالہ{مساوات_ترکیبی_نصف_جمع_کار} کی مدد سے حاصل کیا گیا، اور جن سے شکل \حوالہ{شکل_ترکیبی_ثنائی_متوازی_دو_ضرب_کار} میں پیش، دو بِٹ متوازی ثنائی ضرب کار حاصل ہو گا۔
\begin{align*}
p_0&=a_0b_0\\
p_1&=(a_1b_0)\oplus(a_0b_1)\\
p_2&=(a_1b_1)\oplus(a_1b_0a_0b_1)\\
p_3&=a_1b_1a_1b_0a_0b_1=a_1a_0b_1b_0
\end{align*}


\begin{figure}
\centering
\begin{tikzpicture}
\pgfmathsetmacro{\kxstep}{2}
\pgfmathsetmacro{\kystep}{2}
\pgfmathsetmacro{\kystepa}{2.5}
\pgfmathsetmacro{\ksh}{0.4}
\pgfmathsetmacro{\ksha}{1.25}
\draw(0,3.5*\kystep)node[and port,scale=1,number inputs=2,anchor=out](u0){u0} (u0.out)node[right]{$p_0$};
\draw(0,2.5*\kystep)node[xor port,scale=1,number inputs=2,anchor=out](u1){u1}(u1.out)node[right]{$p_1$};
\draw(0,1*\kystep)node[xor port,scale=1,number inputs=2,anchor=out](u2){u2}(u2.out)node[right]{$p_2$};
\draw(0,0*\kystep)node[and port,scale=1,number inputs=4,anchor=out](u3){u3}(u3.out)node[right]{$p_3$};
\draw(u1.in 1)--++(0,\ksh)--++(-\kxstep,0)node[and port,scale=1,number inputs=2,anchor=out](u4){u4}
 (u4.in 1)node[left]{$a_1$} (u4.in 2)node[left]{$b_0$};
\draw(u1.in 2)--++(0,-\ksh)--++(-\kxstep,0)node[and port,scale=1,number inputs=2,anchor=out](u5){u5}
 (u5.in 1)node[left]{$a_0$} (u5.in 2)node[left]{$b_1$};
\draw(u0.in 1)--($(u4.in 1)!(u0.in 1)!(u4.in 2)$)node[left]{$a_0$};
\draw(u0.in 2)--($(u4.in 1)!(u0.in 2)!(u4.in 2)$)node[left]{$b_0$};
\draw(u2.in 1)--++(0,\ksh)--++(-\kxstep,0)node[and port,scale=1,number inputs=2,anchor=out](u6){u6}
 (u6.in 1)node[left]{$a_1$} (u6.in 2)node[left]{$b_1$};
\draw(u2.in 2)--++(0,-\ksh)--++(-\kxstep,0)node[and port,scale=1,number inputs=4,anchor=out](u7){u7}
 (u7.in 1)node[left]{$a_0$} (u7.in 2)node[left]{$a_1$} (u7.in 3)node[left]{$b_0$} (u7.in 4)node[left]{$b_1$};
 \draw(u3.in 1)--($(u4.in 1)!(u3.in 1)!(u4.in 2)$)node[left]{$a_0$}(u3.in 2)--($(u4.in 1)!(u3.in 2)!(u4.in 2)$)node[left]{$a_1$}(u3.in 3)--($(u4.in 1)!(u3.in 3)!(u4.in 2)$)node[left]{$b_0$} (u3.in 4)--($(u4.in 1)!(u3.in 4)!(u4.in 2)$)node[left]{$b_1$};
\end{tikzpicture}
\caption{دو بِٹ ثنائی متوازی ضرب کار}
\label{شکل_ترکیبی_ثنائی_متوازی_دو_ضرب_کار}
\end{figure}


 اگرچہ زیادہ بِٹ ضرب کار اس طریقہ کار سے تشکیل دیے جا سکتے ہیں؛ بد قسمتی سے، اعداد کے بِٹ کی تعداد بڑھانے سے ضرب کار میں درکار گیٹوں کی تعداد بہت تیزی سے بڑھتی ہے (محض آٹھ یا سولہ بِٹ ضرب کار میں بھی مستعمل گیٹوں کی تعداد بہت زیادہ ہو گی)، لہٰذا ایسا کرنا مہنگا ثابت ہو گا۔ عموماً زیادہ بِٹ کے ضرب کار مکمل جمع کار کی مدد سے حاصل کیے جاتے ہیں۔ اس طریقہ کو تین بِٹ ثنائی اعداد کی ضرب کو مثال بنا کر سیکھتے ہیں۔

تین بِٹ اعداد \عددی{b_2b_1b_0} اور \عددی{a_2a_1a_0} کی ضرب درج ذیل ہے، جس سے شکل \حوالہ{شکل_ترکیبی_تین_بٹ_ضرب_کار} میں پیش تین بِٹ ثنائی ضرب کار حاصل ہو گا۔ اس طریقہ کار سے با آسانی زیادہ بِٹ کے ثنائی ضرب کار بنائے جا سکتے ہیں۔
\begin{gather}
\begin{aligned}\label{مساوات_ترکیبی_تین_بٹ_ضرب_کار}
\begin{array}{cccccc}
&&&b_2& b_1&b_0\\
&&&a_2& a_1&a_0\\
\cline{4-6}
&&&a_0b_2& a_0b_1&a_0b_0\\
&&a_1b_2&a_1b_1&a_1b_0&\\
&a_2b_2&a_2b_1&a_2b_0&\\
\cline{1-6}
p_5&p_4&p_3&p_2&p_1&p_0
\end{array}
\end{aligned}
\end{gather}
%
\begin{figure}
\centering
\begin{tikzpicture}
\pgfmathsetmacro{\kxa}{3.5}
\pgfmathsetmacro{\kya}{1.25}
\draw(0*\kxa,2*\kya)node[and port,scale=1,number inputs=2](u2){u2};
\draw(0*\kxa,1*\kya)node[and port,scale=1,number inputs=2](u1){u1};
\draw(0*\kxa,0*\kya)node[and port,scale=1,number inputs=2](u0){u0};
\draw(1*\kxa,2*\kya)node[and port,scale=1,number inputs=2](u5){u5};
\draw(1*\kxa,1*\kya)node[and port,scale=1,number inputs=2](u4){u4};
\draw(1*\kxa,0*\kya)node[and port,scale=1,number inputs=2](u3){u3};
\draw(2*\kxa,2*\kya)node[and port,scale=1,number inputs=2](u8){u8};
\draw(2*\kxa,1*\kya)node[and port,scale=1,number inputs=2](u7){u7};
\draw(2*\kxa,0*\kya)node[and port,scale=1,number inputs=2](u6){u6};
\draw(u0.in 1)node[left]{$a_0$} (u0.in 2)node[left]{$b_2$} (u0.out)node[right]{$A_2$};
\draw(u1.in 1)node[left]{$a_0$} (u1.in 2)node[left]{$b_1$} (u1.out)node[right]{$A_1$};
\draw(u2.in 1)node[left]{$a_0$} (u2.in 2)node[left]{$b_0$} (u2.out)node[right]{$A_0$};
\draw(u3.in 1)node[left]{$a_1$} (u3.in 2)node[left]{$b_2$} (u3.out)node[right]{$B_2$};
\draw(u4.in 1)node[left]{$a_1$} (u4.in 2)node[left]{$b_1$} (u4.out)node[right]{$B_1$};
\draw(u5.in 1)node[left]{$a_1$} (u5.in 2)node[left]{$b_0$} (u5.out)node[right]{$B_0$};
\draw(u6.in 1)node[left]{$a_2$} (u6.in 2)node[left]{$b_2$} (u6.out)node[right]{$C_2$};
\draw(u7.in 1)node[left]{$a_2$} (u7.in 2)node[left]{$b_1$} (u7.out)node[right]{$C_1$};
\draw(u8.in 1)node[left]{$a_2$} (u8.in 2)node[left]{$b_0$} (u8.out)node[right]{$C_0$};
\end{tikzpicture}\vspace{1cm}
\begin{tikzpicture}
\pgfmathsetmacro{\kpin}{0.50}
\pgfmathsetmacro{\kul}{0.5}
\pgfmathsetmacro{\kxbase}{1.5}
\pgfmathsetmacro{\kysep}{2.25}
\pgfmathsetmacro{\kxdim}{1.5}
\pgfmathsetmacro{\kydim}{1}
\pgfmathsetmacro{\kxsep}{\kxbase+\kxdim-2*\kul}
\draw[thick](0,0)node[below right]{$u_9$} rectangle ++(\kxdim,\kydim);
\draw[thick](\kxsep,0)node[below right]{$u_{10}$} rectangle ++(\kxdim,\kydim);
\draw[thick](2*\kxsep,0)node[below right]{$u_{11}$} rectangle ++(\kxdim,\kydim);
\draw[thick](\kxbase,\kysep)node[below right]{$u_{12}$} rectangle ++(\kxdim,\kydim);
\draw[thick](\kxbase+\kxsep,\kysep)node[below right]{$u_{13}$} rectangle ++(\kxdim,\kydim);
\draw[thick](\kxbase+2*\kxsep,\kysep)node[below right]{$u_{14}$} rectangle ++(\kxdim,\kydim);
\draw(\kul,\kydim)node[below]{$z$}--++(0,\kpin)node[above]{$C_2$};
\draw(\kxsep+\kxdim-\kul,\kydim)node[below]{$y$}--++(0,\kpin)node[above]{$C_1$};
\draw(2*\kxsep+\kxdim-\kul,\kydim)node[below]{$y$}--++(0,\kpin)node[above]{$C_0$};
\draw(0,\kydim/2)node[right]{$c_{\text{خ}}$}--++(-\kpin,0)--++(0,-\kydim/2-\kpin)node[below]{$p_5$};
\draw(\kxdim-\kul,0)node[above]{$s$}--++(0,-\kpin)node[below]{$p_4$};
\draw(\kxsep+\kxdim-\kul,0)node[above]{$s$}--++(0,-\kpin)node[below]{$p_3$};
\draw(2*\kxsep+\kxdim-\kul,0)node[above]{$s$}--++(0,-\kpin)node[below]{$p_2$};
\draw(\kxdim-\kul,\kydim)node[below]{$y$}--++(0,\kysep-\kydim/2)--++(\kxbase-\kxdim+\kul,0)
node[right]{$c_{\text{خ}}$};
\draw(\kxsep+\kul,\kydim)node[below]{$z$}--++(0,\kysep-\kydim)node[above]{$s$};
\draw(2*\kxsep+\kul,\kydim)node[below]{$z$}--++(0,\kysep-\kydim)node[above]{$s$};
\draw(\kxbase+\kxdim,\kysep+\kydim/2)node[left]{$c_{\text{د}}$}--++(\kxsep-\kxdim,0)node[right]{$c_{\text{خ}}$};
\draw(\kxbase+\kxsep+\kxdim,\kysep+\kydim/2)node[left]{$c_{\text{د}}$}--++(\kxsep-\kxdim,0)
node[right]{$c_{\text{خ}}$};
\draw(\kxbase+2*\kxsep+\kxdim,\kysep+\kydim/2)node[left]{$c_{\text{د}}$}--++(\kpin/2,0)node[right]{$0$};
\draw(\kxdim,\kydim/2)node[left]{$c_{\text{د}}$}--++(\kxsep-\kxdim,0)node[right]{$c_{\text{خ}}$};
\draw(\kxsep+\kxdim,\kydim/2)node[left]{$c_{\text{د}}$}--++(\kxsep-\kxdim,0)node[right]{$c_{\text{خ}}$};
\draw(2*\kxsep+\kxdim,\kydim/2)node[left]{$c_{\text{د}}$}--++(\kpin/2,0)node[right]{$0$};
\draw(\kxbase+\kul,\kysep+\kydim)node[below]{$z$}--++(0,\kpin)node[above]{$0$};
\draw(\kxbase+\kxdim-\kul,\kysep+\kydim)node[below]{$y$}--++(0,\kpin)node[above]{$B_2$};
\draw(\kxbase+\kxsep+\kul,\kysep+\kydim)node[below]{$z$}--++(0,\kpin)node[above]{$A_2$};
\draw(\kxbase+\kxsep+\kxdim-\kul,\kysep+\kydim)node[below]{$y$}--++(0,\kpin)node[above]{$B_1$};
\draw(\kxbase+2*\kxsep+\kul,\kysep+\kydim)node[below]{$z$}--++(0,\kpin)node[above]{$A_1$};
\draw(\kxbase+2*\kxsep+\kxdim-\kul,\kysep+\kydim)node[below]{$y$}--++(0,\kpin)node[above]{$B_0$};
\draw(\kxbase+2*\kxsep+\kxdim-\kul,\kysep)node[above]{$s$}--++(0,-\kysep-\kpin)node[below]{$p_1$};
\draw(\kxbase+3*\kxsep+\kul,\kysep+\kydim+\kpin)node[above]{$A_0$}--++(0,-\kpin-\kydim-\kysep-\kpin)node[below]{$p_0$};
\end{tikzpicture}
\caption{تین بِٹ ثنائی ضرب کار }
\label{شکل_ترکیبی_تین_بٹ_ضرب_کار}
\end{figure}

اس شکل میں \عددی{9} ضرب گیٹ اور \عددی{ 6} مکمل جمع کار مستعمل ہیں۔ ضرب گیٹ \عددی{u_1} مداخل \عددی{a_0} اور \عددی{b_1} کا منطقی ضرب \عددی{A_1=a_0b_1} دے گا، جو مکمل جمع کار \عددی{u_{14}} کا \عددی{z} مداخل ہے۔ شکل میں \عددی{u_1} کے مخارج سے \عددی{u_{14}} کے \عددی{z} مداخل تک تار نظر پوش کرتے ہوئے دونوں کو ایک نام \عددی{(A_1)} سے پکارا گیا ہے۔ دو نقطوں کو ایک نام سے پکارنا، دونوں کو آپس میں تار سے جوڑنے کے مترادف ہے۔

\باب{معاصر ترتیبی منطق اور ادوار}
منطق میں، عموماً، دو متضاد صورتیں سامنے آتی ہیں، مثلاً، بلند اور پست، صادق اور کاذب، صادق اور کاذب، وغیرہ؛ جنہیں عددی برقیات میں \عددی{1} اور \عددی{0} سے ظاہر کیا جاتا ہے۔یوں، اگر بلند کو \عددی{1} سے ظاہر کیا جائے، تب پست کو \عددی{0} ظاہر کرے گا، اور اگر بلند کو \عددی{0}سے ظاہر کیا جائے، تب پست کو \عددی{1} سے ظاہر کیا جائے گا۔اگر صادق کو \عددی{1} سے ظاہر کیا جائے، تب کاذب کو \عددی{0} ظاہر کرے گا۔ اگر صادق کو \عددی{1} سے ظاہر کیا جائے، تب کاذب کو \عددی{0} ظاہر کرے گا۔ اس کتاب میں بلند یا صادق کو \عددی{1} جبکہ پست یا کاذب کو \عددی{0} سے ظاہر کیا جائے گا۔

عددی برقیات میں \عددی{1} کو مثبت پانچ وولٹ \عددی{(\SI{+5}{\volt})} اور \عددی{0} کو صفر وولٹ \عددی{(\SI{0}{\volt})} کے برقی دباو سے ظاہر کرنے کو \اصطلاح{ مثبت منطقی نظام}\فرہنگ{منطقی نظام!مثبت}\حاشیہب{positive logic system}\فرہنگ{logic system!positive} کہتے ہیں۔اس کتاب میں یہی نظام استعمال ہو گا۔

ہم اس کو اُلٹ کر کے \عددی{1} کو صفر وولٹ \عددی{(\SI{0}{\volt})} اور \عددی{0} کو مثبت پانچ وولٹ \عددی{(\SI{+5}{\volt})} سے ظاہر کر سکتے ہیں، جو \اصطلاح{منفی منطقی نظام}\فرہنگ{منطقی نظام!منفی}\حاشیہب{negative logic system}\فرہنگ{logic system!negative} کہلاتا ہے۔


اب تک، ہم ثنائی گیٹوں کا مطالعہ کرتے رہے ہیں، جن کا مخارج اُسی لمحہ تبدیل ہو جاتا ہے جس لمحے ان کے مداخل تبدیل ہوں۔عددی برقیات میں ادوار کی ایک اہم قسم ایسی ہے، جو مداخل تبدیل ہونے کے باوجود، مخارج کو اپنے حال میں برقرار رکھ سکتی ہے۔اس قسم کے ادوار \اصطلاح{پلٹ کار}\فرہنگ{پلٹ کار}\حاشیہب{flip flop}\فرہنگ{flip flop} کہلاتے ہیں، جن کے دو متضاد مخارج ہوں گے۔

پلٹ کار ایک ثنائی ہندسہ (ایک بٹ) ذخیرہ کرنے کی صلاحیت رکھتا ہے، لہٰذا اس کو \اصطلاح{حافظہ}\فرہنگ{حافظہ}\حاشیہب{memory}\فرہنگ{memory} کے طور استعمال کیا جا سکتا ہے۔پلٹ کار استعمال کرتے ہوئے \اصطلاح{گنت کار}\فرہنگ{گنت کار}\حاشیہب{counter}\فرہنگ{counter}، وغیرہ تشکیل دیے جاتے ہیں۔اس باب میں پلٹ کار اور اس پر مبنی \اصطلاح{معاصر ادوار} پر غور کیا جائے گا۔ معاصر ادوار وہ ادوار ہیں جن کے تمام حصے قدم ملا کر چلتے ہیں۔


\حصہ{گیٹوں کے اوقات کار} 
ثنائی ادوار کی کارکردگی پر تبصرہ کرنے سے پہلے چند تکنیکی اصطلاحات جاننا ضروری ہے۔شکل \حوالہ{شکل_ترتیبی_کنارے} میں گیٹ کا مخارج بلند ہو کر دوبارہ پست ہوتا دکھایا گیا، جہاں (وقت \عددی{t} کے ساتھ دائیں رخ چلتے ہوئے) پہلے کنارے کو \اصطلاح{کنارہ چڑھائی}\فرہنگ{کنارہ!چڑھائی}\حاشیہب{rising edge}\فرہنگ{edge!rising} یا \اصطلاح{مثبت کنارہ}\فرہنگ{کنارہ!مثبت}\حاشیہب{positive going edge}\فرہنگ{edge!positive going} ، جبکہ دوسرے کو \اصطلاح{کنارہ اترائی}\فرہنگ{کنارہ!اترائی}\حاشیہب{falling edge}\فرہنگ{edge!falling} یا \اصطلاح{منفی کنارہ}\فرہنگ{کنارہ!منفی}\حاشیہب{negative going edge}\فرہنگ{edge!negative going} کہا گیا۔ مخارج کا حال یکدم تبدیل ہوتا دکھایا گیا، جو درست نہیں۔
\begin{figure}
\centering
\begin{tikzpicture}
\pgfmathsetmacro{\kdimX}{2}
\pgfmathsetmacro{\kdimY}{1}
\pgfmathsetmacro{\kpin}{0.75}
\draw[thick](0,0)--++(\kpin,0)--++(0,\kdimY)node[pos=0.5,pin={135:{\text{\RL{\begin{minipage}{0.22\textwidth}
کنارہ چڑھائی،\\
مثبت کنارہ
\end{minipage}
}}}}]{}--++(\kdimX,0)--++(0,-\kdimY)node[pos=0.5,pin={45:{\text{\RL{\begin{minipage}{0.12\textwidth}
کنارہ اترائی،\\
منفی کنارہ
\end{minipage}
}}}}]{}--++(\kpin,0);
\draw[-stealth](\kdimX/2+\kpin/2,-\kpin/2)--++(\kpin,0)node[right]{$t$};
\end{tikzpicture}
\caption{کنارہ چڑھائی اور کنارہ اترائی}
\label{شکل_ترتیبی_کنارے}
\end{figure}

برقیاتی گیٹ نہایت چُست ہوتے ہیں، جو مخارج کو \موٹا{ پست} سے \موٹا{بلند} یا \موٹا{بلند} سے \موٹا{پست} بہت کم دورانیوں میں کرتے ہیں۔یہ دورانیے کم ضرور، لیکن صفر نہیں ہوتے۔ برقی اشارہ، روشنی کی رفتار سے بھی سفر کرتے ہوئے، داخلی پنیا سے خارجی پنیے تک، قابل پیما وقت میں پہنچے گا۔ نفی گیٹ مثال بنا کر حقیقی دورانیوں پر غور کرتے ہیں (جو باقی گیٹوں کے لئے بھی درست ہو گا)۔اشکال پر غور کے دوران یاد رکھیں، وقت بائیں سے دائیں رخ ہو گا، اور تمام معلومات اس حقیقت کو ذہن میں رکھتے ہوئے پیش کی جائیں گی۔

شکل \حوالہ{شکل_ترتیبی_اوقات_کار} میں نفی گیٹ کا مداخل (بالائی ترسیم) اور مخارج (نچلی ترسیم) بیک وقت دکھائے گئے ہیں، جہاں دورانیوں کو بڑھا چڑھا کر پیش کیا گیا ہے۔
\begin{figure}
\centering
\begin{tikzpicture}
\pgfmathsetmacro{\tra}{0.4}
\pgfmathsetmacro{\tfa}{0.5}
\pgfmathsetmacro{\tr}{0.5}
\pgfmathsetmacro{\tf}{0.6}
\pgfmathsetmacro{\td}{0.75}
\pgfmathsetmacro{\th}{2}
\pgfmathsetmacro{\tb}{0.5}
\pgfmathsetmacro{\ta}{1.5}
\pgfmathsetmacro{\h}{1.00}
\pgfmathsetmacro{\g}{0.75}
\pgfmathsetmacro{\cut}{0.15}
\pgfmathsetmacro{\ext}{0.5}
\draw[thick](0,0)--++(\tb,0)--++(\tra,\h)coordinate[pos=0.5](midriseU)--++(\th,0)--++(\tfa,-\h)--++(\ta,0);
\draw[thick](0,-\g)--++(\td+\tb,0)--++(\tf,-\h)coordinate[pos=0.1](tfa)coordinate[pos=0.5](tfb)
coordinate[pos=0.9](tfc)--++(\th,0)--++(\tr,\h)coordinate[pos=0.1](tra)
coordinate[pos=0.5](trb)coordinate[pos=0.9](trc)--++(\ta-\td,0);
\draw(midriseU)++(-\cut/2,0)node[left]{\small$\SI{50}{\percent}$}--++(\cut,0);
\draw[dashed](0,0)++(\tb,0)++(\tra,\h)--++(0,\ext)coordinate(aa) (0,-\g)++(\td+\tb,0)--++(0,\h+\g+\ext)coordinate(bb);
\draw[stealth-](bb)++(0,-\ext/3)--++(0.25,0);
\draw[stealth-](aa)++(0,-\ext/3)--++(-0.5,0)node[left]{\RL{ناپسندیدہ صورت}};
\draw(tfa)++(-\cut/2,0)--++(\cut,0)++(0.9*\tf,0)node[right]{$\small\SI{90}{\percent}$};
\draw(tfb)++(-\cut/2,0)--++(\cut,0)++(0.5*\tf,0)node[right]{$\small\SI{50}{\percent}$};
\draw(tfc)++(-\cut/2,0)--++(\cut,0)++(0.1*\tf,0)node[right,yshift=0.15em]{$\small\SI{10}{\percent}$};
\draw(tra)++(-\cut/2,0)--++(\cut,0);
\draw(trb)++(-\cut/2,0)--++(\cut,0);
\draw(trc)++(-\cut/2,0)--++(\cut,0);
\draw[dashed](midriseU)--++(0,-\h/2-2*\g/3)coordinate(ktd);
\draw[dashed](tfb)--++(0,\h/2+\g);
\draw[stealth-](ktd)++(0,\g/3)coordinate(rgt)--++(-1,0)coordinate(lft)node[left]{\RL{دورانیہ رد عمل}};
\draw[stealth-]($(rgt)!(tfb)!(lft)$)--++(0.5,0);
\draw[dashed](tfa)--++(0,-0.9*\h-\ext)coordinate(fls) (tfc)--++(0,-0.1*\h-\ext)coordinate(fle);
\draw[stealth-](fls)++(0,\ext/3)--++(-0.5,0)node[left]{\RL{دورانیہ اترائی}};
\draw[stealth-](fle)++(0,\ext/3)--++(0.5,0);
\draw[dashed](tra)--++(0,-0.1*\h-\ext)coordinate(frs) (trc)--++(0,-0.9*\h-\ext)coordinate(fre);
\draw[stealth-](frs)++(0,\ext/3)--++(-0.5,0);
\draw[stealth-](fre)++(0,\ext/3)--++(0.5,0)node[right]{\RL{دورانیہ چڑھائی}};
\end{tikzpicture}
\caption{نفی گیٹ کے دورانیے}
\label{شکل_ترتیبی_اوقات_کار}
\end{figure}


 بلند سے پست حال پہنچنے کے دورانیہ کو \اصطلاح{دورانیہ اترائی}\فرہنگ{دورانیہ!اترائی}\حاشیہب{fall time}\فرہنگ{time!fall} اور پست سے بلند پہنچنے کے دورانیہ کو\اصطلاح{دورانیہ چڑھائی}\فرہنگ{دورانیہ!چڑھائی}\حاشیہب{rise time}\فرہنگ{times!rise} کہتے ہیں۔ان دورانیوں کی پیمائش کی وضاحت شکل میں کی گئی ہے۔داخلی برقی اشارہ بھی کسی گیٹ سے آتا ہو گا، لہٰذا یہ بھی پست سے بلند یا بلند سے پست ہونے میں وقت گزارے گا۔
 
مداخل تبدیل ہوتے ہی مخارج تبدیل نہیں ہو جاتا، بلکہ کچھ دیر یوں محسوس ہوتا ہے جیسے مداخل کا مخارج پر کوئی اثر نہیں۔مداخل کے کنارہ چڑھائی پر غور کریں۔مداخل کے بلند ہونے کے باوجود، مخارج کچھ دیر بلند رہتا ہے۔یہ ناقابل قبول صورت حال ہے، جس پر عددی ادوار کے تشکیل کے دوران نظر رکھنی ضروری ہے۔مداخل بلند ہونے کے کچھ وقفہ بعد مخارج نیا حال اختیار کرتا ہے۔اس وقفہ کو \اصطلاح{دورانیہ رد عمل}\فرہنگ{دورانیہ!رد عمل}\حاشیہب{propagation delay}\فرہنگ{propagation delay} کہتے ہیں۔دورانیہ رد عمل ناپنے کی وضاحت شکل میں کی گئی ہے۔برقیاتی گیٹوں کے دورانیہ اترائی، دورانیہ چڑھائی، اور دورانیہ رد عمل، عموماً، چند نینو سیکنڈ ہوں گے۔

کارخانے میں گیٹ سازی کے دوران، اجزاء میں معمولی سے معمولی فرق کی بنا (ایک قسم کے دو) گیٹوں کے دورانیے کبھی ایک جیسے نہیں ہوں گے۔ان میں \عددی{10^{-9}} سیکنڈ کا نہیں تو \عددی{10^{-12}}سیکنڈ کا فرق ضرور ہو گا، جو عمر رسیدگی کے ساتھ اور استعمال کے حالات (درجہ حرارت، نمی، دباو، وغیرہ) سے تبدیل ہوں گے ۔

\ابتدا{مشق}
انٹرنیٹ سے \عددی{74xx} اور \عددی{74Hxx} سلسلہ کے دورانیوں میں فرق دریافت کریں۔
\انتہا{مشق}

\حصہ{پلٹ کار}
شکل \حوالہ{شکل_ترتیبی_ایس_آر} میں \اصطلاح{ایس آر}\فرہنگ{پلٹ کار!ایس آر}\حاشیہب{Set-Reset Flip Flop, (SR FF)}\فرہنگ{SR FF} پلٹ کار کا دور اور جدول پیش ہیں۔پلٹ کار کو، روایتاً، مداخل کے نام\حاشیہد{پلٹ کار کے مداخل انگریزی الفاظ \تحریر{\, Set \,} اور \تحریر{\, Reset \,} کے سر حرف \عددی{\, S\,} اور \عددی{\,R\,} ہیں۔} سے پکارا جاتا ہے، جو یہاں لاطینی حروف \قول{ایس} \حاشیہب{S} اور \قول{آر}\حاشیہب{R} ہیں۔ پلٹ کار کے دو متضاد مخارج ہوں گے، جنہیں \عددی{Q} اور \عددی{\overline{Q}} سے ظاہر کیا جاتا ہے۔یوں، اگر مخارج \عددی{Q} کی قیمت \عددی{1} ہو، تب مخارج \عددی{\overline{Q}} کی قیمت \عددی{0} ہو گی، اور اگر \عددی{Q=0} ہو تب \عددی{\overline{Q}=1} ہو گا۔



\begin{figure}
\centering
\begin{subfigure}{0.45\textwidth}
\centering
\begin{tikzpicture}
\pgfmathsetmacro{\kxsep}{2.5}
\pgfmathsetmacro{\kysep}{2}
\pgfmathsetmacro{\kpin}{0.5}
\draw(0,0)node[nor port,scale=1,number inputs=1](u1){$u_1$};
\draw(0,\kysep)node[nor port,scale=1,number inputs=1](u2){$u_2$};
\draw(u1.out)--++(\kpin,0)node[right]{$\overline{Q}$};
\draw(u2.out)--++(\kpin,0)node[right]{$Q$};
\draw(u1.out)--++(0,\kpin)coordinate(aa) (u2.out)--++(0,-\kpin)coordinate(bb);
\draw(u1.in 1)--++(0,\kpin)coordinate(cc) (u2.in 2)--++(0,-\kpin)coordinate(dd);
\draw(bb)--(cc) (aa)--(dd);
\draw(u1.in 2)--++(-\kpin,0)node[left]{$S$};
\draw(u2.in 1)--++(-\kpin,0)node[left]{$R$};
\end{tikzpicture}
\end{subfigure}\hfill
\begin{subfigure}{0.45\textwidth}
\centering
\begin{otherlanguage}{english}
\begin{tabular}{CC|CCr}
\toprule
S&R&Q_{n+1}&\overline{Q}_{n+1}&\\
\midrule
0&0&Q_n&\overline{Q}_n&\text{\RL{برقرار حال}}\\
0&1&0&1&\text{\RL{پست حال}}\\
1&0&1&0&\text{\RL{بلند حال}}\\
1&1&?&?&\text{\RL{ممنوعہ حال}}\\
\bottomrule
\end{tabular}
\end{otherlanguage}
\end{subfigure}
\caption{بلند فعال مداخل ایس آر پلٹ کار}
\label{شکل_ترتیبی_ایس_آر}
\end{figure}

شکل \حوالہ{شکل_ترتیبی_ایس_آر} میں متمم جمع گیٹ \عددی{u_1} کا مخارج، متمم جمع گیٹ \عددی{u_2} کا ایک مداخل، اور \عددی{u_2} کا مخارج، \عددی{u_1} کا ایک مداخل ہے۔متمم جمع \عددی{u_1} کے مخارج پر نظر رکھیں؛ یہ مخارج، \عددی{u_2} کا ایک مداخل ہے، لہٰذا اس کے مخارج پر اثر انداز ہو گا؛ لیکن \عددی{u_2} کا مخارج \عددی{u_1} کا ایک مداخل ہے، جو \عددی{u_1} کے مخارج پر اثر انداز ہو گا؛ یوں \عددی{u_1} کا مخارج، خود پر اثر انداز ہو گا! اس عمل کو \اصطلاح{باز رسی }\فرہنگ{باز رسی}\حاشیہب{feedback}\فرہنگ{feedback} کہتے ہیں۔

ایسا اشارہ ، مثلاً \عددی{\overline{Q}} ، جو خود پر اثر انداز ہو \اصطلاح{باز رسی اشارہ}\فرہنگ{باز رسی!اشارہ}\حاشیہب{feedback signal}\فرہنگ{feedback!signal} کہلاتا ہے۔ 

یہاں \عددی{Q} اور \عددی{\overline{Q}} دونوں بطور باز رسی اشارات استعمال کیے گئے ہیں۔ آپ دیکھ سکتے ہیں کہ \عددی{Q} کی قیمت جاننے کے لئے \عددی{\overline{Q}} کی قیمت معلوم ہونا ضروری ہے، لیکن \عددی{\overline{Q}} کی قیمت صرف اس صورت معلوم ہو سکتی ہے جب \عددی{Q} کی قیمت معلوم ہو! آئیں اس پلٹ کار کا جدول حاصل کریں۔

ہم پلٹ کار کے ( \عددی{n} قدم گزرنے کے بعد) موجودہ مخارج کو \عددی{Q_n} اور \عددی{\overline{Q}_n} لکھتے ہیں۔اب (باز رسی) مداخل
 \عددی{Q_n}، \عددی{\overline{Q}_n} اور سادہ مداخل \عددی{S} ، \عددی{R} کو دیکھتے ہوئے ( \عددی{n+1} واں قدم گزرنے کے بعد) متوقع مخارج حاصل کرتے ہیں، جنہیں ہم \عددی{Q_{n+1}} اور \عددی{\overline{Q}_{n+1}} لکھتے ہیں۔ اس کی تصوراتی صورت
 شکل \حوالہ{شکل_ترتیبی_اگلا_مخارج} میں پیش ہے۔

\begin{figure}
\centering
\begin{tikzpicture}
\pgfmathsetmacro{\kxsep}{2.5}
\pgfmathsetmacro{\kysep}{2}
\pgfmathsetmacro{\kpin}{0.5}
\draw(0,0)node[nor port,scale=1,number inputs=1](u1){$u_1$};
\draw(0,\kysep)node[nor port,scale=1,number inputs=1](u2){$u_2$};
\draw(u1.out)--++(\kpin,0)node[right]{$\overline{Q}_{n+1}$};
\draw(u2.out)--++(\kpin,0)node[right]{$Q_{n+1}$};
\draw(u1.out)++(0,\kpin)coordinate(aa) (u2.out)++(0,-\kpin)coordinate(bb);
\draw(u1.in 1)--++(0,\kpin)--(bb)node[right]{$Q_n$} (u2.in 2)--++(0,-\kpin)--(aa)node[right]{$\overline{Q}_n$};
%\draw(bb)--(cc) (aa)--(dd);
\draw(u1.in 2)--++(-\kpin,0)node[left]{$S$} (u2.in 1)--++(-\kpin,0)node[left]{$R$};
%\draw(u2.in 1)node[left]{$R$} (u2.in 2)node[left]{$\overline{Q}_n$};
\end{tikzpicture}
\caption{موجودہ مخارج سے اگلے مخارج کا حصول۔}
\label{شکل_ترتیبی_اگلا_مخارج}
\end{figure}

شکل \حوالہ{شکل_ترتیبی_اگلا_مخارج} میں بالائی گیٹ \عددی{(u_2)} کے اگلے مخارج \عددی{Q_{n+1}} کو موجودہ مداخل \عددی{R} اور \عددی{\overline{Q}} کے روپ میں لکھتے ہیں۔
 \begin{align}\label{مساوات_ترتیبی_اگلا_کیو}
 Q_{n+1}=\overline{R+\overline{Q}_n}
 \end{align}
 جیسا آپ نے شکل \حوالہ{شکل_ترتیبی_اوقات_کار} میں دیکھا، گیٹ کا مخارج، دورانیہ رد عمل گزرنے کے بعد، مداخل کے تحت حال اختیار کرتا ہے۔یوں موجودہ \عددی{\overline{Q}_n} اور مداخل \عددی{R} جب نئی قیمت اختیار کریں، گیٹ کچھ دیر بعد نئی قیمت \عددی{Q_{n+1}} اختیار کرتا ہے۔
 
 نچلی گیٹ \عددی{(u_1)} کے مخارج کی مساوات درج ذیل ہو گی۔ یہ گیٹ بھی مداخل تبدیل ہونے کے کچھ دیر بعد مخارج تبدیل کرے گا۔
 \begin{align}\label{مساوات_ترتیبی_اگلا_متمم_کیو}
 \overline{Q}_{n+1}=\overline{S+Q_n}
 \end{align}
بالائی گیٹ کی خارجی مساوات حاصل کرنے کی غرض سے مساوات \حوالہ{مساوات_ترتیبی_اگلا_متمم_کیو} کو مساوات \حوالہ{مساوات_ترتیبی_اگلا_کیو} میں ڈال کر مسئلہ ڈی مارگن سے حل کرتے ہیں۔
\begin{gather}
\begin{aligned}\label{مساوات_ترتیبی_اگلا_کیو_حل}
Q_{n+1}&=\overline{R+(\overline{S+Q_n})}\\
&=\overline{R}(\overline{\overline{S+Q_n}})\\
&=\overline{R}(S+Q_n)
\end{aligned}
\end{gather}


مساوت \حوالہ{مساوات_ترتیبی_اگلا_کیو_حل} میں دائیں ہاتھ کے تین متغیرات \عددی{S}، \عددی{R}، اور \عددی{Q_n} کو آزاد متغیرات تصور کر کے تابع متغیر \عددی{Q_{n+1}} کو جدول\حوالہ{جدول_ترتیبی_ایس_آر_جدول} -الف میں پیش کیا گیا ہے۔(متغیر \عددی{R} مساوات میں \عددی{\overline{R}} کے روپ میں موجود ہے۔)
\begin{table}
\caption{ایس آر پلٹ کار (مساوات \حوالہ{مساوات_ترتیبی_اگلا_کیو_حل} اور مساوات \حوالہ{مساوات_ترتیبی_ایس_آر_متمم})}
\label{جدول_ترتیبی_ایس_آر_جدول}
\centering
\begin{subtable}{0.45\textwidth}
\centering
\begin{otherlanguage}{english}
\begin{tabular}{CCC|C}
\toprule
S&R&Q_n&Q_{n+1}\\
\midrule
0&0&0&0\\
0&0&1&1\\
\midrule
0&1&0&0\\
0&1&1&0\\
\midrule
1&0&0&1\\
1&0&1&1\\
\midrule
1&1&0&0\\
1&1&1&0\\
\bottomrule
\end{tabular}
\end{otherlanguage}
\caption{}
\end{subtable}\hfill
\begin{subtable}{0.45\textwidth}
\centering
\begin{otherlanguage}{english}
\begin{tabular}{CCC|C}
\toprule
S&R&Q_n&\overline{Q}_{n+1}\\
\midrule
0&0&0&1\\
0&0&1&0\\
\midrule
0&1&0&1\\
0&1&1&1\\
\midrule
1&0&0&0\\
1&0&1&0\\
\midrule
1&1&0&0\\
1&1&1&0\\
\bottomrule
\end{tabular}
\end{otherlanguage}
\caption{}
\end{subtable}
\end{table}

اسی طرح شکل \حوالہ{شکل_ترتیبی_اگلا_مخارج} میں نچلی گیٹ کی خارجی مساوات حاصل کرنے کی غرض سے مساوات \حوالہ{مساوات_ترتیبی_اگلا_کیو} کو مساوات \حوالہ{مساوات_ترتیبی_اگلا_متمم_کیو} میں ڈال کر مسئلہ ڈی مارگن سے حل کرتے ہیں۔
\begin{gather}
\begin{aligned}\label{مساوات_ترتیبی_ایس_آر_متمم}
\overline{Q}_{n+1}&=\overline{S+(\overline{R+\overline{Q}_n})}\\
&=\overline{S}(\overline{\overline{R+\overline{Q}_n}})\\
&=\overline{S}(R+\overline{Q}_n)
\end{aligned}
\end{gather}
مساوت \حوالہ{مساوات_ترتیبی_ایس_آر_متمم} میں متغیرات \عددی{S}، \عددی{R}، اور \عددی{Q_n} آزاد متغیرات تصور کر کے تابع متغیر \عددی{\overline{Q}_{n+1}} کو جدول \حوالہ{جدول_ترتیبی_ایس_آر_جدول} -ب میں پیش کیا گیا ہے۔(متغیر \عددی{S} اور \عددی{Q_n} مساوات میں بالترتیب \عددی{\overline{S}} اور \عددی{\overline{Q}_n} کے روپ میں موجود ہیں۔)



 جدول \حوالہ{جدول_ترتیبی_ایس_آر_جدول}-الف اور ب کو \عددی{S} اور \عددی{R} کی قیمتوں کے لحاظ سے چار حصوں میں تقسیم کیا گیا۔پہلے حصہ میں \عددی{S=0} اور \عددی{R=0} ہے، جبکہ \عددی{Q_{n+1}} کی قیمت \عددی{Q_n} کے برابر ہے۔ ہم کہتے ہیں، مداخل \عددی{S=0} اور \عددی{R=0} کی صورت میں ایس آر پلٹ کار \قول{برقرار حال} ہو گا۔ جدول-ب
 میں \عددی{\overline{Q}_{n+1}} کی قیمت، جدول-الف میں \عددی{Q_{n+1}}کی قیمت کی متمم ہے۔ہم چاہتے بھی یہی ہیں (کہ پلٹ کار کے دو مخارج آپس میں متضاد ہوں )۔
 
دوسرے حصہ میں \عددی{S=0} اور \عددی{R=1} ہے، جبکہ \عددی{Q_{n+1}} پست ہو گا۔ہم کہتے ہیں، ان مداخل کے لئے ایس آر پلٹ کار \قول{پست حال} ہو گا۔یہاں بھی (جدول-الف اور ب کے تحت) نئے مخارج ایک دوسرے کے متضاد ہیں۔

تیسرے حصہ میں \عددی{S=1} اور \عددی{R=0} ہے، جبکہ پلٹ کار \قول{بلند حال} ہے۔

چوتھے حصہ میں \عددی{S=1} اور \عددی{R=1} ہے، جبکہ جدول کے تحت \عددی{Q_{n+1}} اور \عددی{\overline{Q}_{n+1}} دونوں پست ہیں، جو ہم نہیں چاہتے، ہم کہتے ہیں پلٹ کار \قول{ممنوعہ حال}( میں) ہے۔ پلٹ کار کی صحیح کارکردگی کے لئے یہ مداخل \قول{ممنوعہ} قرار دیے جاتے ہیں۔ یوں \عددی{S} اور \عددی{R} اکٹھے بلند نہیں کیے جاتے۔

ان حقائق کو شکل \حوالہ{شکل_ترتیبی_ایس_آر} کے جدول میں پیش کیا گیا (جو پلٹ کار کا جدول لکھنے کا درست طریقہ ہے)، جہاں آخری صف میں \عددی{?} لکھ کر واضح کیا جاتا ہے کہ ان صف کے مداخل استعمال نہ کیے جائیں۔ 

\موٹا{ایس آر پلٹ کار کی کارکردگی}
\begin{align}\label{مساوات_ترتیبی_ایس_آر_کارکردگی_نچوڑ}
\begin{array}{c|cr}
SR&Q_{n+1}&\\
\hline
00&Q_n&\text{\RL{برقرار حال}}\\
01&0&\text{\RL{پست حال}}\\
10&1&\text{\RL{بلند حال}}\\
11&?&\text{\RL{ممنوعہ حال}}
\end{array}
\end{align} 
پلٹ کار کی بات کرتے وقت \عددی{Q} کی قیمت کو پلٹ کار کا \اصطلاح{حال}\فرہنگ{حال}\حاشیہب{state}\فرہنگ{state} کہتے ہیں۔ یوں \عددی{Q=1} کی صورت میں پلٹ کار \اصطلاح{بلند حال}\فرہنگ{حال!بلند}\حاشیہب{high state}\فرہنگ{state!high} یا \اصطلاح{صادق حال}\فرہنگ{حال!صادق}\حاشیہب{true state}\فرہنگ{state!true} ، جبکہ \عددی{Q=0} کی صورت میں \اصطلاح{پست حال}\فرہنگ{حال!پست}\حاشیہب{low state}\فرہنگ{state!low} یا \اصطلاح{کاذب حال}\فرہنگ{حال!کاذب}\حاشیہب{false state}\فرہنگ{state!false} کہلائے گا۔ 



جدول سے ظاہر ہے کہ جب \عددی{S} بلند ہو، پلٹ کار بلند حال اختیار کرتا ہے۔یوں، مداخل \عددی{S}، \موٹا{بلند} صورت میں \اصطلاح{فعال}\فرہنگ{فعال}\حاشیہب{active}\فرہنگ{active} ہو گا۔ وہ مداخل جو بلند صورت میں فعال ہو، \اصطلاح{ بلند فعال}\فرہنگ{بلند فعال}\حاشیہب{active high}\فرہنگ{active!high} مداخل کہلاتا ہے۔ وہ مداخل جو پست صورت میں فعال ہو، \اصطلاح{ پست فعال}\فرہنگ{پست فعال}\حاشیہب{active low}\فرہنگ{active!low} کہلاتا ہے۔ جب بلند فعال مداخل، پست ہو، مثلاً، \عددی{S=0} ، ہم کہتے ہیں یہ \اصطلاح{غیر فعال}\فرہنگ{غیر فعال}\حاشیہب{inactive}\فرہنگ{inactive} (حال میں) ہے۔ یوں اس پلٹ کار کا بہتر نام \اصطلاح{بلند فعال مداخل ایس آر پلٹ کار} ہو گا۔

پلٹ کار خود اس صورت فعال کہلاتا ہے جب \عددی{Q=1} ہو۔ پست فعال مداخل اور مخارج (\عددی{\overline{Q}}) کے نام پر لکیر کھینچ کر اس کی پست فعال حیثیت واضح کی جاتی ہے؛ مزید، پلٹ کار کی علامت میں پست فعال (مداخل اور مخارج) پنیوں پر گول دائرہ لگایا جاتا ہے، جو ان کا پست فعال پن ظاہر کرتا ہے (شکل \حوالہ{شکل_ترتیبی_دو_اقسام_ایس_آر} دیکھیں)۔


پلٹ کار کے دونوں مداخل عام طور \موٹا{غیر فعال} رکھے جائیں گے؛ یوں موجودہ پلٹ کار کے مداخل پست رکھے جائیں گے۔پلٹ کار بلند (صادق) حال کرنے کے لئے \عددی{S} اشارہ ایک لمحہ کے لئے بلند (فعال) کر کے واپس پست (غیر فعال)کیا جاتا ہے۔ پہلے سے بلند حال پلٹ کار، اسی حال میں رہے گا، جبکہ پست پلٹ کار، اشارہ ملتے ہی بلند حال اختیار کرے گا۔

اسی طرح پلٹ کار کاذب (پست) حال کرنے کے لئے \عددی{R} اشارہ لمحاتی فعال کیا جاتا ہے۔

مداخل \عددی{S} کو \اصطلاح{فعال کار}\فرہنگ{فعال کار!مداخل}\حاشیہب{set input}\فرہنگ{set!input} مداخل جبکہ \عددی{R} کو \اصطلاح{غیر فعال کار}\فرہنگ{غیر فعال کار!مداخل}\حاشیہب{reset or clear input}\فرہنگ{reset!input} مداخل\فرہنگ{clear!input} کہہ سکتے ہیں۔

آپ نے دیکھا، پلٹ کار درحقیقت مداخل کا (بلند یا پست) حال محفوظ کرتا ہے۔یوں اگر مداخل اشارہ لمحاتی فعال ہونے کے بعد غیر فعال ہو جائے، پلٹ کار (اگلے نئے اشارے تک ) اس کا حال محفوظ رکھتا ہے۔ 

\حصہ{ساعت}
عددی ادوار کی ایک قسم جو \اصطلاح{ہم عصر}\فرہنگ{ہم عصر}\حاشیہب{synchronous}\فرہنگ{synchronous} ادوار کہلاتے ہیں کو، عموماً، مقررہ دورانیے کا مسلسل دہراتا داخلی اشارہ درکار ہو گا، جو \اصطلاح{ساعت}\فرہنگ{ساعت}\حاشیہب{clock}\فرہنگ{clock} کہلاتا ہے۔ ساعت اشارہ شکل \حوالہ{شکل_ترتیبی_ساعت_تفصیل} میں پیش ہے۔اگرچہ اس طرح کی اشکال میں دورانیہ چڑھائی اور دورانیہ اترائی نہیں دکھائے جاتے، امید کی جاتی ہے کہ آپ ان کی موجودگی ہر وقت ذہن میں رکھیں گے۔
\begin{figure}
\centering
\begin{tikzpicture}
\pgfmathsetmacro{\tl}{1}
\pgfmathsetmacro{\th}{1.3}
\pgfmathsetmacro{\kh}{1}
\pgfmathsetmacro{\kpin}{0.2}
\pgfmathsetmacro{\kvs}{0.6}
\draw[thick](-0.5*\tl,0)--++(0.5*\tl,0)--++(0,\kh)coordinate(aa)node[above]{$1$}node[pos=0.5,pin=135:{\RL{پہلا کنارہ چڑھائی}}]{}--++(\th,0)--++(0,-\kh)node[below]{$1$}--++(\tl,0)--++(0,\kh)coordinate(bb)node[above]{$2$}--++(\th,0)--++(0,-\kh)coordinate(cc)node[below]{$2$}--++(\tl,0)coordinate(dd)--++(0,\kh)coordinate(ee)node[above]{$3$}--++(\th,0)coordinate(ff)--++(0,-\kh)node[below]{$3$}node[pos=0.5,pin={45:{\RL{تیسرا کنارہ اترائی}}}]{}--++(0.5*\tl,0);
\draw(aa)++(0,\kvs)--++(0,\kpin) (bb)++(0,\kvs)--++(0,\kpin);
\draw[stealth-stealth]($(aa)+(0,\kvs+\kpin/2)$)--($(bb)+(0,\kvs+\kpin/2)$)node[pos=0.5,fill=white]{$T$}node[pos=0.5,above,yshift=0.5em]{\RL{دوری عرصہ}};
\draw(cc)++(0,-\kvs)--++(0,-\kpin) (dd)++(0,-\kvs)--++(0,-\kpin);
\draw[stealth-stealth]($(cc)+(0,-\kvs-\kpin/2)$)--($(dd)+(0,-\kvs-\kpin/2)$)node[pos=0.5,fill=white]{$t_L$}node[pos=0.5,below,yshift=-0.5em]{\RL{پست دورانیہ}};
\draw(ee)++(0,\kvs)--++(0,\kpin) (ff)++(0,\kvs)--++(0,\kpin);
\draw[stealth-stealth]($(ee)+(0,\kvs+\kpin/2)$)--($(ff)+(0,\kvs+\kpin/2)$)node[pos=0.5,fill=white]{$t_H$}node[pos=0.5,above,yshift=0.5em]{\RL{بلند دورانیہ}};
\end{tikzpicture}
\caption{ساعت}
\label{شکل_ترتیبی_ساعت_تفصیل}
\end{figure}


ہم عصر عددی دور، مہیا کردہ ساعت کے \اصطلاح{تعدد}\فرہنگ{تعدد}\حاشیہب{frequency}\فرہنگ{frequency} کی رفتار سے چلتا ہے، اور اس کے مختلف حصے، ساعت کے کنارہ اترائی یا کنارہ چڑھائی پر بیک وقت حال تبدیل کرتے ہیں۔گویا، ہم عصر دور ساعت کے ساتھ قدم ملا کر چلتا ہے۔


شکل \حوالہ{شکل_ترتیبی_ساعت_تفصیل} میں اوپر جانب کنارہ چڑھائی کی گنتی، جبکہ نیچے جانب کنارہ اترائی کی گنتی دی گئی ہے۔ساتھ ہی، \اصطلاح{دوری عرصہ}\فرہنگ{دوری عرصہ}\حاشیہب{time period}\فرہنگ{time period}، \اصطلاح{بلند دورانیہ}\فرہنگ{دورانیہ!بلند}\حاشیہب{high time, ON time}\فرہنگ{high time} اور\اصطلاح{پست دورانیہ}\فرہنگ{دورانیہ!پست}\حاشیہب{low time, OFF time}\فرہنگ{low time} کی بھی وضاحت کی گئی ہے، جنہیں بالترتیب \عددی{T}، \عددی{t_H}، اور \عددی{t_L} سے ظاہر کیا جاتا ہے۔
یوں \عددی{T=t_H+t_L} ہو گا۔ ساعت کے بلند اور پست دورانیے برابر بھی ہو سکتے ہیں۔ ہمیشہ کی طرح، تعدد \عددی{f} اور دوری عرصہ \عددی{T} کا تعلق درج ذیل ہے، جہاں \عددی{T} کی اکائی \قول{سیکنڈ} اور \عددی{f} کی اکائی \اصطلاح{ہرٹز}\فرہنگ{ہرٹز}\حاشیہب{Hertz, Hz}\فرہنگ{Hertz} ہے
\begin{align*}
f=\frac{1}{T}
\end{align*}

ساعتی اشارہ مختصراً \اصطلاح{ساعت} پکارا جاتا ہے۔ساعت سے مراد متواتر تبدیل ہوتا اشارہ، یا اس کا بلند، یا پست دورانیہ، یا چڑھائی یا اترائی کنارہ ہو گا۔متن سے اس کا مطلوبہ مطلب واضح ہو گا۔جہاں غلط فہمی کا امکان ہو، وہاں وضاحت کی جائے گی۔

ساعت کی بات کرتے ہوئے عموماً ساعت کی \اصطلاح{دھڑکن }\فرہنگ{ساعت!دھڑکن}\حاشیہب{pulse}\فرہنگ{pulse} (جس کو مختصراً \اصطلاح{دھڑکن }کہتے ہیں) کا ذکر ہو گا، جہاں دھڑکن سے مراد ساعت کا بلند حصہ ہو گا۔ یہ اصطلاح کسی بھی اشارے کے لئے استعمال کی جا سکتی ہے جہاں اس سے مراد مستطیل باریک (کم دورانیہ) اشارہ ہو گا۔بلند دھڑکن کے علاوہ پست دھڑکن اور منفی دھڑکن بھی ہو سکتے ہیں۔ 

\حصہ{متمم ضرب گیٹ ایس آر پلٹ کار}
شکل \حوالہ{شکل_ترتیبی_متمم_ضرب_ایس_آر} میں متمم ضرب گیٹ پر مبنی \اصطلاح{پست فعال مداخل ایس آر پلٹ کار}\فرہنگ{ایس آر!پست فعال مداخل}\حاشیہب{active low inputs SR flip flop}\فرہنگ{SR flip flop!active low inputs} دکھایا گیا ہے۔ شکل \حوالہ{شکل_ترتیبی_دو_اقسام_ایس_آر} میں بلند فعال مداخل اور پست فعال مداخل ایس آر پلٹ کار کی علامتیں پیش ہیں۔ پست فعال اشارات، کے نام پر لکیر (\عددی{\overline{S}}، \عددی{\overline{Q}}) اور ان کے پنیوں پر گول دائرے ان کے پست فعال پن ظاہر کرتے ہیں۔


پلٹ کار کے مخارج \عددی{Q}اور \عددی{\overline{Q}} آپس میں متضاد (اُلٹ) حال رہتے ہیں۔آئیں اس پلٹ کار کی کارکردگی، دوسرے نقطہ نظر سے دیکھیں۔
\begin{figure}
\centering
\begin{subfigure}{0.45\textwidth}
\centering
\begin{tikzpicture}
\pgfmathsetmacro{\kxsep}{2.5}
\pgfmathsetmacro{\kysep}{2}
\pgfmathsetmacro{\kpin}{0.5}
\draw(0,0)node[nand port,scale=1,number inputs=2](u1){};
\draw(0,\kysep)node[nand port,scale=1,number inputs=2](u2){};
\draw(u1.out)--++(\kpin,0)node[right]{$\overline{Q}$};
\draw(u2.out)--++(\kpin,0)node[right]{$Q$};
\draw(u1.out)--++(0,\kpin)coordinate(aa) (u2.out)--++(0,-\kpin)coordinate(bb);
\draw(u1.in 1)--++(0,\kpin)coordinate(cc) (u2.in 2)--++(0,-\kpin)coordinate(dd);
\draw(bb)--(cc) (aa)--(dd);
\draw(u1.in 2)--++(-\kpin,0)node[left]{$\overline{R}$};
\draw(u2.in 1)--++(-\kpin,0)node[left]{$\overline{S}$};
\end{tikzpicture}
\end{subfigure}\hfill
\begin{subfigure}{0.45\textwidth}
\centering
\begin{otherlanguage}{english}
\begin{tabular}{CC|CCr}
\toprule
\overline{S}&\overline{R}&Q_{n+1}&\overline{Q}_{n+1}&\\
\midrule
0&0&?&?&\text{\RL{ممنوعہ حال}}\\
0&1&1&0&\text{\RL{بلند حال}}\\
1&0&0&1&\text{\RL{پست حال}}\\
1&1&Q_n&\overline{Q}_n&\text{\RL{برقرار حال}}\\
\bottomrule
\end{tabular}
\end{otherlanguage}
\end{subfigure}
\caption{پست فعال مداخل ایس آر پلٹ کار}
\label{شکل_ترتیبی_متمم_ضرب_ایس_آر}
\end{figure}
%
\begin{figure}
\centering
\begin{subfigure}{0.45\textwidth}
\centering
\begin{tikzpicture}
\kFF[u1]{0}{0}
\foreach \n/\m in {1/S,3/R,4/{\overline{Q}},6/Q}{\draw(u1pl\n)node[]{$\m$};}
\foreach \n in {1,3,4,6}{\draw(u1pb\n)--(u1p\n);}
\foreach \n in {4}{\draw(u1pn\n)node[ocirc]{};}
\end{tikzpicture}
\caption{بلند فعال مداخل ایس آر پلٹ کار}
\end{subfigure}\hfill
\begin{subfigure}{0.45\textwidth}
\centering
\begin{tikzpicture}
\kFF[u1]{0}{0}
\foreach \n/\m in {1/{\overline{S}},3/{\overline{R}},4/{\overline{Q}},6/Q}{\draw(u1pl\n)node[]{$\m$};}
\foreach \n in {1,3,4,6}{\draw(u1pb\n)--(u1p\n);}
\foreach \n in {1,3,4}{\draw(u1pn\n)node[ocirc]{};}
\end{tikzpicture}
\caption{پست فعال مداخل ایس آر پلٹ کار}
\end{subfigure}
\caption{ایس آر پلٹ کار کی دو علامتیں}
\label{شکل_ترتیبی_دو_اقسام_ایس_آر}
\end{figure}

\جزوحصہ{غیر فعال مداخل پلٹ کار، حال برقرار رکھتا ہے}
فرض کریں\موٹا{پست} ایس آر پلٹ کار کے مداخل \موٹا{غیر فعال} ہیں، یعنی \عددی{Q=0}، \عددی{\overline{Q}=1}، \عددی{\overline{S}=1} اور \عددی{\overline{R}=1} ہیں (شکل \حوالہ{شکل_ترتیبی_پست_پلٹ_غیر_فعال_مداخل}-الف)۔یوں، بالائی متمم ضرب گیٹ کے مداخل \عددی{1} اور \عددی{1} ہیں، لہٰذا اس کا مداخل \عددی{0} ہو گا، جو وہ پہلے سے ہے۔ اسی طرح نچلے متمم ضرب گیٹ کے مداخل \عددی{0} اور \عددی{1} ہیں، لہٰذا اس کا مخارج \عددی{1} ہو گا، جو وہ پہلے سے ہے۔
 \begin{figure}
\centering
\begin{subfigure}{0.45\textwidth}
\centering
\begin{tikzpicture}
\pgfmathsetmacro{\kxsep}{2.5}
\pgfmathsetmacro{\kysep}{2}
\pgfmathsetmacro{\kpin}{0.5}
\draw(0,0)node[nand port,scale=1,number inputs=1](u1){};
\draw(0,\kysep)node[nand port,scale=1,number inputs=1](u2){};
\draw(u1.out)--++(\kpin,0)node[right]{$\overline{Q}$}node[above left]{$1$};
\draw(u2.out)--++(\kpin,0)node[right]{$Q$}node[above left]{$0$};
\draw(u1.out)--++(0,\kpin)coordinate(aa) (u2.out)--++(0,-\kpin)coordinate(bb);
\draw(u1.in 1)--++(0,\kpin)coordinate(cc) (u2.in 2)--++(0,-\kpin)coordinate(dd);
\draw(bb)--(cc) (aa)--(dd);
\draw(u1.in 2)--++(-\kpin,0)node[left]{$\overline{R}$}node[above right]{$1$};
\draw(u2.in 1)--++(-\kpin,0)node[left]{$\overline{S}$}node[above right]{$1$};
\end{tikzpicture}
\caption{}
\end{subfigure}\hfill
\begin{subfigure}{0.45\textwidth}
\centering
\begin{tikzpicture}
\pgfmathsetmacro{\kxsep}{2.5}
\pgfmathsetmacro{\kysep}{2}
\pgfmathsetmacro{\kpin}{0.5}
\draw(0,0)node[nand port,scale=1,number inputs=1](u1){};
\draw(0,\kysep)node[nand port,scale=1,number inputs=1](u2){};
\draw(u1.out)--++(\kpin,0)node[right]{$\overline{Q}$}node[above left]{$0$};
\draw(u2.out)--++(\kpin,0)node[right]{$Q$}node[above left]{$1$};
\draw(u1.out)--++(0,\kpin)coordinate(aa) (u2.out)--++(0,-\kpin)coordinate(bb);
\draw(u1.in 1)--++(0,\kpin)coordinate(cc) (u2.in 2)--++(0,-\kpin)coordinate(dd);
\draw(bb)--(cc) (aa)--(dd);
\draw(u1.in 2)--++(-\kpin,0)node[left]{$\overline{R}$}node[above right]{$1$};
\draw(u2.in 1)--++(-\kpin,0)node[left]{$\overline{S}$}node[above right]{$1$};
\end{tikzpicture}
\caption{}
\end{subfigure}
\caption{غیر فعال مداخل کی صورت میں پلٹ کار اپنا حال برقرار رکھتی ہے۔}
\label{شکل_ترتیبی_پست_پلٹ_غیر_فعال_مداخل}
\end{figure}

فرض کریں\موٹا{بلند} پلٹ کار کے مداخل \موٹا{غیر فعال} ہیں، یعنی \عددی{Q=1}، \عددی{\overline{Q}=0}، \عددی{\overline{S}=1} اور \عددی{\overline{R}=1} ہیں (شکل \حوالہ{شکل_ترتیبی_پست_پلٹ_غیر_فعال_مداخل}-ب)۔یوں بالائی متمم ضرب گیٹ کے مداخل \عددی{1} اور \عددی{0} ہیں، لہٰذا اس کا مداخل \عددی{1} ہو گا، جو وہ پہلے سے ہے۔ اسی طرح نچلے متمم ضرب گیٹ کے مداخل \عددی{1} اور \عددی{1} ہیں، لہٰذا اس کا مخارج \عددی{0} ہو گا، جو وہ پہلے سے ہے۔

 شکل \حوالہ{شکل_ترتیبی_پست_پلٹ_غیر_فعال_مداخل} کی دونوں صورتوں پر غور کرنے سے معلوم ہوا کہ \موٹا{غیر فعال مداخل کی صورت میں پلٹ کار اپنا حال برقرار رکھتا ہے}۔ شکل \حوالہ{شکل_ترتیبی_متمم_ضرب_ایس_آر} میں جدول کی آخری صف اس حقیقت کو بیان کرتی ہے، جہاں (اگلا حال) \عددی{Q_{n+1}} موجودہ \عددی{Q_n} کے برابر ہو گا۔

 
\جزوحصہ{مداخل \عددی{S} فعال کرنے سے پلٹ کار بلند حال اختیار کرتا ہے}
تصور کریں ایس آر پلٹ کار کا مداخل \عددی{\overline{S}}، ایک لمحہ فعال کرنے کے بعد دوبارہ غیر فعال کیا جاتا ہے، یعنی لمحاتی طور \عددی{\overline{S}=0} کیا جاتا ہے۔ شکل \حوالہ{شکل_ترتیبی_لمحاتی_ایس_فعال}-الف میں وہ لمحہ پیش ہے جب \عددی{\overline{S}=0} (فعال) ہے۔ بالائی متمم ضرب گیٹ کا کوئی مداخل پست ہونے کی صورت میں اس کا مخارج بلند ہو گا، لہٰذا \عددی{\overline{S}=0} کی صورت میں بالائی گیٹ کا مخارج بلند ہو گا، جیسا شکل میں دکھایا گیا ہے (پلٹ کار کے دونوں گیٹوں کی گزشتہ قیمتیں اس حقیقت پر اثر انداز نہیں ہوں گی)۔یوں نچلے گیٹ کے دونوں مداخل بلند، لہٰذا مخارج پست \عددی{\overline{Q}=0} ہو گا۔ مداخل واپس غیر فعال \عددی{\overline{S}=1} کرنے سے شکل-ب ملتی ہے، لہٰذا پلٹ کار کا حال (\عددی{Q=1} اور \عددی{\overline{Q}=0}) برقرار رہے گا۔ یوں \موٹا{مداخل \عددی{\overline{S}} فعال کرنے سے ایس آر پلٹ کار بلند حال اختیار کرتا ہے}۔

 \begin{figure}
\centering
\begin{subfigure}{0.45\textwidth}
\centering
\begin{tikzpicture}
\pgfmathsetmacro{\kxsep}{2.5}
\pgfmathsetmacro{\kysep}{2}
\pgfmathsetmacro{\kpin}{0.5}
\draw(0,0)node[nand port,scale=1,number inputs=1](u1){};
\draw(0,\kysep)node[nand port,scale=1,number inputs=1](u2){};
\draw(u1.out)--++(\kpin,0)node[right]{$\overline{Q}$};
\draw(u2.out)--++(\kpin,0)node[right]{$Q$}node[above left]{$1$};
\draw(u1.out)--++(0,\kpin)coordinate(aa) (u2.out)--++(0,-\kpin)coordinate(bb);
\draw(u1.in 1)--++(0,\kpin)coordinate(cc) (u2.in 2)--++(0,-\kpin)coordinate(dd);
\draw(bb)--(cc) (aa)--(dd);
\draw(u1.in 2)--++(-\kpin,0)node[left]{$\overline{R}$}node[above right]{$1$};
\draw(u2.in 1)--++(-\kpin,0)node[left]{$\overline{S}$}node[above right]{$0$};
\end{tikzpicture}
\caption{}
\end{subfigure}\hfill
\begin{subfigure}{0.45\textwidth}
\centering
\begin{tikzpicture}
\pgfmathsetmacro{\kxsep}{2.5}
\pgfmathsetmacro{\kysep}{2}
\pgfmathsetmacro{\kpin}{0.5}
\draw(0,0)node[nand port,scale=1,number inputs=1](u1){};
\draw(0,\kysep)node[nand port,scale=1,number inputs=1](u2){};
\draw(u1.out)--++(\kpin,0)node[right]{$\overline{Q}$}node[above left]{$0$};
\draw(u2.out)--++(\kpin,0)node[right]{$Q$}node[above left]{$1$};
\draw(u1.out)--++(0,\kpin)coordinate(aa) (u2.out)--++(0,-\kpin)coordinate(bb);
\draw(u1.in 1)--++(0,\kpin)coordinate(cc) (u2.in 2)--++(0,-\kpin)coordinate(dd);
\draw(bb)--(cc) (aa)--(dd);
\draw(u1.in 2)--++(-\kpin,0)node[left]{$\overline{R}$}node[above right]{$1$};
\draw(u2.in 1)--++(-\kpin,0)node[left]{$\overline{S}$}node[above right]{$1$};
\end{tikzpicture}
\caption{}
\end{subfigure}
\caption{ایک لمحے کے لئے \عددی{\overline{S}} فعال کیا گیا ہے۔}
\label{شکل_ترتیبی_لمحاتی_ایس_فعال}
\end{figure}	

\جزوحصہ{مداخل \عددی{\overline{R}} فعال کرنے سے پلٹ کار پست حال اختیار کرتا ہے}
درج ذیل مشق میں آپ سے یہی ثابت کرنے کی درخواست کی گئی ہے۔

\ابتدا{مشق}
 ثابت کریں کہ \عددی{\overline{S}=1} رکھتے ہوئے ، لمحاتی طور \عددی{\overline{R}=0} کرنے سے ایس آر پلٹ کار \موٹا{پست حال} اختیار کرتا ہے۔
\انتہا{مشق}


\جزوحصہ{حال دوڑ}
ایس آر پلٹ کار کے دونوں مداخل بیکوقت پست کرنے کی اجازت نہیں، چونکہ ایسی صورت میں پلٹ کار غیر یقینی حال اختیار کرتا ہے۔دیکھتے ہیں،ایسا کیوں ہو گا۔

شکل \حوالہ{شکل_ترتیبی_متمم_ضرب_ایس_آر} پر نظر رکھتے ہوئے آگے بڑھیں۔تصور کریں پلٹ کار کے دونوں مداخل بیک وقت پست (فعال) کرنے کے بعد دوبارہ بلند (غیر فعال) کیے جاتے ہیں۔ایسا کرنے کے بعد ہم جاننا چاہتے ہیں پلٹ کار کس حال ہوگا۔

دونوں مداخل بیکوقت پست کرنے سے (بالائی اور نچلے متمم ضرب گیٹ کے مخارج بلند ہوں گے، لہٰذا) پلٹ کار کے دونوں مخارج بیک وقت بلند ہوں گے، جو نا قابل قبول صورت ہے:پلٹ کار کے مخارج \عددی{Q} اور \عددی{\overline{Q}} کا آپس میں متضاد رہنا ضروری ہے۔

 دونوں مداخل بیک وقت یکدم واپس بلند کرنے سے گیٹوں کے مخارج (یکدم حال تبدیل نہیں کرتے، صفحہ \حوالہصفحہ{شکل_ترتیبی_اوقات_کار} پر شکل \حوالہ{شکل_ترتیبی_اوقات_کار} دیکھیں، بلکہ) نئے حال کی طرف روانہ ہوتے ہیں، لیکن، جب تک ان کے مخارج نئے حال اختیار نہیں کرتے، دونوں گیٹوں کے دونوں مداخل بلند ہوں گے (مثلاً \عددی{\overline{S}} بلند کر دیا گیا ہے، اور فی الحال \عددی{\overline{Q}} نئے حال تک نہیں پہنچا، لہٰذا یہ بھی بلند ہے؛ یوں بالائی گیٹ کے دونوں مداخل بلند ہیں )۔ دونوں گیٹ، پست حال کی طرف گامزن ہوں گے۔ گیٹوں کے دورانیوں میں فرق (جو وقت اور حالات کے ساتھ تبدیل ہو سکتے ہیں) کی بنا، ایک گیٹ (جو ہم نہیں جانتے کونسا ہو گا) نئے پست حال تک، دوسرے گیٹ سے پہلے پہنچ کر (دوسرے گیٹ کا مداخل ہونے کی وجہ سے) دوسرے گیٹ کو بلند رہنے پر مجبور کرے گا۔ یوں اگرچہ پلٹ کار کے دونوں مداخل غیر فعال کرنے سے پلٹ کار کے مخارج آپس میں تضاد ہیں، تاہم، ہم جاننے سے قاصر ہیں آیا پلٹ کار بلند یا پست حال ہو گا۔ ایس آر پلٹ کار کے دونوں مداخل فعال کرنے کے بعد دوبارہ بیکوقت غیر فعال کرنے سے پلٹ کار کا حال، متمم ضرب گیٹوں کے بیچ نئے حال تک پہنچنے کے دوڑ پر منحصر ہے۔ اسی لئے اس کو \اصطلاح{حالت دوڑ}\فرہنگ{حالت دوڑ}\حاشیہب{race condition}\فرہنگ{race condition} کہتے ہیں۔ ہم پلٹ کار کو حالت دوڑ میں ڈالنے سے گریز کرتے ہیں۔ حالت دوڑ پر حصہ \حوالہ{حصہ_غیر_معاصر_حالت_دوڑ} میں تفصیل سے غور کیا جائے گا۔
 

\begin{figure}
\centering
\begin{subfigure}{0.45\textwidth}
\centering
\begin{otherlanguage}{english}
\begin{tabular}{CC|CR}
\toprule
\overline{S}&\overline{R}&Q&\text{\RL{حال}}\\
\midrule
1&1&1&\text{\RL{بلند}}\\
0&1&1&\text{\RL{بلند رہے گا}}\\
1&1&1&\text{\RL{برقرار}}\\
0&1&1&\text{\RL{بلند رہے گا}}\\
1&0&0&\text{\RL{پست}}\\
1&0&0&\text{\RL{پست رہے گا}}\\
1&1&0&\text{\RL{برقرار}}\\
1&0&0&\text{\RL{پست رہے گا}}\\
1&1&0&\text{\RL{برقرار}}\\
1&1&0&\text{\RL{برقرار}}\\
0&1&1&\text{\RL{بلند}}\\
1&1&1&\text{\RL{برقرار}}\\
\bottomrule
\end{tabular}
\end{otherlanguage}
\end{subfigure}\hfill
\begin{subfigure}{0.45\textwidth}
\centering
\begin{tikzpicture}
\pgfmathsetmacro{\t}{0.3}
\pgfmathsetmacro{\h}{0.5}
\pgfmathsetmacro{\kyspace}{\h+0.75}
\draw[thick](0,\h)--++(\t,0)--++(0,-\h)--++(\t,0)--++(0,\h)--++(\t,0)--++(0,-\h)--++(\t,0)--++(0,\h)--++(6*\t,0)--++(0,-\h)
--++(\t,0)--++(0,\h)--++(\t,0);
\draw[thick](0,-\kyspace+\h)--++(4*\t,0)--++(0,-\h)--++(2*\t,0)--++(0,\h)--++(\t,0)--++(0,-\h)--++(\t,0)--++(0,\h)--++(4*\t,0);
\draw[thick](0,-2*\kyspace+\h)--++(4*\t,0)--++(0,-\h)--++(6*\t,0)--++(0,\h)--++(2*\t,0);
\draw[thick](0,-3*\kyspace)--++(4*\t,0)--++(0,\h)--++(6*\t,0)--++(0,-\h)--++(2*\t,0);
\draw(0,0)node[above left]{$\overline{S}$} (0,-\kyspace)node[above left]{$\overline{R}$}
 (0,-2*\kyspace)node[above left]{$Q$} (0,-3*\kyspace)node[above left]{$\overline{Q}$};
\foreach \x/\l in {0/1,1/0,2/1,3/0,4/1,5/1,6/1,7/1,8/1,9/1,10/0,11/1}{\draw(\x*\t+\t/2,0)node[above]{$\l$};}
\foreach \x/\l in {0/1,1/1,2/1,3/1,4/0,5/0,6/1,7/0,8/1,9/1,10/1,11/1}{\draw(\x*\t+\t/2,-\kyspace)node[above]{$\l$};}
\foreach \x/\l in {0/1,1/1,2/1,3/1,4/0,5/0,6/0,7/0,8/0,9/0,10/1,11/1}{\draw(\x*\t+\t/2,-2*\kyspace)node[above]{$\l$};}
\foreach \x/\l in {0/0,1/0,2/0,3/0,4/1,5/1,6/1,7/1,8/1,9/1,10/0,11/0}{\draw(\x*\t+\t/2,-3*\kyspace)node[above]{$\l$};}
\draw(12*\t+0.25,0)node[right]{$\SI{0}{\volt}$} (12*\t+0.25,\h)node[right]{$\SI{+5}{\volt}$};
\draw(12*\t+0.25,-\kyspace)node[right]{$\SI{0}{\volt}$} (12*\t+0.25,-\kyspace+\h)node[right]{$\SI{+5}{\volt}$};
\draw(12*\t+0.25,-2*\kyspace)node[right]{$\SI{0}{\volt}$} (12*\t+0.25,-2*\kyspace+\h)node[right]{$\SI{+5}{\volt}$};
\draw(12*\t+0.25,-3*\kyspace)node[right]{$\SI{0}{\volt}$} (12*\t+0.25,-3*\kyspace+\h)node[right]{$\SI{+5}{\volt}$};
\end{tikzpicture}
\end{subfigure}
\caption{ایس آر پلٹ کار کے استعمال کا جدول اور ترسیمات۔}
\label{جدول_ترتیبی_ایس_آر_کام}
\end{figure}

شکل \حوالہ{جدول_ترتیبی_ایس_آر_کام} میں پیش جدول کی پہلے صف میں پلٹ کار بلند \عددی{(Q=1)} اور مداخل غیر فعال ہیں۔صف در صف نیچے چلتے ہوئے دیکھیں، مداخل تبدیل کرنے سے پلٹ کار کیا حال اختیار کرتا ہے۔ ( مداخل کسی خاص ترتیب سے نہیں، بلکہ پلٹ کار کی کارکردگی کی ایک مثال دیکھنے کی غرض سے تبدیل کیے گئے۔ )

مثبت منطقی نظام استعمال کرتے ہوئے، \عددی{(1)}کو \عددی{\SI{+5}{\volt}}، جبکہ \عددی{(0)} کو \عددی{\SI{0}{\volt}} سے ظاہر کیا جائے گا۔یوں مداخل ایس کی فعال صورت \عددی{\overline{S}=0} کو \عددی{\SI{0}{\volt}}، جبکہ غیر فعال صورت \عددی{\overline{S}=1} کو \عددی{\SI{+5}{\volt}} سے ظاہر کیا جائے گا۔ اسی طرح \عددی{Q=0} کو \عددی{\SI{0}{\volt}} اور \عددی{Q=1} کو \عددی{\SI{5}{\volt}} سے ظاہر کیا جائے گا۔ ایسا کرتے ہوئے شکل \حوالہ{جدول_ترتیبی_ایس_آر_کام} میں پیش جدول سے اسی شکل میں پیش ترسیمات حاصل ہوں گی، جہاں موازنہ کے لئے \عددی{\overline{Q}} بھی پیش ہے۔




\حصہ{زیادہ مداخل پلٹ کار}
پلٹ کار کے مداخل دو سے زیادہ ہو سکتے ہیں، جیسا شکل \حوالہ{شکل_ترتیبی_متمم_دو_سے_زیادہ} میں دکھایا گیا ہے۔ یہاں بلند کار مداخل کی تعداد دو ہے، جنہیں \عددی{\overline{S}_a} اور \عددی{\overline{S}_b} کہا گیا ہے، جبکہ پست کار مداخل ایک ہے۔ عام طور تینوں مداخل بلند (غیر فعال) رکھے جائیں گے۔ پلٹ کار بلند حال کرنے کی خاطر \عددی{\overline{S}_a} یا \عددی{\overline{S}_b} یا دونوں کو ایک لمحہ کے لئے پست (فعال) کیا جائے گا، جبکہ پلٹ کار پست حال کرنے کی خاطر \عددی{\overline{R}} ایک لمحہ کے لئے فعال کیا جائے گا۔ حال دوڑ سے بچنے کے لئے ضروری ہے کہ \عددی{\overline{R}} کے ساتھ باقی دو مداخل میں سے کوئی ایک (یا دونوں) اکٹھے فعال نہ کیا جائے۔


\begin{figure}
\centering
\begin{subfigure}{0.45\textwidth}
\centering
\begin{tikzpicture}
\pgfmathsetmacro{\kxsep}{2.5}
\pgfmathsetmacro{\kysep}{2}
\pgfmathsetmacro{\kpin}{0.5}
\draw(0,0)node[nand port,scale=1,number inputs=2](u1){};
\draw(0,\kysep)node[nand port,scale=1,number inputs=3](u2){};
\draw(u1.out)--++(\kpin,0)node[right]{$\overline{Q}$};
\draw(u2.out)--++(\kpin,0)node[right]{$Q$};
\draw(u1.out)--++(0,\kpin)coordinate(aa) (u2.out)--++(0,-\kpin)coordinate(bb);
\draw(u1.in 1)--++(0,\kpin)coordinate(cc) (u2.in 3)--++(0,-\kpin)coordinate(dd);
\draw(bb)--(cc) (aa)--(dd);
\draw(u1.in 2)--++(-\kpin,0)node[left]{$\overline{R}$};
\draw(u2.in 1)--++(-\kpin,0)node[left,yshift=0.25em]{$\overline{S}_a$};
\draw(u2.in 2)--++(-\kpin,0)node[left,yshift=-0.25em]{$\overline{S}_b$};
\end{tikzpicture}
\end{subfigure}\hfill
\begin{subfigure}{0.45\textwidth}
\centering
\begin{otherlanguage}{english}
\begin{tabular}{CCC|CC}
\toprule
\overline{S}_a&\overline{S}_b&\overline{R}&Q_{n+1}&\overline{Q}_{n+1}\\
\midrule
0&0&0&?&?\\
0&0&1&1&0\\
0&1&0&?&?\\
0&1&1&1&0\\
1&0&0&?&?\\
1&0&1&1&0\\
1&1&0&0&1\\
1&1&1&Q_n&\overline{Q}_n\\
\bottomrule
\end{tabular}
\end{otherlanguage}
\end{subfigure}
\caption{زیادہ مداخل ایس آر پلٹ کار}
\label{شکل_ترتیبی_متمم_دو_سے_زیادہ}
\end{figure}


\حصہ{قابل مجاز و معذور پلٹ کار}
 شکل \حوالہ{جدول_ترتیبی_ایس_آر_کام} کی ترسیمات سے واضح ہے، مداخل تبدیل کرتے ہی پلٹ کار نیا حال اختیار کرتا ہے۔اس حصہ میں ایسی پلٹ کار پر غور کیا جائے گا جس کے مداخل کو پلٹ کار کے حال پر اثر انداز ہونے سے روکا جا سکتا ہو۔ شکل\حوالہ{شکل_ترتیبی_قابل_مجاز}الف پر غور کریں جہاں دو متمم ضرب گیٹ کے اضافہ سے قابل قابو پلٹ کار حاصل کیا گیا، جس کے (بلند فعال) مداخل \عددی{S} اور \عددی{R} ہیں، جنہیں عام طور غیر فعال (پست) رکھا جاتا ہے۔پلٹ کار کی علامت شکل-ب بھی پیش ہے۔
 
 اضافی گیٹ کے مخارج کو \عددی{\overline{S}_c} اور \عددی{\overline{R}_c} کہا گیا، جبکہ گیٹوں کو قابو کار اشارہ \عددی{C} فراہم کیا گیا۔ مجاز و معذور بنانے والا قابو کار اشارہ \عددی{C} پست (معذور) کرنے سے \عددی{S} اور \عددی{R} مداخل معذور ہوتے ہیں، \عددی{\overline{S}_c} اور \عددی{\overline{R}_c} بلند رہتے ہیں، اور پلٹ کار اپنا حال برقرار رکھتی ہے۔ قابو کار اشارہ بلند (مجاز) کرنے سے پلٹ کار کے مداخل \عددی{S} اور \عددی{R} مجاز ہو کر پلٹ کار کے حال پر اثر انداز ہوتے ہیں۔ 
 
 شکل-ج میں مجاز و معذور قابو کار اشارہ \عددی{C} کی کارکردگی واضح کی گئی۔ جب تک یہ اشارہ پست (معذور) رہے، \عددی{\overline{S}_c} اور \عددی{\overline{R}_c} بلند ہیں۔ اشارہ \عددی{C} بلند کرنے کے بعد \عددی{S} اور \عددی{R} پلٹ کار کا حال تبدیل کرنے کے قابل ہیں۔ یہ پلٹ کار \موٹا{ مجاز و معذور بلند فعال مداخل ایس آر پلٹ کار} کہلاتا ہے۔
 
 
\begin{figure}
\centering
\begin{subfigure}{0.55\textwidth}
\centering
\begin{tikzpicture}
\pgfmathsetmacro{\kxsep}{2.5}
\pgfmathsetmacro{\kysep}{2}
\pgfmathsetmacro{\kpin}{0.5}
\draw(0,0)node[nand port,scale=1,number inputs=2](u1){};
\draw(0,\kysep)node[nand port,scale=1,number inputs=2](u2){};
\draw(u1.out)--++(\kpin,0)node[right]{$\overline{Q}$};
\draw(u2.out)--++(\kpin,0)node[right]{$Q$};
\draw(u1.out)--++(0,\kpin)coordinate(aa) (u2.out)--++(0,-\kpin)coordinate(bb);
\draw(u1.in 1)--++(0,\kpin)coordinate(cc) (u2.in 2)--++(0,-\kpin)coordinate(dd);
\draw(bb)--(cc) (aa)--(dd);
\draw(u1.in 2)--++(-\kpin,0)node[above]{$\overline{R}_c$}node[nand port,scale=1,number inputs=2,anchor=out](u3){};
\draw(u2.in 1)--++(-\kpin,0)node[above]{$\overline{S}_c$}node[nand port,scale=1,number inputs=2,anchor=out](u4){};
\draw(u3.in 2)--++(-\kpin,0)node[left]{$R$};
\draw(u4.in 1)--++(-\kpin,0)node[left]{$S$};
\draw(u3.in 1)--(u4.in 2)coordinate[pos=0.5](cc) (cc)--++(-\kpin,0)node[left]{$C$};
\end{tikzpicture}
\caption{}
\end{subfigure}\hfill
\begin{subfigure}{0.35\textwidth}
\centering
\begin{tikzpicture}
\kSRFF[u1]{0}{0}
\end{tikzpicture}
\caption{}
\end{subfigure}
\begin{subfigure}{1\textwidth}
\centering
\begin{tikzpicture}
\pgfmathsetmacro{\t}{0.4}
\pgfmathsetmacro{\h}{0.5}
\pgfmathsetmacro{\kysep}{\h+0.25}
\foreach \xha/\xhb in {0/2,14/24}{\draw[thick](\xha*\t,\h)--(\xhb*\t,\h);} 
\foreach \xla/\xlb in {2/14,24/26}{\draw[thick](\xla*\t,0)--(\xlb*\t,0);} 
\foreach \xv in {2,14,24}{\draw[thick](\xv*\t,0)--++(0,\h);}
\draw(0,-0*\kysep)node[above left]{$C$};
\foreach \xha/\xhb in {0/1,5/6,7/8,16/17,18/19,22/23,25/26}{\draw[thick](\xha*\t,-1*\kysep+\h)--(\xhb*\t,-1*\kysep+\h);} 
\foreach \xla/\xlb in {1/5,6/7,8/16,17/18,19/22,23/25}{\draw[thick](\xla*\t,-1*\kysep)--(\xlb*\t,-1*\kysep);} 
\foreach \xv in {1,5,6,7,8,16,17,18,19,22,23,25}{\draw[thick](\xv*\t,-1*\kysep)--++(0,\h);}
\draw(0,-1*\kysep)node[above left]{$S$};
\foreach \xha/\xhb in {3/4,9/10,11/12,13/15,20/21}{\draw[thick](\xha*\t,-2*\kysep+\h)--(\xhb*\t,-2*\kysep+\h);}
\foreach \xla/\xlb in {0/3,4/9,10/11,12/13,15/20,21/26}{\draw[thick](\xla*\t,-2*\kysep)--(\xlb*\t,-2*\kysep);}
\foreach \xv in {3,4,9,10,11,12,13,15,20,21}{\draw[thick](\xv*\t,-2*\kysep)--++(0,\h);}
\draw(0,-2*\kysep)node[above left]{$R$};
\foreach \xha/\xhb in {1/16,17/18,19/22,23/26}{\draw[thick](\xha*\t,-3*\kysep+\h)--(\xhb*\t,-3*\kysep+\h);}
\foreach \xla/\xlb in {0/1,16/17,18/19,22/23}{\draw[thick](\xla*\t,-3*\kysep)--(\xlb*\t,-3*\kysep);}
\foreach \xv in {1,16,17,18,19,22,23}{\draw[thick](\xv*\t,-3*\kysep)--++(0,\h);}
\draw(0,-3*\kysep)node[above left]{$\overline{S}_c$};
\foreach \xha/\xhb in {0/14,15/20,21/26}{\draw[thick](\xha*\t,-4*\kysep+\h)--(\xhb*\t,-4*\kysep+\h);}
\foreach \xla/\xlb in {14/15,20/21}{\draw[thick](\xla*\t,-4*\kysep)--(\xlb*\t,-4*\kysep);}
\foreach \xv in {14,15,20,21}{\draw[thick](\xv*\t,-4*\kysep)--++(0,\h);}
\draw(0,-4*\kysep)node[above left]{$\overline{R}_c$};
\foreach \xha/\xhb in {0/14,16/20,22/26}{\draw[thick](\xha*\t,-5*\kysep+\h)--(\xhb*\t,-5*\kysep+\h);}
\foreach \xla/\xlb in {14/16,20/22}{\draw[thick](\xla*\t,-5*\kysep)--(\xlb*\t,-5*\kysep);}
\foreach \xv in {14,16,20,22}{\draw[thick](\xv*\t,-5*\kysep)--++(0,\h);}
\draw(0,-5*\kysep)node[above left]{$Q$};
\foreach \xha/\xhb in {14/16,20/22}{\draw[thick](\xha*\t,-6*\kysep+\h)--(\xhb*\t,-6*\kysep+\h);}
\foreach \xla/\xlb in {0/14,16/20,22/26} {\draw[thick](\xla*\t,-6*\kysep)--(\xlb*\t,-6*\kysep);}
\foreach \xv in {14,16,20,22}{\draw[thick](\xv*\t,-6*\kysep)--++(0,\h);}
\draw(0,-6*\kysep)node[above left]{$\overline{Q}$};
\end{tikzpicture}
\caption{}
\end{subfigure}
\caption{مجاز و معذور بلند فعال مداخل ایس آر پلٹ کار}
\label{شکل_ترتیبی_قابل_مجاز}
\end{figure}
	
بعض اوقات، پلٹ کار کے عمومی مداخل استعمال کیے بغیر، ہم پلٹ کار کا حال خود تعین کرنا چاہتے ہیں۔عموماً، پلٹ کار کا ابتدائی حال منتخب کرنے کے لئے ایسا کرنا درکار ہو گا۔شکل \حوالہ{شکل_ترتیبی_زبردستی_بلند_قابل_مجاز} میں دو مزید مداخل، \عددی{\overline{\text{اٹھ}}} اور \عددی{\overline{\text{بیٹھ}}}، مہیا کئے گئے ہیں، جنہیں پست کر کے پلٹ کار کو بالترتیب زبردستی بلند اور پست کیا جا سکتا ہے۔
\begin{figure}
\centering
\begin{subfigure}{0.55\textwidth}
\centering
\begin{tikzpicture}
\pgfmathsetmacro{\kxsep}{2.5}
\pgfmathsetmacro{\kysep}{2}
\pgfmathsetmacro{\kpin}{0.5}
\draw(0,0)node[nand port,scale=1,number inputs=3](u1){};
\draw(0,\kysep)node[nand port,scale=1,number inputs=3](u2){};
\draw(u1.out)--++(\kpin,0)node[right]{$\overline{Q}$};
\draw(u2.out)--++(\kpin,0)node[right]{$Q$};
\draw(u1.out)--++(0,\kpin)coordinate(aa) (u2.out)--++(0,-\kpin)coordinate(bb);
\draw(u1.in 1)--++(0,\kpin/2)coordinate(cc) (u2.in 3)--++(0,-\kpin/2)coordinate(dd);
\draw(bb)--(cc) (aa)--(dd);
\draw(u1.in 2)--++(-\kpin,0)node[nand port,scale=1,number inputs=2,anchor=out](u3){};
\draw(u2.in 2)--++(-\kpin,0)node[nand port,scale=1,number inputs=2,anchor=out](u4){};
\draw(u3.in 2)--++(-\kpin,0)node[left]{$R$};
\draw(u4.in 1)--++(-\kpin,0)node[left]{$S$};
\draw(u3.in 1)--(u4.in 2)coordinate[pos=0.5](cc) (cc)--++(-\kpin,0)node[left]{$C$};
\draw(u1.in 3)--++(0,-1.5*\kpin)--++(-2*\kpin,0)node[left]{$\overline{\text{\RL{بیٹھ}}}$};
\draw(u2.in 1)--++(0,1.5*\kpin)--++(-2*\kpin,0)node[left]{$\overline{\text{\RL{اٹھ}}}$};
\end{tikzpicture}
\end{subfigure}\hfill
\begin{subfigure}{0.35\textwidth}
\centering
\begin{tikzpicture}
\pgfmathsetmacro{\kshPa}{0.50}
\kSRFF[u1]{0}{0}
\draw(u1pbu)--++(0,\kshPa)node[above]{$\overline{\text{\RL{اٹھ}}}$};
\draw(u1pbd)--++(0,-\kshPa)node[below]{$\overline{\text{\RL{بیٹھ}}}$};
\draw(u1pnu)node[ocirc]{} (u1pnd)node[ocirc]{};
\end{tikzpicture}
\end{subfigure}
\caption{اٹھ بیٹھ صلاحیت پلٹ کار}
\label{شکل_ترتیبی_زبردستی_بلند_قابل_مجاز}
\end{figure}


\حصہ{آقا غلام پلٹ کار} 	
گزشتہ حصہ میں مجاز و معذور بلند فعال مداخل ایس آر پلٹ کار پر غور کیا گیا۔شکل \حوالہ{شکل_ترتیبی_آقا_غلام} میں ایسے دو پلٹ کار ( پہلا آقا اور دوسرا غلام کہلاتا ہے) اور ایک نفی گیٹ سے \اصطلاح{آقا غلام پلٹ کار}\فرہنگ{پلٹ کار!آقا غلام}\حاشیہب{master slave flip flop}\فرہنگ{flip flop!master slave} تشکیل دیا گیا۔آقا کے مخارج، غلام کے مداخل ہیں۔مزید \عددی{C} پر اشارہ \اصطلاح{ساعت}\فرہنگ{ساعت}\حاشیہب{clock}\فرہنگ{clock} مہیا کیا گیا ہے۔
\begin{figure}
\centering
\begin{subfigure}{1\textwidth}
\centering
\begin{tikzpicture}
\pgfmathsetmacro{\kshPa}{0.50}
\pgfmathsetmacro{\kshXX}{2.5}
\pgfmathsetmacro{\kshYY}{0.75}
\kSRFF[u1]{0}{0}
\kSRFF[u2]{\kshXX}{0}
\draw(u1p6)--(u2p1) (u1p4)--(u2p3);
\draw(u1p1)--++(-\kshPa,0)node[left]{$S$} (u1p3)--++(-\kshPa,0)node[left]{$R$};
\draw(u2p6)node[right]{$Q$} (u2p4)node[right]{$\overline{Q}$};
\draw(u1p2)--++(-\kshPa,0)node[left]{$C$};
\draw(0,-\kshYY)node[not port,scale=0.7,anchor=in](u3){};
\draw(u1p2)|-(u3.in 1) (u3.out)-|(u2p2);
\draw(u1p6)node[above]{$Q_a$} (u1p4)node[above]{$\overline{Q}_a$};
\draw(u1-north)node[above]{\text{\RL{آقا}}};
\draw(u2-north)node[above]{\text{\RL{غلام}}};
\end{tikzpicture}
\caption{}
\end{subfigure}
\begin{subfigure}{1\textwidth}
\centering
\begin{tikzpicture}
\pgfmathsetmacro{\t}{0.4}
\pgfmathsetmacro{\h}{0.5}
\pgfmathsetmacro{\dx}{0.1}
\pgfmathsetmacro{\dy}{0.1}
\pgfmathsetmacro{\kysep}{\h+0.25}
\foreach \xha/\xhb in {0/1,2/3,4/5,6/7,8/9,10/11,12/13,14/15,16/17}{\draw[thick](\xha*\t,\h)--(\xhb*\t,\h);} 
\foreach \xla/\xlb in {1/2,3/4,5/6,7/8,9/10,11/12,13/14,15/16,17/18}{\draw[thick](\xla*\t,0)--(\xlb*\t,0);} 
\foreach \xv in {1,2,3,4,5,6,7,8,9,10,11,12,13,14,15,16,17}{\draw[thick](\xv*\t,0)--++(0,\h);}
\foreach \xv in {1,3,5,7,9,11,13,15,17}{\draw[thick](\xv*\t,\h/2)--++(\dx,\dy) (\xv*\t,\h/2)--++(-\dx,\dy);}
\draw(0,-0*\kysep)node[above left]{$C$};
\draw(\t,\h)++(0,0.1)--++(0,0.2)coordinate[pos=0.5](aa)++(2*\t,0)--++(0,-0.2);
\draw[stealth-stealth] (aa)--++(2*\t,0)node[pos=0.5,above]{$T$};
\foreach \xha/\xhb in {0/1.7,6.4/8.6,11.5/14.4}{\draw[thick](\xha*\t,-1*\kysep+\h)--(\xhb*\t,-1*\kysep+\h);} 
\foreach \xla/\xlb in {1.7/6.4,8.6/11.5,14.4/18}{\draw[thick](\xla*\t,-1*\kysep)--(\xlb*\t,-1*\kysep);} 
\foreach \xv in {1.7,6.4,8.6,11.5,14.4}{\draw[thick](\xv*\t,-1*\kysep)--++(0,\h);}
\draw(0,-1*\kysep)node[above left]{$S$};
\foreach \xha/\xhb in {2.6/5.5,9.5/10.8,15.35/15.65,16.5/18}{\draw[thick](\xha*\t,-2*\kysep+\h)--(\xhb*\t,-2*\kysep+\h);}
\foreach \xla/\xlb in {0/2.6,5.5/9.5,10.8/15.35,15.65/16.5}{\draw[thick](\xla*\t,-2*\kysep)--(\xlb*\t,-2*\kysep);}
\foreach \xv in {2.6,5.5,9.5,10.8,15.35,15.65,16.5}{\draw[thick](\xv*\t,-2*\kysep)--++(0,\h);}
\draw(0,-2*\kysep)node[above left]{$R$};
\foreach \xha/\xhb in {0/2.6,6.2/10,12/16.5}{\draw[thick](\xha*\t,-3*\kysep+\h)--(\xhb*\t,-3*\kysep+\h);}
\foreach \xla/\xlb in {2.6/6.2,10/12,16.5/18}{\draw[thick](\xla*\t,-3*\kysep)--(\xlb*\t,-3*\kysep);}
\foreach \xv in {2.6,6.2,10,12,16.5}{\draw[thick](\xv*\t,-3*\kysep)--++(0,\h);}
\draw(0,-3*\kysep)node[above left]{$Q_a$};
\foreach \xha/\xhb in {1/3,7/11,13/17}{\draw[thick](\xha*\t,-4*\kysep+\h)--(\xhb*\t,-4*\kysep+\h);}
\foreach \xla/\xlb in {3/7,11/13,17/18}{\draw[thick](\xla*\t,-4*\kysep)--(\xlb*\t,-4*\kysep);}
\foreach \xv in {3,7,11,13,17}{\draw[thick](\xv*\t,-4*\kysep)--++(0,\h);}
\draw(0,-4*\kysep)node[above left]{$Q$};
\draw[dashed,thick](0,-4*\kysep)--++(\t,0)--++(0,\h) (0,-4*\kysep+\h)--++(\t,0);
\foreach \x in {1,3,5,7,9,11,13,15,17}{\draw[dotted] (\x*\t,0)--++(0,-4*\kysep+\h);}
\end{tikzpicture}
\caption{}
\end{subfigure}
\caption{ساعت کے کنارہ اترائی پر عمل کار آقا غلام پلٹ کار}
\label{شکل_ترتیبی_آقا_غلام}
\end{figure}



 جتنی دیر ساعت \عددی{(C)} بلند رہے، آقا کے مداخل مجاز، لہٰذا مخارج \عددی{Q_a} اور \عددی{\overline{Q}_a} قابل تبدیل ہوں گے۔ غلام کو \عددی{C} کا متمم \عددی{\overline{C}} مجاز و معذور بناتا ہے، لہٰذا جتنی دیر آقا مجاز ہو، غلام معذور (لہٰذا برقرار حال) ہو گا۔
 
جس لمحہ ساعت پست ہو، آقا اسی لمحہ کے حال میں رہ جائے گا، اور غلام مجاز ہو کر فوراً آقا کے مخارج کے مطابق حال اختیار کر لے گا۔یوں، غلام ہر وقت آقا کی پیروی کرتا ہے۔جتنی دیر ساعت پست رہے، \عددی{Q_a} اور \عددی{\overline{Q}_a} تبدیل نہیں ہو سکتے، لہٰذا غلام حال تبدیل نہیں کرے گا۔


 آپ دیکھ سکتے ہیں، غلام پلٹ کار صرف اور صرف ساعت \عددی{(C)} کے کنارہ اترائی پر حال تبدیل کرتا ہے، جس کی وجہ سے یہ \اصطلاح{کنارہ اترائی پر عمل کار آقا غلام پلٹ کار}\فرہنگ{آقا غلام!کنارہ اترائی پر عمل کار}\حاشیہب{negative edge triggered Master Slave flip flop}\فرہنگ{master slave!negative edge triggered} کہلاتا ہے۔ ساعت کے کنارہ اترائی پر تیر کا نشان اس حقیقت کو ظاہر کرتا ہے۔ساعت کا کنارہ (اترائی)، پلٹ کار کی \اصطلاح{لبلبی}\فرہنگ{لبلبی}\حاشیہب{trigger}\فرہنگ{trigger} ہے، جسے پست کرنے سے، پلٹ کار داخلی اشارے کا عکس لیتا ہے۔

 پلٹ کار کو پہلی مرتبہ برقی طاقت فراہم کرنے سے، حال دوڑ پیدا ہو گی جس کے اختتام پر پلٹ کار بلند یا پست ہو گا۔ شکل میں پہلے کنارہ اترائی سے قبل \عددی{Q} مبہم دکھایا گیا ہے (سایہ دار حصہ) ، جو اس حقیقت کو ظاہر کرتا ہے۔ساعت کے اول کنارہ اترائی پر فعال \عددی{S} کے تحت آقا غلام پلٹ کار یقینی طور پر بلند حال اختیار کرتا ہے۔ (شکل \حوالہ{شکل_ترتیبی_زبردستی_بلند_قابل_مجاز} میں اٹھ بیٹھ قابو اشارات اس طرح مبہم صورت سے نمٹنے کے لئے ہیں۔)
 
شکل \حوالہ{شکل_ترتیبی_آقا_غلام} میں ساعت کے آٹھویں کنارہ اترائی کے بعد پست ساعت کے دوران \عددی{R} بلند ہو کر واپس پست ہوتا ہے، جو آقا غلام پلٹ کار کو پست کرنے میں ہرگز کامیاب نہیں ہو گا۔ پلٹ کار کو بلند یا پست کرنے کے لئے، ضروری ہے کہ داخلی اشارات \عددی{S} اور \عددی{R} کسی مخصوص دورانیے سے زیادہ وقت کے لئے فعال ہوں۔ داخلی اشارہ اس صورت کردار ادا کرتا ہے، جب بلند ساعت اس کا عکس محفوظ کر لے۔ ساعت کے پست دورانیہ \عددی{t_L} (شکل \حوالہ{شکل_ترتیبی_ساعت_تفصیل}) سے زیادہ دیر فعال رہنے والا مداخل اشارہ ، ساعت کے کنارہ اترائی کے فوراً بعد فعال ہونے کی صورت میں بھی ساعت کی اگلی بلندی تک فعال رہے گا، لہٰذا آقا غلام پلٹ کار اس پر ضرور عمل کرے گا۔ البتہ، ایسی صورت میں عین ممکن ہے، کنارہ اترائی پر کوئی مداخل فعال نہ ہو (شکل \حوالہ{شکل_ترتیبی_آقا_غلام} میں چھٹا کنارہ اترائی دیکھیں)، لہٰذا، عین کنارہ اترائی کے لمحہ موجود مداخل کا حال محفوظ کرنے کے لئے ضروری ہے کہ مداخل کم از کم ایک دوری عرصہ \عددی{(T)} دورانیے کے لئے فعال رہے (تسلی کر لیں، اگر یقین نہیں)۔ حصہ \حوالہ{حصہ_ترتیبی_حقیقی_ڈی_پلٹ} میں ایسی پلٹ کار پیش کیا جائے گا، جس کے مداخل پر کم از کم ایک دوری عرصہ فعال رہنے کی شرط مسلط نہیں۔

جدول \حوالہ{جدول_ترتیبی_آقا_غلام} میں کنارہ اترائی پر عمل کار آقا غلام پلٹ کار پیش ہے، جہاں ساعت کے کنارہ اترائی پر پلٹ کار (نیا) حال اختیار کرتا ہے۔بلند اور پست ساعت کے دوران، پلٹ کار حال برقرار رکھتا ہے۔
\begin{table}
\caption{کنارہ اترائی پر عمل کار آقا غلام پلٹ کار}
\label{جدول_ترتیبی_آقا_غلام}
\centering
\begin{otherlanguage}{english}
\begin{tabular}{CCC|CC}
\toprule
C&S&R&Q_{n+1}&\overline{Q}_{n+1}\\
\midrule
0&x&x&Q_n&\overline{Q}_n\\
1&x&x&Q_n&\overline{Q}_n\\
\downarrow&0&0&Q_n&\overline{Q}_n\\
\downarrow&0&1&0&1\\
\downarrow&1&0&1&0\\
\downarrow&1&1&?&?\\
\bottomrule
\end{tabular}
\end{otherlanguage}
\end{table}
 

\begin{figure}
\centering
\begin{subfigure}{0.80\textwidth}
\centering
\begin{tikzpicture}
\pgfmathsetmacro{\kxsep}{2.5}
\pgfmathsetmacro{\kysep}{2}
\pgfmathsetmacro{\kpin}{0.5}
\draw(0,0)node[nand port,scale=1,number inputs=3](u1){};
\draw(0,\kysep)node[nand port,scale=1,number inputs=3](u2){};
\draw(u1.out)--++(\kpin,0)node[right]{$\overline{Q}$};
\draw(u2.out)--++(\kpin,0)node[right]{$Q$};
\draw(u1.out)--++(0,\kpin)coordinate(aa) (u2.out)--++(0,-\kpin)coordinate(bb);
\draw(u1.in 1)--++(0,\kpin/2)coordinate(cc) (u2.in 3)--++(0,-\kpin/2)coordinate(dd);
\draw(bb)--(cc) (aa)--(dd);
\draw(u1.in 2)node[nand port,scale=1,number inputs=2,anchor=out](u3){};
\draw(u2.in 2)node[nand port,scale=1,number inputs=2,anchor=out](u4){};
\draw(u3.in 2)--++(-\kpin,0)node[nand port, scale=1,number inputs=3,anchor=out](u5){};
\draw(u4.in 1)--++(-\kpin,0)node[nand port, scale=1,number inputs=3,anchor=out](u6){};
\draw(u5.out)--++(0,\kpin)coordinate(aa) (u6.out)--++(0,-\kpin)coordinate(bb);
\draw(u5.in 1)--++(0,\kpin/2)coordinate(cc) (u6.in 3)--++(0,-\kpin/2)coordinate(dd);
\draw(bb)--(cc) (aa)--(dd);
\draw(u5.in 2)node[nand port,scale=1,number inputs=2,anchor=out](u7){};
\draw(u6.in 2)node[nand port,scale=1,number inputs=2,anchor=out](u8){};
\draw($(u7.in 1)!0.5!(u8.in 2)$) node[not port,scale=0.8,anchor=in](u9){};
\draw(u3.in 1)--(u4.in 2)coordinate[pos=0.5](cc) (cc)--(u9.out);
\draw(u7.in 2)--++(-\kpin,0)node[left]{$R$};
\draw(u8.in 1)--++(-\kpin,0)node[left]{$S$};
\draw(u7.in 1)--(u8.in 2) (u9.in) to [short,*-]++(-\kpin,0)node[left]{$C$};
\draw(u1.in 3)--++(0,-1.5*\kpin)coordinate(ksit)--($(u7.in 1)!(ksit)!(u8.in 1)$)coordinate(ksitL)
node[left]{$\overline{\text{\RL{بیٹھ}}}$};
\draw(u2.in 1)--++(0,1.5*\kpin)coordinate(kstand)--($(u7.in 1)!(kstand)!(u8.in 1)$)coordinate(kstandL)
node[left]{$\overline{\text{\RL{اٹھ}}}$};
\draw(u5.in 3)--($(ksit)!(u5.in 3)!(ksitL)$);
\draw(u6.in 1)--($(kstand)!(u6.in 1)!(kstandL)$);
\end{tikzpicture}
\end{subfigure}\hfill
\begin{subfigure}{0.20\textwidth}
\centering
\begin{tikzpicture}
\pgfmathsetmacro{\kshPa}{0.4}
\kSRFF[u1]{0}{0}
\draw(u1pbu)--++(0,\kshPa)node[above]{$\overline{\text{اٹھ}}$};
\draw(u1pnu)node[ocirc]{};
\draw(u1pbd)--++(0,-\kshPa)node[below]{$\overline{\text{بیٹھ}}$};
\draw(u1pnd)node[ocirc]{};
\draw(u1pn2)node[ocirc]{};
\end{tikzpicture}
\end{subfigure}
\caption{اٹھ بیٹھ صلاحیت رکھنے اور منفی کنارے پر عمل کرنے والا آقا غلام پلٹ کار}
\label{شکل_ترتیبی_اٹھ_بیٹھ_آقا_غلام}
\end{figure}

بعض اوقات، پلٹ کار کا حال، کنارہ ساعت کا انتظار کیے بغیر، تبدیل کرنا درکار ہو گا۔شکل \حوالہ{شکل_ترتیبی_اٹھ_بیٹھ_آقا_غلام} میں (درکار مقامات پر تین مداخل متمم ضرب گیٹ استعمال کرتے ہوئے) آقا غلام پلٹ کار میں پست فعال مداخل \عددی{\overline{\text{اٹھ}}} اور \عددی{\overline{\text{بیٹھ}}} کا اضافہ کر کے ایسی پلٹ کار تشکیل دیا گیا ہے۔ (برقی تاروں کی تعداد بہت بڑھ گئی ہے۔ بہتر ہو گا صفحہ \حوالہصفحہ{شکل_بوولین_برقی_تار_جوڑ} پر شکل \حوالہ{شکل_بوولین_برقی_تار_جوڑ} ایک مرتبہ دوبارہ دیکھیں۔) عام طور انہیں غیر فعال رکھا جائے گا، البتہ، جب ضرورت پیش آئے، انہیں استعمال کرتے ہوئے، ساعت کے کنارہ اترائی کا انتظار کیے بغیر، پلٹ کار کا حال مرضی کے مطابق منتخب کیا جا سکے گا۔

شکل میں \موٹا{منفی کنارے پر عمل کرنے، اور اٹھ بیٹھ صلاحیت کے ، آقا غلام پلٹ کار} کی علامت بھی پیش ہے، جہاں ساعت \عددی{(C)} پر گول دائرہ \موٹا{منفی}، اور تکون \موٹا{کنارے} کو ظاہر کرتا ہے ۔یوں اس سے مراد \قول{ساعت کے منفی کنارے پر عمل پیرا ہونا} لیا جائے گا۔


\حصہ{ ڈی پلٹ کار} 
\جزوحصہ{آقا غلام پلٹ کار سے حاصل کردہ ڈی پلٹ کار}
آقا غلام پلٹ کار کے ساتھ نفی گیٹ منسلک کرکے \اصطلاح{ڈی پلٹ کار}\فرہنگ{پلٹ کار!ڈی}\حاشیہب{D FF}\فرہنگ{FF!D} حاصل کیا جاتا ہے، جو شکل \حوالہ{شکل_ترتیبی_ڈی_پلٹ} میں پیش ہے۔ پلٹ کار کی علامت میں \عددی{C} واضح طور نہیں لکھا گیا، چونکہ علامت پر داخلی جانب گول دائرہ اور تکون ساعت کے منفی کنارہ کو ظاہر کرتے ہیں (مثبت کنارہ، صرف تکون سے ظاہر کیا جاتا ہے)۔ مداخل \عددی{D} پر کم از کم ایک دوری عرصہ \عددی{(T)} بلند یا پست رہنے کی شرط مسلط ہے۔

 پلٹ کار کی کارکردگی کا جدول بھی شکل \حوالہ{شکل_ترتیبی_ڈی_پلٹ} میں پیش ہے،جس کے تحت، بلند یا پست ساعت کے دوران، مداخل \عددی{D}، پلٹ کار کے حال پر اثر انداز نہیں ہو گا۔ پلٹ کار (صرف) ساعت کے کنارہ اترائی پر \عددی{D} دیکھ کر (نیا) حال اختیار کرتا ہے۔ یوں اس کا نام \اصطلاح{کنارہ اترائی پر عمل کار ڈی پلٹ کار}\فرہنگ{پلٹ کار!ڈی، کنارہ اترائی لبلبی}\حاشیہب{negative edge triggered, D flip flop}\فرہنگ{flip flop!D, negative edge} ہو گا۔ ساعت کو نفی گیٹ سے گزار کر \اصطلاح{کنارہ چڑھائی پر عمل کار ڈی پلٹ کار}\فرہنگ{پلٹ کار!ڈی، کنارہ چڑھائی لبلبی}\حاشیہب{positive edge triggered, D flip flop}\فرہنگ{flip flop!D, positive edge} حاصل ہو گا۔

\begin{figure}
\centering
\begin{subfigure}{0.40\textwidth}
\centering
\begin{tikzpicture}
\pgfmathsetmacro{\kshPa}{0.4}
\kSRFF[u1]{0}{0}
\draw(u1pbu)--++(0,\kshPa)node[above]{$\overline{\text{اٹھ}}$};
\draw(u1pnu)node[ocirc]{};
\draw(u1pbd)--++(0,-\kshPa)node[below]{$\overline{\text{بیٹھ}}$};
\draw(u1pnd)node[ocirc]{};
\draw(u1pn2)node[ocirc]{};
\draw(u1p3)node[not port,scale=1,number inputs=1,anchor=out](u2){} (u2.in)|-coordinate(ss)(u1p1) (ss)--++(-\kshPa,0)coordinate(aal)node[left]{$D$};
\draw(u1p2)node[left]{$C$};
\end{tikzpicture}
\end{subfigure}\hfill
\begin{subfigure}{0.35\textwidth}
\centering
\begin{tikzpicture}
\pgfmathsetmacro{\kshPa}{0.4}
\kDFF[u1]{0}{0}
\draw(u1pbu)--++(0,\kshPa)node[above]{$\overline{\text{اٹھ}}$};
\draw(u1pnu)node[ocirc]{};
\draw(u1pbd)--++(0,-\kshPa)node[below]{$\overline{\text{بیٹھ}}$};
\draw(u1pnd)node[ocirc]{};
\draw(u1pn2)node[ocirc]{};
\end{tikzpicture}
\end{subfigure}\hfill
\begin{subfigure}{0.25\textwidth}
\centering
\begin{otherlanguage}{english}
\begin{tabular}{CC|C}
\toprule
C&D&Q_{n+1}\\
\midrule
0&x&Q_n\\
1&x&Q_n\\
\downarrow&0&0\\
\downarrow&1&1\\
\bottomrule
\end{tabular}
\end{otherlanguage}
\end{subfigure}
\caption{آقا غلام سے حاصل ڈی پلٹ کار}
\label{شکل_ترتیبی_ڈی_پلٹ}
\end{figure}



 شکل \حوالہ{شکل_ترتیبی_ڈی_پلٹ_اوقات} میں ڈی پلٹ کار کی کارکردگی کی مثال پیش ہے۔ آقا غلام پلٹ کار کے \عددی{R} مداخل سے چھٹکارا حاصل کرنے کی بدولت، ڈی پلٹ کار کسی صورت \قول{حال دوڑ} سے دو چار نہیں ہو گا۔ساعت کے اول کنارہ اترائی سے قبل، پلٹ کار کا حال مبہم ہے، جس کو سیاہ کر کے (بلند و پست دونوں) دکھایا گیا ہے۔
 \begin{figure}
 \centering
 \begin{otherlanguage}{english}
 \begin{tikztimingtable}[%
timing/.style={x=4ex,y=3ex},
timing/rowdist=5ex,
every node/.style={inner sep=0,outer sep=0},
timing/c/arrow tip=latex, %and this set the style
timing/c/falling arrows,
timing/slope=0.0, %0.1 is good
thick,
]
$C$& cCN(A1)CCN(A2)CCN(A3)CCN(A4)CCCCN(A5)CCN(A6)0.5C\\
 $\overline{\texturdu{\RL{اٹھ}}}$&4HN(B1)L9H\\
 $\overline{\texturdu{\RL{بیٹھ}}}$&10.5HN(C1)0.25L2.75H0.5H\\
 $D$&2H5L5H2L\\
 $Q$&uUN(E1)2HN(E2)0.5LN(F1)1.5HN(E3)2L3N(E4)3HN(G1)L2N(E5)2HN(E6)0.5L\\
\extracode
\begin{pgfonlayer}{background}
\begin{scope}[semitransparent ,dashed]
%\vertlines[darkgray,dotted]{3.6,7.5,11.5,15.5,19.5,23.5,27.5}
\foreach \n in {1,2,3,4,5,6}\draw(A\n.south)--(E\n.north);
\foreach \n in {1}\draw(B\n.south)--(F\n.north);
\foreach \n in {1}\draw(C\n.south)--(G\n.north);
\end{scope}
\end{pgfonlayer}
\end{tikztimingtable}
\end{otherlanguage}
\caption{کنارہ اترائی پر عمل کار ڈی پلٹ کار کی کارکردگی کی مثال}
\label{شکل_ترتیبی_ڈی_پلٹ_اوقات}
\end{figure}


شکل \حوالہ{شکل_ترتیبی_تعدد_تقسیم_دو} میں \موٹا{کنارہ چڑھائی} پر عمل کار ڈی پلٹ کار کا \عددی{\overline{Q}} مداخل \عددی{D} سے جوڑ کر، پلٹ کار کو ساعت \عددی{(C)} فراہم کی گئی ۔
 شکل-ب میں ساعت کے اول کنارہ چڑھائی پر توجہ دیں۔ یہاں \عددی{\overline{Q}=1} ہے، لہٰذا \عددی{D} بلند ہو گا اور ساعت کے کنارہ چڑھائی پر پلٹ کار اس کا عکس محفوظ کرتے ہوئے بلند حال اختیار کرتی ہے۔ پلٹ کار کا مخارج \عددی{\overline{Q}} کچھ دیر بعد نیا حال \عددی{\overline{Q}=0} اختیار کرے گا، لیکن اس وقت تک ساعت کا کنارہ گزر چکا ہو گا۔ ساعت کے اگلے کنارہ چڑھائی پر \عددی{\overline{Q}=0} دیکھ کر پلٹ کار پست ہو گا۔ آپ دیکھ سکتے ہیں کہ \عددی{Q} (یا \عددی{\overline{Q}}) کا تعدد ساعت کے تعدد کا نصف ہے۔

\begin{figure}
\centering
\begin{subfigure}{0.35\textwidth}
\centering
\begin{tikzpicture}
\pgfmathsetmacro{\kshA}{0.25}
\pgfmathsetmacro{\kshB}{1.5}
\kDFF[u1]{0}{0}
\draw(u1p2)--++(-\kshA,0)node[left]{$C$};
\draw(u1p4)--++(\kshA,0)--++(0,\kshB)-|(u1p1);
\end{tikzpicture}
\caption{}
\end{subfigure}\hfill
\begin{subfigure}{0.55\textwidth}
\centering
 \begin{otherlanguage}{english}
 \begin{tikztimingtable}[%
timing/.style={x=4ex,y=3ex},
timing/rowdist=5ex,
every node/.style={inner sep=0,outer sep=0},
timing/c/arrow tip=latex, %and this set the style
timing/c/rising arrows,
timing/slope=0.1, %0.1 is good
thick,
]
$C$& 0.5CN(A1)CCN(A2)CCN(A3)CCN(A4)C\\
$Q$&0.6C2{H}2{L}2{H}0.9L\\
$\overline{Q}$&0.6H2{L}2{H}2{L}0.9H\\
\extracode
\begin{pgfonlayer}{background}
\begin{scope}[semitransparent ,dashed]
%\vertlines[darkgray,dotted]{3.6,7.5,11.5,15.5,19.5,23.5,27.5}
\foreach \n in {1,2,3,4}{\draw(A\n.south)--(A\n |- row3.south);}
\end{scope}
\end{pgfonlayer}
\end{tikztimingtable}
\end{otherlanguage}
\caption{}
\end{subfigure}
\caption{تعدد دو سے تقسیم کیا گیا}
\label{شکل_ترتیبی_تعدد_تقسیم_دو}
\end{figure}


کنارہ اترائی پر عمل کار پلٹ کار کے استعمال میں اس بات کو یقینی بنانا ضروری ہے کہ مداخل، ساعت کے کنارہ اترائی کے دوران، تبدیل نہ ہو۔حقیقتاً، کنارہ اترائی کے آغاز سے چند لمحات قبل سے لے کر، کنارہ گزرنے کے چند لمحات بعد تک، مداخل \عددی{D} کا برقرار ایک حال میں رہنا ضروری ہے۔ ان لمحات کو بالترتیب \اصطلاح{ دورانیہ تیاری}\فرہنگ{دورانیہ!تیاری}\حاشیہب{setup time}\فرہنگ{time!setup} اور
 \اصطلاح{ دورانیہ ٹھیراؤ}\فرہنگ{دورانیہ!ٹھیراؤ}\حاشیہب{hold time}\فرہنگ{time!hold} کہتے ہیں۔دورانیہ تیاری اور دورانیہ ٹھیراؤ کی معلومات پلٹ کار کے تخلیق کار مہیا کرتے ہیں۔ کنارہ چڑھائی پر عمل کار پلٹ کار کی صورت میں مداخل کو دوران چڑھائی تبدیل نہیں ہونے دیا جاتا۔ 


\حصہ{ڈی پلٹ کار}\شناخت{حصہ_ترتیبی_حقیقی_ڈی_پلٹ}
گزشتہ حصہ میں آقا غلام پلٹ کار سے ڈی پلٹ کار حاصل کیا گیا، جس کے مداخل پر، کم از کم ایک دوری عرصہ دورانیہ کے لئے حال برقرار رکھنے کی شرط مسلط ہے ۔شکل \حوالہ{شکل_ترتیبی_ڈی_پلٹ_بہتر} میں نسبتاً بہتر، (کنارہ چڑھائی پر عمل کار) ڈی پلٹ کار پیش ہے، جو \موٹا{واقعی}، ساعت کے کنارہ چڑھائی پر (نیا )حال اختیار کرتا ہے، اور جو\اصطلاح{ وسیع پیمانہ مخلوط ادوار}\فرہنگ{مخلوط دور!وسیع پیمانہ}\حاشیہب{very large scale integration (VLSI)}\فرہنگ{VLSI} میں با کثرت مستعمل ہے۔

\begin{figure}
\centering
\begin{tikzpicture}
\pgfmathsetmacro{\kxsep}{2.5}
\pgfmathsetmacro{\kysep}{1.5}
\pgfmathsetmacro{\kpin}{0.5}
\pgfmathsetmacro{\kpina}{0.3}
\draw(0,0)node[nand port,scale=1,number inputs=2](u6){$u6$};
\draw(0,\kysep)node[nand port,scale=1,number inputs=2](u5){$u5$};
\draw(u6.out)--++(\kpin/2,0)node[right]{$\overline{Q}$};
\draw(u5.out)--++(\kpin/2,0)node[right]{$Q$};
\draw(u6.out)--++(0,\kpin)coordinate(aa) (u5.out)--++(0,-\kpin)coordinate(bb);
\draw(u6.in 1)--++(0,\kpina)coordinate(cc) (u5.in 2)--++(0,-\kpina)coordinate(dd);
\draw(bb)--(cc) (aa)--(dd);
\draw(u5.in 1)--++(-\kpin,0)node[above right]{$\overline{S}$}
node[nand port,scale=1,number inputs=2,anchor=out](u2){$u2$};
\draw(u2.out)++(0,\kysep)node[nand port,scale=1,number inputs=2,anchor=out](u1){$u1$};
\draw(u6.in 2)--++(-\kpin,0)node[above right]{$\overline{R}$}
node[nand port,scale=1,number inputs=3,anchor=out](u3){$u3$};
\draw(u3.out)++(0,-\kysep)node[nand port,scale=1,number inputs=2,anchor=out](u4){$u4$};
\draw(u2.out)--++(0,\kpin)coordinate(aa) (u1.out)--++(0,-\kpin)coordinate(bb);
\draw(u2.in 1)--++(0,\kpina)coordinate(cc) (u1.in 2)--++(0,-\kpina)coordinate(dd);
\draw(bb)--(cc) (aa)--(dd);
\draw(u4.out)--++(0,\kpin)coordinate(aa) (u3.out)--++(0,-\kpin)coordinate(bb);
\draw(u4.in 1)--++(0,\kpina)coordinate(cc) (u3.in 3)--++(0,-\kpina)coordinate(dd);
\draw(bb)--(cc) (aa)--(dd);
\draw(u1.in 1)--++(0,1.5*\kpina)coordinate(uu)--($(u5.out)!(uu)!(u6.out)$)--++(2*\kpin,0)|-(u4.out);
\draw(u3.in 1)--++(0,\kpina)coordinate(aa) (u2.out) to [short,*-]++(0,-\kpina)--(aa); 
\draw(u2.in 2)--++(-\kpin,0)coordinate(upup)|-coordinate(lolo)(u3.in 2) ($(upup)!0.5!(lolo)$)--++(-\kpin,0)node[left]{$C$};
\draw(u4.in 2)--++(-2*\kpin,0)node[left]{$D$};
\end{tikzpicture}
\caption{کنارہ چڑھائی پر عمل کار ڈی پلٹ کار}
\label{شکل_ترتیبی_ڈی_پلٹ_بہتر}
\end{figure}


اس پلٹ کار کی بناوٹ میں تین ایس آر پلٹ کار مستعمل ہیں۔ گیٹ \عددی{u1}، \عددی{u2} ایک ایس آر، گیٹ \عددی{u3}، \عددی{u4} دوسرا، اور گیٹ \عددی{u5}، \عددی{u6} تیسرا ایس آر پلٹ کار تشکیل دیتے ہیں۔ تیسرا ایس آر پلٹ کار خارجی ہے جو \عددی{\overline{S}} اور \عددی{\overline{R}} کے مطابق مخارج\عددی{Q} اور \عددی{\overline{Q}} فراہم کرتا ہے۔ برقرار حال کے لئے \عددی{\overline{S}=1} اور \عددی{\overline{R}=1} درکار ہے،
 \عددی{\overline{S}=0} اور \عددی{\overline{R}=1} بلند حال ، جبکہ \عددی{\overline{S}=1} اور \عددی{\overline{R}=0} پست حال دے گا، اور \عددی{\overline{S}=0} اور \عددی{\overline{R}=0} ممنوعہ ہے۔ مداخل \عددی{\overline{S}} اور \عددی{\overline{R}} باقی دو ایس آر پلٹ کار پر منحصر ہیں، جنہیں بیرونی اشارات \عددی{D} (مواد) اور \عددی{C} (ساعت) تعین کرتے ہے۔
\begin{figure}
\centering
\begin{subfigure}{0.45\textwidth}
\centering
\begin{tikzpicture}
\pgfmathsetmacro{\kxsep}{2.5}
\pgfmathsetmacro{\kysep}{1.5}
\pgfmathsetmacro{\kpin}{0.5}
\pgfmathsetmacro{\kpina}{0.3}
\draw(0,0)node[nand port,scale=1,number inputs=2,anchor=out](u2){$u2$};
\draw(0,\kysep)node[nand port,scale=1,number inputs=2,anchor=out](u1){$u1$};
\draw(0,-\kysep)node[nand port,scale=1,number inputs=3,anchor=out](u3){$u3$};
\draw(0,-2*\kysep)node[nand port,scale=1,number inputs=2,anchor=out](u4){$u4$};
\draw(u2.out)--++(0,\kpin)coordinate(aa) (u1.out)--++(0,-\kpin)coordinate(bb);
\draw(u2.in 1)--++(0,\kpina)coordinate(cc) (u1.in 2)--++(0,-\kpina)coordinate(dd);
\draw(bb)--(cc) (aa)--(dd);
\draw(u4.out)--++(0,\kpin)coordinate(aa) (u3.out)--++(0,-\kpin)coordinate(bb);
\draw(u4.in 1)--++(0,\kpina)coordinate(cc) (u3.in 3)--++(0,-\kpina)coordinate(dd);
\draw(bb)--(cc) (aa)--(dd);
\draw(u1.in 1)--++(0,1.5*\kpina)coordinate(uu)--($(u5.out)!(uu)!(u6.out)$)--++(2*\kpin,0)|-(u4.out);
\draw(u3.in 1)--++(0,\kpina)coordinate(aa) (u2.out) to [short,*-]++(0,-\kpina)--(aa); 
\draw(u2.in 2)--++(-\kpin,0)coordinate(upup)|-coordinate(lolo)(u3.in 2) ($(upup)!0.5!(lolo)$)--++(-\kpin,0)node[left]{$C=0$};
\draw(u4.in 2)--++(-2*\kpin,0)node[left]{$D=0$};
\draw(u2.out)--++(\kpin,0)node[right]{$\overline{S}$};
\draw(u3.out)--++(\kpin,0)node[right]{$\overline{R}$};
\draw(u1.out)node[right]{$0$};
\draw(u2.out)node[above right]{$1$};
\draw(u3.out)node[above right]{$1$};
\draw(u4.out)node[above right]{$1$};
\end{tikzpicture}
\caption{پست مواد، پست ساعت}
\end{subfigure}\hfill
\begin{subfigure}{0.45\textwidth}
\centering
\begin{tikzpicture}
\pgfmathsetmacro{\kxsep}{2.5}
\pgfmathsetmacro{\kysep}{1.5}
\pgfmathsetmacro{\kpin}{0.5}
\pgfmathsetmacro{\kpina}{0.3}
\draw(0,0)node[nand port,scale=1,number inputs=2,anchor=out](u2){$u2$};
\draw(0,\kysep)node[nand port,scale=1,number inputs=2,anchor=out](u1){$u1$};
\draw(0,-\kysep)node[nand port,scale=1,number inputs=3,anchor=out](u3){$u3$};
\draw(0,-2*\kysep)node[nand port,scale=1,number inputs=2,anchor=out](u4){$u4$};
\draw(u2.out)--++(0,\kpin)coordinate(aa) (u1.out)--++(0,-\kpin)coordinate(bb);
\draw(u2.in 1)--++(0,\kpina)coordinate(cc) (u1.in 2)--++(0,-\kpina)coordinate(dd);
\draw(bb)--(cc) (aa)--(dd);
\draw(u4.out)--++(0,\kpin)coordinate(aa) (u3.out)--++(0,-\kpin)coordinate(bb);
\draw(u4.in 1)--++(0,\kpina)coordinate(cc) (u3.in 3)--++(0,-\kpina)coordinate(dd);
\draw(bb)--(cc) (aa)--(dd);
\draw(u1.in 1)--++(0,1.5*\kpina)coordinate(uu)--($(u5.out)!(uu)!(u6.out)$)--++(2*\kpin,0)|-(u4.out);
\draw(u3.in 1)--++(0,\kpina)coordinate(aa) (u2.out) to [short,*-]++(0,-\kpina)--(aa); 
\draw(u2.in 2)--++(-\kpin,0)coordinate(upup)|-coordinate(lolo)(u3.in 2) ($(upup)!0.5!(lolo)$)--++(-\kpin,0)node[left]{$C=0$};
\draw(u4.in 2)--++(-2*\kpin,0)node[left]{$D=1$};
\draw(u2.out)--++(\kpin,0)node[right]{$\overline{S}$};
\draw(u3.out)--++(\kpin,0)node[right]{$\overline{R}$};
\draw(u1.out)node[right]{$1$};
\draw(u2.out)node[above right]{$1$};
\draw(u3.out)node[above right]{$1$};
\draw(u4.out)node[above right]{$0$};
\end{tikzpicture}
\caption{بلند مواد، پست ساعت}
\end{subfigure}
\begin{subfigure}{0.45\textwidth}
\centering
\begin{tikzpicture}
\pgfmathsetmacro{\kxsep}{2.5}
\pgfmathsetmacro{\kysep}{1.5}
\pgfmathsetmacro{\kpin}{0.5}
\pgfmathsetmacro{\kpina}{0.3}
\draw(0,0)node[nand port,scale=1,number inputs=2,anchor=out](u2){$u2$};
\draw(0,\kysep)node[nand port,scale=1,number inputs=2,anchor=out](u1){$u1$};
\draw(0,-\kysep)node[nand port,scale=1,number inputs=3,anchor=out](u3){$u3$};
\draw(0,-2*\kysep)node[nand port,scale=1,number inputs=2,anchor=out](u4){$u4$};
\draw(u2.out)--++(0,\kpin)coordinate(aa) (u1.out)--++(0,-\kpin)coordinate(bb);
\draw(u2.in 1)--++(0,\kpina)coordinate(cc) (u1.in 2)--++(0,-\kpina)coordinate(dd);
\draw(bb)--(cc) (aa)--(dd);
\draw(u4.out)--++(0,\kpin)coordinate(aa) (u3.out)--++(0,-\kpin)coordinate(bb);
\draw(u4.in 1)--++(0,\kpina)coordinate(cc) (u3.in 3)--++(0,-\kpina)coordinate(dd);
\draw(bb)--(cc) (aa)--(dd);
\draw(u1.in 1)--++(0,1.5*\kpina)coordinate(uu)--($(u5.out)!(uu)!(u6.out)$)--++(2*\kpin,0)|-(u4.out);
\draw(u3.in 1)--++(0,\kpina)coordinate(aa) (u2.out) to [short,*-]++(0,-\kpina)--(aa); 
\draw(u2.in 2)--++(-\kpin,0)coordinate(upup)|-coordinate(lolo)(u3.in 2) ($(upup)!0.5!(lolo)$)--++(-\kpin,0)node[left]{$C=0$};
\draw(u4.in 2)--++(-2*\kpin,0)node[left]{$D=0$};
\draw(u2.out)--++(\kpin,0)node[right]{$\overline{S}$};
\draw(u3.out)--++(\kpin,0)node[right]{$\overline{R}$};
\draw(u1.out)node[right]{$0$};
\draw(u2.out)node[above right]{$1$};
\draw(u3.out)node[above right]{$0$};
\draw(u4.out)node[above right]{$1$};
\end{tikzpicture}
\caption{پست مواد، بلند ساعت}
\end{subfigure}\hfill
\begin{subfigure}{0.45\textwidth}
\centering
\begin{tikzpicture}
\pgfmathsetmacro{\kxsep}{2.5}
\pgfmathsetmacro{\kysep}{1.5}
\pgfmathsetmacro{\kpin}{0.5}
\pgfmathsetmacro{\kpina}{0.3}
\draw(0,0)node[nand port,scale=1,number inputs=2,anchor=out](u2){$u2$};
\draw(0,\kysep)node[nand port,scale=1,number inputs=2,anchor=out](u1){$u1$};
\draw(0,-\kysep)node[nand port,scale=1,number inputs=3,anchor=out](u3){$u3$};
\draw(0,-2*\kysep)node[nand port,scale=1,number inputs=2,anchor=out](u4){$u4$};
\draw(u2.out)--++(0,\kpin)coordinate(aa) (u1.out)--++(0,-\kpin)coordinate(bb);
\draw(u2.in 1)--++(0,\kpina)coordinate(cc) (u1.in 2)--++(0,-\kpina)coordinate(dd);
\draw(bb)--(cc) (aa)--(dd);
\draw(u4.out)--++(0,\kpin)coordinate(aa) (u3.out)--++(0,-\kpin)coordinate(bb);
\draw(u4.in 1)--++(0,\kpina)coordinate(cc) (u3.in 3)--++(0,-\kpina)coordinate(dd);
\draw(bb)--(cc) (aa)--(dd);
\draw(u1.in 1)--++(0,1.5*\kpina)coordinate(uu)--($(u5.out)!(uu)!(u6.out)$)--++(2*\kpin,0)|-(u4.out);
\draw(u3.in 1)--++(0,\kpina)coordinate(aa) (u2.out) to [short,*-]++(0,-\kpina)--(aa); 
\draw(u2.in 2)--++(-\kpin,0)coordinate(upup)|-coordinate(lolo)(u3.in 2) ($(upup)!0.5!(lolo)$)--++(-\kpin,0)node[left]{$C=0$};
\draw(u4.in 2)--++(-2*\kpin,0)node[left]{$D=1$};
\draw(u2.out)--++(\kpin,0)node[right]{$\overline{S}$};
\draw(u3.out)--++(\kpin,0)node[right]{$\overline{R}$};
\draw(u1.out)node[right]{$1$};
\draw(u2.out)node[above right]{$0$};
\draw(u3.out)node[above right]{$1$};
\draw(u4.out)node[above right]{$0$};
\end{tikzpicture}
\caption{بلند مواد، بلند ساعت}
\end{subfigure}
\caption{کنارہ چڑھائی پر عمل کار ڈی پلٹ کار کی کارکردگی۔}
\label{شکل_ترتیبی_ڈی_پلٹ_کام}
\end{figure}

شکل \حوالہ{شکل_ترتیبی_ڈی_پلٹ_کام} میں دور کی کارکردگی کی وضاحت کی گئی ہے، جہاں صرف گیٹ \عددی{u1} تا \عددی{u4} کو دکھاتے ہوئے تمام (چار) ممکنہ صورتیں پیش کی گئی ہیں۔گیٹ \عددی{u2} اور \عددی{u3} کے مخارج \عددی{\overline{S}} اور \عددی{\overline{R}} شکل \حوالہ{شکل_ترتیبی_ڈی_پلٹ_بہتر} کے گیٹ \عددی{u5} اور \عددی{u6} کے ساتھ جڑے ہیں، جو ڈی پلٹ کار کے مخارج \عددی{Q} اور \عددی{\overline{Q}} مہیا کرتے ہیں۔ شکل \حوالہ{شکل_ترتیبی_ڈی_پلٹ_کام} -الف اور ب میں پست ساعت \عددی{(C=0)} کی صورت میں \عددی{D=0} اور \عددی{D=1} کے لئے گیٹوں کے ثنائی مخارج پیش ہیں۔ دونوں اشکال میں \عددی{C=0} کی بدولت \عددی{u2} اور \عددی{u3} کے مخارج ، \عددی{D} کی قیمت سے قطع نظر، بلند ہوں گے، لہٰذا \عددی{\overline{S}=1} اور \عددی{\overline{R}=1} ہو گا، جس کے تحت \عددی{u5}،\عددی{u6} (پر مبنی تیسرا) پلٹ کار برقرار حال ہو گا۔ جب \عددی{D=0} ہو، \عددی{u4} کا مخارج \عددی{1} ہو گا، جو (\عددی{u2} کے بلند مخارج کے ساتھ مل کر) \عددی{u1} کا مخارج \عددی{0} کرے گا۔ جب \عددی{D=1} ہو، (چونکہ \عددی{u3} بلند ہے لہٰذا)\عددی{u4} پست \عددی{(0)} ہو گا، جس کی بنا پر \عددی{u1} بلند \عددی{(1)} ہو گا۔ ساعت \عددی{0} کی صورت میں، جو \عددی{D} سے قطع نظر ڈی پلٹ کار برقرار حال رکھتا ہے، یہی دو ممکنات پائے جاتے ہیں۔


کنارہ چڑھائی سے قبل ایک غیر مبہم وقت کے لئے، جو \ترچھا{دورانیہ تیاری} کہلاتا ہے، مداخل \عددی{D} کی قیمت لازماً مستقل رکھنی ہو گی۔ دورانیہ تیاری گیٹ \عددی{u4} اور \عددی{u1} کے دورانیہ رد عمل کا مجموعہ ہے، چونکہ \عددی{D} میں تبدیلی ان گیٹوں کے مخارج پر اثر انداز ہو تی ہے۔ اب فرض کریں دورانیہ تیاری میں \عددی{D} تبدیل نہیں ہوتا، جبکہ ساعت (پست حال سے) بلند \عددی{(1)} ہوتا ہے۔ یہ صورت شکل \حوالہ{شکل_ترتیبی_ڈی_پلٹ_کام}-ج اور د میں پیش ہے۔اگر \عددی{C=1} ہونے کے لمحے پر \عددی{D=0} ہو، تب \عددی{\overline{S}=1} رہتا ہے، جبکہ \عددی{\overline{R}} تبدیل ہو کر \عددی{0} ہو جائے گا (شکل-ج)۔ یوں (شکل \حوالہ{شکل_ترتیبی_ڈی_پلٹ_بہتر} میں) ڈی پلٹ کار کا مخارج \عددی{Q} پست \عددی{(0)} حال اختیار کرے گا۔ اب اگر \عددی{C=1} (یعنی بلند حال) کے دوران، \عددی{D} کی قیمت تبدیل ہو، (\عددی{\overline{R}} کی بدولت جو \عددی{0} ہے) \عددی{u4} بلند \عددی{(1)} رہے گا۔ گیٹ \عددی{u4} صرف اس وقت حال تبدیل کر سکتا ہے جب ساعت دوبارہ پست \عددی{(0)} ہو؛ لیکن اس وقت \عددی{\overline{S}} اور \عددی{\overline{R}} دونوں \عددی{1} ہوں گے، اور ڈی پلٹ کار برقرار حال ہو گا۔ البتہ، ساعت کے کنارہ چڑھائی کے بعد ایک غیر مبہم دورانیہ کے لئے، جو \ترچھا{دورانیہ ٹھیراؤ} کہلاتا ہے، \عددی{D} کی قیمت تبدیل نہیں ہونی چاہیے۔ دورانیہ ٹھیراؤ گیٹ \عددی{u3} کے دورانیہ رد عمل کے برابر ہے، چونکہ ، \عددی{D} کی قیمت سے قطع نظر، \عددی{u4} کا مخارج \عددی{1} پر رکھنے کے لئے \عددی{\overline{R}} کا \عددی{0} ہونا لازمی ہے۔
 
 اگر \عددی{C=1} ہونے کے لمحے پر \عددی{D=1} ہو، تب \عددی{\overline{S}} تبدیل ہو کر \عددی{0} ہو گا، جبکہ \عددی{R} کی قیمت \عددی{1} رہے گی (شکل-د)، جس کی بنا پر (شکل \حوالہ{شکل_ترتیبی_ڈی_پلٹ_بہتر} میں) ڈی پلٹ کار کا مخارج \عددی{Q} بلند \عددی{(1)} ہو گا۔ بلند ساعت \عددی{(C=1)} کے دوران، \عددی{D} کی تبدیلی \عددی{\overline{S}} اور \عددی{\overline{R}} پر اثر انداز نہیں ہو گی، چونکہ \عددی{\overline{S}} پست \عددی{(0)} ہے جو \عددی{u1} کو \عددی{1} رکھے گا۔ جب \عددی{C} واپس \عددی{0} ہو، \عددی{\overline{S}} اور \عددی{\overline{R}} دونوں \عددی{1} حال اختیار کر کے \عددی{Q} برقرار رکھیں گے۔
 
 خلاصہ کچھ یوں ہے۔ ساعت کے کنارہ چڑھائی پر \عددی{D} کی قیمت \عددی{Q} کو منتقل ہوتی ہے۔ بلند ساعت کے دوران \عددی{D} میں تبدیلیاں \عددی{Q} پر اثر انداز نہیں ہوتیں۔مزید، ساعت کا کنارہ اترائی اور پست ساعت، \عددی{Q} پر اثر انداز نہیں ہوتے۔
 
 اشارہ \عددی{D=0} گیٹ \عددی{u4} اور \عددی{u1} سے گزر کر \عددی{u1} کو پست کرتا ہے، جو \عددی{u2} کو بلند کیے رکھتا ہے۔یوں ساعت کے کنارہ چڑھائی سے (\عددی{u4} اور \عددی{u1} کے مجموعی دورانیہ رد عمل کے برابر وقفہ) دورانیہ تیاری کے برابر وقت قبل، ضروری ہے کہ\عددی{D} کی قیمت مستقل صورت اختیار کر لے۔اسی طرح \عددی{\overline{R}=0} جو (\عددی{D} کی قیمت سے قطع نظر) \عددی{u4} کو بلند کیے رکھتا ہے، کے لئے ضروری ہے کہ \عددی{D} کی قیمت کنارہ چڑھائی کے بعد دورانیہ ٹھیراؤ (جو \عددی{u3} کے دورانیہ رد عمل کے برابر ہے) کے لئے تبدیل نہ ہو۔
 
 آقا غلام پلٹ کار کی طرح، کنارہ پر عمل کار پلٹ کار، ترتیبی ادوار میں باز رسی کے مسائل سے چھٹکارا دیتا ہے۔ اس قسم کا ڈی پلٹ کار استعمال کرتے وقت دورانیہ تیاری اور دورانیہ ٹھیراؤ پر توجہ دینی ہو گی۔ 

ترتیبی ادوار میں مختلف پلٹ کار استعمال کرتے وقت، اس بات کو یقینی بنائیں کہ تمام پلٹ کار بیکوقت (یعنی تمام پلٹ کار ساعت کے کنارہ اترائی پر یا تمام پلٹ کار کنارہ چڑھائی پر) حال تبدیل کرتے ہوں۔ وہ پلٹ کار جو منتخب کنارہ کے مخالف کنارے پر حال تبدیل کرتے ہوں، کی ساعت نفی گیٹ سے گزار کر ، منتخب کنارے کے ہم عصر بنایا جا سکتا ہے۔


\ابتدا{مشق}
انٹرنیٹ سےڈی پلٹ کار کے معلوماتی صفحات اتاریں۔(ا) اس مخلوط دور میں کتنے ڈی پلٹ کار ہیں؟ (ب) یہ پلٹ کار ساعت کے کس کنارے پر عمل کار ہے؟
\انتہا{مشق}


\حصہ{جے کے پلٹ کار } 
ڈی پلٹ کار استعمال کر کے مختلف اقسام کے پلٹ کار تشکیل دیے جا سکتے ہیں، جن میں \اصطلاح{جےکے پلٹ کار}\فرہنگ{پلٹ کار!جے کے}\حاشیہب{JK FF}\فرہنگ{flip flop!JK} اور \اصطلاح{ٹی پلٹ کار}\فرہنگ{پلٹ کار!ٹی}\حاشیہب{T FF}\فرہنگ{flip flop!T} بہت مقبول ہیں۔ساعت کے کنارہ چڑھائی پر عمل کار جے کے پلٹ کار کی بناوٹ شکل \حوالہ{شکل_ترتیبی_جے_کے_پلٹ} میں، اور کارکردگی جدول \حوالہ{جدول_ترتیبی_جے_کے_مداخل_مساوات}-ب میں پیش ہے۔ کنارہ اترائی پر عمل کار جے کے پلٹ کار بھی پایا جاتا ہے۔

شکل میں مداخل \عددی{D} ذیل ہو گا، جہاں پلٹ کار کے موجودہ مخارج \عددی{Q_n} اور \عددی{\overline{Q}_n} لکھے گئے ہیں۔ 
\begin{align}\label{مساوات_ترتیبی_جے_کے_مداخل}
D=J\overline{Q}_n+\overline{K}Q_n
\end{align}
ساعت کے اگلے کنارہ چڑھائی پر ڈی پلٹ کار اس مداخل کے تحت حال اختیار کرتا ہے،لہٰذا جے کے پلٹ کار کی کارکردگی کی مساوات درج ذیل ہو گی، جہاں موجودہ مخارج \عددی{Q_n} اور اگلا \عددی{Q_{n+1}} ہے۔
\begin{align}\label{مساوات_ترتیبی_جے_کے_کارکردگی}
Q_{n+1}=J\overline{Q}_n+\overline{K}Q_n
\end{align}
مساوات \حوالہ{مساوات_ترتیبی_جے_کے_مداخل} کو جدول \حوالہ{جدول_ترتیبی_جے_کے_مداخل_مساوات}-الف میں پیش کیا گیا ہے۔جدول کی پہلی صف میں پلٹ کار کا موجودہ حال \عددی{Q_n=0} ، اور مداخل \عددی{J=0} اور \عددی{K=0} ہیں، لہٰذا مساوات \حوالہ{مساوات_ترتیبی_جے_کے_مداخل} کے تحت \عددی{D=0} ہو گا۔یوں ساعت کے اگلے کنارہ چڑھائی پر پلٹ کار پست حال اختیار کرتے ہوئے موجودہ حال برقرار رکھتا ہے۔ جدول کی دوسری صف میں موجودہ حال \عددی{Q_n=1} جبکہ مداخل \عددی{J=0} اور \عددی{K=0} ہیں، جن سے \عددی{D=1} حاصل ہو گا، لہٰذا ساعت کے اگلے کنارہ چڑھائی پر پلٹ کار بلند حال اختیار کرتے ہوئے موجودہ حال برقرار رکھتا ہے۔

آپ نے دیکھا کہ \عددی{J=0}، \عددی{K=0} کی صورت میں پلٹ کار برقرار حال \عددی{(Q_{n+1}=Q_n)} ہو گا۔ جدول کے اضافی خانے میں یہ معلومات درج کی گئی ہے۔ تسلی کر لیں (اگلے مشق میں ایسا کرنے کو کہا گیا ہے) کہ جدول میں \عددی{D} اور \عددی{Q_{n+1}} کی تمام معلومات مساوات \حوالہ{مساوات_ترتیبی_جے_کے_مداخل} کے عین مطابق ہیں۔اس جدول کی بہتر صورت جدول-ب ہے، جہاں غیر ضروری معلومات روپوش کی گئی ، اور کنارہ چڑھائی کی معلومات فراہم کی گئی۔

\موٹا{جے کےپلٹ کار کی کارکردگی} درج ذیل ہے۔
\begin{align}\label{مساوات_ترتیبی_جے_کے_عام_فہم_کارکردگی}
\begin{array}{c|cr}
JK&Q_{n+1}&\\
\hline
00&Q_n&\text{\RL{برقرار حال}}\\
01&0&\text{\RL{پست حال}}\\
10&1&\text{\RL{بلند حال}}\\
11&\overline{Q}_n&\text{\RL{متمم حال}}
\end{array}
\end{align}
اس مساوات کی پہلی تین صورتوں میں، \عددی{J} اور \عددی{K} بالترتیب \عددی{S} اور \عددی{R} مداخل کا کردار ادا کرتے ہیں، یعنی فعال \عددی{J}، پلٹ کار کو (ساعت کے عمل کار کنارہ پر) بلند حال، اور فعال \عددی{K} اسے پست حال کرتا ہے۔البتہ یہاں دونوں مداخل فعال ہونے کی اجازت ہے، جو حال متمم کرتے ہیں۔ دونوں مداخل غیر فعال ہونے کی صورت میں پلٹ کار موجودہ حال برقرار رکھتا ہے۔

\begin{figure}
\centering
\begin{subfigure}{0.7\textwidth}
\centering
\begin{tikzpicture}
\pgfmathsetmacro{\kshPa}{0.50}
\pgfmathsetmacro{\kshPb}{0.25}
\pgfmathsetmacro{\kshY}{0.5}
\kDFF[u5]{0}{0}
\draw(u5p2)node[left]{$C$};
\draw(u5p1)--++(-\kshPb,0)node[nor port,scale=1,number inputs=2,anchor=out](u1){$u1$};
\draw(u1.in 1)--++(0,\kshY)node[nand port,scale=1,number inputs=2,anchor=out](u2){$u2$};
\draw(u1.in 2)--++(0,-\kshY)node[nand port,scale=1,number inputs=2,anchor=out](u3){$u3$};
\draw(u3.in 1)node[not port,scale=1,anchor=out](u4){} (u4.in)node[left]{$K$};
\draw(u2.in 2)--(u2.in 2 -| u4.in)node[left]{$J$};
\draw(u5p6)--++(2*\kshPb,0)node[right]{$Q$}coordinate[pos=0.5](QQ);
\draw(u5p4)--++(2*\kshPb,0)node[right]{$\overline{Q}$};
\draw(u3.in 2)--++(0,-1*\kshPa)-|(QQ);
\draw(u2.in 1)--++(0,\kshPa)-|(u5p4);
\draw($(u4.in)!0.4!(u4.out)$)node[font=\small]{$u4$};
\draw(u5-south-west)node[above left]{$u_5$};
\end{tikzpicture}
\end{subfigure}\hfill
\begin{subfigure}{0.30\textwidth}
\centering
\begin{tikzpicture}
\kJKFF[u1]{0}{0}
\end{tikzpicture}
\end{subfigure}
\caption{جے کے پلٹ کار کی بناوٹ اور علامت۔}
\label{شکل_ترتیبی_جے_کے_پلٹ}
\end{figure}


\begin{table}
\caption{کنارہ چڑھائی پر عمل کار جے کے پلٹ کار}
\label{جدول_ترتیبی_جے_کے_مداخل_مساوات}
\centering
\begin{subtable}[t]{0.45\textwidth}
\caption{}
\centering
\begin{otherlanguage}{english}
\begin{tabular}{CCC|C|C}
\toprule
J&K&Q_n&D&Q_{n+1}\\
\midrule
0&0&0&0&\multirow{2}{*}{$Q_n$}\\
0&0&1&1&\\
\midrule
0&1&0&0&\multirow{2}{*}{$0$}\\
0&1&1&0&\\
\midrule
1&0&0&1&\multirow{2}{*}{$1$}\\
1&0&1&1&\\
\midrule
1&1&0&1&\multirow{2}{*}{$\overline{Q}_n$}\\
1&1&1&0&\\
\bottomrule
\end{tabular}
\end{otherlanguage}
\end{subtable}\hfill
\begin{subtable}[t]{0.45\textwidth}
\caption{}
\centering
\begin{otherlanguage}{english}
\begin{tabular}{CCC|Cr}
\toprule
C&J&K&Q_{n+1}&\\
\midrule
\uparrow&0&0&Q_n&\text{\RL{برقرار حال}}\\
\uparrow&0&1&0&\text{\RL{پست حال}}\\
\uparrow&1&0&1&\text{\RL{بلند حال}}\\
\uparrow&1&1&\overline{Q}_n&\text{\RL{متمم حال}}\\
\bottomrule
\end{tabular}
\end{otherlanguage}
\end{subtable}
\end{table}


\ابتدا{مشق}
جدول \حوالہ{جدول_ترتیبی_جے_کے_مداخل_مساوات}-الف اور ب کی تصدیق کریں۔
\انتہا{مشق}


\جزوحصہ{ٹی پلٹ کار}
جے کے پلٹ کار کے دونوں مداخل آپس میں جوڑنے سے \اصطلاح{ٹی پلٹ کار}\فرہنگ{پلٹ کار!ٹی}\حاشیہب{T FF}\فرہنگ{FF!T} حاصل ہو گا، جو شکل \حوالہ{شکل_ترتیبی_ٹی_پلٹ} میں بمع علامت اور جدول پیش ہے۔

پست مداخل \عددی{(T=0)} کی صورت میں ٹی پلٹ کار برقرار حال رہے گا، جبکہ بلند مداخل \عددی{(T=1)} کی صورت میں ساعت کے کنارہ چڑھائی پر متمم حال اختیار کرے گی۔ یوں بلند \عددی{T} کی صورت میں بلند پلٹ کار اگلے کنارہ چڑھائی پر پست ہوگا، جبکہ پست پلٹ کار اگلے کنارہ چڑھائی پر بلند ہو گا۔ 

ٹی پلٹ کار کی مساوات ، جے کے پلٹ کار کی مساوات \حوالہ{مساوات_ترتیبی_جے_کے_کارکردگی} سے حاصل کرتے ہیں۔
\begin{gather}
\begin{aligned}\label{مساوات_ترتیبی_ٹی_پلٹ_کی_مساوات}
Q_{n+1}&=J\overline{Q}_n+\overline{K}Q_n\\
&=T\overline{Q}_n+\overline{T}Q_n\\
&=T\oplus Q_n
\end{aligned}
\end{gather}
مساوات کے حصول میں \عددی{J} اور \عددی{K} دونوں کی جگہ \عددی{T} استعمال کیا گیا۔
\begin{figure}
\centering
\begin{subfigure}{0.30\textwidth}
\centering
\begin{tikzpicture}
\pgfmathsetmacro{\kshPa}{0.5}
\kJKFF[u1]{0}{0}
\draw(u1p3)--(u1p1)--++(-\kshPa,0)node[left]{$T$};
\draw(u1p2)--++(-\kshPa,0)node[left]{$C$};
\end{tikzpicture}
\end{subfigure}\hfill
\begin{subfigure}{0.30\textwidth}
\centering
\begin{tikzpicture}
\kTFF[u1]{0}{0}
\end{tikzpicture}
\end{subfigure}\hfill
\begin{subfigure}{0.30\textwidth}
\centering
\begin{otherlanguage}{english}
\begin{tabular}{CC|C}
\toprule
C&T&Q_{n+1}\\
\midrule
0&x&Q_n\\
1&x&Q_n\\
\uparrow&0&Q_n\\
\uparrow&1&\overline{Q}_n\\
\bottomrule
\end{tabular}
\end{otherlanguage}
\end{subfigure}
\caption{ٹی پلٹ کار کی بناوٹ اور علامت}
\label{شکل_ترتیبی_ٹی_پلٹ}
\end{figure}

\ابتدا{مشق}
ٹی پلٹ کار کے جدول کی تصدیق کریں۔
\انتہا{مشق}
\ابتدا{مشق}
انٹرنیٹ سے \عددی{74xx} اور \عددی{40xx} سلسلہ میں جے کے اور ٹی پلٹ کار تلاش کریں۔
\انتہا{مشق}


 
 
\حصہ{ثنائی گنت کار}
شکل \حوالہ{شکل_ترتیبی_تعدد_تقسیم_دو} میں پیش دور تین مرتبہ استعمال کر کے شکل \حوالہ{شکل_ترتیبی_ثنائی_گنت_کار} حاصل ہو گا۔ بائیں جانب سے اول پلٹ کار \عددی{(u_0)} کا مخارج \عددی{Q_0}، دوم پلٹ کار کا مخارج \عددی{Q_1} اور \عددی{u_2} کا مخارج \عددی{Q_2} پکارا گیا ہے۔

 پلٹ کار \عددی{u_0} ساعت \عددی{(C)} کا تعدد \عددی{2} سے تقسیم کرتا ہے۔ اس کے دونوں مخارج شکل میں پیش ہیں، جو ساعت کے کنارہ چڑھائی پر حال تبدیل کرتے ہیں، اور جن کا تعدد \عددی{C} کے تعدد کا نصف ہے۔ اشارہ \عددی{\overline{Q}_0} پلٹ کار \عددی{u_1} کو بطور ساعت مہیا کیا گیا ہے، جس کو \عددی{u_1} دو سے تقسیم کر تا ہے۔ یوں \عددی{Q_1} کا تعدد \عددی{C} کے تعدد سے \عددی{4} گنّا کم ہو گا۔ پلٹ کار \عددی{u_1} کا مخارج \عددی{\overline{Q}_1}، تیسرے پلٹ کار کی ساعت ہے جو اسے \عددی{2} سے تقسیم کرے گا، لہٰذا \عددی{Q_2} کا تعدد \عددی{C} کے تعدد سے \عددی{8} گنّا کم ہو گا۔

\begin{figure}
\centering
\begin{subfigure}{1\textwidth}
\centering
\begin{tikzpicture}
\pgfmathsetmacro{\kdisX}{2.5}
\pgfmathsetmacro{\kdisPa}{0.5}
\pgfmathsetmacro{\kdisPb}{1.0}
\kDFF[u0]{0}{0}
\kDFF[u1]{\kdisX}{0}
\kDFF[u2]{2*\kdisX}{0}
\draw(u0p4)--++(\kdisPa,0)|-coordinate(mid)(u1p2) (mid)--++(0,\kdisPb)-|(u0p1);
\draw(u1p4)--++(\kdisPa,0)|-coordinate(mid)(u2p2) (mid)--++(0,\kdisPb)-|(u1p1);
\draw(u2p4)--++(\kdisPa,0)coordinate(lst)--($(u0p2)!(lst)!(u2p2)$)--++(0,\kdisPb)-|(u2p1);
\draw(u0p2)node[left]{$C$} (u0p6)node[above]{$Q_0$} (u1p6)node[above]{$Q_1$} (u2p6)node[above]{$Q_2$};
\draw(u0-south-west)node[above left]{$u_0$} (u1-south-west)node[above left]{$u_1$}
 (u2-south-west)node[above left]{$u_2$};
\end{tikzpicture}
\end{subfigure}
\begin{subfigure}{1\textwidth}
\centering
\begin{otherlanguage}{english}
 \begin{tikztimingtable}[%
timing/.style={x=2ex,y=3ex},
timing/rowdist=4ex,
every node/.style={inner sep=0,outer sep=0},
timing/c/arrow tip=latex, %and this set the style
timing/c/rising arrows,
timing/slope=0, %0.1 is good
timing/dslope=0.1,
thick,
]
%\tikztimingmetachar{R}{[|/utils/exec=\setcounter{new}{0}|]}
%\usetikztiminglibrary[new={char=Q,reset char=R}]{counters}
%[timing/counter/new={char=c, base=2,digits=3,max value=7, wraps ,text style={font=\normalsize}}] 12{2c} \\ 
$C$& H22{C}\\
$Q_0$&5{LLHH}LLH\\
$\overline{Q}_0$&HH10{2C}C\\
$Q_1$&2{4L4H}4L3H\\
$\overline{Q}_1$&HHHH2{4C4C}3C\\
$Q_2$&{8L8H}7L\\
$\overline{Q}_2$&8H8C7C\\
$\small Q_2Q_1Q_0$&[timing/counter/new={char=c, base=2,digits=3,max value=7, wraps ,text style={font=\normalsize}}] 12{2c} \\ 
\extracode
\begin{pgfonlayer}{background}
\begin{scope}[semitransparent ,dashed]
%\vertlines[darkgray,dotted]{3.6,7.5,11.5,15.5,19.5,23.5,27.5}
\foreach \n in {1,3,...,11} \draw(4*\n ex-2ex,-4ex+1.25ex)node[]{$0$};
\foreach \n in {2,4,...,12} \draw(4*\n ex-2ex,-4ex+1.25ex)node[]{$1$};
\foreach \n in {1,2,5,6,9,10} \draw(4*\n ex-2ex,-12 ex+1.25ex)node[]{$0$};
\foreach \n in {3,4,7,8,11,12} \draw(4*\n ex-2ex,-12 ex+1.25ex)node[]{$1$};
\foreach \n in {1,2,3,4,9,10,11,12} \draw(4*\n ex-2ex,-20 ex+1.25ex)node[]{$0$};
\foreach \n in {5,6,7,8} \draw(4*\n ex-2ex,-20 ex+1.25ex)node[]{$1$};
\draw(4*4 ex-2ex,-12 ex+1.25ex) circle (0.25cm and 1.75cm);
\draw(4*4 ex-2ex,-7*\rowdist+1.25ex) circle (0.25cm and 0.5cm);
%\foreach \n in {1}\draw(B\n.south)--(F\n.north);
%\foreach \n in {1}\draw(C\n.south)--(G\n.north);
\end{scope}
\end{pgfonlayer}
\end{tikztimingtable}
\end{otherlanguage}
\end{subfigure}
\caption{تین ہندسی ثنائی گنت کار}
\label{شکل_ترتیبی_ثنائی_گنت_کار}
\end{figure}

 پلٹ کار کے مخارج، ثنائی عدد کے تین ہندسے تصور کر کے، \عددی{Q_2Q_1Q_0} روپ میں لکھیں۔ شکل \حوالہ{شکل_ترتیبی_ثنائی_گنت_کار} کے آخری صف میں یہ عدد پیش ہے، جہاں تینوں پلٹ کار ابتدائی طور پست تصور کیے گئے۔ نقطہ دار گھیرے میں \عددی{Q_0=1} (بلند)، \عددی{Q_1=1} (بلند)،اور \عددی{Q_2=0} (پست) ہیں جنہیں \عددی{Q_2Q_1Q_0=011} لکھا پیش کیا گیا ہے، جو اعشاری تین کے برابر ہے۔ یہ دور ساعت کا کنارہ چڑھائی، (تین ہندسی ثنائی عدد کے روپ میں) گنتا ہے، جس کی بنا پر اس کا نام \اصطلاح{تین ہندسی، ثنائی گنت کار}\فرہنگ{گنت کار!ثنائی تین ہندسی}\حاشیہب{three bit binary counter}\فرہنگ{counter!binary, three bit} ہے۔ 
 
گنت کار صفر \عددی{(000_2)} تا سات \عددی{(111_2)} (یعنی آٹھ، \عددی{2^3}، کنارے) گنتی کرنے کے بعد دوبارہ صفر \عددی{(000_2)} سے شروع کرتا ہے۔ ساعت \عددی{C} کی بجائے گنت کار کو کوئی بھی عددی اشارہ گنتی کے لئے فراہم کیا جا سکتا ہے۔ گنت کار اشارے کے کنارہ چڑھائی کی گنتی کر کے نتیجہ مہیا کرے گا۔

ڈی پلٹ کار کی تعداد \عددی{4} کر کے، سولہ \عددی{(2^4=16)} کنارے گننے کے قابل گنت کار بنایا جا سکتا ہے جو صفر \عددی{(0000_2)} تا پندرہ \عددی{(1111_2)} گنتی کرے گا۔ یوں \عددی{n} پلٹ کار پر مشتمل ثنائی گنت کار \عددی{2^n} کنارے گننے کے قابل ہو گا۔ 



\حصہ{سلسلہ وار ثنائی جمع کار}
شکل \حوالہ{شکل_ترتیبی_ثنائی_سلسلہ_وار_جمع_کار} میں مکمل جمع کار \عددی{(u_1)} اور ڈی پلٹ کار \عددی{(u_2)} کی مدد سے اصطلاح{سلسلہ وار ثنائی جمع کار}\فرہنگ{جمع کار!ثنائی سلسلہ وار}\حاشیہب{binary serial counter}\فرہنگ{counter!binary,serial} تشکیل دیا گیا ہے (مکمل جمع کار کی ڈبہ علامت کو یوں بنایا گیا ہے کہ دور میں صفائی پیدا ہو)۔مکمل جمع کار کو جمع کرنے والے دو ثنائی اعداد \عددی{x} اور \عددی{y} سلسلہ وار فراہم کئے جاتے ہیں۔کمتر رتبی بٹ سے شروع کر کے ساعت کے ہر کنارہ چڑھائی پر دونوں اعداد کے اگلے بِٹ فراہم کئے جاتے ہیں۔کسی بھی قدم پر ڈی پلٹ کار حاصل جمع (یعنی مکمل جمع کا خارجی حاصل) ذخیرہ کر کے اگلے قدم پر مکمل جمع کو بطور داخلی حاصل مہیا کرتا ہے۔مجموعہ کے حصول سے قبل ڈی پلٹ کار زبردستی پست کیا جاتا ہے تا کہ پہلا داخلی حاصل صفر ہو۔آپ دیکھ سکتے ہیں کہ \عددی{s} پر سلسلہ وار دونوں ثنائی اعداد کا مجموعہ خارج ہو گا۔ 
\begin{figure}
\centering
\begin{tikzpicture}
\kfulladder[u1]{0}{0}
\kDFF[u2]{-0.5}{-2.25}
\draw(u2pbd)--(u2pd);
\draw(u2pnd)node[ocirc]{};
\draw(u2p1)|-(u1co);
\draw(u2p6)-|(u1ci);
\draw(u2p2)node[left]{$C$};
\draw(u1s)node[right]{$s$};
\draw(u1y)node[above]{$y$} (u1z)node[above]{$z$};
\draw(u2pd)--++(\kpin,0)node[right]{$\overline{\text{بیٹھ}}$};
\end{tikzpicture}
\caption{سلسلہ وار ثنائی جمع کار}
\label{شکل_ترتیبی_ثنائی_سلسلہ_وار_جمع_کار}
\end{figure}

اس باب کے آخر میں آپ سے گزارش کی جائے گی کہ سلسلہ وار ثنائی جمع کار استعمال کرتے ہوئے دو ثنائی اعداد جمع کریں۔

\حصہ{معاصر ترتیبی ادوار کا تجزیہ}
ساعت پر عمل کار، پلٹ کار پر مبنی ادوار \اصطلاح{معاصر ترتیبی ادوار}\فرہنگ{معاصر!ترتیبی ادوار}\حاشیہب{synchronous sequential circuits}\فرہنگ{synchronous!sequential circuits} کہلاتے ہیں، جو پلٹ کار کے موجودہ حال اور مداخل دیکھ کر نئے حال اختیار کرتے ہیں۔معاصر ترتیبی ادوار، عموماً، کنارہ ساعت کے ساتھ قدم ملا کر چلتے ہیں۔ہم زیادہ تر کنارہ ساعت پر عمل کار ترتیبی ادوار پر تبصرہ کریں گے ( جو متن سے واضح ہو گا)۔معاصر ترتیبی ادوار میں ترکیبی حصے کا موجود ہونا لازم نہیں۔

کنارہ پر عمل کار معاصر ترتیبی ادوار کنارہ ساعت پر نیا حال اختیار کرتے ہیں۔موجودہ حال نئے حال پر اثر انداز ہو سکتا ہے، لہٰذا نئے حال دریافت کرتے وقت موجودہ حال (کو بھی) مداخل تصور کریں۔ترکیبی ادوار کی طرح ترتیبی ادوار کا جدول، جو \اصطلاح{حال کا جدول}\فرہنگ{حال کا جدول}\حاشیہب{state table}\فرہنگ{state!table} کہلاتا ہے، نئے حال دریافت کرنے میں مدد گار ثابت ہوگا۔نیا حال \اصطلاح{مساوات حال}\فرہنگ{حال!مساوات}\حاشیہب{state equation}\فرہنگ{state!equation} سے بھی حاصل کیا جا سکتا ہے۔ دونوں طریقوں پر غور مثالوں کی مدد سے کرتے ہیں۔

\جزوحصہ{مساوات حال}
 دور کے موجودہ حال اور موجودہ مداخل کے روپ میں، مساوات حال دور کے اگلے حال بیان کرتی ہیں۔کنارہ ساعت پر دور اگلے (نئے) حال اختیار کرتا ہے۔یوں، ساعت کے \عددی{n} کنارے گزرنے کے بعد حال کو موجودہ حال تصور کر کے، اس کے لئے اشاریہ \عددی{n} استعمال کرتے ہوئے ،مثلاً \عددی{Q(n)}، اگلا حال \عددی{Q(n+1)} ہو گا۔

\begin{figure}
\centering
\begin{tikzpicture}
\pgfmathsetmacro{\kdisX}{2}
\pgfmathsetmacro{\kdisY}{2.5}
\pgfmathsetmacro{\shY}{0.5}
\pgfmathsetmacro{\shPa}{0.5}
\pgfmathsetmacro{\shPb}{0.75}
\pgfmathsetmacro{\shPc}{1.5}
\kDFF[u1]{0}{0}
\kDFF[u2]{0}{-\kdisY}
\draw(u1-south-west)node[above right]{\small $u1$};
\draw(u2-south-west)node[above right]{\small $u2$};
\draw(u1p1)--++(-4*\shPa,0)node[pos=0.5,above]{$\overline{xQ_0+x\overline{Q}_1}$}node[nor port, scale=1,number inputs=2,anchor=out](u3){u3};
\draw(u3.in 1)--++(0,\shY)node[above left]{$x Q_0$}--++(-2*\shPa,0)node[and port,scale=1,number inputs=2,anchor=out](u4){u4};
\draw(u3.in 2)--++(0,-\shY)node[above left]{$x \overline{Q}_1$}--++(-2*\shPa,0)node[and port,scale=1,number inputs=2,anchor=out](u5){u5};
\draw(u2p1)--++(-4*\shPa,0)node[above right,xshift=-0.5ex]{$\overline{\overline{Q}_0 Q_1}$}node[nand port,scale=1,number inputs=2,anchor=out](u6){u6};
\draw(0,-1*\kdisY-0.75)node[nor port, scale=1,number inputs=3,anchor=in 1](u7){u7};
\draw(u1p2)--(u2p2)--++(0,-\shPa)--++(-\shPa/2,0)node[left]{$C$};
\draw(u2p6)--++(0,\shPb)-|(u6.in 1);
\draw(u2p4)--++(0,-\shPb)coordinate(yyy)-|(u5.in 2);
\draw(u7.in 1)--(u7.in 1|-yyy);
\draw(u1p6)--++(0,\shPc)-|(u4.in 1);
\draw(u5.in 1)--++(-\shPa,0)node[left]{$x$}coordinate[pos=0.5](xxStart) (xxStart)node[circle,fill,inner sep=1pt]{}|-(u4.in 2) (xxStart)|-(u7.in 3);
\draw(u6.in 2)--++(-\shPa,0)coordinate(qq)node[circle,fill,inner sep=1pt]{}|-(u7.in 2) (qq)--++(0,\shPc-0.25)-|(u1p4);
\draw(u1p6)--++(\shPa,0)node[right]{$Q_0$};
\draw(u1p4)--++(\shPa,0)node[right]{$\overline{Q}_0$};
\draw(u2p6)--++(\shPa,0)node[right]{$Q_1$};
\draw(u2p4)--++(\shPa,0)coordinate(rgt)node[right]{$\overline{Q}_1$};
\draw(u7.out)--(u7.out-|rgt)node[right]{$y$}node[below right,shift={(-2ex,-1ex)}]
{$\overline{x+\overline{Q}_0+\overline{Q}_1}$};
\end{tikzpicture}
\caption{ترتیبی دور بطور مثال}
\label{شکل_ترتیبی_بطور_مثال}
\end{figure}
 شکل \حوالہ{شکل_ترتیبی_بطور_مثال} مثال بنا کر آگے بڑھتے ہیں، جہاں کنارہ چڑھائی پر عمل کار ڈی پلٹ کار مستعمل ہیں۔ موجودہ مداخل \عددی{x(n)} جبکہ موجودہ مخارج \عددی{Q_0(n)} اور \عددی{Q_1(n)} ہیں۔ان تینوں کو مداخل تصور کر کے \عددی{D_0} کی ترکیبی مساوات لکھتے ہیں۔ضرب گیٹ \عددی{u4} کا مخارج \عددی{xQ_0} اور \عددی{u5} کا \عددی{x\overline{Q}_1} ہے، جو متمم جمع \عددی{u3} کے مداخل ہیں، لہٰذا (بالائی پلٹ کار کا مداخل) \عددی{D_0} جو \عددی{u3} کا مخارج ہے، ان کے منطقی جمع کا متمم ہو گا۔
\begin{align*}
D_0(n)=\overline{x(n) Q_0(n) +x(n) \overline{Q}_1(n)} 
\end{align*}
اس مساوات میں ہر جزو کے ساتھ \عددی{(n)} چسپاں کر کے واضح کیا گیا کہ یہ موجودہ متغیرات ہیں۔ساعت کے کنارہ چڑھائی پر ی \عددی{u1} اس مساوات کے مطابق اگلا حال اختیار کرے گا۔یوں، نیا حال \عددی{Q_0(n+1)} درج ذیل ہو گا۔
\begin{align}\label{مساوات_ترتیبی_مثال_ترتیبی_دور_الف}
Q_0(n+1)=\overline{x(n) Q_0(n) +x(n) \overline{Q}_1(n)} 
\end{align}
اسی طرح متمم ضرب \عددی{u6} کے مداخل \عددی{\overline{Q}_0}، \عددی{Q_1} لہٰذا مخارج \عددی{\overline{\overline{Q}_0Q_1}} ہو گا، جو پلٹ کار \عددی{u2} کا مداخل \عددی{D_1} ہے۔ یوں اس پلٹ کار کا اگلا حال درج ذیل ہو گا۔
\begin{align}\label{مساوات_ترتیبی_مثال_ترتیبی_دور_ب}
Q_1(n+1)=\overline{\overline{Q}_0(n)Q_1(n)}
\end{align}
تیسرا مخارج \عددی{y} ہے جو متمم جمع \عددی{u7} کا مخارج \عددی{\overline{x+\overline{Q}_0+\overline{Q}_1}} ہے، اور جو ساعت کا تابع نہیں، لہٰذا \عددی{y} صرف موجودہ حال اور مداخل پر منحصر ہے، یعنی یہ ہر صورت موجودہ مخارج ہو گا۔
\begin{align}\label{مساوات_ترتیبی_مثال_ترتیبی_دور_پ}
y(n)=\overline{x(n)+\overline{Q}_0(n)+\overline{Q}_1(n)}
\end{align}
مساوات \حوالہ{مساوات_ترتیبی_مثال_ترتیبی_دور_الف} تا \حوالہ{مساوات_ترتیبی_مثال_ترتیبی_دور_پ} میں بار بار \عددی{(n)} اور \عددی{(n+1)} لکھنے سے گریز کرتے ہوئے درج ذیل لکھا جا سکتا ہے۔
\begin{gather}
\begin{aligned}\label{مساوات_ترتیبی_مثال_ترتیبی_دور_ت}
Q_0&=\overline{x Q_0 +x \overline{Q}_1} \\
Q_1&=\overline{\overline{Q}_0Q_1}\\
y&=\overline{x+\overline{Q}_0+\overline{Q}_1}
\end{aligned}
\end{gather}


\جزوحصہ{حال کا جدول}
معاصر حال جدول میں لکھے جا سکتے ہیں۔شکل \حوالہ{شکل_ترتیبی_بطور_مثال} کی مثال آگے بڑھاتے ہوئے مساوات \حوالہ{مساوات_ترتیبی_مثال_ترتیبی_دور_ت} سے جدول لکھتے ہیں۔موجودہ مداخل \عددی{(x)} اور موجودہ حال (\عددی{Q_0}، \عددی{Q_1}) آزاد متغیرات، جبکہ اگلے مخارج اور حال تابع متغیرات تصور کریں۔یوں \عددی{x(n)} ، \عددی{Q_0(n)} ، اور \عددی{Q_1(n)} آزاد متغیر تصور کر کے ان کی تمام ترتیب (\عددی{000_2} تا \عددی{111_2}) لکھیں۔مساوات \حوالہ{مساوات_ترتیبی_مثال_ترتیبی_دور_ت} سے ہر ترتیب کے مطابقتی اگلے حال \عددی{Q_0(n+1)}، \عددی{Q_1(n+1)}، اور اگلے مخارج \عددی{y(n)} حاصل کر کے جدول میں درج کریں۔یوں جدول \حوالہ{جدول_ترتیبی_جدول_حال_برائے_بطور_مثال} حاصل ہو گا، جو \اصطلاح{حال کا جدول}\فرہنگ{حال کا جدول}\حاشیہب{state table}\فرہنگ{state!table} کہلاتا ہے۔

\begin{table}
\caption{حال کا جدول ( برائے مساوات \حوالہ{مساوات_ترتیبی_مثال_ترتیبی_دور_ت})}
\label{جدول_ترتیبی_جدول_حال_برائے_بطور_مثال}
\centering
\begin{otherlanguage}{english}
\begin{tabular}{CCCCCC}
\toprule
\text{\RL{موجودہ حال}}& \multicolumn{2}{c}{\text{\RL{اگلا حال}}} && \multicolumn{2}{c}{\text{\RL{موجودہ مخارج}}}\\
\cline{2-3} \cline{5-6}
 & x=0&x=1&\phantom{x}&x=0&x=1\\
\midrule
Q_1Q_0&Q_1Q_0&Q_1Q_0&&y&y\\
\midrule
00&11&10&&0&0\\
01&11&10&&0&0\\
10&01&01&&0&0\\
11&11&10&&1&0\\
\bottomrule
\end{tabular}
\end{otherlanguage}
\end{table}

\جزوحصہ{ حال کا خاکہ}\شناخت{حصہ_ترتیبی_خاکہ_حال}
  حال  کے جدول میں موجود معلومات کا خاکہ بنایا جا سکتا ہے جو \اصطلاح{حال کا خاکہ}\فرہنگ{حال کا خاکہ}\حاشیہب{state diagram}\فرہنگ{state!diagram} کہلاتا ہے۔ جدول \حوالہ{جدول_ترتیبی_جدول_حال_برائے_بطور_مثال} کا حال کا خاکہ شکل \حوالہ{شکل_ترتیبی_خاکہ_حال} میں پیش ہے۔ 
 
  حال  کے خاکہ میں دور کا حال گول دائروں سے ظاہر کیا جاتا ہے، جبکہ موجودہ حال سے اگلے حال منتقلی تیر دار لکیر سے ظاہر کی جاتی ہے، جس کی دم موجودہ حال پر اور سر اگلے حال پر رکھا جاتا ہے۔ تیر دار لکیر پر دو اعداد لکھے جاتے ہیں، جن کے بیچ ترچھی لکیر کھینچی جاتی ہے۔ وہ داخلی قیمت جو انتقال کا سبب بنتی ہے، ترچھی لکیر کے اوپر اور موجودہ مخارج نیچے لکھا جاتا ہے۔
 
 شکل \حوالہ{شکل_ترتیبی_بطور_مثال} کے ترتیبی دور میں دو پلٹ کار مستعمل ہیں، جن کا حال \عددی{Q_1Q_0} لکھ کر \عددی{00}، \عددی{01}، \عددی{10}، اور \عددی{11} ممکن حال ہیں۔حال \عددی{00} سے \عددی{10} انتقال کی تیردار لکیر پر \عددی{1/0} لکھا گیا ہے، جس کے تحت انتقال \عددی{x=1} کی بدولت پیش آیا اور \عددی{y=0} ہے۔
 
حال کا خاکہ دیکھ کر کئی حقائق با آسانی واضح ہوں گے۔ مثلاً، خاکہ دیکھ کر واضح ہے یہ دور کسی دوسرے حال سے \عددی{00} منتقل نہیں ہو گا؛ حال \عددی{10} سے یہ اگلے قدم میں \عددی{01} منتقل ہو گا، جس کے بعد جب تک \عددی{x=1} رہے حال تبدیل نہیں ہو گا اور \عددی{x=0} کرنے سے حال \عددی{11} حاصل ہو گا، جس سے نکلنے کا کوئی راستہ موجود نہیں۔

حال کا خاکہ اور حال کا جدول ایک ہی معلومات دو مختلف طریقوں سے پیش کرتے ہیں۔ دونوں میں پیش معلومات ہر طرح یکساں ہے۔ 
\begin{figure}
\centering
\begin{tikzpicture}
\pgfmathsetmacro{\rad}{0.30}
\draw(0,0)node[]{$00$} node[draw,thick,circle,inner sep=0.25cm](aa){} (2,1.5)node[]{$10$}node[draw,thick,circle,inner sep=0.25cm](ba){} (3,0)node[]{$01$}node[draw,thick,circle,inner sep=0.25cm](ab){} (1,-1)node[]{$11$} node[draw,thick,circle,inner sep=0.25cm](bb){};
\draw[-stealth](aa) to [out=90,in=180]node[pos=0.5,above left]{$1/0$}(ba);
\draw[-stealth](aa) to [out=-90,in=180]node[pos=0.5,below left]{$0/0$}(bb);
\draw[-stealth](ba) to [out=-90,in=180]node[pos=0.5,below left]{$0/0$}(ab);
\draw[-stealth](ba) to [out=0,in=90]node[pos=0.5,above right]{$1/0$}(ab);
\draw[-stealth](ab) to [out=45,in=90] ++(1,0)node[right]{$1/0$} to [out=-90,in=-45] (ab);
\draw[-stealth](bb) to [out=-45,in=0] ++(0,-1)node[below]{$0/1$} to [out=180,in=-135](bb);
\draw[-stealth](ab) to [out=-90,in=0]node[pos=0.5,below right]{$0/0$}(bb);
\end{tikzpicture}
\caption{حال کا خاکہ (برائے شکل \حوالہ{شکل_ترتیبی_بطور_مثال})}
\label{شکل_ترتیبی_خاکہ_حال}
\end{figure}
	


\جزوحصہ{ڈی پلٹ کار پر مبنی ترتیبی دور}
ترتیبی ادوار کے حل کی مزید مثالوں پر غور کرتے ہیں۔پہلی مثال ڈی پلٹ کار پر مبنی ہے جو شکل \حوالہ{شکل_ترتیبی_ڈی_پلٹ_دور} میں پیش ہے۔ دور میں ایک پلٹ کار پایا جاتا ہے جس کا مخارج \عددی{A}لکھ کر، مداخل \عددی{A(x+y)} ہو گا۔

\begin{figure}
\centering
\begin{tikzpicture}
\pgfmathsetmacro{\kshPa}{0.5}
\pgfmathsetmacro{\kshPb}{0.5}
\kDFF[u1]{0}{0}
\draw(u1p1) --++(-3*\kshPa,0)node[above right]{$(x+y)A$}node[and port, scale=1,number inputs=2,anchor=out](u2){} (u2.in 2)--++(-2*\kshPa,0)node[above,xshift=2ex]{$x+y$}node[or port,scale=1,number inputs=2,anchor=out](u3){};
\draw(u3.in 1)node[left]{$x$} (u3.in 2)node[left]{$y$} (u1p2)--++(0,-\kshPa)node[below]{$C$};
\draw(u2.in 1)--++(0,\kshPb)-|(u1p6) --++(\kshPa,0)node[right]{$A$};
\end{tikzpicture}
\caption{ڈی پلٹ کار پر مبنی ترتیبی دور۔}
\label{شکل_ترتیبی_ڈی_پلٹ_دور}
\end{figure}

 ساعت کے کنارہ چڑھائی پر ڈی پلٹ کار مداخل کے تحت نیا حال اختیار کرتا ہے، لہٰذا اگلے حال کی مساوات درج ذیل ہو گی
\begin{align*}
A(n+1)=A(n)(x(n)+y(n))
\end{align*}
جس کی سادہ صورت ذیل ہے۔
\begin{align*}
A=A(x+y)
\end{align*}
اس مساوات کے نتائج شکل \حوالہ{شکل_ڈی_خاکہ_اور_جدول} میں جدول میں پیش ہیں۔حال کا خاکہ اور اس کا سادہ روپ (نچلا خاکہ) بھی شکل پیش ہیں۔ پلٹ کار کے حال \عددی{0} اور \عددی{1} دائروں میں رکھے گئے ہیں، جبکہ ان کے بیچ انتقال تیر دار لکیر سے دکھایا گیا ہے۔ تیر دار لکیروں پر مداخل \عددی{xy} کی موجودہ قیمتیں لکھی گئی ہیں۔ ایک ہی حال میں رہنے کے تمام ممکنات کو اکٹھا بھی لکھا جا سکتا ہے، جیسے نچلے خاکہ میں کیا گیا ہے۔ آپ دیکھ سکتے ہیں کہ حال \عددی{1} سے \عددی{0} اس وقت انتقال ہو گا جب مداخل \عددی{00} ہو۔ باقی تمام حال میں پلٹ کار موجودہ حال برقرار رکھتا ہے۔مزید، حال \عددی{0} سے حال \عددی{1} منتقلی کا کوئی راستہ موجود نہیں۔

\begin{figure}
\centering
\begin{subfigure}{0.35\textwidth}
\centering
\begin{otherlanguage}{english}
\begin{tabular}{CCC|C}
\toprule
\multicolumn{3}{c|}{\text{\RL{موجودہ }}}&{\text{\RL{اگلا }}}\\
\midrule
A&x&y&A\\
\midrule
0&0&0&0\\
0&0&1&0\\
0&1&0&0\\
0&1&1&0\\
1&0&0&0\\
1&0&1&1\\
1&1&0&1\\
1&1&1&1\\
\bottomrule
\end{tabular}
\end{otherlanguage}
\end{subfigure}\hfill
\begin{subfigure}{0.55\textwidth}
\centering
\begin{tikzpicture}
\draw(0,0)node[]{$0$}node[draw,thick,circle,inner sep=0.25cm](a){} (2.5,0)node[]{$1$}node[draw,thick,circle,inner sep=0.25cm](b){};
\draw[-stealth] (b) to [out=160,in=10]node[pos=0.5,above]{$00$}(a);
\draw[-stealth] (b) to [out=110,in=180]++(0,1)node[above]{$11$} to [out=0,in=60] (b);
\draw[-stealth] (b) to [out=30,in=90]++(1,0)node[right]{$10$} to [out=-90,in=-30] (b);
\draw[-stealth] (b) to [out=-110,in=180]++(0,-1)node[below]{$01$} to [out=0,in=-60] (b);
\draw[-stealth] (a) to [out=45,in=0]++(0,1)node[above]{$00$} to [out=180,in=90] (a);
\draw[-stealth] (a) to [out=110,in=70]++(-0.75,0.5)node[above left]{$01$} to [out=-110,in=180] (a);
\draw[-stealth] (a) to [out=-160,in=135]++(-0.75,-0.5)node[below left]{$10$} to [out=-45,in=-90] (a);
\draw[-stealth] (a) to [out=-70,in=-90]++(0.75,-0.5)node[below right]{$11$} to [out=90,in=-20] (a);
\end{tikzpicture}
\begin{tikzpicture}
\draw(0,0)node[]{$0$}node[draw,thick,circle,inner sep=0.25cm](a){} (2.5,0)node[]{$1$}node[draw,thick,circle,inner sep=0.25cm](b){};
\draw[-stealth] (b) to [out=160,in=10]node[pos=0.5,above]{$00$}(a);
\draw[-stealth] (b) to [out=-30,in=60]++(0.75,-0.5)node[below,yshift=-1ex]{$10,01,11$} to [out=-120,in=-90] (b);
\draw[-stealth] (a) to [out=110,in=70]++(-0.75,0.5)node[above]{$00,01,10,11$} to [out=-110,in=180] (a);
\end{tikzpicture}
\end{subfigure}
\caption{ حال کا جدول اور حال کا خاکہ (برائے شکل \حوالہ{شکل_ترتیبی_ڈی_پلٹ_دور})}
\label{شکل_ڈی_خاکہ_اور_جدول}
\end{figure}


\جزوحصہ{جے کے پلٹ کار پر مبنی ترتیبی دور}
شکل \حوالہ{شکل_ترتیبی_جے_کے_مثال} میں جے کے پلٹ کار پر مبنی ترتیبی دور پیش ہے۔بالا پلٹ کار کا حال \عددی{Q_A} اور مداخل \عددی{J_A}، \عددی{K_A} ہیں، جبکہ زیریں پلٹ کار کا حال \عددی{Q_B} اور مداخل \عددی{J_B}، \عددی{K_B} ہیں۔

\begin{figure}
\centering
\begin{tikzpicture}
\pgfmathsetmacro{\kshPa}{0.5}
\pgfmathsetmacro{\kshPb}{0.75}
\pgfmathsetmacro{\kshPc}{2.0}
\kJKFF[u1]{0}{0}
\kJKFF[u2]{0}{-\kshPc}
\draw(u1p1)--++(0,\kshPb)--++(-\kshPc,0)node[above right]{$Q_B\overline{x}$}node[and port,scale=1,number inputs=2,anchor=out](u3){};
\draw(u1p3)--++(-\kshPc,0)node[above right]{$Q_B x$}node[and port,scale=1,number inputs=2,anchor=out](u4){};
\draw(u2p3)--++(-\kshPc,0)node[above right]{$\overline{Q_A \oplus x}$}node[xnor port,scale=1,number inputs=2,anchor=out](u5){};
\draw(u3.in 1)--++(-\kshPa,0)node[not port,scale=1,anchor=out](u6){} (u6.in)--++(-\kshPa,0)node[left]{$x$};
\draw(u5.in 2)--++(-\kshPa,0)node[left]{$Q_A$};
\draw(u3.in 2)--(u4.in 1);
\draw(u4.in 1)--++(-\kshPa,0)node[left]{$Q_B$};
\draw(u6.in)|-(u5.in 1);
\draw(u4.in 2)coordinate(aabb)--(aabb -| u6.in);
\draw(u2p1)--(u2p1 -|u6.in);
\draw(u1p2)node[left]{$C$}(u2p2)node[left]{$C$};
\draw(u1p6)node[right]{$Q_A$} (u2p6)node[right]{$Q_B$};
\end{tikzpicture}
\caption{جے کے پلٹ کار پر مبنی ترتیبی دور}
\label{شکل_ترتیبی_جے_کے_مثال}
\end{figure}

 دور میں متمم بلا شرکت جمع گیٹ کا ایک مداخل \عددی{Q_A} ہے جو بالائی پلٹ کار کا موجودہ حال ہے۔ پلٹ کار کے مخارج سے گیٹ کے مداخل تک تار کھینچنے کی بجائے دونوں کا نام \عددی{(Q_A)} رکھا گیا ہے۔ جب بھی دو مقامات کا ایک نام رکھا جائے، انہیں آپس میں برقی طور جڑا تصور کریں۔ یوں ، دونوں ضرب گیٹ کا ایک ایک مداخل زیریں پلٹ کار کے مخارج سے جڑا ہے۔

مداخل کی مساوات ذیل ہیں۔
\begin{gather}
\begin{aligned}\label{مساوات_ترتیبی_جے_کے_بطور_مثال}
J_A&=\overline{x}Q_B\\
K_A&=xQ_B\\
J_B&=x\\
K_B&=\overline{x\oplus Q_A}
\end{aligned}
\end{gather}
ان مساوات سے جدول \حوالہ{جدول_ترتیبی_جے_کے_بطور_مثال} حاصل ہو گا، جس سے اضافی مواد نکال کر حال کا جدول حاصل ہو گا (شکل \حوالہ{شکل_ترتیبی_حال_جے_کے}) ۔  حال  کے جدول سے حاصل حال کا خاکہ بھی شکل میں پیش ہے۔

مساوات \حوالہ{مساوات_ترتیبی_جے_کے_بطور_مثال} سے جدول \حوالہ{جدول_ترتیبی_جے_کے_بطور_مثال} لکھتے ہوئے موجودہ حال \عددی{Q_A}، \عددی{Q_B} اور مداخل \عددی{x} کی تمام ممکنات \عددی{000_2} تا \عددی{111_2} لکھیں (جدول میں بائیں ہاتھ تین قطاریں)۔ ہر صف کے لئے پلٹ کار کے مطابقتی موجودہ مداخل \عددی{J_A}، \عددی{K_A}، \عددی{J_B}،اور \عددی{K_B} مساوات \حوالہ{مساوات_ترتیبی_جے_کے_بطور_مثال} سے حاصل کریں۔ یوں پہلی صف کے لئے، جہاں موجودہ قیمتیں \عددی{Q_A=0}، \عددی{Q_B=0}، اور \عددی{x=0} ہیں، درج ذیل حاصل ہو گا۔
\begin{align*}
J_A&=\overline{x}Q_B=\overline{0}\cdot 0=1\cdot 0=0\\
K_A&=xQ_B=0\cdot 0=0\\
J_B&=x=0\\
K_B&=\overline{x\oplus Q_A}=\overline{0\oplus 0}=\overline{0}=1
\end{align*}
انہیں جدول کی پہلی صف میں درج کریں۔پلٹ کار کے موجودہ مداخل جانتے ہوئے ساعت کے اگلے کنارہ چڑھائی پر اگلے حال مساوات \حوالہ{مساوات_ترتیبی_جے_کے_کارکردگی} \عددی{(Q(n+1)=J\overline{Q}_n+\overline{K}Q_n)} یا مساوات \حوالہ{مساوات_ترتیبی_جے_کے_عام_فہم_کارکردگی} سے 
\begin{align*}
Q_A&=J_A\overline{Q}_A+\overline{K}_AQ_A=0\cdot \overline{0}+\overline{0}\cdot 0=0\cdot 1+1\cdot 0=0+0=0\\
Q_B&=J_B\overline{Q}_B+\overline{K}_BQ_B=0\cdot \overline{0}+\overline{1}\cdot 0= 
\end{align*}
 حاصل کر کے جدول کی پہلی صف میں درج کریں۔ باقی صف کے لئے مواد حاصل کے کے جدول بھریں۔
 
\begin{table}
\caption{جے کے پلٹ کار دور کی مساوات \حوالہ{مساوات_ترتیبی_جے_کے_بطور_مثال} سے حاصل جدول}
\label{جدول_ترتیبی_جے_کے_بطور_مثال}
\centering
\begin{otherlanguage}{english}
\begin{tabular}{CCC|CCCC|CC}
\toprule
\multicolumn{3}{c|}{\text{\RL{موجودہ مداخل اور حال}}} &\multicolumn{4}{c|}{\text{\RL{پلٹ کار کے مداخل}}}&\multicolumn{2}{c}{\text{\RL{اگلے حال}}}\\
\midrule
Q_A&Q_B&x&J_A&K_A&J_B&K_B&Q_A&Q_B\\
\midrule
0&0&0&0&0&0&1&0&0\\
0&0&1&0&0&1&0&0&1\\
\midrule
0&1&0&1&0&0&1&1&0\\
0&1&1&0&1&1&0&0&1\\
\midrule
1&0&0&0&0&0&0&1&0\\
1&0&1&0&0&1&1&1&1\\
\midrule
1&1&0&1&0&0&0&1&1\\
1&1&1&0&1&1&1&0&0\\
\bottomrule
\end{tabular}
\end{otherlanguage}
\end{table}

\begin{figure}
\centering
\begin{subfigure}{0.45\textwidth}
\centering
\begin{otherlanguage}{english}
\begin{tabular}{CCC}
\toprule
\text{\RL{موجودہ حال}} & \multicolumn{2}{c}{\text{\RL{اگلا حال}}}\\
\cline{2-3}
& x=0&x=1\\
Q_AQ_B&Q_AQ_B&Q_AQ_B\\
\midrule
00&00&01\\
01&10&01\\
10&10&11\\
11&11&00\\
\bottomrule
\end{tabular}
\end{otherlanguage}
\end{subfigure}\hfill
\begin{subfigure}{0.45\textwidth}
\centering
\begin{tikzpicture}
\draw(0,0)node[]{$00$}node[draw,thick,circle,inner sep=0.25cm,](aa){} 
(1,1)node[]{$11$}node[draw,thick,circle,inner sep=0.25cm,](bb){}
(2,0)node[]{$10$}node[draw,thick,circle,inner sep=0.25cm,](ba){}
(1,-1)node[]{$01$}node[draw,thick,circle,inner sep=0.25cm,](ab){};
\draw[-stealth](aa) to [out=-90,in=180]node[pos=0.5,below left]{$1$}(ab);
\draw[-stealth](ab) to [out=0,in=-90]node[pos=0.5,below right]{$0$}(ba);
\draw[-stealth](ba) to [out=90,in=0]node[pos=0.5,below]{$1$}(bb);
\draw[-stealth](bb) to [out=180,in=90]node[pos=0.5,above left]{$1$}(aa);
\draw[-stealth](aa) to [out=135,in=90]++(-1,0)node[left]{$0$} to [out=-90,in=-135](aa);
\draw[-stealth](ab) to [out=-135,in=180]++(0,-1)node[below]{$1$} to [out=0,in=-45](ab);
\draw[-stealth](ba) to [out=-30,in=-90]++(1,0)node[right]{$0$} to [out=90,in=30](ba);
\draw[-stealth](bb) to [out=45,in=0]++(0,1)node[above]{$0$} to [out=180,in=135](bb);
\end{tikzpicture}
\end{subfigure}
\caption{حال کا جدول اور حال کا خاکہ برائے شکل \حوالہ{شکل_ترتیبی_جے_کے_مثال}}
\label{شکل_ترتیبی_حال_جے_کے}
\end{figure}

 
 آپ \عددی{J} اور \عددی{K} کی مساوات استعمال کر کے بھی \عددی{Q} تلاش کر سکتے ہیں۔
 \begin{align*}
 Q_A(n+1)&=J_A\overline{Q}_A+\overline{K}_AQ_A=(\overline{x}Q_B)\overline{Q}_A+(\overline{xQ_B})Q_A\\
 Q_B(n+1)&=J_B\overline{Q}_B+\overline{K}_BQ_B=x\overline{Q}_B+(\overline{\overline{x\oplus Q_A}})Q_B
 \end{align*}

 حال  کے خاکہ (شکل \حوالہ{شکل_ترتیبی_حال_جے_کے}) پر توجہ دیں۔ حال \عددی{00} سے \عددی{01} اور یہاں سے \عددی{10} اور اس کے بعد \عددی{11} جایا جا سکتا ہے، جس کے بعد دوبارہ \عددی{00} سے پوری کہانی شروع ہو گی۔ یہ \عددی{00} تا \عددی{11} ثنائی گنت کار معلوم ہوتا ہے۔ ماسوائے حال \عددی{11} کے، ہر مرتبہ \عددی{x} تبدیل کرنے سے حال تبدیل ہو گا۔ یوں \عددی{00} میں جب تک \عددی{x=0} رہے، دور اسی حال میں رہتا ہے، البتہ \عددی{x} بلند کرنے سے \عددی{01} حال حاصل ہو گا، جہاں اس وقت تک رہا جائے گا جب تک \عددی{x=1} رہے۔



\جزوحصہ{ٹی پلٹ کار کی مدد سے ترتیبی دور کا جائزہ}
شکل \حوالہ{شکل_ترتیبی_ٹی_پلٹ_ترتیبی_مثال} میں ٹی پلٹ کار پر مبنی ترتیبی دور پیش ہے۔ پلٹ کار کے حال \عددی{A} اور \عددی{B} سے ظاہر کیے گئے ہیں۔یوں پہلے پلٹ کار کا مداخل \عددی{T_A} اور دوسرے کا \عددی{T_B} ہے۔
\begin{figure}
\centering
\begin{tikzpicture}
\pgfmathsetmacro{\kshY}{2.00}
\pgfmathsetmacro{\kshPa}{0.75}
\pgfmathsetmacro{\kshPb}{0.75}
\kTFF[u1]{0}{0}
\kTFF[u2]{0}{-\kshY}
\draw(u1p1)--++(-\kshPa,0)node[and port,scale=1,number inputs=2,anchor=out](u3){};
\draw(u2p1)node[nor port,scale=1,number inputs=2,anchor=out](u4){};
\draw(u3.out)--++(0,-\kshPb)-|(u4.in 1);
\draw(u1p2)--(u2p2)--++(0,-\kshPb)--++(-0.5,0)node[left]{$C$};
\draw(u3.in 1)node[left]{$A$} (u3.in 2)node[left]{$\overline{B}$} (u4.in 2)node[left]{$x$};
\draw(u1p6)node[right]{$A$} (u1p4)node[right]{$\overline{A}$}
(u2p6)node[right]{$B$} (u2p4)node[right]{$\overline{B}$};
\end{tikzpicture}
\caption{ٹی پلٹ کار پر مبنی ترتیبی دور}
\label{شکل_ترتیبی_ٹی_پلٹ_ترتیبی_مثال}
\end{figure}

پلٹ کار کا اگلا حال مساوات \حوالہ{مساوات_ترتیبی_ٹی_پلٹ_کی_مساوات} سے ملتا ہے جسے یہاں دوبارہ پیش کرتے ہیں۔
\begin{align*}
Q_{n+1}=T\oplus Q_n
\end{align*}
موجودہ ضرورت کے تحت مساوات سے درج ذیل لکھا جاتا ہے۔
\begin{gather}
\begin{aligned}\label{مساوات_ترتیبی_ٹی_مثال_حال}
A_{n+1}&=T_A\oplus A=T_A \overline{A}+\overline{T}_AA\\
B_{n+1}&=T_B\oplus B=T_B\overline{B}+\overline{T}_BB
\end{aligned}
\end{gather}

پلٹ کار کے مداخل کی مساوات شکل \حوالہ{شکل_ترتیبی_ٹی_پلٹ_ترتیبی_مثال} سے حاصل کرتے ہیں۔
\begin{align*}
T_A&=A\overline{B}\\
T_B&=\overline{A\overline{B}+x}
\end{align*}
ان مساوات کو مساوات \حوالہ{مساوات_ترتیبی_ٹی_مثال_حال} میں ڈالنے سے پلٹ کار کے حال کی مساواتیں حاصل ہوں گی:
\begin{align*}
A_{n+1}&=(A\overline{B})\oplus A\\
B_{n+1}&=(\overline{A\overline{B}+x})\oplus B
\end{align*}
جن سے جدول \حوالہ{جدول_ترتیبی_ٹی_پلٹ_بطور_مثال}-الف ملتا ہے۔ مداخل \عددی{x} اور موجودہ حال \عددی{A} اور \عددی{B} کو پہلی تین قطاروں میں لکھا گیا ہے۔ ان کی تمام ترتیب (\عددی{000_2} تا \عددی{111_2}) پہلی تین قطاروں میں بھر کر، ہر صف کے لئے مطابقتی موجودہ مداخل حاصل کیے جاتے ہیں، جنہیں دائیں قطاروں میں لکھا گیا ہے۔ موجودہ مداخل سے ساعت کے اگلے کنارہ چڑھائی پر اگلے حال حاصل ہوں گے۔ جدول \حوالہ{جدول_ترتیبی_ٹی_پلٹ_بطور_مثال}-الف سے جدول-ب لکھا جا سکتا ہے، جو حال کا جدول کہلاتا ہے۔
\begin{table}
\caption{ٹی پلٹ کار دور (شکل \حوالہ{شکل_ترتیبی_ٹی_پلٹ_ترتیبی_مثال}) کا حال کا جدول}
\label{جدول_ترتیبی_ٹی_پلٹ_بطور_مثال}
\centering
\begin{subtable}{0.6\textwidth}
\caption{}
\centering
\begin{otherlanguage}{english}
\begin{tabular}{CCC|CC|CC}
\toprule
\multicolumn{3}{c}{\text{\RL{موجودہ مواد}}}&\multicolumn{2}{|c|}{\text{\RL{اگلا حال}}} &\multicolumn{2}{c}{\text{\RL{مداخل}}}\\
\midrule
A&B&x&A&B&T_A&T_B\\
\midrule
0&0&0&0&1&0&1\\
0&0&1&0&0&0&0\\
\midrule
0&1&0&0&0&0&1\\
0&1&1&0&1&0&0\\
\midrule
1&0&0&0&0&1&0\\
1&0&1&0&0&1&0\\
\midrule
1&1&0&1&0&0&1\\
1&1&1&1&1&0&0\\
\bottomrule
\end{tabular}
\end{otherlanguage}
\end{subtable}\hfill
\begin{subtable}{0.4\textwidth}
\caption{}
\centering
\begin{otherlanguage}{english}
\begin{tabular}{CCC}
\toprule
\text{\RL{موجودہ}}&\multicolumn{2}{c}{\text{\RL{اگلا حال}}}\\
\cline{2-3}
&x=0&x=1\\
AB&AB&AB\\
\midrule
00&01&00\\
01&00&01\\
10&00&00\\
11&10&11\\
\bottomrule
\end{tabular}
\end{otherlanguage}
\end{subtable}
\end{table}

 حال  کے جدول کے مواد کو  حال  کے خاکہ کی صورت میں شکل \حوالہ{شکل_ترتیبی_ڈی_پلٹ_بطور_مثال_خاکہ_حال} میں پیش کیا گیا ہے۔ جدول \حوالہ{جدول_ترتیبی_ٹی_پلٹ_بطور_مثال}-ب میں \عددی{AB} کو ساتھ ساتھ لکھ کر ایک حال تصور کریں۔یوں \عددی{00}، \عددی{01}،\عددی{10}،اور \عددی{11} حال ممکن ہیں۔  حال  کے خاکہ میں حال کو گول دائرہ میں لکھا جاتا ہے، اور ایک حال سے دوسرے حال (یا اسی حال) انتقال کو تیر دار لکیر سے ظاہر کیا جاتا ہے،جن پر آزاد مداخل \عددی{(x)} کی وہ قیمت درج کی جاتی ہے، جو انتقال کا سبب بنتی ہے۔ مثلاً، جدول-ب کی پہلی صف میں موجودہ حال \عددی{00} ہے؛ اب \عددی{x=1} کی صورت میں دور اسی حال \عددی{(00)} میں رہتا ہے، جس کو حال  کے خاکہ میں \عددی{00}حال سے ابتدا اور اختتام کرنے والی تیر دار لکیر سے ظاہر کیا گیا ہے، جس پر \عددی{1} لکھا گیا ہے؛ البتہ \عددی{x=0} کی صورت میں دور حال \عددی{01} اختیار کرتا ہے،جس کو \عددی{00} سے \عددی{01} جانے والی تیر دار لکیر ظاہر کرتی ہے، جس پر \عددی{0} لکھا گیا ہے ۔

\begin{figure}
\centering
\begin{tikzpicture}
\draw(0,0.5)node[]{$00$}node[draw,thick,circle,inner sep=0.25cm](aa){};
\draw(-1,0)node[]{$01$}node[draw,thick,circle,inner sep=0.25cm](ab){};
\draw(1.5,0.5)node[]{$10$}node[draw,thick,circle,inner sep=0.25cm](ba){};
\draw(1.5,-0.5)node[]{$11$}node[draw,thick,circle,inner sep=0.25cm](bb){};
\draw[-stealth](aa) to [out=70,in=0]++(0,0.75)node[above]{$1$} to [out=180,in=110] (aa);
\draw[-stealth](ab) to [out=-120,in=-180]++(0,-0.75)node[below]{$1$} to [out=0,in=-70] (ab);
\draw[-stealth](bb) to [out=180,in=90]++(-0.75,-0.5)node[left]{$1$} to [out=-90,in=-90] (bb);
\draw[-stealth](aa) to [out=150,in=90] node[pos=0.5,above left]{$0$}(ab);
\draw[-stealth](ab) to [out=-30,in=-90] node[pos=0.5,below right]{$0$}(aa);
\draw[-stealth](ba) to [out=-150,in=-30] node[pos=0.5,below]{$0,1$}(aa);
\draw[-stealth](bb) to [out=30,in=-30] node[pos=0.5,right]{$0$}(ba);
\end{tikzpicture}
\caption{حال کا خاکہ برائے شکل \حوالہ{شکل_ترتیبی_ٹی_پلٹ_ترتیبی_مثال} اور جدول \حوالہ{جدول_ترتیبی_ٹی_پلٹ_بطور_مثال}}
\label{شکل_ترتیبی_ڈی_پلٹ_بطور_مثال_خاکہ_حال}
\end{figure}


\حصہ{میلی اور مُور نمونہ}
 ترتیبی دور میں مداخل، مخارج اور اندرونی حال پائے جاتے ہیں۔ترتیبی ادوار کے دو نمونے پائے جاتے ہیں، جنہیں \اصطلاح{میلی نمونہ}\فرہنگ{میلی نمونہ}\حاشیہب{Mealy}\فرہنگ{Mealy} اور \اصطلاح{مُور نمونہ}\فرہنگ{مور نمونہ}\حاشیہب{Moore}\فرہنگ{Moore} کہتے ہیں۔میلی نمونہ میں مخارج کا دارومدار موجودہ مداخل اور موجودہ اندونی حال پر ، جبکہ مُور نمونہ میں صرف موجودہ حال پر ہو گا۔یہ دو نمونے شکل \حوالہ{شکل_ترتیبی_مور_میلی} میں پیش ہیں۔
 
 ان اشکال میں مداخل تیر دار لکیر پر ترچھی لکیر کھینچ کر \عددی{X} لکھا گیا ہے، جو مداخل ثنائی ہندسوں (بِٹ) کی تعداد بیان کرتا ہے۔ یوں \عددی{X=8} کی صورت میں ایک ایک بٹ کے آٹھ مداخل ہوں گے۔ حافظہ کے مداخل اور مخارج کی تعداد برابر ہو گی، لہٰذا اس کے مداخل (یا مخارج) پر \عددی{Y} لکھنے کے بعد مخارج (یا مداخل) پر صرف ترچھی لکیر کھینچنا کافی ہو گا۔
\begin{figure}
\centering
\begin{subfigure}{0.45\textwidth}
\centering
\begin{tikzpicture}
\pgfmathsetmacro{\kdimX}{1.5}
\pgfmathsetmacro{\kdimY}{0.75}
\pgfmathsetmacro{\ksepY}{1.00}
\pgfmathsetmacro{\kshPa}{1.0}
\pgfmathsetmacro{\kshPb}{0.5}
\pgfmathsetmacro{\kdel}{0.1}
\draw[thick](0,0) rectangle ++(\kdimX,\kdimY)node[pos=0.5]{\text{\RL{ترکیبی منطق}}};
\draw[thick](0,-\ksepY) rectangle ++(\kdimX,\kdimY)node[pos=0.5]{\text{\RL{حافظہ}}};
\draw[stealth-] (0,0.75*\kdimY)--++(-\kshPa,0)coordinate[pos=0.5](aa)node[left]{مداخل};
\draw(aa)++(-\kdel,-\kdel)--++(2*\kdel,2*\kdel)node[above]{$X$};
\draw[-stealth] (\kdimX,0.75*\kdimY)--++(\kshPa,0)coordinate[pos=0.5](bb)node[right]{مخارج};
\draw(bb)++(-\kdel,-\kdel)--++(2*\kdel,2*\kdel)node[above]{$Z$};
\draw[-stealth](\kdimX,0.25*\kdimY)--++(\kshPb,0)|-(\kdimX,-\ksepY+0.5*\kdimY);
\draw[-stealth](0,-\ksepY+0.5*\kdimY)--++(-\kshPb,0)|-(0,0.25*\kdimY);
\draw(-\kshPb,-\ksepY/2+\kdimY/2-0.125*\kdimY)++(\kdel,-\kdel)--++(-2*\kdel,2*\kdel)node[left]{$Y$};
\draw(\kdimX+\kshPb,-\ksepY/2+\kdimY/2-0.125*\kdimY)++(-\kdel,-\kdel)--++(2*\kdel,2*\kdel)node[right]{$Y$};
\end{tikzpicture}
\caption{میلی نمونہ}
\end{subfigure}\hfill
\begin{subfigure}{0.45\textwidth}
\centering
\begin{tikzpicture}
\pgfmathsetmacro{\kdimX}{1.5}
\pgfmathsetmacro{\kdimY}{0.75}
\pgfmathsetmacro{\ksepY}{1.00}
\pgfmathsetmacro{\kshPa}{1.0}
\pgfmathsetmacro{\kshPb}{0.5}
\pgfmathsetmacro{\kdel}{0.1}
\draw[thick](0,0) rectangle ++(\kdimX,\kdimY)node[pos=0.5]{\text{\RL{ترکیبی منطق}}};
\draw[thick](0,-\ksepY) rectangle ++(\kdimX,\kdimY)node[pos=0.5]{\text{\RL{حافظہ}}};
\draw[thick](0,-2*\ksepY) rectangle ++(\kdimX,\kdimY)node[pos=0.5]{\text{\RL{ترکیبی منطق}}};
\draw[stealth-] (0,0.75*\kdimY)--++(-\kshPa,0)coordinate[pos=0.5](aa)node[left]{مداخل};
\draw(aa)++(-\kdel,-\kdel)--++(2*\kdel,2*\kdel)node[above]{$X$};
\draw[-stealth] (\kdimX,-2*\ksepY+0.75*\kdimY)--++(\kshPa,0)coordinate[pos=0.5](bb)node[right]{مخارج};
\draw(bb)++(-\kdel,-\kdel)--++(2*\kdel,2*\kdel)node[above]{$Z$};
\draw[-stealth](\kdimX,0.25*\kdimY)--++(\kshPb,0)|-(\kdimX,-\ksepY+0.5*\kdimY);
\draw[-stealth](0,-\ksepY+0.5*\kdimY)--++(-\kshPb,0)coordinate(dd)|-(0,0.25*\kdimY);
\draw(-\kshPb/2,-\ksepY+\kdimY/2)++(\kdel,-\kdel)--++(-2*\kdel,2*\kdel);
\draw(\kdimX+\kshPb,-\ksepY/2+\kdimY/2-0.125*\kdimY)++(-\kdel,-\kdel)--++(2*\kdel,2*\kdel)node[right]{$Y$};
\draw[-stealth](dd)|-(0,-2*\ksepY+\kdimY/2);
\end{tikzpicture}
\caption{مور نمونہ}
\end{subfigure}
\caption{مور اور میلی نمونے}
\label{شکل_ترتیبی_مور_میلی}
\end{figure}

%???KKK
%this heading makes no sense

\جزوحصہ{حال اور ان کی مقرری}
حصہ \حوالہ{حصہ_ترتیبی_خاکہ_حال} میں  حال  کے خاکہ پر غور کیا گیا۔ ان خاکوں میں پلٹ کار کے مخارج کی بجائے دیگر ناموں سے حال ظاہر کر کے حال  کا خاکہ سمجھنا آسان بنایا جا سکتا ہے (درج ذیل مثال دیکھیں)۔


\ابتدا{مثال}\شناخت{مثال_ترتیبی_حال_کے_نام}
ایسے ایک مداخل، ایک مخارج معاصر ترتیبی دور کا حال کا خاکہ تیار کریں، جو \عددی{110_2} مداخل کے حصول پر \عددی{1} خارج کرتا ہو۔ بلند رتبی بِٹ پہلا بِٹ تصور کریں۔ایسے دور کو \اصطلاح{ترتیب شناس}\فرہنگ{ترتیب!شناس}\حاشیہب{sequence detector}\فرہنگ{sequence!detector} کہتے ہیں۔ 

\begin{figure}
\centering
\begin{tikzpicture}[every text node part/.style={align=center}]
\draw(0,0)node[draw,thick,ellipse,inner sep=0pt](a){\text{\RL{\,\,\, ابتدا \,\,\,}} \\ $0$};
\draw(2,1)node[draw,thick,ellipse,inner sep=0pt](b){\text{\RL{پہلا ایک مل گیا}} \\ $0$};
\draw(6,1)node[draw,thick,ellipse,inner sep=0pt](c){\text{\RL{دوسرا ایک مل گیا}} \\ $0$};
\draw(8,0)node[draw,thick,ellipse,inner sep=0pt](d){\text{\RL{ترتیب مل گئی}} \\ $1$};
\draw[-stealth](a) to [out=90,in=180] node[pos=0.5,above left]{$1$}(b);
\draw[-stealth](b) to [out=30,in=150] node[pos=0.5,above]{$1$}(c);
\draw[-stealth](c) to [out=0,in=60] node[pos=0.5,above right]{$0$}(d);
\draw[-stealth](d) to [out=180,in=-20] node[pos=0.5,above]{$0,1$}(a);
\draw[-stealth](a) to [out=-170,in=-90]++(-1.5,0)node[left]{$0$} to [out=90,in=170] (a);
\draw[-stealth](c) to [out=110,in=180]++(0,1)node[above]{$1$} to [out=0,in=70] (c);
\draw[-stealth](b) to [out=-100,in=10] node[pos=0.5,above left]{$0$}(a);
\end{tikzpicture}
\caption{حال کو الفاظ سے پکار کر خاکہ بہتر سمجھ آتا ہے (مثال \حوالہ{مثال_ترتیبی_حال_کے_نام})}
\label{شکل_ترتیبی_مثال_حال_کے_نام}
\end{figure}
\ترچھا{حل:}\quad
شکل \حوالہ{شکل_ترتیبی_مثال_حال_کے_نام} میں حال کا خاکہ پیش ہے، جسے دیکھ کر دور کی کارکردگی سمجھنا آسان ہے۔ دائرے میں حال کا نام، اور نام کے نیچے \عددی{0} یا \عددی{1} موجودہ مخارج ظاہر کرتا ہے۔
\انتہا{مثال}

\حصہ{معاصر ترتیبی ادوار کی بناوٹ}
گزشتہ حصے میں مختلف اقسام کے پلٹ کار استعمال کر کے معاصر ترتیبی ادوار تشکیل دیے گئے۔ان ادوار کے حصول کا باضابطہ طریقہ کار درج ذیل ہے۔
\begin{enumerate}
 \item
 مسئلہ کے بیان سے حال کا خاکہ تیار کریں۔
 \item
 درکار حال کی تعداد کم کریں۔
 \item
 ہر حال (کو ظاہر کرنے) کی منفرد ثنائی قیمت منتخب کریں۔
 \item
 حال کا جدول حاصل کریں۔
 \item
 پلٹ کار (کی قسم) کا انتخاب کریں۔
 \item
 پلٹ کار کی داخلی اور خارجی سادہ ترین مساوات حاصل کریں۔
 \item
 ان مساوات سے معاصر ترتیبی دور تشکیل دیں۔
\end{enumerate}
 
\ابتدا{مثال}\شناخت{مثال_ترتیبی_تین_ایک_شناس}
 ایسا معاصر ترتیب شناس تشکیل دیں جو تین متواتر \عددی{1}مداخل کے حصول پر \عددی{1}خارج کرے۔ 
\begin{figure}
\centering
\begin{subfigure}{0.60\textwidth}
\centering
\begin{tikzpicture}
\pgfmathsetmacro{\shPa}{1.5}
\draw[](0,0)node[]{$a/0$}node[draw,thick,circle,inner sep=0.25cm](aa){} (aa.west)node[left]{\text{\RL{ابتدا}}};
\draw[](1*\shPa,0)node[]{$b/0$}node[draw,thick,circle,inner sep=0.25cm](ab){};
\draw[](2*\shPa,0)node[]{$c/0$}node[draw,thick,circle,inner sep=0.25cm](ba){};
\draw[](3*\shPa,0)node[]{$d/1$}node[draw,thick,circle,inner sep=0.25cm](bb){};
\draw[-stealth](aa) to [out=70,in=0]++(0,0.75)node[above]{$0$} to [out=180,in=110] (aa);
\draw[-stealth](aa) to [out=45,in=135] node[pos=0.5,above]{$1$}(ab);
\draw[-stealth](ab) to [out=-90,in=-30]node[pos=0.6,above]{$0$} (aa);
\draw[-stealth](ab) to [out=45,in=135] node[pos=0.5,above]{$1$}(ba);
\draw[-stealth](ba) to [out=-135,in=-45]node[pos=0.25,above]{$0$} (aa);
\draw[-stealth](ba) to [out=45,in=135] node[pos=0.5,above]{$1$}(bb);
\draw[-stealth](bb) to [out=70,in=0]++(0,0.75)node[above]{$1$} to [out=180,in=110] (bb);
\draw[-stealth](bb) to [out=-135,in=-60]node[pos=0.25,above]{$0$} (aa);
\end{tikzpicture}
\end{subfigure}\hfill
\begin{subfigure}{0.4\textwidth}
\centering
\begin{otherlanguage}{english}
\begin{tabular}{CC|CC}
\toprule
\multicolumn{2}{c|}{\text{\RL{موجودہ}}}&\text{\RL{اگلا}}&\text{\RL{موجودہ}}\\
\text{\RL{حال}}&\text{\RL{مداخل}}&\text{\RL{حال}}&\text{\RL{مخارج}}\\
\midrule
a&0&a&0\\
a&1&b&0\\
b&0&a&0\\
b&1&c&0\\
c&0&a&0\\
c&1&d&0\\
d&0&a&1\\
d&1&d&1\\
\bottomrule
\end{tabular}
\end{otherlanguage}
\end{subfigure}
\caption{ترتیب شناس کا حال کا خاکہ ( مثال \حوالہ{مثال_ترتیبی_تین_ایک_شناس})}
\label{شکل_ترتیبی_تین_ایک_شناس}
\end{figure}

\ترچھا{حل:}\quad
ترتیب شناس کی کارکردگی کے بیان سے شکل \حوالہ{شکل_ترتیبی_تین_ایک_شناس} کا حال کا خاکہ کھینچا جاتا ہے۔گول دائروں میں ترچھی لکیر سے اوپر حال کا نام اور نیچے مخارج کی قیمت لکھی گئی ہے۔ شناس کا ابتدائی حال \عددی{a} اور مخارج پست \عددی{(0)} ہے۔ پہلی  \عددی{1} کی  حصول کے بعد حال \عددی{b} اور مخارج پست ہو گا۔ دوسری  \عددی{1} کے بعد حال \عددی{c} اور مخارج پست، تیسری \عددی{1} کے بعد حال \عددی{d} اور مخارج بلند ہو گا۔ مزید \عددی{1} ملنے سے شناس حال \عددی{d} میں رہتے ہوئے مخارج بلند رکھتا ہے۔ کسی بھی موقع پر \عددی{0} کا حصول، شناس کو واپس ابتدائی حال \عددی{a} منتقل کرتا ہے۔ حال  کے خاکہ سے حاصل جدول، شکل \حوالہ{شکل_ترتیبی_تین_ایک_شناس} میں پیش ہے، جس میں بائیں ہاتھ موجودہ مداخل اور موجودہ حال، جبکہ دائیں ہاتھ اگلا حال اور موجودہ مخارج درج ہیں۔

حال  کے خاکہ سے واضح ہے کہ حال کی تعداد چار ہے، جنہیں دو بِٹ کا ثنائی عدد ظاہر کر سکتا ہے۔
\begin{gather}
\begin{aligned}\label{مساوات_ترتیبی_شناس_حال_انتخاب}
a&=00\\
b&=01\\
c&=10\\
d&=11
\end{aligned}
\end{gather}
(آپ کوئی دوسری انتخاب کر سکتے ہیں۔ مشق \حوالہ{مشق_ترتیبی_شناس} دیکھیں۔) دو بِٹ کے لئے دو پلٹ کار درکار ہوں گے۔ ہم ڈی پلٹ کار منتخب کر کے، ان کے مخارج \عددی{A} اور \عددی{B}، اور مداخل \عددی{D_A} اور \عددی{D_B} لکھتے ہیں۔

ثنائی علامت استعمال کرتے ہوئے شکل \حوالہ{شکل_ترتیبی_تین_ایک_شناس} میں پیش جدول دوبارہ جدول \حوالہ{جدول_ترتیبی_ترتیب_شناس_متواتر_ایک} میں پیش کیا گیا ہے، جس سے ڈی پلٹ کار کی درج ذیل مساوات اخذ ہوتی ہیں۔
\begin{align*}
A(n+1)=D_A(A,B,x)&=\sum(3,5,7)\\
B(n+1)=D_B(A,B,x)&=\sum(1,5,7)\\
y(A,B,x)&=\sum(6,7)
\end{align*}

\begin{table}
\caption{ترتیب شناس  کا حال کا جدول}
\label{جدول_ترتیبی_ترتیب_شناس_متواتر_ایک}
\centering
\begin{otherlanguage}{english}
\begin{tabular}{CCC|CCC}
\toprule
\multicolumn{3}{c|}{\text{\RL{موجودہ}}}&\multicolumn{2}{c}{\text{\RL{اگلا}}}&\text{\RL{موجودہ}}\\
\cline{4-5}
A&B&x&A&B&y\\
\midrule
0&0&0&0&0&0\\
0&0&1&0&1&0\\
0&1&0&0&0&0\\
0&1&1&1&0&0\\
1&0&0&0&0&0\\
1&0&1&1&1&0\\
1&1&0&0&0&1\\
1&1&1&1&1&1\\
\bottomrule
\end{tabular}
\end{otherlanguage}
\end{table}
%
\begin{figure}
\centering
\begin{subfigure}{0.3\textwidth}
\centering
\begin{tikzpicture}
\pgfmathsetmacro{\kxstep}{1}
\pgfmathsetmacro{\kystep}{1}
\pgfmathsetmacro{\kpin}{0.75}
\pgfmathsetmacro{\kmv}{0.15}
\pgfmathsetmacro{\kmva}{0.10}
\draw[xstep=\kxstep,ystep=\kystep](0,0) grid (2*\kxstep,-4*\kystep);
\draw(0,0)--++(135:\kpin)node[pos=0.75,above right]{$x$}node[pos=0.75,below left]{$AB$};
\foreach \kx/\xlb in {0/0,1/1}{\draw(\kx*\kxstep+\kxstep/2,0)node[above]{$\xlb$};}
\foreach \ky/\ylb in {0/{00},1/{01},2/{11},3/{10}}{\draw(0,-\ky*\kystep-\kystep/2)node[left]{$\ylb$};}
\foreach \ky/\ylb in {0/{0},1/{0},2/{0},3/{0}}{\draw(\kxstep/2,-\ky*\kystep-\kystep/2)node[]{$\ylb$};}
\foreach \ky/\ylb in {0/{0},1/{1},2/{1},3/{1}}{\draw(\kxstep+\kxstep/2,-\ky*\kystep-\kystep/2)node[]{$\ylb$};}
\draw[gray,dashed] ($(\kxstep,-1*\kystep)+(\kmv,-\kmv)$) rectangle ($(2*\kxstep,-3*\kystep)+(-\kmv,\kmv)$);
\draw[gray,dashed] ($(\kxstep,-2*\kystep)+(1.5*\kmv,-1.5*\kmv)$) rectangle ($(2*\kxstep,-4*\kystep)+(-1.5*\kmv,1.5*\kmv)$);
\draw(1*\kxstep,-4*\kystep)node[below]{$D_A=xA+xB$};
\end{tikzpicture}
\end{subfigure}\hfill
\begin{subfigure}{0.3\textwidth}
\centering
\begin{tikzpicture}
\pgfmathsetmacro{\kxstep}{1}
\pgfmathsetmacro{\kystep}{1}
\pgfmathsetmacro{\kpin}{0.75}
\pgfmathsetmacro{\kmv}{0.15}
\pgfmathsetmacro{\kmva}{0.10}
\draw[xstep=\kxstep,ystep=\kystep](0,0) grid (2*\kxstep,-4*\kystep);
\draw(0,0)--++(135:\kpin)node[pos=0.75,above right]{$x$}node[pos=0.75,below left]{$AB$};
\foreach \kx/\xlb in {0/0,1/1}{\draw(\kx*\kxstep+\kxstep/2,0)node[above]{$\xlb$};}
\foreach \ky/\ylb in {0/{00},1/{01},2/{11},3/{10}}{\draw(0,-\ky*\kystep-\kystep/2)node[left]{$\ylb$};}
\foreach \ky/\ylb in {0/{0},1/{0},2/{0},3/{0}}{\draw(\kxstep/2,-\ky*\kystep-\kystep/2)node[]{$\ylb$};}
\foreach \ky/\ylb in {0/{1},1/{0},2/{1},3/{1}}{\draw(\kxstep+\kxstep/2,-\ky*\kystep-\kystep/2)node[]{$\ylb$};}
\draw[gray,dashed] ($(\kxstep,-0*\kystep)+(\kmv,\kmv)$) --++ ($(0*\kxstep,-1*\kystep)+(-0*\kmv,0*\kmv)$)--++($(1*\kxstep,0)+(-2*\kmv,0)$)--++($(0,\kystep)+(0,0*\kmv)$);
\draw[gray,dashed] ($(\kxstep,-4*\kystep)+(\kmv,-\kmv)$) --++ ($(0*\kxstep,1*\kystep)+(-0*\kmv,0*\kmv)$)--++($(1*\kxstep,0)+(-2*\kmv,0)$)--++($(0,-\kystep)+(0,0*\kmv)$);
\draw[gray,dashed] ($(\kxstep,-2*\kystep)+(1.5*\kmv,-1.5*\kmv)$) rectangle ($(2*\kxstep,-4*\kystep)+(-1.5*\kmv,1.5*\kmv)$);
\draw(1*\kxstep,-4*\kystep)node[below]{$D_B=xA+x\overline{B}$};
\end{tikzpicture}
\end{subfigure}\hfill
\begin{subfigure}{0.3\textwidth}
\centering
\begin{tikzpicture}
\pgfmathsetmacro{\kxstep}{1}
\pgfmathsetmacro{\kystep}{1}
\pgfmathsetmacro{\kpin}{0.75}
\pgfmathsetmacro{\kmv}{0.15}
\pgfmathsetmacro{\kmva}{0.10}
\draw[xstep=\kxstep,ystep=\kystep](0,0) grid (2*\kxstep,-4*\kystep);
\draw(0,0)--++(135:\kpin)node[pos=0.75,above right]{$x$}node[pos=0.75,below left]{$AB$};
\foreach \kx/\xlb in {0/0,1/1}{\draw(\kx*\kxstep+\kxstep/2,0)node[above]{$\xlb$};}
\foreach \ky/\ylb in {0/{00},1/{01},2/{11},3/{10}}{\draw(0,-\ky*\kystep-\kystep/2)node[left]{$\ylb$};}
\foreach \ky/\ylb in {0/{0},1/{0},2/{1},3/{0}}{\draw(\kxstep/2,-\ky*\kystep-\kystep/2)node[]{$\ylb$};}
\foreach \ky/\ylb in {0/{0},1/{0},2/{1},3/{0}}{\draw(\kxstep+\kxstep/2,-\ky*\kystep-\kystep/2)node[]{$\ylb$};}
\draw[gray,dashed] ($(0*\kxstep,-2*\kystep)+(\kmv,-\kmv)$) rectangle ($(2*\kxstep,-3*\kystep)+(-\kmv,\kmv)$);
\draw(1*\kxstep,-4*\kystep)node[below]{$y=AB$};
\end{tikzpicture}
\end{subfigure}
\caption{کارناف نقشے برائے مثال \حوالہ{مثال_ترتیبی_تین_ایک_شناس} }
\label{شکل_ترتیبی_مثال_کارناف}
\end{figure}

جدول \حوالہ{جدول_ترتیبی_ترتیب_شناس_متواتر_ایک} سے شکل \حوالہ{شکل_ترتیبی_مثال_کارناف} کے کارناف نقشے بنا کر درج ذیل سادہ مساوات حاصل ہوتی ہیں،جن سے 
شکل \حوالہ{شکل_ترتیبی_شناس_مثال} حاصل ہو گا۔
\begin{align*}
D_A&=Ax+Bx\\
D_B&=Ax+\overline{B}x\\
y&=AB
\end{align*}
%
\begin{figure}
\centering
\begin{tikzpicture}
\pgfmathsetmacro{\kshPa}{0.75}
\pgfmathsetmacro{\kshPb}{0.35}
\pgfmathsetmacro{\kshPc}{0.5}
\pgfmathsetmacro{\kshYa}{2.5}
\kDFF[u1]{0}{0}
\kDFF[u2]{0}{-\kshYa}
\draw($(u1p6)!0.5!(u2p6)$)++(3*\kshPa,0)node[and port, scale=1,number inputs=2,anchor=out](u3){};
\draw(u1p6)-|(u3.in 1) (u2p6)-|(u3.in 2) (u3.out)node[right]{$y$};
\draw(u1p1)--++(-\kshPa,0)node[or port,scale=1,number inputs=2,anchor=out](u4){};
\draw(u4.in 1)--++(0,\kshPb)node[and port,scale=1,number inputs=2,anchor=out](u5){};
\draw(u4.in 2)--++(0,-\kshPb)node[and port,scale=1,number inputs=2,anchor=out](u6){};
\draw(u2p1)--++(-\kshPa,0)node[or port,scale=1,number inputs=2,anchor=out](u7){};
\draw(u7.in 1)--++(0,\kshPb)node[and port,scale=1,number inputs=2,anchor=out](u8){};
\draw(u7.in 2)--++(0,-\kshPb)node[and port,scale=1,number inputs=2,anchor=out](u9){};
\draw(u1p6)node[above]{$A$} (u2p6)node[above]{$B$};
\draw(u5.in 1)--++(-\kshPc,0)node[left]{$A$} (u5.in 2)--(u6.in 1) (u5.in 2)--++(-\kshPc,0)node[left]{$x$}
 (u6.in 2)--++(-\kshPc,0) node[left]{$B$};
\draw(u8.in 1)--++(-\kshPc/2,0)coordinate(kaa)--(kaa |- u5.in 1) (u8.in 2)--(u9.in 1) (u8.in 2)--(u6.in 1)
 (u9.in 2)--++(-\kshPc,0)node[left]{$\overline{B}$};
 \draw(u1p2)--(u2p2)--++(0,-\kshPc)--++(-\kshPc,0)node[left]{$C$};
 \draw(u4.out)node[above]{$D_A$} (u7.out)node[above]{$D_B$};
 \draw(u2p4)node[above]{$\overline{B}$};
\end{tikzpicture}
\caption{ترتیب شناس (مثال \حوالہ{مثال_ترتیبی_تین_ایک_شناس})}
\label{شکل_ترتیبی_شناس_مثال}
\end{figure}
ترتیب شناس ابتدائی پست حال میں \عددی{\overline{\text{بیٹھ}}} اشارہ کی مدد سے لایا جاتا ہے، جو شکل میں نہیں دکھایا گیا۔
\انتہا{مثال}
%
\ابتدا{مشق}\شناخت{مشق_ترتیبی_شناس}
مساوات \حوالہ{مساوات_ترتیبی_شناس_حال_انتخاب} میں حال کے اظہار کا ایک انتخاب دکھایا گیا ہے۔آپ کوئی دوسرا انتخاب کر سکتے ہیں، مثلاً \عددی{a=01}، \عددی{b=10}، \عددی{c=11}، اور \عددی{d=00} جس سے دوسرا دور حاصل ہو گا۔ یہ دور حاصل کریں۔
\انتہا{مشق}

\حصہء{سوالات}
\ابتدا{سوال}
%6.1
 ثابت کریں  جے کے پلٹ کے مخارج  \عددی{\overline{Q}_{n+1}} کی مساوات    درج ذیل ہے۔
 \begin{align*}
 \overline{Q}_{n+1}=\overline{J}\,\overline{Q}+KQ
 \end{align*}
\انتہا{سوال}
\ابتدا{سوال}  
%6.2
 شکل میں ضرب گیٹ کا دورانیہ رد عمل  \عددی{10} نینو سیکنڈ جبکہ جمع گیٹ کا \عددی{15} نینو سیکنڈ ہے۔تینوں مداخل بیک وقت تبدیل کیے جاتے ہیں۔ کتنی دیر بعد مخارج  \عددی{F_1}اور \عددی{F_2}  مستحکم حال میں ہوں گے؟
 \begin{center}
 \begin{tikzpicture}
 \pgfmathsetmacro{\kpin}{0.4}
 \draw(0,0)node[or port,anchor=out](u0){};
 \draw(u0.in 1)--++(-\kpin,0)node[above]{$F_1$}node[and port,anchor=out](u1){};
 \draw(u1.in 1)node[left]{$A$} (u1.in 2)node[left]{$B$};
 \draw(u0.in 2)--++(0,-\kpin)coordinate(kk)--(kk -| u1.in 1)node[left]{$C$};
 \draw(u0.out)node[right]{$F_2$};
 \end{tikzpicture}
 \end{center}
 جواب: \عددی{\SI{10}{\nano\second}} ، \عددی{\SI{25}{\nano\second}}
\انتہا{سوال}
\ابتدا{سوال}
%6.3
 ایک کمپیوٹر  \عددی{\SI{2}{\giga\hertz}}  ساعتی اشارے  سے چلتا ہے۔یہ  اشارہ تیس فی صد وقت بلند رہتا ہے جبکہ اس کا دورانیہ اترائی  پانچ فی صد اور دورانیہ چڑھائی پانچ فی صد وقت لیتے ہیں۔ساعتی اشارے  کا دوری عرصہ ،دورانیہ چڑھائی اور پست دورانیہ حاصل کریں۔ 
 
 جواب:  \عددی{\SI{5e-10}{\second}}،  \عددی{\SI{2.5e-11}{\second}}، \عددی{\SI{3e-10}{\second}}
\انتہا{سوال}
\ابتدا{سوال}
% 6.4
 جمع متمم گیٹ پر مبنی  متعدد (بلند فعال)  مداخل ایس آر پلٹ کے مداخل  ترسیم کیے گئے ہیں۔اس کا مخارج  ترسیم کریں۔
 
 \begin{center}
 \begin{otherlanguage}{english}
 \begin{tikztimingtable}[%
timing/.style={x=4ex,y=3ex},
timing/rowdist=5ex,
every node/.style={inner sep=0,outer sep=0},
timing/c/arrow tip=latex, %and this set the style
timing/c/rising arrows,
timing/slope=0.0, %0.1 is good
thick,
]
%$C$& 0.5CN(A1)CCN(A2)CCN(A3)CCN(A4)C\\
$S$&lL{H}4{L}{H}5{L}1.5{H}3.5{L}\\
$R_a$&3{L}HL{h}5{L}1{H}5{L}\\
$R_b$&10{L}2{H}2{L}h2{L}\\
\extracode
%\begin{pgfonlayer}{background}
%\begin{scope}[semitransparent ,dashed]
%\vertlines[darkgray,dotted]{3.6,7.5,11.5,15.5,19.5,23.5,27.5}
%\foreach \n in {1,2,3,4}{\draw(A\n.south)--(A\n |- row3.south);}
%\end{scope}
%\end{pgfonlayer}
\end{tikztimingtable}
\end{otherlanguage}
\end{center}

جواب:
\begin{center}
 \begin{otherlanguage}{english}
 \begin{tikztimingtable}[%
timing/.style={x=4ex,y=3ex},
timing/rowdist=5ex,
every node/.style={inner sep=0,outer sep=0},
timing/c/arrow tip=latex, %and this set the style
timing/c/rising arrows,
timing/slope=0.0, %0.1 is good
thick,
]
%$C$& 0.5CN(A1)CCN(A2)CCN(A3)CCN(A4)C\\
$S$&lL{H}4{L}{H}5{L}1.5{H}3.5{L}\\
$R_a$&3{L}HL{h}5{L}1{H}5{L}\\
$R_b$&10{L}2{H}2{L}h2{L}\\
$Q$&uUhH3{L}l3{H}h2{L}lhH2{L}l\\
\extracode
%\begin{pgfonlayer}{background}
%\begin{scope}[semitransparent ,dashed]
%\vertlines[darkgray,dotted]{3.6,7.5,11.5,15.5,19.5,23.5,27.5}
%\foreach \n in {1,2,3,4}{\draw(A\n.south)--(A\n |- row3.south);}
%\end{scope}
%\end{pgfonlayer}
\end{tikztimingtable}
\end{otherlanguage}
\end{center}

\انتہا{سوال}
\ابتدا{سوال} 
%6.5
 آقا و غلام پلٹ کے مداخل  ترسیم  کیے گئے ہیں۔آقا مخارج \عددی{Q_a}اور  غلام  مخارج \عددی{Q} ترسیم کریں۔
 \begin{center}
 \begin{otherlanguage}{english}
 \begin{tikztimingtable}[%
timing/.style={x=4ex,y=3ex},
timing/rowdist=5ex,
every node/.style={inner sep=0,outer sep=0},
timing/c/arrow tip=latex, %and this set the style
timing/c/falling arrows,
timing/slope=0.0, %0.1 is good
thick,
]
%$C$& cCN(A1)CCN(A2)CCN(A3)CCN(A4)CCCCN(A5)CCN(A6)0.5C\\
 $C$&LHhLlHLlHHHHLLlHHhLl0.5l\\
 $S$&LlhLhLLhLlLhLLLHHLLL0.5l\\
 $R$&LLLLl0.5hLLlhlLLLLLLlHhL\\
\extracode
%\begin{pgfonlayer}{background}
%\begin{scope}[semitransparent ,dashed]
%\vertlines[darkgray,dotted]{3.6,7.5,11.5,15.5,19.5,23.5,27.5}
%\foreach \n in {1,2,3,4,5,6}\draw(A\n.south)--(E\n.north);
%\foreach \n in {1}\draw(B\n.south)--(F\n.north);
%\foreach \n in {1}\draw(C\n.south)--(G\n.north);
%\end{scope}
%\end{pgfonlayer}
\end{tikztimingtable}
\end{otherlanguage}
\end{center}
جواب:
 \begin{center}
 \begin{otherlanguage}{english}
 \begin{tikztimingtable}[%
timing/.style={x=4ex,y=3ex},
timing/rowdist=5ex,
every node/.style={inner sep=0,outer sep=0},
timing/c/arrow tip=latex, %and this set the style
timing/c/falling arrows,
timing/slope=0.0, %0.1 is good
thick,
]
%$C$& cCN(A1)CCN(A2)CCN(A3)CCN(A4)CCCCN(A5)CCN(A6)0.5C\\
 $C$&LHhLlHLlHHHHLLlHHhLl0.5l\\
 $S$&LlhLhLLhLlLhLLLHHLLL0.5l\\
 $R$&LLLLl0.5hLLlhlLLLLLLlHhL\\
  $Q_a$&lLHHHLLllLHHHHHH0.5hLLl\\
   $Q$&lLLHHhLLLLLlHHHHHLl0.5l\\
\extracode
%\begin{pgfonlayer}{background}
%\begin{scope}[semitransparent ,dashed]
%\vertlines[darkgray,dotted]{3.6,7.5,11.5,15.5,19.5,23.5,27.5}
%\foreach \n in {1,2,3,4,5,6}\draw(A\n.south)--(E\n.north);
%\foreach \n in {1}\draw(B\n.south)--(F\n.north);
%\foreach \n in {1}\draw(C\n.south)--(G\n.north);
%\end{scope}
%\end{pgfonlayer}
\end{tikztimingtable}
\end{otherlanguage}
\end{center}
\انتہا{سوال}
\ابتدا{سوال}
%6.6
 شکل  \حوالہ{شکل_ترتیبی_ثنائی_سلسلہ_وار_جمع_کار}    میں سلسلہ وار ثنائی جمع کار  پیش ہے۔اسے استعمال کرتے ہوئے \عددی{10110011_2} اور  \عددی{00110011_2} قدم با قدم   جمع کریں۔ ہر قدم پر  تمام مقامات پر متغیرات دریافت کریں۔
\انتہا{سوال}
\ابتدا{سوال}
%6.7
ایک ترتیبی  دور جس کے مداخل \عددی{x} اور  \عددی{y} جبکہ مخارج \عددی{z} ہے   میں دو ڈی پلٹ،  \عددی{A} اور \عددی{B} مستعمل ہیں ۔ دور کی مساوات درج ذیل   ہیں۔ یاد رہے ہم \عددی{A(t+1)} کو اگلا حال  جبکہ \عددی{A(t)} کو موجودہ حال یا باز رسی اشارہ   تصور کر سکتے ہیں۔
\begin{align*}
A(t+1)&=\overline{x}y+xA(t)\\
B(t+1)&=\overline{x}B(t)+xA(t)\\
z(t)&=x\overline{B}(t)
\end{align*}
\begin{enumerate}[a.]
\item
 ترتیبی دور  بنائیں۔ 
\item
  ان مساوات سے حال کا جدول حاصل کریں۔ 
\item
  حال  کے جدول سے حال کا خاکہ حاصل کریں۔
\end{enumerate}

جواب:
\begin{center}
\begin{tikzpicture}
\pgfmathsetmacro{\kpin}{0.5}
\pgfmathsetmacro{\kpina}{0.4}
\pgfmathsetmacro{\ksepY}{2.25}
\kDFF[u0]{0}{0}
\kDFF[u1]{0}{-\ksepY}
\draw(u0p1)--++(-\kpin,0)node[or port,number inputs=2,anchor=out](u2){};
\draw(u1p1)--++(-\kpin,0)node[or port,number inputs=2,anchor=out](u3){};
\draw(u2.in 1)--++(0,\kpina)--++(-\kpina,0)node[and port,anchor=out](u4){};
\draw(u2.in 2)--++(-\kpina,0)--++(0,-\kpina)node[and port,anchor=out](u5){};
\draw(u2.in 1)--(u3.in 1);
\draw(u3.in 2)node[and port,anchor=out](u6){};
\draw(u5.in 2)--++(-\kpin,0)node[not port,scale=0.7,anchor=out](u8){};
\draw(u8.in)--++(-3*\kpin,0)coordinate[pos=0.25](klftM)coordinate[pos=0.5](kkk)coordinate(klft)node[left]{$x$};
\draw(u5.in 1)--(u5.in 1 -|klft)node[left]{$y$};
\draw(klftM)node[draw,circle,fill=black,inner sep=0pt,minimum size=4pt]{}|-(u4.in 2);
\draw(u4.in 1)--++(0,1.2*\kpin)coordinate(kA)-|(u0p6);
\draw(u8.out)|-(u6.in 2);
\draw(u6.in 1)--++(0,2.25*\kpin)-|(u1p6);
\draw(u0p2)--(u1p2)--++(0,-2*\kpin)--++(-\kpin,0)node[left]{$C$};
\draw(u0p6)--++(\kpin,0)node[right]{$A$};
\draw(u1p6)--++(\kpin,0)node[right]{$B$};
\draw(0,-\ksepY-2.5*\kpin)node[and port,anchor=in 1](u10){};
\draw(u1p4)--++(0,-2*\kpin)-|(u10.in 1);
\draw(klftM)|-(u10.in 2);
\end{tikzpicture}
\end{center}
%
\begin{center}
\begin{otherlanguage}{english}
\begin{tabular}{CCCCC}
\toprule
&\multicolumn{2}{c}{x=1}&\multicolumn{2}{c}{x=0}\\
AB&y=1&y=0&y=1&y=0\\
\midrule
00&00&00&10&00\\
01&00&00&11&01\\
10&11&11&10&00\\
11&11&11&11&01\\
\bottomrule
\end{tabular}
\end{otherlanguage}
\end{center}
\انتہا{سوال}
\ابتدا{سوال}
%6.8
 مداخل \عددی{x} اور دو جے کے پلٹ،  \عددی{A} اور \عددی{B} ، پر مبنی ترتیبی دور درج ذیل مساوات   پر پورا اترتا ہے۔
 \begin{align*}
 J_A&=\overline{B}\\
 K_A&=x\\
 J_B&=A\\
 K_B&=x
 \end{align*}
 \begin{enumerate}[a.]
\item
ان سے حال کی مساوات  \عددی{A(t+1)} اور  \عددی{B(t+1)} حاصل کریں۔
\item 
ان  مساوات سے  حال کا خاکہ بنائیں۔
\end{enumerate}
%
جواب:
\begin{align*}
A(t+1)&=\overline{B}\,\overline{A}+\overline{x} A\\
B(t+1)&=A\overline{B}+xB
\end{align*}
\begin{center}
\begin{otherlanguage}{english}
\begin{tabular}{CCC}
\toprule
\text{\RL{موجودہ حال}}&\multicolumn{2}{c}{\text{\RL{اگلا حال}}}\\
\cline{2-3}
AB&x=1&x=0\\
\midrule
00&10&10\\
01&00&01\\
10&01&11\\
11&00&11\\
\bottomrule
\end{tabular}
\end{otherlanguage}
\end{center}
\انتہا{سوال}
\ابتدا{سوال}
%6.9
 دو  ڈی پلٹ، \عددی{A} اور \عددی{B}، استعمال کر کے مداخل \عددی{x} کا   ترتیبی دور تخلیق دیں جو بالترتیب  \عددی{00}، \عددی{01}، \عددی{10}،  اور \عددی{11} حال اختیار کر سکتا ہو۔بلند  مداخل  کی صورت میں بڑھتی  گنتی   اور پست مداخل  کی صورت میں گھٹتی گنتی  حاصل کرنی ہے۔ بڑھتی گنتی کی صورت میں \عددی{11} کو  پہنچنے کے بعد بلند  مداخل  کی صورت میں  دور اسی حال میں رہنا چاہیے ۔گھٹتی گنتی کرتے ہوئے \عددی{00} کو پہنچنے کے بعد پست مداخل کی صورت میں دور \عددی{00} میں رہنا چاہیے۔
 
 جواب: 
 
 \begin{minipage}{0.35\textwidth}
 \centering
 \begin{otherlanguage}{english}
 \begin{tabular}{CCC|CC}
 \toprule
 x&A&B&D_A&D_B\\
 \midrule
 0&0&0&0&0\\
 0&0&1&0&0\\
 0&1&0&0&1\\
 0&1&1&1&0\\
 1&0&0&0&1\\
 1&0&1&1&0\\
 1&1&0&1&1\\
 1&1&1&1&1\\
 \end{tabular}
 \end{otherlanguage}
 \end{minipage}\hfill
\begin{minipage}{0.65\textwidth}
\centering
\begin{tikzpicture}
\pgfmathsetmacro{\kpin}{1}
\pgfmathsetmacro{\ksepY}{3.75}
\kDFF[u0]{0}{0}
\kDFF[u1]{0}{-\ksepY}
\draw(u0p6)node[right]{$A$} (u1p6)node[right]{$B$};
\draw(u0p1)--++(-\kpin,0)node[or port,anchor=out](u2){u2};
\draw(u2.in 1)--++(0,\kpin)node[and port,anchor=out](u3){u3};
\draw(u2.in 2)--++(0,-\kpin)node[and port,anchor=out](u4){u4};
\draw(u3.in 1)--++(0,\kpin)node[xor port,anchor=out](u5){u5};
\draw(u5.in 1)node[left]{$x$}  (u5.in 2)node[left]{$A$};
\draw(u3.in 2)--(u3.in 2 -| u5.in 2)node[left]{$B$};
\draw(u4.in 2)--++(0,-\kpin)node[xor port,anchor=out](u6){u6};
\draw(u6.in 1)node[left]{$A$}  (u6.in 2)node[left]{$B$};
\draw(u4.in 1)--(u4.in 1 -| u5.in 2)node[left]{$x$};

\draw(u1p1)--++(-\kpin,0)node[or port,anchor=out](u20){u20};
\draw(u20.in 1)--++(0,\kpin)node[and port,anchor=out](u30){u30};
\draw(u20.in 2)--++(0,-\kpin)node[and port,anchor=out](u40){u40};
\draw(u30.in 1)--++(0,\kpin)node[xor port,anchor=out](u50){u50};
\draw(u50.in 1)node[left]{$x$}  (u50.in 2)node[left]{$A$};
\draw(u30.in 2)--(u30.in 2 -| u5.in 2)node[left]{$\overline{B}$};
\draw(u40.in 1)--(u40.in 1 -| u50.in 2)node[left]{$x$};
\draw(u40.in 2)--(u40.in 2 -| u50.in 2)node[left]{$A$};
\draw(u0p2)--(u1p2)--++(0,-\kpin)node[below]{$C$};
\end{tikzpicture} 
\end{minipage}
\انتہا{سوال}
\ابتدا{سوال}
%6.10
 گزشتہ سوال میں  مداخل  \عددی{e} کا اضافہ کریں۔ بلند  \عددی{e} کی صورت میں دور جوں کا توں چلتا ہو جبکہ پست \عددی{e} کی صورت میں  دور اپنا حال برقرار رکھتا ہو۔
 
 جواب: ساعت \عددی{C} کو ضرب گیٹ سے گزاریں۔ ضرب گیٹ کا دوسرا مداخل \عددی{e} ہو گا۔
\انتہا{سوال}
\ابتدا{سوال}
%6.11
 پچھلے سوال   میں  مداخل کی تعداد میں مزید اضافہ کرتے ہوئے مداخل \عددی{s}  کا اضافہ کریں۔ مداخل \عددی{s}  بلند کرنے سے دور کو حال \عددی{00} اختیار کر لینا چاہیے جبکہ پست  \عددی{s} کی صورت میں دور  کو  پہلے کی طرح کام کرنا چاہیے۔
 
 جواب:دونوں  ڈی پلٹ  کے بلند فعال  \اصطلاح{ زبردستی پست } مداخل کو \عددی{s}  فراہم کریں۔
\انتہا{سوال}
%==============


\باب{دفتر}
ایک پلٹ کار ایک ثنائی ہندسے (بِٹ) کی معلومات ذخیرہ کر سکتا ہے۔ آٹھ بِٹ معلومات ذخیرہ کرنے کے لئے آٹھ پلٹ کار درکار ہوں گے۔\اصطلاح{دفتر}\فرہنگ{دفتر}\حاشیہب{register}\فرہنگ{register} سے مراد وہ دور ہے جو معلومات ذخیرہ، اور ایک جگہ سے دوسری جگہ منتقل کر نے کی صلاحیت رکھتا ہو۔یوں، \عددی{n} بِٹ دفتر سے مراد \عددی{n} پلٹ کار پر مبنی وہ دور ہو گا، جو \عددی{n} بِٹ ذخیرہ اور منتقل کر سکے۔معلومات کے انتقال کا انداز (سلسلہ وار یا متوازی) دور کے ترکیبی حصہ پر منحصر ہو گا۔

\begin{figure}
\centering
\begin{subfigure}{0.45\textwidth}
\centering
\begin{tikzpicture}
\pgfmathsetmacro{\kshYa}{2.5}
\pgfmathsetmacro{\kshPa}{1.0}
\pgfmathsetmacro{\kshPb}{0.75}
\pgfmathsetmacro{\kshPc}{1.0}
\pgfmathsetmacro{\kshPd}{0.5}
\pgfmathsetmacro{\kshPe}{0.5}
\kDFF[u0]{0}{0};
\kDFF[u1]{0}{\kshYa};
\kDFF[u2]{0}{2*\kshYa};
\kDFF[u3]{0}{3*\kshYa};
\draw(u0p6)node[right]{$B_0$} (u1p6)node[right]{$B_1$} (u2p6)node[right]{$B_2$} (u3p6)node[right]{$B_3$};
\draw(u0p1)--++(-\kshPa,0)node[left]{$A_0$} (u1p1)--++(-\kshPa,0)node[left]{$A_1$} 
(u2p1)--++(-\kshPa,0)node[left]{$A_2$} (u3p1)--++(-\kshPa,0)node[left]{$A_3$};
\draw(u0p2)--++(-\kshPd,0) (u1p2)--++(-\kshPd,0) (u2p2)--++(-\kshPd,0) (u3p2)--++(-\kshPd,0);
\draw(u3p2)++(-\kshPd,0)--++(0,-3*\kshYa-2*\kshPd)node[below]{$C$}; 
\end{tikzpicture}
\caption{}
\end{subfigure}\hfill
\begin{subfigure}{0.45\textwidth}
\centering
\begin{tikzpicture}
\pgfmathsetmacro{\kshYa}{2.5}
\pgfmathsetmacro{\kshPa}{1.0}
\pgfmathsetmacro{\kshPb}{0.75}
\pgfmathsetmacro{\kshPc}{1.0}
\pgfmathsetmacro{\kshPd}{0.5}
\pgfmathsetmacro{\kshPe}{0.5}
\kDFFud[u0]{0}{0};
\kDFFud[u1]{0}{\kshYa};
\kDFFud[u2]{0}{2*\kshYa};
\kDFFud[u3]{0}{3*\kshYa};
\draw(u0p6)node[right]{$B_0$} (u1p6)node[right]{$B_1$} (u2p6)node[right]{$B_2$} (u3p6)node[right]{$B_3$};
\draw(u0p1)--++(-\kshPa,0)node[left]{$A_0$} (u1p1)--++(-\kshPa,0)node[left]{$A_1$} 
(u2p1)--++(-\kshPa,0)node[left]{$A_2$} (u3p1)--++(-\kshPa,0)node[left]{$A_3$};
\draw(u0pd)--++(-\kshPb,0) (u1pd)--++(-\kshPb,0) (u2pd)--++(-\kshPb,0) (u3pd)--++(-\kshPb,0);
\draw(u0pu)--++(-\kshPc,0) (u1pu)--++(-\kshPc,0) (u2pu)--++(-\kshPc,0) (u3pu)--++(-\kshPc,0);
\draw(u3pd)++(-\kshPb,0)--++(0,-3*\kshYa-\kshPd)coordinate(blw)node[below right]{$\overline{\text{\RL{بیٹھ}}}$};
\draw(u0pu)++(-\kshPc,0)--++(0,3*\kshYa+\kshPd)node[above]{$\overline{\text{\RL{اٹھ}}}$};
\draw(u0p2)--++(-\kshPd,0) (u1p2)--++(-\kshPd,0) (u2p2)--++(-\kshPd,0) (u3p2)--++(-\kshPd,0);
\draw(u3p2)++(-\kshPd,0)--++(0,-3*\kshYa)coordinate(cc)--(cc|-blw)node[below]{$C$}; 
\end{tikzpicture}
\caption{}
\end{subfigure}
\caption{چار بِٹ دفتر۔}
\label{شکل_دفتر_چار_بٹ_ڈی}
\end{figure}
 سادہ ترین چار بِٹ دفتر شکل \حوالہ{شکل_دفتر_چار_بٹ_ڈی} میں پیش ہے۔شکل-الف میں مداخل \عددی{A} جبکہ مخارج \عددی{B} ہے۔ مداخل کے چار بِٹ \عددی{A_0}، \عددی{A_1}، \عددی{A_2}، اور \عددی{A_3}،جبکہ مخارج کے چار بِٹ \عددی{B_0}، \عددی{B_1}، \عددی{B_2}، اور \عددی{B_3} ہیں ۔

ساعت کے کنارہ چڑھائی پر داخلی چار بِٹ پلٹ کار کو منتقل ہو جاتے ہیں۔ ہم کہتے ہیں دفتر میں مواد کا اندراج ہو گیا، یا مواد دفتر میں درج ہو گیا، یا مواد دفتر میں لکھ لیا گیا۔ساعت کے اگلے کنارہ چڑھائی تک یہ چار بِٹ معلومات دفتر میں محفوظ، اور مخارج پر دستیاب ہو گی۔

شکل \حوالہ{شکل_دفتر_چار_بٹ_ڈی}-ب میں بلند اور پست صلاحیت کا پلٹ کار استعمال کیا گیا۔یوں، ساعت کے کنارہ چڑھائی کا انتظار کیے بغیر، تمام خارجی بِٹ زبردستی بلند یا پست کیے جا سکتے ہیں۔ زبردستی پست کرنے سے دفتر صاف ہو کر \عددی{0000_2}، جبکہ زبردستی بلند کرنے سے \عددی{1111_2} خارج کرتا ہے۔ 

اس دور میں پلٹ کار کی تعداد \عددی{n} کر کے \عددی{n} بِٹ دفتر تشکیل دیا جا سکتا ہے۔ ہر بِٹ کا متمم بھی دفتر کے مخارج سے دستیاب ہو گا۔ یوں \عددی{B_0} کا متمم \عددی{\overline{B}_0} مطابقتی پلٹ کار کے \عددی{\overline{Q}} سے دستیاب ہو گا۔ 


\حصہ{سلسلہ وار دفتر}
\جزوحصہ{دائیں انتقال دفتر}
شکل \حوالہ{شکل_دفتر_دائیں_منتقل} میں (سلسلہ وار) \اصطلاح{ دائیں انتقال دفتر}\فرہنگ{دفتر!دائیں انتقال}\حاشیہب{shift right register}\فرہنگ{register!shift right} پیش ہے، جہاں (متواتر) ایک پلٹ کار کا مخارج، دوسرے کا مداخل ہے، اور ثنائی مواد، \عددی{x}، بائیں (جانب) سے مہیا کیا گیا ہے۔شکل میں زبردستی پست پن نہیں دکھایا گیا تا کہ اصل مضمون پر توجہ رہے، تاہم تصور کریں ساعت کے پہلے کنارہ چڑھائی سے قبل، تمام پلٹ کار زبردستی پست کیے گئے۔
\begin{figure}
\centering
\begin{tikzpicture}
\pgfmathsetmacro{\kshXa}{2.5}
\pgfmathsetmacro{\kshPa}{1.00}
\pgfmathsetmacro{\kshPb}{0.5}
\kDFF[u0]{0}{0};
\kDFF[u1]{-\kshXa}{0};
\kDFF[u2]{-2*\kshXa}{0};
\kDFF[u3]{-3*\kshXa}{0};
\draw(u3p6)node[above]{$Q_3$}--(u2p1) (u2p6)node[above]{$Q_2$}--(u1p1) (u1p6)node[above]{$Q_1$}--(u0p1) (u0p6)node[above]{$Q_0$} (u3p1)--++(-\kshPb,0)node[left]{$x$};
\draw(u0p2)--++(0,-\kshPa) (u1p2)--++(0,-\kshPa) (u2p2)--++(0,-\kshPa) (u3p2)--++(0,-\kshPa)coordinate(cc);
\draw(u0p2)++(0,-\kshPa)--(cc)--++(-\kshPb,0)node[left]{$C$};
\draw(u0-north-west)node[above]{$u_0$};
\draw(u1-north-west)node[above]{$u_1$};
\draw(u2-north-west)node[above]{$u_2$};
\draw(u3-north-west)node[above]{$u_3$};
\path(u1-south)++(0,-1);
\end{tikzpicture}
\caption{دائیں انتقال دفتر}
\label{شکل_دفتر_دائیں_منتقل}
\end{figure}


ساعت کے پہلے کنارہ چڑھائی پر \عددی{u_0} کو \عددی{Q_1=0}، \عددی{u_1} کو \عددی{Q_2=0}، \عددی{u_2} کو \عددی{Q_3=0}، اور \عددی{u_4} کو \عددی{x=1} مواد فراہم ہے، جنہیں پلٹ کار، ساعت کے کنارہ چڑھائی پر، مخارج منتقل کرتے ہیں۔یوں پہلے کنارہ چڑھائی گزرنے کے بعد \عددی{Q_0=0}، \عددی{Q_1=0}، \عددی{Q_2=0}، اور \عددی{Q_3=1} ہو گا۔یاد رہے، ساعت کے کنارہ چڑھائی کے دوران، پلٹ کار گزشتہ حال میں رہتا ہے، اور نیا مواد کنارہ گزرنے کے بعد مخارج کو پہنچتا ہے۔ آپ نے دیکھا، یہ دور، مواد کی دائیں رخ نقل مکانی کرتا ہے، جس کی وجہ سے اس کو\موٹا{دائیں انتقال دفتر} کہتے ہیں۔ 

ساعت کے دوسرے کنارہ چڑھائی کے وقت، \عددی{u_0} کو \عددی{Q_1=0}، \عددی{u_1} کو \عددی{Q_2=0}، \عددی{u_2} کو \عددی{Q_3=1}، اور \عددی{u_4} کو \عددی{x} (جو \عددی{0} یا \عددی{1} ہو گا) مواد فراہم ہے، لہٰذا ساعت کا دوسرا کنارہ چڑھائی گزرنے کے بعد \عددی{Q_0=0}، \عددی{Q_1=0}، \عددی{Q_2=1}، اور \عددی{Q_3=x} ہو گا۔

دور کو سلسلہ وار فراہم بائیں سے مواد، سلسلہ وار دائیں پلٹ کے مخارج \عددی{Q_0} سے اسی ترتیب میں حاصل کیا جا سکتا ہے۔

\جزوحصہ{بائیں انتقال دفتر}
شکل \حوالہ{شکل_دفتر_بائیں_منتقل} میں (سلسلہ وار) \اصطلاح{بائیں انتقال دفتر}\فرہنگ{دفتر! بائیں انتقال}\حاشیہب{shift left register}\فرہنگ{register!shift left} دکھایا گیا ہے، جو مواد کی بائیں نقل مکانی کرتا ہے۔اس کی بناوٹ بالکل دائیں انتقال دفتر کی طرح ہے۔فرق صرف اتنا ہے، بائیں انتقال دفتر میں دایاں پلٹ کار کا مخارج پڑوسی بایاں پلٹ کار کا مداخل ہے۔

ساعت کے کنارہ چڑھائی پر دایاں پلٹ کار فراہم کردہ مواد \عددی{x} کی نقل حاصل کر کے \عددی{Q_0} پر خارج کرتا ہے۔ اگلے کنارہ پر یہ مواد \عددی{Q_1} کو منتقل ہو گا۔ آپ دیکھ سکتے ہیں کہ یہاں مواد دائیں سے فراہم کیا گیا ہے، جو دور میں سے گزرتے ہوئے بائیں منتقل ہو گا۔

\begin{figure}
\centering
\begin{tikzpicture}
\pgfmathsetmacro{\kshXa}{2.5}
\pgfmathsetmacro{\kshPa}{1.0}
\pgfmathsetmacro{\kshPb}{0.3}
\pgfmathsetmacro{\kshPc}{0.3}
\kuDFF[u0]{0*\kshXa}{0}
\kuDFF[u1]{-1*\kshXa}{0}
\kuDFF[u2]{-2*\kshXa}{0}
\kuDFF[u3]{-3*\kshXa}{0}
\path(u0p6)--(u1p4)coordinate[pos=0.5](aau);
\path(u0p1)--(u1p3)coordinate[pos=0.5](aad);
\draw(u0p6)--(aau)--(aad)--(u1p1);
\path(u1p6)--(u2p4)coordinate[pos=0.5](aau);
\path(u1p1)--(u2p3)coordinate[pos=0.5](aad);
\draw(u1p6)--(aau)--(aad)--(u2p1);
\path(u2p6)--(u3p4)coordinate[pos=0.5](aau);
\path(u2p1)--(u3p3)coordinate[pos=0.5](aad);
\draw(u2p6)--(aau)--(aad)--(u3p1);
\draw(u0p1)--++(\kshPa,0)coordinate(rgtin);
\kinright[j1]{rgtin}
\draw(j1east)node[right]{$x$};
\draw(u0p6)--++(0,\kshPb)node[above]{$Q_0$};
\draw(u1p6)--++(0,\kshPb)node[above]{$Q_1$};
\draw(u2p6)--++(0,\kshPb)node[above]{$Q_2$};
\draw(u3p6)--++(0,\kshPb)node[above]{$Q_3$};
\draw(u0p2)--++(0,-\kshPc);
\draw(u1p2)--++(0,-\kshPc);
\draw(u2p2)--++(0,-\kshPc);
\draw(u3p2)--++(0,-\kshPc);
\draw(u0p2)++(0,-\kshPc)--($(u3p2)+(0,-\kshPc)$)--++(-\kshPa,0)node[left]{$C$};
\end{tikzpicture}
\caption{بائیں انتقال دفتر}
\label{شکل_دفتر_بائیں_منتقل}
\end{figure}


\جزوحصہ{ دائیں و بائیں انتقال دفتر}
شکل \حوالہ{شکل_دفتر_بائیں_و_دائیں} میں (سلسلہ وار) بائیں و دائیں انتقال دفتر پیش ہے جو مواد کی بائیں یا دائیں نقل مکانی کی صلاحیت رکھتا ہے۔ مخارج \عددی{Q_2} پلٹ کار کے مداخل \عددی{D} اور اس سے منسلک جمع گیٹ اور (دو) ضرب گیٹ پر توجہ رکھیں۔ قابو اشارہ \عددی{(\text{دائیں}/{\overline{\text{بائیں}}})} بلند ہونے کی صورت میں، دایاں ضرب گیٹ معذور جبکہ بایاں مجاز ہو کر، جمع گیٹ تک \عددی{Q_3} پہنچاتے ہیں جو \عددی{D} پر دستیاب اور ساعت کے اگلے کنارہ چڑھائی پر پلٹ کار میں درج ہو کر بطور \عددی{Q_2} خارج ہو گا۔یوں مواد \عددی{Q_3} سے \عددی{Q_2} یعنی دائیں منتقل ہوا۔اس کے برعکس قابو اشارہ پست ہونے کی صورت میں، دایاں ضرب گیٹ مجاز اور بایاں معذور ہو کر، جمع گیٹ تک \عددی{Q_1} پر موجود مواد پہنچاتے ہیں، جو آخر کار \عددی{Q_2} پہنچتا ہے، اور یوں مواد بائیں منتقل ہوتا ہے۔

 بائیں ترین پلٹ کار کو بیرونی مواد \عددی{y} جبکہ دائیں ترین کو \عددی{x} فراہم کیا گیا ہے۔ قابو اشارہ ان میں سے ایک منتخب کر تا ہے جو مطلوبہ سمت ( دائیں یا بائیں) منتقل ہو گا۔ 

بائیں نقل مکانی کے دوران \عددی{x} پر میسر مواد ساعت کے کنارہ چڑھائی پر \عددی{Q_0} پہنچتا ہے۔ اگلے کنارہ پر یہی مواد \عددی{Q_1}، اس سے اگلے پر \عددی{Q_2} اور آخر میں \عددی{Q_3} پہنچتا ہے۔ دائیں نقل مکانی کی صورت میں \عددی{y} پر موجود مواد الٹ رخ \عددی{Q_3} سے \عددی{Q_0} نقل مکانی کرتا ہے۔
\begin{figure}
\centering
\begin{tikzpicture}
\pgfmathsetmacro{\ksepYa}{2.75}
\pgfmathsetmacro{\kpinA}{0.5}
\pgfmathsetmacro{\kpinB}{0.4}
\kuDFF[u0]{0}{0}
\kuDFF[u1]{-1*\ksepYa}{0}
\kuDFF[u2]{-2*\ksepYa}{0}
\kuDFF[u3]{-3*\ksepYa}{0}
\draw(u0p1)node[or port,scale=1,number inputs=2,rotate=90,anchor=out](u5){};
\draw(u5.in 1)--++(-\kpinB,0)node[and port,scale=1,number inputs=2,rotate=90,anchor=out](u6){};
\draw(u5.in 2)--++(\kpinB,0)node[and port,scale=1,number inputs=2,rotate=90,anchor=out](u7){};
\draw(u1p1)node[or port,scale=1,number inputs=2,rotate=90,anchor=out](u8){};
\draw(u8.in 1)--++(-\kpinB,0)node[and port,scale=1,number inputs=2,rotate=90,anchor=out](u9){};
\draw(u8.in 2)--++(\kpinB,0)node[and port,scale=1,number inputs=2,rotate=90,anchor=out](u10){};
\draw(u2p1)node[or port,scale=1,number inputs=2,rotate=90,anchor=out](u11){};
\draw(u11.in 1)--++(-\kpinB,0)node[and port,scale=1,number inputs=2,rotate=90,anchor=out](u12){};
\draw(u11.in 2)--++(\kpinB,0)node[and port,scale=1,number inputs=2,rotate=90,anchor=out](u13){};
\draw(u3p1)node[or port,scale=1,number inputs=2,rotate=90,anchor=out](u14){};
\draw(u14.in 1)--++(-\kpinB,0)node[and port,scale=1,number inputs=2,rotate=90,anchor=out](u15){};
\draw(u14.in 2)--++(\kpinB,0)node[and port,scale=1,number inputs=2,rotate=90,anchor=out](u16){};
\draw(u0p6)node[above]{$Q_0$};
\draw(u1p6)node[above]{$Q_1$};
\draw(u2p6)node[above]{$Q_2$};
\draw(u3p6)node[above]{$Q_3$};
\draw(u7.in 2)--++(\kpinA,0)coordinate(rgtin) (u6.in 1)node[below]{$Q_1$};
\kinright[j2]{rgtin}
\draw(j2east)node[right]{$x$};
\draw(u10.in 2)node[below]{$Q_0$} (u9.in 1)node[below]{$Q_2$};
\draw(u13.in 2)node[below]{$Q_1$} (u12.in 1)node[below]{$Q_3$};
\draw(u16.in 2)node[below]{$Q_2$} (u15.in 1)--++(-\kpinA,0)coordinate(lft);
\kinleft[j1]{lft}
\draw(j1west)node[left]{$y$};
\draw(u0p2)--(u3p2)--(u3p2 -|lft)node[left]{$C$};
\draw(u16.in 1)--++(0,-3*\kpinA)node[not port,scale=1,anchor=out](u17){};
\draw(u17.out)-|(u7.in 1);
\draw(u10.in 1)coordinate(aa)--(aa |- u17.out);
\draw(u13.in 1)coordinate(aa)--(aa |- u17.out);
\draw(u6.in 2)--++(0,-1.3*\kpinA)-|coordinate(bb)(u15.in 2) (bb)-|(u17.in)--++(-\kpinA,0)node[left]{$\text{دائیں}/{\overline{\text{بائیں}}}$};
\draw(u9.in 2)coordinate(aa)--(aa |- bb);
\draw(u12.in 2)coordinate(aa)--(aa |- bb);
\end{tikzpicture}
\caption{بائیں و دائیں انتقال دفتر}
\label{شکل_دفتر_بائیں_و_دائیں}
\end{figure}


\حصہ{متوازی بھرائی دفتر}
بعض اوقات، دفتر میں بیک وقت مواد چڑھانے کی ضرورت پیش آتی ہے۔ شکل \حوالہ{شکل_دفتر_متوازی_انتقال_دفتر} میں \اصطلاح{دائیں انتقال، متوازی بھرائی دفتر}\فرہنگ{دفتر!متوازی بھرائی}\حاشیہب{parallel load, right shift register}\فرہنگ{register!parallel load} پیش ہے، جس میں متوازی مواد بیک وقت چڑھانا ممکن ہے۔یہ مختصراً \موٹا{متوازی دائیں انتقال دفتر} کہلاتا ہے۔

\begin{figure}
\centering
\begin{tikzpicture}
\pgfmathsetmacro{\ksepYa}{2.75}
\pgfmathsetmacro{\kpinA}{0.5}
\pgfmathsetmacro{\kpinB}{0.4}
\pgfmathsetmacro{\kpinC}{2.00}
\kuDFF[u0]{0}{0}
\kuDFF[u1]{-1*\ksepYa}{0}
\kuDFF[u2]{-2*\ksepYa}{0}
\kuDFF[u3]{-3*\ksepYa}{0}
\draw(u0p1)node[or port,scale=1,number inputs=2,rotate=90,anchor=out](u5){};
\draw(u5.in 1)--++(-\kpinB,0)node[and port,scale=1,number inputs=2,rotate=90,anchor=out](u6){};
\draw(u5.in 2)--++(\kpinB,0)node[and port,scale=1,number inputs=2,rotate=90,anchor=out](u7){};
\draw(u1p1)node[or port,scale=1,number inputs=2,rotate=90,anchor=out](u8){};
\draw(u8.in 1)--++(-\kpinB,0)node[and port,scale=1,number inputs=2,rotate=90,anchor=out](u9){};
\draw(u8.in 2)--++(\kpinB,0)node[and port,scale=1,number inputs=2,rotate=90,anchor=out](u10){};
\draw(u2p1)node[or port,scale=1,number inputs=2,rotate=90,anchor=out](u11){};
\draw(u11.in 1)--++(-\kpinB,0)node[and port,scale=1,number inputs=2,rotate=90,anchor=out](u12){};
\draw(u11.in 2)--++(\kpinB,0)node[and port,scale=1,number inputs=2,rotate=90,anchor=out](u13){};
\draw(u3p1)node[or port,scale=1,number inputs=2,rotate=90,anchor=out](u14){};
\draw(u14.in 1)--++(-\kpinB,0)node[and port,scale=1,number inputs=2,rotate=90,anchor=out](u15){};
\draw(u14.in 2)--++(\kpinB,0)node[and port,scale=1,number inputs=2,rotate=90,anchor=out](u16){};
\kupinAnchors
\kuBuffer[u17]{\pfx}{\kpinB+\pfy}
\draw(u17pout)--++(0,\kpinA)node[above]{$D_0$};
\kuBuffer[u18]{-1*\ksepYa+\pfx}{\kpinB+\pfy}
\draw(u18pout)--++(0,\kpinA)node[above]{$D_1$};
\kuBuffer[u19]{-2*\ksepYa+\pfx}{\kpinB+\pfy}
\draw(u19pout)--++(0,\kpinA)node[above]{$D_2$};
\kuBuffer[u20]{-3*\ksepYa+\pfx}{\kpinB+\pfy}
\draw(u20pout)--++(0,\kpinA)node[above]{$D_3$};
\draw(u0p6)++(0,0.25)--++(\kpinA,0)coordinate(kkt);
\koutright[j0]{kkt}
\draw(j0east)node[right]{\text{\RL{سلسلہ وار مخارج}}};
\draw(u0p6)--++(0,\kpinB);
\draw(u1p6)--++(0,\kpinB);
\draw(u2p6)--++(0,\kpinB);
\draw(u3p6)--++(0,\kpinB);
\draw(u0p6)node[above left]{$Q_0$};
\draw(u1p6)node[above left]{$Q_1$};
\draw(u2p6)node[above left]{$Q_2$};
\draw(u3p6)node[above left]{$Q_3$};
\draw(u7.in 2)--++(0,-\kpinC)node[below]{$z_0$} (u6.in 1)node[below]{$Q_1$};
\draw(u10.in 2)--++(0,-\kpinC)node[below]{$z_1$} (u9.in 1)node[below]{$Q_2$};
\draw(u13.in 2)--++(0,-\kpinC)node[below]{$z_2$} (u12.in 1)node[below]{$Q_3$};
\draw(u16.in 2)--++(0,-\kpinC)node[below]{$z_3$} (u15.in 1)--++(-\kpinA,0)coordinate(lft);
\kinleft[j1]{lft}
\draw(j1west)node[left]{$y$};
\draw(u0p2)--(u3p2)--(u3p2 -|lft)node[left]{$C$};
\draw(u16.in 1)--++(0,-3*\kpinA)node[not port,scale=1,anchor=out](u21){};
\draw(u21.out)-|(u7.in 1);
\draw(u10.in 1)coordinate(aa)--(aa |- u21.out);
\draw(u13.in 1)coordinate(aa)--(aa |- u21.out);
\draw(u6.in 2)--++(0,-1.3*\kpinA)-|coordinate(bb)(u15.in 2) (bb)-|(u21.in)--++(-\kpinA,0)
node[left]{$\overline{\text{متوازی بھرائی}}$};
\draw(u9.in 2)coordinate(aa)--(aa |- bb);
\draw(u12.in 2)coordinate(aa)--(aa |- bb);
\draw(u17pbu)--(u17u);
\draw(u17pnu)node[ocirc]{};
\draw(u18pbu)--(u18u);
\draw(u18pnu)node[ocirc]{};
\draw(u19pbu)--(u19u);
\draw(u19pnu)node[ocirc]{};
\draw(u20pbu)--(u20u);
\draw(u20pnu)node[ocirc]{};
\draw(u17u)--++(0,1.5*\kpinA)coordinate(aa);
\draw(u18u)--++(0,1.5*\kpinA);
\draw(u19u)--++(0,1.5*\kpinA);
\draw(u20u)--++(0,1.5*\kpinA)coordinate(bb);
\draw(aa)--(bb)--++(-\kpinA,0)node[left]{$\overline{\text{\RL{متوازی خارج}}}$};
%\draw(u20pbd)--(u20d);
%\draw(u20pnd)node[ocirc]{};
%\draw(u20pnpin)node[ocirc]{};
%\draw(u20pnpout)node[ocirc]{};
\end{tikzpicture}
\caption{متوازی دائیں انتقال دفتر}
\label{شکل_دفتر_متوازی_انتقال_دفتر}
\end{figure}

پلٹ کار کو جمع گیٹ معلومات فراہم کرتا ہے جس کو دو ضرب گیٹ مواد فراہم کرتے ہیں۔ قابو اشارہ \عددی{\overline{\text{متوازی بھرائی}}} عام طور غیر فعال (بلند) رکھا جاتا ہے۔یوں دایاں ضرب گیٹ معذور جبکہ بایاں گیٹ مجاز ہو کر، بائیں پلٹ کار کا مخارج، جمع گیٹ کے راستے پلٹ کار کو فراہم کرتا ہے، جو ساعت کے اگلے کنارہ چڑھائی پر پلٹ کار میں درج ہو گا۔

 مواد \عددی{z_0} تا \عددی{z_3} پلٹ کار میں چڑھانے کے لئے \عددی{\overline{\text{متوازی بھرائی}}} پست کیا جاتا ہے۔ یوں پلٹ کار کو مواد فراہم کرنے والا بایاں ضرب گیٹ معذور جبکہ دایاں مجاز ہو گا۔ مجاز گیٹ متوازی مواد کو جمع گیٹ کے راستہ پلٹ کار تک پہنچاتا ہے۔
 
یوں پلٹ کار میں مواد سلسلہ وار \عددی{(y)} یا متوازی (\عددی{z_0} تا \عددی{z_3}) بھرا جا سکتا ہے۔
 
شکل میں پلٹ کار کا مخارج، مجاز و معذور صلاحیت مستحکم کار سے منسلک کیا گیا ہے۔ قابو اشارہ \عددی{\overline{\text{\RL{متوازی خارج}}}} پست کر کے پلٹ کار کا مواد \عددی{Q_0} تا \عددی{Q_3} بطور \عددی{D_0} تا \عددی{D_3} حاصل کیا جا سکتا ہے۔قابو اشارہ معذور (بلند) ہونے کی صورت میں مستحکم کار کا مخارج بلند رکاوٹ حال میں ہو گا۔


\حصہ{عالمگیر انتقال دفتر}
ہم مختلف صلاحیت کے دفاتر پر غور کر چکے، جن کی خوبیاں ایک دور میں سموئی جا سکتی ہیں۔ایسا ایک \اصطلاح{عالمگیر انتقال دفتر }\فرہنگ{انتقال دفتر!عالمگیر}\حاشیہب{universal shift register}\فرہنگ{shift register!universal} شکل \حوالہ{شکل_دفتر_عالمگیر_انتقال_دفتر} میں پیش ہے۔

\begin{figure}
\centering
\begin{tikzpicture}
\pgfmathsetmacro{\ksepXa}{2.75}
\pgfmathsetmacro{\ksepXb}{0.20}
\pgfmathsetmacro{\ksepYa}{3.5}
\pgfmathsetmacro{\ksepYb}{1.25}
\pgfmathsetmacro{\ksepYc}{1}%{0.75}
\pgfmathsetmacro{\ksepYd}{1.1}%{0.85}
\pgfmathsetmacro{\ksepYe}{0.65}
\pgfmathsetmacro{\kpinA}{0.5}
\pgfmathsetmacro{\kpinB}{0.4}
\pgfmathsetmacro{\kpinC}{2.00}
\kuDFF[u0]{0}{0}
\kuDFF[u1]{-1*\ksepXa}{0}
\kuDFF[u2]{-2*\ksepXa}{0}
\kuDFF[u3]{-3*\ksepXa}{0}
\kupinAnchors
\kuBuffer[u17]{\pfx}{\kpinB+\pfy}
\draw(u17pout)--++(0,\kpinA/3)node[above]{$D_0$};
\kuBuffer[u18]{-1*\ksepXa+\pfx}{\kpinB+\pfy}
\draw(u18pout)--++(0,\kpinA/3)node[above]{$D_1$};
\kuBuffer[u19]{-2*\ksepXa+\pfx}{\kpinB+\pfy}
\draw(u19pout)--++(0,\kpinA/3)node[above]{$D_2$};
\kuBuffer[u20]{-3*\ksepXa+\pfx}{\kpinB+\pfy}
\draw(u20pout)--++(0,\kpinA/3)node[above]{$D_3$};
\draw(u17pbu)--++(-\kpin,0)--++(0,\ksepYe)coordinate(tt);
\draw(u17pnu)node[ocirc]{};
\draw(u18pbu)--++(-\kpin,0)--++(0,\ksepYe);
\draw(u18pnu)node[ocirc]{};
\draw(u19pbu)--++(-\kpin,0)--++(0,\ksepYe);
\draw(u19pnu)node[ocirc]{};
\draw(u20pbu)--++(-\kpin,0)--++(0,\ksepYe)coordinate(ttt);
\draw(u20pnu)node[ocirc]{};
\draw(u0p6)--++(0,\kpinB);
\draw(u1p6)--++(0,\kpinB);
\draw(u2p6)--++(0,\kpinB);
\draw(u3p6)--++(0,\kpinB);
\draw(u0p6)node[above left]{$Q_0$}++(0,0.25)coordinate(ka);
\draw(u1p6)node[above left]{$Q_1$};
\draw(u2p6)node[above left]{$Q_2$};
\draw(u3p6)node[above right]{$Q_3$}++(0,0.25)coordinate(kb);
\koutright[j0]{ka}
\draw(j0east)node[right,yshift=1ex]{\text{\RL{دائیں خروج}}};
\koutleft[j1]{kb}
\draw(j1west)node[left]{\text{\RL{بائیں خروج}}};
\kupinAnchors									%gives values of pax etc
\kmuxD[u5]{-0*\ksepXa-\kul-2*\kpsep+\pax}{-\ksepYa}
\kmuxD[u6]{-1*\ksepXa-\kul-2*\kpsep+\pax}{-\ksepYa}
\kmuxD[u7]{-2*\ksepXa-\kul-2*\kpsep+\pax}{-\ksepYa}
\kmuxD[u8]{-3*\ksepXa-\kul-2*\kpsep+\pax}{-\ksepYa}
\draw(u0pbd)--++(\kpin,0)--++(0,-\ksepYb)coordinate(ll);
\draw(u0pnd)node[ocirc]{};
\draw(u1pbd)--++(\kpin,0)--++(0,-\ksepYb);
\draw(u1pnd)node[ocirc]{};
\draw(u2pbd)--++(\kpin,0)--++(0,-\ksepYb);
\draw(u2pnd)node[ocirc]{};
\draw(u3pbd)--++(\kpin,0)--++(0,-\ksepYb)coordinate(rr);
\draw(u3pnd)node[ocirc]{};
\draw(rr)++(-\ksepXa-1,0)coordinate(lft);
\draw(rr)--(ll)--++(0,-0.5)node[below]{$\overline{\text{بیٹھ}}$};
\draw(u0p2)--(u3p2)--++(-1,0)node[left]{$C$};
\draw(tt)--(ttt)--++(-0.75,0)node[above]{$\overline{\text{\RL{متوازی خارج}}}$};
\draw(u0p1)--(u5dp2);
\draw(u1p1)--(u6dp2);
\draw(u2p1)--(u7dp2);
\draw(u3p1)--(u8dp2);
\draw(u5ap2)--++(0,\ksepYc)coordinate(aa);
\draw(u5ap1)--++(-\ksepXb,0)--++(0,\ksepYd)coordinate(bb);
\draw(u6ap2)--++(0,\ksepYc);
\draw(u6ap1)--++(-\ksepXb,0)--++(0,\ksepYd);
\draw(u7ap2)--++(0,\ksepYc);
\draw(u7ap1)--++(-\ksepXb,0)--++(0,\ksepYd);
\draw(u8ap2)--++(0,\ksepYc)coordinate(aaa);
\draw(u8ap1)--++(-\ksepXb,0)--++(0,\ksepYd)coordinate(bbb);
\draw(aa)--(aaa)--++(0,0.25)coordinate(abab)node[above, xshift=0.75ex]{$a_1$};
\draw(bb)--(bbb)--(bbb |-abab)node[above,xshift=-0.75ex]{$a_0$};
\draw(u5bp0)node[below]{$y$};
\draw(u5bp1)node[below]{$Q_1$};
\draw(u5bp2)node[below]{$z_0$};
\draw(u5bp3)node[below]{$Q_0$};
\draw(u6bp0)node[below]{$Q_0$};
\draw(u6bp1)node[below]{$Q_2$};
\draw(u6bp2)node[below]{$z_1$};
\draw(u6bp3)node[below]{$Q_1$};
\draw(u7bp0)node[below]{$Q_1$};
\draw(u7bp1)node[below]{$Q_3$};
\draw(u7bp2)node[below]{$z_2$};
\draw(u7bp3)node[below]{$Q_2$};
\draw(u8bp0)node[below]{$Q_2$};
\draw(u8bp1)node[below]{$x$};
\draw(u8bp2)node[below]{$z_3$};
\draw(u8bp3)node[below]{$Q_3$};
\end{tikzpicture}
\caption{چار بِٹ ، عالمگیر  انتقال دفتر}
\label{شکل_دفتر_عالمگیر_انتقال_دفتر}
\end{figure}


بائیں انتقال کے دوران مواد \عددی{y} پر\اصطلاح{ سلسلہ وار داخل }\فرہنگ{سلسلہ وار!داخل}\حاشیہب{serial in}\فرہنگ{serial in} ہو کر آخر کار بائیں \اصطلاح{ خروج }\فرہنگ{خروج}\حاشیہب{output}\فرہنگ{output} سے \اصطلاح{ سلسلہ وار خارج }\فرہنگ{سلسلہ وار!خارج}\حاشیہب{serial out}\فرہنگ{serial out} ہو گا، جبکہ دائیں انتقال کے دوران مواد \عددی{x} سے سلسلہ وار داخل ہو کر آخر کار دائیں خروج سے سلسلہ وار خارج ہو گا۔

شکل \حوالہ{شکل_دفتر_عالمگیر_انتقال_دفتر} میں چار یکساں حصے ہیں، جن کی کارکردگی ایک جیسی ہے۔دایاں حصہ پر غور کرتے ہیں۔


پلٹ کار کے ساتھ \اصطلاح{ چار سے ایک منتخب کنندہ }جوڑا گیا ہے ۔ پتہ کے دو بِٹ \عددی{a_0} اور \عددی{a_1} مداخل میں سے ایک چن کر خارجی پن پہنچاتے ہیں۔مداخل کا انتخاب درج ذیل جدول کے تحت ہو گا۔
 \begin{align*}
 \begin{array}{cc|cr}
 a_1& a_0&D_0&\\
 \toprule
 0&0&Q_0&\text{\RL{حال برقرار}}\\
 0&1&z_0&\text{\RL{متوازی داخل}}\\
 1&0&Q_1&\text{\RL{دائیں انتقال}}\\
 1&1&y&\text{\RL{بائیں انتقال}}
 \end{array}
 \end{align*}

 پتہ \عددی{00_2} مواد \عددی{Q_0} منتخب کر کے پلٹ کار کے مداخل پر مہیا کرتا ہے جو اگلے کنارہ ساعت پر پلٹ کار کے خارجی پن پر خارج ہو گا۔یوں دفتر اپنا حال برقرار رکھے گا (اور مواد دائیں یا بائیں منتقل نہیں ہو گا)۔
 
پتہ \عددی{01_2} مواد \عددی{z_0} پلٹ کار کو مہیا کرے گا جو ساعت کے اگلے کنارہ پلٹ کار کے مخارج پر نمودار ہو گا۔چونکہ \عددی{z_0} متوازی مہیا کردہ مواد ہے لہٰذا متوازی مواد دفتر میں چڑھے گا۔

پتہ \عددی{10_2} پلٹ کار کو \عددی{Q_1} مہیا کرے گا۔یوں موجودہ \عددی{Q_1} ساعت کے اگلے کنارے پر بطور \عددی{Q_0}نمودار ہو گا۔یعنی دفتر مواد دائیں منتقل کرے گا۔

پتہ \عددی{11_2} سلسلہ وار مہیا کردہ مواد \عددی{y} منتخب کرے گا جو ساعت کے اگلے کنارہ پر بطور \عددی{Q_0} نمودار ہو گا۔یوں دفتر مواد بائیں منتقل کرے گا۔

مذکورہ بالا تجزیہ باقی تین حصوں پر لاگو کر کے عالم گیر دفتر کی کارکردگی جدول میں پیش کرتے ہیں۔
 \begin{align*}
 \begin{array}{cc|ccccr}
 a_1& a_0&D_3&D_2&D_1&D_0&\\
 \toprule
 0&0&Q_3&Q_2&Q_1&Q_0&\text{\RL{حال برقرار}}\\
 0&1&z_3&z_2&z_1&z_0&\text{\RL{متوازی داخل}}\\
 1&0&x&Q_3&Q_2&Q_1&\text{\RL{دائیں انتقال}}\\
 1&1&Q_2&Q_1&Q_0&y&\text{\RL{بائیں انتقال}}
 \end{array}
 \end{align*}

\ابتدا{مشق}
انٹرنیٹ سے عالمگیر  انتقال دفتر \عددی{74194} کے معلوماتی صفحات حاصل کریں ۔  یہ کتنے بِٹ کا عالمگیر انتقال  دفتر ہے؟
\انتہا{مشق}

 
 \حصہ{ سلسلہ وار ثنائی جمع کار}
صفحہ \حوالہصفحہ{شکل_ترتیبی_ثنائی_سلسلہ_وار_جمع_کار} پر شکل \حوالہ{شکل_ترتیبی_ثنائی_سلسلہ_وار_جمع_کار} میں سلسلہ وار ثنائی جمع کار پیش ہے جس کو استعمال کرکے شکل  \حوالہ{شکل_دفتر_متعدد_ہندسی_جمع_کار}  میں پیش متعدد بِٹ\اصطلاح{  سلسلہ وار ثنائی جمع کار }\فرہنگ{جمع کار!سلسلہ وار، متعدد بِٹ}\حاشیہب{serial adder}\فرہنگ{serial adder!multibit} حاصل کیا گیا ۔ یہاں  \عددی{n} بِٹ متوازی دائیں انتقال دفتر (ا اور ب) مستعمل ہیں۔
	
\begin{figure}
\centering
\begin{tikzpicture}
\pgfmathsetmacro\cdelX{0.15}
\pgfmathsetmacro\cdelY{0.15}
\pgfmathsetmacro\kshiftX{5}
\pgfmathsetmacro\kshiftY{2}
\pgfmathsetmacro{\kpin}{0.50}
\pgfmathsetmacro{\kpsep}{0.60}			%pin to pin distance
\pgfmathsetmacro{\kulV}{0.50}			%edge clearance
\pgfmathsetmacro{\kdimXa}{2*\kulV+2*\kpsep}
\pgfmathsetmacro{\kdimYa}{2*\kulV+1*\kpsep}		%one spaces between 2 pins
\pgfmathsetmacro{\kdimXb}{2*\kulV+0*\kpsep}
\pgfmathsetmacro{\kdimYb}{2*\kulV+2*\kpsep}		%two spaces between 3 pins
\draw[thick] (0,0) rectangle ++(\kdimXa,\kdimYa)node[pos=0.5]{دفتر-ا};
\draw[thick] (0,-\kshiftY) rectangle ++(\kdimXa,\kdimYa)node[pos=0.5]{دفتر-ب};
\draw[thick] (\kshiftX,0) rectangle ++(\kdimXb,\kdimYb) node[pos=0.5]{جمع کار};
\draw[thick] (\kshiftX,-\kshiftY+\kdimYa) rectangle ++(\kdimXb,-\kdimYb)node[pos=0.5,below]{پلٹ};
\draw[] (\kshiftX+0.5*\kdimXb,-\kshiftY+\kdimYa-\kdimYb)--++(0,-\kulV)--++(-\kpin,0)node[left]{$\overline{\text{\RL{زبردستی پست}}}$};
\draw[thick] (\kshiftX+0.5*\kdimXb,-\kshiftY+\kdimYa-\kdimYb)++(0,-0.05)node[ocirc, fill=none]{};
\draw[thin](\kdimXa,2/3*\kdimYa)--++(\kpin,0)|- (\kshiftX,1*\kulV+2*\kpsep)node[above left]{$y$};
\draw(\kdimXa,-\kshiftY+2/3*\kdimYa)--++(2*\kpin,0)|- (\kshiftX,\kdimYb/2)node[above left]{$z$};
\draw(\kshiftX,-\kshiftY+\kdimYa-\kulV)node[right]{$Q$}--++(-1*\kpin,0)node[rotate=90, above right]{حاصل} |-(\kshiftX,\kulV);
\draw(\kshiftX+\kdimXb,-\kshiftY+\kdimYa-\kulV)node[left]{$D$}--++(1*\kpin,0) |-(\kshiftX+\kdimXb,\kulV)node[left]{$c$}node[above right]{$C$};
\draw(0,\kulV+1*\kpsep)--++(-1*\kpin,0)--++(0,2.5*\kpsep)--++(2*\pin+\kshiftX+\kdimXb+\kpin,0) |-(\kshiftX+\kdimXb,1*\kulV+2*\kpsep)node[left]{$s$}node[above right]{$S$};
\draw[thick] (0,\kulV-\cdelY)--++(\cdelX,\cdelY)--++(-\cdelX,\cdelY);
\draw[thick] (0,-\kshiftY+\kulV-\cdelY)--++(\cdelX,\cdelY)--++(-\cdelX,\cdelY);
\draw[thick] (\kshiftX+\kdimXb,-\kshiftY+\kdimYa-0.5*\kdimYb-\cdelY)--++(-\cdelX,\cdelY)--++(\cdelX,\cdelY);
\draw(\kshiftX+\kdimXb,-\kshiftY+\kdimYa-0.5*\kdimYb)--++(\kpin,0)--++(0,-\kdimYb/2-2*\kpin)coordinate(kL);
\draw(0,\kulV)--++(-\kpin,0)--++(0,-\kshiftY)coordinate(ck)--++(\kpin,0);
\draw(ck)|-(kL) (ck)node[circ]{}--++(-\kpin,0)node[left]{ساعت};
\end{tikzpicture}
\caption{متعدد بِٹ سلسلہ وار ثنائی جمع کار}
\label{شکل_دفتر_متعدد_ہندسی_جمع_کار}
\end{figure}
 ساعت کے پہلے کنارے سے قبل (یعنی مجموعہ لینے سے قبل)، دفتر۔ا میں ثنائی عدد \عددی{y}، دفتر-ب میں ثنائی عدد \عددی{z} متوازی منتقل کیے جاتے ہیں اور زبردستی پست اشارہ لمحاتی پست کر کے ڈی پلٹ کار پست کیا جاتا ہے ( تا کہ مکمل جمع کار کا داخلی حاصل \عددی{0} ہو)۔شکل میں متوازی چڑھائی نہیں دکھائی گئی تا کہ اصل موضوع پر توجہ رہے۔

مکمل جمع کار ان دو ثنائی اعداد کے کم تر رتبی بِٹ اور داخلی حاصل \عددی{0} جمع کر کے جمع \عددی{s_0} اور خارجی حاصل \عددی{c_1} خارج کرتا ہے۔ساعت کے پہلے کنارے پر \عددی{c_1} کو ڈی پلٹ کار محفوظ کر کے اگلے ثنائی بِٹ کی جمع کے دوران مکمل جمع کار کو بطور داخلی حاصل فراہم کرتا ہے جبکہ دفتر-ا اور دفتر-ب اگلے ثنائی بِٹ فراہم کرتے ہیں۔جمع \عددی{s_0} شکل میں دفتر-ا کو سلسلہ وار مداخل کے طور مہیا کیا گیا ہے۔یوں جیسے جیسے دفتر ثنائی عدد \عددی{y} دائیں جانب خارج کرتا ہے ویسے ویسے اس کی جگہ دو اعداد کا مجموعہ جگہ لیتا ہے۔ساعت کے \عددی{n} کنارے گزرنے کے بعد دو ثنائی اعداد کا مجموعہ دفتر-ا میں محفوظ ہو گا جہاں سے اسے متوازی پڑھا جا سکتا ہے جبکہ مجموعے کا آخری حاصل مکمل جمع کار کے مخارج \عددی{c} سے پڑھا جا سکتا ہے۔

\حصہء{سوالات}
\ابتدا{سوال}
%7.1
 چار بٹ  سلسلہ وار دائیں منتقل دفتر میں ابتدائی ثنائی مواد  \عددی{1011} موجود ہے۔دفتر کا مخارج  اسی دفتر کو بطور مداخل مہیا کیا جاتا ہے۔سات ساعت کے کنارے گزرنے کے بعد دفتر میں کیا عدد ہو گا؟
 
 جواب: \عددی{0111}
\انتہا{سوال}
\ابتدا{سوال}
%7.2
  گزشتہ سوال میں دائیں منتقل دفتر کے بجائے بائیں منتقل دفتر استعمال کرتے ہوئے  جواب معلوم کریں۔
  
  جواب: \عددی{1101}
\انتہا{سوال}
\ابتدا{سوال}
%7.3
 گزشتہ دو سوالات میں  ساعت  کے ہر کنارے پر دفتر میں ثنائی عدد  معلوم  کریں۔
\انتہا{سوال}
\ابتدا{سوال}
%7.4
 آٹھ بٹ  سلسلہ وار دائیں منتقل دفتر کا  مخارج  چار بٹ  سلسلہ وار دائیں منتقل دفتر  کو بطور مداخل فراہم کیا جاتا ہے۔آٹھ بٹ دفتر میں ابتدائی مواد \عددی{10110110} پایا جاتا ہے  اور  اسے \عددی{1010}    (کمتر بِٹ سے آغاز کر کے)    فراہم کیا جاتا ہے۔ساعت کے چار  کنارے گزرنے کے بعد ان دفتر  میں کیا اعداد پائے جائیں گے؟
 
 جواب: \عددی{0110}،  \عددی{10101010}
\انتہا{سوال}
\ابتدا{سوال}
%7.5
 گزشتہ سوال میں  بائیں منتقل دفتر استعمال کرتے  ہوئے جواب حاصل کریں۔ چار بِٹ مداخل کا بلند تر بِٹ پہلے فراہم کیا جاتا ہے۔
 
 جواب: \عددی{1011}، \عددی{01101010}
\انتہا{سوال}
\ابتدا{سوال}
%7.6
 آٹھ بٹ کے دو عدد بائیں  انتقال  دفتر استعمال کرتے ہوئے سولہ بٹ کا  بائیں انتقال دفتر حاصل کریں۔
\انتہا{سوال}
\ابتدا{سوال}
%7.7
 شکل   \حوالہ{شکل_دفتر_متعدد_ہندسی_جمع_کار}  میں سلسلہ وار ثنائی جمع کار دکھایا گیا ہے۔ آٹھ بِٹ دفتر۔ا  میں \عددی{11001010}  اور  آٹھ بِٹ دفتر-ب   میں \عددی{11100001} پایا جاتا ہے۔تصور کریں \عددی{\overline{\text{\RL{زبردستی پست}}}}    لمحاتی  پست کرنے کے بعد ساعت کے آٹھ کنارے گزرتے ہیں۔ساعت کا ہر کنارہ گزرنے کے بعد دفتر-ا میں  کیا مواد موجود ہو گا؟
 
 جواب:پہلے کنارے کے بعد دفتر-ا میں \عددی{11100101} ہو گا۔  آخری کنارے کے بعد \عددی{C=1} اور  دفتر-ا میں \عددی{10101011} ہو گا۔
\انتہا{سوال}
\ابتدا{سوال}
%7.8
 سلسلہ وار ثنائی جمع کار سے سلسلہ وار ثنائی منفی کار حاصل کریں۔ منفی  کردہ عدد کا تکملہ  دفتر-ب میں متوازی لکھنا بھی دکھائیں۔
\انتہا{سوال}
%===============


\باب{ گنت کار}
ثنائی گنت کار آپ دیکھ چکے ہیں۔گنت کار کا بنیادی مقصد داخلی \اصطلاح{ برقی اشارے }\فرہنگ{اشارہ!برقی}\حاشیہب{electrical signal}\فرہنگ{signal!electrical}کی گنتی کرنا ہے۔برقی اشارہ اسے بطور ساعت یا سادہ مداخل کے طور پر مہیا کیا جا تا ہے۔

وہ دفتر جس کے خارجی برقی اشارات ثنائی گنتی کے تحت ترتیب وار حال تبدیل کرتے ہوں \اصطلاح{ ثنائی گنت کار } کہلاتا ہے۔ وہ دفتر جس کے خارجی اشارات اعشاری گنتی کے تحت ترتیب وار حال تبدیل کرتے ہوں \اصطلاح{ اعشاری گنت کار } کہلاتا ہے۔

ان کے علاوہ، کوئی بھی دور جو کسی متعین ترتیب کے تحت متواتر حال تبدیل کرتا ہو گنت کار کہلائے گا۔

گنت کار ادوار پر اس باب میں غور کیا جائے گا۔

\حصہ{ثنائی گنت کار}
چار بِٹ ثنائی سیدھی گنتی \عددی{0000_2} تا \عددی{1111_2} ممکن ہے۔اسی طرح الٹی گنتی \عددی{1111_2} سے شروع ہو کر \عددی{0000_2} پر ختم ہو گی۔دونوں صورتوں میں گنتی پوری ہونے کے بعد عموماً دوبارہ نئے سرے سے شروع کی جاتی ہے۔ شکل  \حوالہ{شکل_گنت_کار_سیدھا_الٹا}-الف میں \اصطلاح{ چار بِٹ ثنائی سیدھا گنت کار }\فرہنگ{گنت کار!چار بٹ ثنائی سیدھا}\حاشیہب{four bit binary up counter}\فرہنگ{counter!four bit binary, up} اور شکل- ب میں \اصطلاح{ چار بِٹ ثنائی الٹ گنت کار }\فرہنگ{گنت کار!چار بٹ ثنائی، الٹ}\حاشیہب{four bit binary down counter}\فرہنگ{counter!four bit binary, down} پیش ہیں۔ان کی بناوٹ ملتی جلتی ہے۔
\begin{figure}
\centering
\begin{subfigure}{1\textwidth}
\centering
\begin{tikzpicture}
\pgfmathsetmacro{\ksepX}{2.75}
\pgfmathsetmacro{\kpinA}{0.5}
\pgfmathsetmacro{\kpinB}{0.4}
\kuTFFud[u0]{0}{0}
\kuTFFud[u1]{-1*\ksepX}{0}
\kuTFFud[u2]{-2*\ksepX}{0}
\kuTFFud[u3]{-3*\ksepX}{0}
\draw(u0p6)node[above]{$Q_0$};
\draw(u1p6)node[above]{$Q_1$};
\draw(u2p6)node[above]{$Q_2$};
\draw(u3p6)node[above]{$Q_3$};
\draw(u0p4)--++(0,2*\kpin)--++(-0.65*\ksepX,0)|-(u1p2);
\draw(u1p4)--++(0,2*\kpin)--++(-0.65*\ksepX,0)|-(u2p2);
\draw(u2p4)--++(0,2*\kpin)--++(-0.65*\ksepX,0)|-(u3p2);
\draw(u0p2)--++(2*\kpinA,0)coordinate(clkL);
\kinright[j1]{clkL};
\draw(j1east)node[right]{$C$};
\draw(u3pd)--++(0,-3*\kpinA)-|coordinate(krstN)(u0pd);
\kinright[j2]{krstN};
\draw(j2south)node[below,xshift=0.5*\kpinA]{$\overline{\text{\RL{زبردستی پست}}}$};
\draw(u0pd)--(u0pd |-krstN);
\draw(u1pd)--(u1pd |-krstN);
\draw(u2pd)--(u2pd |-krstN);
\draw(u0pu)--++(0,-2.25*\kpinA)coordinate(ksetN)--(ksetN-| u3pu)coordinate(kHigh);
\kinleft[j3]{kHigh};
\draw(j3west)node[left]{$1$};
\draw(u1pu)--(u1pu|-kHigh);
\draw(u2pu)--(u2pu|-kHigh);
\draw(u3pu)--(u3pu|-kHigh);
\draw(u0p1)|-(ksetN);
\draw(u1p1)--(u1p1|-kHigh);
\draw(u2p1)--(u2p1|-kHigh);
\draw(u3p1)--(u3p1|-kHigh);
\draw(u0-south-east)node[shift={(1*\kpinA,1*\kpinA)}]{(ا)};
\end{tikzpicture}
\end{subfigure}
\begin{subfigure}{1\textwidth}
\centering
\begin{tikzpicture}
\pgfmathsetmacro{\ksepX}{2.75}
\pgfmathsetmacro{\kpinA}{0.5}
\pgfmathsetmacro{\kpinB}{0.4}
\kuTFFud[u0]{0}{0}
\kuTFFud[u1]{-1*\ksepX}{0}
\kuTFFud[u2]{-2*\ksepX}{0}
\kuTFFud[u3]{-3*\ksepX}{0}
\draw(u0p6)--++(0,1*\kpinA)node[above]{$Q_0$};
\draw(u1p6)--++(0,1*\kpinA)node[above]{$Q_1$};
\draw(u2p6)--++(0,1*\kpinA)node[above]{$Q_2$};
\draw(u3p6)--++(0,1*\kpinA)node[above]{$Q_3$};
\draw(u1p6)--++(0,0.5*\kpin)--++(0.65*\ksepX,0)|-(u0p2);
\draw(u2p6)--++(0,0.5*\kpin)--++(0.65*\ksepX,0)|-(u1p2);
\draw(u3p6)--++(0,0.5*\kpin)--++(0.65*\ksepX,0)|-(u2p2);
\draw(u3p2)--++(-2*\kpinA,0)coordinate(clkL);
\kinleft[j1]{clkL};
\draw(j1west)node[left]{$C$};
\draw(u3pd)--++(0,-2.25*\kpinA)-|coordinate(krstN)(u0pd);
\kinright[j2]{krstN};
\draw(j2east)node[right]{$1$};
\draw(u1pd)--(u1pd |-krstN);
\draw(u2pd)--(u2pd |-krstN);
\draw(u3pd)--(u3pd |-krstN)coordinate(kkk);
\draw(u3p1)|-(kkk);
\draw(u2p1)|-(kkk);
\draw(u1p1)|-(kkk);
\draw(u0p1)|-(kkk);
\draw(u0pu)--++(0,-3*\kpinA)coordinate(ksetN)--(ksetN-| u3pu)coordinate(kHigh);
\kinleft[j3]{kHigh};
\draw(j3south)node[below,xshift=-0.5*\kpinA]{$\overline{\text{\RL{زبردستی بلند}}}$};
\draw(u1pu)--(u1pu|-kHigh);
\draw(u2pu)--(u2pu|-kHigh);
\draw(u3pu)--(u3pu|-kHigh);
\draw(u0-south-east)node[shift={(1*\kpinA,1*\kpinA)}]{(ب)};
\end{tikzpicture}
\end{subfigure}
\caption{سیدھا اور الٹا گنت کار}
\label{شکل_گنت_کار_سیدھا_الٹا}
\end{figure}


\اصطلاح{ثنائی گنت کار }\فرہنگ{گنت کار!ثنائی}\حاشیہب{binary counter}\فرہنگ{counter!binary} آپ پہلے بھی دیکھ چکے ہیں۔\اصطلاح{سیدھے گنت کار }میں 
\عددی{\overline{\text{\RL{زبردستی بلند}}}} کو بلند (1) یعنی غیر فعال رکھا جاتا ہے۔گنتی شروع کرنے سے قبل \عددی{\overline{\text{\RL{زبردستی پست}}}} کو لمحاتی پست (0) کر کے گنتی (کی ابتدائی قیمت) \عددی{0000_2} کی جاتی ہے۔ گنتی کے دوران کسی بھی وقت \عددی{\overline{\text{\RL{زبردستی پست}}}} اشارہ پست کر کے گنتی دوبارہ صفر سے شروع کی جا سکتی ہے۔

\اصطلاح{الٹ گنت کار } میں \عددی{\overline{\text{\RL{زبردستی پست}}}} کو غیر فعال رکھا جاتا ہے جبکہ \عددی{\overline{\text{\RL{زبردستی بلند}}}} اشارے کو گنتی شروع کرنے سے قبل لمحاتی فعال کر کے گنتی \عددی{1111_2} سے شروع کی جاتی ہے۔ گنتی کے دوران کسی بھی وقت اس اشارے کو پست کر کے گنتی دوبارہ \عددی{1111_2} سے شروع کی جا سکتی ہے۔

سیدھے گنت کار کو مثال بناتے ہوئے ایک اہم صورت حال پر غور کرتے ہیں۔شکل میں بایاں ترین پلٹ، ساعت کے ( ہر ) کنارہ چڑھائی پر حال تبدیل کرتا ہے۔ساعت کے کنارہ چڑھائی کے کچھ دیر بعد 
\عددی{\overline{Q}_3} حال تبدیل کرے گا۔اس دورانیہ کو پلٹ کا \اصطلاح{ دورانیہ ردعمل }\فرہنگ{دورانیہ!رد عمل}\حاشیہب{propagation time}\فرہنگ{propagation time}کہتے ہیں۔یوں اگلے پلٹ کو، جسے \عددی{\overline{Q}_3} بطور ساعت فراہم کیا گیا ہے ،حال تبدیل کرنے کا خبر اصل ساعت (کے کنارہ چڑھائی) سے کچھ دیر بعد پہنچتا ہے۔اس پلٹ کو بھی مخارج (\عددی{\overline{Q}_2}) تبدیل کرنے کے لئے پلٹ کے دورانیہ رد عمل جتنا وقت درکار ہو گا۔اسی طرح اس سے اگلے پلٹ کو ، جسے \عددی{\overline{Q}_2} بطور ساعت فراہم کیا گیا ہے ، حال تبدیل کرنے کا اشارہ، اصل ساعت (کے کنارہ چڑھائی) سے دورانیہ رد عمل کے دگنے وقت کے برابر تاخیر سے ملے گا۔

آپ دیکھ سکتے ہیں اس دور میں تمام پلٹوں کے مخارج بیک وقت تبدیل نہیں ہوں گے بلکہ مخارج کی تبدیلی بائیں پلٹ سے شروع ہوتی ہے اور بدستور دائیں جانب بڑھتی ہے۔مخارج کی تبدیلی اس دور میں لہر کی طرح گزرتی ہے۔یوں اس طرح ادوار کو \اصطلاح{ لہریا گنت کار }\فرہنگ{گنت کار!لہریا}\حاشیہب{ripple counter}\فرہنگ{counter!ripple} کہتے ہیں۔یوں موجودہ دور\اصطلاح{ لہریا ثنائی گنت کار }\فرہنگ{گنت کار!لہریا، ثنائی}\حاشیہب{binary ripple counter}\فرہنگ{counter!binary, ripple} کہلاتا ہے۔

عین ممکن ہے کہ آخری پلٹ تک سعت کی خبر پہنچنے سے قبل سعت کا نیا اشارہ پہلی پلٹ کو ملے۔ یوں آخری پلٹ گزشتہ ساعت گننے کے مطابق جبکہ پہلی پلٹ نئی سعت گننے کے مطابق ہو گا اور گنتی غلط ہو گی۔متعدد پلٹ پر مبنی لہریا گنت کار میں اس مسئلہ کی توقع رکھیں۔ 

 معاصر گنت کار اس مسئلہ سے پاک ہیں۔آئیں ان پر غور کرتے ہیں۔
 
\حصہ{معاصر گنت کار}
\اصطلاح{معاصر گنت کار } میں تمام پلٹ کو ایک ہی ساعت مہیا کی جاتی ہے لہٰذا تمام پلٹ بیکوقت نیا حال اختیار کرتے ہیں۔ان ادوار میں ہر پلٹ کے مداخل پر ترکیبی دور نصب کر کے ، اسے اگلی ساعت کے کنارے پر ، بلند یا پست ہونے کا اشارہ مہیا کیا جاتا ہے۔پلٹ اگلی ساعت کے کنارے پر اس اشارے کے مطابق حال اختیار کرتا ہے۔یہ فیصلہ کہ اگلی ساعت پر پلٹ بلند یا پست حال اختیار کرے گا ، دور کے موجودہ حال کو دیکھ کر کیا جاتا ہے۔اس طریقہ کار کو چند مثالوں سے سمجھتے ہیں۔

\جزوحصہ{معاصر ثنائی گنت کار}
\اصطلاح{تین بِٹ معاصر ثنائی گنت کار }\فرہنگ{گنت کار!تین بِٹ، معاصر}\حاشیہب{three bit synchronous counter}\فرہنگ{counter!synchronous, three bit} شکل  \حوالہ{شکل_گنت_کار_معاصر_ثنائی_گنتکار}  میں پیش ہے۔ مخارج \عددی{Q_0} کمتر رتبی بِٹ جبکہ \عددی{Q_2} بلند تر رتبی بِٹ ہے۔اس دور کی بناوٹ سیکھتے ہیں۔

\begin{figure}
\centering
\begin{tikzpicture}
\pgfmathsetmacro{\ksepX}{3}
\pgfmathsetmacro{\ksepY}{1.5}
\pgfmathsetmacro{\kpinA}{0.5}
\pgfmathsetmacro{\kpinB}{0.4}
\kuTFFd[u0]{0}{0}
\kuTFFd[u1]{-1*\ksepX}{0}
\kuTFFd[u2]{-2*\ksepX}{0}
\draw(u0p6)--++(0,\kpinA)node[above]{$Q_0$};
\draw(u1p6)--++(0,\kpinA)node[above]{$Q_1$};
\draw(u2p6)--++(0,\kpinA)node[above]{$Q_2$};
\draw(-2.25*\ksepX,-\ksepY)node[and port, number inputs=2,scale=1,rotate=180,anchor=out](u3){};
\draw(u2p1)|-(u3.out);
\draw(u1p1)|-(u3.in 1);
\draw(u1p6)--++(-1.5*\kpinA,0)|-(u3.in 2);
\draw(u0p6)--++(-1.5*\kpinA,0)|-(u1p1 |-u3.in 1);
\draw(u0p1)--++(0,-0.35*\ksepY)coordinate(dd);
\kindown[j1]{dd};
\draw(j1south)node[below]{$1$};
\draw(u2p2)--++(0,-1*\kpinA)--++(2*\ksepX+2*\kpinA,0)coordinate(kclk);
\kinright[j2]{kclk};
\draw(j2east)node[right]{$C$};
\draw(u1p2)|-(kclk);
\draw(u0p2)|-(kclk);
\draw(u2pd)--++(0,-3*\kpin)--++(2*\ksepX,0)coordinate(krst);
\kinright[j3]{krst};
\draw(j3east)node[right]{$\overline{\text{\RL{زبردستی پست}}}$};
\draw(u0pd)|-(krst);
\draw(u1pd)|-(krst);
\end{tikzpicture}
\caption{معاصر ثنائی گنت کار}
\label{شکل_گنت_کار_معاصر_ثنائی_گنتکار}
\end{figure}


جدول \حوالہ{جدول_گنت_کار_تین_بِٹ_معاصر} میں \موٹا{ موجودہ حال } کی قطار میں تین بِٹ ثنائی گنتی لکھی گئی ہے جو کسی بھی لمحے پلٹ کا موجودہ حال پیش کرتی ہے۔موجودہ حال استعمال کرتے ہوئے باقی جدول حاصل ہو گا۔ جدول کی پہلی صف پر غور کریں جہاں موجودہ گنتی یا موجودہ حال \عددی{000_2} ہے۔ہم چاہتے ہیں کہ اگلا عدد \عددی{001_2} ہو ، لہٰذا \موٹا{ اگلے حال }کی پہلی صف میں ہم \عددی{001_2} لکھتے ہیں۔آخری صف میں موجودہ حال \عددی{111_2} ہے۔تین بِٹ استعمال کرتے ہوئے یہیں تک گنتی ممکن ہے۔اس آخری گنتی تک پہنچ کر ہم دوبارہ \عددی{000_2} سے گنتی شروع کرتے ہی، لہٰذا آخری صف میں اگلا حال \عددی{000_2} ہو گا۔یوں موجودہ حال کی دوسری صف در حقیقت اگلے حال کی پہلی صف ہو گی۔اسی طرح موجودہ حال کی تیسری صف اگلے حال کی دوسری صف ہو گی، اور موجودہ حال کی پہلی صف اگلے حال کی آخری صف ہو گی۔
\begin{table}
\caption{معاصر ثنائی گنت کار کے حال}
\label{جدول_گنت_کار_تین_بِٹ_معاصر}
\centering
\begin{otherlanguage}{english}
\begin{tabular}{CCC|CCC|CCC}
\toprule
\multicolumn{3}{c}{\text{\RL{موجودہ حال}}} &\multicolumn{3}{|c|}{\text{\RL{اگلا حال}}} &
\multicolumn{3}{c}{\text{\RL{مداخل}}}\\
\midrule
Q_2&Q_1&Q_0&Q_2&Q_1&Q_0&T_2&T_1&T_0\\
\midrule
0&0&0&0&0&1&0&0&1\\
0&0&1&0&1&0&0&1&1\\
0&1&0&0&1&1&0&0&1\\
0&1&1&1&0&0&1&1&1\\
1&0&0&1&0&1&0&0&1\\
1&0&1&1&1&0&0&1&1\\
1&1&0&1&1&1&0&0&1\\
1&1&1&0&0&0&1&1&1\\
\bottomrule
\end{tabular}
\end{otherlanguage}
\end{table}

 پہلی صف کے کمتر رتبی بِٹ \عددی{Q_0} پر غور کرتے ہیں۔اس بِٹ کی موجودہ قیمت کو موجودہ حال \عددی{Q_0} ظاہر کرتا ہے جو \عددی{0} ہے جبکہ اس کی اگلی قیمت اگلا حال \عددی{Q_0} ظاہر کرتا ہے جو \عددی{1} ہے۔ٹی پلٹ استعمال کرتے ہوئے ساعت کے کنارہ چڑھائی پر پلٹ کا حال \عددی{0} سے \عددی{1} کرنے کی خاطر پلٹ کے مخارج \عددی{T_0} کو بلند کرنا ہو گا۔یہ معلومات جدول \حوالہ{جدول_گنت_کار_ٹی_پلٹ} سے حاصل کی گئی۔ یوں جدول میں \موٹا{مداخل} کا خانہ بنا کر اس کی پہلی صف میں \عددی{T_0} کی قیمت \عددی{1} لکھتے ہیں۔
\begin{table}
\caption{ٹی پلٹ کی کارکردگی}
\label{جدول_گنت_کار_ٹی_پلٹ}
\centering
\begin{otherlanguage}{english}
\begin{tabular}{CL}
T&Q_{n+1}\\
\midrule
0&Q_n\\
1&\overline{Q}_n
\end{tabular}
\end{otherlanguage}
\end{table}


اسی (پہلی) صف میں اگلے بِٹ \عددی{Q_1} پر غور کرتے ہیں۔اس بِٹ کی موجودہ قیمت \عددی{0} ہے اور اس کی اگلی قیمت بھی \عددی{0} ہے، لہٰذا ساعت کے اگلے کنارے پر ہم نہیں چاہتے کہ یہ پلٹ اپنا حال تبدیل کرے۔یوں اس پلٹ کے مداخل \عددی{T_1} کو پست رکھنا ہو گا۔اس طرح \عددی{T_1} کے خانے میں \عددی{0} لکھا جائے گا۔اسی طرز پر تمام صفوں کے تمام مداخل کے لئے جدول کے خانے پُر کیے گئے ہیں۔


دور بنانے کے لئے جدول\حوالہ{جدول_گنت_کار_تین_بِٹ_معاصر} میں \اصطلاح{مداخل }کی قطار استعمال ہو گی جس سے مجموعہ ارکان ضرب کی ترکیب سے درج ذیل مساوات لکھے جا سکتے ہیں۔
\begin{gather}
\begin{aligned}
T_0&=1\\
T_1&=\overline{Q}_2\overline{Q_1} Q_0+\overline{Q}_2 Q_1 Q_0+Q_2\overline{Q_1} Q_0+Q_2Q_1Q_0\\
T_2&=\overline{Q}_2 Q_1 Q_0+Q_2Q_1Q_0
\end{aligned}
\end{gather}
%
\begin{figure}
\centering
\begin{subfigure}{1\textwidth}
\centering
\begin{tikzpicture}
\pgfmathsetmacro{\kxstep}{1}
\pgfmathsetmacro{\kystep}{1}
\pgfmathsetmacro{\kpin}{0.75}
\pgfmathsetmacro{\kmv}{0.15}
\draw[xstep=\kxstep,ystep=\kystep](0,0) grid (4*\kxstep,-2*\kystep);
\draw(0,0)--++(135:1.5*\kpin)node[pos=0.75,above right]{$Q_1Q_0$}node[pos=0.75,below left]{$Q_2$};
\foreach \kx/\xlb in {0/{00},1/{01},2/{11},3/{10}}{\draw(\kx*\kxstep+\kxstep/2,0)node[above]{$\xlb$};}
\foreach \ky/\ylb in {0/0,1/1}{\draw(0,-\ky*\kystep-\kystep/2)node[left]{$\ylb$};}
\foreach \kx/\xlb in {0/{},1/{},2/{1},3/{}}{\draw(\kx*\kxstep+\kxstep/2,-\kystep/2)node[]{$\xlb$};}
\foreach \kx/\xlb in {0/{},1/{},2/{1},3/{}}{\draw(\kx*\kxstep+\kxstep/2,-1.5*\kystep)node[]{$\xlb$};}
\draw[gray,dashed] ($(2*\kxstep,0)+(\kmv,-\kmv)$) rectangle ($(3*\kxstep,-2*\kystep)+(-\kmv,\kmv)$);
\end{tikzpicture}\quad\quad 
\(T_2=Q_1Q_0\)
\caption*{} 					%gives space between subfigures
\end{subfigure}
\begin{subfigure}{1\textwidth}
\centering
\begin{tikzpicture}
\pgfmathsetmacro{\kxstep}{1}
\pgfmathsetmacro{\kystep}{1}
\pgfmathsetmacro{\kpin}{0.75}
\pgfmathsetmacro{\kmv}{0.15}
\draw[xstep=\kxstep,ystep=\kystep](0,0) grid (4*\kxstep,-2*\kystep);
\draw(0,0)--++(135:1.5*\kpin)node[pos=0.75,above right]{$Q_1Q_0$}node[pos=0.75,below left]{$Q_2$};
\foreach \kx/\xlb in {0/{00},1/{01},2/{11},3/{10}}{\draw(\kx*\kxstep+\kxstep/2,0)node[above]{$\xlb$};}
\foreach \ky/\ylb in {0/0,1/1}{\draw(0,-\ky*\kystep-\kystep/2)node[left]{$\ylb$};}
\foreach \kx/\xlb in {0/{},1/{1},2/{1},3/{}}{\draw(\kx*\kxstep+\kxstep/2,-\kystep/2)node[]{$\xlb$};}
\foreach \kx/\xlb in {0/{},1/{1},2/{1},3/{}}{\draw(\kx*\kxstep+\kxstep/2,-1.5*\kystep)node[]{$\xlb$};}
\draw[gray,dashed] ($(\kxstep,0)+(\kmv,-\kmv)$) rectangle ($(3*\kxstep,-2*\kystep)+(-\kmv,\kmv)$);
\end{tikzpicture}\quad\quad
\(T_1=Q_0\)
\caption*{} 					%gives space between subfigures
\end{subfigure}
\begin{subfigure}{1\textwidth}
\centering
\begin{tikzpicture}
\pgfmathsetmacro{\kxstep}{1}
\pgfmathsetmacro{\kystep}{1}
\pgfmathsetmacro{\kpin}{0.75}
\pgfmathsetmacro{\kmv}{0.15}
\draw[xstep=\kxstep,ystep=\kystep](0,0) grid (4*\kxstep,-2*\kystep);
\draw(0,0)--++(135:1.5*\kpin)node[pos=0.75,above right]{$Q_1Q_0$}node[pos=0.75,below left]{$Q_2$};
\foreach \kx/\xlb in {0/{00},1/{01},2/{11},3/{10}}{\draw(\kx*\kxstep+\kxstep/2,0)node[above]{$\xlb$};}
\foreach \ky/\ylb in {0/0,1/1}{\draw(0,-\ky*\kystep-\kystep/2)node[left]{$\ylb$};}
\foreach \kx/\xlb in {0/{1},1/{1},2/{1},3/{1}}{\draw(\kx*\kxstep+\kxstep/2,-\kystep/2)node[]{$\xlb$};}
\foreach \kx/\xlb in {0/{1},1/{1},2/{1},3/{1}}{\draw(\kx*\kxstep+\kxstep/2,-1.5*\kystep)node[]{$\xlb$};}
\draw[gray,dashed] ($(0,0)+(\kmv,-\kmv)$) rectangle ($(4*\kxstep,-2*\kystep)+(-\kmv,\kmv)$);
\end{tikzpicture}\quad\quad
\(T_0=1\)
\end{subfigure}
\caption{تین بِٹ معاصر گنت کار کی سادہ مساواتیں}
\label{شکل_گنت_کار_تین_بٹ_معاصر_کی_سادہ}
\end{figure}

یہ مساوات موجودہ حال کی قیمتیں مدِ نظر رکھ کر لکھی گئی ہیں۔جدول \حوالہ{جدول_گنت_کار_تین_بِٹ_معاصر} میں موجود مواد سے شکل  \حوالہ{شکل_گنت_کار_تین_بٹ_معاصر_کی_سادہ}   میں پیش کارناف نقشوں کی مدد سے درج ذیل سادہ مساواتیں حاصل کی گئی ہیں ۔
\begin{gather}
\begin{aligned}\label{مساوات_گنت_کار_تین_ثنائی}
T_0&=1\\
T_1&=Q_0\\
T_2&=Q_1Q_0
\end{aligned}
\end{gather}

شکل  \حوالہ{شکل_گنت_کار_معاصر_ثنائی_گنتکار} میں تین پلٹوں کو مساوات \حوالہ{مساوات_گنت_کار_تین_ثنائی} سے حاصل برقی اشارات بطور مداخل فراہم کر کے \اصطلاح{ تین بِٹ معاصر ثنائی گنت کار }\فرہنگ{گنت کار!معاصر، تین بِٹ ثنائی}\حاشیہب{three bit synchronous binary counter}\فرہنگ{counter!synchronous, three bit binary}حاصل کیا گیا ہے۔

 جدول \حوالہ{جدول_گنت_کار_تین_بِٹ_معاصر} دیکھ کر بھی مساوات \حوالہ{مساوات_گنت_کار_تین_ثنائی} حاصل کی جا سکتی ہیں۔ اس جدول پر غور کرنے سے دیکھا جا سکتا ہے کہ \عددی{Q_0} ہر ساعت کے کنارے پر تبدیل ہوتا ہے۔ \عددی{T_0} پر \عددی{1} مہیا کرنے سے یہی حاصل ہو گا (جو مساوات \حوالہ{مساوات_گنت_کار_تین_ثنائی} کا پہلا جزو ہے)۔ جدول میں جب بھی \عددی{Q_0} کی قیمت \عددی{1} ہو ، اگلی ساعت کے کنارے پر \عددی{Q_1} کی قیمت تبدیل ہوتی ہے، جو \عددی{T_1} کو \عددی{Q_0} فراہم کرنے سے حاصل ہو گا (یہ درج بالا مساوات کا دوسرا جزو ہے)۔ اسی طرح جدول میں جب بھی \عددی{Q_0} اور \عددی{Q_1} کی قیمتیں بیکوقت \عددی{1} ہوں، اگلی ساعت کے کنارے پر \عددی{Q_2} کی قیمت تبدیل ہوتی ہے۔یوں \عددی{T_2} کو \عددی{Q_1Q_0} فراہم کرنا ہو گا (درج بالا مساوات کا تیسرا جزو)۔ متعدد بِٹ ثنائی گنتی پر غور کرنے سے دیکھا جا سکتا ہے کہ کوئی بھی مخارج، ساعت کے اگلے کنارے، تب حال تبدیل کرتا ہے جب اس سے کمتر تمام مخارج کی قیمتیں بیکوقت \عددی{1} ہوں۔یوں \اصطلاح{ چار بِٹ معاصر ثنائی گنت کار }\فرہنگ{گنت کار!ثنائی، معاصر، چار بِٹ}\حاشیہب{four bit synchronous binary counter}\فرہنگ{counter!synchronous, binary, four bit} کے لئے درج ذیل ہو گا۔
\begin{gather}
\begin{aligned}\label{مساوات_گنت_کار_چار_بِٹ}
T_0&=1\\
T_1&=Q_0\\
T_2&=Q_1Q_0\\
T_3&=Q_2Q_1Q_0
\end{aligned}
\end{gather}

\جزوحصہ{ثنائی  مرموز اعشاری  معاصر گنت کار}
گزشتہ حصے میں تین بِٹ ثنائی گنت کار پر غور کیا گیا، جو \عددی{000_2} تا \عددی{111_8} گنتی کرنے کی صلاحیت رکھتا ہے۔چار بِٹ ثنائی گنت کار \عددی{0000_2} تا \عددی{1111_2} ثنائی گنتی کر سکتا ہے۔چار بِٹ ثنائی گنت کار  کو \عددی{0000_2} تا \عددی{1001_2} گنتی کرنے کا پابند بنانے سے \اصطلاح{ثنائی  مرموز اعشاری گنت کار}\فرہنگ{گنت کار!ثنائی مرموز اعشاری}\حاشیہب{BCD decimal counter}\فرہنگ{counter!decimal, BCD} حاصل ہو گا، جس پر اس حصہ میں غور کیا جائے گا۔

جدول \حوالہ{جدول_گنت_کار_ثنائی_اعشاری} میں ثنائی  مرموز  اعشاری گنت کار کے حال پیش ہیں۔جدول میں \موٹا{ مخارج } \عددی{y} کی قطار کا اضافہ کیا گیا ہے۔مخارج \عددی{y } صفر سے نو تک گنتی پوری ہونے پر ساعت کے ایک\اصطلاح{ دوری عرصہ }\فرہنگ{دوری عرصہ}\حاشیہب{time period}\فرہنگ{time period} کے لئے بلند ہوتا ہے۔ہم آگے دیکھیں گے کہ \عددی{y} استعمال کرتے ہوئے متعدد اعشاری ہندسوں کے گنت کار تخلیق دیے جاتے ہیں۔

\begin{table}
\caption{ثنائی  مرموز  اعشاری گنت کار کے حال}
\label{جدول_گنت_کار_ثنائی_اعشاری}
\centering
\begin{otherlanguage}{english}
\begin{tabular}{CCCC|CCCC|C|CCCC}
\toprule
\multicolumn{4}{c}{\text{\RL{موجودہ حال}}} &\multicolumn{4}{|c|}{\text{\RL{اگلا حال}}} &\text{\RL{مخارج}}&
\multicolumn{4}{c}{\text{\RL{مداخل}}}\\
\midrule
Q_3&Q_2&Q_1&Q_0&Q_3&Q_2&Q_1&Q_0&y&T_3&T_2&T_1&T_0\\
\midrule
0&0&0&0&0&0&0&1&0&0&0&0&1\\
0&0&0&1&0&0&1&0&0&0&0&1&1\\
0&0&1&0&0&0&1&1&0&0&0&0&1\\
0&0&1&1&0&1&0&0&0&0&1&1&1\\
0&1&0&0&0&1&0&1&0&0&0&0&1\\
0&1&0&1&0&1&1&0&0&0&0&1&1\\
0&1&1&0&0&1&1&1&0&0&0&0&1\\
0&1&1&1&1&0&0&0&0&1&1&1&1\\
1&0&0&0&1&0&0&1&0&0&0&0&1\\
1&0&0&1&0&0&0&0&1&1&0&0&1\\
\bottomrule
\end{tabular}
\end{otherlanguage}
\end{table}



\begin{figure}
\centering
\begin{subfigure}{0.45\textwidth}
\centering
\begin{tikzpicture}
\pgfmathsetmacro{\kxstep}{1}
\pgfmathsetmacro{\kystep}{1}
\pgfmathsetmacro{\kpin}{0.75}
\pgfmathsetmacro{\kmv}{0.15}
\draw[xstep=\kxstep,ystep=\kystep](0,0) grid (4*\kxstep,-4*\kystep);
\draw(0,0)--++(135:1.5*\kpin)node[pos=0.75,above right]{$Q_1Q_0$}node[pos=0.75,below left]{$Q_3Q_2$};
\foreach \kx/\xlb in {0/{00},1/{01},2/{11},3/{10}}{\draw(\kx*\kxstep+\kxstep/2,0)node[above]{$\xlb$};}
\foreach \ky/\ylb in {0/{00},1/{01},2/{11},3/{10}}{\draw(0,-\ky*\kystep-\kystep/2)node[left]{$\ylb$};}
\foreach \kx/\xlb in {0/{},1/{},2/{},3/{}}{\draw(\kx*\kxstep+\kxstep/2,-\kystep/2)node[]{$\xlb$};}
\foreach \kx/\xlb in {0/{},1/{},2/{1},3/{}}{\draw(\kx*\kxstep+\kxstep/2,-1.5*\kystep)node[]{$\xlb$};}
\foreach \kx/\xlb in {0/{},1/{1},2/{d},3/{d}}{\draw(\kx*\kxstep+\kxstep/2,-2.5*\kystep)node[]{$\xlb$};}
\foreach \kx/\xlb in {0/{d},1/{d},2/{d},3/{d}}{\draw(\kx*\kxstep+\kxstep/2,-3.5*\kystep)node[]{$\xlb$};}
\draw[gray,dashed] ($(2*\kxstep,-\kystep)+(\kmv,-\kmv)$) rectangle ($(3*\kxstep,-3*\kystep)+(-\kmv,\kmv)$);
\draw[gray,dashed] ($(1*\kxstep,-2*\kystep)+(\kmv,-\kmv)$) rectangle ($(3*\kxstep,-4*\kystep)+(-1.5*\kmv,\kmv)$);
\draw(2*\kxstep,-4.5*\kystep)node[below]{\(T_3=Q_3Q_0+Q_2Q_1Q_0\)};
\end{tikzpicture}
\caption*{} 					%gives space between subfigures
\end{subfigure}\hfill
\begin{subfigure}{0.45\textwidth}
\centering
\begin{tikzpicture}
\pgfmathsetmacro{\kxstep}{1}
\pgfmathsetmacro{\kystep}{1}
\pgfmathsetmacro{\kpin}{0.75}
\pgfmathsetmacro{\kmv}{0.15}
\draw[xstep=\kxstep,ystep=\kystep](0,0) grid (4*\kxstep,-4*\kystep);
\draw(0,0)--++(135:1.5*\kpin)node[pos=0.75,above right]{$Q_1Q_0$}node[pos=0.75,below left]{$Q_3Q_2$};
\foreach \kx/\xlb in {0/{00},1/{01},2/{11},3/{10}}{\draw(\kx*\kxstep+\kxstep/2,0)node[above]{$\xlb$};}
\foreach \ky/\ylb in {0/{00},1/{01},2/{11},3/{10}}{\draw(0,-\ky*\kystep-\kystep/2)node[left]{$\ylb$};}
\foreach \kx/\xlb in {0/{},1/{},2/{1},3/{}}{\draw(\kx*\kxstep+\kxstep/2,-\kystep/2)node[]{$\xlb$};}
\foreach \kx/\xlb in {0/{},1/{},2/{1},3/{}}{\draw(\kx*\kxstep+\kxstep/2,-1.5*\kystep)node[]{$\xlb$};}
\foreach \kx/\xlb in {0/{},1/{},2/{d},3/{d}}{\draw(\kx*\kxstep+\kxstep/2,-2.5*\kystep)node[]{$\xlb$};}
\foreach \kx/\xlb in {0/{d},1/{d},2/{d},3/{d}}{\draw(\kx*\kxstep+\kxstep/2,-3.5*\kystep)node[]{$\xlb$};}
\draw[gray,dashed] ($(2*\kxstep,-0*\kystep)+(\kmv,-\kmv)$) rectangle ($(3*\kxstep,-4*\kystep)+(-\kmv,\kmv)$);
%\draw[gray,dashed] ($(1*\kxstep,-2*\kystep)+(\kmv,-\kmv)$) rectangle ($(3*\kxstep,-4*\kystep)+(-1.5*\kmv,\kmv)$);
\draw(2*\kxstep,-4.5*\kystep)node[below]{\(T_2=Q_1Q_0\)};
\end{tikzpicture}
\caption*{} 					%gives space between subfigures
\end{subfigure}
\begin{subfigure}{0.45\textwidth}
\centering
\begin{tikzpicture}
\pgfmathsetmacro{\kxstep}{1}
\pgfmathsetmacro{\kystep}{1}
\pgfmathsetmacro{\kpin}{0.75}
\pgfmathsetmacro{\kmv}{0.15}
\draw[xstep=\kxstep,ystep=\kystep](0,0) grid (4*\kxstep,-4*\kystep);
\draw(0,0)--++(135:1.5*\kpin)node[pos=0.75,above right]{$Q_1Q_0$}node[pos=0.75,below left]{$Q_3Q_2$};
\foreach \kx/\xlb in {0/{00},1/{01},2/{11},3/{10}}{\draw(\kx*\kxstep+\kxstep/2,0)node[above]{$\xlb$};}
\foreach \ky/\ylb in {0/{00},1/{01},2/{11},3/{10}}{\draw(0,-\ky*\kystep-\kystep/2)node[left]{$\ylb$};}
\foreach \kx/\xlb in {0/{},1/{1},2/{1},3/{}}{\draw(\kx*\kxstep+\kxstep/2,-\kystep/2)node[]{$\xlb$};}
\foreach \kx/\xlb in {0/{},1/{1},2/{1},3/{}}{\draw(\kx*\kxstep+\kxstep/2,-1.5*\kystep)node[]{$\xlb$};}
\foreach \kx/\xlb in {0/{},1/{},2/{d},3/{d}}{\draw(\kx*\kxstep+\kxstep/2,-2.5*\kystep)node[]{$\xlb$};}
\foreach \kx/\xlb in {0/{d},1/{d},2/{d},3/{d}}{\draw(\kx*\kxstep+\kxstep/2,-3.5*\kystep)node[]{$\xlb$};}
\draw[gray,dashed] ($(1*\kxstep,-0*\kystep)+(\kmv,-\kmv)$) rectangle ($(3*\kxstep,-2*\kystep)+(-\kmv,\kmv)$);
%\draw[gray,dashed] ($(1*\kxstep,-2*\kystep)+(\kmv,-\kmv)$) rectangle ($(3*\kxstep,-4*\kystep)+(-1.5*\kmv,\kmv)$);
\draw(2*\kxstep,-4.5*\kystep)node[below]{\(T_1=\overline{Q}_3Q_0\)};
\end{tikzpicture}
\caption*{} 					%gives space between subfigures
\end{subfigure}\hfill
\begin{subfigure}{0.45\textwidth}
\centering
\begin{tikzpicture}
\pgfmathsetmacro{\kxstep}{1}
\pgfmathsetmacro{\kystep}{1}
\pgfmathsetmacro{\kpin}{0.75}
\pgfmathsetmacro{\kmv}{0.15}
\draw[xstep=\kxstep,ystep=\kystep](0,0) grid (4*\kxstep,-4*\kystep);
\draw(0,0)--++(135:1.5*\kpin)node[pos=0.75,above right]{$Q_1Q_0$}node[pos=0.75,below left]{$Q_3Q_2$};
\foreach \kx/\xlb in {0/{00},1/{01},2/{11},3/{10}}{\draw(\kx*\kxstep+\kxstep/2,0)node[above]{$\xlb$};}
\foreach \ky/\ylb in {0/{00},1/{01},2/{11},3/{10}}{\draw(0,-\ky*\kystep-\kystep/2)node[left]{$\ylb$};}
\foreach \kx/\xlb in {0/{1},1/{1},2/{1},3/{1}}{\draw(\kx*\kxstep+\kxstep/2,-\kystep/2)node[]{$\xlb$};}
\foreach \kx/\xlb in {0/{1},1/{1},2/{1},3/{1}}{\draw(\kx*\kxstep+\kxstep/2,-1.5*\kystep)node[]{$\xlb$};}
\foreach \kx/\xlb in {0/{1},1/{1},2/{d},3/{d}}{\draw(\kx*\kxstep+\kxstep/2,-2.5*\kystep)node[]{$\xlb$};}
\foreach \kx/\xlb in {0/{d},1/{d},2/{d},3/{d}}{\draw(\kx*\kxstep+\kxstep/2,-3.5*\kystep)node[]{$\xlb$};}
\draw[gray,dashed] ($(0*\kxstep,-0*\kystep)+(\kmv,-\kmv)$) rectangle ($(4*\kxstep,-4*\kystep)+(-\kmv,\kmv)$);
%\draw[gray,dashed] ($(1*\kxstep,-2*\kystep)+(\kmv,-\kmv)$) rectangle ($(3*\kxstep,-4*\kystep)+(-1.5*\kmv,\kmv)$);
\draw(2*\kxstep,-4.5*\kystep)node[below]{\(T_0=1\)};
\end{tikzpicture}
\caption*{} 					%gives space between subfigures
\end{subfigure}
\begin{subfigure}{0.45\textwidth}
\centering
\begin{tikzpicture}
\pgfmathsetmacro{\kxstep}{1}
\pgfmathsetmacro{\kystep}{1}
\pgfmathsetmacro{\kpin}{0.75}
\pgfmathsetmacro{\kmv}{0.15}
\draw[xstep=\kxstep,ystep=\kystep](0,0) grid (4*\kxstep,-4*\kystep);
\draw(0,0)--++(135:1.5*\kpin)node[pos=0.75,above right]{$Q_1Q_0$}node[pos=0.75,below left]{$Q_3Q_2$};
\foreach \kx/\xlb in {0/{00},1/{01},2/{11},3/{10}}{\draw(\kx*\kxstep+\kxstep/2,0)node[above]{$\xlb$};}
\foreach \ky/\ylb in {0/{00},1/{01},2/{11},3/{10}}{\draw(0,-\ky*\kystep-\kystep/2)node[left]{$\ylb$};}
\foreach \kx/\xlb in {0/{},1/{},2/{},3/{}}{\draw(\kx*\kxstep+\kxstep/2,-\kystep/2)node[]{$\xlb$};}
\foreach \kx/\xlb in {0/{},1/{},2/{},3/{}}{\draw(\kx*\kxstep+\kxstep/2,-1.5*\kystep)node[]{$\xlb$};}
\foreach \kx/\xlb in {0/{},1/{1},2/{d},3/{d}}{\draw(\kx*\kxstep+\kxstep/2,-2.5*\kystep)node[]{$\xlb$};}
\foreach \kx/\xlb in {0/{d},1/{d},2/{d},3/{d}}{\draw(\kx*\kxstep+\kxstep/2,-3.5*\kystep)node[]{$\xlb$};}
\draw[gray,dashed] ($(1*\kxstep,-2*\kystep)+(\kmv,-\kmv)$) rectangle ($(3*\kxstep,-4*\kystep)+(-\kmv,\kmv)$);
%\draw[gray,dashed] ($(1*\kxstep,-2*\kystep)+(\kmv,-\kmv)$) rectangle ($(3*\kxstep,-4*\kystep)+(-1.5*\kmv,\kmv)$);
\draw(2*\kxstep,-4.5*\kystep)node[below]{\(y=Q_3Q_0\)};
\end{tikzpicture}
\caption*{} 					%gives space between subfigures
\end{subfigure}
\caption{کارناف نقشوں سے  ثنائی مرموز اعشاری  معاصر گنت کار کی مساوات کا حصول}
\label{شکل_گنت_کار_کارناف_سے_گنتکار_سادہ_مساوات}
\end{figure}

اس جدول میں \عددی{1010_2} تا \عددی{1111_2} ترتیب استعمال نہیں ہوتے، لہٰذا کارناف نقشوں کی مدد سے پلٹوں کے مداخل \عددی{T_0} تا \عددی{T_3} اور مخارج \عددی{y} کی سادہ مساوات حاصل کرتے وقت انہیں \اصطلاح{ غیر ضروری حال }تصور کیا جاتا ہے۔شکل  \حوالہ{شکل_گنت_کار_کارناف_سے_گنتکار_سادہ_مساوات} میں درج ذیل سادہ مساوات حاصل کرنا دکھایا گیا ہے۔
\begin{gather}
\begin{aligned}
T_0&=1\\
T_1&=\overline{Q}_3Q_0\\
T_2&=Q_1Q_0\\
T_3&=Q_3Q_0+Q_2Q_1Q_0\\
y&=Q_3Q_0
\end{aligned}
\end{gather}

ان مساوات سے حاصل دور شکل \حوالہ{شکل_گنت_کار_ثنائی_مرموز_معاصر_اعشاری}  میں پیش ہے ، جہاں تمام پلٹ کے مداخل پر اضافی ضرب گیٹ نصب کر کے گنتی شروع اور روکنے کی اضافی صلاحیت بھی پیدا کی گئی ہے۔ ان اضافی ضرب گیٹوں کو برقی اشارہ \موٹا{ گِن} مہیا کیا گیا ہے۔یہ اشارہ بلند ہونے کی صورت میں دور گنتی کرتا ہے اور اشارہ پست ہونے کی صورت میں گنتی روکتا ہے۔
\begin{figure}
\centering
\begin{tikzpicture}
\pgfmathsetmacro{\ksepX}{2.5}
\pgfmathsetmacro{\ksepY}{1.5}
\pgfmathsetmacro{\kpinA}{0.5}
\pgfmathsetmacro{\kpinB}{0.4}
\kuTFFd[u0]{0}{0}
\kuTFFd[u1]{-1*\ksepX}{0}
\kuTFFd[u2]{-2*\ksepX}{0}
\kuTFFd[u3]{-3*\ksepX}{0}
\draw(u0p6)node[above]{$Q_0$};
\draw(u1p6)node[above]{$Q_1$};
\draw(u2p6)node[above]{$Q_2$};
\draw(u3p6)node[above]{$Q_3$};
\draw(u0p1)--++(0,-2*\kpinA)node[and port,scale=1,number inputs=2,rotate=90,anchor=out](u4){};
\draw(u1p1)--++(0,-2*\kpinA)node[and port,scale=1,number inputs=2,rotate=90,anchor=out](u5){};
\draw(u2p1)--++(0,-2*\kpinA)node[and port,scale=1,number inputs=2,rotate=90,anchor=out](u6){};
\draw(u3p1)--++(0,-2*\kpinA)node[and port,scale=1,number inputs=2,rotate=90,anchor=out](u7){};
\draw(u7.in 1)node[or port,scale=1,number inputs=2,rotate=90,anchor=out](u8){};
\draw(u8.in 1)--++(-\kpinA,0)node[and port,scale=1,rotate=90,anchor=out,number inputs=2](u9){};
\draw(u8.in 2)--++(\kpinA,0)node[and port,scale=1,rotate=90,anchor=out,number inputs=3](u10){};
\draw(u9.in 1)node[below]{$Q_3$};
\draw(u9.in 2)node[below]{$Q_0$};
\draw(u10.in 1)node[below,xshift=-0.5ex]{$Q_2$};
\draw(u10.in 2)node[below]{$Q_1$};
\draw(u10.in 3)node[below,xshift=0.5ex]{$Q_0$};
\draw(u6.in 1)node[and port,scale=1,number inputs=2,rotate=90,anchor=out](u11){};
\draw(u11.in 1)|-(u11.in 1 |- u10.in 2)node[below]{$Q_1$};
\draw(u11.in 2)|-(u11.in 2 |- u10.in 2)node[below]{$Q_0$};
\draw(u5.in 1)node[and port,scale=1,number inputs=2,rotate=90,anchor=out](u12){};
\draw(u12.in 1)|-(u12.in 1 |- u10.in 2)node[below]{$\overline{Q}_3$};
\draw(u12.in 2)|-(u12.in 2 |- u10.in 2)node[below]{$Q_0$};
\draw(u4.in 1)|-(u4.in 1 |- u10.in 2)node[below]{$1$};
\draw($(u3p6)+(-1*\ksepX,0)$)node[and port,scale=1,number inputs=2,rotate=90,anchor=out](u13){};
\draw(u13.out)node[above]{$y$};
\draw(u13.in 1)--(u13.in 1 |- u9.in 1)node[below]{$Q_3$};
\draw(u13.in 2)--(u13.in 2 |- u9.in 1)node[below]{$Q_0$};
\draw(u7.in 2)--(u4.in 2)coordinate(kcnt);
\kindown[j1]{kcnt};
\draw(j1south)node[left,rotate=90]{گِن};
\draw(u3p2)--(u0p2)--++(1*\kpinA,0)coordinate(kclk)--(kclk |- kcnt)coordinate(kclkin);
\kindown[j2]{kclkin};
\draw(j2south)node[left,rotate=90]{ساعت};
\draw(u0pd)--++(0,-2.5*\kpinA)coordinate(krst)-|(u3pd);
\kindown[j3]{krst};
\draw(j3east)node[below,rotate=90]{$\overline{\text{\RL{زبردستی پست}}}$};
\end{tikzpicture}
\caption{ثنائی مرموز اعشاری  معاصر  گنت کار}
\label{شکل_گنت_کار_ثنائی_مرموز_معاصر_اعشاری}
\end{figure}

شکل  \حوالہ{شکل_گنت_کار_معاصر_ہزار}  میں تین درجی دور بنایا گیا ہے جو \عددی{000_{10}} تا \عددی{999_{10}} گنتی کرتا ہے۔اسے بنانے کی خاطر  تین عدد \اصطلاح{ثنائی  مرموز  اعشاری گنت کار }  (شکل \حوالہ{شکل_گنت_کار_ثنائی_مرموز_معاصر_اعشاری})  استعمال کیے گئے۔اسی طرح مزید درجات جوڑ کر درکار ہندسوں کا گنت کار بنایا جاتا ہے۔   اکائیوں کی  گنتی  \عددی{9_{10}}  کو پہنچنے  پر  اکائی گنت کار  بلند \عددی{y} خارج کرتا ہے جو دہائی گنت کار کے\موٹا{  گِن}  مداخل کو فراہم کیا گیا ہے۔یوں ساعت کے اگلے کنارے پر دہائی کی گنتی میں \عددی{1} کا اضافہ ہو گا۔ اسی طرح \عددی{99_{10}} کو پہنچنے پر سینکڑا گنت کار کا \موٹا{گِن} مداخل بلند ہو گا اور اگلے کنارہ ساعت پر سینکڑا  کی گنتی میں \عددی{1} کا اضافہ ہو گا۔

\begin{figure}
\centering
\begin{tikzpicture}
\pgfmathsetmacro{\ksepX}{3.25}
\pgfmathsetmacro{\ksepY}{2}
\pgfmathsetmacro{\knshift}{0.07}
\pgfmathsetmacro{\kpin}{0.5}
\pgfmathsetmacro{\kpsep}{0.50}
\pgfmathsetmacro{\kul}{0.50}
\pgfmathsetmacro{\kmv}{0.15}
\pgfmathsetmacro{\kxdim}{2*\kul+3*\kpsep}
\pgfmathsetmacro{\kydim}{2.5*\kul+0*\kpsep}
\draw[thick](0,0) rectangle ++(\kxdim,\kydim);
\draw[thick](\kul+0.5*\kpsep-1*\kmv,0)--++(\kmv,\kmv)--++(\kmv,-\kmv);
\foreach \x/\a in {0/{Q_3},1/{Q_2},2/{Q_1},3/{Q_0}}{\draw(\kul+\x*\kpsep,\kydim)node[below]{$\a$}--++(0,\kpin);}
\draw[thick](-\ksepX,0) rectangle ++(\kxdim,\kydim);
\draw[thick](\kul+0.5*\kpsep-1*\kmv-\ksepX,0)--++(\kmv,\kmv)--++(\kmv,-\kmv);
\foreach \x/\a in {0/{Q_3},1/{Q_2},2/{Q_1},3/{Q_0}}{\draw(\kul+\x*\kpsep-\ksepX,\kydim)node[below]{$\a$}--++(0,\kpin);}
\draw[thick](-2*\ksepX,0) rectangle ++(\kxdim,\kydim);
\draw[thick](\kul+0.5*\kpsep-1*\kmv-2*\ksepX,0)--++(\kmv,\kmv)--++(\kmv,-\kmv);
\foreach \x/\a in {0/{Q_3},1/{Q_2},2/{Q_1},3/{Q_0}}{\draw(\kul+\x*\kpsep-2*\ksepX,\kydim)node[below]{$\a$}--++(0,\kpin);}
\draw(\kul+1.5*\kpsep,\kydim+\kpin)node[above]{اکائی};
\draw(\kul+1.5*\kpsep-\ksepX,\kydim+\kpin)node[above]{دہائی};
\draw(\kul+1.5*\kpsep-2*\ksepX,\kydim+\kpin)node[above]{سینکڑا};
\draw(-2*\ksepX+\kxdim,0.5*\kul)coordinate(kna)--++(0.5*\kpin,0)--++(0,-1.5*\kpin)--++(2*\ksepX+0*\kxdim+\kpin,0)coordinate(krst)node[right]{$\overline{\text{\RL{زبردستی پست}}}$};
\draw(-\ksepX+\kxdim,0.5*\kul)coordinate(knb)--++(0.5*\kpin,0)coordinate(aa)--(aa |- krst);
\draw(\kxdim,0.5*\kul)coordinate(knc)--++(0.5*\kpin,0)coordinate(bb)--(bb |- krst);
\draw(kna)++(\knshift,0)node[ocirc]{};
\draw(knb)++(\knshift,0)node[ocirc]{};
\draw(knc)++(\knshift,0)node[ocirc]{};
\draw(-2*\ksepX+\kul+0.5*\kpsep,0)--++(0,-2*\kpin)--++(2*\ksepX+\kxdim,0)coordinate(kclk)node[right]{ساعت};
\draw(-1*\ksepX+\kul+0.5*\kpsep,0)--++(0,-2*\kpin)coordinate(kkbb)--(kkbb |- kclk);
\draw(\kul+0.5*\kpsep,0)--++(0,-2*\kpin)coordinate(kkcc)--(kkcc |- kclk);
\draw(-2*\ksepX+\kul+2.5*\kpin,0)node[above]{گِن}--++(0,-2.5*\kpin)node[and port,scale=1,number inputs=2,anchor=out,rotate=90](u1){};
\draw(-\ksepX,0.5*\kydim)node[right]{$y$}--++(-0.5*\kpin,0) |-(u1.in 2);
\draw(u1.in 1)--++(0,-\kpin)coordinate(kua);
\draw(0,0.5*\kydim)node[right]{$y$}--++(-0.5*\kpin,0)coordinate(kct) |-(kua);
\draw(-1*\ksepX+\kul+2.5*\kpin,0)node[above]{گِن} --++ (0,-0.5*\kpin)coordinate(aaa)--(aaa -| kct);
\draw(\kul+2.5*\kpin,0)node[above]{گِن} --++(0,-3*\kpin)node[below]{1};
\draw(-2*\ksepX,0.5*\kydim)node[right]{$y$}--++(-0.5*\kpin,0);
\end{tikzpicture}
\caption{\عددی{000_{10}} تا \عددی{999_{10}} معاصر گنت کار۔}
\label{شکل_گنت_کار_معاصر_ہزار}
\end{figure}

اس دور کی کارکردگی کچھ یوں ہے۔گنتی شروع کرنے سے قبل \عددی{\overline{\text{\RL{زبردستی پست}}}} کو لمحاتی پست کر کے گنتی \عددی{000_{10}} کر دی جاتی ہے۔ساعت کے کنارہ چڑھائی پر اکائی ہندسے کی گنتی بتدریج بڑھتی ہے؛ اکائی درجے کا مخارج \عددی{y} پست رہتا ہے جو دہائی اور سینکڑا کی گنتی روک کر رکھتا ہے۔گنتی \عددی{009_{10}} تک پہنچتے ہی اکائی درجہ کا مخارج \عددی{y} ایک دوری عرصہ کے لئے بلند ہو گا۔یوں اگلے ساعت کے کنارہ چڑھائی پر اکائی درجہ کا ہندسہ \عددی{9_{10}}سے \عددی{0_{10}} ہو جائے گا، جبکہ دہائی درجے کا ہندسہ \عددی{0_{10}} سے بڑھ کر \عددی{1_{10}} ہو جائے گا اور اسی وقت اکائی کا مخارج \عددی{y} واپس پست حال اختیار کرے گا۔یوں اس سے اگلے ساعت کے کنارے پر صرف اکائی درجہ کی گنتی چالو رہتی ہے جبکہ دہائی اور سینکڑا کی گنتی رکی رہتی ہے۔اسی طرح \عددی{099_{10}} کے بعد اکائی اور دہائی درجات کے مخارج \عددی{y} بلند ہوتے ہیں جس کی وجہ سے اگلے ساعت کے کنارہ چڑھائی پر سینکڑا \عددی{0_{10}} سے بڑھ کر \عددی{1_{10}} ہو جائے گا جبکہ اکائی اور دہائی درجات \عددی{9_{10}} سے \عددی{0_{10}} ہو جائیں گے اور ساتھ ہی ان کے مخارج \عددی{y} دوبارہ پست ہو جائیں گے۔

\ابتدا{مشق}
انٹرنیٹ سے \عددی{7493} اور \عددی{4516} کے معلوماتی صفحات حاصل کریں۔انہیں استعمال کرتے ہوئے متعدد بِٹ گنت کار تخلیق دیں۔
\انتہا{مشق}


\حصہ{دیگر گنت کار}
\جزوحصہ{متغیر لمبائی گنت کار}
چار بِٹ ثنائی گنت کار \عددی{0000_2} تا \عددی{1111_2} گنتی کرتا ہے۔ متوازی دخول استعمال کرکے اس کو دو اعداد کے بیچ گنتی کرنے پر مجبور کیا جا سکتا ہے۔ایسے گنت کار کو ہم \اصطلاح{متغیر لمبائی گنت کار}\فرہنگ{گنت کار!متغیر لمبائی}\حاشیہب{variable length counter}\فرہنگ{counter!variable length}کہیں گے۔جس عدد سے گنتی کا آغاز کرنا ہو وہ عدد دور کو متوازی فراہم کیا جاتا ہے اور جہاں گنتی کا اختتام کرنا ہو وہاں پہنچ کر دور کو مجبور کیا جاتا ہے کہ وہ دوبارہ متوازی فراہم کردہ عدد داخل کر کے گنتی از سرے نو شروع کرے۔

چار بِٹ معاصر ثنائی گنت کار مثال بناتے ہوئے \عددی{0110_2} سے \عددی{1100_2} تک گنتی کرنے والا گنت کار بناتے ہیں، جو شکل  \حوالہ{شکل_گنت_کار_دو_ثنائی_کے_مابین_گنت_کار} میں پیش ہے۔ نقطہ دار مستطیل میں مساوات \حوالہ{مساوات_گنت_کار_تین_ثنائی} سے حاصل دور دکھایا گیا ہے، البتہ یہاں   ہر پلٹ کے ساتھ اضافی دو ضرب گیٹ اور ایک جمع گیٹ جوڑ کر متوازی دخول کی صلاحیت پیدا کی گئی ہے۔                                                                                      

\begin{figure}
\centering
\begin{tikzpicture}
\pgfmathsetmacro{\ksepX}{2.5}
\pgfmathsetmacro{\ksepY}{2}
\pgfmathsetmacro{\knshift}{0.07}
\pgfmathsetmacro{\kpin}{0.5}
\pgfmathsetmacro{\kpsep}{0.50}
\pgfmathsetmacro{\kul}{0.50}
\pgfmathsetmacro{\kmv}{0.15}
\pgfmathsetmacro{\kxdim}{2*\kul+3*\kpsep}
\pgfmathsetmacro{\kydim}{2.5*\kul+0*\kpsep}
\kuTFF[u0]{0}{0}
\kuTFF[u1]{-1*\ksepX}{0}
\kuTFF[u2]{-2*\ksepX}{0}
\kuTFF[u3]{-3*\ksepX}{0}
\draw(u0p6)node[above]{$Q_0$};
\draw(u1p6)node[above]{$Q_1$};
\draw(u2p6)node[above]{$Q_2$};
\draw(u3p6)node[above]{$Q_3$};
\draw(u0p1)node[or port,scale=0.9,rotate=90,anchor=out,number inputs=2](u4){};
\draw(u4.in 2)--++(1*\kpin,0)node[and port,scale=0.9,rotate=90,anchor=out,number inputs=2](u5){};
\draw(u4.in 1)--++(-1*\kpin,0)node[and port,scale=0.9,rotate=90,anchor=out,number inputs=2](u6){};
\draw(u1p1)node[or port,scale=0.9,rotate=90,anchor=out,number inputs=2](u7){};
\draw(u7.in 2)--++(1*\kpin,0)node[and port,scale=0.9,rotate=90,anchor=out,number inputs=2](u8){};
\draw(u7.in 1)--++(-1*\kpin,0)node[and port,scale=0.9,rotate=90,anchor=out,number inputs=2](u9){};
\draw(u2p1)node[or port,scale=0.9,rotate=90,anchor=out,number inputs=2](u10){};
\draw(u10.in 2)--++(1*\kpin,0)node[and port,scale=0.9,rotate=90,anchor=out,number inputs=2](u11){};
\draw(u10.in 1)--++(-1*\kpin,0)node[and port,scale=0.9,rotate=90,anchor=out,number inputs=2](u12){};
\draw(u3p1)node[or port,scale=0.9,rotate=90,anchor=out,number inputs=2](u13){};
\draw(u13.in 2)--++(1*\kpin,0)node[and port,scale=0.9,rotate=90,anchor=out,number inputs=2](u14){};
\draw(u13.in 1)--++(-1*\kpin,0)node[and port,scale=0.9,rotate=90,anchor=out,number inputs=2](u15){};
\draw(u5.in 2)++(6*\kpin,-6*\kpin)node[or port,scale=0.9,rotate=90,anchor=out,number inputs=2](u16){};
\draw(u16.out)--++(0,1.5*\kpin)coordinate(knot)--++(-\kpin,0)node[not port,scale=0.9,rotate=180,anchor=in](u17){};
\draw(u16.in 2)--++(1*\kpin,0)node[or port,scale=0.9,rotate=90,anchor=out,number inputs=2](u18){};
\draw(u16.in 1)--++(-1*\kpin,0)node[nand port,scale=0.9,rotate=90,anchor=out,number inputs=2](u19){};
\draw(u18.in 2)node[below,xshift=0.5ex]{$Q_0$};
\draw(u18.in 1)node[below]{$Q_1$};
\draw(u19.in 2)node[below]{$Q_2$};
\draw(u19.in 1)node[below,xshift=-0.5ex]{$Q_3$};
\draw(u5.in 2)--(u5.in 2 |- u19.in 1)node[below]{$\substack{A_0\\  0}$};
\draw(u8.in 2)--(u8.in 2 |- u19.in 1)node[below]{$\substack{A_1\\  1}$};
\draw(u11.in 2)--(u11.in 2 |- u19.in 1)node[below]{$\substack{A_2\\  1}$};
\draw(u14.in 2)--(u14.in 2 |- u19.in 1)node[below]{$\substack{A_3\\  0}$};
\draw(u14.in 1) |-(u17.out);
\draw(u5.in 1)--(u5.in 1 |- u17.out);
\draw(u8.in 1)--(u8.in 1 |- u17.out);
\draw(u11.in 1)--(u11.in 1 |- u17.out);
\draw(u15.in 1)--++(0,-\kpin)coordinate(knota)-|(knot);
\draw(u6.in 1)--(u6.in 1 |- knota);
\draw(u9.in 1)--(u9.in 1 |- knota);
\draw(u12.in 1)--(u12.in 1 |- knota);
\draw(u12.in 2)--(u12.in 2 |- u16.out)node[and port,scale=0.9,anchor=out,rotate=90,number inputs=2](u20){};
\draw(u20.in 1)node[below]{$Q_1$};
\draw(u20.in 2)node[below]{$Q_0$};
\draw(u15.in 2)--(u15.in 2 |- u16.out)node[and port,scale=0.9,anchor=out,rotate=90,number inputs=3](u21){};
\draw(u21.in 1)node[below,xshift=-0.75ex]{$Q_2$};
\draw(u21.in 2)node[below]{$Q_1$};
\draw(u21.in 3)node[below,xshift=0.75ex]{$Q_0$};
\draw(u6.in 2)--(u6.in 2 |- u21.in 1)coordinate(rectA)node[below]{$1$};
\draw(u9.in 2)--(u9.in 2 |- u21.in 1)node[below]{$Q_0$};
\draw(u3p2)--(u0p2)--++(4*\kpin,0)node[right]{ساعت};
\draw[gray,dashed] (u21.out)++(-3*\kpin,0) rectangle ($(rectA)+(\kpin,-2*\kpin)$);
\end{tikzpicture}
\caption{دو ثنائی اعداد  ، \عددی{0110_2}  اور \عددی{1100_2} ، کے  بیچ گنتی کرنے والا  معاصر گنت کار}
\label{شکل_گنت_کار_دو_ثنائی_کے_مابین_گنت_کار}
\end{figure}

اس دور میں ابتدائی عدد ، جس کو \عددی{A_3A_2A_1A_0} سے ظاہر کیا گیا ہے اور جس کی قیمت \عددی{0110_2} ہے، متوازی داخل کیا جاتا ہے۔ اختتامی عدد \عددی{1100_2} ہے۔ ایک ضرب متمم اور دو جمع گیٹ پر مشتمل دور  اختتامی عدد کو پہچان کر نفی گیٹ کا مداخل پست کرتا ہے اور یوں     ساعت کے اگلے کنارے پر \عددی{0110_2} دور میں متوازی داخل ہو گا۔اس طرح  گنت کار \عددی{0110_2}   اور  \عددی{1100_2}  کے بیچ گنتی کرتا ہے۔

دور میں \عددی{0110_2}پہلی مرتبہ داخل کرنے کا طریقہ نہیں دکھایا گیا۔


\جزوحصہ{بے ترتیب گنت کار}
معاصر ثنائی گنت کار پر بحث کے دوران جدول \حوالہ{جدول_گنت_کار_تین_بِٹ_معاصر} پیش کیا گیا ۔ اس جدول کے \موٹا{موجودہ حال} خانوں میں \عددی{000_2}، \عددی{001_2}، \عددی{011_2}، وغیرہ پُر کر کے باقی جدول حاصل کیا گیا۔ یوں حاصل گنت کار \عددی{000_2} سے بتدریج بڑھتے ہوئے \عددی{111_2} تک گنتا ہے۔

یہ ضروری نہیں کہ گنت کار عام فہم گنتی کی ترتیب میں ہی گننے۔ \موٹا{ موجودہ حال} صفوں میں کوئی بھی ترتیب لکھی جا سکتی ہے۔ فقط اتنا خیال رکھنا ضروری ہے کہ ہر صف میں منفرد عدد لکھا جائے۔ باقی جدول ان اندراج کے مطابق پورا کرنے سے ایسا گنت کار حاصل ہو گا جو \موٹا{موجودہ حال} صفوں میں لکھے گئے اعداد کے مطابق گنتی کرے گا۔ ہم اس کو \اصطلاح{بے ترتیب گنت کار }\فرہنگ{گنت کار!بے ترتیب} پکار سکتے ہیں۔

\ابتدا{مشق}\شناخت{مشق_گنت_کار_بے_ترتیب}
ایسا\اصطلاح{ بے ترتیب گنت کار } تخلیق دیں جو جدول \حوالہ{جدول_گنت_کار_بلا_ترتیب} میں پیش اعداد کی ترتیب کے مطابق گنتا ہو۔ یہ گنت کار \عددی{101_2} سے آغاز کرے گا۔ پہلی ساعت پر \عددی{011_2} اور دوسری ساعت پر \عددی{110_2} دے گا اور \عددی{001_2} تک پہنچنے کے بعد دوبارہ \عددی{101_2} سے گننا شروع کرے گا۔
\begin{table}
\caption{بے ترتیب گنت کار، برائے مشق \حوالہ{مشق_گنت_کار_بے_ترتیب}}
\label{جدول_گنت_کار_بلا_ترتیب}
\centering
\begin{otherlanguage}{english}
\begin{tabular}{CCC}
\toprule
\multicolumn{3}{c}{\text{\RL{موجودہ حال}}}\\
\midrule
Q_2&Q_1&Q_0\\
\midrule
1&0&1\\
0&1&1\\
1&1&0\\
0&1&0\\
1&0&0\\
0&0&0\\
0&0&1\\
\bottomrule
\end{tabular}
\end{otherlanguage}
\end{table}

\انتہا{مشق}

\جزوحصہ{ چھلا گنت کار}
\عددی{n} \اصطلاح{بِٹ چھلا گنت کار}\فرہنگ{گنت کار!چھلا}\حاشیہب{ring counter}\فرہنگ{counter!ring} کے مخارج میں ایک ہی بلند بِٹ گھومتا ہے؛ باقی تمام بِٹ پست رہتے ہیں۔ایک ہی بلند بِٹ کو ساعت کے کنارے پر ایک پلٹ سے دوسرے پلٹ منتقل کیا جاتا ہے۔شکل   \حوالہ{شکل_گنت_کار_چھلا}  میں چار بِٹ چھلا گنت کار پیش ہے، جبکہ جدول \حوالہ{جدول_گنت_کار_چھلا} میں اس کی گنتی پیش کی گئی ہے۔ آغاز  میں \عددی{\overline{\text{\RL{ایک بِٹ بلند}}}}  اشارہ لمحاتی  پست کر کے \عددی{Q_3} بلند جبکہ \عددی{Q_0}، \عددی{Q_1}، اور \عددی{Q_2} پست کیے جاتے ہیں۔ ساعت کے پہلے کنارے پر  \عددی{Q_3} کا مواد \عددی{Q_2} منتقل ہو گا۔ یوں اب \عددی{Q_2} بلند جبکہ باقی بِٹ پست ہوں گے۔ باب کے آخر میں آپ سے گزارش کی جائے گی کہ ایسا  چھلا گنت کار تخلیق دیں جو بلند بِٹ کو مخالف رخ (\عددی{Q_0} سے \عددی{Q_1} جانب)  گھماتا ہو۔
%
\begin{figure}
\centering
\begin{tikzpicture}
\pgfmathsetmacro{\ksepX}{2.5}
\pgfmathsetmacro{\kpin}{0.4}
\kTFFd[u0]{0}{0}
\kTFFd[u1]{-1*\ksepX}{0}
\kTFFd[u2]{-2*\ksepX}{0}
\kTFFu[u3]{-3*\ksepX}{0}
\draw(u3p6)--(u2p1)   (u2p6)--(u1p1)  (u1p6)--(u0p1);
\draw(u0p6)--++(\kpin,0)--++(0,4*\kpin)-|(u3p1);
\draw(u0pd)--(u3pd)--++(-6*\kpin,0)coordinate(bb)coordinate[pos=0.8](aa)node[left]{$\overline{\text{\RL{ایک بِٹ بلند}}}$};
\draw(aa)|-(u3pu);
\draw(u0p2)--++(0,-5.5*\kpin)coordinate(cc)--(cc -|bb)node[left]{ساعت};
\draw(u1p2)--(u1p2 |- cc);
\draw(u2p2)--(u2p2 |- cc);
\draw(u3p2)--(u3p2 |- cc);
\draw(u0p6)node[above]{$Q_0$};
\draw(u1p6)node[above]{$Q_1$};
\draw(u2p6)node[above]{$Q_2$};
\draw(u3p6)node[above]{$Q_3$};
\end{tikzpicture}
\caption{چھلا گنت کار}
\label{شکل_گنت_کار_چھلا}
\end{figure}
%
 چار بِٹ  چھلا گنت کار  میں چار  متغیرات    پائے جاتے ہیں جن کی  سولہ منفرد ترتیب  (\عددی{0000_2} تا \عددی{1111_2}) ممکن ہیں۔جدول  \حوالہ{جدول_گنت_کار_چھلا} میں صرف وہ   صورتیں دکھائی گئی ہیں جو حقیقتاً پائی جاتی ہیں. باقی صورتیں (مثلاً \عددی{1011} یا \عددی{0101}) \اصطلاح{ غیر  دلچسپ } ہیں جنہیں کارناف نقشوں میں \عددی{d} درج کیا جائے گا۔  شکل \حوالہ{شکل_گنت_کار_چار_بِٹ_کارناف} میں مداخل \عددی{T_3} کے لئے  جدول سے کارناف نقشہ پُر کیا گیا ہے، جہاں سے \عددی{T_3=Q_0} حاصل کیا گیا ہے۔ چھلا گنت کار میں آپ دیکھ سکتے ہیں کہ بائیں ترین پلٹ کا مداخل \عددی{T_3}، دائیں ترین پلٹ کے مخارج \عددی{Q_0} سے حاصل کیا گیا ہے۔
\begin{table}
\caption{چار بِٹ چھلا گنت کار}
\label{جدول_گنت_کار_چھلا}
\centering
\begin{otherlanguage}{english}
\begin{tabular}{CCCC|CCCC|CCCC}
\toprule
\multicolumn{4}{c}{\text{\RL{موجودہ حال}}}&\multicolumn{4}{|c|}{\text{\RL{اگلا حال}}}&
\multicolumn{4}{c}{\text{\RL{مداخل }}}\\
\midrule
Q_3&Q_2&Q_1&Q_0&Q_3&Q_2&Q_1&Q_0&T_3&T_2&T_1&T_0\\
\midrule
1&0&0&0&0&1&0&0&0&1&0&0\\
0&1&0&0&0&0&1&0&0&0&1&0\\
0&0&1&0&0&0&0&1&0&0&0&1\\
0&0&0&1&1&0&0&0&1&0&0&0\\
\bottomrule
\end{tabular}
\end{otherlanguage}
\end{table}
\begin{figure}
\centering
\begin{tikzpicture}
\pgfmathsetmacro{\kxstep}{1}
\pgfmathsetmacro{\kystep}{1}
\pgfmathsetmacro{\kpin}{0.75}
\pgfmathsetmacro{\kmv}{0.15}
\draw[xstep=\kxstep,ystep=\kystep](0,0) grid (4*\kxstep,-4*\kystep);
\draw(0,0)--++(135:\kpin)node[pos=0.75,above right]{$Q_1Q_0$}node[pos=0.75,below left]{$Q_3Q_2$};
\foreach \kx/\xlb in {0/{00},1/{01},2/{11},3/{10}}{\draw(\kx*\kxstep+\kxstep/2,0)node[above]{$\xlb$};}
\foreach \ky/\ylb in {0/{00},1/{01},2/{11},3/{10}}{\draw(0,-\ky*\kystep-\kystep/2)node[left]{$\ylb$};}
\foreach \kx/\xlb in {0/{d},1/{1},2/d,3/0}{\draw(\kx*\kxstep+\kxstep/2,-\kystep/2)node[]{$\xlb$};}
\foreach \kx/\xlb in {0/{0},1/{d},2/d,3/d}{\draw(\kx*\kxstep+\kxstep/2,-1.5*\kystep)node[]{$\xlb$};}
\foreach \kx/\xlb in {0/{d},1/{d},2/d,3/d}{\draw(\kx*\kxstep+\kxstep/2,-2.5*\kystep)node[]{$\xlb$};}
\foreach \kx/\xlb in {0/{0},1/{d},2/d,3/d}{\draw(\kx*\kxstep+\kxstep/2,-3.5*\kystep)node[]{$\xlb$};}
\draw[gray,dashed] ($(1*\kxstep,0)+(\kmv,-\kmv)$) rectangle ($(3*\kxstep,-4*\kystep)+(-\kmv,\kmv)$);
\end{tikzpicture}\quad\quad
\(T_3=Q_0\)
\caption{چھلا گنت کار کے مداخل \عددی{T_3} کا حصول۔}
\label{شکل_گنت_کار_چار_بِٹ_کارناف}
\end{figure}


\جزوحصہ{دھڑکن  پیدا کار}
بعض اوقات ہمیں مقررہ دورانیہ کا بلند یا پست اشارہ درکار ہوتا ہے۔تین بِٹ کا معاصر ثنائی الٹ گنت کار استعمال کرتے ہوئے ایسا دور تشکیل دیتے ہیں۔اس دور کو ہم \اصطلاح{ دھڑکن  پیدا کار }\فرہنگ{دھڑکن پیدا کار}\حاشیہب{pulse generator}\فرہنگ{pulse generator} کہیں گے۔ 
\begin{figure}
\centering
\begin{tikzpicture}
\pgfmathsetmacro{\ksepX}{2.5}
\pgfmathsetmacro{\ksepY}{2}
\pgfmathsetmacro{\knshift}{0.07}
\pgfmathsetmacro{\kpin}{0.5}
\pgfmathsetmacro{\kpsep}{0.50}
\pgfmathsetmacro{\kul}{0.50}
\pgfmathsetmacro{\kmv}{0.15}
\pgfmathsetmacro{\kxdim}{2*\kul+1*\kpsep}
\pgfmathsetmacro{\kydim}{2*\kul+4*\kpsep}
\draw[thick](0,0) rectangle ++(\kxdim,\kydim);
\draw(0,\kul-\kmv)--++(\kmv,\kmv)--++(-\kmv,\kmv);
\foreach \x/\a in {0/{},2/{d_0},3/{d_1},4/{d_2}}{\draw(0,\kul+\x*\kpsep)node[right]{$\a$}--++(-\kpin,0);}
\draw(\kxdim+\kpin,\kul+3*\kpsep)node[or port,number inputs=3,scale=1,anchor=in 2](u1){};
\draw(u1.in 1)--++(-\kpin,0)node[left]{$Q_2$};
\draw(u1.in 2)--++(-\kpin,0)node[left]{$Q_1$};
\draw(u1.in 3)--++(-\kpin,0)node[left]{$Q_0$};
\draw(u1.out)--++(\kpin,0)node[right]{دھڑکن};
\draw(u1.out)--++(0,3*\kpin)-|(0.5*\kxdim,\kydim)node[below]{گِن};
\draw(-\kpin,\kul)coordinate(kclk)node[left]{ساعت};
\draw(-\kpin,\kul+3*\kpsep)node[above,rotate=90]{\text{\RL{درکار دورانیہ}}};
\draw(0.5*\kxdim,0)--++(0,-\kpin)coordinate(aa)--(aa -|kclk)node[left]{\text{\RL{متوازی لکھ}}};
\end{tikzpicture}
\caption{دھڑکن پیدا کار}
\label{شکل_گنت_کار_دھڑکن}
\end{figure}

تین بِٹ الٹ گنت کار \عددی{111_2} تا \عددی{000_2} دہراتا ہے۔شکل   \حوالہ{شکل_گنت_کار_دھڑکن}  میں متوازی دخول صلاحیت رکھنے والا تین بِٹ الٹ گنت کار استعمال کیا گیا ہے جو اس دوران گنتی کرے گا جب مداخل \موٹا{گِن} بلند ہو۔اس دور کو تین بِٹ بطور درکار دورانیہ  فراہم کیے جاتے ہیں، جو \موٹا{متوازی لکھ } مداخل لمحاتی بلند کرنے سے گنت کار میں لکھے جاتے ہیں۔جب تک گنت کار کے تینوں خارجی بِٹ  بیکوقت پست \حاشیہد{یہ دور لرزش کا شکار ہو سکتا ہے جس سے بچنے کی بات ہم یہاں نہیں کرتے۔ باب \حوالہ{باب_غیر_معاصر} میں \موٹا{ لرزش}  پر تفصیلاً غور کیا جائے گا۔} نہ ہوں جمع گیٹ بلند رہتا ہے  لہٰذا  گنت کار الٹ گنتی جاری رکھے گا۔جیسے ہی گنت کار \عددی{000_2} کو پہنچتا ہے ، جمع گیٹ کا مخارج پست ہو گا اور گنت کار گنتی روک دے گا۔  یوں تین بِٹ میں پیش درکار  دورانیہ   کے لئے   \موٹا{دھڑکن} بلند رہتا ہے۔


\حصہء{سوالات}
\ابتدا{سوال}
%8.1
 چار بٹ معاصر سیدھا گنت کار کی موجودہ گنتی  \عددی{0101_2} ہے۔ساعت کے کتنے کناروں بعد  \عددی{0000_2} ہو گا؟
 
 جواب: گیارہ کناروں بعد۔
\انتہا{سوال}
\ابتدا{سوال}
%8.2
 سولہ بٹ معاصر گنت کار کی موجودہ گنتی  \عددی{3FA7_{16}}ہے۔ساعت کے کتنے کنارے گزرنے کے بعد \عددی{0000_{16}} ہو گا؟ (ا) تصور کریں یہ سیدھا گنت کار ہے۔ (ب) تصور کریں یہ الٹ گنت کار ہے۔
 
 جواب: (ا)  \عددی{49241_{10}}، (ب)    \عددی{16295_{10}}
\انتہا{سوال}
\ابتدا{سوال}
%8.3
 چار بٹ ثنائی لہریا گنت کار  استعمال کر کے  ثنائی مرموز اعشاری گنت کار  بنایا جا سکتا ہے۔ پس اتنا کرنا ہو گا کہ \عددی{1010_2} پر پہنچ کر  گنتی   فوراً زبردستی \عددی{0000_2} کی جائے۔ زبردستی پست صلاحیت رکھنے والی پلٹ استعمال کرتے ہوئے   دور تخلیق دیں۔ 
\انتہا{سوال}
\ابتدا{سوال}
%8.4
 ڈی پلٹ استعمال کرتے ہوئے چار بٹ معاصر ثنائی گنت کار تشکیل دیں۔ 
\انتہا{سوال}
\ابتدا{سوال}
%8.5
 جے کے پلٹ استعمال کر کے  ایسا معاصر گنت کار تشکیل دیں  جو   \عددی{0}، \عددی{2}، \عددی{3}، اور \عددی{7}   کا گردان کرے۔ جدول  لکھ کر سے شروع کریں۔ گنت کار میں زبردستی پست کا مداخل رکھیں تا کہ \عددی{0} سے گردان شروع کی جائے۔
 
 جواب:
 \begin{center}
 \begin{otherlanguage}{english}
 \begin{tabular}{CCC|CCC}
 \toprule
 \multicolumn{3}{c|}{\text{\RL{موجودہ گنتی}}}&\multicolumn{3}{c}{\text{\RL{اگلی گنتی}}}\\
 Q_2&Q_1&Q_0&Q_2&Q_1&Q_0\\
 \midrule
 0&0&0&0&1&0\\
 0&0&1&d&d&d\\
 0&1&0&0&1&1\\
 0&1&1&1&1&1\\
 1&0&0&d&d&d\\
 1&0&1&d&d&d\\
 1&1&0&d&d&d\\
 1&1&1&0&0&0\\
 \bottomrule
 \end{tabular}
 \end{otherlanguage}
 \end{center}
\انتہا{سوال}
\ابتدا{سوال}
%8.6
 ٹی پلٹ استعمال کرتے ہوئے ایسا  چار بٹ ثنائی معاصر گنت کار تشکیل دیں جو صفر  \عددی{(0000_2)} سے چودہ  \عددی{(1110_2)} تک جفت گنتی کے بعد ایک  \عددی{(0001_2)} سے پندرہ  \عددی{(1111_2)} تک طاق گنتی کرے اور اس ترتیب کو دہراتا ر ہو۔ابتدا \عددی{0000_2} سے کریں۔
\انتہا{سوال}
\ابتدا{سوال}
%Q8.6b
ایسا چار بِٹ چھلا گنت کار تخلیق دیں جو بلند بِٹ کو \عددی{Q_0} سے \عددی{Q_1} رخ گھماتا ہو۔
\انتہا{سوال}
\ابتدا{سوال}
%8.7
 شکل  \حوالہ{شکل_گنت_کار_دھڑکن}  میں  دھڑکن پیدا کار (دورانیہ پیدا کار ) دکھایا گیا ہے۔ ساعت کا تعدد \عددی{\SI{10}{\mega\hertz}}  اور درکار دورانیہ \عددی{\SI{500}{\nano\second}}  ہے۔درکار دورانیہ کے تین بٹ کیا ہوں گے؟
 
 جواب: \عددی{110_2}
\انتہا{سوال}
%???KKK
\ابتدا{سوال}
%8.8
 کارناف نقشے  استعمال  کر کے  مساوات \حوالہ{مساوات_گنت_کار_چار_بِٹ} حاصل کریں۔گنت کار کے جدول سے ابتدا کریں۔
\انتہا{سوال}
\ابتدا{سوال}
%8.9
جےکے پلٹ استعمال کرتے ہوئے  مساوات  \حوالہ{مساوات_گنت_کار_چار_بِٹ} کی متبادل مساوات   کیا ہو ں گی؟
\انتہا{سوال}
%=============


\باب{ حافظہ}
ایک پلٹ ایک\اصطلاح{ ثنائی ہندسہ }\فرہنگ{ثنائی ہندسہ} معلومات (مواد) ذخیرہ کرنے کی صلاحیت رکھتا ہے۔ثنائی ہندسے کو \اصطلاح{بِٹ }\فرہنگ{بِٹ}\حاشیہب{bit}\فرہنگ{bit} بھی کہتے ہیں۔یوں ایک پلٹ ایک ثنائی ہندسہ \اصطلاح{حافظہ }\فرہنگ{حافظہ}\حاشیہب{memory}\فرہنگ{memory} کے طور پر کام کر سکتا ہے۔آٹھ پلٹ جوڑ کر آٹھ ثنائی ہندسہ حافظہ حاصل کیا جا سکتا ہے۔اسی طرح \عددی{n} بِٹ پلٹ سے \عددی{n} بِٹ حافظہ بنایا جا سکتا ہے۔آٹھ ثنائی بِٹ کو ایک \اصطلاح{ ہشتمی عدد }\فرہنگ{ہشتمی عدد} یا ایک \اصطلاح{ بائٹ }\فرہنگ{بائٹ}\حاشیہب{byte}\فرہنگ{byte} کہتے ہیں۔حافظہ میں رکھے گئے مواد کو\اصطلاح{ لفظ }\فرہنگ{لفظ}\حاشیہب{word}\فرہنگ{word} کہتے ہیں۔حافظہ میں \اصطلاح{ الفاظ } کی لمبائی قطعی ہوتی ہے۔ یوں آٹھ بِٹ لفظ ایک بائٹ پر مشتمل ہو گا جبکہ سولہ بِٹ لفظ دو بائٹ پر مشتمل ہو گا۔کمپیوٹر میں موجود کل حافظے کی پیمائش بائٹ میں بیان کی جاتی ہے۔یوں دو سو الفاظ کا حافظہ جس میں ہر لفظ ایک بائٹ پر مشتمل ہو\موٹا{ دو سو بائٹ حافظہ }کہلائے گا۔حافظہ میں مواد داخل کرنے کو مواد\اصطلاح{ لکھنا }\فرہنگ{لکھنا}\حاشیہب{write}\فرہنگ{write} یا حافظہ لکھنا کہتے ہیں جبکہ حافظہ سے مواد کے حصول کو مواد \اصطلاح{ پڑھنا }\فرہنگ{پڑھنا}\حاشیہب{read}\فرہنگ{read} یا حافظہ پڑھنا کہتے ہیں۔اس باب میں انہیں قسم کے برقیاتی حافظہ پر غور کیا جائے گا۔


حافظوں کی دو اہم قسمیں ہیں۔حافظہ کی پہلی قسم ، جو \اصطلاح{عارضی حافظہ }\فرہنگ{حافظہ!عارضی}\حاشیہب{random access memory, RAM}\فرہنگ{memory!RAM} کہلاتا ہے، میں معلومات اس وقت تک محفوظ رہتی ہے جتنی دیر حافظے کو درکار برقی طاقت مہیا کی جائے۔کسی بھی وقت ، عارضی حافظے میں کسی بھی مقام پر معلومات لکھی یا اس مقام سے معلومات پڑھی جا سکتی ہے۔ معلومات کا، حافظہ میں کسی بھی مقام پر لکھنے یا اس سے پڑھنے میں درکار وقت تمام مقامات کے لئے تقریباً برابر ہو گا۔اس دورانیہ کو\اصطلاح{ حافظہ کا دورانیہ رسائی }\فرہنگ{حافظہ!دورانیہ رسائی}یا مختصراً \اصطلاح{ دورانیہ رسائی }\فرہنگ{دورانیہ رسائی}\حاشیہب{access time}\فرہنگ{access time} کہتے ہیں۔


دوسری قسم کا حافظہ ، جو \اصطلاح{ پختہ حافظہ }\فرہنگ{حافظہ!پختہ}\حاشیہب{ROM, read only memory}\فرہنگ{memory!ROM} کہلاتا ہے، میں برقی طاقت کی عدم موجودگی میں بھی مواد محفوظ رہتا ہے تاہم اس سے معلومات پڑھنے کی خاطر حافظے کو درکار برقی طاقت فراہم کرنا لازم ہے۔ پختہ حافظہ سے معلومات کسی بھی وقت کسی بھی مقام سے پڑھی جا سکتی ہے۔حافظے کے تمام مقامات سے مواد پڑھنے کے لئے درکار وقت، جو حافظہ کا \اصطلاح{ دورانیہ رسائی }کہلاتا ہے، تقریباً ایک جیسا ہو گا۔ عام استعمال میں پختہ حافظہ سے معلومات صرف پڑھی جاتی ہے۔پختہ حافظوں کی مختلف اقسام میں معلومات محفوظ کرنے کے طریقے ایک دوسرے سے مختلف ہوں گے۔ایک قسم کے پختہ حافظہ میں معلومات صرف اور صرف ایک مرتبہ لکھی جا سکتی ہے، لہٰذا اسے صرف ایک مرتبہ معلومات کی لکھائی کے لئے استعمال کیا جا سکتا ہے۔اس کو \اصطلاح{ایک مرتبہ قابل لکھائی پختہ حافظہ }\فرہنگ{پختہ حافظہ!ایک مرتبہ قابل لکھائی}\حاشیہب{one time programmable read only memory, OTP}\فرہنگ{OTP} کہتے ہیں۔دوسری قسم کی پختہ حافظہ میں معلومات بار بار لکھی جا سکتی ہے تاہم ایسا کرنے سے پہلے اس سے پرانی معلومات\موٹا{ مٹانی} ضروری ہے۔جدید پختہ حافظہ سے معلومات برق کی مدد سے مٹائی جاتی ہے۔ایسے پختہ حافظہ کو \اصطلاح{برق مٹتا پختہ حافظہ }\فرہنگ{پختہ حافظہ!برق مٹتا}\حاشیہب{electrically erasable read only memory, EEROM, \(E^2PROM\)}\فرہنگ{ROM!EEROM} کہتے ہیں۔شروع میں پختہ حافظہ کی ایک قسم کو شعاع سے مٹایا جاتا تھا۔اس کو \اصطلاح{ شعاع مٹتا پختہ حافظہ }\فرہنگ{پختہ حافظہ!شعاع مٹتا}\حاشیہب{UV erasable read only memory, UV erasable ROM}\فرہنگ{ROM!UV erasable} کہتے ہیں۔

کاغذ پر لکھائی کو مٹانے سے صاف ستھرا کاغذ ملتا ہے ۔ پلٹ ہر صورت بلند یا پست حال ہوتا ہے لہٰذا اس سے مواد کاغذ کی طرح نہیں مٹایا جا سکتا۔ لکھائی سے صاف حافظہ سے مراد وہ حافظہ ہو گا جس کے تمام بِٹ بلند \عددی{(1)} ہوں۔ جدول \حوالہ{جدول_حافظہ_خالی} میں آٹھ بِٹ لمبائی کے چار لفظ حافظہ استعمال کرتے ہوئے مواد سے بھرے اور خالی حافظہ کی وضاحت کی گئی ہے۔ یقیناً ، حافظہ کے تمام بِٹ پر \عددی{1} لکھنا اور حافظے سے مواد مٹانا ایک جیسا ہو گا۔
\begin{table}
\caption{حافظہ سے مواد مٹانے کا مفہوم}
\label{جدول_حافظہ_خالی}
\centering
\begin{subtable}[t]{0.48\textwidth}
\centering
\begin{tabular}{C}
\toprule
1011\,0101\\
0000\,0000\\
1111\,1111\\
0110\,0110\\
\bottomrule
\end{tabular}
\caption{مواد سے بھرا حافظہ}
\end{subtable}%
\begin{subtable}[t]{0.48\textwidth}
\centering
\begin{tabular}{C}
\toprule
1111\,1111\\
1111\,1111\\
1111\,1111\\
1111\,1111\\
\bottomrule
\end{tabular}
\caption{مواد سے خالی حافظہ}
\end{subtable}
\end{table}


\حصہ{عارضی حافظہ}
اس حصے میں عارضی حافظے کی بناوٹ پر غور کیا جائے گا۔ایک بِٹ حافظہ بنیادی طور ایک پلٹ ہوگا،جس میں مواد لکھنے اور پڑھنے کی صلاحیت موجود ہو گی۔ حافظہ عموماً کثیر تعداد بِٹوں پر مشتمل ہوگا لہٰذا حافظہ میں ہر پلٹ تک، لکھنے اور پڑھنے کی خاطر ، رسائی ضروری ہے۔شکل\حوالہء{ 9.1 } میں\اصطلاح{ ثنائی عارضی حافظے کی اکائی }\فرہنگ{عارضی حافظہ!اکائی}\حاشیہب{binary memory cell}\فرہنگ{memory!binary cell} ، جس کو مختصراً\اصطلاح{ اکائی حافظہ }\فرہنگ{حافظہ!اکائی}\حاشیہب{unit memory}\فرہنگ{memory!unit} کہتے ہیں، کی بناوٹ اور علامت پیش ہے، جہاں مواد ذخیرہ کرنے کے لئے ایس آر پلٹ استعمال کیا گیا ہے۔حقیقت میں کئی طریقے مستعمل ہیں جن پر بعد میں غور کیا جائے گا۔ 


 اکائی حافظہ سے رجوع کے لئے اس کا \اصطلاح{ منتخب} اشارہ بلند کیا جاتا ہے اور مواد لکھنے کی خاطر ساتھ ہی \عددی{\overline{\text{لکھ}}/\text{پڑھ}} پست کر کے داخلی مواد فراہم کیا جاتا ہے جبکہ مواد پڑھنے کی خاطر \عددی{\overline{\text{لکھ}}/\text{پڑھ}} بلند کر کے مواد پڑھا جاتا ہے۔

متعدد بِٹ حافظہ اس اکائی حافظہ کی مدد سے حاصل ہو گا۔شکل \حوالہء{9.2 } میں چار بِٹ لفظ کا حافظہ پیش ہے جہاں تمام اکائی حافظوں کے "منتخب" قابو اشارے ایک ساتھ اور "\عددی{\overline{\text{لکھ}}/\text{پڑھ}} " ایک ساتھ جوڑے گئے ہیں۔یوں لفظ کے چاروں بِٹ بیک وقت منتخب ہوتے ہیں اور اس میں مواد \عددی{D} بیک وقت لکھا، یا ذخیرہ مواد بیک وقت پڑھا جا سکتا ہے۔

اس طرح کے کئی الفاظ جوڑ کر متعدد لفظ حافظہ حاصل کیا جا سکتا ہے۔شکل \حوالہء{9.3 } میں چار الفاظ جوڑ کر چار لفظ حافظہ تخلیق دیا گیا ہے۔

متعدد لفظ حافظہ کی تمام اکائیوں کا "منتخب "اشارہ عام صورت میں پست رہتا ہے۔یوں حافظہ کے کسی بھی لفظ تک رسائی ممکن نہیں ہو گی۔حافظہ میں مواد لکھنے کی خاطر مواد \عددی{Z} داخلی راستے فراہم کر کے \عددی{\overline{\text{لکھ}}/\text{پڑھ}} پست رکھ کر مطلوبہ مقام کا "منتخب " اشارہ بلند کیا جاتا ہے۔یوں مواد مطلوبہ لفظ کے مقام پر لکھا جاتا ہے۔ فرض کریں ہم اعشاری تین \عددی{(3_{10})} کے ثنائی علامتی روپ \عددی{0011_2} کو حافظہ کے لفظ \عددی{2} کے مقام پر لکھنا چاہتے ہیں۔ہم مداخل پر \عددی{0011_2} مہیا کر کے \عددی{\overline{\text{لکھ}}/\text{پڑھ}} پست رکھ کر لفظ \عددی{2} کے "منتخب" اشارے کو بلند کریں گے۔ایسا کرنے سے شکل \حوالہء{9.3} میں لفظ \عددی{2} پر \عددی{0011_2} لکھا جائے گا۔یاد رہے کہ اس دوران باقی" منتخب " اشارے پست رہیں گے۔اسی لفظ کو پڑھنے کے لئے ہم \عددی{\overline{\text{لکھ}}/\text{پڑھ}} بلند رکھ کر لفظ \عددی{2} کا "منتخب" بلند کریں گے۔ایسا کرنے سے مخارج \عددی{D} پر \عددی{0011_2} خارج ہو گا جہاں سے اسے پڑھا جا سکتا ہے۔


حقیقی حافظہ میں الفاظ تک رسائی \موٹا{ پتہ } کے ذریعے کی جاتی ہے۔چار لفظ حافظہ میں الفاظ تک رسائی ، دو بِٹ پتہ استعمال کرتے ہوئے دو سے چار شناخت کار کی مدد سے ممکن ہے۔شکل \حوالہء{9.4 } میں یہ عمل پیش کیا گیا ہے۔

عارضی حافظہ کا استعمال جدول \حوالہ{جدول_حافظہ_عارضی_استعمال} میں دکھایا گیا ہے۔ \موٹا{مجاز } پست ہونے کی صورت میں حافظہ \اصطلاح{بلند رکاوٹی حال }\فرہنگ{حال!بلند رکاوٹی}\حاشیہب{high impedance state}\فرہنگ{state!high impedance} اختیار کر کے بیرونی ادوار سے مکمل منقطع ہو گا۔
\begin{table}
\caption{عارضی حافظے کا استعمال}
\label{جدول_حافظہ_عارضی_استعمال}
\centering
\begin{otherlanguage}{english}
\begin{tabular}{CCCC|R}
\toprule
\text{\RL{مجاز}} & \overline{\text{\RL{لکھ}}}/\text{\RL{پڑھ}} & A_1 & A_0 & \multicolumn{1}{c}{\text{\RL{عمل}}}\\
\midrule
0& \times &\times &\times &\text{\RL{بلند رکاوٹی حال}}\\
1&0&0&0&\text{\RL{لفظ \عددی{0} کے مقام پر لکھ}}\\
1&0&0&1&\text{\RL{لفظ \عددی{1} کے مقام پر لکھ}}\\
1&0&1&0&\text{\RL{لفظ \عددی{2} کے مقام پر لکھ}}\\
1&0&1&1&\text{\RL{لفظ \عددی{3} کے مقام پر لکھ}}\\
1&1&0&0&\text{\RL{لفظ \عددی{0} کے مقام سے پڑھ}}\\
1&1&0&1&\text{\RL{لفظ \عددی{1} کے مقام سے پڑھ}}\\
1&1&1&0&\text{\RL{لفظ \عددی{2} کے مقام سے پڑھ}}\\
1&1&1&1&\text{\RL{لفظ \عددی{3} کے مقام سے پڑھ}}\\
\bottomrule
\end{tabular}
\end{otherlanguage}
\end{table}


شکل \حوالہء{ 9.4 } میں چار بِٹ جمع گیٹ کی ایک نئی علامت استعمال کی گئی ہے ۔گیٹ کا ایک مداخل دکھایا گیا ہے جس پر چھوٹی ترچھی لکیر کے ساتھ \عددی{4} لکھ کر اس بات کی وضاحت کی گئی ہے کہ دراصل یہ چار داخلی جمع گیٹ ہے۔اس طرح کی علامت میں گیٹ کے مداخل علیحدہ علیحدہ نہیں دکھائے جاتے بلکہ تمام مداخل ایک داخلی تار سے ظاہر کیے جاتے ہیں۔یوں دور کا نقشہ کاغذ پر کھینچتے ہوئے ہوئے تاروں کے ہجوم سے نجات حاصل ہوتی ہے اور دور صاف ستھرا نظر آتا ہے۔یاد رہے کہ ایسا صرف دور صاف ستھرا نظر آنے کے لئے کیا جاتا ہے۔یوں حافظہ کے گزشتہ دو اشکال ایک ہی دور بنانے کے دو طریقے ہیں۔

اسی طرز پر متعدد لفظ حافظے کی علامت بھی بنائی جاتی ہے۔دس بِٹ پتہ سے \عددی{2^{10}=1024_{10}} یعنی تقریباً ایک ہزار مقامات تک رسائی ممکن ہے۔کمپیوٹر کی دنیا میں کلو (ہزار) سے مراد \عددی{1024_{10}}لیا جاتا ہے۔یوں دو کلو سے مراد \عددی{2048_{10}} ہو گا۔


شکل \حوالہء{9.6 } میں \اصطلاح{ مستحکم کار } کے استعمال پر غور کریں۔\موٹا{ مجاز } اور \عددی{\overline{\text{\RL{لکھ}}}/\text{\RL{پڑھ}}} دونوں بلند ہونے کی صورت میں حافظہ میں ذخیرہ مواد \عددی{D} پر خارج ہو گا جبکہ مجاز بلند اور \عددی{\overline{\text{\RL{لکھ}}}/\text{\RL{پڑھ}}} پست ہو نے کی صورت میں \عددی{D} پر مہیا مواد حافظہ میں لکھا جائے گا۔یوں \عددی{D} بطور مداخل و مخارج کام کرتا ہے۔

جدید عارضی حافظوں میں کثیر تعداد کے الفاظ ذخیرہ کرنے کی گنجائش ہوتی ہے۔شکل\حوالہء{ 9.7 }-ا میں چار لفظ حافظے کے \اصطلاح{ مخلوط دور }\فرہنگ{مخلوط دور}\حاشیہب{integrated circuit, IC}\فرہنگ{IC, integrated circuit} کی علامت دکھائی گئی ہے جہاں لفظ کے چار داخلی و خارجی بِٹوں کو \عددی{D} کی بجائے \عددی{I/O} کہا گیا ہے۔

شکل-ب میں مجاز کی جگہ \عددی{\overline{\text{\RL{مجاز}}}} استعمال کیا گیا ہے ، جو شکل -ا کے مجاز مداخل پر نفی گیٹ نصب کرنے سے حاصل ہو گا؛ مزید \عددی{\overline{\text{\RL{لکھ}}}/\text{\RL{پڑھ}}} کو مختصراً \عددی{\overline{\text{\RL{لکھ}}}} پکار کر پنیا پر گول دائرہ ڈال کر اس کا پست فعال پن ظاہر کیا گیا ہے۔یوں \عددی{\overline{\text{\RL{لکھ}}}} پست ہو نے کی صورت میں \موٹا{ حافظے میں }مواد لکھا اور بلند صورت میں \موٹا{حافظے سے } مواد پڑھا جاتا ہے۔

شکل - ج میں بارہ بِٹ پتہ ، ایک بائٹ لفظ عارضی حافظے کی علامت دکھائی گئی ہے۔بارہ بِٹ پتہ \عددی{2^{12}=4096_{10}} بائٹ تک رسائی ممکن بناتا ہے لہٰذا یہ چار کلو بائٹ عارضی حافظہ کی علامت ہے۔اس مخلوط دور میں \عددی{\overline{\text{\RL{بیدار}}}} مداخل کا اضافہ کیا گیا ہے جو \موٹا{پست فعال }ہے۔اس پر اب بات کرتے ہیں۔


 مخلوط دور میں متعدد گیٹ پائے جاتے ہیں اور جدید برقیاتی آلات کئی مخلوط ادوار پر مشتمل ہوتے ہیں۔یہ سب برقی طاقت سے چلتے ہیں۔ہم کہتے ہیں برقی طاقت انہیں \اصطلاح{ بیدار } رکھتی ہے۔برقیاتی آلات عموماً بیٹری سے برقی طاقت حاصل کرتے ہیں۔ درکار برقی طاقت کم کرنے سے بیٹری زیادہ دیر کارآمد رہتی ہے۔
 
برقیاتی آلات میں مختلف مخلوط ادوار کی ضرورت مختلف لمحات پر ہو گی۔ان لمحات کے علاوہ انہیں بیدار رکھنے سے بلا ضرورت برقی توانائی ضائع ہو گی۔غیر مستعمل مخلوط ادوار کی برقی طاقت منقطع نہیں کی جا سکتی ہے۔عارضی حافظے کی مثال لیتے ہوئے ہم جانتے ہیں کہ برقی طاقت نہ ملنے پر ان میں مواد محفوظ نہیں رہتا ، البتہ یہ ممکن ہے کہ عارضی حافظے کو صرف اتنی برقی طاقت مہیا کی جائے کہ یہ صرف مواد محفوظ رکھنے کے قابل ہو ، یعنی اسے نڈھال سی کیفیت میں ڈالا جا سکتا ہے۔عارضی حافظے کے مخلوط دور میں \عددی{\overline{\text{\RL{بیدار}}}} مداخل اس مقصد کے لئے مہیا کیا گیا ہے۔جس لمحے پر مخلوط دور کی ضرورت ہو، \عددی{\overline{\text{\RL{بیدار}}}} پست (فعال) کر کے اسے جگایا جاتا ہے اور استعمال کے بعد فوراً دوبارہ نڈھال کر دیا جاتا ہے۔نڈھال صورت میں مخلوط دور بیرونی دنیا سے، دو طرفہ مستحکم کار کی مدد سے، مکمل طور پر منقطع رہتا ہے اور اس میں نہ کچھ لکھا جا سکتا ہے اور نہ ہی اس سے کچھ پڑھا جا سکتا ہے۔نڈھال حال میں حافظہ کمتر برقی توانائی صرف کرتا ہے۔عام طور شناخت کار کی مدد سے بیدار کیے جانے والے مخلوط دور کی شناخت کی جاتی ہے۔

چار لفظ حافظے کی تصوراتی تصویر شکل \حوالہء{9.8 } میں دکھائی گئی ہے جہاں دو بِٹ پتہ اور چار بِٹ مواد ثنائی روپ میں لکھے گئے ہیں۔شکل میں ایک کلو بائٹ حافظے کی تصوراتی تصویر بھی پیش ہے جہاں مواد کو ثنائی جبکہ پتہ کو اعشاری روپ میں لکھا گیا ہے۔

\ابتدا{مشق}
عارضی حافظہ \عددی{6116 }کے معلوماتی صفحات سے اس کی استعداد " کلو بائٹ " میں معلوم کریں۔
\انتہا{مشق}



\حصہ{ پختہ حافظہ}
پختہ حافظے سے مراد ا وہ حافظہ ہے جس میں مواد برقی طاقت کی عدم موجودگی میں بھی محفوظ رہتا ہو۔پختہ حافظہ کا بنیادی استعمال وہاں ہو گا جہاں مواد تبدیل نہ ہو۔

عارضی حافظے کی طرح پختہ حافظہ بھی مختلف لمبائی کے الفاظ پر مشتمل ہو گا۔لفظوں تک رسائی پتہ کے ذریعہ ہو گی؛ \عددی{n} بِٹ پتہ کے پختہ حافظہ میں \عددی{2^n} لفظ ہوں گے۔

بائٹ لمبائی چار لفظ پختہ حافظے کی اندرونی ساخت شکل \حوالہء{9.9 } میں دکھائی گئی ہے جس کی بہتر صورت شکل \حوالہء{ 9.10 } پیش کرتی ہے ، جہاں چار داخلی جمع گیٹ کی صاف شکل استعمال کی گئی ہے۔دو سے چار شناخت کار، پتہ کے دو بِٹ سے چار مقامات تک رسائی ممکن بناتا ہے۔یوں چار الفاظ تک رسائی ممکن ہو گی۔



شکل \حوالہء{9.9 } میں بالکل نیا غیر استعمال شدہ پختہ حافظہ دکھایا گیا ہے۔پتہ \عددی{00_2} کی صورت میں دو سے چار شناخت کار \عددی{y_0} بلند کر کے لفظ \عددی{0} چنے گا۔ تمام جمع گیٹ بلند ہوں گے اور \عددی{D} پر \عددی{11111111_2} خارج ہو گا۔ پتہ \عددی{01_2} لفظ \عددی{1} چنے گا اور \عددی{D} پر \عددی{11111111_2} خارج ہو گا۔آپ تسلی کر لیں کہ چاروں پتہ پر یہی مواد ملتا ہے۔کسی بھی نئے غیر استعمال شدہ پختہ حافظے کے ہر لفظ کے تمام بِٹ بلند \عددی{(1)} ہوں گے۔


آپ نے دیکھا کہ بلند \عددی{y_0} کی صورت میں تمام جمع گیٹ کو یہی بلند اشارہ ملتا ہے اور یوں تمام جمع گیٹ کے مخارج بلند ہو ں گے۔جمع گیٹ سے \عددی{y_0} کا جوڑ منقطع کرنے سے \عددی{y_0} جمع گیٹ تک نہیں پہنچے گا۔شکل \حوالہء{ 9.11 } میں دائیں چار جمع گیٹ \عددی{y_0} سے منقطع ہیں لہٰذا \عددی{y_0} بلند کر کے لفظ \عددی{0} پڑھنے سے \عددی{D} پر \عددی{11110000_2} ملتا ہے۔یہاں ایک بات ذہن نشین کریں :ایسے اشکال میں جمع گیٹ کا منقطع مداخل جمع گیٹ کے مخارج پر اثر انداز نہیں ہو گا۔

امید کی جاتی ہے آپ پختہ حافظہ میں لکھائی کا عمل بخوبی سمجھ گئے ہوں گے۔پختہ حافظے میں جوڑوں کو توڑ کر مواد لکھا جاتا ہے۔اس قسم حافظہ میں ہر جوڑ دراصل ایک\اصطلاح{ برقی فتیلہ }\فرہنگ{فتیلہ}\حاشیہب{electric fuse}\فرہنگ{fuse} (فیوز ) ہو تا ہے ۔ فتیلے کی استعداد سے زیادہ برقی رو فتیلے سے گزار کر اسے پگھلا کر جوڑ منقطع کیا جاتا ہے۔


حافظہ میں لکھا مواد شکل\حوالہء{ 9.8 } کی طرح جدول میں لکھا جاتا ہے۔اس جدول میں باری باری ایک لفظ کو دیکھتے ہوئے جس بِٹ کے مقام پر \عددی{0} ہو، حافظہ کے اندر اس لفظ کے اس بِٹ کا جوڑ تباہ کیا جاتا ہے۔

% i donot see the round circles mentioned in this para and the data is wrong, y0 holds 10110010

شکل \حوالہء{ 9.11 } میں جمع گیٹوں کے مداخل اور دو سے چار شناخت کار کے مخارج کے بیچ جوڑ گول دائروں سے ظاہر کیے گئے ہیں۔حافظہ میں لکھا گیا مواد بھی شکل \حوالہء{ 9.12 } میں پیش کیا گیا ہے۔ان اشکال میں غیر تباہ شدہ جوڑ صلیبی نشان \عددی{(\times)} سے ظاہر کیے جاتے ہیں۔اس شکل کو بخوبی سمجھنا ضروری ہے۔

اب تک چار لفظ حافظہ پر بات کی گئی جس کی وجہ سے \عددی{4} داخلی جمع گیٹ استعمال کیے گئے۔ایک لفظ \عددی{8} بِٹ ہونے کی وجہ سے کل \عددی{8} جمع گیٹ استعمال کیے گئے۔یوں ان حافظوں میں کل \عددی{8\times 4} یعنی بتیس \عددی{(32)} جوڑ یا فتیلے ہوں گے۔ آپ دیکھ سکتے ہیں کہ \عددی{n} بِٹ پتے کے حافظے میں \عددی{2^n} لفظ ہوں گے لہٰذا ایسے حافظے میں \عددی{2^n} داخلی جمع گیٹ ہوں گے۔اگر حافظے کا ایک لفظ \عددی{m} بِٹ ہو تب جمع گیٹوں کی تعداد \عددی{m} ہو گی۔یوں حافظے میں جوڑوں کی تعداد \عددی{m\times 2^n} ہو گی۔

\اصطلاح{شعاع مٹتا پختہ حافظہ } میں بار بار لکھائی ممکن ہے۔ان میں جوڑ، برقی فتیلہ سے نہیں بنائے جاتے بلکہ ان جوڑ کو ایک\اصطلاح{ سوئچ }\فرہنگ{سوئچ}\حاشیہب{switch}\فرہنگ{switch} تصور کریں جنہیں مخصوص طریقے سے برقی طاقت کے ذریعہ منقطع کیا جا تا ہے۔منقطع جوڑوں کو دوبارہ جوڑنے کی خاطر حافظے کو شعاع میں کچھ دیر رکھا جاتا ہے۔

جدید\اصطلاح{ برق مٹتا پختہ حافظوں } میں بار بار لکھائی ممکن ہے۔ان حافظوں میں لکھائی برقی دباو سے کی جاتی ہے اور اسے صاف بھی برقی دباو سے کیا جاتا ہے۔

پختہ حافظہ میں لکھائی \اصطلاح{ مخلوط ادوار برنامہ نویس }\فرہنگ{مخلوط دور!برنامہ نویس}\حاشیہب{IC programmer}\فرہنگ{IC!programmer} کی مدد سے کی جاتی ہے۔
	

\حصہ{حافظہ کی استعداد بڑھانے کی ترکیب}
عارضی حافظوں ( کے مخلوط ادوار ) کے قابو مداخل عموماً \عددی{\overline{\text{\RL{بیدار}}}}، \عددی{\overline{\text{\RL{مجاز}}}}اور \عددی{\overline{\text{\RL{لکھ}}}/\text{\RL{پڑھ}}} جبکہ پختہ حافظوں کے \عددی{\overline{\text{\RL{بیدار}}}} اور \عددی{\overline{\text{\RL{مجاز}}}} ہوں گے۔اس حصے میں ہم تصور کرتے ہیں کہ حافظوں کے قابو اشارات صرف \عددی{\overline{\text{\RL{بیدار}}}} اور \عددی{\overline{\text{\RL{لکھ}}}/\text{\RL{پڑھ}}} ہیں جنہیں استعمال کرتے ہوئے ایک سے زیادہ حافظے آپس میں جوڑنا دکھایا جائے گا۔حقیقت میں عموماً \عددی{\overline{\text{\RL{بیدار}}}} کے علاوہ تمام حافظوں کے ایک جیسے قابو مداخل ایک ساتھ جوڑے جاتے ہیں۔یوں تمام حافظوں کے \عددی{\overline{\text{\RL{مجاز}}}} مداخل اکٹھے جوڑے جائیں گے اور اسی طرح تمام کے \عددی{\overline{\text{\RL{لکھ}}}/\text{\RL{پڑھ}}} ایک ساتھ جوڑے جائیں گے۔

\جزوحصہ{دو عدد \عددی{4\times 4} حافظے سلسلہ وار جوڑ کر ایک عدد \عددی{8\times 4} حافظہ کا حصول}
کبھی کبھار درکار استعداد کا حافظہ میسر نہیں ہو گا۔ایسی صورت میں ایک سے زیادہ حافظے اکٹھے جوڑ کر درکار بائٹ ذخیرہ کرنا ممکن بنایا جاتا ہے۔شکل\حوالہء{ 9.13 }-ا میں \عددی{4\times 4} کے دو حافظے جوڑ کر دگنی استعداد کا \عددی{8\times 4} حافظہ حاصل کیا گیا۔چھوٹے حافظوں کو حافظہ-0 اور حافظہ-1 کہا گیا ہے۔ شکل-ا میں ایک جیسے پتہ بِٹ ساتھ ساتھ جوڑے گئے ہیں یعنی حافظہ-0 کا \عددی{A_0} حافظہ-1 کے \عددی{A_0} سے جوڑا گیا ہے، اور حافظہ-0 کا \عددی{A_1} حافظہ-1 کے \عددی{A_1} سے جوڑا گیا ہے۔اسی طرح ایک جیسے مواد بِٹ ساتھ ساتھ جوڑے گئے ہیں یعنی حافظہ-0 کے \عددی{D_0}، \عددی{D_1}، \عددی{D_2} اور \عددی{D_3} بالترتیب حافظہ-1 کے \عددی{D_0}، \عددی{D_1}، \عددی{D_2} اور \عددی{D_3} سے جوڑے گئے ہیں۔البتہ حافظہ-0 کا \عددی{\overline{\text{\RL{بیدار}}}} مداخل (جسے \عددی{\overline{\text{\RL{بیدار0}}}}کہا گیا ہے) سیدھا \عددی{A_2} کے ساتھ ملایا گیا ہے جبکہ حافظہ-1 کا \عددی{\overline{\text{\RL{بیدار}}}} مداخل (جسے \عددی{\overline{\text{\RL{بیدار1}}}} کہا گیا ہے) نفی گیٹ کے ذریعہ \عددی{A_2} سے جوڑا گیا ہے۔



شکل \حوالہء{9.14 }-ا میں تین پتہ بِٹ کی تمام ترتیب دی گئی ہیں۔ (شکل \حوالہء{9.13} کو دیکھتے ہوئے آگے پڑھیں۔) پست \عددی{A_2} سے مراد پست \عددی{\overline{\text{\RL{بیدار0}}}} اور بلند \عددی{\overline{\text{\RL{بیدار1}}}} ہوگا جس سے حافظہ-0 جاگ اٹھتا ہے اور حافظہ-1 نڈھال رہتا ہے۔اسی طرح بلند \عددی{A_2} سے \عددی{\overline{\text{\RL{بیدار0}}}} بلند اور \عددی{\overline{\text{\RL{بیدار1}}}} پست ہو گا جس سے حافظہ-0 نڈھال اور حافظہ-1 جاگ اٹھے گا ۔

یوں پست \عددی{A_2} کی صورت میں پتہ کے باقی دو بِٹ \عددی{A_0} اور \عددی{A_1} حافظہ-0 کے مختلف مقامات تک رسائی ممکن بنائیں گے۔ پتہ \عددی{000_2} حافظہ-0 کے صفرویں مقام اور پتہ \عددی{011_2} حافظہ-0 کے تیسرے مقام تک رسائی دیتا ہے۔



اسی طرح بلند \عددی{A_2} کی صورت میں پتہ کے باقی دو بِٹ \عددی{A_0} اور \عددی{A_1} حافظہ-1کے مختلف مقامات تک رسائی ممکن بنائیں گے۔ پتہ \عددی{000_2} حافظہ-1کے صفرویں اور پتہ \عددی{011_2} حافظہ-1کے تیسرے مقام تک رسائی دیتا ہے۔


گزشتہ دو نثر پاروں کا خلاصہ درج ذیل ہے۔ دو عدد چار لفظ حافظے مل کر ایک عدد آٹھ لفظ حافظے کے طور پر کام کرتے ہیں۔الفاظ کی لمبائی جوں کی توں چار بِٹ رہتی ہے۔اس طرح پتہ \عددی{000_2} کل حافظے کے صفرویں مقام تک رسائی دیتا ہے، پتہ \عددی{011_2} کل حافظے کے تیسرے، پتہ \عددی{100_2} کل حافظہ کے چوتھے اور پتہ \عددی{111_2} ساتویں مقام تک رسائی دیتا ہے ۔ یوں دو عدد حافظے جوڑ کر ایک عدد حافظہ حاصل کیا جا سکتا ہے اور ان کی اندرونی ساخت پر ہر وقت غور کرنے کی ضرورت نہیں۔شکل \حوالہء{ 9.13 }-ب میں اس حقیقت کو مدِ نظر رکھتے ہوئے ان دو حافظوں بمع نفی گیٹ کو بطور ایک \عددی{8\times 4} حافظہ دکھایا گیا ہے جس کے تین پتہ بِٹ اور چار مواد بِٹ ہیں۔اسی طرح شکل \حوالہء{9.14 }-ب میں تین بِٹ پتہ کی نسبت سے دونوں حافظوں کے مقامات دکھائے گئے ہیں۔اس شکل سے واضح ہے کہ دو چھوٹے حافظوں کو پتہ کے لحاظ سے علیحدہ علیحدہ مقامات پر رکھا گیا ہے اور حافظہ۔0 کے آخری لفظ کے اگلے مقام پر حافظہ۔1 کا صفرواں لفظ پایا جاتا ہے۔یوں پتہ کے لحاظ سے ان دو حافظوں کو سلسلہ وار قریب رکھا گیا ہے۔ دو یا دو سے زیادہ حافظے جوڑتے وقت اس طرح کی تصوراتی شکل ذہن میں بنایا کریں۔
	
مذکورہ بالا میں \عددی{4\times 4} استعداد کے حافظے استعمال کیے گئے جنہیں دو پتہ بِٹ \عددی{A_0} اور \عددی{A_1} درکار تھے۔ان دو بِٹ کو استعمال کر کے بیدار حافظے کے مختلف مقامات تک رسائی حاصل کی جاتی ہے جبکہ اگلا پتہ بِٹ \عددی{A_2} استعمال کر کے ان حافظوں کو پتہ کے لحاظ سے مختلف مقامات پر رکھا گیا۔یہی طریقہ کار زیادہ استعداد کے حافظوں کے ساتھ بھی استعمال کیا جا سکتا ہے۔یوں دو عدد دس بِٹ پتہ کے حافظے جوڑتے وقت \عددی{A_0} تا \عددی{A_9} بیدار حافظہ کے مختلف مقامات تک رسائی دیں گے جبکہ \عددی{A_{10}} انہیں جداگانہ بیدار کرے گا۔ 

\جزوحصہ{ تین \عددی{16\times 8} حافظے سلسلہ وار جوڑ کر ایک \عددی{48\times 8} حافظے کا حصول}
شکل \حوالہء{9.15}-ا میں پست مخارج شناخت کار استعمال کر کے تین \عددی{ 16\times 8} حافظے (حافظہ-0، حافظہ-1، حافظہ-2) سلسلہ وار جوڑے گئے ہیں۔تین حافظوں کے ایک جیسے پتہ بِٹ ساتھ ساتھ جوڑے گئے ہیں۔ یوں تینوں کے \عددی{A_0} ایک ساتھ جڑے ہیں، وغیرہ۔ اسی طرح ایک جیسے مواد بِٹ ساتھ ساتھ جوڑے گئے ہیں، لہٰذا تینوں \عددی{D_0} ایک ساتھ جڑے ہیں، وغیرہ۔ تاہم ان کے \عددی{\overline{\text{\RL{بیدار}}}} مداخل علیحدہ علیحدہ رکھے گئے ہیں تا کہ کسی ایک وقت پر صرف ایک حافظے کا \عددی{\overline{\text{\RL{بیدار}}}} فعال ( پست ) کر کے \عددی{A_0} تا \عددی{A_3} کے ذریعہ اس ایک حافظے کے سولہ مقامات تک رسائی حاصل کی جا سکے۔

 شناخت کار کو پتہ بِٹ \عددی{A_4} اور \عددی{A_5} بطور مداخل فراہم کیے گئے جبکہ اس کے مخارج \عددی{\overline{y_0}}، \عددی{\overline{y_1}}، \عددی{\overline{y_2}}، اور \عددی{\overline{y_3}} ہیں، جو مطلوبہ حافظے کی شناخت کرتے ہیں۔\اصطلاح{شناخت کار } کا نام یہیں سے نکلا ہے۔


جیسا آپ جانتے ہیں، شناخت کار کے مداخل کی ہر ترتیب ایک منفرد مخارج چنتی ہے۔شکل-ب میں شناخت کار کا جدول دیا گیا ہے جس میں دائیں جانب ایک اضافی قطار بنائی گئی ہے۔آئیں اس جدول پر غور کرتے ہیں۔پست \عددی{A_4} اور پست \عددی{A_5} کی صورت میں \عددی{\overline{y_0}} پست ہو گا جو حافظہ۔0 کے \عددی{\overline{\text{\RL{بیدار0}}}} کے ساتھ جڑا ہے۔یوں \عددی{A_5A_4=00}حافظہ۔0 کی شناخت کر کے اسے بیدار کرتا ہے۔ \عددی{A_5A_4=00} رکھتے ہوئے باقی چار پتہ بِٹ آزادانہ طور پر بلند یا پست کیے جا سکتے ہیں یعنی \عددی{A_3A_2A_1A_0} کی قیمت \عددی{0000_2} تا \عددی{1111_2} ہو سکتی ہے ، جو حافظہ۔0 کے سولہ مقامات تک رسائی ممکن بناتا ہے۔حافظہ-0 کے تمام مقامات تک رسائی کے لئے یوں پتہ بِٹ \عددی{A_5A_4A_3A_2A_1A_0} کی قیمت \عددی{000000_2} تا \عددی{001111_2} ہو گی۔جدول کی دائیں قطار میں یہ حدود درج ہیں اور شکل-ج میں نچلے سولہ خانے ان مقامات کو ظاہر کرتے ہیں۔حافظہ۔0 کا آخری مقام کل حافظہ کے مقام \عددی{001111_2} پر پایا جاتا ہے۔

بلند \عددی{A_4} اور پست \عددی{A_5} کی صورت میں \عددی{\overline{y_1}} پست ہو گا جو \عددی{\overline{\text{\RL{بیدار1}}}} سے جڑا ہے۔یوں \عددی{A_5A_4=01} حافظہ۔1 کی شناخت کر کے اسے بیدار کرتا ہے۔ \عددی{A_5A_4=01} رکھتے ہوئے باقی چار پتہ بِٹ آزادانہ طور پر بلند یا پست کیے جا سکتے ہیں یعنی \عددی{A_3A_2A_1A_0} کی قیمت \عددی{0000_2} تا \عددی{1111_2} ہو سکتی ہے، جو حافظہ۔1 کے سولہ مقامات تک رسائی دیتا ہے۔حافظہ -1 کے مختلف مقامات تک رسائی کے لئے \عددی{A_5A_4A_3A_2A_1A_0} کی قیمت \عددی{010000_2} تا \عددی{011111_2} ہو گی۔جدول کی دائیں قطار میں یہ حدود درج ہیں۔شکل-ج میں نیچے سے سولہ خانے چھوڑ کر اگلے سولہ خانے ان مقامات کو ظاہر کرتے ہیں۔جیسا پہلے ذکر کیا گیا، حافظہ۔0 کا آخری مقام کل حافظہ کے مقام \عددی{001111_2} پر پایا جاتا ہے جبکہ حافظہ۔1 کا صفرواں مقام اس سے اگلے مقام یعنی \عددی{010000_2} پر پایا جاتا ہے۔شکل-ج سے ظاہر ہے جہاں حافظہ۔0 کا اختتام ہے وہیں سے حافظہ۔1 کی شروعات ہوتی ہے۔

پست \عددی{A_4} اور بلند \عددی{A_5} پست \عددی{\overline{y_2}} دے گا جو کہ کسی بھی حافظے کے ساتھ نہیں جڑا۔ یوں \عددی{A_5A_4=10} کسی بھی حافظے کی شناخت نہیں کرتے لہٰذا باقی چار پتہ بِٹ کی قیمتیں \عددی{0000_2} تا \عددی{1111_2} کرنے سے کسی بھی حافظے کی کسی بھی مقام تک رسائی نہیں ہو گی۔ یوں پتہ \عددی{100000_2} تا \عددی{101111_2} حافظے کے کسی بھی مقام تک رسائی نہیں دیں گے لہٰذا اس خطے میں نہ مواد لکھا جا سکتا ہے اور نہ ہی اس خطے سے مواد پڑھا جا سکتا ہے۔ جدول کی دائیں قطار میں یہ حدود درج ہیں۔ شکل-ج میں ان مقامات کو \موٹا{ خالی مقامات }ظاہر کیا گیا ہے۔

بلند \عددی{A_4} اور بلند \عددی{A_5} پست \عددی{\overline{y_3}} دے کر حافظہ۔3 کو بیدار کرتا ہے۔ \عددی{A_5A_4=11} رکھتے ہوئے باقی چار پتہ بِٹ کی قیمتیں \عددی{0000_2} تا \عددی{1111_2} کرنے حافظہ-3 کے سولہ مقامات تک رسائی ہو گی۔یوں \عددی{A_5A_4A_3A_2A_1A_0} کی قیمت \عددی{110000_2} تا \عددی{111111_2} کرنے سے حافظہ۔3 کے سولہ مقامات تک رسائی ہو گی۔ جدول کی دائیں قطار میں یہ حدود درج ہیں۔شکل-ج میں بالائی سولہ خانے ان مقامات کو ظاہر کرتے ہیں۔۔شکل-ج میں ظاہر ہے کہ جہاں خالی مقامات کا اختتام ہوتا ہے وہیں سے حافظہ۔3 شروع ہوتا ہے۔

یہاں کل چھ پتہ بِٹ \عددی{A_0} تا \عددی{A_5} استعمال کیے گئے جو چونسٹھ \عددی{(2^6=64)} مقامات تک رسائی دے سکتے ہیں۔ہم نے سولہ سولہ لفظ کے تین حافظے استعمال کرتے ہوئے اڑتالیس \عددی{ (16\times 3=48)} مقامات استعمال کیے جبکہ سولہ \عددی{(64-48=16)} مقامات (\موٹا{خالی مقامات}) کا استعمال نہیں کیا گیا۔اگرچہ ان تین حافظوں کو سلسلہ وار جوڑا گیا ہے ، تاہم ان میں صرف حافظہ۔0 اور حافظہ۔1 قریب قریب ہیں جبکہ حافظہ۔3 دور رکھا گیا ہے۔ہم سولہ لفظ کا مزید ایک حافظہ شناخت کار کے ساتھ جوڑ کر تمام چونسٹھ مقامات بروئے کار لا سکتے ہیں۔



\جزوحصہ{دو \عددی{4\times 4} حافظے متوازی جوڑ کر \عددی{4\times 8} حافظے کا حصول}
شکل \حوالہء{ 9.16 }-ا میں دو \عددی{4\times 4} حافظے متوازی جوڑ کر ایک \عددی{4\times 8} حافظہ حاصل کیا گیا ہے۔دونوں حافظے بیک وقت بیدار ہوتے ہیں اور پتہ کے دو بِٹ \عددی{A_0} اور \عددی{A_1} دونوں حافظوں کے چار مقام تک رسائی دیتے ہیں۔حافظہ۔0 کے مواد کو \عددی{D_0} تا \عددی{D_3} جبکہ حافظہ۔1 کے مواد کو \عددی{D_4} تا \عددی{D_7} تصور کر کے 
ان (\عددی{D_0} تا \عددی{D_7})آٹھ بِٹوں کو ایک بائٹ تصور کیا جا سکتا ہے۔اس طرح آپس میں متوازی جڑے دو حافظوں کو \عددی{4\times 8} استعداد کا ایک حافظہ تصور کیا جا سکتا ہے جسے شکل-ب میں تصوراتی شکل دی گئی ہے۔ 

\حصہ{حافظہ کے اوقات کار}
حافظہ عموماً \اصطلاح{خرد عامل کار}\فرہنگ{خرد عامل کار}\حاشیہب{microprocessor}\فرہنگ{microprocessor} ( \اصطلاح{مائکروپراسیسر }\فرہنگ{مائکروپراسیسر}) کے ساتھ منسلکہ استعمال کیا جاتا ہے۔عام طور پر مخلوط ادوار کوئی مخصوص کام سرانجام دینے کے لئے تخلیق کیے جاتے ہیں۔ خرد عامل کار ان سے مختلف نوعیت کا مخلوط دور ہے جو \اصطلاح{ احکامات }\فرہنگ{احکامات}\حاشیہب{commands}\فرہنگ{commands} پر چلتا ہے۔ان احکامات کو تبدیل کر کے مائکروپراسیسر سے مختلف کام لیے جا سکتے ہیں۔ یہ احکامات (پہلے سے) پختہ حافظے میں لکھے جاتے ہے جہاں سے مائکروپراسیسر انہیں پڑھ کر ان کی تعمیل کرتا ہے۔مائکروپراسیسر کے ساتھ عموماً عارضی حافظہ منسلک کیا جاتا ہے جہاں یہ عارضی مواد لکھ کر ذخیرہ کر سکتا ہے اور جہاں سے یہ مواد پڑھ سکتا ہے۔ مختلف صنعت کاروں کے تخلیق کردہ خرد عامل کار کے اپنے اپنے مخصوص احکامات ہوں گے جنہیں یہ سمجھ سکتا ہے اور جن پر یہ عمل کر سکتا ہے۔کسی بھی مائکروپراسیسر کے تمام احکامات کو اس مائکروپراسیسر کی \اصطلاح{ مادری زبان}\فرہنگ{مادری زبان}\حاشیہب{assembly language}\فرہنگ{assembly language} کہتے ہیں جبکہ کسی ایک حکم کو \اصطلاح{ ہدایت }\فرہنگ{ہدایت}\حاشیہب{instruction}\فرہنگ{instruction}کہتے ہیں۔


خرد عامل کار بیرونی جڑے مخلوط ادوار کے ساتھ گفتگو بذریعہ پتہ ، مواد اور قابو اشارات کرتا ہے۔شکل \حوالہء{ 9.17 }-ا میں خرد عامل کار بیرونی جڑے عارضی حافظہ سے گفتگو کر رہا ہے۔اس گفتگو کا مقصد حافظہ میں مواد لکھنا ہے۔گفتگو کا آغاز اس وقت ہوتا ہے جب خرد عامل کار درکار عارضی حافظے کا پتہ خارج کرتا ہے۔اس پتے کی چند ہندسے عارضی حافظہ کی اور باقی حافظہ میں لکھنے کے مقام کی نشاندہی کرتے ہیں۔شناخت کار چند ہی لمحوں میں پتے (کی چند ثنائی ہندسوں) سے درکار عارضی حافظے کے مخلوط دور کی شناخت کر کے اسے بیدار کرتا ہے۔اس عمل کو حافظے کا قابو مداخل "پست " کرنا ظاہر کرتا ہے۔خرد عامل کار خارجی قابو اشارہ \عددی{\overline{\text{\RL{لکھ}}}/\text{\RL{پڑھ}}} پست کر کے حافظہ کو خبر دار کرتا ہے کہ خرد عامل کار حافظہ میں مواد لکھنا چاہتا ہے اور ساتھ ہی اس مواد کو خارج کرتا ہے۔ اس مواد کو \موٹا{ درست مواد } لکھ کر ظاہر کیا گیا ہے۔حافظہ اس مواد کو \عددی{\overline{\text{\RL{لکھ}}}/\text{\RL{پڑھ}}} اشارے کے کنارہ چڑھائی پر مطلوبہ مقام پر (جس کی نشاندہی باقی پتہ بِٹ کرتے ہیں) محفوظ کرتا ہے۔خرد عامل کار کسی بھی ایسے عمل کے دوران پتہ برقرار رکھتا ہے۔ پتے کی تبدیلی کو دو لکیروں کی آپس میں جگہ بدلنے سے ظاہر کیا گیا ہے۔

شکل \حوالہء{9.17}-ب میں خرد عامل کار حافظہ سے مواد پڑھنا چاہتا ہے۔اس گفتگو میں خرد عامل کار \عددی{\overline{\text{\RL{لکھ}}}/\text{\RL{پڑھ}}} بلند رکھ کر پتہ خارج کرتا ہے۔ اس پتے کی چند ہندسے عارضی حافظہ کی اور باقی حافظہ سے مواد پڑھنے کے مقام کی نشاندہی کرتے ہیں۔شناخت کار چند ہی لمحوں میں (پتے کے چند ہندسوں سے) حافظے کی نشاندہی کر کے اسے کو خبردار کرتا ہے کہ خرد عامل کار حافظے سے مواد پڑھنا چاہتا ہے۔حافظہ بیدار ہوتے ہی اس کوشش میں لگ جاتا ہے کہ درکار مقام سے مواد حاصل کر کے خرد عامل کار کے حوالے کرے۔ایسا کرنے کے لئے حافظہ کو کچھ وقت درکار ہوتا ہے جسے حافظہ کا \اصطلاح{دورانیہ رسائی }\فرہنگ{حافظہ!دورانیہ رسائی}\حاشیہب{access time}\فرہنگ{memory!access time} کہتے ہیں۔حافظہ مطلوبہ مقام سے مواد حاصل کر کے خارج کرتا ہے۔اس مواد کو "درست مواد " کہا گیا ہے۔خرد عامل کار اس مواد کو پڑھ کر اگلا \اصطلاح{ ہدایت } پختہ حافظہ سے بڑھ کر اگلے حکم کی تعمیل کرتا ہے۔

\ابتدا{مشق}
انٹرنیٹ سے \عددی{6116} اور \عددی{2732} حافظوں کے دورانیہ رسائی حاصل کریں۔
\انتہا{مشق} 

\حصہ{پختہ حافظہ سے ترکیبی ادوار کا حصول}
اس کتاب کے حصہ \حوالہ{حصہ_ترکیبی_منطق_شناخت_کار_سے_تفاعل_حصول} میں شناخت کار کے ساتھ ایک جمع گیٹ استعمال کر کے تفاعل کا حصول دکھایا گیا ۔ \عددی{n} بِٹ پتہ والے شناخت کار کے \عددی{2^n} مخارج دراصل پتہ بِٹوں کے تمام ممکنہ \اصطلاح{ مجموعہ ارکان ضرب} ہوتے ہیں۔ ہر تفاعل کو مجموعہ ارکان ضرب کے روپ میں لکھ کر اسے شناخت کار کے مطلوبہ مخارج اور ایک جمع گیٹ سے حاصل کیا جا سکتا ہے۔ 

\عددی{m} بِٹ لفظ پختہ حافظہ میں شناخت کار اور \عددی{m} جمع گیٹ موجود ہوتے ہیں لہٰذا اس کو \عددی{m} تفاعل کے حصول کے لئے \اصطلاح{ تشکیل }\فرہنگ{تشکیل}\حاشیہب{configure}\فرہنگ{configure} دیا جا سکتا ہے۔ یوں شکل \حوالہء{9.12 } کو درج ذیل آٹھ تفاعل (اگرچہ \عددی{D_6} تفاعل \عددی{D_0} دہراتا ہے) حاصل کرنے والا دور تصور کیا جا سکتا ہے۔
\begin{gather}
\begin{aligned}
D_7&=\sum (0,3)\\
D_6&=\sum(1,2)\\
D_5&=\sum (1,2,3)\\
D_4&=\sum(3)\\
D_3&=\sum(0,1)\\
D_2&=\sum (0,2)\\
D_1&=\sum(3)\\
D_0&=\sum(1,2)
\end{aligned}
\end{gather}

ان تفاعل کو ایک مختلف نقطہ نظر سے دیکھتے ہیں۔کمتر دو بِٹ \عددی{D_0} اور \عددی{D_1} کو ایک ساتھ \عددی{D_1D_0} دیکھیں تو یہ مداخل \عددی{A_0} اور \عددی{A_1} جمع کرنے والا نصف جمع کار ہے۔اسی طرح \عددی{D_2} دراصل \عددی{\overline{A}_0} اور \عددی{D_3} دراصل \عددی{\overline{A}_1} ہے۔اسی طرح \عددی{D_4} دراصل دونوں مداخل کا \موٹا{منطقی ضرب} جبکہ \عددی{D_5} ان کا \موٹا{ منطقی جمع}، \عددی{D_6} \موٹا{ بلا شرکت جمع } اور \عددی{D_7} ان کا \موٹا{متمم بلا شرکت جمع}ہے۔

\باب{قابل تشکیل ترکیبی منطقی ادوار}
پختہ حافظہ استعمال کرتے ہوئے تفاعل کا حصول گزشتہ باب میں دکھایا گیا۔ \عددی{m} بِٹ پتہ  پختہ حافظہ میں تمام ممکنہ   \عددی{2^m} ارکان ضرب موجود ہوتے ہیں جنہیں جمع گیٹوں سے جوڑ کر درکار تفاعل حاصل کیے  جا سکتے ہیں۔ پختہ حافظہ\اصطلاح{   قابل تشکیل ترکیبی منطقی ادوار }\فرہنگ{قابل تشکیل منطقی  دور}\حاشیہب{programmable logic devices (PLDs)}\فرہنگ{PLD, programmable logic device}، جن پر یہاں غور کیا جائے گا،  کی  ایک  قسم ہے۔

قابل تشکیل ترکیبی منطقی ادوار کی پہلی قسم  \اصطلاح{ قابل تشکیل جمع  ترکیبی منطقی ادوار }\فرہنگ{ترکیبی منطقی ادوار!قابل تشکیل جمع} \حاشیہب{programmable array logic (PAL)}\فرہنگ{PAL, programmable array logic} ہے  ، جن میں پہلا  صف ضرب گیٹ اور دوسرا   جمع گیٹ  کا ہوتا ہے اور  جو مجموعہ ارکان ضرب کی صورت  میں تفاعل دیتے ہیں۔  ضرب گیٹوں  کی صف میں داخلی برقی جوڑ اٹل  جبکہ دوسری صف کے جمع گیٹوں کے داخلی برقی جوڑ قابل تشکیل ہوتے ہیں۔پختہ حافظہ  اس قسم میں شمار ہوتا ہے۔

قابل تشکیل ترکیبی منطقی ادوار کی دوسری قسم  \اصطلاح{ قابل تشکیل ضرب ترکیبی منطقی ادوار }\فرہنگ{ترکیبی منطقی ادوار!قابل تشکیل ضرب} \حاشیہب{programmable logic array (PLA)}\فرہنگ{PLA, programmable logic array}   ہے  ، جن میں پہلا  صف ضرب گیٹ اور دوسرا   جمع گیٹ  کا ہوتا ہے اور  جو مجموعہ ارکان ضرب کی صورت  میں تفاعل دیتے ہیں۔  پہلی صف کے ضرب گیٹوں کے داخلی برقی جوڑ  قابل تشکیل جبکہ  دوسری صف کے جمع گیٹوں کے داخلی برقی جوڑ اٹل ہوتے ہیں۔

تیسری اور سب سے زیادہ  لچک دار قابل تشکیل ترکیبی منطقی ادوار کی قسم میں پہلی صف کے ضرب گیٹوں کے داخلی جوڑ اور دوسری صف کے جمع گیٹوں کے داخلی جوڑ  دونوں قابل تشکیل ہوتے ہیں۔انہیں\اصطلاح{  قابل تشکیل ضرب و جمع ترکیبی منطقی ادوار  }\فرہنگ{ترکیبی منطقی ادوار!قابل تشکیل ضرب و جمع}\حاشیہب{CPLD, complex programmable logic devices}\فرہنگ{CPLD} کہتے ہیں۔

مذکورہ  بالا ادوار   پروگرامر (مخلوط دور برنامہ نویس)   سے تشکیل دیے جاتے ہیں۔


\جزوحصہ{قابل تشکیل ضرب ترکیبی منطقی ادوار}
قابل تشکیل ضرب  ترکیبی منطقی ادوار کی عمومی ساخت شکل \حوالہء{ 10.1  } میں دکھائی گئی ہے جہاں دور کے چار مداخل اور تین مخارج ہیں۔ان ادوار میں عموماً کئی مخارج اشارے  بھی  بطور مداخل   استعمال کیے جاتے ہیں  جیسے  یہاں \عددی{F_2}   استعمال کیا گیا  ہے۔

دکھائے گئے  دور کے تین یکساں حصے ہیں۔ہر حصہ میں دس     مداخل  تین ضرب گیٹ ہیں جو  تین مداخل ایک جمع گیٹ کو جاتے ہیں۔ضرب گیٹ کے مداخل قابل تشکیل جبکہ جمع گیٹ کے مداخل اٹل ہیں۔دور کے کُل چار مداخل ہیں جنہیں مستحکم کار سے گزار کر ان  کے  متمم بھی  ضرب گیٹ کو مہیا کیے گئے ہیں۔اس دور میں  \عددی{10} داخلی کُل\عددی{9} جمع گیٹ ہیں لہٰذا  اس میں  \عددی{9\times 10=90} فتیلے ہوں گے۔

عام دستیاب ادوار میں مداخل اور مخارج  کی تعداد اس سے زیادہ ہو گی، مثلاً  ان  میں سولہ مداخل، آٹھ مخارج اور آٹھ یکساں اندرونی حصے ہو سکتے ہیں جن میں ہر حصہ آٹھ ضرب اور ایک جمع گیٹ پر مشتمل ہو گا۔مزید خارجی   اشاروں   پر مستحکم کار نصب ہو سکتے ہیں جنہیں   بلند  رکاوٹی حال کیا جا سکتا ہے۔

آئیں اس دور کو استعمال کرتے ہوئے  درج ذیل تفاعل حاصل کرتے ہیں جو ارکان ضرب کے روپ میں دیے گئے ہیں۔
\begin{gather}
\begin{aligned}
F_0(A,B,C,D)&=\sum(4,5,10,14)\\
F_1(A,B,C,D)&=\sum(0,1,5,7,9,13,14,15)\\
F_2(A,B,C,D)&=\sum(0,1,5,7,14,15)
\end{aligned}
\end{gather}
ان تفاعل کا سادہ  روپ درج ذیل ہے۔
\begin{gather}
\begin{aligned}
F_0&=\overline{A}B\overline{C}+AC\overline{D}\\
F_1&=\overline{A}\,\overline{B}\,\overline{C}+\overline{A}BD+ABC+A\overline{B}C=F_2+A\overline{B}C\\
F_2&=\overline{A}\,\overline{B}\,\overline{C}+\overline{A}BD+ABC
\end{aligned}
\end{gather}

ان مساواتوں میں کوئی بھی ضربی رکن تین سے زیادہ مداخل پر مشتمل نہیں لہٰذا  درج بالا تفاعلات کو  شکل \حوالہء{ 10.1  }  میں پیش  قابل تشکیل ترکیبی منطقی دور استعمال کر کے حاصل کیا جا سکتا ہے۔شکل \حوالہء{10.2 } میں تفاعلات کا دور دکھایا گیا ہے جہاں موجود  جوڑ صلیبی نشان  سے ظاہر کیے  گئے ہیں۔ باقی جوڑ منقطع کیے گئے ہیں۔


\جزوحصہ{قابل تشکیل ضرب و  جمع ترکیبی منطقی ادوار}
ان ادوار میں بھی پہلی صف ضرب گیٹ اور دوسری صف جمع گیٹوں کی ہوتی ہے البتہ ان میں ضرب گیٹوں اور جمع گیٹوں کے تمام جوڑ قابل تشکیل ہوتے ہیں۔یوں  استعمال کے نکتہ نظر سے  یہ نہایت لچک دار ہوتے ہیں۔ 


شکل\حوالہء{ 10.3 } میں قابل تشکیل ضرب و جمع ترکیبی منطقی دور دکھایا گیا ہے۔اس دور میں تمام ضرب گیٹوں کے داخلی جوڑ    اور تمام جمع گیٹوں کے داخلی جوڑ قابل تشکیل ہیں۔اس دور میں آٹھ داخلی چھ  ضرب گیٹ اور چھ  داخلی تین جمع گیٹ ہیں۔یوں اس میں کُل جوڑ \عددی{66} ہوں گے۔

اس شکل میں درج ذیل  تین تفاعل حاصل کیے گئے ہیں جہاں صلیبی  نشان سلامت جوڑ کو   ظاہر کرتے ہیں۔ ان تفاعل کے حصول میں چار ضرب گیٹ اور تینوں جمع گیٹ کی ضرورت پیش آئی، جبکہ دو ضرب گیٹ زیر استعمال نہیں آئے۔
\begin{gather}
\begin{aligned}
D_2&=\overline{A_0}\,\overline{A_1} A_2 \overline{A_3}\\
D_1&=\overline{A_1}A_2 \overline{A_3}\\
D_2&=A_0\overline{A_1}A_3+\overline{A_0}\,\overline{A_3}
\end{aligned}
\end{gather}
 یہاں دکھایا گیا قابل تشکیل ضرب و جمع ترکیبی منطقی دور صرف سمجھانے کی خاطر تھا۔حقیقی ادوار میں کئی گنا زیادہ مداخل،مخارج،  اور گیٹ ہوں گے۔ثنائی تفاعل کی سادہ ترین  صورت حاصل کر کے اسے مخلوط دور میں ڈالا جاتا ہے۔سادہ ترین  روپ کا حصول ، جو عموماً ایک مشکل کام ہو گا   ، کمپیوٹر کے ذریعے  کیا جاتا ہے۔ منقطع ہونے والے فتیلوں کی معلومات  بھی  کمپیوٹر فراہم کرتا ہے۔فتیلے مخلوط ادوار کا پروگرامر منقطع کرتا ہے۔


\حصہ{ قابل تشکیل ترتیبی ادوار}
جیسا اس باب کی  شروع میں ذکر ہوا، \اصطلاح{ وسیع پیمانے کے مخلوط ادوار}\فرہنگ{مخلوط دور!وسیع پیمانہ}\حاشیہب{large scale integration (LSI)}\فرہنگ{LSI, large scale integration}  ترتیبی بناوٹ رکھتے ہیں۔قابل تشکیل ترکیبی ادوار کے ساتھ پلٹ منسلک کر کے قابل تشکیل ترتیبی ادوار حاصل کیے جاتے ہیں۔اس طرح کے  یکساں  کئی حصے ایک  مخلوط دور پر  میں ڈال  کر  \اصطلاح{پیچیدہ قابل تشکیل ترتیبی ادوار  }\فرہنگ{قابل تشکیل!پیچیدہ ترتیبی دور}\حاشیہب{complex PLD (CPLD)}\فرہنگ{CPLD, complex PLD} بنائے جاتے ہیں۔ان ادوار میں تمام  انفرادی حصوں کے مابین،  قابل تشکیل ترکیبی ادوار کی طرح، برقی جوڑوں  (فتیلوں)  کا جال  بچھایا جاتا ہے ، اور بیرونی مداخل کے ساتھ ساتھ  دور کے  مخارج بطور مداخل استعمال کیے جا سکتے ہیں۔

\اصطلاح{انتہائی وسیع پیمانے کے مخلوط ادوار }\فرہنگ{مخلوط دور!انتہائی وسیع پیمانہ}\حاشیہب{very large scale integration (VLSI)}\فرہنگ{VLSI, very large scale integration} کی بناوٹ صف در صف گیٹوں پر مبنی ہوتی ہے۔ایسے جدید مخلوط ادوار میں گیٹوں کی تعداد اربوں میں  ہوتی ہے۔

انتہائی وسیع پیمانے کے مخلوط ادوار کا ذکر کرتے ہوئے \موٹا{  مُور } کی  پیشن گوئی کا ذکر کرنا لازم ہے جنہوں نے \سن{1965 } میں پیشن گوئی کی کہ مخلوط ادوار میں گیٹوں کی تعداد ہر دو سال میں دگنی ہو گی۔یہ پیشن گوئی جسے \اصطلاح{ مُور کا قانون }\فرہنگ{قانون!مور}\حاشیہب{Moore's law}\فرہنگ{Moore's law} کہتے ہیں اب تک درست ثابت ہوتا آ رہا ہے۔

انتہائی وسیع پیمانہ  مخلوط  دور تشکیل دینے کی خاطر  تفاعل میں مستعمل  گیٹ اور ان کے بیچ جوڑ کی معلومات  مخلوط دور تیار کرنے والے صنعت کار کو فراہم  کیا جاتا ہے۔ مخلوط دور بناتے وقت اس معلومات کے تحت گیٹوں کے بیچ  درکار جوڑ بنا دیے جاتے ہیں۔کبھی کبھار صنعت کار صارف کے ضرورت کے مطابق مخلوط دور تیار کرتا ہے۔ایسے تیار کیے جانے والے ادوار کو\اصطلاح{ خصوصی استعمال کے مخلوط ادوار  }\فرہنگ{مخلوط دور!خصوصی استعمال}\حاشیہب{application specific integrated circuit (ASIC)}\فرہنگ{ASIC} کہتے ہیں۔

اس سلسلہ کی آخری قسم \اصطلاح{ موقع  پر قابل  تشکیل گیٹ صف }\فرہنگ{موقع پر قابل تشکیل گیٹ صف}\حاشیہب{field programmable gate array (FPGA)}\فرہنگ{FPGA} ہے جو دراصل انتہائی وسیع پیمانہ مخلوط ادوار کی وہ قسم ہے جسے   صارف خود تشکیل دے سکتا ہے۔انہیں بار بار تشکیل دیا جا سکتا ہے۔ ان  ادوار میں گیٹ، پلٹ، شناخت کار، عارضی حافظہ اور اس قسم کے دیگر ادوار پائے جاتے ہیں۔موقع  پر قابل  تشکیل گیٹ صف استعمال کرنے کی خاطر کمپیوٹر کا بھرپور استعمال کیا جاتا ہے۔ \اصطلاح{کمپیوٹر کی مدد سے تیار }\فرہنگ{کمپیوٹر کی مدد سے تیار}\حاشیہب{computer aided design (CAD)}\فرہنگ{CAD} کرنے کی خاطر کئی  کمپیوٹر پروگرام استعمال کیے جا سکتے ہیں۔ 

\ابتدا{مشق}
انٹرنیٹ سے \عددی{EPM7032} مخلوط دور کے معلوماتی صفحات حاصل کریں۔(ا) اس میں کتنے یکساں حصے ہیں؟ (ب) کیا ہر حصے میں پلٹ بھی پایا جاتا ہے؟
\انتہا{مشق}

\باب{ غیر معاصر ترتیبی ادوار }
وسیع پیمانہ عددی ادوار عموماً معاصر ادوار کے طرز پر بنائے جاتے ہیں۔ان کے اگلے حال مکمل طور پر موجودہ حال سے حاصل ہوتے ہیں۔ حال صرف ساعت کے کنارے پر تبدیل ہوتے ہیں اور باقی اوقات کے لئے انہیں غیر متغیر تصور کیا جا سکتا ہے۔ساعت کے کنارے سے چند لمحات قبل تا چند لمحات بعد تک تمام حال کا پائیدار ہونا یقینی بنایا جاتا ہے۔یوں کنارہ ساعت پر معلوم حال پائے جاتے ہیں جن سے اگلے پر یقین حال حاصل ہوتے ہیں۔ 

اس کے برعکس غیر معاصر ادوار کے حال کسی بھی لمحہ تبدیل ہو سکتے ہیں جس سے حالت دوڑ اور دیگر مسائل کھڑے ہوتے ہیں جن پر اس باب میں غور کیا جائے گا۔

غیر معاصر ادوار کی اپنی ایک اہمیت ہے۔یہ ساعت کے کنارے کا انتظار کیے بغیر اشارہ کو ردعمل کر سکتے ہیں۔عموماً کسی بھی عددی دور میں کچھ حصہ معاصر اور کچھ غیر معاصر ہو گا۔

شکل\حوالہء{ 11.1 } میں نہایت سادہ دور دکھایا گیا ہے جس کو سرسری نظر سے دیکھ کر یوں محسوس ہوتا ہے کہ ضرب گیٹ کا مخارج کبھی بلند نہیں ہو سکتا۔غور کرنے سے ثابت ہوتا ہے کہ مسئلہ اتنا سادہ نہیں۔ جب بھی مداخل \عددی{A} حال تبدیل کرے اس کے چند لمحوں بعد نفی گیٹ کا مخارج حال تبدیل کرے گا۔یہ \اصطلاح{ تاخیر }\فرہنگ{تاخیر}\حاشیہب{delay}\فرہنگ{delay} نفی گیٹ کے \اصطلاح{دورانیہ ردِ عمل } کی بدولت ہے۔شکل میں \عددی{A} اور \عددی{\overline{A}} کے خط کھینچتے ہوئے یہ تاخیر دکھائی گئی ہے۔اگر ضرب گیٹ کا دورانیہ ردِ عمل صفر ہوتا تب ضرب گیٹ کا مخارج ان دو مداخل کے مطابق حال \عددی{Y_0} اختیار کرتا۔حقیقتاً ضرب گیٹ کو بھی ردِ عمل کے لئے چند لمحات درکار ہوں گے لہٰذا ضرب گیٹ کا مخارج \عددی{Y} ہو گا۔

آپ دیکھ سکتے ہیں ضرب گیٹ کا مخارج غیر مطلوبہ طور پر ، نفی گیٹ کے دورانیہ ردِ عمل کے برابر دورانیے کے لئے، بلند ہو گا۔اس طرح کے ، غیر مطلوبہ نہایت کم دورانیہ کے لئے ،حال کی تبدیلی کو\اصطلاح{ برقی لرزش} یا مختصراً \اصطلاح{ لرزش}\فرہنگ{ لرزش}\حاشیہب{glitch}\فرہنگ{glitch} کہتے ہیں۔ برقی لرزش مثبت یا منفی ہو سکتی ہے لہٰذا موجودہ لرزش کو مثبت لرزش کہیں گے۔ لرزش نہایت کم دورانیے کی دھڑکن تصور کی جا سکتی ہے، تاہم لرزش کی اصطلاح عموماً غیر مطلوبہ دھڑکن کے لئے استعمال کی جاتی ہے اور ان سے معاصر ادوار کو پاک رکھا جاتا ہے۔

 لرزش کی وجہ سے ادوار \اصطلاح{عبوری حال }\فرہنگ{عبوری حال } \حاشیہب{transition state}\فرہنگ{state!transition} اختیار کرتے ہیں۔اس باب میں عبوری حال پر تفصیلاً بحث ہو گی۔ 

آپ نے دیکھا کہ ضرب گیٹ تک اشارہ \عددی{\overline{A}} پہنچنے میں تاخیر کی بدولت لرزش پیدا ہوئی۔تاخیر کی مزید ایک مثال دیکھتے ہیں۔

برقی تار میں برقی دباو کی رفتار تقریباً خلاء میں روشنی کی رفتار \حاشیہب{خلاء میں روشنی کی رفتار \عددی{3\times 10^8} میٹر فی سیکنڈ ہے۔}کے برابر ہوتی ہے۔یوں ایک نینو سیکنڈ میں برقی دباو تقریباً \عددی{3\times 10^8\times 10^{-9}=0.3} میٹر یعنی \عددی{30} سنٹی میٹر فاصلہ طے کرتا ہے۔آئیے دیکھتے ہیں اگر پچھلی مثال تبدیل کر کے نفی گیٹ کی جگہ \عددی{30} سینٹی میٹر برقی تار لگائی جائے اور ضرب گیٹ کی جگہ بلا شرکت جمع گیٹ نصب کیا جائے تو دور کا ردِ عمل کیسا ہوگا ( شکل \حوالہء{11.2 }دیکھیں)۔


 اشارہ \عددی{A} گیٹ کے ایک داخلی پن پر مہیا کیا گیا ہے جبکہ یہی اشارہ تیس سنٹی میٹر برقی تار سے گزار کر دوسرے داخلی پن پر مہیا کیا گیا ہے جہاں اشارے کو \عددی{A_t} کہا گیا ہے۔ تار کو بل دار لکیر سے ظاہر کیا گیا ہے۔یوں اشارہ \عددی{A_t} گیٹ کے دوسرے پن تک تاخیر سے پہنچتا ہے۔اشارہ \عددی{A} بلند یا پست ہونے کے ایک نینو سیکنڈ بعد اشارہ \عددی{A_t} بلند یا پست ہو گا۔گیٹ کا دورانیہ ردِ عمل نظر انداز کرتے ہوئے گیٹ کا مخارج \عددی{Y_0} ہوگا۔ گیٹ کا دورانیہ ردِ عمل مدِ نظر رکھتے ہوئے مخارج \عددی{Y} ہوگا۔گیٹ کے خارجی اشارے میں دو بلند برقی لرزشیں دیکھنے کو ملتی ہیں جن کے دورانیے برقی تار میں تاخیر کے برابر ہیں۔یوں اشارے کی راہ میں تاخیر، حافظہ کی طرح، معلومات لمحاتی طور یاد رکھنے کی صلاحیت رکھتی ہیں۔

آپ نے دیکھا مختلف طرز کی تاخیر دور میں لرزشیں پیدا کرتی ہیں۔جہاں\اصطلاح{ باز رسی اشارہ }\فرہنگ{باز رسی اشارہ}\حاشیہب{feedback signal}\فرہنگ{feedback signal} تاخیر سے پہنچ کر مخارج تبدیل کرتا ہو وہاں دوران تاخیر مخارج اور تاخیر کے بعد مخارج مختلف ہوں گے جس سے \اصطلاح{ نا پائیدار حالت }\فرہنگ{حالت!نا پائیدار}\حاشیہب{unstable condition}\فرہنگ{unstable condition} پیدا ہو گی۔

 جب بھی ایک سے زیادہ اشارے بیک وقت تبدیل ہوں، گیٹ اور برقی تاروں میں نا قابل معلوم تاخیر کی بدولت ، ان کے اثرات جاننا تقریباً ناممکن ہو گا۔اس مسئلے سے بچنے کی خاطر غیر معاصر ادوار درج ذیل دو شرائط کے تحت بنائے جاتے ہیں: (ا) ایک وقت پر صرف ایک اشارہ تبدیل ہو؛ (ب) اشاروں کی تبدیلی کے درمیان اتنا وقفہ دیا جائے کہ تاخیر کے باوجود دور پائیدار حال اختیار کرتا ہو۔ان شرائط کے تحت چلنے کو \اصطلاح{بنیادی طریق کار }\فرہنگ{بنیادی طریقہ کار}\حاشیہب{fundamental mode}\فرہنگ{fundamental mode} کے تحت چلنا کہتے ہیں۔

\حصہ{ تجزیہ}
\اصطلاح{غیر معاصر ترتیبی ادوار }\فرہنگ{ترتیبی دور!غیر معاصر}\حاشیہب{asynchronous combinational circuit}\فرہنگ{asynchronous!combinational circuit} سے مراد ایسے ادوار ہیں جن میں (ا) بغیر ساعت والے پلٹ پائے جائیں اور یا (ب) ان میں ایک یا ایک سے زیادہ مخارج بطور \اصطلاح{ باز رسی اشارات } استعمال ہوں۔جیسے اُوپر ذکر کیا گیا،مختلف نوعیت کی تاخیر کی بنا پر باز رسی اشارات لمحاتی طور پر حافظہ کی صلاحیت رکھتے ہیں۔

جب خارجی اشارہ، مثلاً \عددی{D}، بطور داخلی اشارہ استعمال ہو کر اپنی ہی قیمت \عددی{(D)} تعین کرنے میں کردار ادا کر تا ہو، یہ \اصطلاح{باز رسی اشارہ }\فرہنگ{باز رسی اشارہ}\حاشیہب{feedback signal}\فرہنگ{feedback signal} کہلاتا ہے۔

اس حصہ میں بغیر پلٹ ادوار پر غور کیا جائے گا۔پلٹ والے دور پر اگلے حصہ میں غور کیا جائے گا۔

\جزوحصہ{عبوری جدول}
غیر معاصر ترتیبی ادوار پر غور ان کے \اصطلاح{ عبوری جدول }\فرہنگ{عبوری جدول}\حاشیہب{transition table}\فرہنگ{transition table} کی مدد سے کیا جاتا ہے۔یہ طریقہ شکل \حوالہء{11.3} میں دیے گئے دور کی مدد سے سیکھتے ہیں۔

پلٹ کی غیر موجودگی کے باوجود اس کو ترتیبی دور اس لئے کہیں گے کہ خارجی اشارے \عددی{A} اور \عددی{B} بطور باز رسی اشارات، \عددی{a} اور \عددی{b}، استعمال کیے گئے ہیں۔دور سے خارجی حال کی مساوات لکھتے ہیں۔
\begin{gather}
\begin{aligned}
A&=(b+x)\cdot (a+\overline{x})\\
B&=(b+x)\cdot (\overline{a}+\overline{x})
\end{aligned}
\end{gather}
مساوات حاصل کرتے وقت باز رسی اشاروں کو عام مداخل تصور کریں۔یوں \عددی{x} کو بیرونی مداخل جبکہ \عددی{a} اور \عددی{b} کو اندرونی مداخل تصور کریں۔ ان مساوات میں \عددی{a} اور \عددی{b} \موٹا{ موجودہ مخارج } جبکہ \عددی{A} اور \عددی{B} \موٹا{ اگلے مخارج }ہیں۔ ان مساوات سے جدول \حوالہ{جدول_غیر_معاصر_باز رسی} حاصل ہو گا جس سے عبوری جدول کا حصول شکل \حوالہء{11.4 } میں دکھایا گیا ہے۔
\begin{table}
\caption{دور کا بوولین جدول}
\label{جدول_غیر_معاصر_باز رسی}
\centering
\begin{otherlanguage}{english}
\begin{tabular}{CCC|CC}
\toprule
a&b&x&A&B\\
\midrule
0&0&0&0&0\\
0&0&1&0&1\\
0&1&0&1&1\\
0&1&1&0&1\\
1&0&0&1&1\\
1&0&1&1&0\\
1&1&0&0&0\\
1&1&1&1&0\\
\bottomrule
\end{tabular}
\end{otherlanguage}
\end{table}

جدول \حوالہ{جدول_غیر_معاصر_باز رسی} میں پیش \اصطلاح{حال کے متغیرات}\فرہنگ{حال کے متغیرات}\حاشیہب{state variables}\فرہنگ{state variables} \عددی{A} اور \عددی{B} کی معلومات کو علیحدہ علیحدہ کارناف نقشوں کی طرز پر لکھا گیا ہے جس سے عبوری جدول کے حصول میں آسانی پیدا ہوتی ہے۔کارناف نقشوں کی بائیں جانب قطار کی صورت میں اندرونی مداخل \عددی{ab} کی قیمتیں جبکہ اوپر جانب صف کی صورت میں بیرونی مداخل \عددی{x} کی قیمتیں لکھی جاتی ہیں۔

\اصطلاح{عبوری جدول } میں \عددی{A} اور \عددی{B} کی قیمتیں ساتھ ساتھ \عددی{AB} لکھی جاتی ہیں۔کارناف نقشوں کی آخری صفوں کی دائیں قطاروں میں \عددی{A} کی قیمت \عددی{1} جبکہ \عددی{B} کی قیمت \عددی{0} ہے۔عبوری جدول کی نچلی صف اور دائیں قطار کے خانے میں ان قیمتوں کو ساتھ ساتھ \عددی{10} لکھا گیا ہے۔ اس عمل کی وضاحت نقطہ دار لکیروں سے کی گئی ہے۔

عبوری جدول میں صف در صف چلتے ہوئے جب بھی صف میں موجودہ مخارج \عددی{ab} اور اگلے مخارج \عددی{AB} کی قیمت یکساں ہو، \عددی{AB} کی قیمت دائرے میں بند کریں۔دائرہ میں بند حال پائیدار (مستحکم) جبکہ باقی نا پائیدار یعنی \اصطلاح{عبوری } \فرہنگ{عبوری حال }\حاشیہب{transient state}\فرہنگ{transient state} ہوں گے۔

شکل\حوالہء{ 11.5 } پر نظر رکھ کر \اصطلاح{عبوری جدول }کے استعمال پر غور کرتے ہیں۔جدول کی \عددی{ab=00} صف اور \عددی{x=0} قطار میں واقع خانے کو \اصطلاح{ابتدائی خانہ}\حاشیہد{کسی بھی مستحکم حال خانے کو ابتدائی خانہ منتخب کیا جا سکتا ہے۔}کہا گیا ہے، جس میں \عددی{ab=00} اور \عددی{x=0} کی صورت میں \عددی{AB} کی قیمت درج ہے۔ فرض کریں ابتدائی خانہ دور کا ابتدائی حال ظاہر کرتا ہے۔

اب اگر \عددی{ab=00} رکھتے ہوئے بیرونی مداخل \عددی{x} کی قیمت \عددی{0} سے \عددی{1} کر دی جائے تو عبوری جدول کے مطابق \عددی{AB} کی قیمت \عددی{00} سے \عددی{01} ہو جائے گی۔ یوں موجودہ حال \عددی{ab} اور اگلے حال \عددی{AB} کی قیمتیں مختلف ہوں گی جو عبوری حال کی نشانی ہے اور جس میں دور زیادہ دیر نہیں رہ سکتا۔ برقی تاروں میں تاخیر کے بعد \عددی{ab} کی قیمت \عددی{01} ہو جائے گی جبکہ \عددی{x} اپنی نئی قیمت \عددی{(1)} برقرار رکھے گا۔یوں دور تاخیر کے بعد عبوری جدول کی \عددی{x=1} قطار اور \عددی{ab=01} صف پر پائے جانے والے خانے تک پہنچے گا جہاں \عددی{AB} اور \عددی{ab} دونوں کی قیمت \عددی{01} ہے، جو مستحکم حال کو ظاہر کرتا ہے ( اور اسی لئے دائرے میں بند دکھایا گیا ہے)۔ اس پورے مرحلہ کو" پہلا قدم " کہا گیا ہے۔پہلے قدم کو تیر دار لکیر سے ظاہر کیا گیا ہے جو عبوری خانے سے گزر کر مستحکم خانے پر اختتام پذیر ہوتا ہے۔

 مستحکم (پائیدار) حال سے ابتدا کرتے ہوئے \عددی{x} کی قیمت تبدیل کرنے سے دور کچھ لمحوں کے لئے عبوری حال اختیار کر گیا۔یہ صورت زیادہ دیر برقرار نہیں رہی۔تاروں میں تاخیر کے بعد باز رسی اشارے تبدیل ہوئے اور دور دوبارہ مستحکم حال اختیار کر گیا۔عموماً ادوار کا عمل اسی طرح ہو گا۔

اسی طرح \عددی{ab=01} رکھتے ہوئے \عددی{x} کی قیمت \عددی{1} سے \عددی{0} کرنے سے عبوری جدول کے مطابق دور \عددی{x=0} قطار اور \عددی{ab=01} صف کے خانے میں درج حال \عددی{AB=11} اختیار کرے گا۔اس مرتبہ بھی \عددی{AB} اور \عددی{ab} مختلف ہیں (جو عبوری حال کو ظاہر کرتا ہے) لہٰذا دور اس سے نکلنے کی کوشش کرے گا۔برقی تاروں میں تاخیر کے بعد \عددی{AB} کی نئی قیمتوں کی خبر \عددی{ab} کے مقام تک پہنچے گی لہٰذا \عددی{ab} کی قیمت بھی \عددی{11} ہو جائے گی۔یوں دور \عددی{x=0} قطار اور \عددی{ab=11} صف میں درج (دائرے میں بند) مستحکم حال \عددی{AB=11} اختیار کر ے گا۔اسی طرح چلاتے ہوئے \عددی{x} کی قیمت بار بار تبدیل کرنے سے دور بالترتیب \عددی{00} ، \عددی{01}، \عددی{11}، اور \عددی{10} مستحکم حال اختیار کرے گا۔ہر مرتبہ \عددی{10} تک پہنچ کر یہی ترتیب دوبارہ دہرائی جائے گی۔شکل میں تیر دار لکیروں سے یہ مراحل دکھائے گئے ہیں۔

 دور کا حال \عددی{AB} کی بجائے \عددی{ABx} لکھا جاتا ہے۔ یوں \عددی{000}، \عددی{011}، \عددی{110}، اور \عددی{101} \اصطلاح{ مستحکم حال }جبکہ \عددی{001}، \عددی{010}، \عددی{111}، اور \عددی{100} \اصطلاح{ عبوری حال }ہیں۔
 
عبوری جدول کی ہر صف میں ،عموماً ، کم از کم ایک مستحکم حال ضرور پایا جاتا ہے۔ایسا نہ ہونے کی صورت میں اس صف میں پہنچ کر دور عبوری حال اختیار کرے گا۔

عبوری جدول حاصل کرنے کا طریقہ کار یہاں بیان کرتے ہیں۔
\begin{itemize}
\item
دور میں تمام \اصطلاح{باز رسی اشاروں }اور \اصطلاح{باز رسی دائروں }\حاشیہب{feedback loops} کی نشاندہی کریں۔
\item
 کسی بھی ترتیب سے باز رسی دائروں کے مخارج کی شناخت \عددی{A}، \عددی{B}، \عددی{C}، وغیرہ جبکہ اسی ترتیب سے ان کے باز رسی اشارات کی شناخت \عددی{a}، \عددی{b}، \عددی{c}، وغیرہ سے کریں۔
\item
 بیرونی اور اندرونی مداخل کی صورت میں تمام مخارج کے بوولین تفاعل حاصل کریں۔
\item 
 ان تفاعل کے کارناف نقشے بنائیں۔
\item 
 تمام کارناف نقشوں کو ایک عبوری جدول میں یکجا کریں۔عبوری جدول کے خانوں میں \عددی{ABC\cdots} قیمتیں جبکہ جدول کے بائیں جانب ہر صف میں \عددی{abc\cdots} قیمتیں اسی ترتیب سے لکھیں۔
\item 
 جہاں \عددی{ABC\cdots} اور اسی صف میں \عددی{abc\cdots} کی قیمت یکساں ہو، وہاں \عددی{ABC\cdots} کو دائرے میں بند کریں۔
 \end{itemize}
عبوری جدول کے حصول کے بعد بیرونی مداخل تبدیل کر کے دور کے عبوری حال پر غور کیا جا سکتا ہے۔ 

\جزوحصہ{بہاو کا جدول}
شکل\حوالہء{ 11.4 } میں عبوری جدول لکھتے ہوئے خانوں میں بوولین طرز پر حال درج کیے گئے۔دو مخارج کی صورت میں چار حال ( \عددی{00}، \عددی{01}، \عددی{10}، اور \عددی{11} ) ممکن ہیں جنہیں نام بھی دیے جا سکتے ہیں۔مثلاً حال \عددی{00} کو حال \عددی{a} پکارا جا سکتا ہے۔ اسی طرح \عددی{01} کو حال \عددی{b}، \عددی{10} کو حال \عددی{c}، اور \عددی{11} کو حال \عددی{d} نام دیے جا سکتے ہیں۔عبوری جدول میں یہ نام استعمال کر کے، شکل \حوالہء{ 11.6} میں پیش، \اصطلاح{ بہاو کا جدول }\فرہنگ{جدول!بہاو کا}\حاشیہب{flow table}\فرہنگ{table!flow} حاصل ہو گا۔

شکل \حوالہء{ 11.6} میں پیش بہاو کے جدول کے ہر صف میں صرف ایک مستحکم حال پایا جاتا ہے۔پہلی صف میں صرف \عددی{000} اور دوسری صف میں صرف \عددی{011} مستحکم حال پائے جاتے ہیں۔ ایسا جدول جس کی ہر صف میں صرف ایک مستحکم حال پایا جاتا ہو \اصطلاح{ اوّلی بہاو کا جدول }\فرہنگ{بہاو کا جدول!اوّلی}\حاشیہب{primitive flow table}\فرہنگ{flow table!primitive} کہلاتا ہے۔


شکل\حوالہء{ 11.7 } میں ایک ایسا بہاو کا جدول پیش کیا گیا ہے جس کی صفوں میں ایک سے زیادہ مستحکم حال پائے جاتے ہیں۔مثلاً ، پہلی صف میں مستحکم حال \عددی{000}، \عددی{011}، اور \عددی{010} ہیں۔ایسے جدول کو \اصطلاح{ غیر اوّلی بہاو کا جدول }\فرہنگ{بہاو کا جدول!غیر اوّلی}\حاشیہب{non primitive flow table}\فرہنگ{flow table!non primitive} کہتے ہیں۔

 بہاو کے جدول سے دور حاصل کرنے کے لئے پہلے عبوری جدول حاصل کیا جاتا ہے۔بہاو کے جدول کے دو صف ہیں لہٰذا دور کے دو حال ہوں گے۔دو ممکنہ صورتوں کو ایک بِٹ عدد ظاہر کر سکتا ہے۔یوں حال \عددی{a} کو \عددی{0} اور حال \عددی{b} کو \عددی{1} لکھ کر عبوری جدول حاصل کرتے ہیں، جو شکل\حوالہء{ 11.7 } میں دکھایا گیا ہے۔ دور کے اگلے مخارج کو \عددی{Y} اور موجودہ مخارج کو \عددی{y} سے ظاہر کر کے عبوری جدول سے \عددی{Y} کا تفاعل حاصل کرتے ہیں۔
\begin{align}
Y=\overline{x}_1x_0+x_1y
\end{align}

اس تفاعل کا دور شکل\حوالہء{ 11.8 } میں دکھایا گیا ہے۔

 شکل\حوالہء{ 11.7 } میں پیش بہاو کے جدول کے استعمال پر شکل \حوالہء{ 11.9 } کی مدد سے غور کرتے ہیں۔ فرض کریں بیرونی مداخل \عددی{x_1x_0} کی قیمت \عددی{00} ہے،یعنی \عددی{x=00} ، اور دور حال \عددی{a} میں ہے۔اگر \عددی{x_1} تبدیل کیے بغیر \عددی{x_0} کی قیمت \عددی{1} کر دی جائے،یعنی \عددی{x=01} کر دی جائے، تو عبوری جدول کے مطابق دور چند لمحوں کے لئے عبوری حال \عددی{b} اختیار کرنے کے بعد مستحکم حال \عددی{b} اختیار کر ے گا۔اب اگر \عددی{x_0} کی قیمت \عددی{1} رکھتے ہوئے \عددی{x_1} کی قیمت بھی \عددی{1} کر دی جائے،یعنی \عددی{x=11} کر دی جائے، تو حال \عددی{b} برقرار رہے گا۔اس اختتامی خانے کو \موٹا{ پہلا اختتامی خانہ }کہا گیا ہے۔ابتدائی خانے سے \موٹا{ پہلے اختتامی خانے }تک پہنچنے کا عمل تین تیر دار لکیروں سے ظاہر کیا گیا ہے جہاں پہلا تیر مستحکم حال \عددی{a} سے عبوری حال \عددی{b} کا حصول جبکہ دوسرا تیر یہاں سے مستحکم حال \عددی{b} کا حصول ظاہر کرتا ہے۔تیسرا تیر مستحکم حال \عددی{b} سے مستحکم حال \عددی{b} میں ہی رہنے کو ظاہر کرتا ہے۔

اس کے برعکس، ابتدائی خانے سے آغاز کرتے ہوئے \عددی{x_1} برقرار اور \عددی{x_0} تبدیل کرنے کی بجائے ہم \عددی{x_0} کی قیمت \عددی{0}رکھتے ہوئے \عددی{x_1} کی قیمت \عددی{1} کرتے ہیں، یعنی \عددی{x=10} کرتے ہیں۔ بہاو کے جدول کے مطابق حال \عددی{a} برقرار رہے گا۔اب اگر \عددی{x_0} کی قیمت بھی \عددی{1} کر دی جائے، یعنی \عددی{x=11} کر دی جائے، تو اختتامی حال برقرار \عددی{a} رہے گا۔اس اختتامی خانے کو \موٹا{دوسرا اختتامی خانہ } کہا گیا ہے۔

آپ نے دیکھا اختتامی حال بیرونی مداخل کی تبدیلی کی ترتیب پر منحصر ہے۔اس مثال میں ابتدائی بیرونی مداخل \عددی{00} جبکہ اختتامی بیرونی مداخل \عددی{11} ہیں۔ یاد رہے \موٹا{ بنیادی طریق کار } کی شرائط کے تحت، (دور کی درست کارکردگی کے لئے ضروری ہے کہ) ایک سے زیادہ بیرونی مداخل بیک وقت تبدیل نہ کیے جائیں۔ یوں \عددی{00} سے آغاز کر کے ہم سیدھا \عددی{11} نہیں کر سکتے۔ایسا کرنے سے ( ناقابل معلوم تاخیر کی بنا پر) درست اختتامی حال جاننا ناممکن ہو گا۔


\جزوحصہ{ حالت دوڑ}\شناخت{حصہ_غیر_معاصر_حالت_دوڑ}
\اصطلاح{حالت دوڑ }\فرہنگ{حالت دوڑ}\حاشیہب{race condition}\فرہنگ{race condition} کا تذکرہ ایس آر پلٹ پر تبصرے کے دوران کیا گیا۔اس حصے میں اس پر تفصیلاً گفتگو کی جائے گی۔حالت دوڑ اس صورت کو کہتے ہیں جب بیرونی اشارے کی تبدیلی ایک سے زیادہ حال تبدیل کرتا ہو۔نا معلوم تاخیر کی بنا پر حال کی تبدیلی مکمل طور پر جاننا ممکن نہیں ہو گا۔مثلاً ، فرض کریں دو حال دور کا موجودہ مستحکم حال \عددی{00} ہے اور بیرونی مداخل تبدیلی کرنے سے دونوں حال تبدیل ہوتے ہیں ، اور دور آخر کار \عددی{11} مستحکم حال اختیار کر تا ہے۔ پہلی باز رسی راہ کی تاخیر دوسری باز رسی راہ کی تاخیر سے کم ہو نے کی صورت میں دور مستحکم حال \عددی{00} سے عبوری حال \عددی{10} اور آخر کار مستحکم حال \عددی{11} اختیار کرے گا جبکہ دوسری راہ کی تاخیر پہلی راہ کی تاخیر سے کم ہو نے کی صورت میں دور عبوری حال \عددی{01} سے گزر کر مستحکم حال \عددی{11} تک پہنچے گا۔آپ نے دیکھا کہ (نامعلوم تاخیر کی بنا پر) حال تبدیل ہونے کی ترتیب جاننا ممکن نہیں۔

جب عبوری حال کی تبدیلی کی ترتیب اختتامی حال متعین کرنے میں کردار ادا کرتی ہو اور دور دو مختلف اختتامی مستحکم حال اختیار کر نے کی صلاحیت رکھتا ہو وہاں دوڑ کو \اصطلاح{بحرانی دوڑ }\فرہنگ{دوڑ!بحرانی}\حاشیہب{critical race}\فرہنگ{race!critical} کہیں گے۔ سودمند استعمال کے لئے ضروری ہے کہ دور میں بحرانی دوڑ کی صورت پیدا نہ ہوتی ہو۔جہاں عبوری حال کی تبدیلی کی ترتیب اختتامی مستحکم حال پر اثر انداز نہ ہوتی ہو وہاں دوڑ کو\اصطلاح{ غیر بحرانی دوڑ }\فرہنگ{دوڑ!غیر بحرانی}\حاشیہب{non-critical race}\فرہنگ{race!non-critical} کہیں گے۔



شکل\حوالہء{ 11.10 } میں بحرانی دوڑ کی ایک مثال دکھائی گئی ہے جہاں بیرونی مداخل \عددی{x} اور حال \عددی{y_1y_0x} ہے۔ حال کو \اصطلاح{مکمل حال } \عددی{y_1y_0x} لکھتے ہوئے حال \عددی{000} سے آغاز کر کے بیرونی مداخل \عددی{0} سے \عددی{1} کرنے سے دور اختتامی حال کی جانب دوڑ لگائے گا۔نا معلوم تاخیر کی بنا پر ہم نہیں جانتے دور تین ممکنہ حال \عددی{011}، \عددی{111}، اور \عددی{101} میں سے کس حال کو پہلے پہنچے گا۔ یہ تینوں عبوری حال پہلی صف میں دکھائے گئے ہیں۔ عبوری حال \عددی{011} پہلے پہنچنے کی صورت میں دور یہاں سے ہوتے ہوئے اختتامی مستحکم حال \عددی{011} اختیار کر ے گا، جس کو دوسری صف میں دائرے میں بند دکھایا گیا ہے۔اگر دونوں باز رسی راہ میں مائل تاخیر برابر ہوں ، دور عبوری حال \عددی{111} پہلے پہنچے گا اور یہاں سے ہوتے ہوئے اختتامی مستحکم حال \عددی{111} اختیار کر ے گا، جس کو تیسری صف میں دائرہ میں بند دکھایا گیا ہے۔تیسری صورت میں دور عبوری حال \عددی{101} پہلے پہنچتا ہے جہاں سے یہ آخری صف کی جانب رواں ہو گا ، لیکن آخری صف ازخود عبوری حال ہے لہٰذا دور اس عبوری حال سے بھی گزر کر آخر کار تیسری صف کے اختتامی مستحکم حال \عددی{111} پہنچے گا۔اس مثال میں دو اختتامی حال ممکن ہیں۔یہ دریافت کرنا ناممکن ہے کہ دور ان میں سے کس اختتامی حال کو پہنچے گا۔ شکل میں بائیں جانب \عددی{x=0} کی قطار اس لئے خالی رکھی گئی ہے کہ ہم صرف \عددی{x=0} سے \عددی{x=1} کرتے ہوئے دور پر غور کر رہے ہیں جس میں بائیں قطار کے اندراجات درکار نہیں۔

شکل\حوالہء{ 11.11 } میں \موٹا{بحرانی دوڑ } کی دوسری مثال دکھائی گئی ہے جہاں تین اختتامی حال ممکن ہیں۔ \اصطلاح{ مکمل مستحکم حال } \عددی{000} سے آغاز کرتے ہوئے بیرونی مداخل \عددی{x} کی قیمت \عددی{1} کر نے سے دور اختتامی حال کی طرف دوڑ لگائے گا۔بالکل اُوپر مثال کی طرح،تین ممکنہ عبوری حال ہیں۔ایک عبوری حال \عددی{011} ہے جہاں سے یہ دوسری صف میں دکھائے اختتامی مستحکم حال \عددی{011} پہنچے گا۔دوسرا عبوری حال \عددی{111} ہے جہاں سے یہ تیسری صف کے اختتامی مستحکم حال \عددی{111} پہنچے گا اور تیسرا عبوری حال \عددی{101} ہے جہاں سے یہ آخری صف میں دکھائے اختتامی مستحکم حال \عددی{101} پہنچے گا۔ نامعلوم تاخیر کی بنا پر یہ جاننا ممکن نہیں کہ دور حقیقت میں کس اختتامی حال کو پہنچے گا۔

اب غیر بحرانی دوڑ کی ایک مثال دیکھتے ہیں جو شکل \حوالہء{11.12 } میں دکھائی گئی ہے۔اس مثال میں \عددی{000} سے آغاز کرتے ہوئے تین عبوری حال ممکن ہیں۔ ایک عبوری حال \عددی{011} ہے جہاں سے دور دوسری صف کے عبوری حال \عددی{111} اور اس کے بعد تیسری صف کے عبوری حال \عددی{101} سے گزر کر آخر کار چوتھی صف کے اختتامی مستحکم حال \عددی{101} پہنچے گا۔ دوسرا عبوری حال \عددی{111} ہے جہاں سے دور تیسری صف کے عبوری حال \عددی{101} سے ہوتے ہوئے آخر کار آخری صف کے اختتامی مستحکم حال \عددی{101} پہنچے گا۔ تیسرا عبوری حال \عددی{101} ہے جہاں سے گزر کر دور آخری صف کے اختتامی مستحکم حال \عددی{101} پہنچے گا۔

اس مثال میں اگرچہ تین مختلف ممکنات موجود ہیں تاہم اختتامی مستحکم حال سب کا ایک ہے لہٰذا یہ غیر بحرانی دوڑ ہو گی۔ 

 مخصوص اور منفرد عبوری حال سے گزر کر اختتامی مستحکم حال اختیار کرنے کو \اصطلاح{ پھیرا }\فرہنگ{پھیرا}\حاشیہب{cycle}\فرہنگ{cycle} لگانا کہتے ہیں۔اس کی مثال شکل \حوالہء{11.13 } میں دی گئی ہے۔ان اشکال میں حالت دوڑ نہیں پائی جاتی چونکہ ایک وقت میں صرف ایک مخارج حال تبدیل کرتا ہے ، البتہ اختتامی حال تک پہنچنے کی خاطر دور کو مخصوص اور منفرد عبوری حال سے گزرنا ہو گا۔

شکل - الف میں مستحکم حال \عددی{00} سے آغاز کرتے ہوئے عبوری حال \عددی{10} کے بعد عبوری حال \عددی{11} سے گزر کر اختتامی مستحکم حال \عددی{01} پہنچا گیا۔ شکل-ب میں مستحکم حال \عددی{00} سے آغاز کرتے ہوئے عبوری حال \عددی{10} کے راستے اختتامی مستحکم حال \عددی{11} اختیار کیا گیا۔


\جزوحصہ{توازن اور ارتعاش }
 ایسا دور جو \موٹا{پھیرے} لگاتے ہوئے کسی بھی اختتامی مستحکم حال تک نہ پہنچ پائے \اصطلاح{غیر مستحکم دور }\فرہنگ{غیر مستحکم دور}\حاشیہب{unstable circuit}\فرہنگ{unstable circuit} کہلاتا ہے۔شکل \حوالہء{ 11.14 } میں اس کی مثال دکھائی گئی ہے جہاں بیرونی مداخل \عددی{1} کرنے سے دور مستحکم حال تک پہنچے بغیر عبوری حال سے عبوری حال منتقل ہو گا۔	ایسے ادوار بطور\اصطلاح{ مرتعش }\فرہنگ{مرتعش}\حاشیہب{oscillator}\فرہنگ{oscillator} استعمال کیے جاتے ہیں۔ ادوار کو کبھی بھی غیر مستحکم نہیں ہونے دیا جاتا ماسوائے جب انہیں بطور مرتعش استعمال کرنا مقصد ہو۔

\حصہ{حالت دوڑ سے پاک ثنائی علامتوں کا تقرر}
حالت دوڑ کی صورت اس وقت پیدا ہو گی ہے جب ایک سے زیادہ مخارج بیک وقت حال تبدیل کرنے کی کوشش کریں۔بحرانی دوڑ کی صورت میں ادوار قابل استعمال نہیں رہتے۔اس حصے میں بحرانی دوڑ کے خاتمے پر غور کیا جائے گا۔یاد رہے (\موٹا{بنیادی طریقہ کار } پر چلنے کے تحت) ایک وقت پر غیر معاصر دور کا صرف ایک مداخل تبدیل ہو سکتا ہے ، لہٰذا یہ حصہ پڑھتے ہوئے ایک سے زیادہ مداخل کی تبدیلی کی فکر مت کریں۔

جن ادوار میں ایک وقت پر صرف ایک مخارج حال تبدیل کرنے کی کوشش کرتا ہو، وہ حالت دوڑ سے دو چار نہیں ہوتے۔اس حقیقت کو بروئے کار لاتے ہوئے حالت دوڑ ختم کی جاتی ہے۔

عبوری جدول کے حصول کے بعد اس میں درج حال کو ثنائی علامتیں تعین کی جاتی ہیں۔ ان حال کو \اصطلاح{ہمسایہ } ثنائی علامتیں مختص کرنے سے جن کے مابین عبوری جدول میں تبادلہ پایا جاتا ہو بحرانی دوڑ سے پاک دور حاصل ہو گا۔دو ایسے ثنائی اعداد \اصطلاح{ ہمسایہ اعداد }\فرہنگ{ہمسایہ اعداد}\حاشیہب{adjacent numbers}\فرہنگ{adjacent numbers} کہلاتے ہیں جن میں صرف ایک ہندسے کا فرق ہو۔یوں \عددی{1010} اور \عددی{1110} ہمسایہ اعداد ہیں چونکہ ان میں صرف ایک بِٹ مختلف ہے۔اسی طرح \عددی{1110} اور \عددی{0110} آپس میں ہمسایہ ہیں جبکہ \عددی{1010} اور \عددی{0110} آپس میں ہمسایہ نہیں۔

اس ترکیب کو شکل \حوالہء{11.15 }-ا میں دی مثال کی مدد سے دیکھتے ہیں جس میں چار صف ہیں۔یوں دو بِٹ\اصطلاح{ حال کا متغیر } \عددی{f_1f_0} اس کے چار ممکنہ حال بیان کر سکتا ہے۔ہم حال \عددی{a} کے لئے \عددی{f=00}، حال \عددی{b} کے لئے \عددی{f=01}،حال \عددی{c} کے لئے \عددی{f=11}، اور حال \عددی{d} کے لئے \عددی{f=10} حال کے متغیر منتخب کر کے دیکھتے ہیں کیا نتائج رو نما ہوتے ہیں۔

 پہلی صف میں \عددی{x} کی قیمت \عددی{00} سے \عددی{01} کرنے سے حال تبدیل ہو کر \عددی{a}سے \عددی{b} ہو گا، لہٰذا حال کا متغیر \عددی{f} تبدیل ہو کر \عددی{00} سے \عددی{01} ہو گا۔چونکہ حال کے متغیر کا صرف ایک بِٹ تبدیل ہو ا لہٰذا حالت دوڑ پیدا نہیں ہو گی۔اس کے برعکس، پہلی صف میں \عددی{x} کی قیمت \عددی{00} سے \عددی{10} کرنے سے حال تبدیل ہو کر \عددی{a} سے \عددی{c} ہو گا لہٰذا \عددی{f} کی قیمت \عددی{00} سے تبدیل ہو کر \عددی{11} ہو گی۔چونکہ \عددی{f} کے دو ہندسے بیکوقت تبدیل ہونے کی کوشش کرتے ہیں لہٰذا حالت دوڑ پیدا ہو گی۔ یوں دو بِٹ حال کا متغیر تقرر کرنے سے حالت دوڑ پیدا ہو گی۔ایسی صورت میں دو سے زیادہ بِٹ حال کا متغیر استعمال کر کے دیکھا جاتا ہے کہ آیا حالت دوڑ سے چھٹکارا ممکن ہے۔

کبھی کبھار چار صف عبوری جدول میں دو بِٹ حال کا متغیر یوں تقرر کرنا ممکن ہو گا کہ حالت دوڑ پیدا نہ ہو۔

شکل \حوالہء{11.15 } -ب میں حال کے متغیر کی ترتیب بدل کر حالت دوڑ سے بچنے کی (نا کام) کوشش کی گئی ہے۔ یہاں \عددی{a}، \عددی{b}، \عددی{c}، اور \عددی{d} کے لئے بالترتیب \عددی{f=00}، \عددی{f=01}، \عددی{f=10}، اور \عددی{f=11} مختص کیے گئے۔پہلی صف میں \عددی{a}سے \عددی{b} کرنے سے \عددی{f} کی قیمت \عددی{00} سے تبدیل ہو کر \عددی{01} ، جبکہ \عددی{a} سے \عددی{c} کرنے سے \عددی{f} کی قیمت \عددی{00} سے \عددی{10} ہو گی۔ دونوں صورتوں میں \عددی{f} کا صرف ایک بِٹ تبدیل ہو گا، لہٰذا پہلی صف میں حالت دوڑ پیدا نہیں ہو گا۔ البتہ دوسری صف میں \عددی{x} کی قیمت \عددی{01} سے \عددی{11} کرنے سے حال تبدیل ہو کر \عددی{b} سے \عددی{c} ہو گا اور یوں \عددی{f} کی قیمت \عددی{01} سے \عددی{10} ہو گی۔ حال کے متغیر کے دو بِٹ کی تبدیلی سے مراد حالت دوڑ ہے۔

مذکورہ بالا دو مثالوں سے ظاہر ہے کہ موجودہ مسئلے میں دو بِٹ حال کا متغیر مختص کرنے سے حالت دوڑ سے نجات حاصل کرنا ممکن نہیں۔ایسی صورت میں حالت دوڑ سے پاک حال کا متغیر منتخب کرنے کے لئے ہم \اصطلاح{ایک بلند بِٹ تقرری }\فرہنگ{ایک بلند بِٹ تقرری}\حاشیہب{one hot bit assignment}\فرہنگ{one hot bit assignment} کا طریقہ استعمال کرتے ہیں، جس کا استعمال نہایت آسان ہے۔آئیے اسی مثال پر اسے استعمال کرتے ہیں۔

شکل\حوالہء{ 11.16 } میں حال کا متغیر چار بِٹ رکھا گیا ہے اور اس میں ایک وقت پر صرف ایک بِٹ بلند ہے۔یوں حال \عددی{a} ، \عددی{b}، \عددی{c}، اور \عددی{d} کے لئے حال کے متغیر بالترتیب \عددی{0001}، \عددی{0010}، \عددی{0100}، اور \عددی{1000} مقرر کیے گئے۔


شکل \حوالہء{11.16 } میں جدول کی پہلی صف میں مداخل کی قیمت \عددی{00} سے \عددی{01} کرنے سے دور حال \عددی{a} سے حال \عددی{b} منتقل ہوتا ہے۔یوں حال کا متغیر \عددی{0001} سے \عددی{0010} ہو گا اور اس میں دو بِٹ کی تبدیلی حالت دوڑ پیدا کرے گی۔اس سے بچنے کے لئے جدول میں ایک نیا عبوری حال، \عددی{e}، شامل کیا جاتا ہے ۔حال \عددی{a} سے \عددی{b} پہنچنے کے لئے اس عبوری حال سے گزرنا لازمی بنایا جاتا ہے۔عبوری حال \عددی{e} کے لئے حال کا متغیر یوں مقرر کیا جاتا ہے کہ یہ \عددی{a} اور \عددی{b}دونوں کا ہمسایہ عدد ہو۔ایسا عدد \عددی{0011} ہے۔یوں \عددی{e} کے لئے حال کا متغیر \عددی{0011} مقرر کیا جاتا ہے اور جدول کو تبدیل کر کے \عددی{x=01}کی قطار کے حال \عددی{a} کی صف میں \عددی{b} کی بجائے \عددی{e} لکھا جاتا ہے جبکہ اسی قطار میں حال \عددی{e} کی صف میں \عددی{b} لکھا جاتا ہے۔ایسا کرنے سے جدول تبدیل ہو کر شکل \حوالہء{11.17 } اختیار کرتا ہے۔


اب پہلی صف میں مداخل \عددی{00} سے \عددی{01} کرنے سے دور حال \عددی{a} سے عبوری حال \عددی{e} اختیار کرتے ہوئے آخر کار اختتامی مستحکم حال \عددی{b} پہنچتا ہے۔اس عمل کو نقطہ دار تیر دار لکیروں سے ظاہر کیا گیا ہے۔ اس پورے عمل میں ہر قدم پر حال کے متغیر کا صرف ایک بِٹ تبدیل ہوتا ہے لہٰذا حالت دوڑ پیدا نہیں ہو گی ۔ عبوری حال \عددی{e} کی صف میں باقی خانے خالی رکھے گئے ہیں۔ان میں سے کچھ خانے زیر استعمال آئیں گے اور کچھ نہیں۔استعمال میں نہ آنے والے خانے خالی رکھے جاتے ہیں اور ان خانوں کی قیمت\اصطلاح{ غیر ضروری }\فرہنگ{غیر ضروری}\حاشیہب{don't care}\فرہنگ{don't care} ہو گی۔

پہلی صف میں مداخل \عددی{00}سے \عددی{10} کرنے سے شکل \حوالہء{11.17 } میں حال \عددی{a} سے حال \عددی{c}حاصل ہو گا۔حال کا متغیر \عددی{0001} سے تبدیل ہو کر \عددی{0100} ہونا چاہے گا۔البتہ ایسا کرنے سے حالت دوڑ پیدا ہو گی، جس سے ہم مذکورہ بالا طریقے سے چھٹکارا حاصل کرتے ہیں۔

اس حالت دوڑ سے بچنے کے لئے جدول میں عبوری حال، \عددی{f} ، شامل کیا جاتا ہے اور حال \عددی{a} سے عبوری حال \عددی{f} کے ذریعہ حال \عددی{c} پہنچا جاتا ہے۔عبوری حال \عددی{f} کے لئے حال کا متغیر یوں مقرر کیا جاتا ہے کہ یہ \عددی{a} اور \عددی{c} دونوں کا ہمسایہ عدد ہو۔ایسا عدد \عددی{0101} ہے۔یوں ح \عددی{f} کے لئے حال کا متغیر \عددی{0101} مقرر کیا جاتا ہے اور جدول کو تبدیل کر کے \عددی{x=10} کی قطار میں حال \عددی{a} کی صف \عددی{c} کو تبدیل کر کے \عددی{f} لکھا جاتا ہے جبکہ اسی قطار میں حال \عددی{f} کی صف میں \عددی{c} لکھا جاتا ہے۔ایسا کرنے سے شکل \حوالہء{ 11.18} ملتا ہے۔

یہی طریقہ کار تمام خانوں کے لئے دہرایا جاتا ہے۔ایسا کرنے سے شکل \حوالہء{11.19 } حاصل ہو گا۔آپ سے گزارش کی جاتی ہے کہ یہ جدول خود حاصل کریں۔تسلی کر لیں کہ اس جدول میں کسی بھی حال سے دوسرے حال تک پہنچنے میں حالت دوڑ پیدا نہیں ہوتی۔


\حصہ{ عبوری جدول کی مدد سے پلٹ کا تجزیہ}
عبوری جدول استعمال کر کے سے اس حصے میں پلٹ کا تجزیہ کیا جائے گا۔چند مثالوں کے بعد حصہ \حوالہ{حصہ_غیر_معاصر_قدم_با_قدم} میں اس طریقہ کار پر قدم با قدم غور کیا جائے گا۔

\جزوحصہ{ایس آر پلٹ}
عبوری جدول استعمال کر کے سب سے پہلے ایس آر پلٹ پر غور کرتے ہیں۔شکل \حوالہء{11.20 } میں اُوپر ایس آر پلٹ اور نیچے اسی کو بطور \موٹا{ باز رسی دور } پیش کیا گیا ہے جہاں\اصطلاح{ باز رسی اشارہ } \عددی{q} کی پہچان آسان ہے۔

 حال کے متغیر \عددی{Q} کو بطور باز رسی اشارہ \عددی{q} استعمال کیا گیا ہے۔یوں حال کا متغیر \عددی{Q}، اندرونی مداخل \عددی{q} جبکہ بیرونی مداخل \عددی{S} اور \عددی{R} ہیں۔انہیں استعمال کرتے ہوئے عبوری جدول حاصل کی گئی ہے (شکل \حوالہء{11.20 } دیکھیں)۔آئیے اس پلٹ کا تجزیہ اس کے عبوری جدول کی مدد سے کریں۔پلٹ کا جدول صداقت مندرجہ ذیل ہے۔
 \begin{align*}
 \begin{array}{cc|cc}
 \toprule
 S&R&Q_{n+1}&\overline{Q}_{n+1}\\
 \midrule
 0&0&Q_n&\overline{Q}_n\\
 0&1&0&1\\
 1&0&1&0\\
 1&1&0&0\\
 \bottomrule
 \end{array}
\end{align*}
 جدول سے ظاہر ہے کہ جمع متمم گیٹ پر مبنی ایس آر پلٹ استعمال کرتے ہوئے دونوں مداخل بیکوقت بلند کرنے کی اجازت نہیں۔ دونوں مداخل بیکوقت بلند کرنے سے پلٹ کے مخارج \عددی{Q} اور \عددی{\overline{Q}} بیکوقت پست ہوں گے جبکہ ہر صورت ان کا آپس میں متضاد رہنا ضروری ہے۔درج ذیل مساوات پر پورا اترنے سے یہ شرط پوری ہو گی۔
 \begin{align}
 S\cdot R=0
 \end{align}

شکل \حوالہء{11.21 } پر نظر رکھ کر آگے پڑھیں۔عبوری جدول کی \عددی{SR=01} قطار اور \عددی{q=0} صف میں مستحکم حال پایا جاتا ہے جہاں حال کا متغیر پست \عددی{(Q=0)} ہے۔ عبوری جدول کے تحت \عددی{SR=00} کرنے سے حال کا متغیر پست رہے گا۔شکل -الف میں نقطہ دار تیر دار لکیر اس عمل کو ظاہر کرتی ہے۔

اسی طرح \عددی{SR=10} کی صورت میں پلٹ کا بلند مستحکم حال \عددی{q=1} کی صف میں پایا جاتا ہے۔عبوری جدول کے مطابق \عددی{SR=00} کرنے سے پلٹ بلند حال میں رہے گا، جیسا شکل-ب میں دکھایا گیا ہے۔یہ دونوں اعمال پلٹ کے بوولین جدول سے بھی واضح ہیں۔

اب دیکھتے ہیں \عددی{SR=11} سے آغاز کرتے ہوئے \عددی{SR=00} کرنے سے کیا صورت پیدا ہوتی ہے۔یاد رہے ان ادوار کو \موٹا{ بنیادی طریق کار }کے تحت چلایا جاتا ہے جہاں ایک سے زیادہ بیرونی مداخل تبدیل کرنے کی اجازت نہیں۔بہرحال پھر بھی دیکھتے ہیں کہ ایسا کرنے سے کیا مسائل کھڑے ہوتے ہیں۔بوولین جدول کے مطابق \عددی{SR=00}کرنے سے قبل \عددی{Q} اور \عددی{\overline{Q}} دونوں پست ہوں گے نا کہ آپس میں متضاد جبکہ کسی بھی پلٹ کے لئے لازم ہے کہ اس کے دونوں مخارج ہر وقت متضاد حال ہوں۔ساتھ ہی، عبوری جدول کے تحت اگر \عددی{S}پہلے پست حال اختیار کر لے تو اختتامی حال \عددی{0} ہو گا جبکہ اگر \عددی{R} پہلے پست ہو تب اختتامی حال \عددی{1} ہو گا۔چونکہ قبل از وقت یہ جاننا ممکن نہیں کہ \عددی{S} یا \عددی{R} پہلے پست ہو گا لہٰذا اختتامی حال جاننا ممکن نہیں۔ دور کا یوں استعمال غیر یقینی صورت پیدا کرے گا۔


\جزوحصہ{ ساعت کے کنارہ پر چلتا ہوا ڈی پلٹ}
شکل \حوالہء{11.22 } ڈی پلٹ دکھایا گیا ہے جو ساعت کے کنارہ پر چلتا ہے۔ڈی پلٹ میں اندرونی باز رسی دور پایا جاتا ہے جس کے اندرونی حال کے متغیرات \عددی{S} اور \عددی{R} جبکہ باز رسی اشارات \عددی{s} اور \عددی{r} ہیں\حاشیہد{اس کتاب میں ضرب متمم گیٹ پر مبنی ایس آر پلٹ کے مداخل عموماً \عددی{\overline{S}} اور \عددی{\overline{R}} لکھے گئے ہیں۔ یہاں \عددی{S} اور \عددی{R} لکھا گیا ہے۔ امید کی جاتی ہے کہ اس سے پریشانی پیدا نہیں ہو گی۔}۔شکل میں ڈی پلٹ کو دوبارہ باز رسی دور کے طرز پر بنایا گیا ہے تا کہ باز رسی اشارات \عددی{s} اور \عددی{r} کی پہچان آسان ہو۔

اس دور میں \عددی{S} اور \عددی{R} حال کے متغیرات ، \عددی{s} اور \عددی{r} باز رسی اشارات ، جبکہ \عددی{C} اور \عددی{D} بیرونی مداخل ہیں۔یوں درج ذیل لکھا جا سکتا ہے۔
\begin{gather}
\begin{aligned}
A&=\overline{sB}\\
B&=\overline{Dr}\\
S&=\overline{AC}=\overline{A}+\overline{C}=\overline{\overline{sB}}+\overline{C}=sB+\overline{C}=s(\overline{rD})+\overline{C}\\
&=s(\overline{r}+\overline{D})+\overline{C}\\
R&=\overline{BCs}=\overline{B}+\overline{C}+\overline{s}=\overline{\overline{Dr}}+\overline{C}+\overline{s}\\
&=Dr+\overline{C}+\overline{s}
\end{aligned}
\end{gather}

 ان مساوات سے حاصل \عددی{S} اور \عددی{R} کے بوولین جدول کو کارناف نقشوں کی طرز پر لکھ کر شکل\حوالہء{ 11.23 } میں دکھایا گیا عبوری جدول حاصل کیا گیا۔\اصطلاح{مکمل حال }\فرہنگ{حال!مکمل}\حاشیہب{complete state}\فرہنگ{state!complete} \عددی{srCD} کی صورت میں لکھتے ہوئے اس جدول پر غور کرتے ہیں۔


فرض کریں جس لمحے پلٹ کو برقی طاقت مہیا کر کے زندہ کیا جاتا ہے اس لمحے ساعت، \عددی{C}، اور بیرونی مداخل، \عددی{D} ،دونوں پست ہیں۔عبوری جدول کے مطابق دور \عددی{CD=00} کی قطار میں ہوگا۔اس قطار میں تین خانے \عددی{0000}، \عددی{0100}، اور \عددی{1000} عبوری حال کے متغیر ظاہر کرتے ہیں۔ ان خانوں میں عبوری حال \عددی{SR=11} ہے۔چوتھا خانہ،\عددی{1100}، مستحکم حال \عددی{SR=11} ظاہر کرتا ہے۔ اگر برقی طاقت کی فراہمی کے لمحے تاخیر ایسی ہوں کہ دور ان تین عبوری خانوں میں سے کسی ایک میں داخل ہو تو وہ یہاں سے جلد \عددی{sr=11} کی صف پہنچ کر مستحکم حال اختیار کر ے گا۔اگر زندہ ہوتے ہی دور سیدھا \عددی{1100} خانے میں داخل ہو تب وہ یہی رہے گا۔

اس کے برعکس برقی طاقت مہیا کرنے کے لمحے اگر \عددی{C=1} اور \عددی{D=1} ہو تب عبوری جدول کے مطابق دور \عددی{0111} یا \عددی{1011} مستحکم حال پہنچ کر یہی رہے گا، جبکہ \عددی{C=1} اور \عددی{D=0} کی صورت میں دور \عددی{0110} یا \عددی{1010} حال میں ہو گا۔

پست ساعت کی صورت میں حال کے متغیر \عددی{SR} کی قیمت \عددی{11} رہتی ہے۔عبوری جدول میں \عددی{CD=00} اور \عددی{CD=01} کی دو قطاریں اس حقیقت کو ظاہر کرتی ہیں جہاں تمام \عددی{SR} کی قیمت \عددی{11} ہے۔ہم جانتے ہیں ایس آر پلٹ کے دونوں مداخل بلند ہونے کی صورت میں پلٹ اپنا حال برقرار رکھتی ہے۔یوں شکل \حوالہء{11.22 } میں خارجی پلٹ اپنا حال برقرار رکھے گی۔

پست ساعت، \عددی{C=0}، اور پست \عددی{D} کی صورت میں مستحکم حال کا متغیر \عددی{SR} حاصل کرنے کی خاطر ہم عبوری جدول کی \عددی{CD=00} قطار میں دیکھتے ہیں جہاں ہمیں \موٹا{ مکمل حال } \عددی{srCD=1100} بطور مستحکم حال ملتا ہے۔جدول کے اس خانے میں \عددی{a} لکھ کر اسے اجاگر کیا گیا ہے۔یہاں \عددی{SR=11} کی بدولت خارجی پلٹ اپنا حال برقرار رکھے گی۔

پست ساعت اور بلند \عددی{D} کی صورت میں \عددی{CD=01} کی قطار میں مستحکم حال \عددی{1101} پایا جاتا ہے جہاں \عددی{SR=11} ہے اور یوں خارجی پلٹ اپنا حال برقرار رکھے گی۔جدول کے اس خانے میں \عددی{b} لکھ کر اسے اجاگر کیا گیا ہے۔

فرض کریں دور مستحکم حال \عددی{1100}،یعنی خانہ \عددی{a} ،میں ہے جب بیرونی مداخل \عددی{C} بلند ہوتا ہے۔بیرونی مداخل \عددی{C} جس لمحہ \عددی{0} سے \عددی{1} ہوتا ہے اس لمحے کو ساعت کا \اصطلاح{کنارہ چڑھائی }\فرہنگ{کنارہ!چڑھائی}\حاشیہب{rising edge}\فرہنگ{edge!rising} کہتے ہیں۔ یوں \عددی{D=0} کی صورت میں ساعت کے کنارہ چڑھائی پر دور خانہ \عددی{a} کی صف میں رہتے ہوئے، \عددی{CD=00} سے \عددی{CD=10} کی قطار میں داخل ہو کر عبوری حال \عددی{1110} اختیار کرتا ہے۔اس عبوری حال کو خانہ \عددی{e} کہا گیا ہے، جہاں سے دور جلد اختتامی مستحکم حال \عددی{1010} پہنچے گا جس کو خانہ \عددی{m} ظاہر کرتا ہے۔ حال \عددی{1010} میں حال کا متغیر \عددی{SR=10} ہے۔خارجی پلٹ \عددی{SR=10} کی صورت میں پست حال اختیار کر ے گی لہٰذا \عددی{Q=0} ہو جائے گا۔اس قدم کو خانہ \عددی{a}سے خانہ \عددی{e} کے راستے خانہ \عددی{m} تک تیر دار لکیر سے ظاہر کیا گیا ہے۔ خلاصہ یہ ہے کہ \عددی{D=0} کی صورت میں ساعت کے کنارہ چڑھائی پر \عددی{Q=0} ہو جائے گا یعنی ڈی پلٹ پست حال اختیار کرے گی۔

اس پورے عمل پر دوبارہ غور کرتے ہیں۔ساعت کے کنارہ چڑھائی آتے ہی دور عبوری حال \عددی{1110} سے گزر کر مستحکم حال \عددی{1010} اختیار کرتا ہے۔ان دونوں حال میں \عددی{SR=10} رہتا ہے اور یوں عبوری حال سے گزرتے ہوئے لرزش پیدا نہیں ہوگی۔آگے پڑھتے ہوئے تسلی کر لیں کہ ہر قدم پر کسی بھی عبوری حال سے گزرتے وقت \عددی{SR} کی قیمت وہی ہو گی جو اس قدم کے اختتامی حال میں ہو گی۔یوں ان لمحات پر لرزش سے کسی قسم کی غیر یقینی صورت پیدا نہیں ہو گی۔

اسی طرح مکمل حال \عددی{srCD=1101} میں موجود دور ، ساعت کے کنارہ چڑھائی پر ، عبوری حال \عددی{1111} سے گزر کر مستحکم حال \عددی{0111} اختیار کرے گا۔اس قدم کو خانہ \عددی{b} سے خانہ \عددی{k} کے راستے خانہ \عددی{n} تک تیر دار لکیر ظاہر کرتی ہے۔یہ قدم بلند بیرونی مداخل \عددی{D=1} اور ساعت کے کنارہ چڑھائی پر \عددی{SR=01} کی صورت میں ہونے والا عمل ہے جس سے داخلی پلٹ بلند ہو کر ڈی پلٹ کا مخارج بلند \عددی{(Q=1)} کرتا ہے۔

ساعت کے \اصطلاح{ کنارہ اترائی } پر ہونے والے عمل کو نقطہ دار تیر دار لکیروں سے ظاہر کیا گیا ہے۔انہیں آپ خود سمجھ سکتے ہیں۔یہ دونوں لکیریں یہ حقیقت واضح کرتی ہیں کہ ساعت کے کنارہ اترائی پر عبوری حال اور اختتامی مستحکم حال دونوں میں \عددی{SR=11} ہو گا لہٰذا بیرونی پلٹ اپنا حال برقرار رکھے گی اور یوں ساعت کے کنارہ اترائی پر ڈی پلٹ کے حال میں کسی قسم کی تبدیلی رو نما نہیں ہوگی۔

ایک آخری بات اس پلٹ کے حوالے سے کرتے ہیں۔شکل \حوالہء{ 11.22 } میں \عددی{R} پیدا کرنے والے ضرب متمم گیٹ کو \عددی{S} بطور داخلی اشارہ مہیا کیا گیا ہے، جس کی بدولت \عددی{S} اور \عددی{R} کسی صورت بیکوقت پست نہیں ہو سکتے۔یاد رہے کہ \عددی{S} اور \عددی{R} دونوں بیکوقت پست ہونے سے بیرونی پلٹ کے دونوں مخارج بلند ہو جائیں گے جو کہ ناقابل قبول صورت ہو گی۔یوں عبوری جدول میں \عددی{0010} اور \عددی{0011} کے خانے کوئی معنی نہیں رکھتے۔ان خانوں کو \عددی{x} لکھ کر اجاگر کیا گیا ہے۔

%???KKK
\جزوحصہ{ایس آر پلٹوں پر مبنی غیر معاصر ادوار کا قدم با قدم تجزیہ}\شناخت{حصہ_غیر_معاصر_قدم_با_قدم}
مذکورہ بالا مثالوں میں استعمال کیے گئے طریقہ کار کو یہاں بیان کرتے ہیں۔پلٹ کے اپنے باز رسی اشارات کو نظر انداز کرتے ہیں۔
\begin{itemize}
\item
 تمام پلٹوں کے مخارج کو \عددی{Y_i}سے ظاہر کریں جہاں \عددی{i=0,1,2,\cdots}ہے۔ مخارج سے حاصل باز رسی اشارے کو اس مخارج کا \عددی{i} استعمال کرتے ہوئے \عددی{y_i} لکھیں۔ یوں \عددی{Y_3} سے حاصل باز رسی اشارہ \عددی{y_3} کہلائے گا۔
\item
 تمام پلٹوں کے \عددی{S_i} اور \عددی{R_i} مداخل کی مساوات حاصل کریں۔
\item
 جمع متمم گیٹ پر مبنی ایس آر پلٹ کے لئے تسلی کر لیں کہ \عددی{SR=0} ہے جبکہ ضرب متمم گیٹ پر مبنی ایس آر پلٹ کے لئے \عددی{\overline{S}\,\overline{R}=0} ہونا ضروری ہے۔ایسا نہ ہونے کی صورت میں پلٹ غلط نتائج دے سکتا ہے۔
\item 
\عددی{S_i} اور \عددی{R_i} دیکھ کر تمام پلٹ کے \عددی{Y_i} حاصل کریں۔
\item
 ہر \عددی{Y_i} کو کارناف نقشے کے طرز پر لکھیں۔ان نقشوں کی بائیں جانب قطار میں باز رسی اشارات \عددی{y} جبکہ نقشوں کے اُوپر صف میں بیرونی مداخل \عددی{x} لکھیں جہاں \عددی{y} سے مراد \عددی{\cdots y_3y_2y_1y_0} جبکہ \عددی{x} سے مراد \عددی{\cdots x_3x_2x_1x_0} ہے۔
\item
 ان نقشوں کو عبوری جدول میں یکجا کریں۔ ان نقشوں کے خانوں میں \عددی{Y} لکھیں، جہاں \عددی{Y} سے مراد \عددی{\cdots Y_3Y_2Y_1Y_0} ہے۔
\item
 وہ خانے جن میں \عددی{Y=y} ہو، مستحکم حال ظاہر کرتے ہیں۔انہیں دائرہ میں بند کریں۔یوں عبوری جدول حاصل ہو گا۔
\end{itemize}


\باب{سادہ ترین کمپیوٹر}
اس باب میں کمپیوٹر کی سادہ ترین ساخت پر غور کیا جائے گا۔ سادہ ہونے کے باوجود اس میں   کئی   اعلٰی  تصورات شامل ہیں۔ اس باب کو پڑھنے اور   سمجھنے  کے بعد آپ جدید کمپیوٹر کی بناؤٹ سمجھ پائیں گے۔

\حصہ{بناؤٹ}
سادہ کمپیوٹر کی  بناوٹ شکل \حوالہ{شکل_کمپیوٹر_سادہ_ترین} میں پیش ہے۔ یہ ایک مکمل کمپیوٹر ہے۔   دفاتر کے    وہ خروج جو آٹھ بِٹ گزر گاہ     سے جڑے ہیں،  \اصطلاح{سہ حالی}\فرہنگ{حال!سہ}\حاشیہب{tri-state}\فرہنگ{state!tri}    ہیں؛  جو  مواد کی  منظم ترسیل ممکن بناتا  ہے۔ آٹھ بِٹ  گزر گاہ سے مراد آٹھ برقی تاریں ہیں جو    ذیلی ادوار (مثلاً حافظہ، جمع و منفی کار)  کے مابین مواد کی ترسیل ممکن بناتے ہیں۔ دفاتر کے باقی خروج \اصطلاح{   دو حالی  }\فرہنگ{حال!دو}\حاشیہب{two-state}\فرہنگ{state!two} ہیں؛ یہ خروج   ان ڈبہ ادوار کو مسلسل   معلومات (مواد، پتہ، شمار وغیرہ)  فراہم کرتے ہیں جن سے یہ منسلک ہیں۔

\begin{figure}
\centering
\begin{tikzpicture}
\pgfmathsetmacro{\klshift}{0.25}
\pgfmathsetmacro{\knshift}{0.07}
\pgfmathsetmacro{\kmv}{0.15}
\pgfmathsetmacro{\knshift}{0.07}
\pgfmathsetmacro{\kpin}{0.50}
\pgfmathsetmacro{\kpina}{1}
\pgfmathsetmacro{\kpsep}{0.40}			%pin to pin distance
\pgfmathsetmacro{\kW}{\kpsep}
\pgfmathsetmacro{\kulV}{0.40}			%edge clearance along vertical edge
\pgfmathsetmacro{\kulH}{0.50}
\pgfmathsetmacro{\kdimX}{2*\kulH+4*\kpsep}
\pgfmathsetmacro{\kdimY}{2*\kulV+4*\kpsep}		%two spaces between 3 pins
\pgfmathsetmacro{\ksepX}{\kdimX+2*\kpina+1*\kpsep+\kW}
\pgfmathsetmacro{\ksepY}{\kdimY+\kpin+0.5*\kpsep}
\draw(0,0)  [thick] rectangle ++(\kdimX,\kdimY)node[pos=0.5,rectangle,inner sep=0pt,text width=1.5cm,align=center]{\RTL{ہدایت گنت کار}};
\draw(0,\kulV+0*\kpsep)--++(-\kpin,0)node[left]{$E_P$};
\draw(0,\kulV+1*\kpsep)--++(-\kpin,0)node[left]{$\overline{CLR}$};
\draw(0,\kulV+2*\kpsep)--++(-\kpin,0)node[left]{$\overline{CLK}$};
\draw(0,\kulV+4*\kpsep)--++(-\kpin,0)node[left]{$C_P$};
\draw(0,\kulV+2*\kpsep-\kmv)--++(\kmv,\kmv)--++(-\kmv,\kmv);
\draw(0,\kulV+2*\kpsep)++(-\knshift,0)node[ocirc]{};
\draw(0,\kulV+1*\kpsep)++(-\knshift,0)node[ocirc]{};
\draw(\kdimX,\kulV+1.5*\kpsep)--++(\kpina,0);
\draw(\kdimX,\kulV+2.5*\kpsep)--++(\kpina,0);
\draw(\kdimX,\kulV+1.5*\kpsep)++(\kpina,0)++(-0.5*\kpsep,-0.5*\kpsep)--++(1*\kpsep,1*\kpsep)--++(-1*\kpsep,1*\kpsep);
\draw(\kdimX,\kulV+2*\kpsep)++(0.5*\kpina,0)node[]{$4$};

\draw(0,-\ksepY)  [thick] rectangle ++(\kdimX,\kdimY)node[pos=0.5,rectangle,inner sep=0pt,text width=2cm,align=center]{\RTL{برنامہ نویس\\ اور\\ دفتر پتہ}};
\draw(0,-\ksepY+\kulV+2*\kpsep)--++(-\kpina,0)node[left]{$CLK$};
\draw(0,-\ksepY+\kulV+2*\kpsep-\kmv)--++(\kmv,\kmv)--++(-\kmv,\kmv);
\draw(0,-\ksepY+\kulV+4*\kpsep)--++(-\kpina,0)node[left]{$\overline{L}_M$};
\draw(0,-\ksepY+\kulV+4*\kpsep)++(-\knshift,0)node[ocirc]{};
\draw(\kulH+0*\kpsep,-\ksepY)--++(0,-\kpin);
\draw(\kulH+1*\kpsep,-\ksepY)--++(0,-\kpin);
\draw(\kulH+0*\kpsep,-\ksepY)++(0,-\kpin)++(-0.5*\kpsep,0.5*\kpsep)--++(\kpsep,-\kpsep)--++(\kpsep,\kpsep);
\draw(\kulH+0.5*\kpsep,-\ksepY)++(0,-0.5*\kpin)node[]{$4$};
\draw(\kulH+3*\kpsep,-\ksepY)--++(0,-\kpin);
\draw(\kulH+4*\kpsep,-\ksepY)--++(0,-\kpin);
\draw(\kulH+3*\kpsep,-\ksepY)++(0,-\kpin)++(-0.5*\kpsep,0.5*\kpsep)--++(\kpsep,-\kpsep)--++(\kpsep,\kpsep);
\draw(\kulH+3.5*\kpsep,-\ksepY)++(0,-0.5*\kpin)node[]{$4$};
\draw(\kdimX,-\ksepY+\kulV+1.5*\kpsep)++(0.5*\kpsep,0)--++(\kpina,0);
\draw(\kdimX,-\ksepY+\kulV+2.5*\kpsep)++(0.5*\kpsep,0)--++(\kpina,0);
\draw(\kdimX,-\ksepY+\kulV+2*\kpsep)--++(1*\kpsep,-1*\kpsep);
\draw(\kdimX,-\ksepY+\kulV+2*\kpsep)--++(1*\kpsep,1*\kpsep);
\draw(\kdimX,-\ksepY+\kulV+2*\kpsep)++(0.5*\kpina,0)node[]{$4$};

\draw(0,-2*\ksepY)  [thick] rectangle ++(\kdimX,\kdimY)node[pos=0.5,rectangle,inner sep=0pt,text width=2.5cm,align=center]{\RTL{$16\times 8$\\ عارضی حافظہ}};
\draw(0,-2*\ksepY+\kulV)--++(-\kpina,0)node[left]{$\overline{CE}$}; 
\draw(0,-2*\ksepY+\kulV)++(-\knshift,0)node[ocirc]{};
\draw(\kdimX,-2*\ksepY+\kulV+1.5*\kpsep)--++(\kpina,0);
\draw(\kdimX,-2*\ksepY+\kulV+2.5*\kpsep)--++(\kpina,0);
\draw(\kdimX,-2*\ksepY+\kulV+1.5*\kpsep)++(\kpina,0)++(-0.5*\kpsep,-0.5*\kpsep)--++(1*\kpsep,1*\kpsep)--++(-1*\kpsep,1*\kpsep);
\draw(\kdimX,-2*\ksepY+\kulV+2*\kpsep)++(0.5*\kpina,0)node[]{$8$};

\draw(0,-3*\ksepY)  [thick] rectangle ++(\kdimX,\kdimY)node[pos=0.5,rectangle,inner sep=0pt,text width=2cm,align=center]{\RTL{دفتر ہدایت}};
\draw(0,-3*\ksepY+\kulV+0*\kpsep)--++(-\kpina,0)node[left]{$\overline{E}_I$};
\draw(0,-3*\ksepY+\kulV+1*\kpsep)--++(-\kpina,0)node[left]{$CLR$};
\draw(0,-3*\ksepY+\kulV+2*\kpsep)--++(-\kpina,0)node[left]{$CLK$};
\draw(0,-3*\ksepY+\kulV+4*\kpsep)--++(-\kpina,0)node[left]{$\overline{L}_I$};
\draw(0,-3*\ksepY+\kulV+0*\kpsep)++(-\knshift,0)node[ocirc]{};
\draw(0,-3*\ksepY+\kulV+4*\kpsep)++(-\knshift,0)node[ocirc]{};
\draw(\kdimX,-3*\ksepY+\kulV+0*\kpsep)--++(\kpina,0);
\draw(\kdimX,-3*\ksepY+\kulV+1*\kpsep)--++(\kpina,0);
\draw(\kdimX,-3*\ksepY+\kulV+0*\kpsep)++(\kpina,0)++(-0.5*\kpsep,-0.5*\kpsep)--++(1*\kpsep,1*\kpsep)--++(-1*\kpsep,1*\kpsep);
\draw(\kdimX,-3*\ksepY+\kulV+0.5*\kpsep)++(0.5*\kpina,0)node[]{$4$};
\draw(\kdimX,-3*\ksepY+\kulV+3*\kpsep)++(0.5*\kpsep,0)--++(\kpina,0);
\draw(\kdimX,-3*\ksepY+\kulV+4*\kpsep)++(0.5*\kpsep,0)--++(\kpina,0);
\draw(\kdimX,-3*\ksepY+\kulV+3.5*\kpsep)--++(1*\kpsep,-1*\kpsep);
\draw(\kdimX,-3*\ksepY+\kulV+3.5*\kpsep)--++(1*\kpsep,1*\kpsep);
\draw(\kdimX,-3*\ksepY+\kulV+3.5*\kpsep)++(0.5*\kpina,0)node[]{$8$};
\draw(\kulH+1.5*\kpsep,-3*\ksepY)--++(0,-\kpin);
\draw(\kulH+2.5*\kpsep,-3*\ksepY)--++(0,-\kpin);
\draw(\kulH+1.5*\kpsep,-3*\ksepY)++(0,-\kpin)++(-0.5*\kpsep,0.5*\kpsep)--++(\kpsep,-\kpsep)--++(\kpsep,\kpsep);
\draw(\kulH+2*\kpsep,-3*\ksepY)++(0,-0.5*\kpin)node[]{$4$};

\draw(0,-4*\ksepY)  [thick] rectangle ++(\kdimX,\kdimY)node[pos=0.5,rectangle,inner sep=0pt,text width=3cm,align=center]{\RTL{قابو  و ترتیب کار}};
\draw[-latex](\kdimX,-4*\ksepY+\kulV+0*\kpsep)--++(\kpina,0)node[right]{$\overline{CLR}$}; 
\draw[-latex](\kdimX,-4*\ksepY+\kulV+1*\kpsep)--++(\kpina,0)node[right]{$CLR$}; 
\draw[-latex](\kdimX,-4*\ksepY+\kulV+2*\kpsep)--++(\kpina,0)node[right]{$\overline{CLK}$}; 
\draw[-latex](\kdimX,-4*\ksepY+\kulV+3*\kpsep)--++(\kpina,0)node[right]{$CLK$}; 
\draw(\kulH+1.5*\kpsep,-4*\ksepY)--++(0,-\kpin);
\draw(\kulH+2.5*\kpsep,-4*\ksepY)--++(0,-\kpin);
\draw(\kulH+1.5*\kpsep,-4*\ksepY)++(0,-\kpin)++(-0.5*\kpsep,0.5*\kpsep)--++(\kpsep,-\kpsep)--++(\kpsep,\kpsep);
\draw(\kulH+2*\kpsep,-4*\ksepY)++(0,-0.5*\kpin)node[]{$12$};
\draw(\kulH+2*\kpsep,-4*\ksepY)++(0,-1*\kpin-\kpsep)node[below]{$C_PE_P\overline{L}_M\overline{CE}\,\,\, \overline{L}_I\overline{E}_I\overline{L}_AE_A\,\,S_UE_U\overline{L}_B\overline{L}_O$};
\draw(\kdimX,-4*\ksepY+\kulV+0*\kpsep)++(\knshift,0)node[ocirc]{};
\draw(\kdimX,-4*\ksepY+\kulV+2*\kpsep)++(\knshift,0)node[ocirc]{};

\draw(\kdimX+\kpina+0.5*\kpsep,\kdimY)  [thick] rectangle ++(\kW,-4*\ksepY+\kpin);
\draw(\kdimX+\kpina+0.5*\kpsep,\kdimY)++(0.5*\kW,0)node[above, text width=1.5cm,align=center]{\RTL{$W$ گزر گاہ}}node[below]{$8$};

\draw(\ksepX,-0*\ksepY)  [thick] rectangle ++(\kdimX,\kdimY)node[pos=0.5,rectangle,inner sep=0pt,text width=1cm,align=center]{\RTL{دفتر\\ $A$}};
\draw(\ksepX,-0*\ksepY+\kulV+0*\kpsep)--++(-\kpina,0); 
\draw(\ksepX,-0*\ksepY+\kulV+1*\kpsep)--++(-\kpina,0); 
\draw(\ksepX,-0*\ksepY+\kulV+0*\kpsep)++(-\kpina,0)++(0.5*\kpsep,-0.5*\kpsep)--++(-\kpsep,\kpsep)--++(\kpsep,\kpsep);
\draw(\ksepX,-0*\ksepY+\kulV+0.5*\kpsep)++(-0.5*\kpina,0)node[]{$8$};
\draw(\ksepX,-0*\ksepY+\kulV+3*\kpsep)++(-0.5*\kpsep,0)--++(-\kpina,0); 
\draw(\ksepX,-0*\ksepY+\kulV+4*\kpsep)++(-0.5*\kpsep,0)--++(-\kpina,0); 
\draw(\ksepX,-0*\ksepY+\kulV+3*\kpsep)++(-1*\kpsep,-0.5*\kpsep)--++(\kpsep,\kpsep)--++(-\kpsep,\kpsep);
\draw(\ksepX,-0*\ksepY+\kulV+3.5*\kpsep)++(-0.5*\kpina,0)node[]{$8$};
\draw(\ksepX+\kulH+1.5*\kpsep,-0*\ksepY)--++(0,-\kpin);
\draw(\ksepX+\kulH+2.5*\kpsep,-0*\ksepY)--++(0,-\kpin);
\draw(\ksepX+\kulH+1.5*\kpsep,-0*\ksepY)++(0,-\kpin)++(-0.5*\kpsep,0.5*\kpsep)--++(\kpsep,-\kpsep)--++(\kpsep,\kpsep);
\draw(\ksepX+\kulH+2*\kpsep,-0*\ksepY)++(0,-0.5*\kpin)node[]{$8$};
\draw(\ksepX+\kdimX,-0*\ksepY+\kulV+0*\kpsep)--++(\kpin,0)node[right]{$E_A$};
\draw(\ksepX+\kdimX,-0*\ksepY+\kulV+2*\kpsep)--++(\kpin,0)node[right]{$CLK$};
\draw(\ksepX+\kdimX,-0*\ksepY+\kulV+4*\kpsep)--++(\kpin,0)node[right]{$\overline{L}_A$};
\draw(\ksepX+\kdimX,-0*\ksepY+\kulV+2*\kpsep)++(0,-\kmv)--++(-\kmv,\kmv)--++(\kmv,\kmv);
\draw(\ksepX+\kdimX,-0*\ksepY+\kulV+4*\kpsep)++(\knshift,0)node[ocirc]{};
\draw(\ksepX,-1*\ksepY)  [thick] rectangle ++(\kdimX,\kdimY)node[pos=0.5,rectangle,inner sep=0pt,text width=1cm,align=center]{\RTL{جمع و منفی کار}};
\draw(\ksepX+\kdimX,-1*\ksepY+\kulV+0*\kpsep)--++(\kpin,0)node[right]{$E_U$};
\draw(\ksepX+\kdimX,-1*\ksepY+\kulV+4*\kpsep)--++(\kpin,0)node[right]{$S_U$};
\draw(\ksepX,-1*\ksepY+\kulV+1.5*\kpsep)--++(-\kpina,0); 
\draw(\ksepX,-1*\ksepY+\kulV+2.5*\kpsep)--++(-\kpina,0); 
\draw(\ksepX,-1*\ksepY+\kulV+1.5*\kpsep)++(-\kpina,0)++(0.5*\kpsep,-0.5*\kpsep)--++(-\kpsep,\kpsep)--++(\kpsep,\kpsep);
\draw(\ksepX,-1*\ksepY+\kulV+2*\kpsep)++(-0.5*\kpina,0)node[]{$8$};
\draw(\ksepX+\kulH+1.5*\kpsep,-1*\ksepY)++(0,-0.5*\kpsep)--++(0,-\kpin);
\draw(\ksepX+\kulH+2.5*\kpsep,-1*\ksepY)++(0,-0.5*\kpsep)--++(0,-\kpin);
\draw(\ksepX+\kulH+1.5*\kpsep,-1*\ksepY)++(-0.5*\kpsep,-\kpsep)--++(\kpsep,\kpsep)--++(\kpsep,-\kpsep);
\draw(\ksepX+\kulH+2*\kpsep,-1*\ksepY)++(0,-0.5*\kpsep-0.5*\kpin)node[]{$8$};


\draw(\ksepX,-2*\ksepY)  [thick] rectangle ++(\kdimX,\kdimY)node[pos=0.5,rectangle,inner sep=0pt,text width=1cm,align=center]{\RTL{دفتر\\ $B$}};
\draw(\ksepX,-2*\ksepY+\kulV+1.5*\kpsep)++(-0.5*\kpsep,0)--++(-\kpina,0); 
\draw(\ksepX,-2*\ksepY+\kulV+2.5*\kpsep)++(-0.5*\kpsep,0)--++(-\kpina,0); 
\draw(\ksepX,-2*\ksepY+\kulV+1.5*\kpsep)++(-1*\kpsep,-0.5*\kpsep)--++(\kpsep,\kpsep)--++(-\kpsep,\kpsep);
\draw(\ksepX,-2*\ksepY+\kulV+2*\kpsep)++(-0.5*\kpina-0.5*\kpsep,0)node[]{$8$};
\draw(\ksepX+\kdimX,-2*\ksepY+\kulV+2*\kpsep)--++(\kpina,0)node[right]{$CLK$}; 
\draw(\ksepX+\kdimX,-2*\ksepY+\kulV+4*\kpsep)--++(\kpina,0)node[right]{$\overline{L}_B$}; 
\draw(\ksepX+\kdimX,-2*\ksepY+\kulV+2*\kpsep)++(0,-\kmv)--++(-\kmv,\kmv)--++(\kmv,\kmv);
\draw(\ksepX+\kdimX,-2*\ksepY+\kulV+4*\kpsep)++(\knshift,0)node[ocirc]{};

\draw(\ksepX,-3*\ksepY)  [thick] rectangle ++(\kdimX,\kdimY)node[pos=0.5,rectangle,inner sep=0pt,text width=2cm,align=center]{\RTL{خارجی دفتر}};
\draw(\ksepX,-3*\ksepY+\kulV+1.5*\kpsep)++(-0.5*\kpsep,0)--++(-\kpina,0); 
\draw(\ksepX,-3*\ksepY+\kulV+2.5*\kpsep)++(-0.5*\kpsep,0)--++(-\kpina,0); 
\draw(\ksepX,-3*\ksepY+\kulV+1.5*\kpsep)++(-1*\kpsep,-0.5*\kpsep)--++(\kpsep,\kpsep)--++(-\kpsep,\kpsep);
\draw(\ksepX,-3*\ksepY+\kulV+2*\kpsep)++(-0.5*\kpina-0.5*\kpsep,0)node[]{$8$};
\draw(\ksepX+\kdimX,-3*\ksepY+\kulV+2*\kpsep)--++(\kpina,0)node[right]{$CLK$}; 
\draw(\ksepX+\kdimX,-3*\ksepY+\kulV+4*\kpsep)--++(\kpina,0)node[right]{$\overline{L}_O$}; 
\draw(\ksepX+\kdimX,-3*\ksepY+\kulV+2*\kpsep)++(0,-\kmv)--++(-\kmv,\kmv)--++(\kmv,\kmv);
\draw(\ksepX,-3*\ksepY)++(\kulH+1.5*\kpsep,0)--++(0,-\kpin);
\draw(\ksepX,-3*\ksepY)++(\kulH+2.5*\kpsep,0)--++(0,-\kpin);
\draw(\ksepX,-3*\ksepY)++(\kulH+2*\kpsep,0)++(0,-0.5*\kpin)node[]{$8$};
\draw(\ksepX,-3*\ksepY)++(\kulH+1.5*\kpsep,0)++(0,-\kpin)++(-0.5*\kpsep,0.5*\kpsep)--++(\kpsep,-\kpsep)--++(\kpsep,\kpsep);
\draw(\ksepX+\kdimX,-3*\ksepY+\kulV+4*\kpsep)++(\knshift,0)node[ocirc]{};

\draw(\ksepX,-4*\ksepY)  [thick] rectangle ++(\kdimX,\kdimY)node[pos=0.5,rectangle,inner sep=0pt,text width=2cm,align=center]{\RTL{ثنائی نمائشی تختی}};
\end{tikzpicture}
\caption{سادہ ترین کمپیوٹر کی بناوٹ}
\label{شکل_کمپیوٹر_سادہ_ترین}
\end{figure}

سادہ ترین کمپیوٹر کے مختلف حصے واضح کرنے کی غرض سے شکل \حوالہ{شکل_کمپیوٹر_سادہ_ترین}  بنایا گیا ہے۔ اسی لئے تمام    قابو اشارات  ایک ڈبہ جسے \اصطلاح{ قابو مرکز }\فرہنگ{قابو مرکز}\حاشیہب{control unit}\فرہنگ{control unit}کہتے ہیں ، تمام داخلی  اور خارجی ادوار ایک ڈبہ جسے\اصطلاح{ دخول و خروج مرکز }\فرہنگ{دخول و خروج مرکز}\حاشیہب{input-output unit}\فرہنگ{input output unit} کہتے ہیں، وغیرہ،   میں نہیں رکھے گئے ہیں۔

شکل \حوالہ{شکل_کمپیوٹر_سادہ_ترین} میں پیش کئی دفاتر  آپ پہلے سے  جانتے ہیں۔  ہر ڈبے کی مختصر  خصوصیات بیان کرتے ہیں؛ ان پر تفصیلی گفتگو بعد میں کی جائے گی۔

\جزوحصہء{ہدایت گنت کار}
حافظہ  کے شروع میں\اصطلاح{ برنامہ }\فرہنگ{برنامہ}\حاشیہب{program}\فرہنگ{program} (پروگرام)  رکھا جاتا ہے۔ پہلا ہدایت ثنائی پتہ \عددی{0000} پر، دوسرا ہدایت پتہ \عددی{0001}، اور تیسرا ہدایت \عددی{0010} پر ہو گا۔\اصطلاح{ ہدایت گنت کار }\فرہنگ{ہدایت گنت کار}\حاشیہب{program counter}\فرہنگ{program counter}،  جو قابو مرکز کا حصہ ہے، \عددی{0000} تا \عددی{1111} گردان کرتا ہے۔ اس کا کام حافظہ کو وہ پتہ فراہم کرنا ہے جس سے اگلا ہدایت پڑھ کر عمل میں لایا جائے گا۔ یہ کام درج ذیل طریقے سے سرانجام ہو گا۔

کمپیوٹر کی ہر دوڑ  سے قبل ہدایت گنت کار   \عددی{0000}  کر دیا جاتا ہے۔ جب کمپیوٹر کی دوڑ شروع  ہوتی ہے ہدایت گنت کار   حافظہ کو پتہ \عددی{0000} فراہم کرتا ہے۔ اس کے بعد ہدایت گنت کار   ایک قدم بڑھا کر \عددی{0001}  کر دیا جاتا ہے۔ پہلا ہدایت (مقام \عددی{0000} سے) پڑھ کر اس پر عمل کیا جاتا ہے، جس کے بعد ہدایت گنت کار حافظہ کو پتہ \عددی{0001} بھیجتا ہے اور   ہدایت گنت کار ایک قدم بڑھا کر \عددی{0010} کر دیا جاتا ہے۔ دوسرا ہدایت پڑھنے اور اس پر عمل کرنے کے بعد ہدایت گنت کار حافظہ کو \عددی{0010} پتہ بھیجتا ہے۔ اس طرح، ہدایت گنت کار ہر وقت اگلی  ہدایت  پر نظر جمائے رکھتا ہے۔

گویا ہدایت گنت کار اس شخص کی طرح ہے جو ہدایت کی فہرست  کی طرف اشارہ کرتے ہوئے کہتا ہے یہ کام  پہلے کریں، یہ کام دوسرے نمبر پر کریں، یہ تیسرے نمبر پر کریں، وغیرہ۔ اسی لئے ہدایت گنت کار بعض اوقات \اصطلاح{  اشارہ گر   }\فرہنگ{اشارہ گر}\حاشیہب{pointer}\فرہنگ{pointer} کہلاتا ہے؛ یہ حافظہ میں اس مقام کی طرف اشارہ کرتا ہے جہاں کوئی   اہم معلومات درج ہو گی۔

\جزوحصہء{برنامہ نویس   اور  دفتر پتہ}
ہدایت گنت کار کے نیچے برنامہ نویس   اور دفتر پتہ کا  ڈبہ ہے۔ شکل \حوالہ{شکل_کمپیوٹر_برنامہ_نویسی} میں برنامہ نویس پیش ہے (     صفحہ \حوالہصفحہ{شکل_حافظہ_میں_مواد_کی_لکھائی} پر شکل \حوالہ{شکل_حافظہ_میں_مواد_کی_لکھائی} سمجھیں)   جس  کے ذریعہ سوئچوں  کی مدد سے   عارضی حافظہ کو  \عددی{4} پتہ اور \عددی{8} مواد   بِٹ فراہم کر  کے بھرا جاتا ہے۔ یاد رہے کمپیوٹر کی  (با مقصد)دوڑ سے قبل عارضی حافظہ میں برنامہ  لکھنا لازمی ہے۔ یہ دور جو 

حافظے کے پتہ  کا دفتر  (دفتر پتہ  )  اس کمپیوٹر کے عارضی حافظے کا حصہ ہے۔ کمپیوٹر  کی دوڑ کے دوران، ہدایت گنت کار  میں موجود پتہ  اس   (دفتر پتہ)     میں نقل  کیا جاتا ہے۔  دفتر پتہ   چند لمحوں بعد یہ پتہ عارضی  حافظہ کو فراہم کرتا ہے، جہاں سے   اگلی ہدایت پڑھی جاتی ہے۔

\begin{figure}
\centering
 \ctikzset{bipoles/resistor/height=0.15}
 \ctikzset{bipoles/resistor/width=0.4}
  \ctikzset{bipoles/nos/height=0.15}
 \ctikzset{bipoles/nos/width=0.4}
\begin{circuitikz}
\pgfmathsetmacro{\ksepX}{6}
\pgfmathsetmacro{\ksepY}{8}
\pgfmathsetmacro{\kr}{1}
\pgfmathsetmacro{\knshift}{0.07}
\pgfmathsetmacro{\kpsep}{0.50}
\pgfmathsetmacro{\kpsepr}{0.25}
\pgfmathsetmacro{\kpin}{0.5}
\pgfmathsetmacro{\kpina}{2.5}
\pgfmathsetmacro{\kpinb}{\kpsep+2*\kpsepr}%{4}
\pgfmathsetmacro{\kul}{0.50}
\pgfmathsetmacro{\kmv}{0.15}
\pgfmathsetmacro{\kxdim}{2*\kul+3*\kpsep}
\pgfmathsetmacro{\kydim}{2*\kul+5*\kpsep}
\draw[thick](0,0)rectangle++(\kxdim,\kydim)node[pos=0.5,below]{$74189$}node[pos=0.5,above]{\text{\RL{حافظہ}}};
\draw(1*\kul+0*\kpsep,\kydim)node[below]{$A_3$}node[shift={(-1.5ex,2ex)}]{$13$}--++(0,\kpina)to [R]++(0,\kr)coordinate(kpA);
\draw(1*\kul+1*\kpsep,\kydim)node[below]{$A_2$}node[shift={(-1.5ex,2ex)}]{$14$}--++(0,\kpina)to [R]++(0,\kr);
\draw(1*\kul+2*\kpsep,\kydim)node[below]{$A_1$}node[shift={(-1.5ex,2ex)}]{$15$}--++(0,\kpina)to [R]++(0,\kr);
\draw(1*\kul+3*\kpsep,\kydim)node[below]{$A_0$}node[shift={(-1ex,2ex)}]{$1$}--++(0,\kpina)to [R]++(0,\kr)coordinate(kpB);
\draw(kpA)--(kpB)--++(\kpin,0)node[right]{$\SI{+5}{\volt}$};
\draw(1*\kul+0*\kpsep,\kydim)++(0,3*\kpsep+3*\kpsepr)coordinate(kaa)--++(-1*\kpsep,0) to [nos,o-o,invert,mirror]++(-\kpin,0)--++(-0.5*\kpsep,0)coordinate(gndA)node[left]{$a_3$};
\draw(1*\kul+1*\kpsep,\kydim)++(0,3*\kpsep+2*\kpsepr)coordinate(kab)--++(-2*\kpsep,0) to [nos,o-o,invert,mirror]++(-\kpin,0)--++(-0.5*\kpsep,0)node[left]{$a_2$};
\draw(1*\kul+2*\kpsep,\kydim)++(0,3*\kpsep+1*\kpsepr)coordinate(kac)--++(-3*\kpsep,0) to [nos,o-o,invert,mirror]++(-\kpin,0)--++(-0.5*\kpsep,0)node[left]{$a_1$};
\draw(1*\kul+3*\kpsep,\kydim)++(0,3*\kpsep+0*\kpsepr)coordinate(kad)--++(-4*\kpsep,0) to [nos,o-o,invert,mirror]++(-\kpin,0)--++(-0.5*\kpsep,0)coordinate(gndB)node[left]{$a_0$};
\draw(gndA)--(gndB)node [ground]{};
\draw(0,\kul+1*\kpsep)node[right]{$D_4$}node[left,xshift=-0.5ex,yshift=1.5ex]{$4$}--++(-\kpinb-\kpsepr,0)--++(0,-2*\kpsep) to [nos,o-o,invert,mirror]++(0,-\kpin)--++(0,-\kpin)coordinate(kA);%
\draw(0,\kul+2*\kpsep)node[right]{$D_5$}node[left,xshift=-0.5ex,yshift=1.5ex]{$6$}--++(-\kpinb-2*\kpsepr,0)--++(0,-3*\kpsep) to [nos,o-o,invert,mirror]++(0,-\kpin)--++(0,-\kpin);
\draw(0,\kul+3*\kpsep)node[right]{$D_6$}node[left,xshift=-0.5ex,yshift=1.5ex]{$10$}--++(-\kpinb-3*\kpsepr,0)--++(0,-4*\kpsep) to [nos,o-o,invert,mirror]++(0,-\kpin)--++(0,-\kpin);
\draw(0,\kul+4*\kpsep)node[right]{$D_7$}node[left,xshift=-0.5ex,yshift=1.5ex]{$12$}--++(-\kpinb-4*\kpsepr,0)--++(0,-5*\kpsep) to [nos,o-o,invert,mirror]++(0,-\kpin)node[above left]{$d_7$}--++(0,-\kpin)coordinate(kB);
\draw(kA)--(kB)node[ground]{};
\draw(-\kpsep-1*\kpsepr,\kul+0*\kpsep)--++(0,4*\kpsep+\kul)to [R]++(0,\kr)coordinate(kpC);
\draw(-\kpsep-3*\kpsepr,\kul+1*\kpsep)--++(0,3*\kpsep+\kul)to [R]++(0,\kr);
\draw(-\kpsep-4*\kpsepr,\kul+2*\kpsep)--++(0,2*\kpsep+\kul)to [R]++(0,\kr);
\draw(-\kpsep-5*\kpsepr,\kul+3*\kpsep)--++(0,1*\kpsep+\kul)to [R]++(0,\kr);
\draw(-\kpsep-6*\kpsepr,\kul+4*\kpsep)--++(0,0*\kpsep+\kul)to [R]++(0,\kr)coordinate(kpD);
\draw(kpC)--(kpD)--++(-0.5*\kpin,0)node[left]{$\SI{5}{\volt}$};
\draw(0,\kul+0*\kpsep)node[right]{$\overline{\text{\RL{لکھ}}}$}node[left,xshift=-0.5ex,yshift=1.5ex]{$3$}--++(-\kpinb+\kpsepr,0)--++(0,-5.5*\kpsep)coordinate(kkWW) to [nos,o-o,invert,mirror]++(0,-\kpin)node[right]{$S_1$}--++(0,-\kpin)node[ground]{};
\draw(0,\kul+0*\kpsep)++(-\knshift,0)node[ocirc]{};
\draw(1.5*\kul,0)node[below right]{$16$}--++(0,-\kpin)--++(-1*\kpin,0)node[left]{$\SI{5}{\volt}$} (1.5*\kul+\kpsep,0)node[below right]{$8$}--++(0,-\kpin)node[ground]{};
\draw(1*\kul+3*\kpsep,0)node[below right]{$2$}node[above]{$\overline{\text{\RL{بیدار}}}$}
--++(0,-0.5*\kpin)coordinate(kCE);
\draw(1*\kul+3*\kpsep,0)++(0,-\knshift)node[ocirc]{};
\draw(\kxdim,\kul+1*\kpsep)node[right,xshift=0.5ex,yshift=1.5ex]{$5$}--++(1.5*\kpin,0)node[right]{$\overline{O}_4$};
\draw(\kxdim,\kul+2*\kpsep)node[right,xshift=0.5ex,yshift=1.5ex]{$7$}--++(1.5*\kpin,0)node[right]{$\overline{O}_5$};
\draw(\kxdim,\kul+3*\kpsep)node[right,xshift=0.5ex,yshift=1.5ex]{$9$}--++(1.5*\kpin,0)node[right]{$\overline{O}_6$};
\draw(\kxdim,\kul+4*\kpsep)node[right,xshift=0.5ex,yshift=1.5ex]{$11$}--++(1.5*\kpin,0)node[right]{$\overline{O}_7$};
\draw(\kxdim,\kul+1*\kpsep)++(\knshift,0)node[ocirc]{};
\draw(\kxdim,\kul+2*\kpsep)++(\knshift,0)node[ocirc]{};
\draw(\kxdim,\kul+3*\kpsep)++(\knshift,0)node[ocirc]{};
\draw(\kxdim,\kul+4*\kpsep)++(\knshift,0)node[ocirc]{};

\draw[thick](\ksepX,0*\ksepY)rectangle++(\kxdim,\kydim)node[pos=0.5,below]{$74189$}node[pos=0.5,above]{\text{\RL{حافظہ}}};
\draw(1*\kul+0*\kpsep+\ksepX,\kydim+0*\ksepY)node[below]{$A_3$}node[shift={(-1.5ex,2ex)}]{$13$}|-(kaa)node[draw,circle,fill=black,inner sep=1.5pt]{};
\draw(1*\kul+1*\kpsep+\ksepX,\kydim+0*\ksepY)node[below]{$A_2$}node[shift={(-1.5ex,2ex)}]{$14$}|-(kab)node[draw,circle,fill=black,inner sep=1.5pt]{};
\draw(1*\kul+2*\kpsep+\ksepX,\kydim+0*\ksepY)node[below]{$A_1$}node[shift={(-1.5ex,2ex)}]{$15$}|-(kac)node[draw,circle,fill=black,inner sep=1.5pt]{};
\draw(1*\kul+3*\kpsep+\ksepX,\kydim+0*\ksepY)node[below]{$A_0$}node[shift={(-1ex,2ex)}]{$1$}|-(kad)node[draw,circle,fill=black,inner sep=1.5pt]{};
\draw(\ksepX,\kul+1*\kpsep+0*\ksepY)node[right]{$D_0$}node[left,xshift=-0.5ex,yshift=1.5ex]{$4$}--++(-\kpinb+\kpsepr,0)--++(0,-2*\kpsep) to [nos,o-o,invert,mirror]++(0,-\kpin)--++(0,-\kpin)coordinate(kA);%
\draw(\ksepX,\kul+2*\kpsep+0*\ksepY)node[right]{$D_1$}node[left,xshift=-0.5ex,yshift=1.5ex]{$6$}--++(-\kpinb-0*\kpsepr,0)--++(0,-3*\kpsep) to [nos,o-o,invert,mirror]++(0,-\kpin)--++(0,-\kpin);
\draw(\ksepX,\kul+3*\kpsep+0*\ksepY)node[right]{$D_2$}node[left,xshift=-0.5ex,yshift=1.5ex]{$10$}--++(-\kpinb-1*\kpsepr,0)--++(0,-4*\kpsep) to [nos,o-o,invert,mirror]++(0,-\kpin)--++(0,-\kpin);
\draw(\ksepX,\kul+4*\kpsep+0*\ksepY)node[right]{$D_3$}node[left,xshift=-0.5ex,yshift=1.5ex]{$12$}--++(-\kpinb-2*\kpsepr,0)--++(0,-5*\kpsep) to [nos,o-o,invert,mirror]++(0,-\kpin)node[above left]{$d_3$}--++(0,-\kpin)coordinate(kB);
\draw(\ksepX-\kpsep-1*\kpsepr,\kul+1*\kpsep+0*\ksepY)--++(0,3*\kpsep+\kul)to [R]++(0,\kr)coordinate(kpC);
\draw(\ksepX-\kpsep-2*\kpsepr,\kul+2*\kpsep+0*\ksepY)--++(0,2*\kpsep+\kul)to [R]++(0,\kr);
\draw(\ksepX-\kpsep-3*\kpsepr,\kul+3*\kpsep+0*\ksepY)--++(0,1*\kpsep+\kul)to [R]++(0,\kr);
\draw(\ksepX-\kpsep-4*\kpsepr,\kul+4*\kpsep+0*\ksepY)--++(0,0*\kpsep+\kul)to [R]++(0,\kr)coordinate(kpD);
\draw(kpC)--(kpD)--++(-0.5*\kpin,0)node[left]{$\SI{5}{\volt}$};
\draw(\ksepX,\kul+0*\kpsep+0*\ksepY)node[right]{$\overline{\text{\RL{لکھ}}}$}node[left,xshift=-0.5ex,yshift=1.5ex]{$3$}coordinate(kwr);%--++(-\kpinb+\kpsepr,0)--++(0,-3*\kpsep) to [nos,o-o,invert,mirror]++(0,-\kpin)node[right]{$S_1$}--++(0,-\kpin)coordinate(kA);
\draw(kwr)--++(-\kpin,0)|-($(kkWW)+(0,0.3)$);
\draw(\ksepX,\kul+0*\kpsep+0*\ksepY)++(-\knshift,0)node[ocirc]{};
\draw(kA)--(kB)node[ground]{};
\draw(1.5*\kul+\ksepX,0*\ksepY)node[below right]{$16$}--++(0,-\kpin)--++(-0.5*\kpin,0)node[left]{$\SI{5}{\volt}$} (1.5*\kul+\kpsep+\ksepX,0)node[below right]{$8$}--++(0,-\kpin)node[ground]{};
\draw(1*\kul+3*\kpsep+\ksepX,0*\ksepY)node[below right]{$2$}node[above]{$\overline{\text{\RL{بیدار}}}$}
--++(0,-5*\kpin)node[spdt,xscale=-1,rotate=-90,anchor=in](sw){};
\draw(1*\kul+3*\kpsep+\ksepX,0*\ksepY)++(0,-\knshift)node[ocirc]{};
\draw(sw.out 1)node[ground]{} (sw.out 2)node[above right,yshift=3ex]{$S_2$}--++(2*\kpin,0)node[right]{$\overline{E}_R$};
\draw(sw.out 1)node[left]{\text{\RL{برنامہ نویسی}}};
\draw(sw.out 2)node[below right]{\text{\RL{دوڑ}}};
\draw(sw.in)++(0,0.2)-|(kCE);
\draw(\kxdim+\ksepX,\kul+1*\kpsep+0*\ksepY)node[right,xshift=0.5ex,yshift=1.5ex]{$5$}--++(1.5*\kpin,0)node[right]{$\overline{O}_0$};
\draw(\kxdim+\ksepX,\kul+2*\kpsep+0*\ksepY)node[right,xshift=0.5ex,yshift=1.5ex]{$7$}--++(1.5*\kpin,0)node[right]{$\overline{O}_1$};
\draw(\kxdim+\ksepX,\kul+3*\kpsep+0*\ksepY)node[right,xshift=0.5ex,yshift=1.5ex]{$9$}--++(1.5*\kpin,0)node[right]{$\overline{O}_2$};
\draw(\kxdim+\ksepX,\kul+4*\kpsep+0*\ksepY)node[right,xshift=0.5ex,yshift=1.5ex]{$11$}--++(1.5*\kpin,0)node[right]{$\overline{O}_3$};
\draw(\kxdim+\ksepX,\kul+1*\kpsep+0*\ksepY)++(\knshift,0)node[ocirc]{};
\draw(\kxdim+\ksepX,\kul+2*\kpsep+0*\ksepY)++(\knshift,0)node[ocirc]{};
\draw(\kxdim+\ksepX,\kul+3*\kpsep+0*\ksepY)++(\knshift,0)node[ocirc]{};
\draw(\kxdim+\ksepX,\kul+4*\kpsep+0*\ksepY)++(\knshift,0)node[ocirc]{};
\end{circuitikz}
\caption{برنامہ نویس}
\label{شکل_کمپیوٹر_برنامہ_نویسی}
\end{figure}

\جزوحصہء{عارضی حافظہ}
کمپیوٹر کی دوڑ سے قبل   \عددی{16\times 8} عارضی  حافظہ  میں  ہدایت اور درکار مواد لکھا جاتا ہے۔ کمپیوٹر کی دوڑ کے دوران، حافظہ کو دفتر پتہ \عددی{4} بِٹ پتہ فراہم کرتا ہے ؛    جہاں سے ہدایت یا مواد  پڑھ  کر \عددی{W} گزرگاہ پر رکھ دیا جاتا ہے جسے  کمپیوٹر کا کوئی دوسرا حصے استعمال کر سکتا ہے۔عارضی حافظہ کے مخارج \عددی{\overline{O}_0} تا \عددی{\overline{O}_7} آٹھ برقی تاروں کے ذریعہ کمپیوٹر کے  باقی  حصوں کے ساتھ جڑا ہے۔ ان آٹھ تاروں کو  \عددی{W} گزرگاہ کہتے ہیں۔

\جزوحصہء{دفتر ہدایت}
 قابو مرکز کا ایک حصہ  \اصطلاح{دفتر ہدایت  }\فرہنگ{دفتر ہدایت}\حاشیہب{instruction register}\فرہنگ{instruction register} ہے۔حافظہ سے ہدایت پڑھنے کی خاطر کمپیوٹر   جو عمل  سرانجام دیتا ہے اس کو \اصطلاح{ہدایت پڑھ عمل }\فرہنگ{ہدایت پڑھ عمل}\حاشیہب{memory read operation}\فرہنگ{operation!memory read} کہتے ہیں۔  حافظہ کے   مخاطب  مقام   پر موجود ہدایت (یا مواد) کو  یہ عمل \عددی{W} گزرگاہ پر رکھتا ہے۔ ساتھ ہی   ساعت کے اگلے مثبت کنارے پر  دفتر  ہدایت بھرائی کے لئے تیار کر دیا جاتا ہے۔
 
 دفتر ہدایت    میں موجود معلومات کو دو حصوں میں تقسیم کیا جاتا ہے۔  نچلے   (زیریں) چار بِٹ سہ حالی مخارج ہے جو بوقت ضرورت \عددی{W} گزرگاہ پر ڈال دیا جاتا ہے جبکہ   بالا چار بِٹ  دو حالی مخارج ہے جو سیدھا  قابو و ترتیب کار  کو مہیا کیا جاتا ہے۔
 
 \جزوحصہء{قابو و ترتیب کار}
 کمپیوٹر کی ہر دوڑ سے قبل  ہدایت گنت کار کو \عددی{\overline{CLR}} اور دفتر   ہدایت کو \عددی{CLR}  اشارہ بھیجا جاتا ہے ، جو ہدایت گنت کار   \عددی{0000}     کرتا ہے اور دفتر ہدایت   میں موجود ہدایت  زائل  کرتا  ہے۔

تمام مستحکم کار دفاتر کو ساعتی اشارہ \عددی{CLK} بھیجا جاتا ہے جو کمپیوٹر کے  مختلف اعمال   ہم قدم    کرتے ہوئے یقینی بناتا ہے کہ سب کچھ اپنے اپنے  وقت پر  ہو۔ دوسرے لفظوں میں،   دفاتر کے مابین معلومات کا تبادلہ مشترک ساعت \عددی{CLK} کے مثبت کنارے پر ہو۔ دھیان رہے، ہدایت گنت کار کو \عددی{\overline{CLK}} اشارہ بھی فراہم کیا گیا ہے۔

قابو و ترتیب کار \عددی{12} بِٹ  لفظ خارج کرتا ہے جو باقی کمپیوٹر کو قابو کرتا ہے۔ وہ \عددی{12} برقی تار جن  پر یہ لفظ ترسیل ہوتا ہے \اصطلاح{ قابو گزرگاہ }\فرہنگ{گزرگاہ!قابو}\حاشیہب{control bus}\فرہنگ{bus!control} کہلاتا ہے۔

بارہ بِٹ قابو  لفظ   درج ذیل ہے۔
\begin{align*}
\text{\RL{قابو}}=C_PE_P\overline{L}_M\overline{CE}\,\,\, \overline{L}_I\overline{E}_I\overline{L}_AE_A\,\,\,S_UE_U\overline{L}_B\overline{L}_O
\end{align*}

 ساعت \عددی{CLK} کے  اگلے   مثبت کنارے پر  دفاتر کا عمل اس لفظ کے تحت ہو گا۔ مثلاً، بلند \عددی{E_P} اور پست \عددی{\overline{L}_M} کی صورت میں ساعت کے اگلے مثبت کنارے پر ہدایت گنت کار کی معلومات   دفتر پتہ میں نقل ہو گی۔ اسی طرح، پست \عددی{\overline{CE}} اور پست \عددی{\overline{L}_A} کی صورت میں  ساعت کے اگلے مثبت کنارے پر  دفتر \عددی{A}   میں عارضی حافظہ کا مخاطب   لفظ نقل ہو گا۔انتقال    مواد  کی وقتیہ  ترسیمات پر غور  ( جس سے ہم جان پائیں گے   یہ انتقال  کیسے اور کب ہو ں گے) بعد میں کیا جائے گا ۔

\جزوحصہء{دفتر \عددی{A}}
 کمپیوٹر کی دوڑ کے دوران   حاصل نتائج دفتر \عددی{A} میں ذخیرہ کیے جاتے ہیں۔ شکل \حوالہ{شکل_کمپیوٹر_سادہ_ترین} میں \عددی{A} کے دو مخارج  دکھائے گئے ہیں۔اس کا دو حالی مخارج   سیدھا جمع و منفی کار کو جاتا ہے۔ تین حالی مخارج \عددی{W} گزرگاہ کو جاتا ہے۔ یوں \عددی{A} کا آٹھ بِٹ لفظ  جمع و منفی کار کو    مسلسل فراہم ہو گا؛ یہی لفظ  بلند \عددی{E_A} کی صورت میں \عددی{W} گزرگاہ  پر بھی ڈالا جائے گا۔
 
 \جزوحصہء{جمع و منفی کار}
 یہاں  تکملہ  \عددی{2} کا   جمع و منفی کار مستعمل ہے۔ پست \عددی{S_U}  کی صورت میں شکل \حوالہ{شکل_کمپیوٹر_سادہ_ترین} میں جمع و منفی کار کا مخارج درج ذیل ہو گا۔
 \begin{align*}
 S=A+B
 \end{align*}
 بلند \عددی{S_U} کی صورت میں  جمع و منفی کار درج ذیل دیگا جہاں \عددی{B'} سے مراد \عددی{B} کا  اساس \عددی{2} تکملہ ہے۔(یاد رہے، \عددی{2} کا تکملہ علامت تبدیل کرنے کے مترادف ہے۔)
  \begin{align*}
 S=A+B'
 \end{align*}
 جمع و منفی کار غیر معاصر ہے (یعنی اس کی کارکردگی ساعت پر منحصر نہیں)؛  یوں   جیسے ہی داخلی الفاظ تبدیل ہوں، اس کا مخارج تبدیل ہو گا۔ بلند \عددی{E_U} کی صورت میں یہ مخارج \عددی{W} گزرگاہ پر ڈالا جائے گا۔
 
\جزوحصہء{دفتر \عددی{B}}
دفتر \عددی{B} حسابی   اعمال میں  استعمال کیا جاتا ہے۔ پست \عددی{\overline{L}_B} کی صورت میں ساعت کے مثبت کنارے  پر \عددی{W} گزرگاہ پر موجود لفظ \عددی{B} میں نقل ہو گا۔ دفتر \عددی{B} کا دو حالی مخارج مسلسل جمع و منفی کار کو فراہم کیا جاتا ہے ۔ یہ عدد \عددی{A} میں موجود عدد کے ساتھ جمع یا اس سے منفی ہو گا۔

\جزوحصہء{خارجی دفتر}
کسی بھی مسئلے کو حل کرنے کے بعد حاصل نتیجہ دفتر \عددی{A} میں     ہو گا۔ یہ نتیجہ بیرونی دنیا کو بتانا مقصود ہو گا۔یہ کام \اصطلاح{ خارجی دفتر }\فرہنگ{خارجی دفتر}\حاشیہب{output register}\فرہنگ{register!output} کے سپرد ہے۔ بلند \عددی{E_A} اور پست \عددی{\overline{L}_O} کی صورت میں ساعت کے  اگلے  مثبت کنارے پر \عددی{A} میں موجود معلومات خارجی دفتر میں نقل کی جاتی ہے۔

چونکہ خارجی دفتر کے ذریعہ    مواد کمپیوٹر سے باہر منتقل ہوتا ہے لہٰذا اسے عموماً\اصطلاح{ خارجی  روزن }\فرہنگ{خارجی روزن}\حاشیہب{output port}\فرہنگ{port!output} بھی  کہتے ہیں۔ خارجی روزن  \اصطلاح{ ملاپی ادوار }\فرہنگ{دور!ملاپی}\حاشیہب{interface circuits}\فرہنگ{interface circuit} سے منسلک ہو گا جو بیرونی آلات  مثلاً  \اصطلاح{      پرنٹر}\فرہنگ{پرنٹر}\حاشیہب{printer}\فرہنگ{printer}،   سات کلی نمائشی تختی،  کمپیوٹر کا شیشہ، وغیرہ چلاتے ہیں۔

\جزوحصہء{ثنائی نمائشی تختی}
ثنائی نمائشی تختی  آٹھ \اصطلاح{  نوری ڈایوڈ  }\فرہنگ{نوری ڈایوڈ}\حاشیہب{LED}\فرہنگ{LED} پر مبنی ہے۔خارجی  روزن کے ہر بِٹ کے ساتھ ایک نوری ڈایوڈ منسلک ہے۔ یوں ثنائی نمائشی تختی  پر خارجی دفتر میں موجود  معلومات    ثنائی روپ میں نظر آئے گی۔

\جزوحصہء{خلاصہ}
اس کمپیوٹر کا قابو مرکز ہدایت گنت کار، ہدایت دفتر، اور قابو و ترتیب کار  (جو قابو لفظ، ساعت \عددی{CLK}، اور زائل  اشارہ \عددی{CLR}  پیدا کرتا ہے) پر مشتمل ہے۔ کمپیوٹر کا\اصطلاح{ حسابی  مرکز }\فرہنگ{حسابی  مرکز}\حاشیہب{arithmetic logic unit, ALU}\فرہنگ{ALU}  دفتر \عددی{A}، دفتر \عددی{B}، اور جمع و منفی کار پر مشتمل ہے۔کمپیوٹر کا حافظہ  دفتر پتہ اور \عددی{16\times 8} عارضی حافظہ پر مشتمل ہے۔  درآمدی  سوئچ، خارجی روزن، اور ثنائی نمائشی تختی  مل کر   دخول و خروج مرکز دیتے ہیں۔

\حصہ{ہدایات کی فہرست} 
کمپیوٹر کی با مقصد دوڑ سے قبل اس کے حافظہ  میں ہدایات قدم با قدم  بھرنا لازم ہے۔البتہ، ایسا کرنے سے پہلے  آپ کو یہ ہدایات  جاننی  ہو گی۔ان ہدایات سے مراد وہ اعمال ہیں جو یہ کمپیوٹر سرانجام دے سکتا ہے۔ اس کمپیوٹر کی ہدایات کی فہرست  پر اب غور کرتے ہیں۔ہدایت کا مجموعہ کمپیوٹر کی \اصطلاح{ مادری زبان }\فرہنگ{زبان!مادری}\فرہنگ{مادری زبان}\حاشیہب{ assembly language}\فرہنگ{language!assembly} کہلاتی ہے۔

\جزوحصہء{نقل الف}
حافظہ کے مقام \عددی{0000_2} پر موجود معلومات کو ہم  \عددی{R_0} کہتے ہیں، مقام \عددی{0001_2} پر  \عددی{R_1} ہو گا، وغیرہ ۔ یوں \عددی{R_0} مقام \عددی{0H} پر محفوظ ہے ، \عددی{R_1} پتہ \عددی{1H} پر، \عددی{R_2} پتہ \عددی{2H} پر، وغیرہ، جہاں \عددی{0H} سے مراد   \عددی{0_{16}} ہے۔اساس \عددی{16} اعداد کے آخر میں  زیرنوشت \عددی{16} لکھنے کی بجائے ہم عدد کے آخر میں \عددی{H} لکھتے ہیں۔

\موٹا{نقل الف  } اس کمپیوٹر کی ایک ہدایت ہے جو  کہتی ہے دفتر ا  الف میں مواد نقل کریں۔ پوری ہدایت میں  اس مواد کا اساس سولہ پتہ بھی دیا جاتا ہے جو دفتر \عددی{A} میں بھرا جائے گا، لہٰذا مکمل ہدایت درج ذیل ہے جو جدول \حوالہ{جدول_کمپیوٹر_ہدایات} میں  پیش ہے۔
\begin{align*}
\text{\RL{نقل  الف \فاصلہء   پتہ}}
\end{align*}
 یوں\قول{ نقل الف \عددی{8H}  } کہتی ہے کہ عارضی حافظہ کے پتہ \عددی{8H} پر درج معلومات کو دفتر \عددی{A} میں نقل کریں۔ اس ہدایت پر عمل کرنے کے بعد دفتر \عددی{A} میں  اور حافظہ کے مقام  \عددی{8H} پر ایک جیسا مواد  پایا جائے گا۔ یوں  درج ذیل صورت میں
\begin{align*}
R_8=1111\,0000
\end{align*}
جو کہتی ہے  مقام \عددی{R_8} پر  ثنائی معلومات  \عددی{1111\,0000} محفوظ ہے،  ذیل ہدایت
\begin{align*}
8H \quad \text{\RL{نقل الف }}
\end{align*}
پر عمل کرنے کے بعد درج ذیل ہو گا۔
\begin{align*}
A=1111\,0000
\end{align*}
آپ نے دیکھا  یہ  ہدایت دفتر \عددی{A} میں معلومات نقل کرتے ہوئے  حافظہ میں درج معلومات پر   اثر انداز نہیں ہوتی۔

اسی طرح \قول{نقل الف \عددی{AH}}  مقام \عددی{10_{10}} سے دفتر \عددی{A} میں معلومات  نقل کرے گی، اور \قول{نقل الف \عددی{FH}} مقام \عددی{F_{16}} سے معلومات دفتر \عددی{A} میں نقل کرے گی۔

\جزوحصہء{جمع}
کمپیوٹر کی یہ ہدایت دو اعداد جمع کرنے کو کہتی  ہے۔پہلا عدد دفتر \عددی{A} میں ہو گا جبکہ دوسرے عدد کا پتہ مکمل ہدایت میں شامل ہو گا؛ نتیجہ دفتر \عددی{A} میں محفوظ ہو گا، لہٰذا دفتر \عددی{A} میں پہلے سے موجود مواد زائل ہو گا۔یوں   اگر دفتر \عددی{A} میں \عددی{2_{10}} اور حافظہ کے مقام \عددی{9H} پر \عددی{3_{10}} ہو:
\begin{align*}
A&=0000\,0010\\
R_9&=0000\,0011
\end{align*}
تب ذیل ہدایت
\begin{align*}
9H\quad \text{\RL{جمع}}
\end{align*} 
پر عمل کرنے کے لئے درج ذیل اقدام پر عمل کرنا ہو گا۔ پہلے قدم پر،  دفتر  \عددی{B} میں \عددی{R_9} ڈالا جائے گا:
\begin{align*}
B=0000\,0011
\end{align*}
جس کے فوراً بعد  جمع و منفی کار   الف اور ب کا مجموعہ
\begin{align*}
\text{مجموعہ}=0000\,0101
\end{align*}
معلوم  کرتا ہے۔دوسرے قدم پر،   یہ مجموعہ دفتر \عددی{A} میں ڈالا جاتا ہے۔
\begin{align*}
A&=0000\,0101
\end{align*}

جب  بھی \قول{جمع}  کی ہدایت پر عمل کیا جائے درج بالا اقدام اٹھانے ہوں گے؛ دیے گئے پتہ سے مواد دفتر  \عددی{B}  میں ڈال کر جمع  و منفی کار  سے مجموعہ حاصل کرنے کے بعد نتیجہ دفتر \عددی{A} میں ڈالا جاتا ہے۔چونکہ   دفتر \عددی{A} میں پہلے سے موجود مواد  کے اوپر  نیا مواد (حاصل جمع) لکھا جاتا ہے لہٰذا  دفتر \عددی{A} کا پرانا  مواد زائل ہو گا۔اسی طرح چونکہ دفتر  \عددی{B} میں دیے گئے پتے کا مواد ڈالا  کیا جاتا ہے لہٰذا دفتر  \عددی{B}  کا پرانا مواد بھی زائل ہو گا۔ اس طرح  \قول{جمع \عددی{9H}} پر عمل کرنے سے دفتر \عددی{A} کا مواد اور \عددی{R_9} کا مجموعہ دفتر \عددی{A} میں  حاصل ہو گا۔ \قول{جمع \عددی{FH}} پر عمل کے بعد دفتر \عددی{A} میں \عددی{R_F} اور دفتر \عددی{A} کا مجموعہ پایا جائے گا۔

\begin{table}
\caption{کمپیوٹر کی مادری زبان کی ہدایات}
\label{جدول_کمپیوٹر_ہدایات}
\centering
\begin{tabular}{r|r}
\toprule
\multicolumn{1}{c|}{ہدایت}& \multicolumn{1}{c}{عمل}\\
\midrule
نقل الف \فاصلہء پتہ& دفتر \عددی{A} میں حافظہ سے مواد نقل کریں\\
جمع \فاصلہء پتہ& دفتر \عددی{A} کے ساتھ حافظہ کا مواد جمع کریں\\
منفی\فاصلہء پتہ&دفتر \عددی{A} سے حافظہ کا مواد منفی کریں\\
برآمد&دفتر \عددی{A} کا مواد  ر خارجی  دفتر میں ڈالیں\\
رک& کام کرنا روک دیں\\
\bottomrule
\end{tabular}
\end{table}

\جزوحصہء{منفی}
دو اعداد منفی کرنے کے لئے کمپیوٹر کی ہدایت \اصطلاح{منفی} ہے جو دفتر \عددی{A} میں موجود عدد سے  دیا گیا عدد منفی کر کے نتیجہ دفتر \عددی{A} میں دے گی۔ مکمل ہدایت میں منفی ہونے والے عدد کے مقام کا پتہ بھی شامل ہو گا۔
\begin{align*}
\text{\RL{منفی\فاصلہء پتہ}}
\end{align*}
یوں \قول{منفی \عددی{CH}} کا مطلب ہے دفتر \عددی{A} میں موجود مواد سے حافظہ  کے مقام \عددی{CH} پر موجود مواد \عددی{R_C} منفی کر کے نتیجہ دفتر \عددی{A} میں ڈالیں۔

مثال کی خاطر فرض کریں دفتر \عددی{A} میں  اعشاری \عددی{7} اور  حافظہ کے مقام \عددی{CH} پر اعشاری \عددی{3} پایا جاتا ہے۔
\begin{align*}
A&=0000\,0111\\
R_C&=0000\,0011
\end{align*}
\قول{منفی\عددی{CH}} پر عمل درج ذیل اقدام اٹھانے سے ہو گا۔ پہلے قدم پر،  دفتر \عددی{B}  میں \عددی{R_C}  ڈالا  کیا جاتا ہے: 
\begin{align*}
B=0000\,0011
\end{align*}
جس کے فوراً بعد جمع و منفی کار دفتر \عددی{A} اور ب کا فرق:
\begin{align*}
\text{فرق}=0000\,0100
\end{align*}
  معلوم کرتا ہے۔دوسرے قدم پر یہ فرق  دفتر \عددی{A} میں  ڈالا جاتا ہے۔
\begin{align*}
A=0000\,0100
\end{align*}

منفی کی  تمام ہدایت  پر عمل درج بالا اقدام کے ذریعہ ہو گا؛ دیے گئے پتہ پر موجود مواد حافظہ سے دفتر \عددی{B}  میں ڈال کر جمع و منفی کار کو مہیا کیا جاتا ہے جو فوراً ان کا فرق معلوم کرتا ہے۔ یہ فرق دفتر \عددی{A} میں ڈالا جاتا ہے۔ یوں \قول{منفی \عددی{CH}} پر عمل کرتے ہوئے \عددی{R_C} کو دفتر \عددی{A} سے منفی کر کے نتیجہ دفتر \عددی{A} میں ڈالا جائے گا۔ \قول{منفی \عددی{EH}} مقام \عددی{EH}  پر موجود مواد \عددی{R_E} کو دفتر \عددی{A} سے منفی کر کے نتیجہ دفتر \عددی{A} میں ڈالتا ہے۔

\جزوحصہء{برآمد}
کمپیوٹر کی ہدایت  \اصطلاح{برآمد}  کہتی ہے دفتر \عددی{A} کا مواد خارجی دفتر میں ڈالیں۔اس ہدایت پر عمل کرنے کے بعد دفتر \عددی{A} کا مواد کمپیوٹر سے باہر دستیاب ہو گا جہاں سے آپ نتیجہ دیکھ سکتے ہیں۔

اس ہدایت پر عمل کرنے کے لئے  حافظہ سے رجوع کرنے کی ضرورت نہیں لہٰذا اس ہدایت میں پتہ درکار نہیں ہے۔

\جزوحصہء{رک}
یہ ہدایت  ، جو برنامے کی آخری ہدایت ہو گی، کمپیوٹر کو  مزید ہدایات پر عمل کرنے ے روکتی ہے۔یہ ہدایت،  جملہ مکمل ہونے کے بعد    (جملے کے آخر میں) \اصطلاح{  ختمہ  }\فرہنگ{ختمہ}\حاشیہب{fullstop}\فرہنگ{fullstop} کے مترادف ہے۔ ہر برنامے کے آخر میں یہ ہدایت ضروری ہے ؛  ورنہ کمپیوٹر  بے باق   دوڑتا رہے گا  اور بے مقصد (اور غلط) نتائج فراہم کرتا رہے گا۔

رک کی ہدایت از خود مکمل ہے۔ اس پر عمل کرنے کی خاطر حافظہ سے رجوع کرنے کی ضرورت نہیں لہٰذا اس ہدایت میں پتے کی شمولیت نہیں ہو گی۔

\جزوحصہء{حافظہ سے رجوع کرنے والے  راجع  ہدایات}
نقل الف، جمع، اور منفی کی ہدایات حافظہ سے رجوع کرنی ہیں لہٰذا یہ\اصطلاح{  راجع ہدایات }\فرہنگ{راجع ہدایات}\حاشیہب{memory-reference instructions}\فرہنگ{memory reference instructions} کہلاتی ہیں۔ اس کے برعکس برآمد اور  رک حافظہ سے رجوع نہیں کرتی ہیں لہٰذا یہ ہدایات غیر راجع  ہیں۔

\جزوحصہء{\عددی{8080} اور \عددی{8085}}
وسیع  پیمانے پر استعمال ہونے والا  پہلا \اصطلاح{   خرد عامل  کار  }\فرہنگ{خرد عامل کار}\حاشیہب{microprocessor}\فرہنگ{microprocessor} (مائکروپراسیسر ) \عددی{8080}  تھا۔ اس کی کل \عددی{72} ہدایات ہیں۔اس خرد عامل کار \عددی{8085} ہے  جو انہیں ہدایات پر چلتا ہے۔  اس باب کے  سادہ ترین کمپیوٹر کو حقیقتاً قابل استعمال بنانے کی غرض سے ہم  اس کی ہدایات کو \عددی{8080/8085} کی ہدایت کے  ہم آہنگ بناتے ہیں۔ دوسرے لفظوں میں  نقل، جمع، منفی، برآمد، اور رک \عددی{8080/8085} کے بھی ہدایات ہیں۔

\ابتدا{مثال}\شناخت{مثال_کمپیوٹر_برنامہ_الف}
سادہ ترین کمپیوٹر کا ایک برنامہ پیش ہے۔
\begin{center}
\begin{tabular}{rr}
پتہ& ہدایات\\[0.5ex]
0H& نقل  \عددی{9H}\\
1H&جمع \عددی{AH}\\
2H&جمع \عددی{BH}\\
3H&منفی \عددی{CH}\\
4H&برآمد\\
5H&رک
\end{tabular}
\end{center}
حافظہ میں برنامہ سے اوپر درج ذیل مواد پایا جاتا ہے۔
\begin{center}
\begin{tabular}{rr}
پتہ& مواد\\[0.5ex]
6H&FFH\\
7H&FFH\\
8H&FFH\\
9H&01H\\
AH&02H\\
BH&03H\\
CH&04H\\
DH&FFH\\
EH&FFH\\
FH&FFH
\end{tabular}
\end{center}
یہ ہدایات  کیا کریں گے؟

حل:\quad
برنامہ نچلے حافظہ میں \عددی{0H} تا \عددی{5H} مقامات پر رکھا گیا ہے۔ پہلی ہدایت حافظہ کے مقام \عددی{9H} سے مواد \عددی{01H} دفتر \عددی{A} میں نقل کرتی ہے۔
\begin{align*}
A=01H
\end{align*}
دوسری ہدایت مقام \عددی{AH}  کا مواد دفتر \عددی{A} کے ساتھ جمع کر کے نتیجہ دفتر \عددی{A} میں ڈالتی ہے۔
\begin{align*}
A=01H+02H=03H
\end{align*}
تیسری ہدایت حافظہ کے مقام \عددی{BH} کے مواد کو دفتر \عددی{A} (جس میں اس وقت \عددی{03H} موجود ہے) کے ساتھ جمع کر کے نتیجہ دفتر \عددی{A} منتقل کرتی ہے۔
\begin{align*}
A=03H+03H=06H
\end{align*}
چوتھی ہدایت مقام \عددی{CH} کے مواد کو دفتر \عددی{A} سے منفی کرکے نتیجہ دفتر \عددی{A} میں ڈالتی ہے۔
\begin{align*}
A=06H-04H=02H
\end{align*}
پانچویں ہدایت دفتر \عددی{A} کے مواد کو خارجی دفتر میں منتقل کرتی ہے۔خارجی دفتر کے ساتھ ثنائی نمائشی تختی   منسلک ہے جس پر یہ مواد ثنائی روپ میں نظر آئے گا۔یوں  نوری ڈایوڈ درج ذیل دکھائیں گے۔
\begin{align*}
0000\,0010
\end{align*}
آخری ہدایت   \اصطلاح{رک}  ہے جو کمپیوٹر کر  کو مزید ہدایات پر عمل کرنے سے روکتی ہے۔
\انتہا{مثال}

\حصہ{کمپیوٹر  کی برنامہ  نویسی}
کمپیوٹر کے  حافظہ میں ہدایات اور مواد بھرنے کے لئے ہمیں ایسی زبان استعمال کرنی ہو گی جو کمپیوٹر  سمجھ سکے۔ جدول \حوالہ{جدول_کمپیوٹر_رموز}  میں  کمپیوٹر  کے \اصطلاح{رموز }\فرہنگ{رمز}\حاشیہب{operation codes, op codes}\فرہنگ{op code} پیش ہیں۔ یوں \قول{ نقل  الف} کی ہدایت کے لئے کمپیوٹر \عددی{0000} کا  ثنائی رمز استعمال کرتا ہے۔\قول{ جمع } کے لئے \عددی{0001}، \قول{ منفی } کے لئے \عددی{0010}، \قول{  برآمد }کے لئے \عددی{1110}، اور \قول{ رک } کے لئے \عددی{1111} استعمال ہو گا۔ 
\begin{table}
\caption{سادہ ترین کمپیوٹر کے رمز}
\label{جدول_کمپیوٹر_رموز}
\centering
\begin{tabular}{rr}
\toprule
ہدایت&رمز\\
\midrule
نقل&0000\\
جمع&0001\\
منفی&0010\\
برآمد&1110\\
رک&1111\\
\bottomrule
\end{tabular}
\end{table}

\begin{figure}

\end{figure}

جیسا پہلے ذکر کیا گیا،(صفحہ \حوالہصفحہ{مثال_حافظہ_برنامہ_نویس} پر مثال \حوالہ{مثال_حافظہ_برنامہ_نویس} دیکھیں) برنامہ نویس (شکل \حوالہ{شکل_کمپیوٹر_برنامہ_نویسی})  سوئچ  کے ذریعہ  حافظہ میں معلومات ڈالتا ہے۔ ان سوئچ کو یوں استعمال کیا گیا ہے کہ منقطع (کھڑا ) سوئچ \عددی{1} اور غیر منقطع ( بیٹھا  یا چالو) سوئچ \عددی{0} دیتا ہے۔ برنامہ نویسی کے دوران سوئچ \عددی{d_4} تا \عددی{d_7}   ہدایت  کے رمز کے مطابق رکھے جاتے ہیں جبکہ \عددی{d_0} تا \عددی{d_3}  ہدایت کے باقی\اصطلاح{ زیر عمل }\فرہنگ{زیر عمل}\حاشیہب{operand}\فرہنگ{operand} حصہ کے مطابق رکھے جاتے ہیں۔

مثلاً،  فرض کریں ہم درج ذیل  ہدایات حافظہ میں بھرنا چاہتے ہیں۔
\begin{center}
\begin{tabular}{rrr}
پتہ&\multicolumn{2}{c}{ہدایت}\\[1ex]
$0H$& نقل& $ّFH$\\
$1H$&جمع& $EH$\\
$2H$&رک&
\end{tabular}
\end{center}

سب سے پہلے ایک ایک ہدایت کا ثنائی روپ  حاصل کرتے ہیں۔
\begin{center}
\begin{tabular}{rrrr}
نقل&$FH$&=&$0000\,1111$\\
جمع & $EH$&=&$0001\,1110$\\
رک&&=&$1111\,xxxx$
\end{tabular}
\end{center}
پہلی ہدایت  \قول{نقل \عددی{FH}} ہے جس   کے دو حصے ہیں۔ اس کا پہلا حصہ  ہدایت \قول{نقل } ہے جس کا ثنائی رمز  \عددی{0000}   ہے؛ اس کا دوسرا حصہ   \عددی{FH} ہے جو     اس مقام کا پتہ ہے جہاں سے مواد لیا جائے گا۔یہ ہدایت کا  \اصطلاح{زیر عمل }\فرہنگ{زیر عمل}\حاشیہب{operand}\فرہنگ{operand} حصہ ہے۔ اس پتے کا ثنائی مماثل \عددی{1111} ہے۔   یوں \قول{جمع \عددی{FH}} کی جگہ  ان کے ثنائی مماثل جوڑ کر \عددی{0000\,1111} حاصل کیا گیا ہے۔ دوسری ہدایت میں جمع کا رمز \عددی{0001} اور زیر عمل حصہ \عددی{EH} کا ثنائی مماثل \عددی{1110} ہے۔ ان کو  ساتھ ساتھ لکھ کر \عددی{0001\,1110} حاصل کیا گیا ہے۔ آخری ہدایت  میں رک کا رمز \عددی{1111} ہے جبکہ اس کا کوئی زیر عمل حصہ نہیں پایا جاتا، لہٰذا زیر عمل حصہ غیر مطلوبہ ہے جس میں کچھ بھی لکھا جا سکتا ہے۔ اس غیر مطلوبہ حصہ کو \عددی{xxxx} سے ظاہر کیا گیا ہے۔یوں \عددی{1111\,xxxx} حاصل کیا گیا ہے۔

اب \عددی{S_2} کو بٹھا کر (زمین   سے  جوڑ کر)   پتہ  اور مواد کے  سوئچ    قدم با قدم   درج ذیل رکھیں، جہاں \قول{ک} سے مراد  کھڑا   یعنی منقطع  سوئچ ہے جو \عددی{1} کو ظاہر کرتا ہے،  \قول{ب} سے مراد بیٹھا یا غیر منقطع (چالو) سوئچ ہے کو \عددی{0}   دیگا، اور \قول{x} سے مراد یہ کہ  سوئچ کسی بھی حالت میں  (منقطع یا غیر منقطع)   ہو سکتا ہے۔
\begin{center}
\begin{tabular}{rr}
\multicolumn{1}{c}{پتہ}& \multicolumn{1}{c}{مواد}\\[1ex]
ب ب ب ب & ک ک ک ک   ب ب ب ب\\
ک ب ب ب & ب ک ک ک  ک ب ب ب\\
ب ک ب ب & \,\,\,\,\,
 x\,\quad  x\,\quad  x\, \quad x \quad
  ک ک ک ک
\end{tabular}
\end{center}

ہر قدم پر  پتہ اور  مواد سوئچ  مطلوبہ حالت میں رکھ کر   \عددی{S_1} کو بٹھا  کر دوبارہ کھڑا کریں۔ تینوں پتہ پر مواد لکھنے کے بعد \عددی{S_2} کو کھڑا کریں۔حافظہ کے ابتدائی تین  مقامات پر اب درج ذیل پایا جائے گا۔
\begin{center}
\begin{tabular}{LL}
\multicolumn{1}{c}{\text{\RL{پتہ}}}& \multicolumn{1}{c}{\text{\RL{مواد}}}\\[1ex]
0000&0000\,1111\\
0001&0001\,1110\\
0010&1111\,xxxx
\end{tabular}
\end{center}

آپ نے دیکھا کہ ہم کمپیوٹر کی \اصطلاح{ مادری زبان  } میں اردو  کے الفاظ مثلاً  \قول{نقل}، اور  \قول{جمع} استعمال کر کے کمپیوٹر  کو ہدایات جاری کرتے ہیں۔ کمپیوٹر از خود \قول{  ثنائی زبان  } سمجھتا ہے جو   \اصطلاح{مشینی زبان}\فرہنگ{زبان!مشینی}\فرہنگ{مشینی زبان}\حاشیہب{machine language}\فرہنگ{language!machine}\فرہنگ{machine language}    کہلاتی ہے۔  مشینی  زبان میں  \عددی{0} اور \عددی{1}  سے الفاظ بنائے جاتے ہیں۔ درج ذیل مثال  ان  زبانوں میں فرق اجاگر کرتا ہے۔

\ابتدا{مثال}
گزشتہ مثال میں دیے گئے  برنامے کا ترجمہ مشینی  زبان میں کریں۔

حل:\quad
مثال \حوالہ{مثال_کمپیوٹر_برنامہ_الف} کا برنامہ جو مادری زبان میں ہے  ذیل ہے۔
\begin{center}
\begin{tabular}{rr}
پتہ& ہدایات\\[0.5ex]
0H& نقل  \عددی{9H}\\
1H&جمع \عددی{AH}\\
2H&جمع \عددی{BH}\\
3H&منفی \عددی{CH}\\
4H&برآمد\\
5H&رک
\end{tabular}
\end{center}
اس کا ترجمہ مشینی  زبان میں کرتے ہیں۔
\begin{center}
\begin{tabular}{RR}
\multicolumn{1}{c}{\text{\RL{پتہ}}}& \multicolumn{1}{c}{\text{\RL{ہدایت}}}\\[0.5ex]
0000&0000\,1001\\
0001&0001\,1010\\
0010&0001\,1011\\
0011&0010\,1100\\
0100&1110\,xxxx\\
0101&1111\,xxxx
\end{tabular}
\end{center}

اس ثنائی برنامہ میں  ہدایت کے چار  بلند  تر رتبی بِٹ \قول{    عمل } کو ظاہر کرتے ہیں جبکہ چار کم تر رتبی بِٹ  \قول{پتہ } فراہم کرتے ہیں۔ بعض اوقات ہم  چار بلند تر رتبی بِٹ کو \اصطلاح{ جزو  ہدایت }\فرہنگ{جزو ہدایت}\حاشیہب{instruction field}\فرہنگ{instruction field} اور چار کم تر رتبی بِٹ کو  \اصطلاح{جزو  پتہ }\فرہنگ{جزو پتہ}\حاشیہب{address field}\فرہنگ{address field} کہتے ہیں۔
\begin{align*}
\underbrace{XXXX}_{\text{\RL{جزو ہدایت}}}\, \underbrace{YYYY}_{\text{\RL{جزو پتہ}}}=\text{\RL{ہدایت}}
\end{align*}
\انتہا{مثال}
\ابتدا{مثال}
درج  ذیل  حساب کرنے کے لئے کمپیوٹر کا برنامہ  لکھیں۔ تمام اعداد اعشاری ہیں۔
\begin{align*}
16+20+24-32
\end{align*}

حل:\quad
گزشتہ مثال کا برنامہ لے کر  حافظہ کے مقام \عددی{9H} تا \عددی{CH} میں بالترتیب  مواد \عددی{16}، \عددی{20}، \عددی{24}، اور \عددی{32}  کے اساس سولہ مماثل لکھ کر    درج ذیل مطلوبہ برنامہ حاصل ہو گا۔(اعشاری \عددی{16} کا اساس سولہ مماثل \عددی{10H} ہے۔)
\begin{center}
\begin{tabular}{RR}
\multicolumn{1}{c}{\text{\RL{پتہ}}}& \multicolumn{1}{c}{\text{\RL{ہدایت}}}\\[0.5ex]
0H& \LDA{9H}\\
1H&\ADD{AH}\\
2H&\ADD{BH}\\
3H&\SUB{CH}\\
4H&\OUT\\
5H&\HLT\\
6H&XX\\
7H&XX\\
8H&XX\\
9H&10H\\
AH&14H\\
BH&18H\\
CH&20H\\
DH&XX\\
EH&XX\\
FH&XX
\end{tabular}
\end{center}
اس کا ترجمہ مشینی  زبان میں کرتے ہیں۔
\begin{center}
\begin{tabular}{RR}
\multicolumn{1}{c}{\text{\RL{پتہ}}}& \multicolumn{1}{c}{\text{\RL{ہدایت}}}\\[0.5ex]
0000&0000\,1001 \\
0001&0001\,1010\\
0010&0001\,1011\\
0011&0010\,1100\\
0100&1110\,xxxx\\
0101&1111\,xxxx\\
0110&xxxx\,xxxx\\
0111&xxxx\,xxxx\\
1000&xxxx\,xxxx\\
1001&0001\,0000\\
1010&0001\,0100\\
1011&0001\,1000\\
1100&0010\,0000\\
1101&xxxx\,xxxx\\
1110&xxxx\,xxxx\\
1111&xxxx\,xxxx
\end{tabular}
\end{center}
یاد رہے برنامے کی پہلی ہدایت حافظہ کے مقام \عددی{0000}  سے پڑھی جاتی ہے،   دوسری  مقام \عددی{0001} سے پڑھی جاتی ہے، وغیرہ، لہٰذا  برنامہ زیریں حافظہ میں اور مواد بالا میں رکھا  گیا ہے۔ غیر مستعمل مقامات  میں معلومات   کو   \عددی{xxxx\,xxxx}  دکھایا گیا ہے۔
\انتہا{مثال}
\ابتدا{مثال}
درج بالا مثال میں حاصل ثنائی برنامہ کو اساس سولہ کے روپ میں لکھیں۔ ثنائی روپ کی بجائے ہم عموماً  برنامے کا اساس سولہ روپ استعمال کرتے ہیں۔

حل:\quad
\begin{center}
\begin{tabular}{RR}
\multicolumn{1}{c}{\text{\RL{پتہ}}}& \multicolumn{1}{c}{\text{\RL{ہدایت}}}\\[0.5ex]
0H&09H \\
1H&1AH\\
2H&1BH\\
3H&2CH\\
4H&EXH\\
5H&FXH\\
6H&XXH\\
7H&XXH\\
8H&XXH\\
9H&10H\\
AH&14H\\
BH&18H\\
CH&20H\\
DH&XXH\\
EH&XXH\\
FH&XXH
\end{tabular}
\end{center}
اساس سولہ میں لکھی گئی زبان بھی مشینی زبان کہلاتی ہے۔

  مشینی زبان میں منفی عدد کا اساس \عددی{2} تکملہ  استعمال کیا جاتا ہے۔ مثال کے طور پر ، \عددی{-03H}  کی بجائے \عددی{FDH} حافظہ میں ڈالا جائے گا۔
\انتہا{مثال}

\حصہ{بازیابی  پھیرا}
کمپیوٹر کی خودکار کارکردگی کا دارومدار \قول{ قابو مرکز } پر ہے۔ حافظہ سے باری باری ایک  ہدایت    اٹھانے اور اس پر عمل کرنے  کے  احکامات قابو مرکز جاری کرتا ہے۔ہدایت  اٹھانے اور اس پر عمل کرنے کے دوران کمپیوٹر مختلف \اصطلاح{   وقتیہ حال }\فرہنگ{وقتیہ حال}\حاشیہب{timing states}\فرہنگ{timing states} (\عددی{T} حال)   سے گزرتا ہے، جس میں دفاتر  کا  مواد  تبدیل ہوتا ہے۔ آئیں وقتیہ حال پر غور کریں۔

\جزوحصہء{چھلا گنت کار}
اس کمپیوٹر میں  چھلا گنت کار  مستعمل ہے جو شکل \حوالہ{شکل_کمپیوٹر_چھلا} میں پیش ہے۔ مخلوط دور \عددی{74107} میں دو عدد جے کے پلٹ کار  پائے جاتے ہیں لہٰذا تین مخلوط دور استعمال کیے گئے۔ اس  مخلوط دور میں زبردستی پست   کا مداخل موجود ہے، تاہم اس میں زبردستی بلند کا مداخل موجود نہیں۔  استعمال سے پہلا ایک مرتبہ  چھلا گنت کار کو ابتدائی حال میں لانا ضروری ہے جس میں صرف ایک مخارج بلند ہو۔  زبردستی پست مداخل پلٹ کے   مخارج پس کر تا ہے جبکہ ہمیں ایک مخارج بلند چاہیے۔ اسی لئے  بایاں ترین پلٹ باقی سے مختلف طریقے سے استعمال کیا گیا ہے۔ پست حال میں اس کا \عددی{\overline{Q}} بلند ہو گا جو ساعت کے کنارہ اترائی پر اگلی پلٹ کو منتقل ہو گا۔

\begin{figure}
\centering
\begin{subfigure}{1\textwidth}
\centering
\begin{tikzpicture}
\pgfmathsetmacro{\ksepX}{2}
\pgfmathsetmacro{\kpin}{0.4}
\def\pnd{pnd}
\def\p{p}			%pin tip
\def\pb{pb}			%pin base
\def\pbd{pbd}			%pin base
\def\pd{pd}
\def\pQ{p6}
\kJKFF[u0]{0}{0}
\kJKFF[u1]{-1*\ksepX}{0}
\kJKFF[u2]{-2*\ksepX}{0}
\kJKFF[u3]{-3*\ksepX}{0}
\kJKFF[u4]{-4*\ksepX}{0}
\kFF[u5]{-5*\ksepX}{0}
\def\kref{u5}
\draw(u0pn2)node[ocirc]{};
\draw(u1pn2)node[ocirc]{};
\draw(u2pn2)node[ocirc]{};
\draw(u3pn2)node[ocirc]{};
\draw(u4pn2)node[ocirc]{};
\foreach \n in {1,2,3,4,6}{\draw[thin](\kref\pb\n)--(\kref\p\n);}
\foreach \n/\lbl in {1/K,3/J,4/Q,6/{\overline{Q}}}{\draw(\kref\pl\n)node[]{$\lbl$};}
\foreach \n in {2}{\draw[thin](\kref\pcd\n)--(\kref\pcm\n)--(\kref\pcu\n);}
\foreach \n in {2,6}{\draw(\kref\pn\n)node[ocirc]{};}
\draw(u5p6)--(u4p1) (u4p6)--(u3p1) (u3p6)--(u2p1)   (u2p6)--(u1p1)  (u1p6)--(u0p1) (u0p6)--++(0,2*\kpin)-|(u5p1);
\draw(u5p4)--(u4p3) (u4p4)--(u3p3) (u3p4)--(u2p3)   (u2p4)--(u1p3)  (u1p4)--(u0p3) (u0p4)--++(0,-4*\kpin) -|(u5p3);
\draw(u0p2)--++(0,-4.5*\kpin)--++(-5*\ksepX-\kpin,0)coordinate(kBot)|-coordinate(kclk)(u5p2)   (kclk)--++(-\kpin,0)coordinate(klft)node[left]{$CLK$};
\draw(u1p2)--(u1p2 |- kBot) (u2p2)--(u2p2 |- kBot) (u3p2)--(u3p2 |- kBot) (u4p2)--(u4p2 |- kBot);
\foreach \n in {0,1,2,3,4,5}{\draw(u\n\pbd)--(u\n\pd) (u\n\pnd)node[ocirc]{};}
\draw(u0pd)--++(0,-3*\kpin)coordinate(kRES)--(kRES -|klft)coordinate(kR)node[left]{$\overline{CLR}$};
\draw(u1pd)--(u1pd |- kR) (u2pd)--(u2pd |- kR) (u3pd)--(u3pd |- kR) (u4pd)--(u4pd |- kR) (u5pd)--(u5pd |- kR);
\foreach \n/\a in {0/6,1/5,2/4,3/3,4/2,5/1}{\draw[thin](u\n\pQ)++(0.5*\kpin,0)--++(0,3*\kpin)node[above]{$T_{\a}$};}
\draw(u0p6)--++(0.5*\kpin,0);
\draw(u0pbd)node[below right]{\scriptsize{$13$}};
\draw(u2pbd)node[below right]{\scriptsize{$13$}};
\draw(u4pbd)node[below right]{\scriptsize{$13$}};
\draw(u1pbd)node[below right]{\scriptsize{$10$}};
\draw(u3pbd)node[below right]{\scriptsize{$10$}};
\draw(u5pbd)node[below right]{\scriptsize{$10$}};
\foreach \n/\a in {1/11,2/9,3/8}{\draw(u5pb\n)node[above,xshift=-1.25ex]{\scriptsize{$\a$}};}
\foreach \n/\a in {4/5,6/6}{\draw(u5pb\n)node[above,xshift=1.25ex]{\scriptsize{$\a$}};}
%
\foreach \n/\a in {1/1,2/12,3/4}{\draw(u0pb\n)node[above,xshift=-1.25ex]{\scriptsize{$\a$}};}
\foreach \n/\a in {4/2,6/3}{\draw(u0pb\n)node[above,xshift=1.25ex]{\scriptsize{$\a$}};}
\foreach \n/\a in {1/1,2/12,3/4}{\draw(u2pb\n)node[above,xshift=-1.25ex]{\scriptsize{$\a$}};}
\foreach \n/\a in {4/2,6/3}{\draw(u2pb\n)node[above,xshift=1.25ex]{\scriptsize{$\a$}};}
\foreach \n/\a in {1/1,2/12,3/4}{\draw(u4pb\n)node[above,xshift=-1.25ex]{\scriptsize{$\a$}};}
\foreach \n/\a in {4/2,6/3}{\draw(u4pb\n)node[above,xshift=1.25ex]{\scriptsize{$\a$}};}
%
\foreach \n/\a in {1/8,2/9,3/11}{\draw(u1pb\n)node[above,xshift=-1.25ex]{\scriptsize{$\a$}};}
\foreach \n/\a in {4/6,6/5}{\draw(u1pb\n)node[above,xshift=1.25ex]{\scriptsize{$\a$}};}
\foreach \n/\a in {1/8,2/9,3/11}{\draw(u3pb\n)node[above,xshift=-1.25ex]{\scriptsize{$\a$}};}
\foreach \n/\a in {4/6,6/5}{\draw(u3pb\n)node[above,xshift=1.25ex]{\scriptsize{$\a$}};}
\draw(-0.25*\ksepX,-1.75cm)node[rectangle,inner sep=0pt,text width=0.5cm,align=center]{\small{$u36$\\ $74107$}};
\draw(-2.25*\ksepX,-1.75cm)node[rectangle,inner sep=0pt,text width=0.5cm,align=center]{\small{$u37$\\ $74107$}};
\draw(-4.25*\ksepX,-1.75cm)node[rectangle,inner sep=0pt,text width=0.5cm,align=center]{\small{$u38$\\ $74107$}};
\end{tikzpicture}
\caption{}
\end{subfigure}
\begin{subfigure}{1\textwidth}
\centering
\begin{tikzpicture}
\pgfmathsetmacro{\kpin}{0.5}
\pgfmathsetmacro{\kpina}{2.5}
\pgfmathsetmacro{\kpinb}{4}
\pgfmathsetmacro{\kpsep}{0.50}
\pgfmathsetmacro{\kul}{0.50}
\pgfmathsetmacro{\kmv}{0.15}
\pgfmathsetmacro{\kxdim}{2*\kul+5*\kpsep}
\pgfmathsetmacro{\kydim}{2*\kul+1*\kpsep}
\draw[thick](0,0)rectangle++(\kxdim,\kydim)node[pos=0.5]{\text{\RL{چھلا گنت کار}}};
\foreach \n/\a in {0/6,1/5,2/4,3/3,4/2,5/1}{\draw[thin](\kul+\n*\kpsep,0)--++(0,-\kpin)node[below]{$T_{\a}$};}
\foreach \n/\a in {0/{\overline{CLR}},1/{CLK}}{\draw[thin](\kxdim,\kul+\n*\kpsep)--++(\kpin,0)node[right]{$\a$};}
\draw[thin](\kxdim,\kul)++(0.07,0)node[ocirc]{}  (\kxdim,\kul+\kpsep)++(0.07,0)node[ocirc]{} 
(\kxdim,\kul+\kpsep-\kmv)--++(-\kmv,\kmv)--++(\kmv,\kmv);
\end{tikzpicture}
\caption{}
\end{subfigure}
\begin{subfigure}{1\textwidth}
\centering
\begin{otherlanguage}{english}
 \begin{tikztimingtable}[%
timing/.style={x=4ex,y=3ex},
timing/rowdist=6ex,
every node/.style={inner sep=0,outer sep=0},
%timing/c/arrow tip=latex, %and this set the style
%timing/c/rising arrows,
timing/slope=0, %0.1 is good
timing/dslope=0,
thick,
]
%\tikztimingmetachar{R}{[|/utils/exec=\setcounter{new}{0}|]}
%\usetikztiminglibrary[new={char=Q,reset char=R}]{counters}
%[timing/counter/new={char=c, base=2,digits=3,max value=7, wraps ,text style={font=\normalsize}}] 12{2c} \\ 
%$C$& H22{C}\\
%$\texturdu{\RL{پتہ}}$&3D{} 20D{[scale=1.5]\texturdu{\RL{درست پتہ}}} 3D{}\\
$CLK$&HN(a)LHN(b)LHN(c)LHN(d)LHN(e)LHN(f)LHN(g)LHN(h)L\\
\\
$T_1$&LHHLLLLLLLLLLHHL\\
$T_2$&LLLHHLLLLLLLLLLL\\
$T_3$&LLLLLHHLLLLLLLLL\\
$T_4$&LLLLLLLHHLLLLLLL\\
$T_5$&LLLLLLLLLHHLLLLL\\
$T_6$&LLLLLLLLLLLHHLLL\\
\extracode
\begin{pgfonlayer}{background}
\begin{scope}[]
\draw [latex-latex] (a|-row2.north) --node[fill=white]{$T_1$}node[below,yshift=-1ex]{\texturdu{حال}} (b|-row2.north);
\draw [latex-latex] (b|-row2.north) --node[fill=white]{$T_2$}node[below,yshift=-1ex]{\texturdu{حال}} (c|-row2.north);
\draw [latex-latex] (c|-row2.north) --node[fill=white]{$T_3$}node[below,yshift=-1ex]{\texturdu{حال}} (d|-row2.north);
\draw [latex-latex] (d|-row2.north) --node[fill=white]{$T_4$}node[below,yshift=-1ex]{\texturdu{حال}} (e|-row2.north);
\draw [latex-latex] (e|-row2.north) --node[fill=white]{$T_5$}node[below,yshift=-1ex]{\texturdu{حال}} (f|-row2.north);
\draw [latex-latex] (f|-row2.north) --node[fill=white]{$T_6$}node[below,yshift=-1ex]{\texturdu{حال}} (g|-row2.north);
\draw [latex-latex] (g|-row2.north) --node[fill=white]{$T_1$}node[below,yshift=-1ex]{\texturdu{حال}} (h|-row2.north);
\foreach \n in {a,b,c,d,e,f,g,h}{\draw[thin]($(\n|-row1.south)+(0,-1ex)$)--++(0,-4ex);}
%%\vertlines[darkgray,dotted]{3.6,7.5,11.5,15.5,19.5,23.5,27.5}
%\foreach \n in {1,3,...,11} \draw(4*\n ex-2ex,-4ex+1.25ex)node[]{$0$};
%\foreach \n in {2,4,...,12} \draw(4*\n ex-2ex,-4ex+1.25ex)node[]{$1$};
%\foreach \n in {1,2,5,6,9,10} \draw(4*\n ex-2ex,-12 ex+1.25ex)node[]{$0$};
%\foreach \n in {3,4,7,8,11,12} \draw(4*\n ex-2ex,-12 ex+1.25ex)node[]{$1$};
%\foreach \n in {1,2,3,4,9,10,11,12} \draw(4*\n ex-2ex,-20 ex+1.25ex)node[]{$0$};
%\foreach \n in {5,6,7,8} \draw(4*\n ex-2ex,-20 ex+1.25ex)node[]{$1$};
%\draw(4*4 ex-2ex,-12 ex+1.25ex) circle (0.25cm and 1.75cm);
%\draw(4*4 ex-2ex,-7*\rowdist+1.25ex) circle (0.25cm and 0.5cm);
%%\foreach \n in {1}\draw(B\n.south)--(F\n.north);
%%\foreach \n in {1}\draw(C\n.south)--(G\n.north);
\end{scope}
\end{pgfonlayer}
\end{tikztimingtable}
\end{otherlanguage}
\caption{}
\end{subfigure}
\caption{(ا) چھلا گنت کار،  (ب) ڈبہ شکل،  (ج) ساعت،اور وقتیہ ترسیمات۔}
\label{شکل_کمپیوٹر_چھلا}
\end{figure}

شکل \حوالہ{شکل_کمپیوٹر_چھلا}  -ب میں   گنت کار کی ڈبہ شکل   جبکہ  شکل-د میں ساعت اور وقتیہ ترسیمات  پیش ہیں۔ چھلا گنت کار کا مخارج درج ذیل ہے۔
\begin{align*}
\bold{T}=T_6T_5T_4T_3T_2T_1
\end{align*}
کمپیوٹر کی دوڑ کے آغاز میں چھلا لفظ درج ذیل ہو گا۔
\begin{align*}
\bold{T}=000001
\end{align*}
یک بعد دیگرے ساعت   کی دھڑکن  ذیل چھلا الفاظ پیدا کرتا ہے۔
\begin{align*}
\bold{T}&=000010\\
\bold{T}&=000100\\
\bold{T}&=001000\\
\bold{T}&=010000\\
\bold{T}&=100000\\
\end{align*}
اس کے بعد چھلا گنت کار \عددی{000001}  پہنچتا ہے اور دوبارہ چکر کاٹنا  شروع کرتا  ہے۔ یہ عمل مسلسل چلتا ہے۔ ہر ایک چھلا لفظ ایک \عددی{T} پھیرا ظاہر کرتا ہے۔

شکل-ج میں وقتیہ ترسیمات پیش ہیں۔ ابتدائی \عددی{T_1} حال  کا آغاز ساعت کے پہلے کنارہ اترائی پر  اور اختتام اگلے کنارہ اترائی پر  ہو گا۔ اس \عددی{T} حال میں چھلا گنت کار کا \عددی{T_1} بِٹ بلند  رہے گا۔

اگلے حال میں \عددی{T_2} بلند ہو گا؛ اس سے اگلے میں \عددی{T_3}؛ اس کے بعد \عددی{T_4}؛ وغیرہ۔ جیسا آپ دیکھ سکتے ہیں چھلا گنت کار چھ \عددی{T} حال پیدا کرتا ہے۔ ان چھ \عددی{T} حال کے دوران   (ہر) ایک ہدایت اٹھایا جاتا ہے اور اس پر عمل کیا جاتا ہے۔

جیسا دکھایا گیا ہے، ساعت کا کنارہ  چڑھائی   نصف \عددی{T} حال  گزرنے کے بعد (یعنی وسط میں )  آتا ہے۔ یہ ایک اہم حقیقت ہے جس پر جلد روشنی ڈالی جائے گی۔

\جزوحصہء{پتہ حال}
برنامہ گنت کار سے حافظہ کو پتہ  \عددی{T_1} حال   کے دوران منتقل ہوتا ہے، لہٰذا یہ \اصطلاح{پتہ حال}\فرہنگ{حال!پتہ}\حاشیہب{address state}\فرہنگ{state!address}  کہلاتا ہے۔شکل  \حوالہ{شکل_کمپیوٹر_اجاگر_حصے_بازیابی_پھیرا}-الف میں کمپیوٹر کے وہ حصے گہری سیاہی سے  اجاگر کیے گئے ہیں جو  \عددی{T_1}  حال  کے دوران  فعال ہیں (غیر فعال حصے ہلکی سیاہی میں دکھائے گئے ہیں؛ مزید،  ڈبہ ادوار  کے مختصر  نام لکھ گئے ہیں)۔

پتہ حال کے دوران \عددی{E_P} اور \عددی{\overline{L}_M} فعال جبکہ باقی تمام بِٹ غیر فعال ہوں گے۔ یوں اس حال کے دوران  قابو و ترتیب کار  درج ذیل قابو لفظ خارج کرتا ہے۔
\begin{align*}
\text{\RL{قابو}}&=C_PE_P\overline{L}_M\overline{CE}\quad  \overline{L}_I\overline{E}_I\overline{L}_AE_A\quad S_UE_U\overline{L}_B\overline{L}_O\\
&=\,\,0\,\,\,\,\,1\,\,\,\,\,0\,\,\,\,\,1\quad\,\, 1\,\,\,\,1\,\,\,\,1\,\,\,\,0\quad \quad 0\,\,\,\,0\,\,\,\,1\,\,\,\,1
\end{align*}

\جزوحصہء{بڑھوتری حال}
شکل  \حوالہ{شکل_کمپیوٹر_اجاگر_حصے_بازیابی_پھیرا}-ب میں کمپیوٹر کے وہ حصے اجاگر کیے گئے ہیں جو \عددی{T_2} حال کے دوران فعال ہیں۔ اس  حال میں گنت کار  کا  شمار (گنتی ) ایک قدم بڑھایا جاتا ہے لہٰذا اس کو\اصطلاح{ بڑھوتری حال}\فرہنگ{حال!بڑھوتری}\حاشیہب{increment state}\فرہنگ{state!increment} کہتے ہیں۔ بڑھوتری حال کے دوران قابو و ترتیب کار درج ذیل قابو لفظ خارج کرتا ہے۔
\begin{align*}
\text{\RL{قابو}}&=C_PE_P\overline{L}_M\overline{CE}\quad  \overline{L}_I\overline{E}_I\overline{L}_AE_A\quad S_UE_U\overline{L}_B\overline{L}_O\\
&=\,\,1\,\,\,\,\,0\,\,\,\,\,1\,\,\,\,\,1\quad\,\, 1\,\,\,\,1\,\,\,\,1\,\,\,\,0\quad \quad 0\,\,\,\,0\,\,\,\,1\,\,\,\,1
\end{align*}
جیسا آپ دیکھ سکتے ہیں \عددی{C_P} فعال ہو گا۔

\جزوحصہء{حافظہ حال}
  حافظہ سے ہدایت دفتر کو \عددی{T_3} حال کے دوران ہدایت منتقل کی جاتی ہے۔ یہ ہدایت فراہم کردہ پتہ کے مقام  سے پڑھی جاتی ہے۔اس حال کے دوران فعال حصے  شکل  \حوالہ{شکل_کمپیوٹر_اجاگر_حصے_بازیابی_پھیرا}-ج میں دکھائے گئے ہیں۔ اس حال میں صرف \عددی{\overline{CE}} اور \عددی{\overline{L}_I} قابو بِٹ فعال ہوں گے۔اس حال کے دوران قابو و ترتیب کار  درج ذیل قابو لفظ خارج کرتا ہے۔
  \begin{align*}
\text{\RL{قابو}}&=C_PE_P\overline{L}_M\overline{CE}\quad  \overline{L}_I\overline{E}_I\overline{L}_AE_A\quad S_UE_U\overline{L}_B\overline{L}_O\\
&=\,\,0\,\,\,\,\,0\,\,\,\,\,1\,\,\,\,\,0\quad\,\, 0\,\,\,\,1\,\,\,\,1\,\,\,\,0\quad \quad 0\,\,\,\,0\,\,\,\,1\,\,\,\,1
\end{align*}

\begin{figure}
\centering
\begin{subfigure}{0.30\textwidth}
\centering
\begin{tikzpicture}
\pgfmathsetmacro{\klshift}{0.25}
\pgfmathsetmacro{\knshift}{0.07}
\pgfmathsetmacro{\kmv}{0.15}
\pgfmathsetmacro{\knshift}{0.07}
\pgfmathsetmacro{\kpin}{0.30}
\pgfmathsetmacro{\kpina}{0.30}
\pgfmathsetmacro{\kpsep}{0.15}			%pin to pin distance
\pgfmathsetmacro{\kW}{\kpsep}
\pgfmathsetmacro{\kulV}{0.20}			%edge clearance along vertical edge
\pgfmathsetmacro{\kulH}{0.20}
\pgfmathsetmacro{\kdimX}{2*\kulH+4*\kpsep}
\pgfmathsetmacro{\kdimY}{2*\kulV+4*\kpsep}		%two spaces between 3 pins
\pgfmathsetmacro{\ksepX}{\kdimX+2*\kpina+1*\kpsep+\kW}
\pgfmathsetmacro{\ksepY}{\kdimY+\kpin+0.5*\kpsep}
\draw(0,0)  [thick] rectangle ++(\kdimX,\kdimY)node[pos=0.5,rectangle,inner sep=0pt,text width=1.5cm,align=center]{\RTL{گنتکار}};
\draw(0,\kulV+0*\kpsep)--++(-\kpin,0)node[left]{$E_P$};
\draw(\kdimX,\kulV+1.5*\kpsep)--++(\kpina,0);
\draw(\kdimX,\kulV+2.5*\kpsep)--++(\kpina,0);
\draw(\kdimX,\kulV+1.5*\kpsep)++(\kpina,0)++(-0.5*\kpsep,-0.5*\kpsep)--++(1*\kpsep,1*\kpsep)--++(-1*\kpsep,1*\kpsep);

\draw(0,-\ksepY)  [thick] rectangle ++(\kdimX,\kdimY)node[pos=0.5,rectangle,inner sep=0pt,text width=1cm,align=center]{\RTL{دفتر پتہ}};
\draw(0,-\ksepY+\kulV+4*\kpsep)--++(-\kpin,0)node[left]{$\overline{L}_M$};
\draw(0,-\ksepY+\kulV+4*\kpsep)++(-\knshift,0)node[ocirc]{};
\draw[inactivePart](\kulH+0*\kpsep,-\ksepY)--++(0,-\kpin);
\draw[inactivePart](\kulH+1*\kpsep,-\ksepY)--++(0,-\kpin);
\draw[inactivePart](\kulH+0*\kpsep,-\ksepY)++(0,-\kpin)++(-0.5*\kpsep,0.5*\kpsep)--++(\kpsep,-\kpsep)--++(\kpsep,\kpsep);
\draw[inactivePart](\kulH+3*\kpsep,-\ksepY)--++(0,-\kpin);
\draw[inactivePart](\kulH+4*\kpsep,-\ksepY)--++(0,-\kpin);
\draw[inactivePart](\kulH+3*\kpsep,-\ksepY)++(0,-\kpin)++(-0.5*\kpsep,0.5*\kpsep)--++(\kpsep,-\kpsep)--++(\kpsep,\kpsep);
\draw(\kdimX,-\ksepY+\kulV+1.5*\kpsep)++(0.5*\kpsep,0)--++(\kpina,0);
\draw(\kdimX,-\ksepY+\kulV+2.5*\kpsep)++(0.5*\kpsep,0)--++(\kpina,0);
\draw(\kdimX,-\ksepY+\kulV+2*\kpsep)--++(1*\kpsep,-1*\kpsep);
\draw(\kdimX,-\ksepY+\kulV+2*\kpsep)--++(1*\kpsep,1*\kpsep);

\draw [thick,inactivePart](0,-2*\ksepY)  rectangle ++(\kdimX,\kdimY);%node[pos=0.5,rectangle,inner sep=0pt,text width=1cm,align=center]{\footnotesize{\RTL{حافظہ}}};
\draw[inactivePart](\kdimX,-2*\ksepY+\kulV+1.5*\kpsep)--++(\kpina,0);
\draw[inactivePart](\kdimX,-2*\ksepY+\kulV+2.5*\kpsep)--++(\kpina,0);
\draw[inactivePart](\kdimX,-2*\ksepY+\kulV+1.5*\kpsep)++(\kpina,0)++(-0.5*\kpsep,-0.5*\kpsep)--++(1*\kpsep,1*\kpsep)--++(-1*\kpsep,1*\kpsep);

\draw[thick,inactivePart](0,-3*\ksepY)  rectangle ++(\kdimX,\kdimY);%node[pos=0.5,rectangle,inner sep=0pt,text width=1cm,align=center]{\RTL{دفتر ہدایت}};
\draw[inactivePart](\kdimX,-3*\ksepY+\kulV+0*\kpsep)--++(\kpina,0);
\draw[inactivePart](\kdimX,-3*\ksepY+\kulV+1*\kpsep)--++(\kpina,0);
\draw[inactivePart](\kdimX,-3*\ksepY+\kulV+0*\kpsep)++(\kpina,0)++(-0.5*\kpsep,-0.5*\kpsep)--++(1*\kpsep,1*\kpsep)--++(-1*\kpsep,1*\kpsep);
\draw[inactivePart](\kdimX,-3*\ksepY+\kulV+3*\kpsep)++(0.5*\kpsep,0)--++(\kpina,0);
\draw[inactivePart](\kdimX,-3*\ksepY+\kulV+4*\kpsep)++(0.5*\kpsep,0)--++(\kpina,0);
\draw[inactivePart](\kdimX,-3*\ksepY+\kulV+3.5*\kpsep)--++(1*\kpsep,-1*\kpsep);
\draw[inactivePart](\kdimX,-3*\ksepY+\kulV+3.5*\kpsep)--++(1*\kpsep,1*\kpsep);
\draw[inactivePart](\kulH+1.5*\kpsep,-3*\ksepY)--++(0,-\kpin);
\draw[inactivePart](\kulH+2.5*\kpsep,-3*\ksepY)--++(0,-\kpin);
\draw[inactivePart](\kulH+1.5*\kpsep,-3*\ksepY)++(0,-\kpin)++(-0.5*\kpsep,0.5*\kpsep)--++(\kpsep,-\kpsep)--++(\kpsep,\kpsep);

\draw(0,-4*\ksepY)  [thick] rectangle ++(\kdimX,\kdimY)node[pos=0.5,rectangle,inner sep=0pt,text width=1.5cm,align=center]{\RTL{قابو    }};
\draw(\kulH+1.5*\kpsep,-4*\ksepY)--++(0,-\kpin);
\draw(\kulH+2.5*\kpsep,-4*\ksepY)--++(0,-\kpin);
\draw(\kulH+1.5*\kpsep,-4*\ksepY)++(0,-\kpin)++(-0.5*\kpsep,0.5*\kpsep)--++(\kpsep,-\kpsep)--++(\kpsep,\kpsep);
\draw(\kulH+2*\kpsep,-4*\ksepY)++(0,-1*\kpin-\kpsep)node[below,rectangle,inner sep=0pt, text width=1cm]{\RTL{قابو لفظ}};

\draw(\kdimX+\kpina+0.5*\kpsep,\kdimY)  [thick] rectangle ++(\kW,-4*\ksepY+\kpin);

\begin{scope}[inactivePart]
\draw(\ksepX,-0*\ksepY)  [thick] rectangle ++(\kdimX,\kdimY);%node[pos=0.5,rectangle,inner sep=0pt,text width=1cm,align=center]{\RTL{$A$}};
\draw(\ksepX,-0*\ksepY+\kulV+0*\kpsep)--++(-\kpina,0); 
\draw(\ksepX,-0*\ksepY+\kulV+1*\kpsep)--++(-\kpina,0); 
\draw(\ksepX,-0*\ksepY+\kulV+0*\kpsep)++(-\kpina,0)++(0.5*\kpsep,-0.5*\kpsep)--++(-\kpsep,\kpsep)--++(\kpsep,\kpsep);
\draw(\ksepX,-0*\ksepY+\kulV+3*\kpsep)++(-0.5*\kpsep,0)--++(-\kpina,0); 
\draw(\ksepX,-0*\ksepY+\kulV+4*\kpsep)++(-0.5*\kpsep,0)--++(-\kpina,0); 
\draw(\ksepX,-0*\ksepY+\kulV+3*\kpsep)++(-1*\kpsep,-0.5*\kpsep)--++(\kpsep,\kpsep)--++(-\kpsep,\kpsep);
\draw(\ksepX+\kulH+1.5*\kpsep,-0*\ksepY)--++(0,-\kpin);
\draw(\ksepX+\kulH+2.5*\kpsep,-0*\ksepY)--++(0,-\kpin);
\draw(\ksepX+\kulH+1.5*\kpsep,-0*\ksepY)++(0,-\kpin)++(-0.5*\kpsep,0.5*\kpsep)--++(\kpsep,-\kpsep)--++(\kpsep,\kpsep);

\draw(\ksepX,-1*\ksepY)  [thick] rectangle ++(\kdimX,\kdimY);%node[pos=0.5,rectangle,inner sep=0pt,text width=1cm,align=center]{\RTL{جمع و منفی کار}};
\draw(\ksepX,-1*\ksepY+\kulV+1.5*\kpsep)--++(-\kpina,0); 
\draw(\ksepX,-1*\ksepY+\kulV+2.5*\kpsep)--++(-\kpina,0); 
\draw(\ksepX,-1*\ksepY+\kulV+1.5*\kpsep)++(-\kpina,0)++(0.5*\kpsep,-0.5*\kpsep)--++(-\kpsep,\kpsep)--++(\kpsep,\kpsep);
\draw(\ksepX+\kulH+1.5*\kpsep,-1*\ksepY)++(0,-0.5*\kpsep)--++(0,-\kpin);
\draw(\ksepX+\kulH+2.5*\kpsep,-1*\ksepY)++(0,-0.5*\kpsep)--++(0,-\kpin);
\draw(\ksepX+\kulH+1.5*\kpsep,-1*\ksepY)++(-0.5*\kpsep,-\kpsep)--++(\kpsep,\kpsep)--++(\kpsep,-\kpsep);


\draw(\ksepX,-2*\ksepY)  [thick] rectangle ++(\kdimX,\kdimY);%node[pos=0.5,rectangle,inner sep=0pt,text width=1cm,align=center]{\RTL{$B$}};
\draw(\ksepX,-2*\ksepY+\kulV+1.5*\kpsep)++(-0.5*\kpsep,0)--++(-\kpina,0); 
\draw(\ksepX,-2*\ksepY+\kulV+2.5*\kpsep)++(-0.5*\kpsep,0)--++(-\kpina,0); 
\draw(\ksepX,-2*\ksepY+\kulV+1.5*\kpsep)++(-1*\kpsep,-0.5*\kpsep)--++(\kpsep,\kpsep)--++(-\kpsep,\kpsep);

\draw(\ksepX,-3*\ksepY)  [thick] rectangle ++(\kdimX,\kdimY);%node[pos=0.5,rectangle,inner sep=0pt,text width=1cm,align=center]{\RTL{خارجی دفتر}};
\draw(\ksepX,-3*\ksepY+\kulV+1.5*\kpsep)++(-0.5*\kpsep,0)--++(-\kpina,0); 
\draw(\ksepX,-3*\ksepY+\kulV+2.5*\kpsep)++(-0.5*\kpsep,0)--++(-\kpina,0); 
\draw(\ksepX,-3*\ksepY+\kulV+1.5*\kpsep)++(-1*\kpsep,-0.5*\kpsep)--++(\kpsep,\kpsep)--++(-\kpsep,\kpsep);
\draw(\ksepX,-3*\ksepY)++(\kulH+1.5*\kpsep,0)--++(0,-\kpin);
\draw(\ksepX,-3*\ksepY)++(\kulH+2.5*\kpsep,0)--++(0,-\kpin);
\draw(\ksepX,-3*\ksepY)++(\kulH+1.5*\kpsep,0)++(0,-\kpin)++(-0.5*\kpsep,0.5*\kpsep)--++(\kpsep,-\kpsep)--++(\kpsep,\kpsep);

\draw(\ksepX,-4*\ksepY)  [thick] rectangle ++(\kdimX,\kdimY);%node[pos=0.5,rectangle,inner sep=0pt,text width=1cm,align=center]{\RTL{تختی}};
\end{scope}
\end{tikzpicture}
\caption{}
\end{subfigure}\hfill
\begin{subfigure}{0.30\textwidth}
\centering
\begin{tikzpicture}
\pgfmathsetmacro{\klshift}{0.25}
\pgfmathsetmacro{\knshift}{0.07}
\pgfmathsetmacro{\kmv}{0.15}
\pgfmathsetmacro{\knshift}{0.07}
\pgfmathsetmacro{\kpin}{0.30}
\pgfmathsetmacro{\kpina}{0.30}
\pgfmathsetmacro{\kpsep}{0.15}			%pin to pin distance
\pgfmathsetmacro{\kW}{\kpsep}
\pgfmathsetmacro{\kulV}{0.20}			%edge clearance along vertical edge
\pgfmathsetmacro{\kulH}{0.20}
\pgfmathsetmacro{\kdimX}{2*\kulH+4*\kpsep}
\pgfmathsetmacro{\kdimY}{2*\kulV+4*\kpsep}		%two spaces between 3 pins
\pgfmathsetmacro{\ksepX}{\kdimX+2*\kpina+1*\kpsep+\kW}
\pgfmathsetmacro{\ksepY}{\kdimY+\kpin+0.5*\kpsep}
\draw(0,0)  [thick] rectangle ++(\kdimX,\kdimY)node[pos=0.5,rectangle,inner sep=0pt,text width=1.5cm,align=center]{\RTL{گنتکار}};
\draw(0,\kulV+4*\kpsep)--++(-\kpin,0)node[left]{$C_P$};
\draw[inactivePart](\kdimX,\kulV+1.5*\kpsep)--++(\kpina,0);
\draw[inactivePart](\kdimX,\kulV+2.5*\kpsep)--++(\kpina,0);
\draw[inactivePart](\kdimX,\kulV+1.5*\kpsep)++(\kpina,0)++(-0.5*\kpsep,-0.5*\kpsep)--++(1*\kpsep,1*\kpsep)--++(-1*\kpsep,1*\kpsep);

\begin{scope}[inactivePart]
\draw(0,-\ksepY)  [thick] rectangle ++(\kdimX,\kdimY);%node[pos=0.5,rectangle,inner sep=0pt,text width=1cm,align=center]{\RTL{دفتر پتہ}};
\draw(\kulH+0*\kpsep,-\ksepY)--++(0,-\kpin);
\draw(\kulH+1*\kpsep,-\ksepY)--++(0,-\kpin);
\draw(\kulH+0*\kpsep,-\ksepY)++(0,-\kpin)++(-0.5*\kpsep,0.5*\kpsep)--++(\kpsep,-\kpsep)--++(\kpsep,\kpsep);
\draw(\kulH+3*\kpsep,-\ksepY)--++(0,-\kpin);
\draw(\kulH+4*\kpsep,-\ksepY)--++(0,-\kpin);
\draw(\kulH+3*\kpsep,-\ksepY)++(0,-\kpin)++(-0.5*\kpsep,0.5*\kpsep)--++(\kpsep,-\kpsep)--++(\kpsep,\kpsep);
\draw(\kdimX,-\ksepY+\kulV+1.5*\kpsep)++(0.5*\kpsep,0)--++(\kpina,0);
\draw(\kdimX,-\ksepY+\kulV+2.5*\kpsep)++(0.5*\kpsep,0)--++(\kpina,0);
\draw(\kdimX,-\ksepY+\kulV+2*\kpsep)--++(1*\kpsep,-1*\kpsep);
\draw(\kdimX,-\ksepY+\kulV+2*\kpsep)--++(1*\kpsep,1*\kpsep);

\draw(0,-2*\ksepY)  [thick] rectangle ++(\kdimX,\kdimY);%node[pos=0.5,rectangle,inner sep=0pt,text width=1cm,align=center]{\footnotesize{\RTL{حافظہ}}};
\draw(\kdimX,-2*\ksepY+\kulV+1.5*\kpsep)--++(\kpina,0);
\draw(\kdimX,-2*\ksepY+\kulV+2.5*\kpsep)--++(\kpina,0);
\draw(\kdimX,-2*\ksepY+\kulV+1.5*\kpsep)++(\kpina,0)++(-0.5*\kpsep,-0.5*\kpsep)--++(1*\kpsep,1*\kpsep)--++(-1*\kpsep,1*\kpsep);

\draw(0,-3*\ksepY)  [thick] rectangle ++(\kdimX,\kdimY);%node[pos=0.5,rectangle,inner sep=0pt,text width=1cm,align=center]{ \RTL{دفتر ہدایت}};
\draw(\kdimX,-3*\ksepY+\kulV+0*\kpsep)--++(\kpina,0);
\draw(\kdimX,-3*\ksepY+\kulV+1*\kpsep)--++(\kpina,0);
\draw(\kdimX,-3*\ksepY+\kulV+0*\kpsep)++(\kpina,0)++(-0.5*\kpsep,-0.5*\kpsep)--++(1*\kpsep,1*\kpsep)--++(-1*\kpsep,1*\kpsep);
\draw(\kdimX,-3*\ksepY+\kulV+3*\kpsep)++(0.5*\kpsep,0)--++(\kpina,0);
\draw(\kdimX,-3*\ksepY+\kulV+4*\kpsep)++(0.5*\kpsep,0)--++(\kpina,0);
\draw(\kdimX,-3*\ksepY+\kulV+3.5*\kpsep)--++(1*\kpsep,-1*\kpsep);
\draw(\kdimX,-3*\ksepY+\kulV+3.5*\kpsep)--++(1*\kpsep,1*\kpsep);
\draw(\kulH+1.5*\kpsep,-3*\ksepY)--++(0,-\kpin);
\draw(\kulH+2.5*\kpsep,-3*\ksepY)--++(0,-\kpin);
\draw(\kulH+1.5*\kpsep,-3*\ksepY)++(0,-\kpin)++(-0.5*\kpsep,0.5*\kpsep)--++(\kpsep,-\kpsep)--++(\kpsep,\kpsep);
\end{scope}

\draw(0,-4*\ksepY)  [thick] rectangle ++(\kdimX,\kdimY)node[pos=0.5,rectangle,inner sep=0pt,text width=1.5cm,align=center]{\RTL{قابو    }};
\draw(\kulH+1.5*\kpsep,-4*\ksepY)--++(0,-\kpin);
\draw(\kulH+2.5*\kpsep,-4*\ksepY)--++(0,-\kpin);
\draw(\kulH+1.5*\kpsep,-4*\ksepY)++(0,-\kpin)++(-0.5*\kpsep,0.5*\kpsep)--++(\kpsep,-\kpsep)--++(\kpsep,\kpsep);
\draw(\kulH+2*\kpsep,-4*\ksepY)++(0,-1*\kpin-\kpsep)node[below,rectangle,inner sep=0pt, text width=1cm]{\RTL{قابو لفظ}};

\begin{scope}[inactivePart]
\draw(\kdimX+\kpina+0.5*\kpsep,\kdimY)  [thick] rectangle ++(\kW,-4*\ksepY+\kpin);

\draw(\ksepX,-0*\ksepY)  [thick] rectangle ++(\kdimX,\kdimY);%node[pos=0.5,rectangle,inner sep=0pt,text width=1cm,align=center]{\RTL{$A$}};
\draw(\ksepX,-0*\ksepY+\kulV+0*\kpsep)--++(-\kpina,0); 
\draw(\ksepX,-0*\ksepY+\kulV+1*\kpsep)--++(-\kpina,0); 
\draw(\ksepX,-0*\ksepY+\kulV+0*\kpsep)++(-\kpina,0)++(0.5*\kpsep,-0.5*\kpsep)--++(-\kpsep,\kpsep)--++(\kpsep,\kpsep);
\draw(\ksepX,-0*\ksepY+\kulV+3*\kpsep)++(-0.5*\kpsep,0)--++(-\kpina,0); 
\draw(\ksepX,-0*\ksepY+\kulV+4*\kpsep)++(-0.5*\kpsep,0)--++(-\kpina,0); 
\draw(\ksepX,-0*\ksepY+\kulV+3*\kpsep)++(-1*\kpsep,-0.5*\kpsep)--++(\kpsep,\kpsep)--++(-\kpsep,\kpsep);
\draw(\ksepX+\kulH+1.5*\kpsep,-0*\ksepY)--++(0,-\kpin);
\draw(\ksepX+\kulH+2.5*\kpsep,-0*\ksepY)--++(0,-\kpin);
\draw(\ksepX+\kulH+1.5*\kpsep,-0*\ksepY)++(0,-\kpin)++(-0.5*\kpsep,0.5*\kpsep)--++(\kpsep,-\kpsep)--++(\kpsep,\kpsep);

\draw(\ksepX,-1*\ksepY)  [thick] rectangle ++(\kdimX,\kdimY);%node[pos=0.5,rectangle,inner sep=0pt,text width=1cm,align=center]{\RTL{جمع و منفی کار}};
\draw(\ksepX,-1*\ksepY+\kulV+1.5*\kpsep)--++(-\kpina,0); 
\draw(\ksepX,-1*\ksepY+\kulV+2.5*\kpsep)--++(-\kpina,0); 
\draw(\ksepX,-1*\ksepY+\kulV+1.5*\kpsep)++(-\kpina,0)++(0.5*\kpsep,-0.5*\kpsep)--++(-\kpsep,\kpsep)--++(\kpsep,\kpsep);
\draw(\ksepX+\kulH+1.5*\kpsep,-1*\ksepY)++(0,-0.5*\kpsep)--++(0,-\kpin);
\draw(\ksepX+\kulH+2.5*\kpsep,-1*\ksepY)++(0,-0.5*\kpsep)--++(0,-\kpin);
\draw(\ksepX+\kulH+1.5*\kpsep,-1*\ksepY)++(-0.5*\kpsep,-\kpsep)--++(\kpsep,\kpsep)--++(\kpsep,-\kpsep);
\draw(\ksepX,-2*\ksepY)  [thick] rectangle ++(\kdimX,\kdimY);%node[pos=0.5,rectangle,inner sep=0pt,text width=1cm,align=center]{\RTL{$B$}};
\draw(\ksepX,-2*\ksepY+\kulV+1.5*\kpsep)++(-0.5*\kpsep,0)--++(-\kpina,0); 
\draw(\ksepX,-2*\ksepY+\kulV+2.5*\kpsep)++(-0.5*\kpsep,0)--++(-\kpina,0); 
\draw(\ksepX,-2*\ksepY+\kulV+1.5*\kpsep)++(-1*\kpsep,-0.5*\kpsep)--++(\kpsep,\kpsep)--++(-\kpsep,\kpsep);

\draw(\ksepX,-3*\ksepY)  [thick] rectangle ++(\kdimX,\kdimY);%node[pos=0.5,rectangle,inner sep=0pt,text width=1cm,align=center]{\RTL{خارجی دفتر}};
\draw(\ksepX,-3*\ksepY+\kulV+1.5*\kpsep)++(-0.5*\kpsep,0)--++(-\kpina,0); 
\draw(\ksepX,-3*\ksepY+\kulV+2.5*\kpsep)++(-0.5*\kpsep,0)--++(-\kpina,0); 
\draw(\ksepX,-3*\ksepY+\kulV+1.5*\kpsep)++(-1*\kpsep,-0.5*\kpsep)--++(\kpsep,\kpsep)--++(-\kpsep,\kpsep);
\draw(\ksepX,-3*\ksepY)++(\kulH+1.5*\kpsep,0)--++(0,-\kpin);
\draw(\ksepX,-3*\ksepY)++(\kulH+2.5*\kpsep,0)--++(0,-\kpin);
\draw(\ksepX,-3*\ksepY)++(\kulH+1.5*\kpsep,0)++(0,-\kpin)++(-0.5*\kpsep,0.5*\kpsep)--++(\kpsep,-\kpsep)--++(\kpsep,\kpsep);

\draw(\ksepX,-4*\ksepY)  [thick] rectangle ++(\kdimX,\kdimY);%node[pos=0.5,rectangle,inner sep=0pt,text width=1cm,align=center]{\RTL{تختی}};
\end{scope}
\end{tikzpicture}
\caption{}
\end{subfigure}\hfill
\begin{subfigure}{0.30\textwidth}
\centering
\begin{tikzpicture}
\pgfmathsetmacro{\klshift}{0.25}
\pgfmathsetmacro{\knshift}{0.07}
\pgfmathsetmacro{\kmv}{0.15}
\pgfmathsetmacro{\knshift}{0.07}
\pgfmathsetmacro{\kpin}{0.30}
\pgfmathsetmacro{\kpina}{0.30}
\pgfmathsetmacro{\kpsep}{0.15}			%pin to pin distance
\pgfmathsetmacro{\kW}{\kpsep}
\pgfmathsetmacro{\kulV}{0.20}			%edge clearance along vertical edge
\pgfmathsetmacro{\kulH}{0.20}
\pgfmathsetmacro{\kdimX}{2*\kulH+4*\kpsep}
\pgfmathsetmacro{\kdimY}{2*\kulV+4*\kpsep}		%two spaces between 3 pins
\pgfmathsetmacro{\ksepX}{\kdimX+2*\kpina+1*\kpsep+\kW}
\pgfmathsetmacro{\ksepY}{\kdimY+\kpin+0.5*\kpsep}

\begin{scope}[inactivePart]
\draw(0,0)  [thick] rectangle ++(\kdimX,\kdimY);%node[pos=0.5,rectangle,inner sep=0pt,text width=1.5cm,align=center]{\RTL{گنتکار}};
\draw(\kdimX,\kulV+1.5*\kpsep)--++(\kpina,0);
\draw(\kdimX,\kulV+2.5*\kpsep)--++(\kpina,0);
\draw(\kdimX,\kulV+1.5*\kpsep)++(\kpina,0)++(-0.5*\kpsep,-0.5*\kpsep)--++(1*\kpsep,1*\kpsep)--++(-1*\kpsep,1*\kpsep);
\end{scope}

\draw(0,-\ksepY)  [thick] rectangle ++(\kdimX,\kdimY)node[pos=0.5,rectangle,inner sep=0pt,text width=1cm,align=center]{\RTL{دفتر پتہ}};
\draw(\kulH+0*\kpsep,-\ksepY)--++(0,-\kpin);
\draw(\kulH+1*\kpsep,-\ksepY)--++(0,-\kpin);
\draw(\kulH+0*\kpsep,-\ksepY)++(0,-\kpin)++(-0.5*\kpsep,0.5*\kpsep)--++(\kpsep,-\kpsep)--++(\kpsep,\kpsep);
\draw(\kulH+3*\kpsep,-\ksepY)--++(0,-\kpin);
\draw(\kulH+4*\kpsep,-\ksepY)--++(0,-\kpin);
\draw(\kulH+3*\kpsep,-\ksepY)++(0,-\kpin)++(-0.5*\kpsep,0.5*\kpsep)--++(\kpsep,-\kpsep)--++(\kpsep,\kpsep);
\draw[inactivePart](\kdimX,-\ksepY+\kulV+1.5*\kpsep)++(0.5*\kpsep,0)--++(\kpina,0);
\draw[inactivePart](\kdimX,-\ksepY+\kulV+2.5*\kpsep)++(0.5*\kpsep,0)--++(\kpina,0);
\draw[inactivePart](\kdimX,-\ksepY+\kulV+2*\kpsep)--++(1*\kpsep,-1*\kpsep);
\draw[inactivePart](\kdimX,-\ksepY+\kulV+2*\kpsep)--++(1*\kpsep,1*\kpsep);

\draw(0,-2*\ksepY)  [thick] rectangle ++(\kdimX,\kdimY)node[pos=0.5,rectangle,inner sep=0pt,text width=1cm,align=center]{\footnotesize{\RTL{حافظہ}}};
\draw(0,-2*\ksepY+\kulV)--++(-\kpin,0)node[left]{$\overline{CE}$}; 
\draw(0,-2*\ksepY+\kulV)++(-\knshift,0)node[ocirc]{};
\draw(\kdimX,-2*\ksepY+\kulV+1.5*\kpsep)--++(\kpina,0);
\draw(\kdimX,-2*\ksepY+\kulV+2.5*\kpsep)--++(\kpina,0);
\draw(\kdimX,-2*\ksepY+\kulV+1.5*\kpsep)++(\kpina,0)++(-0.5*\kpsep,-0.5*\kpsep)--++(1*\kpsep,1*\kpsep)--++(-1*\kpsep,1*\kpsep);

\draw(0,-3*\ksepY)  [thick] rectangle ++(\kdimX,\kdimY)node[pos=0.5,rectangle,inner sep=0pt,text width=1cm,align=center]{ \RTL{دفتر ہدایت}};
\draw(0,-3*\ksepY+\kulV+4*\kpsep)--++(-\kpin,0)node[left]{$\overline{L}_I$};
\draw(0,-3*\ksepY+\kulV+4*\kpsep)++(-\knshift,0)node[ocirc]{};
\draw[inactivePart](\kdimX,-3*\ksepY+\kulV+0*\kpsep)--++(\kpina,0);
\draw[inactivePart](\kdimX,-3*\ksepY+\kulV+1*\kpsep)--++(\kpina,0);
\draw[inactivePart](\kdimX,-3*\ksepY+\kulV+0*\kpsep)++(\kpina,0)++(-0.5*\kpsep,-0.5*\kpsep)--++(1*\kpsep,1*\kpsep)--++(-1*\kpsep,1*\kpsep);
\draw(\kdimX,-3*\ksepY+\kulV+3*\kpsep)++(0.5*\kpsep,0)--++(\kpina,0);
\draw(\kdimX,-3*\ksepY+\kulV+4*\kpsep)++(0.5*\kpsep,0)--++(\kpina,0);
\draw(\kdimX,-3*\ksepY+\kulV+3.5*\kpsep)--++(1*\kpsep,-1*\kpsep);
\draw(\kdimX,-3*\ksepY+\kulV+3.5*\kpsep)--++(1*\kpsep,1*\kpsep);
\draw[inactivePart](\kulH+1.5*\kpsep,-3*\ksepY)--++(0,-\kpin);
\draw[inactivePart](\kulH+2.5*\kpsep,-3*\ksepY)--++(0,-\kpin);
\draw[inactivePart](\kulH+1.5*\kpsep,-3*\ksepY)++(0,-\kpin)++(-0.5*\kpsep,0.5*\kpsep)--++(\kpsep,-\kpsep)--++(\kpsep,\kpsep);

\draw(0,-4*\ksepY)  [thick] rectangle ++(\kdimX,\kdimY)node[pos=0.5,rectangle,inner sep=0pt,text width=1.5cm,align=center]{\RTL{قابو    }};
\draw(\kulH+1.5*\kpsep,-4*\ksepY)--++(0,-\kpin);
\draw(\kulH+2.5*\kpsep,-4*\ksepY)--++(0,-\kpin);
\draw(\kulH+1.5*\kpsep,-4*\ksepY)++(0,-\kpin)++(-0.5*\kpsep,0.5*\kpsep)--++(\kpsep,-\kpsep)--++(\kpsep,\kpsep);
\draw(\kulH+2*\kpsep,-4*\ksepY)++(0,-1*\kpin-\kpsep)node[below,rectangle,inner sep=0pt, text width=1cm]{\RTL{قابو لفظ}};

\draw(\kdimX+\kpina+0.5*\kpsep,\kdimY)  [thick] rectangle ++(\kW,-4*\ksepY+\kpin);

\begin{scope}[inactivePart]
\draw(\ksepX,-0*\ksepY)  [thick] rectangle ++(\kdimX,\kdimY);%node[pos=0.5,rectangle,inner sep=0pt,text width=1cm,align=center]{\RTL{$A$}};
\draw(\ksepX,-0*\ksepY+\kulV+0*\kpsep)--++(-\kpina,0); 
\draw(\ksepX,-0*\ksepY+\kulV+1*\kpsep)--++(-\kpina,0); 
\draw(\ksepX,-0*\ksepY+\kulV+0*\kpsep)++(-\kpina,0)++(0.5*\kpsep,-0.5*\kpsep)--++(-\kpsep,\kpsep)--++(\kpsep,\kpsep);
\draw(\ksepX,-0*\ksepY+\kulV+3*\kpsep)++(-0.5*\kpsep,0)--++(-\kpina,0); 
\draw(\ksepX,-0*\ksepY+\kulV+4*\kpsep)++(-0.5*\kpsep,0)--++(-\kpina,0); 
\draw(\ksepX,-0*\ksepY+\kulV+3*\kpsep)++(-1*\kpsep,-0.5*\kpsep)--++(\kpsep,\kpsep)--++(-\kpsep,\kpsep);
\draw(\ksepX+\kulH+1.5*\kpsep,-0*\ksepY)--++(0,-\kpin);
\draw(\ksepX+\kulH+2.5*\kpsep,-0*\ksepY)--++(0,-\kpin);
\draw(\ksepX+\kulH+1.5*\kpsep,-0*\ksepY)++(0,-\kpin)++(-0.5*\kpsep,0.5*\kpsep)--++(\kpsep,-\kpsep)--++(\kpsep,\kpsep);
\draw(\ksepX,-1*\ksepY)  [thick] rectangle ++(\kdimX,\kdimY);%node[pos=0.5,rectangle,inner sep=0pt,text width=1cm,align=center]{\RTL{جمع و منفی کار}};
\draw(\ksepX,-1*\ksepY+\kulV+1.5*\kpsep)--++(-\kpina,0); 
\draw(\ksepX,-1*\ksepY+\kulV+2.5*\kpsep)--++(-\kpina,0); 
\draw(\ksepX,-1*\ksepY+\kulV+1.5*\kpsep)++(-\kpina,0)++(0.5*\kpsep,-0.5*\kpsep)--++(-\kpsep,\kpsep)--++(\kpsep,\kpsep);
\draw(\ksepX+\kulH+1.5*\kpsep,-1*\ksepY)++(0,-0.5*\kpsep)--++(0,-\kpin);
\draw(\ksepX+\kulH+2.5*\kpsep,-1*\ksepY)++(0,-0.5*\kpsep)--++(0,-\kpin);
\draw(\ksepX+\kulH+1.5*\kpsep,-1*\ksepY)++(-0.5*\kpsep,-\kpsep)--++(\kpsep,\kpsep)--++(\kpsep,-\kpsep);


\draw(\ksepX,-2*\ksepY)  [thick] rectangle ++(\kdimX,\kdimY);%node[pos=0.5,rectangle,inner sep=0pt,text width=1cm,align=center]{\RTL{$B$}};
\draw(\ksepX,-2*\ksepY+\kulV+1.5*\kpsep)++(-0.5*\kpsep,0)--++(-\kpina,0); 
\draw(\ksepX,-2*\ksepY+\kulV+2.5*\kpsep)++(-0.5*\kpsep,0)--++(-\kpina,0); 
\draw(\ksepX,-2*\ksepY+\kulV+1.5*\kpsep)++(-1*\kpsep,-0.5*\kpsep)--++(\kpsep,\kpsep)--++(-\kpsep,\kpsep);

\draw(\ksepX,-3*\ksepY)  [thick] rectangle ++(\kdimX,\kdimY);%node[pos=0.5,rectangle,inner sep=0pt,text width=1cm,align=center]{\RTL{خارجی دفتر}};
\draw(\ksepX,-3*\ksepY+\kulV+1.5*\kpsep)++(-0.5*\kpsep,0)--++(-\kpina,0); 
\draw(\ksepX,-3*\ksepY+\kulV+2.5*\kpsep)++(-0.5*\kpsep,0)--++(-\kpina,0); 
\draw(\ksepX,-3*\ksepY+\kulV+1.5*\kpsep)++(-1*\kpsep,-0.5*\kpsep)--++(\kpsep,\kpsep)--++(-\kpsep,\kpsep);
\draw(\ksepX,-3*\ksepY)++(\kulH+1.5*\kpsep,0)--++(0,-\kpin);
\draw(\ksepX,-3*\ksepY)++(\kulH+2.5*\kpsep,0)--++(0,-\kpin);
\draw(\ksepX,-3*\ksepY)++(\kulH+1.5*\kpsep,0)++(0,-\kpin)++(-0.5*\kpsep,0.5*\kpsep)--++(\kpsep,-\kpsep)--++(\kpsep,\kpsep);

\draw(\ksepX,-4*\ksepY)  [thick] rectangle ++(\kdimX,\kdimY);%node[pos=0.5,rectangle,inner sep=0pt,text width=1cm,align=center]{\RTL{تختی}};
\end{scope}
\end{tikzpicture}
\caption{}
\end{subfigure}
\caption{
بازیابی پھیرا: (ا) \عددی{T_1} حال؛ (ب) \عددی{T_2} حال؛ (ج) \عددی{T_3} حال۔
}
\label{شکل_کمپیوٹر_اجاگر_حصے_بازیابی_پھیرا}
\end{figure}


\جزوحصہء{بازیابی پھیرا}
پتہ حال، بڑھوتری حال، اور حافظہ حال مل کر \اصطلاح{ بازیابی پھیرا }\فرہنگ{پھیرا!بازیابی}\حاشیہب{fetch cycle}\فرہنگ{cycle!fetch} دیتے ہیں۔ پتہ حال کے دوران \عددی{E_P} اور \عددی{\overline{L}_M} فعال ہوں گے؛ یوں برنامہ گنت کار   \عددی{W} گزرگاہ کے ذریعہ  دفتر پتہ کو  تیار کرتا ہے۔ جیسا شکل  \حوالہ{شکل_کمپیوٹر_چھلا}-ج   میں دکھایا گیا،  ساعت کا مثبت کنارہ  نصف  پتہ حال  گزرنے کے بعد (یعنی  پتہ حال کے وسط میں)  آ تا ہے؛ اور یوں  گنت کار کی معلومات دفتر پتہ میں  درج کرتا ہے۔

بڑھوتری حال کے دوران صرف \عددی{C_P} قابو بِٹ فعال ہو گا۔ یہ بِٹ برنامہ گنت کار  کو ساعت کے مثبت کنارہ گننے کی اجازت دیتا ہے۔ بڑھوتری حال کے وسط میں ساعت کا مثبت کنارہ آئے گا ، جو برنامہ گنت کار کی گنتی میں  \عددی{1}  کا اضافہ کرے گا۔

حافظہ حال کے دوران \عددی{\overline{CE}} اور \عددی{\overline{L}_I} فعال ہوں گے۔یوں،  حافظہ کے مقام پتہ پر موجود  لفظ کی رسائی ، \عددی{W} گزرگاہ کے ذریعہ ، دفتر ہدایت تک    ہو گی۔ حافظہ حال کے وسط میں   ساعت کا  آنے والا مثبت کنارہ  دفتر ہدایت میں یہ لفظ  درج کرتا ہے۔

\حصہ{تعمیلی پھیرا}
اگلے تین حال ( \عددی{T_4}، \عددی{T_5}، اور \عددی{T_6} ) کمپیوٹر کا\اصطلاح{ تعمیلی پھیرا  }\فرہنگ{پھیرا!تعمیلی}\حاشیہب{execution cycle}\فرہنگ{cycle!execution}کہلاتے   ہیں۔تعمیلی پھیرا  کے دوران دفاتر میں معلومات کا انتقال  اس ہدایت پر منحصر ہے جس کی تعمیل کی جا رہی ہو۔ مثلاً،  \قول{نقل \عددی{9H}}  کی تعمیل کے دوران دفاتر میں معلومات کا انتقال \قول{جمع \عددی{BH}}  کی تعمیل کے دوران دفاتر میں معلومات کے انتقال  سے مختلف ہو گا۔آئیں     اب مختلف ہدایات  کی تعمیل  کے لئے\قول{ قابو  طریقہ کار } پر غور کریں۔

\جزوحصہء{طریق نقل}
اس گفتگو کو آگے بڑھانے کے لئے فرض کریں دفتر ہدایت میں نقل \عددی{9H} بھرا گیا ہے۔
\begin{align*}
0000\, 1001=\text{\RL{دفتر ہدایت}}
\end{align*}
جزو ہدایت \عددی{0000}  قابو و ترتیب کار  کو   \عددی{T_4} حال کے دوران جاتا ہے، جہاں  اس کی  رمز کشائی  ہو گی؛  جزو پتہ \عددی{1001} دفتر پتہ  میں ڈالا جاتا ہے۔ شکل   \حوالہ{شکل_کمپیوٹر_اجاگر_حصے_تعمیلی_پھیرا}-الف میں \عددی{T_4} حال کے دوران فعال حصے اجاگر کیے گئے ہیں۔ جیسا آپ دیکھ سکتے ہیں، \عددی{\overline{E}_I} اور \عددی{\overline{L}_M} فعال ہیں، جبکہ باقی تمام قابو بِٹ غیر فعال ہیں۔

دوران \عددی{T_5} حال ، \عددی{\overline{CE}} اور \عددی{\overline{L}_A} پست ہوں گے۔ یوں ساعت کے اگلے کنارہ  چڑھائی پر  حافظہ کے مقام پتہ  سے  مواد  کا لفظ دفتر \عددی{A}  میں نقل  ہو گا (شکل  \حوالہ{شکل_کمپیوٹر_اجاگر_حصے_تعمیلی_پھیرا}-ب  دیکھیں)۔

\عددی{T_6}\اصطلاح{ فارغ حال }\فرہنگ{حال!فارغ}\حاشیہب{nop, no operation}\فرہنگ{nop} ہے۔اس (تیسرے تعمیلی ) حال کے دوران تمام دفاتر غیر فعال ہیں (شکل  \حوالہ{شکل_کمپیوٹر_اجاگر_حصے_تعمیلی_پھیرا}-ج دیکھیں)۔ یوں قابو و ترتیب کار ایسا قابو لفظ خارج کرتا ہے جس کے تمام بِٹ غیر فعال ہوں گے۔  فارغ حال میں  کوئی کام سرانجام نہیں ہو گا۔

\begin{figure}
\centering
\begin{subfigure}{0.30\textwidth}
\centering
\begin{tikzpicture}
\pgfmathsetmacro{\klshift}{0.25}
\pgfmathsetmacro{\knshift}{0.07}
\pgfmathsetmacro{\kmv}{0.15}
\pgfmathsetmacro{\knshift}{0.07}
\pgfmathsetmacro{\kpin}{0.30}
\pgfmathsetmacro{\kpina}{0.30}
\pgfmathsetmacro{\kpsep}{0.15}			%pin to pin distance
\pgfmathsetmacro{\kW}{\kpsep}
\pgfmathsetmacro{\kulV}{0.20}			%edge clearance along vertical edge
\pgfmathsetmacro{\kulH}{0.20}
\pgfmathsetmacro{\kdimX}{2*\kulH+4*\kpsep}
\pgfmathsetmacro{\kdimY}{2*\kulV+4*\kpsep}		%two spaces between 3 pins
\pgfmathsetmacro{\ksepX}{\kdimX+2*\kpina+1*\kpsep+\kW}
\pgfmathsetmacro{\ksepY}{\kdimY+\kpin+0.5*\kpsep}

\begin{scope}[inactivePart]
\draw(0,0)  [thick] rectangle ++(\kdimX,\kdimY);%node[pos=0.5,rectangle,inner sep=0pt,text width=1.5cm,align=center]{\RTL{گنتکار}};
\draw(\kdimX,\kulV+1.5*\kpsep)--++(\kpina,0);
\draw(\kdimX,\kulV+2.5*\kpsep)--++(\kpina,0);
\draw(\kdimX,\kulV+1.5*\kpsep)++(\kpina,0)++(-0.5*\kpsep,-0.5*\kpsep)--++(1*\kpsep,1*\kpsep)--++(-1*\kpsep,1*\kpsep);
\end{scope}

\draw(0,-\ksepY)  [thick] rectangle ++(\kdimX,\kdimY)node[pos=0.5,rectangle,inner sep=0pt,text width=1cm,align=center]{\RTL{دفتر پتہ}};
\draw(0,-\ksepY+\kulV+4*\kpsep)--++(-\kpin,0)node[left]{$\overline{L}_M$};
\draw(0,-\ksepY+\kulV+4*\kpsep)++(-\knshift,0)node[ocirc]{};
\draw[inactivePart](\kulH+0*\kpsep,-\ksepY)--++(0,-\kpin);
\draw[inactivePart](\kulH+1*\kpsep,-\ksepY)--++(0,-\kpin);
\draw[inactivePart](\kulH+0*\kpsep,-\ksepY)++(0,-\kpin)++(-0.5*\kpsep,0.5*\kpsep)--++(\kpsep,-\kpsep)--++(\kpsep,\kpsep);
\draw[inactivePart](\kulH+3*\kpsep,-\ksepY)--++(0,-\kpin);
\draw[inactivePart](\kulH+4*\kpsep,-\ksepY)--++(0,-\kpin);
\draw[inactivePart](\kulH+3*\kpsep,-\ksepY)++(0,-\kpin)++(-0.5*\kpsep,0.5*\kpsep)--++(\kpsep,-\kpsep)--++(\kpsep,\kpsep);
\draw(\kdimX,-\ksepY+\kulV+1.5*\kpsep)++(0.5*\kpsep,0)--++(\kpina,0);
\draw(\kdimX,-\ksepY+\kulV+2.5*\kpsep)++(0.5*\kpsep,0)--++(\kpina,0);
\draw(\kdimX,-\ksepY+\kulV+2*\kpsep)--++(1*\kpsep,-1*\kpsep);
\draw(\kdimX,-\ksepY+\kulV+2*\kpsep)--++(1*\kpsep,1*\kpsep);

\begin{scope}[inactivePart]
\draw [thick,](0,-2*\ksepY)  rectangle ++(\kdimX,\kdimY);%node[pos=0.5,rectangle,inner sep=0pt,text width=1cm,align=center]{\footnotesize{\RTL{حافظہ}}};
\draw(\kdimX,-2*\ksepY+\kulV+1.5*\kpsep)--++(\kpina,0);
\draw(\kdimX,-2*\ksepY+\kulV+2.5*\kpsep)--++(\kpina,0);
\draw(\kdimX,-2*\ksepY+\kulV+1.5*\kpsep)++(\kpina,0)++(-0.5*\kpsep,-0.5*\kpsep)--++(1*\kpsep,1*\kpsep)--++(-1*\kpsep,1*\kpsep);
\end{scope}

\draw[thick,](0,-3*\ksepY)  rectangle ++(\kdimX,\kdimY)node[pos=0.5,rectangle,inner sep=0pt,text width=1cm,align=center]{\RTL{دفتر ہدایت}};
\draw(0,-3*\ksepY+\kulV+0*\kpsep)--++(-\kpin,0)node[left]{$\overline{E}_I$};
\draw(0,-3*\ksepY+\kulV+0*\kpsep)++(-0.07,0)node[ocirc]{};
\draw(\kdimX,-3*\ksepY+\kulV+0*\kpsep)--++(\kpina,0);
\draw(\kdimX,-3*\ksepY+\kulV+1*\kpsep)--++(\kpina,0);
\draw(\kdimX,-3*\ksepY+\kulV+0*\kpsep)++(\kpina,0)++(-0.5*\kpsep,-0.5*\kpsep)--++(1*\kpsep,1*\kpsep)--++(-1*\kpsep,1*\kpsep);
\draw[inactivePart](\kdimX,-3*\ksepY+\kulV+3*\kpsep)++(0.5*\kpsep,0)--++(\kpina,0);
\draw[inactivePart](\kdimX,-3*\ksepY+\kulV+4*\kpsep)++(0.5*\kpsep,0)--++(\kpina,0);
\draw[inactivePart](\kdimX,-3*\ksepY+\kulV+3.5*\kpsep)--++(1*\kpsep,-1*\kpsep);
\draw[inactivePart](\kdimX,-3*\ksepY+\kulV+3.5*\kpsep)--++(1*\kpsep,1*\kpsep);
\draw(\kulH+1.5*\kpsep,-3*\ksepY)--++(0,-\kpin);
\draw(\kulH+2.5*\kpsep,-3*\ksepY)--++(0,-\kpin);
\draw(\kulH+1.5*\kpsep,-3*\ksepY)++(0,-\kpin)++(-0.5*\kpsep,0.5*\kpsep)--++(\kpsep,-\kpsep)--++(\kpsep,\kpsep);

\draw(0,-4*\ksepY)  [thick] rectangle ++(\kdimX,\kdimY)node[pos=0.5,rectangle,inner sep=0pt,text width=1.5cm,align=center]{\RTL{قابو    }};
\draw(\kulH+1.5*\kpsep,-4*\ksepY)--++(0,-\kpin);
\draw(\kulH+2.5*\kpsep,-4*\ksepY)--++(0,-\kpin);
\draw(\kulH+1.5*\kpsep,-4*\ksepY)++(0,-\kpin)++(-0.5*\kpsep,0.5*\kpsep)--++(\kpsep,-\kpsep)--++(\kpsep,\kpsep);
\draw(\kulH+2*\kpsep,-4*\ksepY)++(0,-1*\kpin-\kpsep)node[below,rectangle,inner sep=0pt, text width=1cm]{\RTL{قابو لفظ}};

\draw(\kdimX+\kpina+0.5*\kpsep,\kdimY)  [thick] rectangle ++(\kW,-4*\ksepY+\kpin); %گزرگاہ

\begin{scope}[inactivePart]
\draw(\ksepX,-0*\ksepY)  [thick] rectangle ++(\kdimX,\kdimY);%node[pos=0.5,rectangle,inner sep=0pt,text width=1cm,align=center]{\RTL{$A$}};
\draw(\ksepX,-0*\ksepY+\kulV+0*\kpsep)--++(-\kpina,0); 
\draw(\ksepX,-0*\ksepY+\kulV+1*\kpsep)--++(-\kpina,0); 
\draw(\ksepX,-0*\ksepY+\kulV+0*\kpsep)++(-\kpina,0)++(0.5*\kpsep,-0.5*\kpsep)--++(-\kpsep,\kpsep)--++(\kpsep,\kpsep);
\draw(\ksepX,-0*\ksepY+\kulV+3*\kpsep)++(-0.5*\kpsep,0)--++(-\kpina,0); 
\draw(\ksepX,-0*\ksepY+\kulV+4*\kpsep)++(-0.5*\kpsep,0)--++(-\kpina,0); 
\draw(\ksepX,-0*\ksepY+\kulV+3*\kpsep)++(-1*\kpsep,-0.5*\kpsep)--++(\kpsep,\kpsep)--++(-\kpsep,\kpsep);
\draw(\ksepX+\kulH+1.5*\kpsep,-0*\ksepY)--++(0,-\kpin);
\draw(\ksepX+\kulH+2.5*\kpsep,-0*\ksepY)--++(0,-\kpin);
\draw(\ksepX+\kulH+1.5*\kpsep,-0*\ksepY)++(0,-\kpin)++(-0.5*\kpsep,0.5*\kpsep)--++(\kpsep,-\kpsep)--++(\kpsep,\kpsep);

\draw(\ksepX,-1*\ksepY)  [thick] rectangle ++(\kdimX,\kdimY);%node[pos=0.5,rectangle,inner sep=0pt,text width=1cm,align=center]{\RTL{جمع و منفی کار}};
\draw(\ksepX,-1*\ksepY+\kulV+1.5*\kpsep)--++(-\kpina,0); 
\draw(\ksepX,-1*\ksepY+\kulV+2.5*\kpsep)--++(-\kpina,0); 
\draw(\ksepX,-1*\ksepY+\kulV+1.5*\kpsep)++(-\kpina,0)++(0.5*\kpsep,-0.5*\kpsep)--++(-\kpsep,\kpsep)--++(\kpsep,\kpsep);
\draw(\ksepX+\kulH+1.5*\kpsep,-1*\ksepY)++(0,-0.5*\kpsep)--++(0,-\kpin);
\draw(\ksepX+\kulH+2.5*\kpsep,-1*\ksepY)++(0,-0.5*\kpsep)--++(0,-\kpin);
\draw(\ksepX+\kulH+1.5*\kpsep,-1*\ksepY)++(-0.5*\kpsep,-\kpsep)--++(\kpsep,\kpsep)--++(\kpsep,-\kpsep);


\draw(\ksepX,-2*\ksepY)  [thick] rectangle ++(\kdimX,\kdimY);%node[pos=0.5,rectangle,inner sep=0pt,text width=1cm,align=center]{\RTL{$B$}};
\draw(\ksepX,-2*\ksepY+\kulV+1.5*\kpsep)++(-0.5*\kpsep,0)--++(-\kpina,0); 
\draw(\ksepX,-2*\ksepY+\kulV+2.5*\kpsep)++(-0.5*\kpsep,0)--++(-\kpina,0); 
\draw(\ksepX,-2*\ksepY+\kulV+1.5*\kpsep)++(-1*\kpsep,-0.5*\kpsep)--++(\kpsep,\kpsep)--++(-\kpsep,\kpsep);

\draw(\ksepX,-3*\ksepY)  [thick] rectangle ++(\kdimX,\kdimY);%node[pos=0.5,rectangle,inner sep=0pt,text width=1cm,align=center]{\RTL{خارجی دفتر}};
\draw(\ksepX,-3*\ksepY+\kulV+1.5*\kpsep)++(-0.5*\kpsep,0)--++(-\kpina,0); 
\draw(\ksepX,-3*\ksepY+\kulV+2.5*\kpsep)++(-0.5*\kpsep,0)--++(-\kpina,0); 
\draw(\ksepX,-3*\ksepY+\kulV+1.5*\kpsep)++(-1*\kpsep,-0.5*\kpsep)--++(\kpsep,\kpsep)--++(-\kpsep,\kpsep);
\draw(\ksepX,-3*\ksepY)++(\kulH+1.5*\kpsep,0)--++(0,-\kpin);
\draw(\ksepX,-3*\ksepY)++(\kulH+2.5*\kpsep,0)--++(0,-\kpin);
\draw(\ksepX,-3*\ksepY)++(\kulH+1.5*\kpsep,0)++(0,-\kpin)++(-0.5*\kpsep,0.5*\kpsep)--++(\kpsep,-\kpsep)--++(\kpsep,\kpsep);

\draw(\ksepX,-4*\ksepY)  [thick] rectangle ++(\kdimX,\kdimY);%node[pos=0.5,rectangle,inner sep=0pt,text width=1cm,align=center]{\RTL{تختی}};
\end{scope}
\end{tikzpicture}
\caption{}
\end{subfigure}\hfill
\begin{subfigure}{0.30\textwidth}
\centering
\begin{tikzpicture}
\pgfmathsetmacro{\klshift}{0.25}
\pgfmathsetmacro{\knshift}{0.07}
\pgfmathsetmacro{\kmv}{0.15}
\pgfmathsetmacro{\knshift}{0.07}
\pgfmathsetmacro{\kpin}{0.30}
\pgfmathsetmacro{\kpina}{0.30}
\pgfmathsetmacro{\kpsep}{0.15}			%pin to pin distance
\pgfmathsetmacro{\kW}{\kpsep}
\pgfmathsetmacro{\kulV}{0.20}			%edge clearance along vertical edge
\pgfmathsetmacro{\kulH}{0.20}
\pgfmathsetmacro{\kdimX}{2*\kulH+4*\kpsep}
\pgfmathsetmacro{\kdimY}{2*\kulV+4*\kpsep}		%two spaces between 3 pins
\pgfmathsetmacro{\ksepX}{\kdimX+2*\kpina+1*\kpsep+\kW}
\pgfmathsetmacro{\ksepY}{\kdimY+\kpin+0.5*\kpsep}

\begin{scope}[inactivePart]
\draw(0,0)  [thick] rectangle ++(\kdimX,\kdimY);%node[pos=0.5,rectangle,inner sep=0pt,text width=1.5cm,align=center]{\RTL{گنتکار}};
\draw(\kdimX,\kulV+1.5*\kpsep)--++(\kpina,0);
\draw(\kdimX,\kulV+2.5*\kpsep)--++(\kpina,0);
\draw(\kdimX,\kulV+1.5*\kpsep)++(\kpina,0)++(-0.5*\kpsep,-0.5*\kpsep)--++(1*\kpsep,1*\kpsep)--++(-1*\kpsep,1*\kpsep);
\end{scope}

\draw(0,-\ksepY)  [thick] rectangle ++(\kdimX,\kdimY)node[pos=0.5,rectangle,inner sep=0pt,text width=1cm,align=center]{\RTL{دفتر پتہ}};
\draw(\kulH+0*\kpsep,-\ksepY)--++(0,-\kpin);
\draw(\kulH+1*\kpsep,-\ksepY)--++(0,-\kpin);
\draw(\kulH+0*\kpsep,-\ksepY)++(0,-\kpin)++(-0.5*\kpsep,0.5*\kpsep)--++(\kpsep,-\kpsep)--++(\kpsep,\kpsep);
\draw(\kulH+3*\kpsep,-\ksepY)--++(0,-\kpin);
\draw(\kulH+4*\kpsep,-\ksepY)--++(0,-\kpin);
\draw(\kulH+3*\kpsep,-\ksepY)++(0,-\kpin)++(-0.5*\kpsep,0.5*\kpsep)--++(\kpsep,-\kpsep)--++(\kpsep,\kpsep);
\draw[inactivePart](\kdimX,-\ksepY+\kulV+1.5*\kpsep)++(0.5*\kpsep,0)--++(\kpina,0);
\draw[inactivePart](\kdimX,-\ksepY+\kulV+2.5*\kpsep)++(0.5*\kpsep,0)--++(\kpina,0);
\draw[inactivePart](\kdimX,-\ksepY+\kulV+2*\kpsep)--++(1*\kpsep,-1*\kpsep);
\draw[inactivePart](\kdimX,-\ksepY+\kulV+2*\kpsep)--++(1*\kpsep,1*\kpsep);

\draw(0,-2*\ksepY)  [thick] rectangle ++(\kdimX,\kdimY)node[pos=0.5,rectangle,inner sep=0pt,text width=1cm,align=center]{\footnotesize{\RTL{حافظہ}}};
\draw(0,-2*\ksepY+\kulV+0*\kpsep)--++(-\kpin,0)node[left]{$\overline{CE}$};
\draw(0,-2*\ksepY+\kulV+0*\kpsep)++(-0.07,0)node[ocirc]{};
\draw(\kdimX,-2*\ksepY+\kulV+1.5*\kpsep)--++(\kpina,0);
\draw(\kdimX,-2*\ksepY+\kulV+2.5*\kpsep)--++(\kpina,0);
\draw(\kdimX,-2*\ksepY+\kulV+1.5*\kpsep)++(\kpina,0)++(-0.5*\kpsep,-0.5*\kpsep)--++(1*\kpsep,1*\kpsep)--++(-1*\kpsep,1*\kpsep);

\draw(0,-3*\ksepY)  [thick] rectangle ++(\kdimX,\kdimY)node[pos=0.5,rectangle,inner sep=0pt,text width=1cm,align=center]{ \RTL{دفتر ہدایت}};
\draw[inactivePart](\kdimX,-3*\ksepY+\kulV+0*\kpsep)--++(\kpina,0);
\draw[inactivePart](\kdimX,-3*\ksepY+\kulV+1*\kpsep)--++(\kpina,0);
\draw[inactivePart](\kdimX,-3*\ksepY+\kulV+0*\kpsep)++(\kpina,0)++(-0.5*\kpsep,-0.5*\kpsep)--++(1*\kpsep,1*\kpsep)--++(-1*\kpsep,1*\kpsep);
\draw[inactivePart](\kdimX,-3*\ksepY+\kulV+3*\kpsep)++(0.5*\kpsep,0)--++(\kpina,0);
\draw[inactivePart](\kdimX,-3*\ksepY+\kulV+4*\kpsep)++(0.5*\kpsep,0)--++(\kpina,0);
\draw[inactivePart](\kdimX,-3*\ksepY+\kulV+3.5*\kpsep)--++(1*\kpsep,-1*\kpsep);
\draw[inactivePart](\kdimX,-3*\ksepY+\kulV+3.5*\kpsep)--++(1*\kpsep,1*\kpsep);
\draw(\kulH+1.5*\kpsep,-3*\ksepY)--++(0,-\kpin);
\draw(\kulH+2.5*\kpsep,-3*\ksepY)--++(0,-\kpin);
\draw(\kulH+1.5*\kpsep,-3*\ksepY)++(0,-\kpin)++(-0.5*\kpsep,0.5*\kpsep)--++(\kpsep,-\kpsep)--++(\kpsep,\kpsep);


\draw(0,-4*\ksepY)  [thick] rectangle ++(\kdimX,\kdimY)node[pos=0.5,rectangle,inner sep=0pt,text width=1.5cm,align=center]{\RTL{قابو    }};
\draw(\kulH+1.5*\kpsep,-4*\ksepY)--++(0,-\kpin);
\draw(\kulH+2.5*\kpsep,-4*\ksepY)--++(0,-\kpin);
\draw(\kulH+1.5*\kpsep,-4*\ksepY)++(0,-\kpin)++(-0.5*\kpsep,0.5*\kpsep)--++(\kpsep,-\kpsep)--++(\kpsep,\kpsep);
\draw(\kulH+2*\kpsep,-4*\ksepY)++(0,-1*\kpin-\kpsep)node[below,rectangle,inner sep=0pt, text width=1cm]{\RTL{قابو لفظ}};


\draw(\kdimX+\kpina+0.5*\kpsep,\kdimY)  [thick] rectangle ++(\kW,-4*\ksepY+\kpin);

\draw(\ksepX,-0*\ksepY)  [thick] rectangle ++(\kdimX,\kdimY)node[pos=0.5,rectangle,inner sep=0pt,text width=1cm,align=center]{\RTL{$A$}};
\draw(\ksepX+\kdimX,-0*\ksepY+\kulV+4*\kpsep)--++(\kpin,0)node[right]{$\overline{L}_A$}; 
\draw(\ksepX+\kdimX,-0*\ksepY+\kulV+4*\kpsep)++(0.07,0)node[ocirc]{};
\draw[inactivePart](\ksepX,-0*\ksepY+\kulV+0*\kpsep)--++(-\kpina,0); 
\draw[inactivePart](\ksepX,-0*\ksepY+\kulV+1*\kpsep)--++(-\kpina,0); 
\draw[inactivePart](\ksepX,-0*\ksepY+\kulV+0*\kpsep)++(-\kpina,0)++(0.5*\kpsep,-0.5*\kpsep)--++(-\kpsep,\kpsep)--++(\kpsep,\kpsep);
\draw(\ksepX,-0*\ksepY+\kulV+3*\kpsep)++(-0.5*\kpsep,0)--++(-\kpina,0); 
\draw(\ksepX,-0*\ksepY+\kulV+4*\kpsep)++(-0.5*\kpsep,0)--++(-\kpina,0); 
\draw(\ksepX,-0*\ksepY+\kulV+3*\kpsep)++(-1*\kpsep,-0.5*\kpsep)--++(\kpsep,\kpsep)--++(-\kpsep,\kpsep);
\draw[inactivePart](\ksepX+\kulH+1.5*\kpsep,-0*\ksepY)--++(0,-\kpin);
\draw[inactivePart](\ksepX+\kulH+2.5*\kpsep,-0*\ksepY)--++(0,-\kpin);
\draw[inactivePart](\ksepX+\kulH+1.5*\kpsep,-0*\ksepY)++(0,-\kpin)++(-0.5*\kpsep,0.5*\kpsep)--++(\kpsep,-\kpsep)--++(\kpsep,\kpsep);

\begin{scope}[inactivePart]
\draw(\ksepX,-1*\ksepY)  [thick] rectangle ++(\kdimX,\kdimY);%node[pos=0.5,rectangle,inner sep=0pt,text width=1cm,align=center]{\RTL{جمع و منفی کار}};
\draw(\ksepX,-1*\ksepY+\kulV+1.5*\kpsep)--++(-\kpina,0); 
\draw(\ksepX,-1*\ksepY+\kulV+2.5*\kpsep)--++(-\kpina,0); 
\draw(\ksepX,-1*\ksepY+\kulV+1.5*\kpsep)++(-\kpina,0)++(0.5*\kpsep,-0.5*\kpsep)--++(-\kpsep,\kpsep)--++(\kpsep,\kpsep);
\draw(\ksepX+\kulH+1.5*\kpsep,-1*\ksepY)++(0,-0.5*\kpsep)--++(0,-\kpin);
\draw(\ksepX+\kulH+2.5*\kpsep,-1*\ksepY)++(0,-0.5*\kpsep)--++(0,-\kpin);
\draw(\ksepX+\kulH+1.5*\kpsep,-1*\ksepY)++(-0.5*\kpsep,-\kpsep)--++(\kpsep,\kpsep)--++(\kpsep,-\kpsep);
\draw(\ksepX,-2*\ksepY)  [thick] rectangle ++(\kdimX,\kdimY);%node[pos=0.5,rectangle,inner sep=0pt,text width=1cm,align=center]{\RTL{$B$}};
\draw(\ksepX,-2*\ksepY+\kulV+1.5*\kpsep)++(-0.5*\kpsep,0)--++(-\kpina,0); 
\draw(\ksepX,-2*\ksepY+\kulV+2.5*\kpsep)++(-0.5*\kpsep,0)--++(-\kpina,0); 
\draw(\ksepX,-2*\ksepY+\kulV+1.5*\kpsep)++(-1*\kpsep,-0.5*\kpsep)--++(\kpsep,\kpsep)--++(-\kpsep,\kpsep);

\draw(\ksepX,-3*\ksepY)  [thick] rectangle ++(\kdimX,\kdimY);%node[pos=0.5,rectangle,inner sep=0pt,text width=1cm,align=center]{\RTL{خارجی دفتر}};
\draw(\ksepX,-3*\ksepY+\kulV+1.5*\kpsep)++(-0.5*\kpsep,0)--++(-\kpina,0); 
\draw(\ksepX,-3*\ksepY+\kulV+2.5*\kpsep)++(-0.5*\kpsep,0)--++(-\kpina,0); 
\draw(\ksepX,-3*\ksepY+\kulV+1.5*\kpsep)++(-1*\kpsep,-0.5*\kpsep)--++(\kpsep,\kpsep)--++(-\kpsep,\kpsep);
\draw(\ksepX,-3*\ksepY)++(\kulH+1.5*\kpsep,0)--++(0,-\kpin);
\draw(\ksepX,-3*\ksepY)++(\kulH+2.5*\kpsep,0)--++(0,-\kpin);
\draw(\ksepX,-3*\ksepY)++(\kulH+1.5*\kpsep,0)++(0,-\kpin)++(-0.5*\kpsep,0.5*\kpsep)--++(\kpsep,-\kpsep)--++(\kpsep,\kpsep);

\draw(\ksepX,-4*\ksepY)  [thick] rectangle ++(\kdimX,\kdimY);%node[pos=0.5,rectangle,inner sep=0pt,text width=1cm,align=center]{\RTL{تختی}};
\end{scope}
\end{tikzpicture}
\caption{}
\end{subfigure}\hfill
\begin{subfigure}{0.30\textwidth}
\centering
\begin{tikzpicture}
\pgfmathsetmacro{\klshift}{0.25}
\pgfmathsetmacro{\knshift}{0.07}
\pgfmathsetmacro{\kmv}{0.15}
\pgfmathsetmacro{\knshift}{0.07}
\pgfmathsetmacro{\kpin}{0.30}
\pgfmathsetmacro{\kpina}{0.30}
\pgfmathsetmacro{\kpsep}{0.15}			%pin to pin distance
\pgfmathsetmacro{\kW}{\kpsep}
\pgfmathsetmacro{\kulV}{0.20}			%edge clearance along vertical edge
\pgfmathsetmacro{\kulH}{0.20}
\pgfmathsetmacro{\kdimX}{2*\kulH+4*\kpsep}
\pgfmathsetmacro{\kdimY}{2*\kulV+4*\kpsep}		%two spaces between 3 pins
\pgfmathsetmacro{\ksepX}{\kdimX+2*\kpina+1*\kpsep+\kW}
\pgfmathsetmacro{\ksepY}{\kdimY+\kpin+0.5*\kpsep}

\begin{scope}[inactivePart]
\draw(0,0)  [thick] rectangle ++(\kdimX,\kdimY);%node[pos=0.5,rectangle,inner sep=0pt,text width=1.5cm,align=center]{\RTL{گنتکار}};
\draw(\kdimX,\kulV+1.5*\kpsep)--++(\kpina,0);
\draw(\kdimX,\kulV+2.5*\kpsep)--++(\kpina,0);
\draw(\kdimX,\kulV+1.5*\kpsep)++(\kpina,0)++(-0.5*\kpsep,-0.5*\kpsep)--++(1*\kpsep,1*\kpsep)--++(-1*\kpsep,1*\kpsep);


\draw(0,-\ksepY)  [thick] rectangle ++(\kdimX,\kdimY);%node[pos=0.5,rectangle,inner sep=0pt,text width=1cm,align=center]{\RTL{دفتر پتہ}};
\draw(\kulH+0*\kpsep,-\ksepY)--++(0,-\kpin);
\draw(\kulH+1*\kpsep,-\ksepY)--++(0,-\kpin);
\draw(\kulH+0*\kpsep,-\ksepY)++(0,-\kpin)++(-0.5*\kpsep,0.5*\kpsep)--++(\kpsep,-\kpsep)--++(\kpsep,\kpsep);
\draw(\kulH+3*\kpsep,-\ksepY)--++(0,-\kpin);
\draw(\kulH+4*\kpsep,-\ksepY)--++(0,-\kpin);
\draw(\kulH+3*\kpsep,-\ksepY)++(0,-\kpin)++(-0.5*\kpsep,0.5*\kpsep)--++(\kpsep,-\kpsep)--++(\kpsep,\kpsep);
\draw(\kdimX,-\ksepY+\kulV+1.5*\kpsep)++(0.5*\kpsep,0)--++(\kpina,0);
\draw(\kdimX,-\ksepY+\kulV+2.5*\kpsep)++(0.5*\kpsep,0)--++(\kpina,0);
\draw(\kdimX,-\ksepY+\kulV+2*\kpsep)--++(1*\kpsep,-1*\kpsep);
\draw(\kdimX,-\ksepY+\kulV+2*\kpsep)--++(1*\kpsep,1*\kpsep);

\draw(0,-2*\ksepY)  [thick] rectangle ++(\kdimX,\kdimY);%node[pos=0.5,rectangle,inner sep=0pt,text width=1cm,align=center]{\footnotesize{\RTL{حافظہ}}};
\draw(\kdimX,-2*\ksepY+\kulV+1.5*\kpsep)--++(\kpina,0);
\draw(\kdimX,-2*\ksepY+\kulV+2.5*\kpsep)--++(\kpina,0);
\draw(\kdimX,-2*\ksepY+\kulV+1.5*\kpsep)++(\kpina,0)++(-0.5*\kpsep,-0.5*\kpsep)--++(1*\kpsep,1*\kpsep)--++(-1*\kpsep,1*\kpsep);

\draw(0,-3*\ksepY)  [thick] rectangle ++(\kdimX,\kdimY);%node[pos=0.5,rectangle,inner sep=0pt,text width=1cm,align=center]{ \RTL{دفتر ہدایت}};
\draw(\kdimX,-3*\ksepY+\kulV+0*\kpsep)--++(\kpina,0);
\draw(\kdimX,-3*\ksepY+\kulV+1*\kpsep)--++(\kpina,0);
\draw(\kdimX,-3*\ksepY+\kulV+0*\kpsep)++(\kpina,0)++(-0.5*\kpsep,-0.5*\kpsep)--++(1*\kpsep,1*\kpsep)--++(-1*\kpsep,1*\kpsep);
\draw(\kdimX,-3*\ksepY+\kulV+3*\kpsep)++(0.5*\kpsep,0)--++(\kpina,0);
\draw(\kdimX,-3*\ksepY+\kulV+4*\kpsep)++(0.5*\kpsep,0)--++(\kpina,0);
\draw(\kdimX,-3*\ksepY+\kulV+3.5*\kpsep)--++(1*\kpsep,-1*\kpsep);
\draw(\kdimX,-3*\ksepY+\kulV+3.5*\kpsep)--++(1*\kpsep,1*\kpsep);
\draw(\kulH+1.5*\kpsep,-3*\ksepY)--++(0,-\kpin);
\draw(\kulH+2.5*\kpsep,-3*\ksepY)--++(0,-\kpin);
\draw(\kulH+1.5*\kpsep,-3*\ksepY)++(0,-\kpin)++(-0.5*\kpsep,0.5*\kpsep)--++(\kpsep,-\kpsep)--++(\kpsep,\kpsep);

\draw(0,-4*\ksepY)  [thick] rectangle ++(\kdimX,\kdimY);%node[pos=0.5,rectangle,inner sep=0pt,text width=1.5cm,align=center]{\RTL{قابو    }};
\end{scope}

\draw(\kulH+1.5*\kpsep,-4*\ksepY)--++(0,-\kpin);
\draw(\kulH+2.5*\kpsep,-4*\ksepY)--++(0,-\kpin);
\draw(\kulH+1.5*\kpsep,-4*\ksepY)++(0,-\kpin)++(-0.5*\kpsep,0.5*\kpsep)--++(\kpsep,-\kpsep)--++(\kpsep,\kpsep);
\draw(\kulH+2*\kpsep,-4*\ksepY)++(0,-1*\kpin-\kpsep)node[below,rectangle,inner sep=0pt, text width=1cm]{\RTL{قابو لفظ}};

\begin{scope}[inactivePart]
\draw(\kdimX+\kpina+0.5*\kpsep,\kdimY)  [thick] rectangle ++(\kW,-4*\ksepY+\kpin);


\draw(\ksepX,-0*\ksepY)  [thick] rectangle ++(\kdimX,\kdimY);%node[pos=0.5,rectangle,inner sep=0pt,text width=1cm,align=center]{\RTL{$A$}};
\draw(\ksepX,-0*\ksepY+\kulV+0*\kpsep)--++(-\kpina,0); 
\draw(\ksepX,-0*\ksepY+\kulV+1*\kpsep)--++(-\kpina,0); 
\draw(\ksepX,-0*\ksepY+\kulV+0*\kpsep)++(-\kpina,0)++(0.5*\kpsep,-0.5*\kpsep)--++(-\kpsep,\kpsep)--++(\kpsep,\kpsep);
\draw(\ksepX,-0*\ksepY+\kulV+3*\kpsep)++(-0.5*\kpsep,0)--++(-\kpina,0); 
\draw(\ksepX,-0*\ksepY+\kulV+4*\kpsep)++(-0.5*\kpsep,0)--++(-\kpina,0); 
\draw(\ksepX,-0*\ksepY+\kulV+3*\kpsep)++(-1*\kpsep,-0.5*\kpsep)--++(\kpsep,\kpsep)--++(-\kpsep,\kpsep);
\draw(\ksepX+\kulH+1.5*\kpsep,-0*\ksepY)--++(0,-\kpin);
\draw(\ksepX+\kulH+2.5*\kpsep,-0*\ksepY)--++(0,-\kpin);
\draw(\ksepX+\kulH+1.5*\kpsep,-0*\ksepY)++(0,-\kpin)++(-0.5*\kpsep,0.5*\kpsep)--++(\kpsep,-\kpsep)--++(\kpsep,\kpsep);
\draw(\ksepX,-1*\ksepY)  [thick] rectangle ++(\kdimX,\kdimY);%node[pos=0.5,rectangle,inner sep=0pt,text width=1cm,align=center]{\RTL{جمع و منفی کار}};
\draw(\ksepX,-1*\ksepY+\kulV+1.5*\kpsep)--++(-\kpina,0); 
\draw(\ksepX,-1*\ksepY+\kulV+2.5*\kpsep)--++(-\kpina,0); 
\draw(\ksepX,-1*\ksepY+\kulV+1.5*\kpsep)++(-\kpina,0)++(0.5*\kpsep,-0.5*\kpsep)--++(-\kpsep,\kpsep)--++(\kpsep,\kpsep);
\draw(\ksepX+\kulH+1.5*\kpsep,-1*\ksepY)++(0,-0.5*\kpsep)--++(0,-\kpin);
\draw(\ksepX+\kulH+2.5*\kpsep,-1*\ksepY)++(0,-0.5*\kpsep)--++(0,-\kpin);
\draw(\ksepX+\kulH+1.5*\kpsep,-1*\ksepY)++(-0.5*\kpsep,-\kpsep)--++(\kpsep,\kpsep)--++(\kpsep,-\kpsep);


\draw(\ksepX,-2*\ksepY)  [thick] rectangle ++(\kdimX,\kdimY);%node[pos=0.5,rectangle,inner sep=0pt,text width=1cm,align=center]{\RTL{$B$}};
\draw(\ksepX,-2*\ksepY+\kulV+1.5*\kpsep)++(-0.5*\kpsep,0)--++(-\kpina,0); 
\draw(\ksepX,-2*\ksepY+\kulV+2.5*\kpsep)++(-0.5*\kpsep,0)--++(-\kpina,0); 
\draw(\ksepX,-2*\ksepY+\kulV+1.5*\kpsep)++(-1*\kpsep,-0.5*\kpsep)--++(\kpsep,\kpsep)--++(-\kpsep,\kpsep);

\draw(\ksepX,-3*\ksepY)  [thick] rectangle ++(\kdimX,\kdimY);%node[pos=0.5,rectangle,inner sep=0pt,text width=1cm,align=center]{\RTL{خارجی دفتر}};
\draw(\ksepX,-3*\ksepY+\kulV+1.5*\kpsep)++(-0.5*\kpsep,0)--++(-\kpina,0); 
\draw(\ksepX,-3*\ksepY+\kulV+2.5*\kpsep)++(-0.5*\kpsep,0)--++(-\kpina,0); 
\draw(\ksepX,-3*\ksepY+\kulV+1.5*\kpsep)++(-1*\kpsep,-0.5*\kpsep)--++(\kpsep,\kpsep)--++(-\kpsep,\kpsep);
\draw(\ksepX,-3*\ksepY)++(\kulH+1.5*\kpsep,0)--++(0,-\kpin);
\draw(\ksepX,-3*\ksepY)++(\kulH+2.5*\kpsep,0)--++(0,-\kpin);
\draw(\ksepX,-3*\ksepY)++(\kulH+1.5*\kpsep,0)++(0,-\kpin)++(-0.5*\kpsep,0.5*\kpsep)--++(\kpsep,-\kpsep)--++(\kpsep,\kpsep);

\draw(\ksepX,-4*\ksepY)  [thick] rectangle ++(\kdimX,\kdimY);%node[pos=0.5,rectangle,inner sep=0pt,text width=1cm,align=center]{\RTL{تختی}};
\end{scope}
\end{tikzpicture}
\caption{}
\end{subfigure}
\caption{
طریق نقل: (ا) \عددی{T_4} حال؛ (ب) \عددی{T_5} حال؛ (ج) \عددی{T_6} حال۔
}
\label{شکل_کمپیوٹر_اجاگر_حصے_تعمیلی_پھیرا}
\end{figure}
%
\begin{figure}
\centering
\begin{otherlanguage}{english}
 \begin{tikztimingtable}[%
timing/.style={x=4ex,y=3ex},
timing/rowdist=6ex,
every node/.style={inner sep=0,outer sep=0},
%timing/c/arrow tip=latex, %and this set the style
%timing/c/rising arrows,
timing/slope=0, %0.1 is good
timing/dslope=0,
thick,
]
%\tikztimingmetachar{R}{[|/utils/exec=\setcounter{new}{0}|]}
%\usetikztiminglibrary[new={char=Q,reset char=R}]{counters}
%[timing/counter/new={char=c, base=2,digits=3,max value=7, wraps ,text style={font=\normalsize}}] 12{2c} \\ 
%$C$& H22{C}\\
%$\texturdu{\RL{پتہ}}$&3D{} 20D{[scale=1.5]\texturdu{\RL{درست پتہ}}} 3D{}\\
$CLK$&HN(a)LN(ca)HN(b)LN(cb)HN(c)LN(cc)HN(d)LN(cd)HN(e)LN(ce)HN(f)LN(cf)HN(g)L\\
$E_P$&LHHLLLLLLLLLLH\\
$\overline{L}_M$&HLL4{H}2{L}5{H}\\
$C_P$&LLLHHLLLLLLLLL\\
$\overline{CE}$&HHHHHLLHHLLHHH\\
$\overline{L}_I$&HHHHHLLHHHHHHH\\
$\overline{E}_I$&HHHHHHHLLHHHHH\\
$\overline{L}_A$&HHHHHHHHHLLHHH\\
\extracode
\begin{pgfonlayer}{background}
\begin{scope}[]
\draw [latex-latex] ($(a|-row1.north)+(0,2ex)$) --node[fill=white]{$T_1$} ($(b|-row1.north)+(0,2ex)$);
\draw [latex-latex] ($(b|-row1.north)+(0,2ex)$) --node[fill=white]{$T_2$} ($(c|-row1.north)+(0,2ex)$);
\draw [latex-latex] ($(c|-row1.north)+(0,2ex)$) --node[fill=white]{$T_3$} ($(d|-row1.north)+(0,2ex)$);
\draw [latex-latex] ($(d|-row1.north)+(0,2ex)$) --node[fill=white]{$T_4$} ($(e|-row1.north)+(0,2ex)$);
\draw [latex-latex] ($(e|-row1.north)+(0,2ex)$) --node[fill=white]{$T_5$} ($(f|-row1.north)+(0,2ex)$);
\draw [latex-latex] ($(f|-row1.north)+(0,2ex)$) --node[fill=white]{$T_6$} ($(g|-row1.north)+(0,2ex)$);
\foreach \n in {a,b,c,d,e,f,g}{\draw[thin]($(\n|-row1.north)+(0,1ex)$)--++(0,2ex);}
\draw[dashed] (ca)--(ca |-row3.south);
\draw[dashed] (cb)--(cb |-row4.north);
\draw[dashed] (cc)--(cc |-row6.south);
\draw[dashed] (cd)--(cd |-row7.south);
\draw[dashed] (ce)--(ce |-row8.south);
%%\vertlines[darkgray,dotted]{3.6,7.5,11.5,15.5,19.5,23.5,27.5}
%\foreach \n in {1,3,...,11} \draw(4*\n ex-2ex,-4ex+1.25ex)node[]{$0$};
%\foreach \n in {2,4,...,12} \draw(4*\n ex-2ex,-4ex+1.25ex)node[]{$1$};
%\foreach \n in {1,2,5,6,9,10} \draw(4*\n ex-2ex,-12 ex+1.25ex)node[]{$0$};
%\foreach \n in {3,4,7,8,11,12} \draw(4*\n ex-2ex,-12 ex+1.25ex)node[]{$1$};
%\foreach \n in {1,2,3,4,9,10,11,12} \draw(4*\n ex-2ex,-20 ex+1.25ex)node[]{$0$};
%\foreach \n in {5,6,7,8} \draw(4*\n ex-2ex,-20 ex+1.25ex)node[]{$1$};
%\draw(4*4 ex-2ex,-12 ex+1.25ex) circle (0.25cm and 1.75cm);
%\draw(4*4 ex-2ex,-7*\rowdist+1.25ex) circle (0.25cm and 0.5cm);
%%\foreach \n in {1}\draw(B\n.south)--(F\n.north);
%%\foreach \n in {1}\draw(C\n.south)--(G\n.north);
\end{scope}
\end{pgfonlayer}
\end{tikztimingtable}
\end{otherlanguage}
\caption{بازیابی اور نقل کی وقتیہ ترسیمات۔}
\label{شکل_کمپیوٹر_بازیابی_وقتیہ}
\end{figure}

شکل  \حوالہ{شکل_کمپیوٹر_بازیابی_وقتیہ}  میں بازیابی اور نقل طریق کی وقتیہ ترسیمات پیش ہیں۔ \عددی{T_1} حال  کے دوران \عددی{E_P} اور \عددی{\overline{L}_M}  فعال ہیں؛  اس حال کے وسط میں  ساعت کا آنے والا  کنارہ چڑھائی ، دفتر پتہ میں   برنامہ گنت کار سے پتہ  منتقل کرتا ہے۔ \عددی{T_2} حال  کے دوران  \عددی{C_P} فعال ہے لہٰذا ساعت کے کنارہ چڑھائی پر  برنامہ گنت کار  کی گنتی میں \عددی{1} کا اضافہ  ہو گا۔ \عددی{T_3} حال کے دوران \عددی{\overline{CE}} اور \عددی{\overline{L}_I} فعال ہیں؛ ساعت کے کنارہ  چڑھائی پر  دفتر ہدایت میں ، پتہ  کی نشاندہی پر حافظہ  کے مطلوبہ (    نشان زد )مقام سے ، لفظ بھرا جائے گا۔ \قول{نقل} کی ہدایت پر عمل درآمد   \عددی{T_4} حال سے شروع ہو گی، جہاں \عددی{\overline{L}_M} اور \عددی{\overline{E}_I} فعال  ہیں؛ دفتر ہدایت میں موجود   جزو پتہ ،  ساعت کے کنارہ چڑھائی پر،  دفتر پتہ میں  منتقل ہو گا۔ \عددی{T_5} حال کے دوران \عددی{\overline{CE}} اور \عددی{\overline{L}_A} فعال ہیں؛ دفتر \عددی{A} میں، ساعت کے کنارہ چڑھائی پر،  حافظہ کے مطلوبہ مقام سے مواد کا لفظ بھرا جائے گا۔ \قول{نقل} ہدایت میں \عددی{T_6} حال   کچھ نہیں کرتا۔ ہم کہتے ہیں یہ فارغ حال ہے۔

\جزوحصہء{طریق جمع}
فرض کریں بازیابی پھیرا کے اختتام پر دفتر ہدایت میں \قول{جمع \عددی{BH}} پایا جاتا ہے۔
\begin{align*}
0001\,1011=\text{\RL{دفتر ہدایت}}
\end{align*}
دوران \عددی{T_4} حال قابو و ترتیب کار کو   جزو ہدایت  اور دفتر پتہ کو جزو پتہ جائے گا (شکل \حوالہ{شکل_کمپیوٹر_جمع_اور_منفی}-الف  دیکھیں)۔ اس حال کے دوران \عددی{\overline{E}_I} اور \عددی{\overline{L}_M} فعال ہوں گے۔

\عددی{T_5} حال کے دوران قابو بِٹ \عددی{\overline{CE}} اور \عددی{\overline{L}_B} فعال ہوں گے۔یوں  پتہ  کی نشاندہی کے مقام پر لفظ    حافظہ  سے  دفتر \عددی{B} میں  لکھا جا سکتا ہے (شکل \حوالہ{شکل_کمپیوٹر_جمع_اور_منفی}-ب)۔ ہمیشہ کی طرح ،   اس حال کے وسط میں آنے والے ساعت کے کنارہ چڑھائی پر مواد دفتر \عددی{B} میں منتقل ہو گا۔

\عددی{T_6} حال  کے دوران، \عددی{E_U} اور \عددی{\overline{L}_A} فعال ہو ں گے؛ لہٰذا دفتر \عددی{A} تک جمع و منفی کار کا مخارج پہنچے گا (شکل  \حوالہ{شکل_کمپیوٹر_جمع_اور_منفی}-ج)۔ اس حال کے وسط میں جمع و منفی کار  کا مخارج دفتر \عددی{A} منتقل ہو گا۔

\begin{figure}
\centering
\begin{subfigure}{0.30\textwidth}
\centering
\begin{tikzpicture}
\pgfmathsetmacro{\klshift}{0.25}
\pgfmathsetmacro{\knshift}{0.07}
\pgfmathsetmacro{\kmv}{0.15}
\pgfmathsetmacro{\knshift}{0.07}
\pgfmathsetmacro{\kpin}{0.30}
\pgfmathsetmacro{\kpina}{0.30}
\pgfmathsetmacro{\kpsep}{0.15}			%pin to pin distance
\pgfmathsetmacro{\kW}{\kpsep}
\pgfmathsetmacro{\kulV}{0.20}			%edge clearance along vertical edge
\pgfmathsetmacro{\kulH}{0.20}
\pgfmathsetmacro{\kdimX}{2*\kulH+4*\kpsep}
\pgfmathsetmacro{\kdimY}{2*\kulV+4*\kpsep}		%two spaces between 3 pins
\pgfmathsetmacro{\ksepX}{\kdimX+2*\kpina+1*\kpsep+\kW}
\pgfmathsetmacro{\ksepY}{\kdimY+\kpin+0.5*\kpsep}

\begin{scope}[inactivePart]
\draw(0,0)  [thick] rectangle ++(\kdimX,\kdimY);%node[pos=0.5,rectangle,inner sep=0pt,text width=1.5cm,align=center]{\RTL{گنتکار}};
\draw(\kdimX,\kulV+1.5*\kpsep)--++(\kpina,0);
\draw(\kdimX,\kulV+2.5*\kpsep)--++(\kpina,0);
\draw(\kdimX,\kulV+1.5*\kpsep)++(\kpina,0)++(-0.5*\kpsep,-0.5*\kpsep)--++(1*\kpsep,1*\kpsep)--++(-1*\kpsep,1*\kpsep);
\end{scope}

\draw(0,-\ksepY)  [thick] rectangle ++(\kdimX,\kdimY)node[pos=0.5,rectangle,inner sep=0pt,text width=1cm,align=center]{\RTL{دفتر پتہ}};
\draw(0,-\ksepY+\kulV+4*\kpsep)--++(-\kpin,0)node[left]{$\overline{L}_M$};
\draw(0,-\ksepY+\kulV+4*\kpsep)++(-\knshift,0)node[ocirc]{};
\draw[inactivePart](\kulH+0*\kpsep,-\ksepY)--++(0,-\kpin);
\draw[inactivePart](\kulH+1*\kpsep,-\ksepY)--++(0,-\kpin);
\draw[inactivePart](\kulH+0*\kpsep,-\ksepY)++(0,-\kpin)++(-0.5*\kpsep,0.5*\kpsep)--++(\kpsep,-\kpsep)--++(\kpsep,\kpsep);
\draw[inactivePart](\kulH+3*\kpsep,-\ksepY)--++(0,-\kpin);
\draw[inactivePart](\kulH+4*\kpsep,-\ksepY)--++(0,-\kpin);
\draw[inactivePart](\kulH+3*\kpsep,-\ksepY)++(0,-\kpin)++(-0.5*\kpsep,0.5*\kpsep)--++(\kpsep,-\kpsep)--++(\kpsep,\kpsep);
\draw(\kdimX,-\ksepY+\kulV+1.5*\kpsep)++(0.5*\kpsep,0)--++(\kpina,0);
\draw(\kdimX,-\ksepY+\kulV+2.5*\kpsep)++(0.5*\kpsep,0)--++(\kpina,0);
\draw(\kdimX,-\ksepY+\kulV+2*\kpsep)--++(1*\kpsep,-1*\kpsep);
\draw(\kdimX,-\ksepY+\kulV+2*\kpsep)--++(1*\kpsep,1*\kpsep);

\begin{scope}[inactivePart]
\draw [thick,](0,-2*\ksepY)  rectangle ++(\kdimX,\kdimY);%node[pos=0.5,rectangle,inner sep=0pt,text width=1cm,align=center]{\footnotesize{\RTL{حافظہ}}};
\draw(\kdimX,-2*\ksepY+\kulV+1.5*\kpsep)--++(\kpina,0);
\draw(\kdimX,-2*\ksepY+\kulV+2.5*\kpsep)--++(\kpina,0);
\draw(\kdimX,-2*\ksepY+\kulV+1.5*\kpsep)++(\kpina,0)++(-0.5*\kpsep,-0.5*\kpsep)--++(1*\kpsep,1*\kpsep)--++(-1*\kpsep,1*\kpsep);
\end{scope}

\draw[thick,](0,-3*\ksepY)  rectangle ++(\kdimX,\kdimY)node[pos=0.5,rectangle,inner sep=0pt,text width=1cm,align=center]{\RTL{دفتر ہدایت}};
\draw(0,-3*\ksepY+\kulV+0*\kpsep)--++(-\kpin,0)node[left]{$\overline{E}_I$};
\draw(0,-3*\ksepY+\kulV+0*\kpsep)++(-0.07,0)node[ocirc]{};
\draw(\kdimX,-3*\ksepY+\kulV+0*\kpsep)--++(\kpina,0);
\draw(\kdimX,-3*\ksepY+\kulV+1*\kpsep)--++(\kpina,0);
\draw(\kdimX,-3*\ksepY+\kulV+0*\kpsep)++(\kpina,0)++(-0.5*\kpsep,-0.5*\kpsep)--++(1*\kpsep,1*\kpsep)--++(-1*\kpsep,1*\kpsep);
\draw[inactivePart](\kdimX,-3*\ksepY+\kulV+3*\kpsep)++(0.5*\kpsep,0)--++(\kpina,0);
\draw[inactivePart](\kdimX,-3*\ksepY+\kulV+4*\kpsep)++(0.5*\kpsep,0)--++(\kpina,0);
\draw[inactivePart](\kdimX,-3*\ksepY+\kulV+3.5*\kpsep)--++(1*\kpsep,-1*\kpsep);
\draw[inactivePart](\kdimX,-3*\ksepY+\kulV+3.5*\kpsep)--++(1*\kpsep,1*\kpsep);
\draw(\kulH+1.5*\kpsep,-3*\ksepY)--++(0,-\kpin);
\draw(\kulH+2.5*\kpsep,-3*\ksepY)--++(0,-\kpin);
\draw(\kulH+1.5*\kpsep,-3*\ksepY)++(0,-\kpin)++(-0.5*\kpsep,0.5*\kpsep)--++(\kpsep,-\kpsep)--++(\kpsep,\kpsep);

\draw(0,-4*\ksepY)  [thick] rectangle ++(\kdimX,\kdimY)node[pos=0.5,rectangle,inner sep=0pt,text width=1.5cm,align=center]{\RTL{قابو    }};
\draw(\kulH+1.5*\kpsep,-4*\ksepY)--++(0,-\kpin);
\draw(\kulH+2.5*\kpsep,-4*\ksepY)--++(0,-\kpin);
\draw(\kulH+1.5*\kpsep,-4*\ksepY)++(0,-\kpin)++(-0.5*\kpsep,0.5*\kpsep)--++(\kpsep,-\kpsep)--++(\kpsep,\kpsep);
\draw(\kulH+2*\kpsep,-4*\ksepY)++(0,-1*\kpin-\kpsep)node[below,rectangle,inner sep=0pt, text width=1cm]{\RTL{قابو لفظ}};

\draw(\kdimX+\kpina+0.5*\kpsep,\kdimY)  [thick] rectangle ++(\kW,-4*\ksepY+\kpin); %گزرگاہ

\begin{scope}[inactivePart]
\draw(\ksepX,-0*\ksepY)  [thick] rectangle ++(\kdimX,\kdimY);%node[pos=0.5,rectangle,inner sep=0pt,text width=1cm,align=center]{\RTL{$A$}};
\draw(\ksepX,-0*\ksepY+\kulV+0*\kpsep)--++(-\kpina,0); 
\draw(\ksepX,-0*\ksepY+\kulV+1*\kpsep)--++(-\kpina,0); 
\draw(\ksepX,-0*\ksepY+\kulV+0*\kpsep)++(-\kpina,0)++(0.5*\kpsep,-0.5*\kpsep)--++(-\kpsep,\kpsep)--++(\kpsep,\kpsep);
\draw(\ksepX,-0*\ksepY+\kulV+3*\kpsep)++(-0.5*\kpsep,0)--++(-\kpina,0); 
\draw(\ksepX,-0*\ksepY+\kulV+4*\kpsep)++(-0.5*\kpsep,0)--++(-\kpina,0); 
\draw(\ksepX,-0*\ksepY+\kulV+3*\kpsep)++(-1*\kpsep,-0.5*\kpsep)--++(\kpsep,\kpsep)--++(-\kpsep,\kpsep);
\draw(\ksepX+\kulH+1.5*\kpsep,-0*\ksepY)--++(0,-\kpin);
\draw(\ksepX+\kulH+2.5*\kpsep,-0*\ksepY)--++(0,-\kpin);
\draw(\ksepX+\kulH+1.5*\kpsep,-0*\ksepY)++(0,-\kpin)++(-0.5*\kpsep,0.5*\kpsep)--++(\kpsep,-\kpsep)--++(\kpsep,\kpsep);

\draw(\ksepX,-1*\ksepY)  [thick] rectangle ++(\kdimX,\kdimY);%node[pos=0.5,rectangle,inner sep=0pt,text width=1cm,align=center]{\RTL{جمع و منفی کار}};
\draw(\ksepX,-1*\ksepY+\kulV+1.5*\kpsep)--++(-\kpina,0); 
\draw(\ksepX,-1*\ksepY+\kulV+2.5*\kpsep)--++(-\kpina,0); 
\draw(\ksepX,-1*\ksepY+\kulV+1.5*\kpsep)++(-\kpina,0)++(0.5*\kpsep,-0.5*\kpsep)--++(-\kpsep,\kpsep)--++(\kpsep,\kpsep);
\draw(\ksepX+\kulH+1.5*\kpsep,-1*\ksepY)++(0,-0.5*\kpsep)--++(0,-\kpin);
\draw(\ksepX+\kulH+2.5*\kpsep,-1*\ksepY)++(0,-0.5*\kpsep)--++(0,-\kpin);
\draw(\ksepX+\kulH+1.5*\kpsep,-1*\ksepY)++(-0.5*\kpsep,-\kpsep)--++(\kpsep,\kpsep)--++(\kpsep,-\kpsep);


\draw(\ksepX,-2*\ksepY)  [thick] rectangle ++(\kdimX,\kdimY);%node[pos=0.5,rectangle,inner sep=0pt,text width=1cm,align=center]{\RTL{$B$}};
\draw(\ksepX,-2*\ksepY+\kulV+1.5*\kpsep)++(-0.5*\kpsep,0)--++(-\kpina,0); 
\draw(\ksepX,-2*\ksepY+\kulV+2.5*\kpsep)++(-0.5*\kpsep,0)--++(-\kpina,0); 
\draw(\ksepX,-2*\ksepY+\kulV+1.5*\kpsep)++(-1*\kpsep,-0.5*\kpsep)--++(\kpsep,\kpsep)--++(-\kpsep,\kpsep);

\draw(\ksepX,-3*\ksepY)  [thick] rectangle ++(\kdimX,\kdimY);%node[pos=0.5,rectangle,inner sep=0pt,text width=1cm,align=center]{\RTL{خارجی دفتر}};
\draw(\ksepX,-3*\ksepY+\kulV+1.5*\kpsep)++(-0.5*\kpsep,0)--++(-\kpina,0); 
\draw(\ksepX,-3*\ksepY+\kulV+2.5*\kpsep)++(-0.5*\kpsep,0)--++(-\kpina,0); 
\draw(\ksepX,-3*\ksepY+\kulV+1.5*\kpsep)++(-1*\kpsep,-0.5*\kpsep)--++(\kpsep,\kpsep)--++(-\kpsep,\kpsep);
\draw(\ksepX,-3*\ksepY)++(\kulH+1.5*\kpsep,0)--++(0,-\kpin);
\draw(\ksepX,-3*\ksepY)++(\kulH+2.5*\kpsep,0)--++(0,-\kpin);
\draw(\ksepX,-3*\ksepY)++(\kulH+1.5*\kpsep,0)++(0,-\kpin)++(-0.5*\kpsep,0.5*\kpsep)--++(\kpsep,-\kpsep)--++(\kpsep,\kpsep);

\draw(\ksepX,-4*\ksepY)  [thick] rectangle ++(\kdimX,\kdimY);%node[pos=0.5,rectangle,inner sep=0pt,text width=1cm,align=center]{\RTL{تختی}};
\end{scope}
\end{tikzpicture}
\caption{}
\end{subfigure}\hfill
\begin{subfigure}{0.30\textwidth}
\centering
\begin{tikzpicture}
\pgfmathsetmacro{\klshift}{0.25}
\pgfmathsetmacro{\knshift}{0.07}
\pgfmathsetmacro{\kmv}{0.15}
\pgfmathsetmacro{\knshift}{0.07}
\pgfmathsetmacro{\kpin}{0.30}
\pgfmathsetmacro{\kpina}{0.30}
\pgfmathsetmacro{\kpsep}{0.15}			%pin to pin distance
\pgfmathsetmacro{\kW}{\kpsep}
\pgfmathsetmacro{\kulV}{0.20}			%edge clearance along vertical edge
\pgfmathsetmacro{\kulH}{0.20}
\pgfmathsetmacro{\kdimX}{2*\kulH+4*\kpsep}
\pgfmathsetmacro{\kdimY}{2*\kulV+4*\kpsep}		%two spaces between 3 pins
\pgfmathsetmacro{\ksepX}{\kdimX+2*\kpina+1*\kpsep+\kW}
\pgfmathsetmacro{\ksepY}{\kdimY+\kpin+0.5*\kpsep}

\begin{scope}[inactivePart]
\draw(0,0)  [thick] rectangle ++(\kdimX,\kdimY);%node[pos=0.5,rectangle,inner sep=0pt,text width=1.5cm,align=center]{\RTL{گنتکار}};
\draw(\kdimX,\kulV+1.5*\kpsep)--++(\kpina,0);
\draw(\kdimX,\kulV+2.5*\kpsep)--++(\kpina,0);
\draw(\kdimX,\kulV+1.5*\kpsep)++(\kpina,0)++(-0.5*\kpsep,-0.5*\kpsep)--++(1*\kpsep,1*\kpsep)--++(-1*\kpsep,1*\kpsep);
\end{scope}

\draw(0,-\ksepY)  [thick] rectangle ++(\kdimX,\kdimY)node[pos=0.5,rectangle,inner sep=0pt,text width=1cm,align=center]{\RTL{دفتر پتہ}};
\draw(\kulH+0*\kpsep,-\ksepY)--++(0,-\kpin);
\draw(\kulH+1*\kpsep,-\ksepY)--++(0,-\kpin);
\draw(\kulH+0*\kpsep,-\ksepY)++(0,-\kpin)++(-0.5*\kpsep,0.5*\kpsep)--++(\kpsep,-\kpsep)--++(\kpsep,\kpsep);
\draw(\kulH+3*\kpsep,-\ksepY)--++(0,-\kpin);
\draw(\kulH+4*\kpsep,-\ksepY)--++(0,-\kpin);
\draw(\kulH+3*\kpsep,-\ksepY)++(0,-\kpin)++(-0.5*\kpsep,0.5*\kpsep)--++(\kpsep,-\kpsep)--++(\kpsep,\kpsep);
\draw[inactivePart](\kdimX,-\ksepY+\kulV+1.5*\kpsep)++(0.5*\kpsep,0)--++(\kpina,0);
\draw[inactivePart](\kdimX,-\ksepY+\kulV+2.5*\kpsep)++(0.5*\kpsep,0)--++(\kpina,0);
\draw[inactivePart](\kdimX,-\ksepY+\kulV+2*\kpsep)--++(1*\kpsep,-1*\kpsep);
\draw[inactivePart](\kdimX,-\ksepY+\kulV+2*\kpsep)--++(1*\kpsep,1*\kpsep);

\draw(0,-2*\ksepY)  [thick] rectangle ++(\kdimX,\kdimY)node[pos=0.5,rectangle,inner sep=0pt,text width=1cm,align=center]{\footnotesize{\RTL{حافظہ}}};
\draw(0,-2*\ksepY+\kulV+0*\kpsep)--++(-\kpin,0)node[left]{$\overline{CE}$};
\draw(0,-2*\ksepY+\kulV+0*\kpsep)++(-0.07,0)node[ocirc]{};
\draw(\kdimX,-2*\ksepY+\kulV+1.5*\kpsep)--++(\kpina,0);
\draw(\kdimX,-2*\ksepY+\kulV+2.5*\kpsep)--++(\kpina,0);
\draw(\kdimX,-2*\ksepY+\kulV+1.5*\kpsep)++(\kpina,0)++(-0.5*\kpsep,-0.5*\kpsep)--++(1*\kpsep,1*\kpsep)--++(-1*\kpsep,1*\kpsep);

\draw(0,-3*\ksepY)  [thick] rectangle ++(\kdimX,\kdimY)node[pos=0.5,rectangle,inner sep=0pt,text width=1cm,align=center]{ \RTL{دفتر ہدایت}};
\draw[inactivePart](\kdimX,-3*\ksepY+\kulV+0*\kpsep)--++(\kpina,0);
\draw[inactivePart](\kdimX,-3*\ksepY+\kulV+1*\kpsep)--++(\kpina,0);
\draw[inactivePart](\kdimX,-3*\ksepY+\kulV+0*\kpsep)++(\kpina,0)++(-0.5*\kpsep,-0.5*\kpsep)--++(1*\kpsep,1*\kpsep)--++(-1*\kpsep,1*\kpsep);
\draw[inactivePart](\kdimX,-3*\ksepY+\kulV+3*\kpsep)++(0.5*\kpsep,0)--++(\kpina,0);
\draw[inactivePart](\kdimX,-3*\ksepY+\kulV+4*\kpsep)++(0.5*\kpsep,0)--++(\kpina,0);
\draw[inactivePart](\kdimX,-3*\ksepY+\kulV+3.5*\kpsep)--++(1*\kpsep,-1*\kpsep);
\draw[inactivePart](\kdimX,-3*\ksepY+\kulV+3.5*\kpsep)--++(1*\kpsep,1*\kpsep);
\draw(\kulH+1.5*\kpsep,-3*\ksepY)--++(0,-\kpin);
\draw(\kulH+2.5*\kpsep,-3*\ksepY)--++(0,-\kpin);
\draw(\kulH+1.5*\kpsep,-3*\ksepY)++(0,-\kpin)++(-0.5*\kpsep,0.5*\kpsep)--++(\kpsep,-\kpsep)--++(\kpsep,\kpsep);


\draw(0,-4*\ksepY)  [thick] rectangle ++(\kdimX,\kdimY)node[pos=0.5,rectangle,inner sep=0pt,text width=1.5cm,align=center]{\RTL{قابو    }};
\draw(\kulH+1.5*\kpsep,-4*\ksepY)--++(0,-\kpin);
\draw(\kulH+2.5*\kpsep,-4*\ksepY)--++(0,-\kpin);
\draw(\kulH+1.5*\kpsep,-4*\ksepY)++(0,-\kpin)++(-0.5*\kpsep,0.5*\kpsep)--++(\kpsep,-\kpsep)--++(\kpsep,\kpsep);
\draw(\kulH+2*\kpsep,-4*\ksepY)++(0,-1*\kpin-\kpsep)node[below,rectangle,inner sep=0pt, text width=1cm]{\RTL{قابو لفظ}};


\draw(\kdimX+\kpina+0.5*\kpsep,\kdimY)  [thick] rectangle ++(\kW,-4*\ksepY+\kpin);

\begin{scope}[inactivePart]
\draw(\ksepX,-0*\ksepY)  [thick] rectangle ++(\kdimX,\kdimY);%node[pos=0.5,rectangle,inner sep=0pt,text width=1cm,align=center]{\RTL{$A$}};
\draw[inactivePart](\ksepX,-0*\ksepY+\kulV+0*\kpsep)--++(-\kpina,0); 
\draw[inactivePart](\ksepX,-0*\ksepY+\kulV+1*\kpsep)--++(-\kpina,0); 
\draw[inactivePart](\ksepX,-0*\ksepY+\kulV+0*\kpsep)++(-\kpina,0)++(0.5*\kpsep,-0.5*\kpsep)--++(-\kpsep,\kpsep)--++(\kpsep,\kpsep);
\draw(\ksepX,-0*\ksepY+\kulV+3*\kpsep)++(-0.5*\kpsep,0)--++(-\kpina,0); 
\draw(\ksepX,-0*\ksepY+\kulV+4*\kpsep)++(-0.5*\kpsep,0)--++(-\kpina,0); 
\draw(\ksepX,-0*\ksepY+\kulV+3*\kpsep)++(-1*\kpsep,-0.5*\kpsep)--++(\kpsep,\kpsep)--++(-\kpsep,\kpsep);
\draw[inactivePart](\ksepX+\kulH+1.5*\kpsep,-0*\ksepY)--++(0,-\kpin);
\draw[inactivePart](\ksepX+\kulH+2.5*\kpsep,-0*\ksepY)--++(0,-\kpin);
\draw[inactivePart](\ksepX+\kulH+1.5*\kpsep,-0*\ksepY)++(0,-\kpin)++(-0.5*\kpsep,0.5*\kpsep)--++(\kpsep,-\kpsep)--++(\kpsep,\kpsep);

\draw(\ksepX,-1*\ksepY)  [thick] rectangle ++(\kdimX,\kdimY);%node[pos=0.5,rectangle,inner sep=0pt,text width=1cm,align=center]{\RTL{جمع و منفی کار}};
\draw(\ksepX,-1*\ksepY+\kulV+1.5*\kpsep)--++(-\kpina,0); 
\draw(\ksepX,-1*\ksepY+\kulV+2.5*\kpsep)--++(-\kpina,0); 
\draw(\ksepX,-1*\ksepY+\kulV+1.5*\kpsep)++(-\kpina,0)++(0.5*\kpsep,-0.5*\kpsep)--++(-\kpsep,\kpsep)--++(\kpsep,\kpsep);
\draw(\ksepX+\kulH+1.5*\kpsep,-1*\ksepY)++(0,-0.5*\kpsep)--++(0,-\kpin);
\draw(\ksepX+\kulH+2.5*\kpsep,-1*\ksepY)++(0,-0.5*\kpsep)--++(0,-\kpin);
\draw(\ksepX+\kulH+1.5*\kpsep,-1*\ksepY)++(-0.5*\kpsep,-\kpsep)--++(\kpsep,\kpsep)--++(\kpsep,-\kpsep);
\end{scope} 

\draw(\ksepX,-2*\ksepY)  [thick] rectangle ++(\kdimX,\kdimY)node[pos=0.5,rectangle,inner sep=0pt,text width=1cm,align=center]{\RTL{$B$}};
\draw(\ksepX+\kdimX,-2*\ksepY+\kulV+4*\kpsep)--++(\kpin,0)node[right]{$\overline{L}_B$};
\draw(\ksepX+\kdimX,-2*\ksepY+\kulV+4*\kpsep)++(\knshift,0)node[ocirc]{};
\draw(\ksepX,-2*\ksepY+\kulV+1.5*\kpsep)++(-0.5*\kpsep,0)--++(-\kpina,0); 
\draw(\ksepX,-2*\ksepY+\kulV+2.5*\kpsep)++(-0.5*\kpsep,0)--++(-\kpina,0); 
\draw(\ksepX,-2*\ksepY+\kulV+1.5*\kpsep)++(-1*\kpsep,-0.5*\kpsep)--++(\kpsep,\kpsep)--++(-\kpsep,\kpsep);

\begin{scope}[inactivePart]
\draw(\ksepX,-3*\ksepY)  [thick] rectangle ++(\kdimX,\kdimY);%node[pos=0.5,rectangle,inner sep=0pt,text width=1cm,align=center]{\RTL{خارجی دفتر}};
\draw(\ksepX,-3*\ksepY+\kulV+1.5*\kpsep)++(-0.5*\kpsep,0)--++(-\kpina,0); 
\draw(\ksepX,-3*\ksepY+\kulV+2.5*\kpsep)++(-0.5*\kpsep,0)--++(-\kpina,0); 
\draw(\ksepX,-3*\ksepY+\kulV+1.5*\kpsep)++(-1*\kpsep,-0.5*\kpsep)--++(\kpsep,\kpsep)--++(-\kpsep,\kpsep);
\draw(\ksepX,-3*\ksepY)++(\kulH+1.5*\kpsep,0)--++(0,-\kpin);
\draw(\ksepX,-3*\ksepY)++(\kulH+2.5*\kpsep,0)--++(0,-\kpin);
\draw(\ksepX,-3*\ksepY)++(\kulH+1.5*\kpsep,0)++(0,-\kpin)++(-0.5*\kpsep,0.5*\kpsep)--++(\kpsep,-\kpsep)--++(\kpsep,\kpsep);

\draw(\ksepX,-4*\ksepY)  [thick] rectangle ++(\kdimX,\kdimY);%node[pos=0.5,rectangle,inner sep=0pt,text width=1cm,align=center]{\RTL{تختی}};
\end{scope}
\end{tikzpicture}
\caption{}
\end{subfigure}\hfill
\begin{subfigure}{0.30\textwidth}
\centering
\begin{tikzpicture}
\pgfmathsetmacro{\klshift}{0.25}
\pgfmathsetmacro{\knshift}{0.07}
\pgfmathsetmacro{\kmv}{0.15}
\pgfmathsetmacro{\knshift}{0.07}
\pgfmathsetmacro{\kpin}{0.30}
\pgfmathsetmacro{\kpina}{0.30}
\pgfmathsetmacro{\kpsep}{0.15}			%pin to pin distance
\pgfmathsetmacro{\kW}{\kpsep}
\pgfmathsetmacro{\kulV}{0.20}			%edge clearance along vertical edge
\pgfmathsetmacro{\kulH}{0.20}
\pgfmathsetmacro{\kdimX}{2*\kulH+4*\kpsep}
\pgfmathsetmacro{\kdimY}{2*\kulV+4*\kpsep}		%two spaces between 3 pins
\pgfmathsetmacro{\ksepX}{\kdimX+2*\kpina+1*\kpsep+\kW}
\pgfmathsetmacro{\ksepY}{\kdimY+\kpin+0.5*\kpsep}

\begin{scope}[inactivePart]
\draw(0,0)  [thick] rectangle ++(\kdimX,\kdimY);%node[pos=0.5,rectangle,inner sep=0pt,text width=1.5cm,align=center]{\RTL{گنتکار}};
\draw(\kdimX,\kulV+1.5*\kpsep)--++(\kpina,0);
\draw(\kdimX,\kulV+2.5*\kpsep)--++(\kpina,0);
\draw(\kdimX,\kulV+1.5*\kpsep)++(\kpina,0)++(-0.5*\kpsep,-0.5*\kpsep)--++(1*\kpsep,1*\kpsep)--++(-1*\kpsep,1*\kpsep);


\draw(0,-\ksepY)  [thick] rectangle ++(\kdimX,\kdimY);%node[pos=0.5,rectangle,inner sep=0pt,text width=1cm,align=center]{\RTL{دفتر پتہ}};
\draw(\kulH+0*\kpsep,-\ksepY)--++(0,-\kpin);
\draw(\kulH+1*\kpsep,-\ksepY)--++(0,-\kpin);
\draw(\kulH+0*\kpsep,-\ksepY)++(0,-\kpin)++(-0.5*\kpsep,0.5*\kpsep)--++(\kpsep,-\kpsep)--++(\kpsep,\kpsep);
\draw(\kulH+3*\kpsep,-\ksepY)--++(0,-\kpin);
\draw(\kulH+4*\kpsep,-\ksepY)--++(0,-\kpin);
\draw(\kulH+3*\kpsep,-\ksepY)++(0,-\kpin)++(-0.5*\kpsep,0.5*\kpsep)--++(\kpsep,-\kpsep)--++(\kpsep,\kpsep);
\draw(\kdimX,-\ksepY+\kulV+1.5*\kpsep)++(0.5*\kpsep,0)--++(\kpina,0);
\draw(\kdimX,-\ksepY+\kulV+2.5*\kpsep)++(0.5*\kpsep,0)--++(\kpina,0);
\draw(\kdimX,-\ksepY+\kulV+2*\kpsep)--++(1*\kpsep,-1*\kpsep);
\draw(\kdimX,-\ksepY+\kulV+2*\kpsep)--++(1*\kpsep,1*\kpsep);

\draw(0,-2*\ksepY)  [thick] rectangle ++(\kdimX,\kdimY);%node[pos=0.5,rectangle,inner sep=0pt,text width=1cm,align=center]{\footnotesize{\RTL{حافظہ}}};
\draw(\kdimX,-2*\ksepY+\kulV+1.5*\kpsep)--++(\kpina,0);
\draw(\kdimX,-2*\ksepY+\kulV+2.5*\kpsep)--++(\kpina,0);
\draw(\kdimX,-2*\ksepY+\kulV+1.5*\kpsep)++(\kpina,0)++(-0.5*\kpsep,-0.5*\kpsep)--++(1*\kpsep,1*\kpsep)--++(-1*\kpsep,1*\kpsep);
\end{scope}

\draw(0,-3*\ksepY)  [thick] rectangle ++(\kdimX,\kdimY)node[pos=0.5,rectangle,inner sep=0pt,text width=1cm,align=center]{ \RTL{دفتر ہدایت}};
\draw[inactivePart](\kdimX,-3*\ksepY+\kulV+0*\kpsep)--++(\kpina,0);
\draw[inactivePart](\kdimX,-3*\ksepY+\kulV+1*\kpsep)--++(\kpina,0);
\draw[inactivePart](\kdimX,-3*\ksepY+\kulV+0*\kpsep)++(\kpina,0)++(-0.5*\kpsep,-0.5*\kpsep)--++(1*\kpsep,1*\kpsep)--++(-1*\kpsep,1*\kpsep);
\draw[inactivePart](\kdimX,-3*\ksepY+\kulV+3*\kpsep)++(0.5*\kpsep,0)--++(\kpina,0);
\draw[inactivePart](\kdimX,-3*\ksepY+\kulV+4*\kpsep)++(0.5*\kpsep,0)--++(\kpina,0);
\draw[inactivePart](\kdimX,-3*\ksepY+\kulV+3.5*\kpsep)--++(1*\kpsep,-1*\kpsep);
\draw[inactivePart](\kdimX,-3*\ksepY+\kulV+3.5*\kpsep)--++(1*\kpsep,1*\kpsep);
\draw(\kulH+1.5*\kpsep,-3*\ksepY)--++(0,-\kpin);
\draw(\kulH+2.5*\kpsep,-3*\ksepY)--++(0,-\kpin);
\draw(\kulH+1.5*\kpsep,-3*\ksepY)++(0,-\kpin)++(-0.5*\kpsep,0.5*\kpsep)--++(\kpsep,-\kpsep)--++(\kpsep,\kpsep);

\draw(0,-4*\ksepY)  [thick] rectangle ++(\kdimX,\kdimY)node[pos=0.5,rectangle,inner sep=0pt,text width=1.5cm,align=center]{\RTL{قابو    }};

\draw(\kulH+1.5*\kpsep,-4*\ksepY)--++(0,-\kpin);
\draw(\kulH+2.5*\kpsep,-4*\ksepY)--++(0,-\kpin);
\draw(\kulH+1.5*\kpsep,-4*\ksepY)++(0,-\kpin)++(-0.5*\kpsep,0.5*\kpsep)--++(\kpsep,-\kpsep)--++(\kpsep,\kpsep);
\draw(\kulH+2*\kpsep,-4*\ksepY)++(0,-1*\kpin-\kpsep)node[below,rectangle,inner sep=0pt, text width=1cm]{\RTL{قابو لفظ}};


\draw(\kdimX+\kpina+0.5*\kpsep,\kdimY)  [thick] rectangle ++(\kW,-4*\ksepY+\kpin);


\draw(\ksepX,-0*\ksepY)  [thick] rectangle ++(\kdimX,\kdimY)node[pos=0.5,rectangle,inner sep=0pt,text width=1cm,align=center]{\RTL{$A$}};
\draw(\ksepX,-0*\ksepY)++(\kdimX,\kulV+4*\kpsep)--++(\kpin,0)node[right]{$\overline{L}_A$};
\draw(\ksepX,-0*\ksepY)++(\kdimX,\kulV+4*\kpsep)++(\knshift,0)node[ocirc]{};
\draw[inactivePart](\ksepX,-0*\ksepY+\kulV+0*\kpsep)--++(-\kpina,0); 
\draw[inactivePart](\ksepX,-0*\ksepY+\kulV+1*\kpsep)--++(-\kpina,0); 
\draw[inactivePart](\ksepX,-0*\ksepY+\kulV+0*\kpsep)++(-\kpina,0)++(0.5*\kpsep,-0.5*\kpsep)--++(-\kpsep,\kpsep)--++(\kpsep,\kpsep);
\draw[inactivePart](\ksepX,-0*\ksepY+\kulV+3*\kpsep)++(-0.5*\kpsep,0)--++(-\kpina,0); 
\draw[inactivePart](\ksepX,-0*\ksepY+\kulV+4*\kpsep)++(-0.5*\kpsep,0)--++(-\kpina,0); 
\draw[inactivePart](\ksepX,-0*\ksepY+\kulV+3*\kpsep)++(-1*\kpsep,-0.5*\kpsep)--++(\kpsep,\kpsep)--++(-\kpsep,\kpsep);
\draw(\ksepX+\kulH+1.5*\kpsep,-0*\ksepY)--++(0,-\kpin);
\draw(\ksepX+\kulH+2.5*\kpsep,-0*\ksepY)--++(0,-\kpin);
\draw(\ksepX+\kulH+1.5*\kpsep,-0*\ksepY)++(0,-\kpin)++(-0.5*\kpsep,0.5*\kpsep)--++(\kpsep,-\kpsep)--++(\kpsep,\kpsep);
\draw(\ksepX,-1*\ksepY)  [thick] rectangle ++(\kdimX,\kdimY)node[pos=0.5,rectangle,inner sep=0pt,text width=1cm,align=center]{\RTL{جمع و منفی کار}};
\draw(\ksepX,-1*\ksepY)++(\kdimX,\kulV+0*\kpsep)--++(\kpin,0)node[right]{$E_U$};
\draw(\ksepX,-1*\ksepY+\kulV+1.5*\kpsep)--++(-\kpina,0); 
\draw(\ksepX,-1*\ksepY+\kulV+2.5*\kpsep)--++(-\kpina,0); 
\draw(\ksepX,-1*\ksepY+\kulV+1.5*\kpsep)++(-\kpina,0)++(0.5*\kpsep,-0.5*\kpsep)--++(-\kpsep,\kpsep)--++(\kpsep,\kpsep);
\draw(\ksepX+\kulH+1.5*\kpsep,-1*\ksepY)++(0,-0.5*\kpsep)--++(0,-\kpin);
\draw(\ksepX+\kulH+2.5*\kpsep,-1*\ksepY)++(0,-0.5*\kpsep)--++(0,-\kpin);
\draw(\ksepX+\kulH+1.5*\kpsep,-1*\ksepY)++(-0.5*\kpsep,-\kpsep)--++(\kpsep,\kpsep)--++(\kpsep,-\kpsep);


\draw(\ksepX,-2*\ksepY)  [thick] rectangle ++(\kdimX,\kdimY)node[pos=0.5,rectangle,inner sep=0pt,text width=1cm,align=center]{\RTL{$B$}};
\draw[inactivePart](\ksepX,-2*\ksepY+\kulV+1.5*\kpsep)++(-0.5*\kpsep,0)--++(-\kpina,0); 
\draw[inactivePart](\ksepX,-2*\ksepY+\kulV+2.5*\kpsep)++(-0.5*\kpsep,0)--++(-\kpina,0); 
\draw[inactivePart](\ksepX,-2*\ksepY+\kulV+1.5*\kpsep)++(-1*\kpsep,-0.5*\kpsep)--++(\kpsep,\kpsep)--++(-\kpsep,\kpsep);

\begin{scope}[inactivePart]
\draw(\ksepX,-3*\ksepY)  [thick] rectangle ++(\kdimX,\kdimY);%node[pos=0.5,rectangle,inner sep=0pt,text width=1cm,align=center]{\RTL{خارجی دفتر}};
\draw(\ksepX,-3*\ksepY+\kulV+1.5*\kpsep)++(-0.5*\kpsep,0)--++(-\kpina,0); 
\draw(\ksepX,-3*\ksepY+\kulV+2.5*\kpsep)++(-0.5*\kpsep,0)--++(-\kpina,0); 
\draw(\ksepX,-3*\ksepY+\kulV+1.5*\kpsep)++(-1*\kpsep,-0.5*\kpsep)--++(\kpsep,\kpsep)--++(-\kpsep,\kpsep);
\draw(\ksepX,-3*\ksepY)++(\kulH+1.5*\kpsep,0)--++(0,-\kpin);
\draw(\ksepX,-3*\ksepY)++(\kulH+2.5*\kpsep,0)--++(0,-\kpin);
\draw(\ksepX,-3*\ksepY)++(\kulH+1.5*\kpsep,0)++(0,-\kpin)++(-0.5*\kpsep,0.5*\kpsep)--++(\kpsep,-\kpsep)--++(\kpsep,\kpsep);

\draw(\ksepX,-4*\ksepY)  [thick] rectangle ++(\kdimX,\kdimY);%node[pos=0.5,rectangle,inner sep=0pt,text width=1cm,align=center]{\RTL{تختی}};
\end{scope}
\end{tikzpicture}
\caption{}
\end{subfigure}
\caption{
طریق جمع و منفی؛ (ا) \عددی{T_4} حال؛ (ب) \عددی{T_5} حال؛ (ج) \عددی{T_6} حال۔
}
\label{شکل_کمپیوٹر_جمع_اور_منفی}
\end{figure}

اتفاق سے،   دورانیہ تیاری اور  دورانیہ ردعمل  کی بدولت دفتر \عددی{A} حالت دوڑ سے دو چار نہیں ہوتا۔ شکل \حوالہء{10.6c} میں  ساعت کے  کنارہ چڑھائی  پر دفتر \عددی{A} کا  مواد تبدیل ہو گا، جس کی وجہ سے جمع و منفی کار کا مخارج تبدیل ہو گا۔ یہ نیا مواد دفتر \عددی{A}  کے مداخل تک پہنچتا ہے، تاہم یہ مواد  ساعت کے کنارہ چڑھائی کے دو  تاخیر  بعد  یہاں پہنچے گا ( پہلی تاخیر   دفتر \عددی{A} اور دوسری  تاخیر جمع و منفی کار کی بدولت ہو گی)۔ اس وقت تک دفتر \عددی{A}  میں مواد لکھنے کا لمحہ گزر چکا ہوگا۔ یوں دفتر \عددی{A} حالت دوڑ (جس میں ساعت کے  ایک  ہی کنارے پر  ایک سے زیادہ مرتبہ مواد بھرا جاتا ہو)  سے دو چار نہیں ہو گا۔

شکل  \حوالہ{شکل_کمپیوٹر_بازیابی_جمع_وقتیہ}  میں بازیابی اور  \قول{طریق جمع  } کی وقتیہ ترسیمات  پیش ہیں۔طریق  بازیابی  ہمیشہ کی طرح \عددی{T_1} حال میں  دفتر پتہ میں برنامہ گنت کار کا مواد منتقل کرتا ہے؛ \عددی{T_2} حال میں  گنت کار کی گنتی میں ایک کا اضافہ کیا جاتا ہے؛ \عددی{T_3} حال میں دفتر ہدایت  کو ،  پتہ کی نشاندہی پر ، حافظہ سے ہدایت منتقل کی جاتی ہے۔

\begin{figure}
\centering
\begin{otherlanguage}{english}
 \begin{tikztimingtable}[%
timing/.style={x=4ex,y=3ex},
timing/rowdist=6ex,
every node/.style={inner sep=0,outer sep=0},
%timing/c/arrow tip=latex, %and this set the style
%timing/c/rising arrows,
timing/slope=0, %0.1 is good
timing/dslope=0,
thick,
]
%\tikztimingmetachar{R}{[|/utils/exec=\setcounter{new}{0}|]}
%\usetikztiminglibrary[new={char=Q,reset char=R}]{counters}
%[timing/counter/new={char=c, base=2,digits=3,max value=7, wraps ,text style={font=\normalsize}}] 12{2c} \\ 
%$C$& H22{C}\\
%$\texturdu{\RL{پتہ}}$&3D{} 20D{[scale=1.5]\texturdu{\RL{درست پتہ}}} 3D{}\\
$CLK$&HN(a)LN(ca)HN(b)LN(cb)HN(c)LN(cc)HN(d)LN(cd)HN(e)LN(ce)HN(f)LN(cf)HN(g)L\\
$E_P$&LHHLLLLLLLLLLH\\
$\overline{L}_M$&HLL4{H}2{L}5{H}\\
$C_P$&LLLHHLLLLLLLLL\\
$\overline{CE}$&HHHHHLLHHLLHHH\\
$\overline{L}_I$&HHHHHLLHHHHHHH\\
$\overline{E}_I$&HHHHHHHLLHHHHH\\
$\overline{L}_B$&HHHHHHHHHLLHHH\\
$E_U$&LLLLLLLLLLLHHL\\
$\overline{L}_A$&HHHHHHHHHHHLLH\\
\extracode
\begin{pgfonlayer}{background}
\begin{scope}[]
\draw [latex-latex] ($(a|-row1.north)+(0,2ex)$) --node[fill=white]{$T_1$} ($(b|-row1.north)+(0,2ex)$);
\draw [latex-latex] ($(b|-row1.north)+(0,2ex)$) --node[fill=white]{$T_2$} ($(c|-row1.north)+(0,2ex)$);
\draw [latex-latex] ($(c|-row1.north)+(0,2ex)$) --node[fill=white]{$T_3$} ($(d|-row1.north)+(0,2ex)$);
\draw [latex-latex] ($(d|-row1.north)+(0,2ex)$) --node[fill=white]{$T_4$} ($(e|-row1.north)+(0,2ex)$);
\draw [latex-latex] ($(e|-row1.north)+(0,2ex)$) --node[fill=white]{$T_5$} ($(f|-row1.north)+(0,2ex)$);
\draw [latex-latex] ($(f|-row1.north)+(0,2ex)$) --node[fill=white]{$T_6$} ($(g|-row1.north)+(0,2ex)$);
\foreach \n in {a,b,c,d,e,f,g}{\draw[thin]($(\n|-row1.north)+(0,1ex)$)--++(0,2ex);}
\draw[dashed] (ca)--(ca |-row3.south);
\draw[dashed] (cb)--(cb |-row4.north);
\draw[dashed] (cc)--(cc |-row6.south);
\draw[dashed] (cd)--(cd |-row7.south);
\draw[dashed] (ce)--(ce |-row8.south);
\draw[dashed] (cf)--(cf |-row10.south);
%%\vertlines[darkgray,dotted]{3.6,7.5,11.5,15.5,19.5,23.5,27.5}
%\foreach \n in {1,3,...,11} \draw(4*\n ex-2ex,-4ex+1.25ex)node[]{$0$};
%\foreach \n in {2,4,...,12} \draw(4*\n ex-2ex,-4ex+1.25ex)node[]{$1$};
%\foreach \n in {1,2,5,6,9,10} \draw(4*\n ex-2ex,-12 ex+1.25ex)node[]{$0$};
%\foreach \n in {3,4,7,8,11,12} \draw(4*\n ex-2ex,-12 ex+1.25ex)node[]{$1$};
%\foreach \n in {1,2,3,4,9,10,11,12} \draw(4*\n ex-2ex,-20 ex+1.25ex)node[]{$0$};
%\foreach \n in {5,6,7,8} \draw(4*\n ex-2ex,-20 ex+1.25ex)node[]{$1$};
%\draw(4*4 ex-2ex,-12 ex+1.25ex) circle (0.25cm and 1.75cm);
%\draw(4*4 ex-2ex,-7*\rowdist+1.25ex) circle (0.25cm and 0.5cm);
%%\foreach \n in {1}\draw(B\n.south)--(F\n.north);
%%\foreach \n in {1}\draw(C\n.south)--(G\n.north);
\end{scope}
\end{pgfonlayer}
\end{tikztimingtable}
\end{otherlanguage}
\caption{بازیابی اور جمع  وقتیہ ترسیمات۔}
\label{شکل_کمپیوٹر_بازیابی_جمع_وقتیہ}
\end{figure}

\عددی{T_4} حال کے دوران، \عددی{\overline{E}_I} اور \عددی{\overline{L}_M} فعال  ہوں گے؛ ساعت کے اگلے کنارہ چڑھائی پر، دفتر پتہ کو   دفتر  ہدایت سے جزو  پتہ منتقل ہو گا۔ \عددی{T_5} حال کے دوران، \عددی{\overline{CE}} اور \عددی{\overline{L}_B} فعال ہوں گے؛  لہٰذا ساعت کے کنارہ چڑھائی پر   دفتر \عددی{B} میں  پتہ کی نشاندہی پر حافظہ سے لفظ منتقل ہو گا۔ \عددی{T_6} حال کے دوران، \عددی{E_U} اور \عددی{\overline{L}_A} فعال ہوں گے؛ دفتر \عددی{A} میں، ساعت کے کنارہ چڑھائی پر، جمع  و منفی کار کا  حاصل نتیجہ منتقل ہو گا۔

\جزوحصہء{طریق منفی}
طریق منفی اور طریق جمع ملتے جلتے ہیں۔ شکل \حوالہ{شکل_کمپیوٹر_جمع_اور_منفی}-الف اور ب میں طریق منفی کے لئے  \عددی{T_4} اور \عددی{T_5}  حال کے دوران  فعال حصے دکھائے گئے ہیں۔ \عددی{T_6} حال کے دوران شکل \حوالہ{شکل_کمپیوٹر_جمع_اور_منفی}-ج کے جمو ع منفی حصے کو بلند \عددی{S_U} بھیجا جاتا ہے۔ وقتیہ ترسیم شکل \حوالہ{شکل_کمپیوٹر_بازیابی_جمع_وقتیہ}    سے تقریباً  مکمل مماثلت رکھتی ہے۔ \عددی{T_1}  تا \عددی{T_5} حال کے دوران پست \عددی{S_U} اور \عددی{T_6} حال کے دوران بلند \عددی{S_U} تصور کریں۔

\جزوحصہء{طریق برآمد}
فرض کریں  بازیابی پھیرا کے آخر میں دفتر ہدایت میں برآمد کی ہدایت موجود ہو۔ یوں درج ذیل ہو گا۔
\begin{align*}
1110\,xxxx=\text{\RL{دفتر ہدایت}}
\end{align*}
قابو و ترتیب کار کو رمز کشائی کے لئے  جزو ہدایت  بھیجا جاتا ہے۔ رمز کشائی کے بعد   قابو و ترتیب کار   خارجی دفتر میں دفتر \عددی{A} کا مواد منتقل کرنے کے لئے قابو لفظ جاری کرتا ہے۔

برآمد کی ہدایت کے دوران فعال حصے شکل  \حوالہ{شکل_کمپیوٹر_برآمد_ہدایت}  میں پیش ہیں۔ چونکہ \عددی{E_A} اور \عددی{\overline{L}_O} فعال ہیں، لہٰذا ساعت کے اگلے کنارہ چڑھائی پر دفتر \عددی{A} کی معلومات  خارجی دفتر میں  ، \عددی{T_4} حال کے دوران ، منتقل ہو گی۔ \عددی{T_5} اور \عددی{T_6} حال فارغ  ہیں۔

\begin{figure}
\centering
\begin{tikzpicture}
\pgfmathsetmacro{\klshift}{0.25}
\pgfmathsetmacro{\knshift}{0.07}
\pgfmathsetmacro{\kmv}{0.15}
\pgfmathsetmacro{\knshift}{0.07}
\pgfmathsetmacro{\kpin}{0.30}
\pgfmathsetmacro{\kpina}{0.30}
\pgfmathsetmacro{\kpsep}{0.15}			%pin to pin distance
\pgfmathsetmacro{\kW}{\kpsep}
\pgfmathsetmacro{\kulV}{0.20}			%edge clearance along vertical edge
\pgfmathsetmacro{\kulH}{0.20}
\pgfmathsetmacro{\kdimX}{2*\kulH+4*\kpsep}
\pgfmathsetmacro{\kdimY}{2*\kulV+4*\kpsep}		%two spaces between 3 pins
\pgfmathsetmacro{\ksepX}{\kdimX+2*\kpina+1*\kpsep+\kW}
\pgfmathsetmacro{\ksepY}{\kdimY+\kpin+0.5*\kpsep}

\begin{scope}[inactivePart]
\draw(0,0)  [thick] rectangle ++(\kdimX,\kdimY);%node[pos=0.5,rectangle,inner sep=0pt,text width=1.5cm,align=center]{\RTL{گنتکار}};
\draw(\kdimX,\kulV+1.5*\kpsep)--++(\kpina,0);
\draw(\kdimX,\kulV+2.5*\kpsep)--++(\kpina,0);
\draw(\kdimX,\kulV+1.5*\kpsep)++(\kpina,0)++(-0.5*\kpsep,-0.5*\kpsep)--++(1*\kpsep,1*\kpsep)--++(-1*\kpsep,1*\kpsep);


\draw(0,-\ksepY)  [thick] rectangle ++(\kdimX,\kdimY);%node[pos=0.5,rectangle,inner sep=0pt,text width=1cm,align=center]{\RTL{دفتر پتہ}};
\draw(\kulH+0*\kpsep,-\ksepY)--++(0,-\kpin);
\draw(\kulH+1*\kpsep,-\ksepY)--++(0,-\kpin);
\draw(\kulH+0*\kpsep,-\ksepY)++(0,-\kpin)++(-0.5*\kpsep,0.5*\kpsep)--++(\kpsep,-\kpsep)--++(\kpsep,\kpsep);
\draw(\kulH+3*\kpsep,-\ksepY)--++(0,-\kpin);
\draw(\kulH+4*\kpsep,-\ksepY)--++(0,-\kpin);
\draw(\kulH+3*\kpsep,-\ksepY)++(0,-\kpin)++(-0.5*\kpsep,0.5*\kpsep)--++(\kpsep,-\kpsep)--++(\kpsep,\kpsep);
\draw(\kdimX,-\ksepY+\kulV+1.5*\kpsep)++(0.5*\kpsep,0)--++(\kpina,0);
\draw(\kdimX,-\ksepY+\kulV+2.5*\kpsep)++(0.5*\kpsep,0)--++(\kpina,0);
\draw(\kdimX,-\ksepY+\kulV+2*\kpsep)--++(1*\kpsep,-1*\kpsep);
\draw(\kdimX,-\ksepY+\kulV+2*\kpsep)--++(1*\kpsep,1*\kpsep);

\draw(0,-2*\ksepY)  [thick] rectangle ++(\kdimX,\kdimY);%node[pos=0.5,rectangle,inner sep=0pt,text width=1cm,align=center]{\footnotesize{\RTL{حافظہ}}};
\draw(\kdimX,-2*\ksepY+\kulV+1.5*\kpsep)--++(\kpina,0);
\draw(\kdimX,-2*\ksepY+\kulV+2.5*\kpsep)--++(\kpina,0);
\draw(\kdimX,-2*\ksepY+\kulV+1.5*\kpsep)++(\kpina,0)++(-0.5*\kpsep,-0.5*\kpsep)--++(1*\kpsep,1*\kpsep)--++(-1*\kpsep,1*\kpsep);
\end{scope}

\draw(0,-3*\ksepY)  [thick] rectangle ++(\kdimX,\kdimY)node[pos=0.5,rectangle,inner sep=0pt,text width=1cm,align=center]{ \RTL{دفتر ہدایت}};
\draw[inactivePart](\kdimX,-3*\ksepY+\kulV+0*\kpsep)--++(\kpina,0);
\draw[inactivePart](\kdimX,-3*\ksepY+\kulV+1*\kpsep)--++(\kpina,0);
\draw[inactivePart](\kdimX,-3*\ksepY+\kulV+0*\kpsep)++(\kpina,0)++(-0.5*\kpsep,-0.5*\kpsep)--++(1*\kpsep,1*\kpsep)--++(-1*\kpsep,1*\kpsep);
\draw[inactivePart](\kdimX,-3*\ksepY+\kulV+3*\kpsep)++(0.5*\kpsep,0)--++(\kpina,0);
\draw[inactivePart](\kdimX,-3*\ksepY+\kulV+4*\kpsep)++(0.5*\kpsep,0)--++(\kpina,0);
\draw[inactivePart](\kdimX,-3*\ksepY+\kulV+3.5*\kpsep)--++(1*\kpsep,-1*\kpsep);
\draw[inactivePart](\kdimX,-3*\ksepY+\kulV+3.5*\kpsep)--++(1*\kpsep,1*\kpsep);
\draw(\kulH+1.5*\kpsep,-3*\ksepY)--++(0,-\kpin);
\draw(\kulH+2.5*\kpsep,-3*\ksepY)--++(0,-\kpin);
\draw(\kulH+1.5*\kpsep,-3*\ksepY)++(0,-\kpin)++(-0.5*\kpsep,0.5*\kpsep)--++(\kpsep,-\kpsep)--++(\kpsep,\kpsep);

\draw(0,-4*\ksepY)  [thick] rectangle ++(\kdimX,\kdimY)node[pos=0.5,rectangle,inner sep=0pt,text width=1.5cm,align=center]{\RTL{قابو    }};

\draw(\kulH+1.5*\kpsep,-4*\ksepY)--++(0,-\kpin);
\draw(\kulH+2.5*\kpsep,-4*\ksepY)--++(0,-\kpin);
\draw(\kulH+1.5*\kpsep,-4*\ksepY)++(0,-\kpin)++(-0.5*\kpsep,0.5*\kpsep)--++(\kpsep,-\kpsep)--++(\kpsep,\kpsep);
\draw(\kulH+2*\kpsep,-4*\ksepY)++(0,-1*\kpin-\kpsep)node[below,rectangle,inner sep=0pt, text width=1cm]{\RTL{قابو لفظ}};


\draw(\kdimX+\kpina+0.5*\kpsep,\kdimY)  [thick] rectangle ++(\kW,-4*\ksepY+\kpin);


\draw(\ksepX,-0*\ksepY)  [thick] rectangle ++(\kdimX,\kdimY)node[pos=0.5,rectangle,inner sep=0pt,text width=1cm,align=center]{\RTL{$A$}};
\draw(\ksepX,-0*\ksepY)++(\kdimX,\kulV+0*\kpsep)--++(\kpin,0)node[right]{$E_A$};
\draw(\ksepX,-0*\ksepY+\kulV+0*\kpsep)--++(-\kpina,0); 
\draw(\ksepX,-0*\ksepY+\kulV+1*\kpsep)--++(-\kpina,0); 
\draw(\ksepX,-0*\ksepY+\kulV+0*\kpsep)++(-\kpina,0)++(0.5*\kpsep,-0.5*\kpsep)--++(-\kpsep,\kpsep)--++(\kpsep,\kpsep);
\draw[inactivePart](\ksepX,-0*\ksepY+\kulV+3*\kpsep)++(-0.5*\kpsep,0)--++(-\kpina,0); 
\draw[inactivePart](\ksepX,-0*\ksepY+\kulV+4*\kpsep)++(-0.5*\kpsep,0)--++(-\kpina,0); 
\draw[inactivePart](\ksepX,-0*\ksepY+\kulV+3*\kpsep)++(-1*\kpsep,-0.5*\kpsep)--++(\kpsep,\kpsep)--++(-\kpsep,\kpsep);
\draw[inactivePart](\ksepX+\kulH+1.5*\kpsep,-0*\ksepY)--++(0,-\kpin);
\draw[inactivePart](\ksepX+\kulH+2.5*\kpsep,-0*\ksepY)--++(0,-\kpin);
\draw[inactivePart](\ksepX+\kulH+1.5*\kpsep,-0*\ksepY)++(0,-\kpin)++(-0.5*\kpsep,0.5*\kpsep)--++(\kpsep,-\kpsep)--++(\kpsep,\kpsep);

\begin{scope}[inactivePart]
\draw(\ksepX,-1*\ksepY)  [thick] rectangle ++(\kdimX,\kdimY);%node[pos=0.5,rectangle,inner sep=0pt,text width=1cm,align=center]{\RTL{جمع و منفی کار}};
\draw(\ksepX,-1*\ksepY)++(\kdimX,\kulV+0*\kpsep)--++(\kpin,0)node[right]{$E_U$};
\draw(\ksepX,-1*\ksepY+\kulV+1.5*\kpsep)--++(-\kpina,0); 
\draw(\ksepX,-1*\ksepY+\kulV+2.5*\kpsep)--++(-\kpina,0); 
\draw(\ksepX,-1*\ksepY+\kulV+1.5*\kpsep)++(-\kpina,0)++(0.5*\kpsep,-0.5*\kpsep)--++(-\kpsep,\kpsep)--++(\kpsep,\kpsep);
\draw(\ksepX+\kulH+1.5*\kpsep,-1*\ksepY)++(0,-0.5*\kpsep)--++(0,-\kpin);
\draw(\ksepX+\kulH+2.5*\kpsep,-1*\ksepY)++(0,-0.5*\kpsep)--++(0,-\kpin);
\draw(\ksepX+\kulH+1.5*\kpsep,-1*\ksepY)++(-0.5*\kpsep,-\kpsep)--++(\kpsep,\kpsep)--++(\kpsep,-\kpsep);


\draw(\ksepX,-2*\ksepY)  [thick] rectangle ++(\kdimX,\kdimY);%node[pos=0.5,rectangle,inner sep=0pt,text width=1cm,align=center]{\RTL{$B$}};
\draw[inactivePart](\ksepX,-2*\ksepY+\kulV+1.5*\kpsep)++(-0.5*\kpsep,0)--++(-\kpina,0); 
\draw[inactivePart](\ksepX,-2*\ksepY+\kulV+2.5*\kpsep)++(-0.5*\kpsep,0)--++(-\kpina,0); 
\draw[inactivePart](\ksepX,-2*\ksepY+\kulV+1.5*\kpsep)++(-1*\kpsep,-0.5*\kpsep)--++(\kpsep,\kpsep)--++(-\kpsep,\kpsep);
\end{scope}

\draw(\ksepX,-3*\ksepY)  [thick] rectangle ++(\kdimX,\kdimY)node[pos=0.5,rectangle,inner sep=0pt,text width=1cm,align=center]{\RTL{خارجی دفتر}};
\draw(\ksepX,-3*\ksepY)++(\kdimX,\kulV+4*\kpsep)--++(\kpin,0)node[right]{$\overline{L}_O$};
\draw(\ksepX,-3*\ksepY)++(\kdimX,\kulV+4*\kpsep)++(\knshift,0)node[ocirc]{};
\draw(\ksepX,-3*\ksepY+\kulV+1.5*\kpsep)++(-0.5*\kpsep,0)--++(-\kpina,0); 
\draw(\ksepX,-3*\ksepY+\kulV+2.5*\kpsep)++(-0.5*\kpsep,0)--++(-\kpina,0); 
\draw(\ksepX,-3*\ksepY+\kulV+1.5*\kpsep)++(-1*\kpsep,-0.5*\kpsep)--++(\kpsep,\kpsep)--++(-\kpsep,\kpsep);
\draw(\ksepX,-3*\ksepY)++(\kulH+1.5*\kpsep,0)--++(0,-\kpin);
\draw(\ksepX,-3*\ksepY)++(\kulH+2.5*\kpsep,0)--++(0,-\kpin);
\draw(\ksepX,-3*\ksepY)++(\kulH+1.5*\kpsep,0)++(0,-\kpin)++(-0.5*\kpsep,0.5*\kpsep)--++(\kpsep,-\kpsep)--++(\kpsep,\kpsep);

\draw(\ksepX,-4*\ksepY)  [thick] rectangle ++(\kdimX,\kdimY)node[pos=0.5,rectangle,inner sep=0pt,text width=1cm,align=center]{\RTL{تختی}};
\end{tikzpicture}
\caption{برآمد ہدایت کے دوران \عددی{T_4} حال۔}
\label{شکل_کمپیوٹر_برآمد_ہدایت}
\end{figure}

شکل \حوالہ{شکل_کمپیوٹر_بازیابی_برآمد_وقتیہ} میں  بازیابی اور برآمد وقتیہ ترسیمات پیش ہیں۔  بازیابی حال   ہمیشہ کی طرح پتہ حال، بڑھوتری حال، اور حافظہ حال پر مشتمل ہو گا۔ \عددی{T_4} حال کے دوران، \عددی{E_A} اور \عددی{\overline{L}_O} فعال ہوں گے؛ لہٰذا ساعت کے اگلے کنارہ چڑھائی پر دفتر \عددی{A} کی معلومات خارجی دفتر کو منتقل ہو گی۔

\begin{figure}
\centering
\begin{otherlanguage}{english}
 \begin{tikztimingtable}[%
timing/.style={x=4ex,y=3ex},
timing/rowdist=6ex,
every node/.style={inner sep=0,outer sep=0},
%timing/c/arrow tip=latex, %and this set the style
%timing/c/rising arrows,
timing/slope=0, %0.1 is good
timing/dslope=0,
thick,
]
%\tikztimingmetachar{R}{[|/utils/exec=\setcounter{new}{0}|]}
%\usetikztiminglibrary[new={char=Q,reset char=R}]{counters}
%[timing/counter/new={char=c, base=2,digits=3,max value=7, wraps ,text style={font=\normalsize}}] 12{2c} \\ 
%$C$& H22{C}\\
%$\texturdu{\RL{پتہ}}$&3D{} 20D{[scale=1.5]\texturdu{\RL{درست پتہ}}} 3D{}\\
$CLK$&HN(a)LN(ca)HN(b)LN(cb)HN(c)LN(cc)HN(d)LN(cd)HN(e)LN(ce)HN(f)LN(cf)HN(g)L\\
$E_P$&LHHLLLLLLLLLLH\\
$\overline{L}_M$&HLL4{H}2{H}5{H}\\
$C_P$&LLLHHLLLLLLLLL\\
$\overline{CE}$&HHHHHLLHHHHHHH\\
$\overline{L}_I$&HHHHHLLHHHHHHH\\
$E_A$&LLLLLLLHHLLLLL\\
$\overline{L}_O$&HHHHHHHLLHHHHH\\
\extracode
\begin{pgfonlayer}{background}
\begin{scope}[]
\draw [latex-latex] ($(a|-row1.north)+(0,2ex)$) --node[fill=white]{$T_1$} ($(b|-row1.north)+(0,2ex)$);
\draw [latex-latex] ($(b|-row1.north)+(0,2ex)$) --node[fill=white]{$T_2$} ($(c|-row1.north)+(0,2ex)$);
\draw [latex-latex] ($(c|-row1.north)+(0,2ex)$) --node[fill=white]{$T_3$} ($(d|-row1.north)+(0,2ex)$);
\draw [latex-latex] ($(d|-row1.north)+(0,2ex)$) --node[fill=white]{$T_4$} ($(e|-row1.north)+(0,2ex)$);
\draw [latex-latex] ($(e|-row1.north)+(0,2ex)$) --node[fill=white]{$T_5$} ($(f|-row1.north)+(0,2ex)$);
\draw [latex-latex] ($(f|-row1.north)+(0,2ex)$) --node[fill=white]{$T_6$} ($(g|-row1.north)+(0,2ex)$);
\foreach \n in {a,b,c,d,e,f,g}{\draw[thin]($(\n|-row1.north)+(0,1ex)$)--++(0,2ex);}
\draw[dashed] (ca)--(ca |-row3.south);
\draw[dashed] (cb)--(cb |-row4.north);
\draw[dashed] (cc)--(cc |-row6.south);
\draw[dashed] (cd)--(cd |-row8.south);
%%\vertlines[darkgray,dotted]{3.6,7.5,11.5,15.5,19.5,23.5,27.5}
%\foreach \n in {1,3,...,11} \draw(4*\n ex-2ex,-4ex+1.25ex)node[]{$0$};
%\foreach \n in {2,4,...,12} \draw(4*\n ex-2ex,-4ex+1.25ex)node[]{$1$};
%\foreach \n in {1,2,5,6,9,10} \draw(4*\n ex-2ex,-12 ex+1.25ex)node[]{$0$};
%\foreach \n in {3,4,7,8,11,12} \draw(4*\n ex-2ex,-12 ex+1.25ex)node[]{$1$};
%\foreach \n in {1,2,3,4,9,10,11,12} \draw(4*\n ex-2ex,-20 ex+1.25ex)node[]{$0$};
%\foreach \n in {5,6,7,8} \draw(4*\n ex-2ex,-20 ex+1.25ex)node[]{$1$};
%\draw(4*4 ex-2ex,-12 ex+1.25ex) circle (0.25cm and 1.75cm);
%\draw(4*4 ex-2ex,-7*\rowdist+1.25ex) circle (0.25cm and 0.5cm);
%%\foreach \n in {1}\draw(B\n.south)--(F\n.north);
%%\foreach \n in {1}\draw(C\n.south)--(G\n.north);
\end{scope}
\end{pgfonlayer}
\end{tikztimingtable}
\end{otherlanguage}
\caption{بازیابی اور برآمد   وقتیہ  ترسیمات۔}
\label{شکل_کمپیوٹر_بازیابی_برآمد_وقتیہ}
\end{figure}

\جزوحصہء{رک}
رک کی ہدایت  پر عمل در آمد  کے دوران کسی دفتر کی ضرورت پیش نہیں آتی، لہٰذا  اس کے لئے طریق  قابو  درکار نہیں ہو گا۔ جب دفتر ہدایت میں درج ذیل موجود ہو
\begin{align*}
1111\, xxxx=\text{\RL{دفتر ہدایت}}
\end{align*}
جزو ہدایت \عددی{1111} قابو و ترتیب کار کو  مواد پر عمل  نہ کرنے کا اشارہ کرتا ہے۔ قابو و ترتیب کار ساعت (جس کے دور پر کچھ دیر میں غور کیا جائے گا)   روک کر کمپیوٹر کو  مزید کام کرنے سے روک لیتا ہے۔

\جزوحصہء{مشینی پھیرا اور ہدایتی پھیرا}
اس سادہ  کمپیوٹر  کے چھ \عددی{T} حال ہیں، جن میں سے تین بازیابی اور تین تعمیلی ہیں ۔ ان چھ حال کو\اصطلاح{ مشینی پھیرا }\فرہنگ{پھیرا!مشینی}\حاشیہب{machine cycle}\فرہنگ{cycle!machine} کہتے ہیں (شکل  \حوالہ{شکل_کمپیوٹر_مشینی_پھیرے}-الف  دیکھیں)۔ ایک مشینی پھیرے میں ایک ہدایت  کی بازیابی اور تعمیل کی جاتی ہے۔ اس سادہ ترین کمپیوٹر کی ساعت  کا تعدد \عددی{\SI{1}{\kilo\hertz}}  ہے، لہٰذا اس کا دوری عرصہ \عددی{\SI{1}{\milli\second}} ہو گا۔ یوں ہر  مشینی پھیرا \عددی{\SI{6}{\milli\second}} لیگا۔

کئی کمپیوٹر میں  ہدایت کی بازیابی اور تعمیل کرنا  ایک سے زائد  مشینی پھیروں میں ممکن ہو گا۔ شکل  \حوالہ{شکل_کمپیوٹر_مشینی_پھیرے}-ب  میں دو  مشینی پھیروں  کی ہدایت کا وقتیہ ترسیم پیش ہے۔ اولین تین \عددی{T} حال  بازیابی پھیرا دیتے ہیں؛ تاہم تعمیلی پھیرے کو اگلے نو \عددی{T} حال درکار ہیں۔  دو مشینی پھیرے کی ہدایت  زیادہ پیچیدہ ہو گی جس کی تعمیل  کے لئے اضافی \عددی{T} حال درکار ہوں گے۔

ایک ہدایت کی بازیابی اور تعمیل کے لئے درکار \عددی{T} حال کو \اصطلاح{ ہدایتی پھیرا }\فرہنگ{پھیرا!ہدایتی}\حاشیہب{instruction cycle}\فرہنگ{cycle!instruction}کہتے ہیں۔ اس سادہ ترین کمپیوٹر میں ہدایتی پھیرا اور مشینی پھیرا ایک  برابر ہیں، جبکہ  شکل \حوالہ{شکل_کمپیوٹر_مشینی_پھیرے}-ب میں ہدایتی پھیرا دو مشینی پھیروں کے برابر ہے۔

\عددی{8080} اور \عددی{8085}  کے ہدایتی پھیرے ایک سے  پانچ مشینی پھیروں کے برابر ہو سکتے ہیں۔ 

\begin{figure}
\centering
\begin{subfigure}{1\textwidth}
\centering
\begin{otherlanguage}{english}
 \begin{tikztimingtable}[%
timing/.style={x=2.75ex,y=3ex},
timing/rowdist=6ex,
every node/.style={inner sep=0,outer sep=0},
%timing/c/arrow tip=latex, %and this set the style
%timing/c/rising arrows,
timing/slope=0, %0.1 is good
timing/dslope=0,
thick,
]
%\tikztimingmetachar{R}{[|/utils/exec=\setcounter{new}{0}|]}
%\usetikztiminglibrary[new={char=Q,reset char=R}]{counters}
%[timing/counter/new={char=c, base=2,digits=3,max value=7, wraps ,text style={font=\normalsize}}] 12{2c} \\ 
%$C$& H22{C}\\
%$\texturdu{\RL{پتہ}}$&3D{} 20D{[scale=1.5]\texturdu{\RL{درست پتہ}}} 3D{}\\
$CLK$&HN(a)LN(ca)HN(b)LN(cb)HN(c)LN(cc)HN(d)LN(cd)HN(e)LN(ce)HN(f)LN(cf)HN(g)L\\
\extracode
\begin{pgfonlayer}{background}
\begin{scope}[]
\foreach \n/\a in {ca/1,cb/2,cc/3,cd/4,ce/5,cf/6}\draw ($(\n |-row1.south)+(0,-3ex)$) node[]{$T_\a$};
\foreach \n in {b,c,e,f}{\draw[thin]($(\n|-row1.south)+(0,-1ex)$)--($(\n|-row2.mid)$);}
\draw($(a|-row1.south)+(0,-1ex)$)--++(0,-14ex);
\draw($(g|-row1.south)+(0,-1ex)$)--++(0,-14ex);
\draw($(d|-row1.south)+(0,-1ex)$)--($(d|-row3.north)+(0,1ex)$);
\draw[stealth-stealth]($(a|-row3.north)+(0,2ex)$)--node[fill=white]{\texturdu{\RL{بازیابی پھیرا}}}($(d|-row3.north)+(0,2ex)$);
\draw[stealth-stealth]($(d|-row3.north)+(0,2ex)$)--node[fill=white]{\texturdu{\RL{تعمیلی پھیرا}}}($(g|-row3.north)+(0,2ex)$);
\draw[stealth-stealth](a|-row3.mid)--node[fill=white]{\texturdu{\RL{مشینی پھیرا}}}(g|-row3.mid);
\draw[stealth-stealth]($(a|-row3.south)+(0,-2ex)$)--node[fill=white]{\texturdu{\RL{ہدایتی پھیرا}}}($(g|-row3.south)+(0,-2ex)$);
%\draw[dashed] (cd)--(cd |-row8.south);
%%\vertlines[darkgray,dotted]{3.6,7.5,11.5,15.5,19.5,23.5,27.5}
%\foreach \n in {1,3,...,11} \draw(4*\n ex-2ex,-4ex+1.25ex)node[]{$0$};
\end{scope}
\end{pgfonlayer}
\end{tikztimingtable}
\end{otherlanguage}
\caption{}
\end{subfigure}
\begin{subfigure}{1\textwidth}
\centering
\begin{otherlanguage}{english}
 \begin{tikztimingtable}[%
timing/.style={x=2.75ex,y=3ex},
timing/rowdist=6ex,
every node/.style={inner sep=0,outer sep=0},
%timing/c/arrow tip=latex, %and this set the style
%timing/c/rising arrows,
timing/slope=0, %0.1 is good
timing/dslope=0,
thick,
]
%\tikztimingmetachar{R}{[|/utils/exec=\setcounter{new}{0}|]}
%\usetikztiminglibrary[new={char=Q,reset char=R}]{counters}
%[timing/counter/new={char=c, base=2,digits=3,max value=7, wraps ,text style={font=\normalsize}}] 12{2c} \\ 
%$C$& H22{C}\\
%$\texturdu{\RL{پتہ}}$&3D{} 20D{[scale=1.5]\texturdu{\RL{درست پتہ}}} 3D{}\\
$CLK$&HN(a)LN(ca)HN(b)LN(cb)HN(c)LN(cc)HN(d)LN(cd)HN(e)LN(ce)HN(f)L
N(cf)HN(g)LN(cg)HN(h)LN(ch)HN(i)LN(ci)HN(j)LN(cj)HN(k)LN(ck)HN(l)LN(cl)HN(m)L\\
\extracode
\begin{pgfonlayer}{background}
\begin{scope}[]
\foreach \n/\a in {ca/1,cb/2,cc/3,cd/4,ce/5,cf/6}\draw ($(\n |-row1.south)+(0,-3ex)$) node[]{$T_\a$};
\foreach \n/\a in {cg/1,ch/2,ci/3,cj/4,ck/5,cl/6}\draw ($(\n |-row1.south)+(0,-3ex)$) node[]{$T_\a$};
\foreach \n in {h,i,j,k,l}{\draw[thin]($(\n|-row1.south)+(0,-1ex)$)--($(\n|-row2.mid)$);}
\draw($(a|-row1.south)+(0,-1ex)$)--++(0,-14ex);
\draw($(g|-row1.south)+(0,-1ex)$)--++(0,-10ex);
\draw($(m|-row1.south)+(0,-1ex)$)--++(0,-14ex);
\draw($(d|-row1.south)+(0,-1ex)$)--($(d|-row3.north)+(0,1ex)$);
\draw[stealth-stealth]($(a|-row3.north)+(0,2ex)$)--node[fill=white]{\texturdu{\RL{بازیابی پھیرا}}}($(d|-row3.north)+(0,2ex)$);
\draw[stealth-stealth]($(d|-row3.north)+(0,2ex)$)--node[fill=white]{\texturdu{\RL{تعمیلی پھیرا}}}($(g|-row3.north)+(0,2ex)$);
\draw[stealth-stealth]($(g|-row3.north)+(0,2ex)$)--node[fill=white]{\texturdu{\RL{تعمیلی پھیرا}}}($(m|-row3.north)+(0,2ex)$);
\draw[stealth-stealth](a|-row3.mid)--node[fill=white]{\texturdu{\RL{مشینی پھیرا}}}(g|-row3.mid);
\draw[stealth-stealth](g|-row3.mid)--node[fill=white]{\texturdu{\RL{مشینی پھیرا}}}(m|-row3.mid);
\draw[stealth-stealth]($(a|-row3.south)+(0,-2ex)$)--node[fill=white]{\texturdu{\RL{ہدایتی پھیرا}}}($(m|-row3.south)+(0,-2ex)$);
%\draw[dashed] (cd)--(cd |-row8.south);
%%\vertlines[darkgray,dotted]{3.6,7.5,11.5,15.5,19.5,23.5,27.5}
%\foreach \n in {1,3,...,11} \draw(4*\n ex-2ex,-4ex+1.25ex)node[]{$0$};
\end{scope}
\end{pgfonlayer}
\end{tikztimingtable}
\end{otherlanguage}
\caption{}
\end{subfigure}
\caption{(ا) ہدایتی پھیرا؛ (ب) دو مشینی پھیروں پر مبنی ہدایتی پھیرا۔}
\label{شکل_کمپیوٹر_مشینی_پھیرے}
\end{figure}

%--------------------------------
\ابتدا{مثال}
\عددی{8080/8085} کا معلوماتی کتابچہ  کہتا ہے\قول{ نقل}  کی ہدایت  کی بازیابی اور تعمیل کے لئے تیرہ \عددی{T} حال درکار ہوں گے۔ اگر کمپیوٹر کی ساعت کا تعدد \عددی{\SI{2.5}{\mega\hertz}} ہو، اس  ہدایت کو کتنا وقت درکار ہو گا؟

حل:\quad
ساعت کا دوری عرصہ درج ذیل ہو گا۔
\begin{align*}
T=\frac{1}{f}=\frac{1}{\SI{2.5}{\mega\hertz}}=\SI{400}{\nano\second}
\end{align*}
چونکہ  ہر ایک \عددی{T} حال کو   \عددی{\SI{400}{\nano\second}} درکار ہیں  اور \قول{نقل} کی ہدایت کی بازیابی اور تعمیل   تیرہ \عددی{T} حال    میں  ممکن  ہے لہٰذا اس ہدایت کو درج ذیل وقت درکار ہو گا۔
\begin{align*}
13\times \SI{400}{\nano\second}=\SI{5.2}{\micro\second}
\end{align*}
\انتہا{مثال}
%------------------------------------
\ابتدا{مثال}
شکل \حوالہ{شکل_کمپیوٹر_کنارہ_چڑھائی_وسط_میں}  میں سادہ کمپیوٹر کے چھ \عددی{T} حال دکھائے گئے ہیں۔ ساعت کا (تیر دار)   کنارہ چڑھائی   نصف حال گزر کر آتا ہے۔ ایسا کیوں ہے؟

\begin{figure}
\centering
\begin{otherlanguage}{english}
 \begin{tikztimingtable}[%
timing/.style={x=4ex,y=3ex},
timing/rowdist=6ex,
every node/.style={inner sep=0,outer sep=0},
timing/c/arrow tip=latex, %and this set the style
timing/c/rising arrows,
timing/slope=0, %0.1 is good
timing/dslope=0,
thick,
]
%\tikztimingmetachar{R}{[|/utils/exec=\setcounter{new}{0}|]}
%\usetikztiminglibrary[new={char=Q,reset char=R}]{counters}
%[timing/counter/new={char=c, base=2,digits=3,max value=7, wraps ,text style={font=\normalsize}}] 12{2c} \\ 
%$C$& H22{C}\\
%$\texturdu{\RL{پتہ}}$&3D{} 20D{[scale=1.5]\texturdu{\RL{درست پتہ}}} 3D{}\\
$CLK$&HN(a)CN(b)CN(c)CN(d)CN(e)CN(f)CN(g)CN(h)CN(i)CN(j)CN(k)CN(l)CN(m)L\\
\extracode
\begin{pgfonlayer}{background}
\begin{scope}[]
\foreach \n in {a,c,e,g,i,k,m}{\draw (\n|-row1.south)++(0,-1ex)--(\n|-row2.mid);}
\foreach \n/\a  in {b/1,d/2,f/3,h/4,j/5,l/6}{\draw(\n|-row1.south)++(0,-3ex)node[]{$T_{\a}$};}
\end{scope}
\end{pgfonlayer}
\end{tikztimingtable}
\end{otherlanguage}
\caption{ساعت کا کنارہ چڑھائی \عددی{T} حال کے وسط میں پایا جاتا ہے۔}
\label{شکل_کمپیوٹر_کنارہ_چڑھائی_وسط_میں}
\end{figure}

حل:\quad
جدید کمپیوٹر کی طرح اس کمپیوٹر میں مواد کا تبادلہ بذریعہ \عددی{W}  گزرگاہ ہوتا ہے۔ تاہم دفتر  کی بغیر مسئلہ  بھرائی اس صورت ممکن ہو گی  جب دورانیہ تیاری اور  دورانیہ  ٹھیراؤ  مطمئن ہوں۔ نصف  دوری عرصہ انتظار   کر  کے دفتر میں  بھرائی ، دورانیہ تیاری کو مطمئن کرتا ہے؛ بھرائی کے بعد نصف دوری عرصہ کا انتظار، دورانیہ ٹھیراؤ کو مطمئن کرتا ہے۔ اسی لئے ساعت کا کنارہ چڑھائی  \عددی{T} حال کے عین وسط میں رکھا جاتا ہے (شکل  \حوالہ{شکل_کمپیوٹر_کنارہ_چڑھائی_وسط_میں})۔

نصف دوری عرصہ انتظار کرنے کی  دوسری وجہ بھی ہے۔ مواد ترسیل کرنے والے دفتر کا \قول{مجاز}   اشارہ فعال کرنے  سے \عددی{W} گزرگاہ  پر مواد ایک دم ڈلتا ہے۔ غیر مطلوبہ  برقی گنجائش اور تاروں کے    امالہ کی بدولت  گزرگاہ  تاروں میں برقی دباو کی درست  سطح  کے حصول میں وقت درکار ہوتا ہے۔ دوسرے لفظوں میں   \عددی{W} گزرگاہ  پر عبوری حال   پیدا ہو گا؛بوقت بھرائی درست مواد یقینی بنانے کے لئے ضروری ہے کہ  ا عبوری حال  کے اختتام  کا  انتظار کیا جائے۔
\انتہا{مثال}
%----------------------------

\حصہ{خرد برنامہ}
ہم جلد اس سادہ کمپیوٹر   کے دوری  نقشہ  پر غور کریں گے، لیکن اس سے قبل بہتر ہو گا ہم اس کی ہدایات کی تعمیل کو ایک  جدول میں،  جسے\اصطلاح{ خرد برنامہ }\فرہنگ{خرد برنامہ}\حاشیہب{microprogram}\فرہنگ{microprogram} کہتے ہیں،  یکجا کریں۔

\جزوحصہء{خرد ہدایات}
ہر ایک \عددی{T} حال کے دوران قابو و ترتیب کار ایک قابو لفظ خارج کرتا ہے۔ یہ  لفظ  کمپیوٹر کے باقی حصوں کو  بتاتا ہے کہ ان  نے کیا کام سرانجام دینا ہے۔ چونکہ   یہ  لفظ مواد پر عمل  کا ایک چھوٹا قدم  پیدا کرتا ہے لہٰذا یہ \اصطلاح{ خرد ہدایت  }\فرہنگ{خرد ہدایت}\حاشیہب{microinstruction}\فرہنگ{microinstruction} کہلاتا ہے۔ شکل \حوالہ{شکل_کمپیوٹر_سادہ_ترین} کو دیکھتے ہوئے   قابو و ترتیب کار سے باقی ادوار کو مسلسل خرد ہدایات جاری ہونا  ہم تصور کر سکتے ہیں۔

\جزوحصہء{کلاں ہدایات}
برنامے کی ہدایات (نقل، جمع، منفی، وغیرہ) کو  بعض اوقات\اصطلاح{ کلاں ہدایات }\فرہنگ{کلاں ہدایات}\حاشیہب{macroinstructions}\فرہنگ{macroinstructions} کہتے ہیں تا کہ ان میں اور خرد ہدایات میں تمیز    ہو۔ سادہ ترین کمپیوٹر کی ہر ایک کلاں ہدایت تین خرد ہدایات پر مشتمل ہے۔ مثلاً، نقل کی کلاں ہدایت جدول \حوالہ{جدول_کمپیوٹر_کلاں_اور_خرد} میں پیش تین خرد ہدایات پر مشتمل ہے۔  آسان بنانے کی غرض سے ہم  خرد ہدایات کو اساس سولہ میں لکھتے ہیں (جدول \حوالہ{جدول_کمپیوٹر_خرد_اساس_سولہ} دیکھیں)۔
\begin{table}
\caption{نقل ہدایت تین خرد ہدایات پر مشتمل ہے۔}
\label{جدول_کمپیوٹر_کلاں_اور_خرد}
\centering
\begin{tabular}{RRRR}
\toprule
\text{\RL{کلاں}}&\text{\RL{حال}}&
C_PE_P\overline{L}_M\overline{CE}\quad  \overline{L}_I\overline{E}_I\overline{L}AE_A\quad S_UE_U\overline{L}_B\overline{L}_O&\text{\RL{فعال}}\\
\midrule
\text{\RL{نقل}}& T_4& 
\,\,0\,\,\,\,\,0\,\,\,\,\,0\,\,\,\,\,1\quad\,\, 1\,\,\,\,0\,\,\,\,1\,\,\,\,0\quad \quad 0\,\,\,\,0\,\,\,\,1\,\,\,\,1&\overline{L}_M, \overline{E}_I\\
& T_5& 
\,\,0\,\,\,\,\,0\,\,\,\,\,1\,\,\,\,\,0\quad\,\, 1\,\,\,\,1\,\,\,\,0\,\,\,\,0\quad \quad 0\,\,\,\,0\,\,\,\,1\,\,\,\,1&\overline{CE}, \overline{L}_A\\
& T_6& 
\,\,0\,\,\,\,\,0\,\,\,\,\,1\,\,\,\,\,1\quad\,\, 1\,\,\,\,1\,\,\,\,1\,\,\,\,0\quad \quad 0\,\,\,\,0\,\,\,\,1\,\,\,\,1&\text{\RL{کوئی نہیں}}\\
\bottomrule
\end{tabular}
\end{table}


\begin{table}
\begin{minipage}[t]{0.45\textwidth}
\centering
\caption{نقل ہدایت  کی اساس سولہ  خرد ہدایات۔}
\label{جدول_کمپیوٹر_خرد_اساس_سولہ}
\begin{tabular}{RRRR}
\toprule
\text{\RL{کلاں}}&\text{\RL{حال}}&
\text{\RL{قابو لفظ}}&\text{\RL{فعال}}\\
\midrule
\text{\RL{نقل}}& T_4& 
1A3H&\overline{L}_M, \overline{E}_I\\
& T_5& 
2C3H&\overline{CE}, \overline{L}_A\\
& T_6& 
3E3H&\text{\RL{کوئی نہیں}}\\
\bottomrule
\end{tabular}
\end{minipage}\hfill
\begin{minipage}[t]{0.45\textwidth}
\centering
\caption{سادہ کمپیوٹر کا خرد  برنامہ}
\label{جدول_کمپیوٹر_خرد_برنامہ}
\begin{tabular}{RRRL}
\toprule
\text{\RL{کلاں}}&\text{\RL{حال}}&
\text{\RL{قابو لفظ}}&\text{\RL{فعال}}\\
\midrule
\text{\RL{نقل}}& T_4& 
1A3H&\overline{L}_M, \overline{E}_I\\
& T_5& 
2C3H&\overline{CE}, \overline{L}_A\\
& T_6& 
3E3H&\text{\RL{کوئی نہیں}}\\
\text{\RL{جمع}}& T_4 & 1A3H&\overline{L}_M, \overline{E}_I\\
&T_5&2E1H&\overline{CE},\overline{L}_B\\
&T_6&3C7H&\overline{L}_A, E_U\\
\text{\RL{منفی}}& T_4& 1A3H&\overline{L}_M, \overline{E}_I\\
&T_5&2E1H&\overline{CE},\overline{L}_B\\
&T_6&3CFH&\overline{L}_A, S_U,E_U\\
\text{\RL{بر آمد}}&T_4&3F2H&E_A,\overline{L}_O\\
&T_5&3E3H&\text{\RL{کوئی نہیں}}\\
&T_6&3E3H&\text{\RL{کوئی نہیں}}\\
\bottomrule
\end{tabular}
\end{minipage}
\end{table}

جدول \حوالہ{جدول_کمپیوٹر_خرد_برنامہ} میں سادہ کمپیوٹر کا خرد برنامہ پیش ہے، جس میں ہر کلاں ہدایت اور  اس کی تعمیل کے لئے درکار خرد ہدایات  دیے گئے ہیں۔ یہ  جدول  سادہ کمپیوٹر کے طریق تعمیل   کا خلاصہ ہے۔ زیادہ جدید ہدایات کے لئے بھی ایسا جدول لکھا جا سکتا ہے۔

\حصہ{سادہ کمپیوٹر کا نقشہ دور}
اس حصے میں سادہ کمپیوٹر کے مکمل نقشہ دور پر غور کیا جائے گا۔ شکل  \حوالہ{شکل_کمپیوٹر_برنامہ_گنتکار}  تا شکل  \حوالہ{شکل_کمپیوٹر_قابو_دور} میں تمام مخلوط ادوار، برقی تاریں، اور اشارات دکھائے گئے ہیں۔ آگے پڑھتے ہوئے ان  اشکال سے رجوع کریں۔جہاں ضرورت ہو، مستعمل  مخلوط ادوار  کی معلومات انٹرنیٹ سے حاصل کریں۔

\جزوحصہء{برنامہ گنت کار}
شکل \حوالہ{شکل_کمپیوٹر_برنامہ_گنتکار} میں مخلوط ادوار \عددی{u1}، \عددی{u2}، اور \عددی{u3} \قول{ برنامہ گنت کار } دیتے ہیں۔مخلوط دور  \عددی{u1}  ، جو \عددی{74LS107} ہے، دوہرا جے کے پلٹ کار ہے، جو پتہ کے  زیریں دو بِٹ \عددی{A_0} اور \عددی{A_1} دیتا ہے۔ \عددی{u2} دوسرا \عددی{74LS107}  ہے جو پتہ کے بالا بِٹ \عددی{A_2} اور \عددی{A_3} دیتا ہے۔ \عددی{u3} \عددی{(74LS126)}   سہ حال چو   مستحکم کار ہے جو برنامہ گنت کار  کے بِٹ \عددی{A_0} تا \عددی{A_3} کو \عددی{W} گزرگاہ پر ضرورت کے وقت ڈالنے کی صلاحیت دیتا ہے۔

کمپیوٹر کی دوڑ سے قبل ، پست \عددی{\overline{CLR}} برنامہ گنت کار کو  زبردستی پست  \عددی{(0000)} کرتا ہے۔ \عددی{T_1} حال کے دوران بلند \عددی{E_P} پتے کو \عددی{W} گزرگاہ پر ڈالتا ہے۔\عددی{T_2}    کے دوران  برنامہ گنت کار کو بلند \عددی{C_P} مہیا کیا جاتا ہے؛  نصف حال گزر کر \عددی{\overline{CLK}} کا  کنارہ اترائی (جو \عددی{CLK} کے کنارہ چڑھائی کے مترادف ہے) برنامہ گنت کار  کی گنتی میں \عددی{1} کا اضافہ کرتا ہے۔

 \عددی{T_3} تا \عددی{T_6} حال کے دوران برنامہ گنت کار  غیر فعال ہو گا۔
 
 شکل \حوالہ{شکل_کمپیوٹر_برنامہ_گنتکار} میں  \عددی{u1} کے پنیا  \عددی{12} کو \عددی{\overline{CLK}} کا اشارہ فراہم کیا گیا ہے جو  درحقیقت شکل \حوالہ{شکل_کمپیوٹر_ساعت_منبع_طاقت} میں  \عددی{u27} کے پنیا \عددی{6} سے آتا ہے۔صفائی کی خاطر،  نقشہ جات میں لمبی تاروں کو کھینچ کر دکھانے سے گریز کیا جاتا ہے۔ایسی تار کے دونوں سروں کو ایک نام دے کر جوڑ ظاہر کیا جاتا ہے۔ یوں   شکل \حوالہ{شکل_کمپیوٹر_برنامہ_گنتکار} میں  \عددی{u1} کے پنیا  \عددی{12} اور   شکل \حوالہ{شکل_کمپیوٹر_ساعت_منبع_طاقت} میں  \عددی{u27} کے پنیا \عددی{6} کو ایک نام \عددی{(\overline{CLK})} دے کر  انہیں آپس میں جڑا ظاہر کیا گیا ہے۔
 
 \جزوحصہء{دفتر پتہ}
 مخلوط دور \عددی{u4}  \عددی{(74LS173)}   چار بِٹ   سہ حال  مستحکم کار ہے ، جو  بطور  \قول{دفتر پتہ } کردار ادا کرتا ہے۔ دھیان  رہے، پنیا \عددی{1} اور \عددی{2}  برقی زمین سے جڑے ہیں، جس کی بدولت \عددی{u4} سہ حال کی بجائے دو حال ہو گا۔ دوسرے  لفظوں میں، چونکہ یہ گزرگاہ سے نہیں جڑا لہٰذا  اس کی سہ حال صلاحیت درکار نہیں۔
 
 \جزوحصہء{دو تا ایک داخلی منتخب کار}
 مخلوط دور  \عددی{u5}  \عددی{(74LS157)} \عددی{2} تا \عددی{1} ریزہ \قول{    داخلی منتخب کار }ہے۔ بایاں ریزہ (پنیے \عددی{2}، \عددی{5}، \عددی{11}، اور \عددی{14})  پتہ سوئچ \عددی{S_1} سے آتا ہے جو \عددی{AA_0} تا \عددی{AA_3} بِٹ دستی  مہیا کرتا ہے۔اس سوئچ کے چار بازووں کو انفرادی کھڑا یا بٹھایا جا سکتا ہے۔(یاد رہے \عددی{74LSxxxx}   سلسلہ کے مخلوط ادوار کا داخلی  پنیا برقی زمین سے جوڑنے سے پنیے پر \عددی{0}   ہو گا، جبکہ آزاد (منقطع)   پنیے   پر \عددی{1}  ہو گا۔)    \عددی{S_1}   کا  بازو  بٹھانے سے \عددی{u5} کے مطابقتی پنیا کو \عددی{0}  جبکہ کھڑا کرنے سے  \عددی{1}   مہیا کیا جاتا ہے۔  دایاں ریزہ (پنیے \عددی{13}، \عددی{10}، \عددی{6}، اور \عددی{3}) دفتر پتہ \عددی{(u4)} سے آتا ہے۔ داخلی منتخب کار \عددی{u5}  کے مخارج تک پہنچنے والے ریزے   کا فیصلہ سوئچ \عددی{S_{2a}}  کرتا ہے۔ جب \عددی{S_{2a}}   \قول{برنامہ لکھ  } کی جانب (یعنی پست)  ہو تب \عددی{S_1} کا  دستی پتہ  \عددی{u5} کے مخارج \عددی{(a_3a_2a_1a_0)}  منتقل ہو گا، اور جب \عددی{S_{2a}} \قول{ دوڑ } کی جانب  (یعنی بلند)    ہو تب دفتر پتہ \عددی{u4} کا مواد  (پتہ) \عددی{u5} کے مخارج کو پہنچے گا۔ سوئچ \عددی{S_2}  کے دو بازو، جنہیں \عددی{S_{2a}} اور \عددی{S_{2b}} کہا گیا ہے، ایک ساتھ کھڑا ہوں گے یا بیٹھیں گے؛ ان کو انفرادی کھڑا کرنا  یا بٹھانا ممکن  نہیں۔
 
 \جزوحصہء{\عددی{16\times 8} عارضی حافظہ}
   \عددی{u6} اور \عددی{u7} مخلوط دور \عددی{74LS189} ہیں۔  \عددی{74LS189} مخلوط دور \عددی{16\times 4} عارضی حافظہ ہے۔  \عددی{u6} اور \عددی{u7} مل کر \عددی{16\times 8} \قول{ عارضی حافظہ } دیتے ہیں۔ سوئچ  \عددی{S_3}  آٹھ بِٹ مواد (\عددی{D_0} تا \عددی{D_7})  فراہم کرتا ہے۔اس کے آٹھ بازووں کو انفرادی کھڑا یا بٹھایا جا  سکتا  ہے۔ سوئچ \عددی{S_4}   (جو  \اصطلاح{داب بتام }\فرہنگ{داب بتام}\حاشیہب{push button}\فرہنگ{push button}ہے)  
   \عددی{\overline{\text{\RL{لکھ}}}\!/\!\text{\RL{پڑھ}}}  اشارہ فراہم کرتا ہے۔ اس کا بازو عام طور      (دکھائی گئی صورت میں)  کھڑا رہتا ہے۔ بازو بٹھانے کے لئے اسے دبانا ہو گا۔ حافظہ کی برنامہ نویسی   \عددی{S_{2b}} (اور \عددی{S_{2a}})   \قول{برنامہ لکھ } پر رکھ کر ہو گی۔\عددی{S_{2b}} کو بٹھا کر \عددی{\overline{CE}} (پنیا \عددی{2})   پست کر کے  درست(  \عددی{AA_0}   تا \عددی{AA_3}) دستی پتہ  اور (\عددی{D_0} تا \عددی{D_3})  دستی مواد بِٹ    رکھ کر  \عددی{S_4} کو ایک لمحے کے لئے دبا کر \عددی{\overline{\text{\RL{لکھ}}}} (پنیا \عددی{3})    لمحاتی پست  کرتے ہوئے حافظہ میں  مطلوبہ پتے  \عددی{(a_3a_2a_1a_0)} پر مواد لکھا جاتا ہے۔
   
یاد رہے برنامہ نویسی کے دوران  \عددی{S_2} (یعنی \عددی{S_{2a}} اور \عددی{S_{2b}}) کے بازو برنامہ لکھ پر ہوں گے جس کی بدولت \عددی{AA_0} تا \عددی{AA_3} دستی  پتہ اور \عددی{D_0} تا \عددی{D_7} دستی  مواد حافظہ کو فراہم ہو گا۔
   
   حافظہ میں برنامہ اور مواد لکھنے کے بعد \عددی{S_2} کا بازو \قول{دوڑ}  پر رکھ کر  کمپیوٹر کو چلنے کے لئے تیار کیا جاتا ہے۔
   
\جزوحصہء{دفتر ہدایت}
 \عددی{u8} اور \عددی{u9} مخلوط دور  \عددی{74LS173} ہیں۔ ایک مخلوط دور  میں سہ حال \عددی{4} بِٹ مستحکم کار دفاتر پائے جاتے ہیں۔ یہ دو مخلوط ادوار مل کر \عددی{8} بِٹ \قول{  دفتر ہدایت   } دیتے ہیں۔ \عددی{u8}  کے \عددی{1} اور \عددی{2} پنیے زمین سے جوڑ کر   مخلوط دور کا مخارج \عددی{I_7I_6I_5I_4} دو حال بنایا گیا ہے۔ یہ ریزہ قابو و ترتیب کار  کے \قول{ ہدایت رمز  کشا  } کو جاتا ہے۔ دفتر ہدایت کے  زیریں ریزہ کو،  جو \عددی{u9} کا مخارج  ہے، \عددی{\overline{E}_I} اشارہ قابو کرتا ہے۔ پست \عددی{\overline{E}_I}  اس ریزہ کو \عددی{W}  گزرگاہ پر ڈالتا ہے۔

\begin{figure}
\centering
\begin{tikzpicture}
\pgfmathsetmacro{\kpin}{0.30}
\pgfmathsetmacro{\kpsep}{0.40}			%pin to pin distance
\pgfmathsetmacro{\kul}{0.40}			%edge clearance along vertical edges
\pgfmathsetmacro{\kdimX}{2*\kul+0*\kpsep}
\pgfmathsetmacro{\kdimY}{2*\kul+2*\kpsep}	
\pgfmathsetmacro{\ksepX}{2.00}
\pgfmathsetmacro{\ksepY}{1.25}
\pgfmathsetmacro{\kpinA}{0.5}
\pgfmathsetmacro{\kpinB}{1.5}
\pgfmathsetmacro{\kmv}{0.15}
\pgfmathsetmacro{\kw}{0.20}
\kSfozSA[uu1]{0}{0}
\kSfozSB[uu2]{\ksepX}{0}
\kSfozSA[uu3]{2*\ksepX}{0}
\kSfozSB[uu4]{3*\ksepX}{0}
\draw[thin](uu1-north)node[above,yshift=1ex,rectangle,inner sep=0pt,text width=2cm,align=center]{u1\\74LS107};
\draw[thin](uu2-north)node[above,yshift=1ex,rectangle,inner sep=0pt,text width=2cm,align=center]{u1\\74LS107};
\draw[thin](uu3-north)node[above,yshift=1ex,rectangle,inner sep=0pt,text width=2cm,align=center]{u2\\74LS107};
\draw[thin](uu4-north)node[above,yshift=1ex,rectangle,inner sep=0pt,text width=2cm,align=center]{u2\\74LS107};
\kSfotsA[uu5]{4*\ksepX}{-\ksepY}
\kSfotsB[uu6]{4*\ksepX}{-2*\ksepY}
\kSfotsC[uu7]{4*\ksepX}{-3*\ksepY}
\kSfotsD[uu8]{4*\ksepX}{-4*\ksepY}
\draw[dashed,lgray]($(uu5pin|-uu5u)+(-0.5ex,2ex)$)rectangle ($(uu8pout|-uu8d)+(2ex,-1ex)$);
\draw[thin](uu5u)node[above,yshift=3ex,rectangle,inner sep=0pt,text width=1cm,align=center]{u3\\74LS126};
\draw[thin] (uu1p6)|-coordinate(cntA)(uu2p2)  (uu2p6)|-coordinate(cntB)(uu3p2)  (uu3p6)|-coordinate(cntC)(uu4p2);
\draw[thin](cntA)|-(uu8pin) (cntB)|-(uu7pin)  (cntC)|-(uu6pin)  (uu4p6)|-coordinate(kAddressLoc)(uu5pin);
\draw[thin](uu1p2)--++(-1*\kpin,0)coordinate(kClft)node[left]{$\overline{CLK}$};
\draw[thin](uu1p3)--(uu1p1)--++(0,\kpinB)coordinate(kCPl);
\draw[thin](uu2p3)--(uu2p1)--++(0,\kpinB);
\draw[thin](uu3p3)--(uu3p1)--++(0,\kpinB);
\draw[thin](uu4p3)--(uu4p1)--++(0,\kpinB)coordinate(kCPr)--(kCPl)--(kCPl-|kClft)node[left]{$C_P$};
\draw[thin](4.25*\ksepX,1*\kdimY)node[]{\text{\RL{برنامہ گنت کار}}};
\draw[thin](uu5u)--++(0,0.5ex)--++(3*\kpin,0)coordinate(kEPtop);
\draw[thin](uu6u)--++(0,0.5ex)--++(3*\kpin,0);
\draw[thin](uu7u)--++(0,0.5ex)--++(3*\kpin,0);
\draw[thin](uu8u)--++(0,0.5ex)--++(3*\kpin,0)coordinate(kEPbot);
\draw[thin](kEPbot)--(kEPtop)--(kEPtop-|kClft)node[left,yshift=-0.5ex]{$E_P$};
\draw[thin](kAddressLoc)node[left]{$A_3$};
\draw[thin](uu6pin-|kAddressLoc)node[above left]{$A_2$};
\draw[thin](uu7pin-|kAddressLoc)node[above left]{$A_1$};
\draw[thin](uu8pin-|kAddressLoc)node[above left]{$A_0$};
\draw[thin](uu4pd)--(uu1pd)--(uu1pd-|kClft)node[left,yshift=0.5ex]{$\overline{CLR}$};
\draw[thin](uu5pout)++(7*\kpin,10*\kpin)node[above]{$w_0$}coordinate(kwa)--($(uu8pout)+(7*\kpin,-35*\kpin)$)coordinate(kwbot);
\draw[thin](kwa)++(-\kw,0)coordinate(kwb)--(kwb|-kwbot);
\draw[thin](kwa)++(-2*\kw,0)coordinate(kwc)--(kwc|-kwbot);
\draw[thin](kwa)++(-3*\kw,0)coordinate(kwd)--(kwd|-kwbot);
\draw[thin](kwa)++(-4*\kw,0)coordinate(kwe)--(kwe|-kwbot);
\draw[thin](kwa)++(-5*\kw,0)coordinate(kwf)--(kwf|-kwbot);
\draw[thin](kwa)++(-6*\kw,0)coordinate(kwg)--(kwg|-kwbot);
\draw[thin](kwa)++(-7*\kw,0)node[above]{$w_7$}coordinate(kwh)--(kwh|-kwbot);
\draw[thin](uu8pout)--(uu8pout-|kwd);
\draw[thin](uu7pout)--(uu7pout-|kwc);
\draw[thin](uu6pout)--(uu6pout-|kwb);
\draw[thin](uu5pout)--(uu5pout-|kwa);
\draw[thin](kwd)node[yshift=6ex,rectangle,inner sep=0pt,text width=2cm,align=center]{\RTL{$W$ گزرگاہ}};
\draw[thick] (3*\ksepX,-8*\ksepY)coordinate(kLftBot)rectangle node[rectangle, inner sep=0pt,align=center,text width=1cm]{u4 \\ 74LS173}++(2*\kul+3*\kpsep,2*\kul+4*\kpsep);
\draw[thin] (2*\ksepX,-8*\ksepY+3*\kpsep)node[]{\text{\RL{دفتر پتہ}}};
\draw[thin](kLftBot)++(\kul+0*\kpsep,2*\kul+4*\kpsep)node[above,xshift=1.25ex]{\scriptsize{$14$}}--++(0,\kpin+4*\kw)coordinate(kd)--(kd-|kwd);
\draw[thin](kLftBot)++(\kul+1*\kpsep,2*\kul+4*\kpsep)node[above,xshift=1.25ex]{\scriptsize{$13$}}--++(0,\kpin+3*\kw)coordinate(kc)--(kc-|kwc);
\draw[thin](kLftBot)++(\kul+2*\kpsep,2*\kul+4*\kpsep)node[above,xshift=1.25ex]{\scriptsize{$12$}}--++(0,\kpin+2*\kw)coordinate(kb)--(kb-|kwb);
\draw[thin](kLftBot)++(\kul+3*\kpsep,2*\kul+4*\kpsep)node[above,xshift=1.25ex]{\scriptsize{$11$}}--++(0,\kpin+1*\kw)coordinate(ka)--(ka-|kwa);
\draw[thin](kLftBot)++(2*\kul+3*\kpsep,1*\kul+3*\kpsep)node[above right]{\scriptsize{$10$}}--++(2*\kpin,0)--++(0,\kpsep);
\draw[thin](kLftBot)++(2*\kul+3*\kpsep,1*\kul+4*\kpsep)node[above right]{\scriptsize{$9$}}--++(3*\kpin,0)node[right]{$\overline{L}_M$};
\draw[thin](kLftBot)++(2*\kul+3*\kpsep,1*\kul+3*\kpsep)++(\knshift,0)node[ocirc]{};
\draw[thin](kLftBot)++(2*\kul+3*\kpsep,1*\kul+4*\kpsep)++(\knshift,0)node[ocirc]{};
\draw[thin](kLftBot)++(2*\kul+3*\kpsep,1*\kul+2*\kpsep)node[above right]{\scriptsize{$7$}}--++(3*\kpin,0)node[right]{$CLK$};
\draw[thin](kLftBot)++(2*\kul+3*\kpsep,1*\kul+2*\kpsep)++(0,-\kmv)--++(-\kmv,\kmv)--++(\kmv,\kmv);
\draw[thin](kLftBot)++(0,1*\kul+0*\kpsep)node[above left]{\scriptsize{$15$}}--++(-2*\kpin,0)coordinate(lftMARa);
\draw[thin](kLftBot)++(0,1*\kul+1*\kpsep)node[above left]{\scriptsize{$8$}}--++(-2*\kpin,0);
\draw[thin](kLftBot)++(0,1*\kul+2*\kpsep)node[above left]{\scriptsize{$2$}}+(-\knshift,0)node[ocirc]{}--++(-2*\kpin,0);
\draw[thin](kLftBot)++(0,1*\kul+3*\kpsep)node[above left]{\scriptsize{$1$}}+(-\knshift,0)node[ocirc]{}--++(-2*\kpin,0)coordinate(lftMARb);
\draw[thin](lftMARb)--(lftMARa)node[ground]{};
\draw[thin](kLftBot)++(0,1*\kul+4*\kpsep)node[above left]{\scriptsize{$16$}}--++(-2*\kpin,0)node[left]{$\SI{5}{\volt}$};
\draw[thick] (kLftBot)++(-4*\kpsep,-4*\ksepY)coordinate(klftMUX)rectangle node[rectangle,inner sep=0pt,text width=1cm, align=center]{u5\\ 74LS157}++ (2*\kul+7*\kpsep,2*\kul+4*\kpsep);
\draw[thin](klftMUX)++(\kul+4*\kpsep,2*\kul+4*\kpsep)node[above,xshift=1.25ex]{\scriptsize{$13$}}--($(kLftBot)+(\kul+0*\kpsep,0)$)node[below,xshift=1.25ex]{\scriptsize{$3$}};
\draw[thin](klftMUX)++(\kul+5*\kpsep,2*\kul+4*\kpsep)node[above ,xshift=1.25ex]{\scriptsize{$10$}}--($(kLftBot)+(\kul+1*\kpsep,0)$)node[below,xshift=1.25ex]{\scriptsize{$4$}};
\draw[thin](klftMUX)++(\kul+6*\kpsep,2*\kul+4*\kpsep)node[above ,xshift=1.25ex]{\scriptsize{$6$}}--($(kLftBot)+(\kul+2*\kpsep,0)$)node[below,xshift=1.25ex]{\scriptsize{$5$}};
\draw[thin](klftMUX)++(\kul+7*\kpsep,2*\kul+4*\kpsep)node[above ,xshift=1.25ex]{\scriptsize{$3$}}--($(kLftBot)+(\kul+3*\kpsep,0)$)node[below,xshift=1.25ex]{\scriptsize{$6$}};
\draw[thin](klftMUX)++(\kul+0*\kpsep,2*\kul+4*\kpsep)node[above ,xshift=1.25ex]{\scriptsize{$14$}}--++(0,\kpin)--++(-5*\kpin,0) to [nos,mirror,invert]node[below]{$S_1$}++(-2*\kpin,0)--++(-2*\kpin,0)coordinate(wrAddBot)node[left]{$AA_3$};
\draw[thin](klftMUX)++(\kul+1*\kpsep,2*\kul+4*\kpsep)node[above ,xshift=1.25ex]{\scriptsize{$11$}}--++(0,2.5*\kpin)--++(-5*\kpin-1*\kpsep,0) to [nos,mirror,invert]++(-2*\kpin,0)--++(-2*\kpin,0)node[left]{$AA_2$};
\draw[thin](klftMUX)++(\kul+2*\kpsep,2*\kul+4*\kpsep)node[above ,xshift=1.25ex]{\scriptsize{$5$}}--++(0,4*\kpin)--++(-5*\kpin-2*\kpsep,0) to [nos,mirror,invert]++(-2*\kpin,0)--++(-2*\kpin,0)node[left]{$AA_1$};
\draw[thin](klftMUX)++(\kul+3*\kpsep,2*\kul+4*\kpsep)node[above ,xshift=1.25ex]{\scriptsize{$2$}}--++(0,5.5*\kpin)--++(-5*\kpin-3*\kpsep,0) to [nos,mirror,invert]node[above,yshift=3ex]{\text{\RL{دستی پتہ}}}++(-2*\kpin,0)--++(-2*\kpin,0)coordinate(wrAddTop)node[left]{$AA_0$};
\draw[thin](wrAddTop)--(wrAddBot)node[ground]{};
\draw[thin](klftMUX)++(0,\kul+0*\kpsep)node[left,yshift=1.25ex]{\scriptsize{$15$}}+(-\knshift,0)node[ocirc]{}--++(-2*\kpin,0);
\draw[thin](klftMUX)++(0,\kul+1*\kpsep)node[left,yshift=1.25ex]{\scriptsize{$8$}}--++(-2*\kpin,0)--++(0,-\kpsep)node[ground]{};
\draw[thin](klftMUX)++(0,\kul+2*\kpsep)node[left,yshift=1.25ex]{\scriptsize{$1$}}--++(-12*\kpin,0) node[spdt,yscale=-1,rotate=0,anchor=out 1](sw){};
\draw[thin](sw.in)node[above]{$S_{2a}$}node[ground]{};
\draw[thin](sw.out 1)node[below]{\text{\RL{برنامہ لکھ}}};
\draw[thin](sw.out 2)node[above,xshift=-1ex]{\text{\RL{دوڑ}}};
\draw[thin](klftMUX)++(\kul+4*\kpsep,0)node[right,yshift=-1.25ex]{\scriptsize{$12$}}--++(0,-2*\kpin)node[below]{$a_3$};
\draw[thin](klftMUX)++(\kul+5*\kpsep,0)node[right,yshift=-1.25ex]{\scriptsize{$9$}}--++(0,-2*\kpin)node[below]{$a_2$};
\draw[thin](klftMUX)++(\kul+6*\kpsep,0)node[right,yshift=-1.25ex]{\scriptsize{$7$}}--++(0,-2*\kpin)node[below]{$a_1$};
\draw[thin](klftMUX)++(\kul+7*\kpsep,0)node[right,yshift=-1.25ex]{\scriptsize{$4$}}--++(0,-2*\kpin)node[below]{$a_0$};
\draw[thin](klftMUX)++(2*\kul+7*\kpsep,2*\kul+2*\kpsep)node[right,xshift=1ex,rectangle,inner sep=0pt,text width=2cm,align=center]{$2\times 1$\\ \text{\RL{داخلی منتخب کار}}};
\end{tikzpicture}
\caption{برنامہ گنت کار}
\label{شکل_کمپیوٹر_برنامہ_گنتکار}
\end{figure}
%===================================================================
\begin{figure}
\centering
\begin{tikzpicture}
\pgfmathsetmacro{\kw}{0.20}
\pgfmathsetmacro{\kpin}{0.50}
\pgfmathsetmacro{\kpsep}{0.40}			%pin to pin distance
\pgfmathsetmacro{\kpinA}{0.5}
\pgfmathsetmacro{\kpinS}{0.4}
\pgfmathsetmacro{\ksepX}{5}
\pgfmathsetmacro{\ksepY}{9}
\kSfoen[uu6]{0}{0}
\draw[thin](uu6-center)node[inner sep=0pt,text width=1cm, align=center,]{u6\\ 74189};
\foreach \n/\lbl in {1/4,2/5,3/6,4/7}\draw[thin] (uu6ap\n) to [nos,mirror,invert]++(-\kpinS,0)--++(-0.5*\kpin,0)node[left]{$D_{\lbl}$};
\draw[thin](uu6ap4)++(-\kpinS-0.5*\kpin,0)coordinate(kA)--(kA|-uu6ap1)coordinate(klft)node[ground]{};
\draw(uu6ap4)node[above left,yshift=2ex]{$S_3$};
\kSfoen[uu7]{\ksepX}{0}
\draw[thin](uu7-center)node[inner sep=0pt,text width=1cm, align=center,]{u7\\ 74189};
\foreach \n/\lbl in {1/0,2/1,3/2,4/3}\draw[thin] (uu7ap\n) to [nos,mirror,invert]++(-\kpinS,0)--++(-0.5*\kpin,0)node[left]{$D_{\lbl}$};
\draw[thin](uu7ap4)++(-\kpinS-0.5*\kpin,0)coordinate(kA)--(kA|-uu7ap1)node[ground]{};
\draw(uu7ap4)node[above left,yshift=2ex]{$S_3$};
\foreach \n/\lbl in {0/3,1/2,2/1,3/0}\draw[thin](uu7dp\n)--++(0,2.5*\kpinA)node[above]{$a_{\lbl}$};
\draw[thin](uu6dp0)--++(0,\kpin+3*\kw)coordinate(kB)--(kB-|uu7dp0);
\draw[thin](uu6dp1)--++(0,\kpin+2*\kw)coordinate(kB)--(kB-|uu7dp1);
\draw[thin](uu6dp2)--++(0,\kpin+1*\kw)coordinate(kB)--(kB-|uu7dp2);
\draw[thin](uu6dp3)--++(0,\kpin+0*\kw)coordinate(kB)--(kB-|uu7dp3);
\draw[thin](uu6cp3)node[ground]{}  (uu7cp3)node[ground]{};
\draw[thin](uu6cp4)node[right]{$\SI{5}{\volt}$}  (uu7cp4)node[right]{$\SI{5}{\volt}$};
\draw[thin](uu7ap0)--++(0,-2*\kpin)coordinate(kC)--(kC-|uu6ap0)coordinate(kswlft)--++(0,-\kpin)--++(-\kpin,0)node[spdt,anchor=out 2](sw){};
\draw[thin](sw.in)node[above]{$S_4$} node[ground]{} (sw.out 1)node[above,xshift=-3ex]{\text{\RL{پڑھ}}} (sw.out 2)node[below,xshift=-3ex]{$\overline{\text{\RL{لکھ}}}$};
\draw[thin](uu6ap0)-|(kswlft);
\draw[thin](uu7cp0)--++(0,-2*\kpin-\kw)coordinate(kD)--($(kD-|uu6ap0)+(2*\kw,0)$)--++(0,-4*\kpin)node[spdt,anchor=in,xscale=-1,yscale=-1](sw){};
\draw[thin](sw.out 1)node[left,xshift=-2ex]{$S_{2b}$}node[right,xshift=2ex,yshift=-3ex]{\text{\RL{برنامہ لکھ}}}node[ground]{};
\draw[thin](sw.out 2)node[above,xshift=2ex]{\text{\RL{دوڑ}}}--++(-\kpin,0)node[left]{$\overline{CE}$};
\draw[thin](uu6cp0)--(uu6cp0|-kD);
\foreach \n in {0,1,2,3,4,5,6,7}\draw[thin](uu7dp3)++(0,2.5*\kpinA)++(3.5*\kpin+7*\kw-\n*\kw,0)coordinate(w\n)--++(0,-1.75*\ksepY);\
\draw[thin](uu7dp3)++(0,2.5*\kpinA)++(3.5*\kpin+0*\kw,0)node[above]{$w_7$}++(7*\kw,0)node[above]{$w_0$};
\draw[thin](uu7bp3)--++(0,-2*\kpin-0*\kw)coordinate(kD)--(kD-| w0);
\draw[thin](uu7bp2)--++(0,-2*\kpin-1*\kw)coordinate(kD)--(kD-| w1);
\draw[thin](uu7bp1)--++(0,-2*\kpin-2*\kw)coordinate(kD)--(kD-| w2);
\draw[thin](uu7bp0)--++(0,-2*\kpin-3*\kw)coordinate(kD)--(kD-| w3);
\draw[thin](uu6bp3)--++(0,-2*\kpin-4*\kw)coordinate(kD)--(kD-| w4);
\draw[thin](uu6bp2)--++(0,-2*\kpin-5*\kw)coordinate(kD)--(kD-| w5);
\draw[thin](uu6bp1)--++(0,-2*\kpin-6*\kw)coordinate(kD)--(kD-| w6);
\draw[thin](uu6bp0)--++(0,-2*\kpin-7*\kw)coordinate(kD)--(kD-| w7);
\kSfoSt[uu8]{0}{-\ksepY}
\draw[thin](uu8-center)node[inner sep=0pt,text width=1cm,align=center]{u8\\ 74LS173};
\kSfoSt[uu9]{\ksepX}{-\ksepY}
\draw[thin](uu9-center)node[inner sep=0pt,text width=1cm,align=center]{u9\\ 74LS173};
\draw[thin](uu9dp3)--++(0,\kpin+0*\kw)coordinate(kD)--(kD-|w0);
\draw[thin](uu9dp2)--++(0,\kpin+1*\kw)coordinate(kD)--(kD-|w1);
\draw[thin](uu9dp1)--++(0,\kpin+2*\kw)coordinate(kD)--(kD-|w2);
\draw[thin](uu9dp0)--++(0,\kpin+3*\kw)coordinate(kD)--(kD-|w3);
\draw[thin](uu8dp3)--++(0,\kpin+4*\kw)coordinate(kD)--(kD-|w4);
\draw[thin](uu8dp2)--++(0,\kpin+5*\kw)coordinate(kD)--(kD-|w5);
\draw[thin](uu8dp1)--++(0,\kpin+6*\kw)coordinate(kD)--(kD-|w6);
\draw[thin](uu8dp0)--++(0,\kpin+7*\kw)coordinate(kD)--(kD-|w7);
\draw[thin](uu8ap2)--++(-\kpin,0)--++(0,-2*\kpsep)node[ground]{}  (uu8ap1)--++(-\kpin,0)  (uu8ap0)--++(-\kpin,0); 
\draw[thin](uu8ap3)--++(-\kpin,0)coordinate(kklft)node[left]{$CLR$}  (uu8ap4)--++(-\kpin,0)node[left]{$\SI{5}{\volt}$};
\draw[thin] (uu9ap4)--++(-\kpin,0)node[left]{$\SI{5}{\volt}$};
\draw[thin](uu9cp2)--++(0,-6*\kpin)coordinate(kD)--(kD-|kklft)node[left]{$CLK$};
\draw[thin](uu8cp2)--(uu8cp2|-kD);
\draw[thin](uu9bp3)--++(0,-0*\kw)coordinate(kD)--(kD-|w0);
\draw[thin](uu9bp2)--++(0,-1*\kw)coordinate(kD)--(kD-|w1);
\draw[thin](uu9bp1)--++(0,-2*\kw)coordinate(kD)--(kD-|w2);
\draw[thin](uu9bp0)--++(0,-3*\kw)coordinate(kD)--(kD-|w3);
\draw[thin](uu9cp4)--++(\kw,0)--++(0,-6*\kpin-3*\kpsep)coordinate(kD)--(kD-|kklft)node[left]{$\overline{L}_I$};
\draw[thin](uu9cp3)--++(\kw,0);
\draw[thin](uu8cp4)--++(\kw,0)coordinate(kE)--(kE|-kD);
\draw[thin](uu8cp3)--++(\kw,0);
\draw[thin](uu9ap2)--++(-\kpin,0)--++(0,-6*\kpin-2*\kpsep)coordinate(kD)--(kD-|kklft)node[left]{$\overline{E}_I$};
\draw[thin](uu9ap1)--++(-\kpin,0);
\draw[thin](uu9ap3)--++(-2.5*\kpin,0)node[ground]{};
\draw[thin](uu9ap0)node[ground]{};
\draw[thin](uu8bp3)--++(0,-\kpin)node[below]{$I_4$};
\draw[thin](uu8bp2)--++(0,-\kpin)node[below]{$I_5$};
\draw[thin](uu8bp1)--++(0,-\kpin)node[below]{$I_6$};
\draw[thin](uu8bp0)--++(0,-\kpin)node[below]{$I_7$};
\draw(uu8ap4)node[shift={(-0.25cm,1.5cm)}]{\text{\RL{دفتر ہدایت}}};
\draw(uu6ap4)node[shift={(-0.25cm,2cm)},inner sep=0pt,align=center]{$16\times 8$\\ \text{\RL{عارضی حافظہ}}};
\end{tikzpicture}
\caption{حافظہ اور دفتر ہدایت}
\label{شکل_کمپیوٹر_حافظہ_دفتر_ہدایت}
\end{figure}
%=====================================================================
\begin{figure}
\centering
\begin{tikzpicture}
\pgfmathsetmacro{\kw}{0.20}
\pgfmathsetmacro{\kpin}{0.30}
\pgfmathsetmacro{\kpsep}{0.40}			%pin to pin distance
\pgfmathsetmacro{\kw}{0.20}
\pgfmathsetmacro{\kul}{0.40}			%edge clearance along vertical edges
\pgfmathsetmacro{\kpinA}{0.5}
\pgfmathsetmacro{\kpinS}{0.4}
\pgfmathsetmacro{\ksepX}{5}
\pgfmathsetmacro{\ksepY}{4}
\kSfoStCW[uu10]{0}{0}
\kSfoStCW[uu11]{0}{-\ksepY}
\kSfotsCW[uu12]{0-\ksepX}{0}
\kSfotsCW[uu13]{0-\ksepX}{-\ksepY}
\draw(uu10-center)node[inner sep=0pt,align=center]{u10\\ 74LS173};
\draw(uu11-center)node[inner sep=0pt,align=center]{u11\\ 74LS173};
\draw(uu12-center)node[inner sep=0pt,align=center]{u12\\ 74LS126};
\draw(uu13-center)node[inner sep=0pt,align=center]{u13\\ 74LS126};
\draw[thin](uu10dp3)++(2*\kul+4*\kpin+7*\kw,2.5cm)coordinate(kWtop)coordinate(w0)--++(0,-60*\kpin)coordinate(kWbot);
\foreach \n in {1,2,3,4,5,6,7} \draw[thin](kWtop)++(-\n*\kw,0)coordinate(w\n)coordinate(kA)--(kA|-kWbot);
\draw[thin](kWtop)node[above]{$w_0$} +(-7*\kw,0)node[above]{$w_7$};
\draw(uu10cp0)coordinate(kA)--(kA-| w4);
\draw(uu10cp1)coordinate(kA)--(kA-| w5);
\draw(uu10cp2)coordinate(kA)--(kA-| w6);
\draw(uu10cp3)coordinate(kA)--(kA-| w7);
\draw(uu11cp0)coordinate(kA)--(kA-| w0);
\draw(uu11cp1)coordinate(kA)--(kA-| w1);
\draw(uu11cp2)coordinate(kA)--(kA-| w2);
\draw(uu11cp3)coordinate(kA)--(kA-| w3);
\draw[thin](uu10dp3)--++(0,\kpin)--++(-4*\kpsep,0)coordinate(kA)node[ground]{} (uu10dp2)--(uu10dp2|-kA)  (uu10dp1)--(uu10dp1|-kA)   (uu10dp0)--(uu10dp0|-kA);
\draw[thin](uu11dp3)--++(0,\kpin)--++(-4*\kpsep,0)coordinate(kA)node[ground]{} (uu11dp2)--(uu11dp2|-kA)  (uu11dp1)--(uu11dp1|-kA)   (uu11dp0)--(uu11dp0|-kA);
\draw(uu10dp4)node[above]{$\SI{5}{\volt}$};
\draw(uu11dp4)node[above]{$\SI{5}{\volt}$};
\draw[thin](uu10bp3)--(uu10bp4)coordinate(kA)--($(kA-| w7)+(-2*\kw,0)$)coordinate(kB)--(kB|-uu11bp3)--++(0,-2*\kpin) -|coordinate(kD)(uu11bp3)  (kD)--++(-3*\kpin,0)coordinate(lft)node[left]{$\overline{L}_A$}  (uu11bp4)--(uu11bp4|- lft); 
\draw(uu10bp2)--++(0,-\kw)coordinate(kA)--($(kA-| w7)+(-3*\kw,0)$)coordinate(kB)--(kB|-uu11bp2)--(uu11bp2-|lft)node[left]{$CLK$};
\foreach \n in {0,1,2,3}\draw(uu12cp\n)--(uu10ap\n);
\foreach \n in {0,1,2,3}\draw(uu13cp\n)--(uu11ap\n);
\draw[thin](uu12ap3)--++(-0*\kw,0)--++(0,\kul+2*\kpin+2*\kw+0*\kpsep)coordinate(kA)--(kA-| w7);
\draw[thin](uu12ap2)--++(-1*\kw,0)--++(0,\kul+2*\kpin+3*\kw+1*\kpsep)coordinate(kA)--(kA-| w6);
\draw[thin](uu12ap1)--++(-2*\kw,0)--++(0,\kul+2*\kpin+4*\kw+2*\kpsep)coordinate(kA)--(kA-| w5);
\draw[thin](uu12ap0)--++(-3*\kw,0)--++(0,\kul+2*\kpin+5*\kw+3*\kpsep)coordinate(kA)--(kA-| w4);
\draw[thin](uu13ap3)--++(-4*\kw,0)--++(0,\ksepY+\kul+2*\kpin+6*\kw+0*\kpsep)coordinate(kA)--(kA-| w3);
\draw[thin](uu13ap2)--++(-5*\kw,0)--++(0,\ksepY+\kul+2*\kpin+7*\kw+1*\kpsep)coordinate(kA)--(kA-| w2);
\draw[thin](uu13ap1)--++(-6*\kw,0)--++(0,\ksepY+\kul+2*\kpin+8*\kw+2*\kpsep)coordinate(kA)--(kA-| w1);
\draw[thin](uu13ap0)--++(-7*\kw,0)--++(0,\ksepY+\kul+2*\kpin+9*\kw+3*\kpsep)coordinate(kA)--(kA-| w0);
\draw[thin](uu12dp0)--(uu12dp3)--(uu12dp3-|uu12cp3)coordinate(kA)--(kA |-lft)--++(0,-\kw)--++(-\kpin,0)node[left]{$E_A$};
\draw[thin](uu13dp0)--(uu13dp3)--(uu13dp3-|uu13cp3);
\draw[thin](uu12bp0)node[ground]{}  (uu13bp0)node[ground]{};
\draw[thin](uu12bp3)node[below]{$\SI{5}{\volt}$}  (uu13bp3)node[below]{$\SI{5}{\volt}$};
\foreach \n in {0,1,2,3} \draw[thin](uu11ap\n)++(-\n*\kw,0)coordinate(kA)--(kA|-  lft);
\foreach \n in {0,1,2,3} \draw[thin](uu10ap\n)++(-5*\kw-\n*\kw,0)coordinate(kA)--(kA|- lft);
\draw[thin] [decorate,decoration={brace,amplitude=5pt,mirror,raise=5pt}]{}(kA|-lft)--++(8*\kw,0)node[pos=0.5,below,inner sep=0pt,align=center,yshift=-3ex]{\text{\RL{مخارج دفتر}}\\ $A$};
\kSfoStCW[uu14]{0}{-2.25*\ksepY}
\kSfoStCW[uu15]{0}{-3.25*\ksepY}
\draw(uu14-center)node[inner sep=0pt,align=center]{u14\\ 74LS173};
\draw(uu15-center)node[inner sep=0pt,align=center]{u15\\ 74LS173};
\foreach \n in {0,1,2,3}\draw[thin](uu15cp\n)coordinate(kA)--(kA-| w\n);
\foreach \n/\m in {0/4,1/5,2/6,3/7}\draw[thin](uu14cp\n)coordinate(kA)--(kA-| w\m);
\draw[thin](uu14dp4)node[above]{$\SI{5}{\volt}$}   (uu15dp4)node[above]{$\SI{5}{\volt}$};
\draw[thin](uu14dp3)--++(0,\kpin)--++(-4*\kpsep,0)node[ground]{}  (uu14dp2)--++(0,\kpin) (uu14dp1)--++(0,\kpin)
(uu14dp0)--++(0,\kpin);
\draw[thin](uu15dp3)--++(0,\kpin)--++(-4*\kpsep,0)node[ground]{}  (uu15dp2)--++(0,\kpin) (uu15dp1)--++(0,\kpin)
(uu15dp0)--++(0,\kpin);
\draw[thin](uu15bp3)--++(\kpsep+\kul+2*\kpin,0)|-(uu14bp3)--++(-4*\kpin,0)node[left]{$\overline{L}_B$};
\draw[thin](uu14bp2)--++(0,-\kpsep)coordinate(kCLKa)--++(-2*\kpin,0)node[left]{$CLK$};
\draw[thin](uu15bp2)--++(0,-\kw)--++(2*\kpsep+\kul+2*\kpin+\kw,0)|-(kCLKa);
\foreach \n in {0,1,2,3} \draw[thin](uu15ap\n)--++(-\n*\kw,0)--++(0,-2*\kul-\n*\kpsep);
\foreach \n in {0,1,2,3} \draw[thin](uu14ap\n)--++(-\n*\kw-5*\kw,0)--++(0,-2*\kul-\n*\kpsep-\ksepY)coordinate(kA);
\draw[thin][decorate,decoration={brace,amplitude=5pt,raise=5pt,mirror}](kA)--++(8*\kw,0)node[pos=0.5,below,yshift=-3ex,inner sep=0pt,align=center]{\text{\RL{مخارج دفتر}}\\ $B$};
\draw[thin]($(uu10bp0)!0.25!(uu11dp0)$)node[]{$A$ \text{\RL{دفتر}}};
\draw[thin]($(uu14bp0)!0.25!(uu15dp0)$)++(-0.75*\ksepX,0)node[]{$B$ \text{\RL{دفتر}}};
\end{tikzpicture}
\caption{دفتر \عددی{A}  اور جمع و منفی کار}
\label{شکل_کمپیوٹر_دفتر_الف_جمع_منفی_کار}
\end{figure}
%------------------
\begin{figure}
\centering
 \ctikzset{bipoles/resistor/height=0.15}
 \ctikzset{bipoles/resistor/width=0.4}
\begin{tikzpicture}
\pgfmathsetmacro{\kr}{1}
\pgfmathsetmacro{\kLx}{0.15}
\pgfmathsetmacro{\kLy}{0.2}
\pgfmathsetmacro{\kw}{0.20}
\pgfmathsetmacro{\kpin}{0.30}
\pgfmathsetmacro{\kpsep}{0.40}			%pin to pin distance
\pgfmathsetmacro{\kw}{0.20}
\pgfmathsetmacro{\kul}{0.40}			%edge clearance along vertical edges
\pgfmathsetmacro{\kpinA}{0.5}
\pgfmathsetmacro{\kpinS}{0.4}
\pgfmathsetmacro{\ksepX}{5.25}
\pgfmathsetmacro{\ksepY}{3.5}
\def\kLED{coordinate(kSt)++(0.05,0.05)--++(0.1,0.1)coordinate(kT)--++(-0.05,0) (kT)--++(0,-0.05)   (kSt)--++(\kLx,0)--++(-\kLx,-\kLy)coordinate(ledB)--++(-\kLx,\kLy)--++(\kLx,0) (ledB)++(-\kLx,0)--++(2*\kLx,0) (ledB)}
\kSfet[uu16]{0}{0}
\kSfet[uu17]{-\ksepX}{0}
\draw[thin](uu16-center)node[inner sep=0pt,align=center]{u16\\ 74LS83};
\draw[thin](uu17-center)node[inner sep=0pt,align=center]{u17\\ 74LS83};
\draw[thin](uu16dp4)--++(0,3*\kw)--++(\kpin,0)node[xor port,number inputs=2,rotate=-90,scale=0.7,anchor=out](uu18A){};
\draw[thin](uu16dp5)--++(0,2*\kw)--++(\kpin+1*\kpinA,0)--++(0,\kw)node[xor port,number inputs=2,rotate=-90,scale=0.7,anchor=out](uu18B){};
\draw[thin](uu16dp6)--++(0,1*\kw)--++(\kpin+2*\kpinA,0)--++(0,2*\kw)node[xor port,number inputs=2,rotate=-90,scale=0.7,anchor=out](uu18C){};
\draw[thin](uu16dp7)--++(0,0*\kw)--++(\kpin+3*\kpinA,0)--++(0,3*\kw)node[xor port,number inputs=2,rotate=-90,scale=0.7,anchor=out](uu18D){};

\draw[thin](uu17dp4)--++(0,3*\kw)--++(\kpin,0)node[xor port,number inputs=2,rotate=-90,scale=0.7,anchor=out](uu19A){};
\draw[thin](uu17dp5)--++(0,2*\kw)--++(\kpin+1*\kpinA,0)--++(0,\kw)node[xor port,number inputs=2,rotate=-90,scale=0.7,anchor=out](uu19B){};
\draw[thin](uu17dp6)--++(0,1*\kw)--++(\kpin+2*\kpinA,0)--++(0,2*\kw)node[xor port,number inputs=2,rotate=-90,scale=0.7,anchor=out](uu19C){};
\draw[thin](uu17dp7)--++(0,0*\kw)--++(\kpin+3*\kpinA,0)--++(0,3*\kw)node[xor port,number inputs=2,rotate=-90,scale=0.7,anchor=out](uu19D){};
\draw[thin](uu19A.in 1)--(uu18D.in 1)--++(3*\kpin,0)node[right]{$S_U$} +(-1.5*\kpin,0)coordinate(kRht);
\draw[thin](uu16cp1)-|(kRht)   (uu16ap1)--(uu17cp1);
\draw[thin](uu16ap0)node[ground]{}  (uu17ap0)node[ground]{};
\draw[thin](uu16ap2)node[left]{$\SI{5}{\volt}$}   (uu17ap2) node[left]{$\SI{5}{\volt}$};
\draw[thin](uu18A.in 2)--++(0,2*\kpin)coordinate(kSubTop);
\draw[thin](uu18B.in 2)coordinate(kA)--(kA|- kSubTop);
\draw[thin](uu18C.in 2)coordinate(kA)--(kA|- kSubTop);
\draw[thin](uu18D.in 2)coordinate(kA)--(kA|- kSubTop);
\draw[thin](uu19A.in 2)coordinate(kA)--(kA|- kSubTop);
\draw[thin](uu19B.in 2)coordinate(kA)--(kA|- kSubTop);
\draw[thin](uu19C.in 2)coordinate(kA)--(kA|- kSubTop);
\draw[thin](uu19D.in 2)coordinate(kA)--(kA|- kSubTop);
\foreach \n in {0,1,2,3}\draw(uu16dp\n)coordinate(kB)--(kB|- kSubTop);
\foreach \n in {0,1,2,3}\draw(uu17dp\n)coordinate(kB)--(kB|- kSubTop);
\draw[dashed,lgray] (uu18A.out)++(-4ex,-2ex) rectangle ($(uu18D.in 1)+(2ex,2ex)$);
\draw(uu18D.out)node[right,xshift=5ex,align=center,black]{u18\\ 74LS86};
\draw[dashed,lgray] (uu19A.out)++(-4ex,-2ex) rectangle ($(uu19D.in 1)+(2ex,2ex)$);
\draw[thin](uu19A.out)node[left,xshift=-15ex,align=center,black]{u19\\ 74LS86};
\draw[thin,decorate,decoration={brace,amplitude=5pt,raise =5pt}](uu18A.in 2 |-kSubTop)--(uu18D.in 2 |- kSubTop)node[pos=0.5,above,align=center,yshift=10pt]{$B$ \text{\RL{مخارج دفتر}}};
\draw[thin,decorate,decoration={brace,amplitude=5pt,raise =5pt}](uu19A.in 2 |- kSubTop)--(uu19D.in 2 |- kSubTop)node[pos=0.5,above,align=center,yshift=10pt]{$B$ \text{\RL{مخارج دفتر}}};
\draw[decorate,decoration={brace,amplitude=5pt,raise=5pt}] (uu16dp0|-kSubTop)--(uu16dp3|-kSubTop)node[above,align=center,pos=0.5,yshift=10pt]{\text{\RL{مخارج}}\\ $A$ \text{\RL{دفتر}}};
\draw[decorate,decoration={brace,amplitude=5pt,raise=5pt}] (uu17dp0|-kSubTop)--(uu17dp3|-kSubTop)node[above,align=center,pos=0.5,yshift=10pt]{\text{\RL{مخارج}}\\ $A$ \text{\RL{دفتر}}};
\draw(uu16-south-east)node[right,xshift=4ex]{\text{\RL{جمع و منفی کار}}};
\kSfots[uu20]{3*\kpsep}{-\ksepY}
\kSfots[uu21]{-\ksepX+3*\kpsep}{-\ksepY}
\draw[thin](uu20-center)node[align=center]{u20\\ 74LS126};
\draw[thin](uu21-center)node[align=center]{u21\\ 74LS126};
\foreach \n/\m in {0/3,1/4,2/5,3/6}\draw[thin](uu20dp\n)--(uu16bp\m);
\foreach \n/\m in {0/3,1/4,2/5,3/6}\draw[thin](uu21dp\n)--(uu17bp\m);
\draw[thin](uu20cp3)node[right]{$\SI{5}{\volt}$}   (uu21cp3)node[right]{$\SI{5}{\volt}$};
\draw[thin](uu20cp0)node[ground]{}  (uu21cp0)node[ground]{};
\draw[thin](uu20ap0)--(uu20ap3)--++(0,3*\kpin)coordinate(kA)--(kA-| uu17ap0)coordinate(klftEU)node[left]{$E_U$};
\draw[thin](uu21ap0)--(uu21ap0|-klftEU);
\draw[thin](uu20dp3)++(6*\kpin+7*\kw,0)coordinate(kwTop);
\foreach \n in {0,1,2,3,4,5,6,7}\draw[thin](kwTop)++(-\n*\kw,0)coordinate(w\n)--++(0,-25*\kpin);
\draw[thin](w0)node[above]{$w_0$}   (w7)node[above]{$w_7$};
\foreach \n/\m in {0/3,1/2,2/1,3/0} \draw[thin](uu20bp\n)--++(0,-3*\kw+\n*\kw)coordinate(kA)--(kA-|w\m);
\foreach \n/\m in {0/7,1/6,2/5,3/4} \draw[thin](uu21bp\n)--++(0,-7*\kw+\n*\kw)coordinate(kA)--(kA-|w\m);
\kSfoSt[uu22]{0}{-3*\ksepY}
\kSfoSt[uu23]{-\ksepX}{-3*\ksepY}
\draw(uu22-center)node[align=center]{u22\\ 74LS173};
\draw(uu23-center)node[align=center]{u23\\ 74LS173};
\foreach \n/\m in {0/3,1/2,2/1,3/0} \draw[thin](uu22dp\n)--++(0,3*\kw-\n*\kw)coordinate(kA)--(kA-|w\m);
\foreach \n/\m in {0/7,1/6,2/5,3/4} \draw[thin](uu23dp\n)--++(0,7*\kw-\n*\kw)coordinate(kA)--(kA-|w\m);
\foreach \n in {0,1,2,3}\draw[thin](uu22bp\n)--++(0,-\kpin) to [R]++(0,-\kr)\kLED node[ground]{};
\foreach \n in {0,1,2,3}\draw[thin](uu23bp\n)--++(0,-\kpin) to [R]++(0,-\kr) \kLED node[ground]{};
\draw[thin](uu22ap3)--(uu22ap0)node[ground]{};
\draw[thin](uu23ap3)--(uu23ap0)node[ground]{};
\draw[thin](uu22ap4)node[left]{$\SI{5}{\volt}$};
\draw[thin](uu23ap4)node[left]{$\SI{5}{\volt}$};
\draw[thin](uu23cp4)--++(\kpin,0)coordinate(kA)--(kA|-uu23bp3)-|coordinate(kC)(uu22cp4)   (kC)--++(3*\kpin,0)coordinate(krht)node[right,yshift=0.5ex]{$\overline{L}_O$};
\draw[thin](uu23cp3)--++(\kpin,0);
\draw[thin](uu23cp2)--($(uu23cp2|-uu23bp3)+(0,-\kpin)$)coordinate(kD)--($(kD-|kC)+(\kpin,0)$)coordinate(kE)|- (uu22cp2)  (kE)--(kE-|krht)node[right,yshift=-0.5ex]{$CLK$};
\draw[thin](uu22cp1)node[right,xshift=10ex]{\text{\RL{خارجی دفتر}}};
\draw[thin](uu22bp3)++(0,-1.25*\kr)node[right,xshift=10ex]{\text{\RL{ثنائی نمائشی تختی}}};
\draw[thin](uu22bp3)++(0,-0.75*\kr)node[right]{$\SI{1}{\kilo\ohm}$};
\draw[thin](uu23bp3)++(0,-0.75*\kr)node[right]{$\SI{1}{\kilo\ohm}$};
\end{tikzpicture}
\caption{جمع و منفی کار اور خارجی دفتر}
\label{شکل_کمپیوٹر_جمع_منفی_کار}
\end{figure}
%==================================================================
\جزوحصہء{دفتر \عددی{A}}
 مخلوط ادوار \عددی{u10} اور \عددی{u11}  ، جو \عددی{74LS173} ہیں  ،\قول{  دفتر \عددی{A} } دیتے ہیں (شکل \حوالہ{شکل_کمپیوٹر_دفتر_الف_جمع_منفی_کار} دیکھیں)۔ دونوں مخلوط دور کے \عددی{1} اور \عددی{2} پنیے زمین سے جوڑ کر   مخارج دو حال بنایا گیا ہے۔دو حال مخارج جمع و منفی کار کو فراہم کیا گیا ہے۔ \عددی{u12} اور \عددی{u13} مخلوط دور (\عددی{74LS126})     سہ حال    سوئچ  ہیں جو بلند \عددی{E_A} کی صورت میں دفتر \عددی{A} کا مخارج  \عددی{W} گزرگاہ پر ڈالتے ہیں۔

\جزوحصہء{جمع و منفی کار}
\عددی{u18} اور \عددی{u19} مخلوط دور \عددی{74LS86} ہیں۔ یہ بلا شرکت جمع گیٹ بطور قابو  کردہ  متمم کار  کا کردار ادا کرتے ہیں۔ پست \عددی{S_U} کی صورت میں  دفتر \عددی{B} کا مواد بغیر تبدیل ہوئے ان  گیٹ سے گزرتا ہے۔بلند \عددی{S_U} کی صورت میں \عددی{B}  کے مواد کا  تکملہ  \عددی{1} ان گیٹوں سے خارج ہو گا اور ساتھ ہی کمتر تر  رتبی بِٹ کے ساتھ \عددی{1} جمع ہو کر تکملہ \عددی{2} دیگا۔

\عددی{u16} اور \عددی{u17} مخلوط دور \عددی{74LS83} ہیں، جو \عددی{4} بِٹ  مکمل جمع کار ہے۔ دونوں کو جوڑ کر \عددی{8}  بِٹ  \قول{مکمل جمع  و منفی کار }حاصل کیا گیا ہے۔ \عددی{u20} اور \عددی{u21}، جو \عددی{74LS126} ہیں، \عددی{8} بِٹ نتیجہ کو سہ حال بنا کر \عددی{W}  گزرگاہ پر ڈالتے ہیں۔

\جزوحصہء{دفتر \عددی{B} اور خارجی دفتر}
\عددی{u14} اور \عددی{u15}، جو \عددی{74LS173} ہیں، مل کر \قول{ دفتر \عددی{B} } دیتے ہیں۔دونوں کے پنیا \عددی{1} اور \عددی{2} زمین سے جوڑ کر مخارج دو حال بنایا گیا ہے۔دفتر \عددی{A}  کے مواد کے ساتھ  دفتر \عددی{B} کا مواد جمع کیا جاتا ہے یا اس سے دفتر \عددی{B} کا مواد  منفی کیا جاتا ہے۔

\عددی{u22} اور \عددی{u23} ، جو \عددی{74LS173} ہیں، \قول{خارجی دفتر} دیتے ہیں۔ خارجی دفتر ثنائی  نمائشی تختی کو  چلاتا ہے۔ نمائشی تختی پر ہم نتائج دیکھ سکتے ہیں۔

\جزوحصہء{بلا ٹپک صاف  و چل }
شکل \حوالہ{شکل_کمپیوٹر_ساعت_منبع_طاقت} میں \قول{بلا ٹپک صاف و چل دور  } پیش ہے، جس کے دو مخارج ہیں؛دفتر  ہدایت  کے لئے  \عددی{\overline{CLR}} جبکہ برنامہ گنت کار اور چھلا گنت کار کے لئے \عددی{CLR} اشارہ۔ \عددی{\overline{CLR}} ساعت چالو کرنے والے پلٹ \عددی{u29} کو بھی جاتا ہے۔ \عددی{S_5} داب بتام ہے جو د (کھائی گئی  صورت میں ) کھڑا رہتا ہے۔ دبانے سے  اس کا بازو   بیٹھ کر  \قول{صاف}    کو زمین سے ملائے  گا، جس سے بلند \عددی{CLR} اور پست \عددی{\overline{CLR}} حاصل ہو گا۔ بتام کو آزاد چھوڑنے سے اس کا بازو کھڑا ہو کر  \قول{چل} کو زمین سے ملا کر پست \عددی{CLR} اور بلند \عددی{\overline{CLR}} پیدا کرے گا۔ یوں داب  بتام  کو دبا کر دونوں اشارے فعال  ملیں گے۔

سوئچ کا باز و ایک بیٹھک سے دوسری  بیٹھک منتقل کرتے وقت  بازو  ٹپکیاں کھا کر بیٹھتا ہے، جس سے متعدد اشارات پیدا ہو سکتے ہیں۔ ہمیں عموماً ایک مستند  اشارہ درکار ہو گا۔ شکل \حوالہ{شکل_کمپیوٹر_ساعت_منبع_طاقت} میں  \عددی{S_6} کا بازو \قول{صاف}  پر  بٹھانے سے ٹپکیوں کی بدولت  \قول{صاف} پر متعدد \عددی{0}    پیدا ہوں گے، تاہم \عددی{u24} کے دو ضرب متمم گیٹ صرف ایک  پست \عددی{\overline{CLR}} پیدا کرتے ہیں۔ گویا، یہ دور سوئچ کو \اصطلاح{ بلا  ٹپک }\فرہنگ{بلا ٹپک}\حاشیہب{debounced}\فرہنگ{debounce} بناتا ہے۔

دھیان رہے کہ \عددی{u24} کا آدھا حصہ  \قول{بلا ٹپک صاف و چل} اور باقی  \قول{بلا ٹپک قدم با قدم} میں  مستعمل ہے۔  \عددی{u24}  مخلوط  دور  \عددی{74LS00}  کو ظاہر کرتا ہے جو  \عددی{2} داخلی  چو ضرب  متمم  گیٹ پر مشتمل ہے۔

\جزوحصہء{بلا ٹپک قدم با قدم دور}
یہ کمپیوٹر دو  طرز  میں چل سکتا ہے؛ دستی   یا خود کار۔ دستی طرز میں  \عددی{S_6} ایک مرتبہ  دبا کر چھوڑنے سے ساعت کی  ایک مکمل   دھڑکن پیدا ہو گی۔ بیٹھا   \عددی{S_6} بلند \عددی{CLK} دیگا؛ کھڑا \عددی{S_6} پست \عددی{CLK } دیگا۔ دوسرے لفظوں میں ، جیسے جیسے آپ \عددی{S_6}    بٹھا کر  کھڑا کرتے ہیں، شکل \حوالہ{شکل_کمپیوٹر_ساعت_منبع_طاقت}  میں پیش،  \قول{ بلا ٹپک قدم با قدم دور } باری باری ایک ایک \عددی{T} حال پیدا کرتا ہے۔ یوں آپ  کمپیوٹر  کو  مختلف \عددی{T} حال سے گزار کر اس کا تفصیلی معائنہ  کر سکتے ہیں، جو  خرابی کی صورت میں کمپیوٹر ٹھیک کرنے میں مددگار ثابت ہو گا۔

\جزوحصہء{بلا ٹپک دستی و خود کار}
\عددی{S_7}\اصطلاح{ ایک  قطب دو چال  }\فرہنگ{سوئچ!ایک قطب دو چال}\حاشیہب{spdt, single-pole double-throw}\فرہنگ{spdt} سوئچ ہے، جو \قول{  دستی   } بیٹھک پر یا  \قول{خود کار } بیٹھک پر بیٹھا  رہ سکتا ہے۔جب سوئچ  دستی بیٹھک پر  ہو، قدم با قدم بتام فعال ہو گا۔جب سوئچ  خود کار بیٹھک پر  بیٹھا ہو، کمپیوٹر خود کار  کام کریگا ۔ \عددی{u25} کے دو ضرب متمم گیٹ \عددی{S_7} کو بلا ٹپک بناتے ہیں۔\عددی{u25} کے باقی دو  ضرب متمم گیٹ  قدم  با قدم ساعت یا خود کار ساعت  میں سے ایک کو \عددی{CLK} اور \عددی{\overline{CLK}} تک پہنچاتے ہیں۔

\جزوحصہء{ساعت مستحکم کار}
\begin{figure}
\centering
 \ctikzset{bipoles/resistor/height=0.15}
 \ctikzset{bipoles/resistor/width=0.4}
  \ctikzset{bipoles/capacitor/height=0.4}
 \ctikzset{bipoles/capacitor/width=0.15}
  \ctikzset{bipoles/diode/height=0.3}
 \ctikzset{bipoles/diode/width=0.3}
 \begin{subfigure}{1\textwidth}
 \centering
\begin{tikzpicture}
\pgfmathsetmacro{\ksepX}{3}
\pgfmathsetmacro{\ksepY}{1.5}
\pgfmathsetmacro{\kr}{1}
\pgfmathsetmacro{\kd}{1.5}
\pgfmathsetmacro{\ks}{1}
\pgfmathsetmacro{\krad}{0.2}
\pgfmathsetmacro{\knshift}{0.07}
\pgfmathsetmacro{\kpsep}{0.50}
\pgfmathsetmacro{\kpsepr}{0.25}
\pgfmathsetmacro{\kpin}{0.5}
\pgfmathsetmacro{\kpina}{2.5}
\pgfmathsetmacro{\kpinb}{\kpsep+2*\kpsepr}%{4}
\pgfmathsetmacro{\kul}{0.50}
\pgfmathsetmacro{\kmv}{0.15}
\kSfzzA[uu24]{0}{-0*\ksepY}
\kSfzzB[uu24]{0}{-1*\ksepY}
\kSfzzC[uu24]{0}{-2*\ksepY}
\kSfzzD[uu24]{0}{-3*\ksepY}
\kSfzzA[uu25]{0}{-4*\ksepY}
\kSfzzB[uu25]{0}{-5*\ksepY}
\draw[thin](uu24A-north)node[above]{74LS00};
\draw[thin](uu25A-north)node[above]{74LS00};
\draw[thin](uu24A-center)node[]{u24};
\draw[thin](uu24B-center)node[]{u24};
\draw[thin](uu24C-center)node[]{u24};
\draw[thin](uu24D-center)node[]{u24};
\draw[thin](uu25A-center)node[]{u25};
\draw[thin](uu25B-center)node[]{u25};
\draw[thin](uu24p3)--++(0,-\kpin)coordinate(AA)  (uu24p5)--++(0,\kpin)--(AA);
\draw[thin](uu24p6)--++(0,\kpin)coordinate(AA)  (uu24p1)--++(0,-\kpin)--(AA);
\draw[thin](uu24p3)--++(\kpin,0)node[right]{$\overline{CLR}$};
\draw[thin](uu24p6)--++(\kpin,0)node[right]{$CLR$};
\draw[thin]($(uu24p1)!0.5!(uu24p5)$)++(-9*\kpin,0)node[spdt,anchor=in](sw5){};
\draw[thin](uu24p2)-|node[left]{\text{\RL{صاف\!/\!چل}}}(sw5.out 1)node[right]{\text{\RL{چل}}} (uu24p4)-|(sw5.out 2)node[right]{\text{\RL{صاف}}}  (sw5.in)node[above]{$S_5$}node[ground]{};

\draw[thin](uu24p8)--++(0,-\kpin)coordinate(AA) ;
\draw[thin](uu24p13)--++(0,\kpin)--(AA) ;
\draw[thin] (uu24p11)--++(0,\kpin)coordinate(AA) (uu24p9)--++(0,-\kpin)--(AA);
\draw[thin]($(uu24p9)!0.5!(uu24p13)$)++(-9*\kpin,0)node[spdt,anchor=in](sw6){};
\draw[thin](uu24p10)-|node[left,yshift=3ex]{\text{\RL{قدم با قدم}}}(sw6.out 1)node[right]{\text{\RL{پست}}} (uu24p12)-|(sw6.out 2)node[right]{\text{\RL{بلند}}}  (sw6.in)node[above]{$S_6$}node[ground]{};

\draw[thin](uu25p3)--++(0,-\kpin)coordinate(AA)  (uu25p5)--++(0,\kpin)--(AA);
\draw[thin](uu25p6)--++(0,\kpin)coordinate(AA)  (uu25p1)--++(0,-\kpin)--(AA);
\draw[thin]($(uu25p1)!0.5!(uu25p5)$)++(-9*\kpin,0)node[spdt,anchor=in](sw7){};
\draw[thin](uu25p2)-|node[left,yshift=3ex]{\text{\RL{دستی\!/\!خود کار}}}(sw7.out 1)node[right]{\text{\RL{دستی}}} (uu25p4)-|(sw7.out 2)node[right]{\text{\RL{خودکار}}}  (sw7.in)node[above]{$S_7$}node[ground]{};
\kSfozA[uu26]{\ksepX}{-3*\ksepY}
\draw[thin](uu26A-center)node[]{u26};
\draw[thin](uu26A-north)node[above]{74LS10};

\draw[thin](uu26p2)--(uu24p11)  (uu26p13)--++(0,2*\kpin)node[above]{$\overline{HLT}$}  (uu26p1)|-(uu25p3);
\kSfzzC[uu25]{\ksepX}{-5*\ksepY-0.5*\kpsep}
\kSfzzD[uu25]{1.75*\ksepX}{-4*\ksepY}
\draw[thin](uu25C-center)node{u25}  (uu25D-center)node{u25};
\draw[thin](uu25p10)--(uu25p6);
\draw[thin](uu25p13)|-(uu26p12);
\draw[thin](uu25p12)|-(uu25p8);
\draw[thin](uu25p11)++(0,-9*\kpin)coordinate(AA);
\kdSfzfAL[uu27]{AA}
\draw[thin](uu25p11)--(uu27p1)  (uu27p2)++(0,-4*\kpin)coordinate(AA);
\kdSfzfBL[uu27]{AA}
\draw[thin](uu27p2)--(uu27p3)  (uu27p4)node[below]{$CLK$};
\draw[thin](uu27p1)--++(-3*\kpin,0)coordinate(kA)++(0,-5*\kpin)coordinate(AA);
\kdSfzfCL[uu27]{AA}
\draw(uu27A-east)node[right,align=center]{u27\\ 74LS04};
\draw(uu27B-east)node[right]{u27};
\draw(uu27C-west)node[above left]{u27};
\draw(uu27p5)--(kA)  (uu27p6)--(uu27p6|-uu27p4)node[below]{$\overline{CLK}$};
\draw[thin](uu27p6) node[,xshift=-0.75cm,align=center]{\text{\RL{ساعت}}\\  \text{\RL{مستحکم کار}}};
\kFFF[uu28]{-\ksepX}{-7.5*\ksepY}
\draw[thin](uu28-center)node[align=center]{u28\\ NE555};
\draw[thin](uu28p7)--++(-1.5*\kpin,0) to [R,l={$\SI{36}{\kilo\ohm}$}]++(0,\kr)node[left,yshift=5ex]{ساعت}coordinate(kA)-|(uu28p4)  (uu28p8)--(uu28p8|-kA) (kA)--++(-\kpin,0)node[left]{$\SI{5}{\volt}$};
\draw[thin](uu28p7)++(-1.5*\kpin,0) to [R,l_={$\SI{18}{\kilo\ohm}$}]++(0,-\kr)coordinate(kA) to [C,l_={$\SI{0.01}{\micro\farad}$}]++(0,-\kr)node[ground]{}  (kA)-|(uu28p6);
\draw[thin](uu28p2)-|(uu28p6);
\draw[thin](uu28p1)node[ground]{}  (uu28p5) to [C,l={\small{$\SI{0.01}{\micro\farad}$}}]++(0,-\kr)node[ground]{};
\kSfozSA[uu29A]{0}{-7.5*\ksepY+\kpsep}
\draw[thin](uu29A-north)node[above,align=center]{u29\\74LS107};
\draw[thin](uu28p3)--(uu29Ap2)   (uu29Ap3)--(uu29Ap1)--++(0,4*\kpin)--++(-3*\kpin,0)node[left]{$\overline{HLT}$}  (uu29Ap6)|-(uu25p9)   (uu29Apd)--++(0,-2*\kpin)node[below]{$\overline{CLR}$};
\end{tikzpicture}
\caption*{}			%to give space between subfigures
\end{subfigure}
\begin{subfigure}{1\textwidth}
\centering
\begin{tikzpicture}
\pgfmathsetmacro{\ksepX}{3}
\pgfmathsetmacro{\ksepY}{1.5}
\pgfmathsetmacro{\kr}{1}
\pgfmathsetmacro{\kd}{1.5}
\pgfmathsetmacro{\ks}{1}
\pgfmathsetmacro{\krad}{0.2}
\pgfmathsetmacro{\knshift}{0.07}
\pgfmathsetmacro{\kpsep}{0.50}
\pgfmathsetmacro{\kpsepr}{0.25}
\pgfmathsetmacro{\kpin}{0.5}
\pgfmathsetmacro{\kpina}{2.5}
\pgfmathsetmacro{\kpinb}{\kpsep+2*\kpsepr}%{4}
\pgfmathsetmacro{\kul}{0.50}
\pgfmathsetmacro{\kmv}{0.15}
\draw[thick](0,0)coordinate(AA)node[below,xshift=2ex]{$3\!/\!8\, \si{\ampere}$}node[ocirc]{}+(0,\knshift) to [out=45,in=-135]++(0.5*\kr,-\knshift)++(0,\knshift)node[ocirc]{}++(\knshift,0)coordinate(BB) ++(1*\kpin,-0.3*\kpin)coordinate(CC) arc (90:-90:\krad)coordinate(aa) arc (90:-90:\krad) arc (90:-90:\krad)coordinate(bb) arc (90:-90:\krad) coordinate(DD)  ($(CC)+(0.4,0)$) --($(DD)+(0.4,0)$) ($(CC)+(0.6,0)$) --($(DD)+(0.6,0)$) ($(aa)+(1.0,0)$)coordinate(EE) arc (90:270:\krad)coordinate(kMid) arc (90:270:\krad)coordinate(FF);
\draw[thin](AA)++(-\knshift,0)--++(-2*\kpin,0)node[above, xshift=5ex,yshift=3ex]{\text{\RL{منبع طاقت}}}coordinate(klft)  (BB)-|(CC)  (DD)--++(0,-0.3*\kpin)coordinate(kA) --(kA-|klft)coordinate(klftL);
\draw($(klft)!0.5!(klftL)$)node[align=center]{$\SI{220}{\volt}$\\  $\SI{50}{\hertz}$};
\draw(kMid)++(1cm,0)coordinate(dA) to [D,l={$D_1$}]++(45:\kd)coordinate(dB) to [D,l={$D_2$}]++(-45:\kd)coordinate(dC) to [D,invert,l={$D_3$}]++(-135:\kd)coordinate(dD) to [D,invert,l={$D_4$}]++(135:\kd);
\draw[thin](dB)--++(0,0.25)coordinate(kT)-|(EE)  (dD)--++(0,-0.25)coordinate(kB)-|(FF);
\draw[thin](dA)--++(-0.35,0)node[above right]{$-$}coordinate(kA)--($(kA|- dD)+(0,-0.4)$)--++(3.5,0)coordinate(krgt)  (dC)--++(0.35,0)node[above left]{$+$}coordinate(kB)--($(kB|-dB)+(0,0.4)$)coordinate(kC)--(kC-|krgt)coordinate(krgtT);
\draw[thin](krgt)node[circle,inner sep=1.5pt, fill=black]{}node[ground]{} to [C,l_={$\SI{1000}{\micro\farad}$}] ++(0,2.5*\kpin)--(krgtT);
\draw[thick]($(krgt)!0.6!(krgtT)$)++(1cm,0) --++(0,0.5*\ks)--++(1.5*\ks,0)coordinate[pos=0.5](aaa)node[pos=0.5,xshift=1.25ex,yshift=1.5ex]{\scriptsize{$1$}}--++(0,-\ks)coordinate[pos=0.5](bbb)node[pos=0.5,xshift=1.25ex,yshift=1.5ex]{\scriptsize{$2$}}--++(-1.5*\ks,0)coordinate[pos=0.5](ccc)node[pos=0.5,xshift=1.25ex,yshift=-1.5ex]{\scriptsize{$3$}}--++(0,0.5*\ks);
\draw[thin](aaa)|-(krgtT)  (ccc)|-coordinate(kA)(krgt) (kA)--++(1cm,0)coordinate(ddd) to [C,l_={$\SI{220}{\micro\farad}$}] (ddd|-bbb)coordinate(kB) --(bbb)  (kB)--++(\kpin,0)node[right]{$\SI{5}{\volt}$};
\draw($(aaa)!0.5!(ccc)$)node[align=center]{u30\\ LM340-5};
\end{tikzpicture}
\end{subfigure}
\caption{ساعت، منبع طاقت، اور  بلا ٹپک صاف و چل۔}
\label{شکل_کمپیوٹر_ساعت_منبع_طاقت}
\end{figure}

%
\begin{figure}
\centering
\begin{tikzpicture}
\pgfmathsetmacro{\ksepX}{3}
\pgfmathsetmacro{\ksepY}{2}
\pgfmathsetmacro{\ksepYnot}{0.8}
\pgfmathsetmacro{\kr}{1}
\pgfmathsetmacro{\kd}{1.5}
\pgfmathsetmacro{\ks}{1}
\pgfmathsetmacro{\krad}{0.2}
\pgfmathsetmacro{\knshift}{0.07}
\pgfmathsetmacro{\kpsep}{0.50}
\pgfmathsetmacro{\kpsepr}{0.25}
\pgfmathsetmacro{\kpin}{0.5}
\pgfmathsetmacro{\kpina}{2.5}
\pgfmathsetmacro{\kpinb}{\kpsep+2*\kpsepr}%{4}
\pgfmathsetmacro{\kul}{0.50}
\pgfmathsetmacro{\kmv}{0.15}
\pgfmathsetmacro{\kW}{0.2}
\kSfzfALocation[uu35]{0,-0*\ksepY}
\kSfzfBLocation[uu35]{0,-1*\ksepY}
\kSfzfCLocation[uu35]{0,-2*\ksepY}
\kSfzfDLocation[uu35]{0,-3*\ksepY}
\draw(uu35A-north)node[above,align=center]{u35\\ 74LS04};
\kSftzALocation[uu32]{uu35p1}
\kSftzBLocation[uu32]{uu35p3}
\kSftzALocation[uu33]{uu35p5}
\kSftzBLocation[uu33]{uu35p9}
\draw(uu32A-center)node[]{u32};
\draw(uu32B-center)node[]{u32};
\draw(uu33A-center)node[]{u33};
\draw(uu33B-center)node[]{u33};
\draw(uu32A-north)node[above,xshift=1ex]{74LS20};
\draw(uu33A-north)node[above,xshift=1ex]{74LS20};
\foreach \n in {0,1,2,3,4,5,6,7}\draw[thin](uu32p5)++(-\n*\kW-\kpin,1*\kpsep+7*\ksepYnot-\n*\ksepYnot)coordinate(ii\n)--++(0,-25*\kpsep-7*\ksepYnot+\n*\ksepYnot);
\foreach \n in {0,1,2,3,4,5,6,7}\draw(ii\n)--++(-7*\kW+\n*\kW,0)coordinate(i\n);
\kSfzfALocation[uu31]{i0}
\kSfzfBLocation[uu31]{i2}
\kSfzfCLocation[uu31]{i4}
\kSfzfDLocation[uu31]{i6}
\draw(uu31A-north)node[above,align=center]{u31\\ 74LS04};
\draw(uu31B-north)node[above,align=center]{u31};
\draw(uu31C-north)node[above,align=center]{u31};
\draw(uu31D-north)node[above,align=center]{u31};
\draw[thin](uu31p1)--++(-\kpin,0)node[left]{$I_4$};
\draw[thin](uu31p3)--++(-\kpin,0)node[left]{$I_5$};
\draw[thin](uu31p5)--++(-\kpin,0)node[left]{$I_6$};
\draw[thin](uu31p9)--++(-\kpin,0)node[left]{$I_7$};
\draw[thin](i1)-| (uu31p1);
\draw[thin](i3)-| (uu31p3);
\draw[thin](i5)-| (uu31p5);
\draw[thin](i7)-| (uu31p9);
\draw[thin](uu35p2)node[right]{\text{\RL{نقل}}};
\draw[thin](uu35p4)node[right]{\text{\RL{جمع}}};
\draw[thin](uu35p6)node[right]{\text{\RL{منفی}}};
\draw[thin](uu35p8)node[right]{\text{\RL{برآمد}}};
\draw(uu33p8)++(0,-\ksepY)coordinate(kA);
\kSftzALocation[uu34]{kA}
\draw[thin](uu34A-center)node{u34};
\draw[thin](uu34p6)-|(uu34p6-|uu35p2)node[right]{$\overline{\text{\RL{رک}}}$};
\draw[thin](uu32p1)--++(-\kpin-6*\kW,0);
\draw[thin](uu32p2)--++(-\kpin-4*\kW,0);
\draw[thin](uu32p4)--++(-\kpin-2*\kW,0);
\draw[thin](uu32p5)--++(-\kpin-0*\kW,0);

\draw[thin](uu32p9)--++(-\kpin-6*\kW,0);
\draw[thin](uu32p10)--++(-\kpin-4*\kW,0);
\draw[thin](uu32p12)--++(-\kpin-2*\kW,0);
\draw[thin](uu32p13)--++(-\kpin-1*\kW,0);

\draw[thin](uu33p1)--++(-\kpin-6*\kW,0);
\draw[thin](uu33p2)--++(-\kpin-4*\kW,0);
\draw[thin](uu33p4)--++(-\kpin-3*\kW,0);
\draw[thin](uu33p5)--++(-\kpin-0*\kW,0);

\draw[thin](uu33p9)--++(-\kpin-7*\kW,0);
\draw[thin](uu33p10)--++(-\kpin-5*\kW,0);
\draw[thin](uu33p12)--++(-\kpin-3*\kW,0);
\draw[thin](uu33p13)--++(-\kpin-0*\kW,0);

\draw[thin](uu34p1)--++(-\kpin-7*\kW,0);
\draw[thin](uu34p2)--++(-\kpin-5*\kW,0);
\draw[thin](uu34p4)--++(-\kpin-3*\kW,0);
\draw[thin](uu34p5)--++(-\kpin-1*\kW,0);
\end{tikzpicture}
\caption{ہدایات کی رمز کشائی (جدول \حوالہ{جدول_کمپیوٹر_رموز} کے تحت)۔}
\label{شکل_کمپیوٹر_ہدایت_رمز_کشائی}
\end{figure}
%----------------------------------
\begin{figure}
\centering
\begin{tikzpicture}
\pgfmathsetmacro{\ksepX}{3}
\pgfmathsetmacro{\ksepY}{2}
\pgfmathsetmacro{\ksepYnot}{0.8}
\pgfmathsetmacro{\kr}{1}
\pgfmathsetmacro{\kd}{1.5}
\pgfmathsetmacro{\ks}{1}
\pgfmathsetmacro{\krad}{0.2}
\pgfmathsetmacro{\knshift}{0.07}
\pgfmathsetmacro{\kpsep}{0.40}
\pgfmathsetmacro{\kpsepr}{0.25}
\pgfmathsetmacro{\kpin}{0.5}
\pgfmathsetmacro{\kpinb}{\kpsep+2*\kpsepr}%{4}
\pgfmathsetmacro{\kul}{0.40}
\pgfmathsetmacro{\kpina}{0.625*\kul}
\pgfmathsetmacro{\kpinb}{2*\kpina+2*\kpsep+1.25*\kul}
\pgfmathsetmacro{\kpinc}{\kpsep+1.25*\kul}
\pgfmathsetmacro{\kpind}{2*\kpsep+2.5*\kul}
\pgfmathsetmacro{\kpine}{3*\kpsep+3.75*\kul}
\pgfmathsetmacro{\kpinf}{1.5*\kpsep+2*\kul}
\pgfmathsetmacro{\kping}{1.5*\kpsep+3.75*\kul}
\pgfmathsetmacro{\kpinh}{3*\kpsep+4*\kul}
\pgfmathsetmacro{\kmv}{0.15}
\pgfmathsetmacro{\kW}{0.2}
\kSfzfBLocation[uu48]{0,0}
\draw[thin](uu48B-north)node[above,align=center]{u48\\ 74LS04};
\draw[thin](uu48p3)--++(-\kW,0)coordinate(kS);
\kSfzzBLocation[uu46]{kS}
\draw[thin](uu46B-center)node[]{u46};
\draw[thin](uu46B-north)node[above,align=center]{74LS00};
\draw[thin](uu46p5)--++(0,\kpina)coordinate(kA);
\draw[thin](uu46p4)--++(0,-\kpina)coordinate(kB);
\kSfzzALocation[uu43]{kB}
\kSfzzBLocation[uu43]{kA}
\path(uu43p6)++(0,1.25*\kul+\kpsep)coordinate(kA);
\kSfzzCLocation[uu43]{kA}
\draw[thin](uu43A-center)node[]{u43};
\draw[thin](uu43B-center)node[]{u43};
\draw[thin](uu43C-center)node[]{u43};
\draw[thin](uu48p4)coordinate(krgt)node[right]{$\overline{L}_B$};
\draw[thin](uu43p8)--(uu43p8-|krgt)node[right]{$\overline{L}_O$};
\path(uu46p6)++(0,-\kpinb)coordinate(kA);
\kSfzzALocation[uu46]{kA}
\draw[thin](uu46A-center)node[]{u46};
\draw[thin](uu46p3)--(uu46p3-|krgt)node[right]{$E_U$};
\draw[thin](uu46p2)--++(0,\kpina)coordinate(kA);
\draw[thin](uu46p1)--++(0,-\kpina)coordinate(kB);
\kSfzzCLocation[uu42]{kB}
\kSfzzDLocation[uu42]{kA}
\draw[thin](uu42C-center)node[]{u42};
\draw[thin](uu42D-center)node[]{u42};
\path(uu42p8)++(0,-\kpinc)coordinate(kA);
\kSfzzBLocation[uu42]{kA}
\draw[thin](uu42B-center)node[]{u42};
\path(uu42p6)--(uu42p6-|uu48p4)coordinate(kA);
\kSfzfALocation[uu48]{kA}
\draw[thin](uu48A-north)node[above]{u48};
\draw[thin](uu48p2)node[right]{$S_U$}   (uu48p1)--(uu42p6);
\path(uu48p2)++(0,-\kpinc)coordinate(kA);
\kSfzfFLocation[uu47]{kA}
\draw[thin](uu47F-south)node[below]{u47};
\draw[thin](uu47p12)node[right]{$E_A$}  (uu47p13)--(uu47p13-|uu42p6)coordinate(kA);
\kSfzzALocation[uu42]{kA}
\draw[thin](uu42A-center)node[]{u42};
\path(uu47p12)++(0,-\kpind)coordinate(kA);
\kSfzfELocation[uu47]{kA}
\draw[thin](uu47E-north)node[above]{u47}  (uu47p10)node[right]{$\overline{L}_A$};
\draw[thin] (uu47p11)--(uu47p11-|uu46p3)coordinate(kA);
\kSfozBLocation[uu45]{kA}
\draw[thin](uu45B-center)node[]{u45};

\draw[thin](uu45p5)--++(0,1.25*\kul)coordinate(kA);
\draw[thin](uu45p3)--++(0,-1.25*\kul)coordinate(kB);
\kSfzzDLocation[uu41]{kA}
\kSfzzBLocation[uu41]{kB}
\kSfzzCLocation[uu41]{uu45p4}
\draw[thin](uu41B-center)node[]{u41};
\draw[thin](uu41C-center)node[]{u41};
\draw[thin](uu41D-center)node[]{u41};

\path(uu47p10)++(0,-\kpine)coordinate(kA);
\kSfzfDLocation[uu47]{kA}
\draw[thin](uu47D-north)node[above]{u47}  (uu47p8)node[right]{$\overline{E}_I$};
\draw[thin] (uu47p9)--(uu47p9-|uu46p3)coordinate(kA);
\kSfozALocation[uu45]{kA}
\draw[thin](uu45A-center)node[]{u45};

\draw[thin](uu45p13)--++(0,1.25*\kul)coordinate(kA);
\draw[thin](uu45p1)--++(0,-1.25*\kul)coordinate(kB);
\kSfzzALocation[uu41]{kA}
\kSfzzCLocation[uu40]{kB}
\kSfzzDLocation[uu40]{uu45p2}
\draw[thin](uu41A-center)node[]{u41};
\draw[thin](uu40C-center)node[]{u40};
\draw[thin](uu40D-center)node[]{u40};
\path(uu47p8)++(0,-\kpinf)coordinate(kA);
\kSfzfCLocation[uu47]{kA}
\draw[thin](uu47C-north)node[above]{u47}  (uu47p6)node[right]{$\overline{L}_I$};
%%
\path(uu47p6)++(0,-\kping)coordinate(kA);
\kSfzfBLocation[uu47]{kA}
\draw[thin](uu47B-north)node[above]{u47}  (uu47p4)node[right]{$\overline{CE}$};
\draw[thin] (uu47p3)--(uu47p3-|uu46p3) +(\kW,0)coordinate(kA);
\kSftzBLocation[uu44]{kA}
\draw[thin](uu44B-center)node[]{u44};

\draw[thin](uu44p13)--++(0,2.5*\kul)--++(-\kW,0)coordinate(kA);
\draw[thin](uu44p12)--++(-\kW,0)--++(0,1.25*\kul)coordinate(kB);
\draw(uu44p10)--++(-\kW,0)coordinate(kC);
\kSfzzBLocation[uu40]{kA}
\kSfzzALocation[uu40]{kB}
\kSfzzDLocation[uu39]{kC}

\draw[thin](uu40B-center)node[]{u40};
\draw[thin](uu40A-center)node[]{u40};
\draw[thin](uu39D-center)node[]{u39};
%
\path(uu47p4)++(0,-\kpinh)coordinate(kA);
\kSfzfALocation[uu47]{kA}
\draw[thin](uu47A-north)node[above]{u47}  (uu47p2)node[right]{$\overline{L}_M$};
\draw[thin] (uu47p1)--(uu47p1-|uu46p3)+(\kW,0)coordinate(kA);
\kSftzALocation[uu44]{kA}
\draw[thin](uu44A-center)node[]{u44};

\draw[thin](uu44p5)--++(0,2.5*\kul)--++(-\kW,0)coordinate(kA);
\draw[thin](uu44p4)--++(-\kW,0)--++(0,1.25*\kul)coordinate(kB);
\draw(uu44p2)--++(-\kW,0)coordinate(kC);
\kSfzzCLocation[uu39]{kA}
\kSfzzBLocation[uu39]{kB}
\kSfzzALocation[uu39]{kC}

\draw[thin](uu39C-center)node[]{u39};
\draw[thin](uu39B-center)node[]{u39};
\draw[thin](uu39A-center)node[]{u39};
%decoded commands
\foreach \n/\lbl in {0/1,1/2,2/3,3/4} \draw[thin](uu43p10)++(-\kpin-\n*\kW,\kul)coordinate(c\lbl)coordinate(kTop)--($(kTop|-uu39p1)+(0,-\kul-\kW-2*\kul)$)coordinate(kBot);
\draw[thin](c1)--++(\kpin,0)--++(0,2*\kW)node[rotate=90,right]{\text{\RL{برآمد}}};
\draw[thin](c2)--++(0,\kW)--++(0.25*\kpin,0)--++(0,1*\kW)node[rotate=90,right]{\text{\RL{منفی}}};
\draw[thin](c3)--++(0,\kW)--++(-0.25*\kpin,0)--++(0,1*\kW)node[rotate=90,right]{\text{\RL{جمع}}};
\draw[thin](c4)--++(-\kpin,0)--++(0,2*\kW)node[rotate=90,right]{\text{\RL{نقل}}};
% T states
\foreach \n/\lbl in {0/1,1/2,2/3,3/4,4/5,5/6} \draw[thin](uu43p10)++(-5*\kpin-6*\kW-\n*\kW,\kul)coordinate(T\lbl)coordinate(kTop)--($(kTop|-uu39p1)+(0,-\kul-\kW-2*\kul)$)coordinate(kBot);
\draw(T1)node[above]{$T_1$}  (T6) node[above]{$T_6$};

\draw[thin](uu43p10)coordinate(kA)--(kA-|T4);
\draw[thin](uu43p9)coordinate(kA)--(kA-|c1);
\draw[thin](uu43p5)coordinate(kA)--(kA-|T5);
\draw[thin](uu43p4)coordinate(kA)--(kA-|c2);
\draw[thin](uu43p2)coordinate(kA)--(kA-|T5);
\draw[thin](uu43p1)coordinate(kA)--(kA-|c3);

\draw[thin](uu42p13)coordinate(kA)--(kA-|T6);
\draw[thin](uu42p12)coordinate(kA)--(kA-|c2);
\draw[thin](uu42p10)coordinate(kA)--(kA-|T6);
\draw[thin](uu42p9)coordinate(kA)--(kA-|c3);

\draw[thin](uu42p5)coordinate(kA)--(kA-|T6);
\draw[thin](uu42p4)coordinate(kA)--(kA-|c2);
\draw[thin](uu42p2)coordinate(kA)--(kA-|T4);
\draw[thin](uu42p1)coordinate(kA)--(kA-|c1);

\draw[thin](uu41p13)coordinate(kA)--(kA-|T6);
\draw[thin](uu41p12)coordinate(kA)--(kA-|c2);
\draw[thin](uu41p10)coordinate(kA)--(kA-|T6);
\draw[thin](uu41p9)coordinate(kA)--(kA-|c3);

\draw[thin](uu41p5)coordinate(kA)--(kA-|T5);
\draw[thin](uu41p4)coordinate(kA)--(kA-|c4);
\draw[thin](uu41p2)coordinate(kA)--(kA-|T4);
\draw[thin](uu41p1)coordinate(kA)--(kA-|c2);

\draw[thin](uu40p13)coordinate(kA)--(kA-|T4);
\draw[thin](uu40p12)coordinate(kA)--(kA-|c3);
\draw[thin](uu40p10)coordinate(kA)--(kA-|T4);
\draw[thin](uu40p9)coordinate(kA)--(kA-|c4);

\draw[thin](uu40p5)coordinate(kA)--(kA-|T5);
\draw[thin](uu40p4)coordinate(kA)--(kA-|c2);
\draw[thin](uu40p2)coordinate(kA)--(kA-|T5);
\draw[thin](uu40p1)coordinate(kA)--(kA-|c3);

\draw[thin](uu39p13)coordinate(kA)--(kA-|T5);
\draw[thin](uu39p12)coordinate(kA)--(kA-|c4);
\draw[thin](uu39p10)coordinate(kA)--(kA-|T4);
\draw[thin](uu39p9)coordinate(kA)--(kA-|c2);

\draw[thin](uu39p5)coordinate(kA)--(kA-|T4);
\draw[thin](uu39p4)coordinate(kA)--(kA-|c3);
\draw[thin](uu39p2)coordinate(kA)--(kA-|T4);
\draw[thin](uu39p1)coordinate(kA)--(kA-|c4);

\draw[thin](T1|-kBot)++(0,\kW+1*\kul)coordinate(kA)--(kA-|uu47p2)node[right]{$E_P$};
\draw[thin](T2|-kBot)++(0,0.5*\kul)coordinate(kA)--(kA-|uu47p2)node[right]{$C_P$};

\draw[thin](uu47p5)--(uu47p5-|T3);
\draw[thin](uu44p9)|-($(uu39p11)!0.5!(uu39p8)$)coordinate(kA)--(kA-|c4)--++(-1.5*\kW,0)--++(0,0.75*\kW)coordinate(kB);
\kSfzfFLocation[uu35]{kB}
\draw[thin](uu35p13)--(uu35p13-|T3);
\draw[thin](uu35F-south)node[below,yshift=-1ex]{u35};
\draw[thin](uu44p1)--++(0,-0.75*\kul+0.5*\kpsep)coordinate(kA)--(kA-|c4)--++(-1.5*\kW,0)coordinate(kB);
\kSfzfELocation[uu35]{kB}
\draw[thin](uu35p11)--(uu35p11-|T1);
\draw[thin](uu35E-north)node[above,yshift=-0.5ex]{u35};
\end{tikzpicture}
\caption{قابو کار}
\label{شکل_کمپیوٹر_قابو_دور}
\end{figure}





%\باب{سوالات}
\ابتدا{سوال}
درج ذیل اعشاری اعداد کو ثنائی روپ  میں لکھیں۔
(ا)	\عددی{33}	(ب)	\عددی{64}	(پ)	\عددی{128}	(ت)	\عددی{256}
(ٹ)	\عددی{4096}	(ث)	\عددی{0.375}	(ج)	\عددی{5.625}	(چ)	\عددی{13.6875}
\انتہا{سوال}
\ابتدا{سوال}
درج ذیل ثنائی اعداد کو اعشاری روپ میں لکھیں۔
(ا)		\عددی{10}		(ب)		\عددی{101}	
(پ)		\عددی{1101}		(ت)		\عددی{11011}
(ٹ)		\عددی{101101011}	(ث)		\عددی{11001010011}		
\انتہا{سوال}
\ابتدا{سوال}
درج ذیل ثنائی اعداد کو اعشاری روپ  میں لکھیں۔
(ا)		\عددی{10.1}		(ب)		\عددی{101.01}
(پ)		\عددی{0.001101}	(ت)		\عددی{1011.01101}
(ٹ)		\عددی{100.001}	(ث)		\عددی{1111.1111}
\انتہا{سوال}
\ابتدا{سوال}
درج ذیل اعشاری اعداد کو اساس سولہ اور اساس آٹھ میں تبدیل کریں۔
(ا)	\عددی{7}	(ب)	\عددی{23}	(پ)	\عددی{32}	(ت)	\عددی{64}		
(ٹ)	\عددی{1024}	(ث)	\عددی{2048}
\انتہا{سوال}
\ابتدا{سوال}
درج ذیل اساس سولہ اعداد کو اساس آٹھ اور ثنائی   روپ  میں لکھیں۔
(ا)	\عددی{7}	(ب)	\عددی{10}	(پ)	\عددی{1A}	(ت)	\عددی{2B3}	
(ٹ)	\عددی{A.BC}	(ث)	\عددی{0.12}	(ج)	\عددی{F0}	(چ)	\عددی{FFFF}	
\انتہا{سوال}
%=======================
%2.1
\ابتدا{سوال}
درج ذیل ثنائی  مجموعے حاصل کریں۔ان سوالات کو اعشاری  روپ  میں بھی حل کریں۔جوابات کا موازنہ کریں۔
(ا)		\عددی{101+110}		(ب)	\عددی{101+11}	
(پ)		\عددی{1101+1011}		(ت)	\عددی{1001+1101}			
(ٹ)		\عددی{1011+101}		(ث)	\عددی{1111+101}		
\انتہا{سوال}
\ابتدا{سوال}
درج ذیل ثنائی اعداد کے سوالات حل کریں۔ان سوالات کو اعشاری  روپ میں بھی حل کریں۔جوابات کا موازنہ کریں۔
(ا)		\عددی{101-110}		(ب)	\عددی{101-111}	
(پ)		\عددی{1101-1111}		(ت)	\عددی{1001-1101}			
(ٹ)		\عددی{1011-101}		(ث)	\عددی{1111-101}
\انتہا{سوال}
\ابتدا{سوال}
درج ذیل ثنائی اعداد کے سوالات حل کریں۔انہیں سوالات کو اعشاری  روپ  میں بھی حل کریں۔جوابات کا موازنہ کریں۔
(ا)		\عددی{10.1-110}		(ب)	\عددی{10.1-101}	
(پ)		\عددی{1.101-11.11}		(ت)	\عددی{10.01-110.1}			
(ٹ)		\عددی{10.11-101.011}	(ث)	\عددی{11.01-111.1}
\انتہا{سوال}
\ابتدا{سوال}
درج ذیل اعشاری سوالات کو ثنائی روپ میں تبدیل کر کے حل کریں۔
(ا)		\عددی{32+64}		(ب)		\عددی{128-256}	
(پ)		\عددی{94.3-121.2}	(ت)		\عددی{22.24+36.09}		
(ٹ)		\عددی{63-1024}	(ث)		\عددی{1024+2056}				
\انتہا{سوال}
\ابتدا{سوال}
درج ذیل اعشاری اعداد کا تکملہ نو اور تکملہ دس حاصل کریں۔
(ا)		\عددی{6}		(ب)			\عددی{8}	
(پ)		\عددی{19}		(ت)			\عددی{205}	
(ٹ)		\عددی{3160029}	(ث)			\عددی{9807568}	
(ج)		\عددی{0.63}		(چ)			\عددی{39.09}	
(ح)		\عددی{3093.9801}	(خ)			\عددی{23409.65487}	
\انتہا{سوال}
\ابتدا{سوال}
درج ذیل ثنائی اعداد کا  تکملہ ایک اور تکملہ دو حاصل کریں۔
(ا)		\عددی{1011}		(ب)			\عددی{1001}	
(پ)		\عددی{111101}	(ت)			\عددی{10101010}	
(ٹ)		\عددی{11.11}		(ث)			\عددی{1101.0011}	
\انتہا{سوال}
\ابتدا{سوال}
درج ذیل اعشاری سوالات کو تکملہ نو اور تکملہ دس  استعمال کرتے ہوئے  حل کریں۔
(ا)		\عددی{4-9}		(ب)			\عددی{9-16}	
(پ)		\عددی{13-23.9}	(ت)		\عددی{303.93-555.078}	
(ٹ)		\عددی{0.045-0.555}	(ث)		\عددی{909.5301-1000}	
\انتہا{سوال}
\ابتدا{سوال}
درج ذیل ثنائی سوالات کو تکملہ ایک اور تکملہ دو سے حل کریں۔
(ا)		\عددی{10-11}		(ب)		\عددی{1010-1101}	
(پ)		\عددی{10.11-11.10}	(ت)		\عددی{1001.1-1101.01}	
(ٹ)		\عددی{1010-101}	(ث)		\عددی{1101.11-0.11}	
\انتہا{سوال}
\ابتدا{سوال}
درج ذیل اعشاری سوالات کو ثنائی روپ  میں تبدیل کر کے حل کریں-(ا) \عددی{3\times 9}			(ب) \عددی{31\times 23}			
(پ)	\عددی{15\times 3.625}		(ت)	 \عددی{1024\times 16}	
(ٹ)	\عددی{2048\times 2048}	(ث) \عددی{65.75\times 11.625}	
\انتہا{سوال}
%================
%3.1
\ابتدا{سوال}	
درج ذیل بوولین مساوات کا جدول لکھیں۔
(ا)	\عددی{XYZ+\overline{X}Y\overline{Z}}	(ب) \عددی{ABC+A\overline{B}C+\overline{A}\,\overline{B}C}			
(پ)	\عددی{A(B+\overline{C})}		(ت)	\عددی{(A+B)(AB+BC+\overline{C}A)}		
(ٹ)	\عددی{A\overline{B}+\overline{A}B}		(ث)	\عددی{A\overline{B}+B\overline{C}}	
\انتہا{سوال}
\ابتدا{سوال}
%3.2	
تفاعل   \عددی{AB+C\overline{D}} کی تکملی شکل \عددی{\overline{AB+C\overline{D}}=(\overline{A}+\overline{B})(\overline{C}+D)} ہے۔ درج  ذیل کی تکملی شکل معلوم کریں۔
(ا)	  \عددی{X+YZ+XY}	(ب)   \عددی{AB(C\overline{D}+\overline{C}D)}
(پ)		  \عددی{\overline{A}\,\overline{B}+A\overline{B}}	(ت)	  \عددی{X\overline{Y}Z+\overline{X}Y}	
(ٹ)   \عددی{(A+B)(B+C)(C+A)}
\انتہا{سوال}
\ابتدا{سوال}	
%3.3
درج ذیل کے ادوار جمع، ضرب اور نفی گیٹوں کی مدد سے بنائیں۔
(ا)	\عددی{AB\overline{C}+\overline{A}\,\overline{B}C}	(ب)	\عددی{A+B(A+\overline{C})}
(پ)	\عددی{\overline{X}\,\overline{Y}(X+\overline{Y})}	(ت)	\عددی{AB+BC+CA}
(ٹ)	\عددی{ABC+\overline{A}B\overline{C}+AB\overline{C}}
\انتہا{سوال}
\ابتدا{سوال}
%3.4	
ڈی مارگن  کلیات کو بوولین جدول سے اخذ کرنے کے طریقہ سے ثابت کریں۔
\انتہا{سوال}
\ابتدا{سوال}
%3.5	
بوولین جدول سے اخذ کرنے کے طریقہ سے درج ذیل ثابت کریں۔
(ا)	\عددی{X\overline{Y}+XY=X}	(ب)	\عددی{X+\overline{X}Y=X+Y}
\انتہا{سوال}
\ابتدا{سوال}
%3.6	
درج ذیل کو مجموعہ ارکان ضرب کی شکل میں لکھیں۔	
(ا)		\عددی{(A+B)(C+D)}	(ب)	\عددی{(A+B)(\overline{B}+C)(A+\overline{C})}
(پ) \عددی{(A+B)(A+B+C)(C+B)}	(ت)	 \عددی{(A+B+C)(\overline{B}+\overline{C})}	
\انتہا{سوال}
\ابتدا{سوال}
%3.7	
درج ذیل کو ضرب   ارکان جمع کی شکل میں لکھیں۔
(ا)	\عددی{X+\overline{Y}Z+\overline{X}\,\overline{Z}}	(ب)	\عددی{XY+\overline{Z}X}		
(پ)	\عددی{X\overline{Y}(\overline{Y}\,\overline{Z}+YZ)}	(ت) \عددی{(A+B\overline{C})(\overline{A}B+\overline{B}A)}
\انتہا{سوال}
\ابتدا{سوال}
%3.8	
 تفاعل \عددی{Y}  درج ذیل صورتوں میں ایک \عددی{(1)}  کے برابر ہو گا۔اگر \عددی{A=0}، \عددی{B=0}،  اور \عددی{C=1}ہو  یا اگر \عددی{A=1}، \عددی{B=1}، اور \عددی{C=0}ہو اور یا اگر \عددی{A=1}، \عددی{B=1}، اور \عددی{C=1}ہو۔ان صورتوں کے علاوہ اس تفاعل کی قیمت صفر  \عددی{(0)} رہتی ہے۔ان معلومات  کو جدول کی شکل میں لکھ کر تفاعل کی مساوات مجموعہ ارکان  ضرب کے   روپ میں حاصل کریں۔
\انتہا{سوال}
\ابتدا{سوال}
%3.9	
گزشتہ سوال میں دیے گئے تفاعل  \عددی{Y} کو \اصطلاح{ ضرب  و جمع  }\فرہنگ{ضرب و جمع}\حاشیہب{AND-OR}\فرہنگ{AND-OR} دور کی شکل میں بنائیں۔یہی  تفاعل \اصطلاح{ضرب متمم  و ضرب متمم  }\فرہنگ{ضرب متمم و ضرب متمم}\حاشیہب{NAND-NAND}\فرہنگ{NAND-NAND} دور سے حاصل کریں۔
\انتہا{سوال}
\ابتدا{سوال}
%3.10	
تفاعل \عددی{Z}   کی قیمت درج ذیل صورتوں میں صفر   \عددی{(0)} ہے۔اگر \عددی{A=0}، \عددی{B=0}، اور \عددی{C=0}ہو یا اگر \عددی{A=1}، \عددی{B=0}، اور  \عددی{C=0}ہو یا اگر \عددی{A=1}، \عددی{B=1}، اور \عددی{C=0}ہو اور یا اگر \عددی{A=1}، \عددی{B=1}، اور  \عددی{C=1}ہو۔ان صورتوں کے علاوہ اس کی قیمت ایک  \عددی{(1)} رہتی ہے۔ان معلومات کو جدول کی شکل میں لکھ کر \عددی{Z}  کی مساوات ضرب  ارکان  جمع کے روپ  میں حاصل کریں۔
\انتہا{سوال}
\ابتدا{سوال}
%3.11
گزشتہ سوال میں دیے  گئے تفاعل  \عددی{Z} کا  جمع  و ضرب دور  بنائیں۔ اسی تفاعل کا   جمع متمم  و جمع متمم دور  بنائیں۔
\انتہا{سوال}
\ابتدا{سوال}
%3.12
جدول میں  \عددی{A}، \عددی{B}، اور \عددی{C} تین آزاد داخلی متغیرات  جبکہ  \عددی{F0}، \عددی{F1}, اور \عددی{F2} تابع متغیرات ہیں۔
\begin{center}
\begin{otherlanguage}{english}
\begin{tabular}{CCC|CCC}
\toprule
A&B&C&F0&F1&F2\\
\midrule
0&0&0&0&1&1\\
0&0&1&1&0&1\\
0&1&0&1&1&0\\
0&1&1&0&0&0\\
1&0&0&1&1&1\\
1&0&1&0&0&1\\
1&1&0&0&0&0\\
1&1&1&0&1&1\\
\bottomrule
\end{tabular}
\end{otherlanguage}
\end{center}
\begin{enumerate}[a.]
\item
تابع متغیرات کو باری باری مجموعہ ارکان ضرب کے روپ  میں لکھیں۔
\item
 منطقی ضرب گیٹ اور منطقی جمع گیٹ استعمال کرتے ہوئے  تابع متغیرات کے ضرب  و جمع دور بنائیں۔
\item
ضرب  و جمع ادوار سے تابع متغیرات کے ضرب متمم  و 	ضرب  متمم ادوار حاصل کریں۔
\item
 تابع متغیرات کو باری باری ضرب ارکان جمع کے روپ میں لکھیں۔
\item
 منطقی جمع گیٹ اور منطقی ضرب گیٹ استعمال کرتے ہوئے  تابع متغیرات  کے جمع   و ضرب دور بنائیں۔
\item
جمع  و  ضرب ادوار سے  تابع متغیرات کے جمع متمم  و جمع متمم  ادوار حاصل کریں۔
\end{enumerate}
\انتہا{سوال}
\ابتدا{سوال}
%3.13
درج ذیل تفاعل مجموعہ ارکان ضرب کی شکل میں ہیں۔انہیں ضرب ارکان جمع کی شکل میں لکھیں۔
(ا)	     \عددی{Z(A,B)=\sum(0,1)}  (ب)  \عددی{F(A,B,C)=\sum(1,3,7)}		
(پ) \عددی{F(A,B,C)=\sum (0,5,7)} (ت)  \عددی{Y(A,B,C)=\sum (0,7)}
(ٹ) \عددی{Z(A,B,C,D)=\sum (0,2,5,12)}
\انتہا{سوال}
%???KKK
\ابتدا{سوال}
%3.14
	درج ذیل تفاعل ضرب ارکان جمع کی شکل میں ہیں۔انہیں مجموعہ ارکان ضرب کی شکل میں لکھیں۔
(ا)	(ب)			
(پ)	
\انتہا{سوال}
\ابتدا{سوال}
%3.15
	انٹرنیٹ 3 سے درج ذیل معلوماتی صفحات 4 حاصل کریں۔یہ گیٹوں کے مخلوط ادوار5  پاکستان کے ہر شہر میں نہایت سستے داموں دستیاب ہیں۔(ا) 7400 (ب) 4011 (پ) 7408 (ت)4081 (ٹ) 4000 (ث)7432 (ج)7404 (چ)4049 (ح)4070  (مثال:7400 کے معلوماتی صفحات حاصل کرنے کی خاطر انٹرنیٹ کے گوگل6 میںلکھیں)
\انتہا{سوال}
\ابتدا{سوال}
%3.16
	گزشتہ سوال میں حاصل کئے گئے معلوماتی صفحات سے 7400 مخلوط دور میں چار گیٹوں کے مخارج کن پنوں پر دستیاب ہیں۔
\انتہا{سوال}
\ابتدا{سوال}
%3.17
	انٹرنیٹ سے تین مداخل والے ضرب گیٹ اور چار مداخل والے جمع گیٹ کے مخلوط ادوار دریافت کریں۔
\انتہا{سوال}
%====================
\ابتدا{سوال}
4.1	کارناف نقشے میں
\انتہا{سوال}
%=============
\ابتدا{سوال}
5.1	شکل 12.1 میں چار مداخل والا دور دیا گیا ہے۔

(ا)	شکل 12.1 کے اندرونی متغیراتاورکے بوولین مساوات 	حاصل 	کریں۔
(ب)	اسی شکل میں خارجی تابع متغیرہکی بوولین مساوات حاصل کریں۔
(پ)	ایک بوولین جدول بنائیں جس میں چار آزاد متغیرات،، 	اورکے تمام ممکنہ ترتیب درج ہوں۔اس بوولین جدول میں،		اورکے خانے بنائیں اور انہیں پُر کریں۔    
\انتہا{سوال}
\ابتدا{سوال}
5.2	ایک ایسا بوولین جدول بنائیں جس کے تین مداخل اور ایک مخارج ہو۔	اس جدول کو یوں پُر کریں کہ مخارج کی قیمت صرف اُس صورت میں ایک 	 کے برابر ہو جب صرف ایک مداخل کی قیمت صفر 	ہو۔اس جدول کی مدد سے مخارج کا ترکیبی دور تشکیل دیں۔
\انتہا{سوال}
\ابتدا{سوال}
5.3	چار مداخل کا ایک ایسا بوولین جدول بنائیں جس کا مخارج صرف اُس 	صورت بلند  ہو جب مخارج ثنائی عدد کی قیمت نو  سے 	کم ہو۔اس تفاعل کا ترکیبی دور تشکیل دیں۔ 
\انتہا{سوال}
\ابتدا{سوال}
5.4	تین مداخل اور تین مخارج والا ایک ایسا بوولین جدول تشکیل دیں جس 	میں مداخل ثنائی عدد کی قیمت سات  سے کم ہونے کی صورت 	میں مخارج کی قیمت مداخل سے ایک زیادہ ہو جبکہ مداخل کی قیمت 	سات کے برابر ہونے کی صورت میں مخارج کی قیمت صفر 	ہو۔
\انتہا{سوال}
\ابتدا{سوال}
5.5	اقلیتی دور ایسے ترکیبی دور کو کہتے ہیں جس کا مداخل اس صورت بلند	 ہوتا ہے جب اس کے زیادہ تر مداخل پست ہوں۔تین 	مداخل والا اقلیتی دور تشکیل دیں۔
\انتہا{سوال}
\ابتدا{سوال}
5.6	ایک ترکیبی دور تشکیل دیں جو اعشاری ہندسے کا اساس نو خارج 	کرے۔ایسے دور کے چار مداخل اور چار مخارج ہوں گے۔
\انتہا{سوال}
\ابتدا{سوال}
5.7	تین بٹ کے دو اعداد کا موازنہ کرنے والا ایسا ترکیبی دور تشکیل دیں 	جس کا مخارج اس صورت بلند ہو جب دونوں اعداد کی قیمتیں 	برابر ہوں۔
\انتہا{سوال}
\ابتدا{سوال}
5.8	چار بٹ کے دو ثنائی اعداد ضرب کرنے والا ترکیبی دور تشکیل دیں۔
\انتہا{سوال}
\ابتدا{سوال}
5.9	جمع متمم گیٹ استعمال کرتے  شناخت کار تشکیل دیں۔
\انتہا{سوال}
\ابتدا{سوال}
5.10	درج ذیل تین تفاعل کو ایک عدد  شناخت کار کی مدد سے حاصل کریں۔اس دور کو شکل 5.25 کی طرز پر تشکیل دیں۔
 \انتہا{سوال}
\ابتدا{سوال}
\انتہا{سوال}
\ابتدا{سوال}
5.11	درج ذیل تفاعل کو داخلی منتخب کار  کی مدد سے حاصل 	کریں۔
\انتہا{سوال}
\ابتدا{سوال}
5.12	مکمل جمع کار کو دو عدد داخلی منتخب کار کی مدد سے حاصل کریں۔
\انتہا{سوال}
\ابتدا{سوال}
5.13	شکل 12.2 میں اعشاری ہندسے کی نمائش کرنے والی تختی7 دکھائی 	گئی ہے جو  سات قابل روشن حصوں پر مبنی ہے۔ان حصوں میں سے 	کسی ایک یا ایک سے زیادہ حصوں کو  بیک وقت روشن کیا جا سکتا 	ہے۔یوں مختلف	حصے روشن کرنے سے اعشاری ہندسے لکھے جا 	سکتے ہیں۔مثلاً	(ب) اور (پ)  بیک وقت روشن کرنے سے  	لکھا جائے گا۔اسی طرح (ا)، (ب)، (پ)، (ت)، (ٹ) اور (ث) بیک 	وقت روشن کرنے سےلکھا جا سکتا ہے۔
		فرض کریں کہ کسی بھی حصے کو بلند  کرنے سے یہ 	حصہ روشن ہو جاتا ہے۔چار مداخل اور سات مخارج والا  ایسا ترکیبی 	دور تشکیل دیں جو مہیا کردہ اعشاری ہندسے کو اس تختی پر دکھلائے۔	اعشاری ہندسہ کو ثنائی علامتی روپ میں مہیا کیا جائے گا۔
		مخلوط دور یہی کام سر انجام دیتا ہے۔
\انتہا{سوال}
\ابتدا{سوال}
5.14	انٹرنیٹ سے سات حصوں والی نمائشی تختی کے معلوماتی صفحات حاصل کریں۔ایس کرنے کی خاطر گوگل میںلکھیں۔
\انتہا{سوال}
%=================
\ابتدا{سوال}
6.1	ثابت کریں کہ جے-کے پلٹ کے مخارجکی مساوات 	یوں ہے
\انتہا{سوال}
\ابتدا{سوال}      	
6.2	شکل میں ضرب گیٹ کا دورانیہ رد عمل نینو سیکنڈ جبکہ جمع گیٹ کانینو سیکنڈ ہے۔تینوں مداخل بیک وقت  تبدیل کئے جاتے ہیں۔  کتنی دیر بعد مخارجاورمتوازن حالت میں ہوں گے۔ (جواب: ،)
\انتہا{سوال}
\ابتدا{سوال}
6.3	ایک کمپیوٹر کے ساعتی اشارہ سے چلتا ہے۔یہ اشارہ تیس فی صد وقت بلند رہتا ہے جبکہ اس کے دورانیہ اترائی اور دورانیہ چڑھائی دونوں پانچ پانچ فی صد وقت لیتے ہیں۔ساعتی اشارہ کا دوری عرصہ ،دورانیہ چڑھائی اور پست دورانیہ حاصل کریں۔ (جواب: ، ،)
\انتہا{سوال}
\ابتدا{سوال}
           6.4	جمع متمم گیٹوں پر مبنی زیادہ مداخل والے ایس-آر پلٹ کے مداخل کو گراف میں تبدیل ہوتے دکھایا گیا ہے۔اس کا مخارج کھینچے۔
\انتہا{سوال}
\ابتدا{سوال}                      
6.5	آقا-غلام پلٹ کے مداخل گراف کئے گئے ہیں۔اورگراف کریں۔
\انتہا{سوال}
\ابتدا{سوال}
6.6	شکل 6.25 میں سلسلہ وار ثنائی جمع کار دکھایا گیا ہے۔اسے استعمال کرتے ہوئے اور کو جمع کریں۔
\انتہا{سوال}
\ابتدا{سوال}
6.7	،مداخل اورمخارج والا ایک ترتیبی دور جس میں دو ڈی پلٹ، اوراستعمال ہوئے ہیں کے مساوات درج ذیل ہیں۔
\انتہا{سوال}
\ابتدا{سوال}
(ا) اس ترتیبی دور کی شکل بنائیں۔ (ب) ان مساوات سے حالات کا جدول حاصل کریں۔ (ج) حالات کے جدول سے حالات کا خاکہ حاصل کریں۔
\انتہا{سوال}
\ابتدا{سوال}
6.8	ایک مداخلاور دو عدد جے-کے پلٹاورپر مبنی ترتیبی دور کے درج ذیل مساوات ہیں۔
(ا) ان سے حالات کے مساواتاور حاصل کریں۔(ب) ان کے حالات کا خاکہ بنائیں۔
\انتہا{سوال}
\ابتدا{سوال}
6.9	دو عدد ڈی پلٹ،اور، استعمال کرتے ہوئے ایک مداخل،، والا ایسا ترتیبی دور تخلیق دیں جو بلترتیب  ،،اورحالتیں اختیار کر سکے۔اگر مداخل بلند ہو تو یہ اوپر جانب کی طرف بڑھے اور اگر مداخل پست ہو تو یہ نیچے کی جانب بڑھے۔اُوپر جانب تک پہنچنے کے بعد مداخل بلند ہونے کی صورت میں یہ اسی حالت میں رہے۔اسی طرح نیچے جانب حالت پہنچ کر مداخل پست ہونے کی صورت یہ اسی حالت میں رہے۔
\انتہا{سوال}
\ابتدا{سوال}
6.10	پچھلے سوال میں مداخلکا اضافہ کریں۔اگر یہ مداخل بلند ہو تو دور بالکل اسی طرح کام کرے جیسے پہلے کرتا تھا جبکہ اگر یہ مداخل پست ہو تو دور جس حالت میں ہو اسی میں رہے۔
\انتہا{سوال}
\ابتدا{سوال}
6.11	پچھلے سوال کے مداخل کی تعداد میں مزید اضافہ کرتے ہوئے مداخل کا اضافہ کریں۔ بلند کرنے سے دور کوحالت اختیار کر لینا چاہیے جبکہپست ہونے کی صورت میں دور بالکل پہلے کی طرح کام کرے۔
\انتہا{سوال}
%==============
\ابتدا{سوال}
7.1	چار بٹ کے سلسلہ وار دائیں منتقل کھاتے میں ابتدائی ثنائی موادموجود ہے۔اس کھاتے کے مخارج کو اسی کھاتے کو بطور مداخل مہیا کیا جاتا ہے۔سات گھڑی کے کنارے گزرنے کے بعد کھاتے میں کیا عدد ہو گا۔
\انتہا{سوال}
\ابتدا{سوال}
7.2	 گزشتہ سوال میں دائیں منتقل کھاتے کے بجائے بائیں منتقل کھاتا استعمال کرتے جواب معلوم کریں۔
\انتہا{سوال}
\ابتدا{سوال}
7.3	گزشتہ دو سوالات میں ہر کنارہ ساعت پر کھاتے میں ثنائی عدد حاصل کریں۔
\انتہا{سوال}
\ابتدا{سوال}
7.4	آٹھ بٹ کے سلسلہ وار دائیں منتقل کھاتے کے مخارج کو چار بٹ کے سلالہ وار دائیں منتقل کھاتے کو بطور مداخل فراہم کیا جاتا ہے۔آٹھ بٹ کھاتے میں ابتدائی موادپایا جاتا ہے جبکہ اسے بطور مداخل متواترفراہم کیا جاتا ہے۔ساعت کے سات کنارے گزرنے کے بعد ان کھاتوں میں کیا اعداد پائے جائیں گے۔
\انتہا{سوال}
\ابتدا{سوال}
7.5	گزشتہ سوال میں چار بٹ کا سلسلہ وار بائیں منتقل کھاتا استعمال کرتے 	ہوئے جواب حاصل کریں۔
\انتہا{سوال}
\ابتدا{سوال}
7.6	آٹھ بٹ کے دو عدد عالمگیر کھاتے استعمال کرتے ہوئے سولہ بٹ کا عالمگیر کھاتا حاصل کریں۔
\انتہا{سوال}
\ابتدا{سوال}
7.7	شکل 7.7 میں سلسلہ وار ثنائی جمع کار دکھایا گیا ہے۔اگر اس شکل میں کھاتا۔ا اور کھاتا-ب دونوں آٹھ آٹھ بٹ کے ہوں اور ان میں ابتدائی ثنائی مواداورپائے جائیں۔تصور کریں کہ ساعت کے آٹھ کنارے گزرتے ہیں۔ساعت کے ہر کنارہ گزرنے کے بعد کھاتا-ا  میں موجود مواد کیا ہو گا۔
\انتہا{سوال}
\ابتدا{سوال}
7.8	سلسلہ وار ثنائی جمع کار سے سلسلہ وار ثنائی منفی کار حاصل کریں۔ایسا کرنے کی خاطر منفی ہونے والے عدد کے تکملہ کو کھاتا-ب میں متوازی لکھنا بھی دکھائیں۔
\انتہا{سوال}
%===============
\ابتدا{سوال}
8.1	چار بٹ معاصر سیدھا گنت کار کی موجودہ گنتیہے۔ساعت کے کتنے کناروں کے بعد یہدے گا۔
\انتہا{سوال}
\ابتدا{سوال}
8.2	سولہ بٹ معاصر گنت کار کی موجودہ گنتیہے۔یہ ساعت کے کتنے کنارے گزرنے کے بعدپڑھے گا۔(ا) تصور کریں کہ یہ سیدھا گنت کار ہے۔ (ب) تصور کریں کہ یہ الٹ گنت کار ہے۔
\انتہا{سوال}
\ابتدا{سوال}
8.3	چار بٹ ثنائی لہر نما گنت کار کو استعمال کرتے ہوئے اعشاری اعداد کے ثنائی علامتی روپ8 کا گنت کار بنایا جا سکتا ہے۔ایسا کرنے کی خاطر ثنائی گنت کار کوکے گنتی پر پہنچتے ہی زبردستی پست کر دیا جاتا ہے۔ایک عدد ضرب متمم گیٹ کے استعمال سے ایسا کرنا ممکن ہوتا ہے۔زبردستی پست صلاحیت رکھنے والے پلٹ استعمال کرتے ہوئے یہ دور تخلیق دیں۔ 
\انتہا{سوال}
\ابتدا{سوال}
8.4	ڈی پلٹ استعمال کرتے ہوئے چار بٹ معاصر ثنائی گنت کار تشکیل دیں۔ 
\انتہا{سوال}
\ابتدا{سوال}
8.5	جے-کے پلٹ استعمال کرتے ہوئے ایسا معاصر گنت کار تشکیل دیں جو اس ترتیب کو دہرائے۔،،اور۔
\انتہا{سوال}
\ابتدا{سوال}
8.6	ٹی پلٹ استعمال کرتے ہوئے چار بٹ کا ثنائی معاصر گنت کار تشکیل دیں جو صفر سے چودہ تک کی جفت گنتی کے بعد ایک سے پندرہ تک طاق گنتی کرنے کے بعد اسی طرح ان دو ترتیب کو دہراتا رہے۔
\انتہا{سوال}
\ابتدا{سوال}
8.7	شکل 8.11 میں دورانیہ پیدا کار دکھایا گیا ہے۔اگر ساعت کا تعددہو تبکے دورانیہ کے لئے درکار دورانیہ کے تین بٹ کیا ہوں گے۔
\انتہا{سوال}
\ابتدا{سوال}
8.8	کارناف نقشوں کے استعمال سے مساوات 8.3 حاصل کریں۔
\انتہا{سوال}
\ابتدا{سوال}
8.9	مساوات 8.3 کے متبادل مساوات جے-کے پلٹ کی خاطر حاصل کریں۔ 
\انتہا{سوال}
%=============
\ابتدا{سوال}
9.1	درج ذیل مختلف جسامت کے حافظہ میں پتہ بٹوں کی تعداد درج ذیل ہے۔ان حافظہ میں الفاظ ذخیرہ کرنے کے کتنے مقام ہیں۔ (ا) (ب) (ج) (د)
\انتہا{سوال}
\ابتدا{سوال}
9.2	حافظہ  کے جسامت کو عموماًلکھا اور پکارا جاتا ہے جہاںاس حافظہ میں الفاظ کی تعداد اورایک لفظ میں بٹ کی تعداد بتلاتے ہیں۔یوں درج ذیل حافظہ میں پتہ کے لئے درکار پن اور ثنائی مواد کے لئے درکار پن کیا ہوں گے۔ (ا) (ب) (ج) (د)
\انتہا{سوال}
\ابتدا{سوال}
9.3	کسی حافظہ کے پتہ پرمواد لکھا ہوا ہے۔اس تک رسائی کے لئے سولہ پتہ بٹ کیا ہوں گے اور اس سے پڑھے جانے والے آٹھ مواد بٹ کیا ہوں گے۔
\انتہا{سوال}
\ابتدا{سوال}
9.4	چار عددحافظہ اور یک عددشناخت کار کی مدد سےحافظہ حاصل کریں۔
\انتہا{سوال}
\ابتدا{سوال}
9.5	دو عددحافظہ کے استعمال سےحافظہ حاصل کریں۔
\انتہا{سوال}
\ابتدا{سوال}
9.6	چار پتہ بٹ اور آٹھ مواد بٹ والے حافظہ کو استعمال کرتے ہوئے نو کا پھاڑا حاصل کرنا ہے۔حافظہ کو ثنائی علامتی روپ میںتاکا اعشاری عدد بطور پتہ فراہم کیا جائے گا۔حافظہ نے مواد بٹوں پر جواب ثنائی علامتی روپ کی شکل میں پیش کرنا ہے۔مثلاً اگر اسے دوفراہم کیا جائے تو یہ اٹھارہ خارج کرے۔ (ا) حافظہ میں لکھی جانے والے مواد کو جدول کی شکل میں لکھیں۔ (ب) حافظہ میں کتنی جگہ باقی رہ جائے گی۔ 
\انتہا{سوال}
\ابتدا{سوال}
9.7	حافظہ استعمال کرتے دئے گئے چار بٹ ثنائی عدد میںکی تعداد معلوم کرنی ہے۔حافظہ کو ثنائی عدد بطور پتہ مہیا کیا جاتا ہے۔حافظہ نے دئے گئے ثنائی عدد میںکی تعداد بطور مواد خارج کرنی ہے۔مثلاً اگر اسےفراہم کیا جائے تو یہیعنی تین خارج کرے۔
\انتہا{سوال}
\ابتدا{سوال}
9.8	انٹرنیٹ سے درج ذیل حافظہ کے معلوماتی  حاصل کر کے ان کی قسم (یعنی پختہ یا عارضی)، جسامت اور دورانیہ رسائی دریافت کریں۔یہ تمام حافظہ مختلف دورانیہ رسائی کی صلاحیت کے لئے دستیاب ہیں۔ (ا)(ب)(پ)(ت)  (ٹ) (ث)(مثال: انٹرنیٹ سےکی معلومات حاصل کرنے کی خاطر گوگل میںلکھیں)
\انتہا{سوال}

\باب{نقشہ دور}
\begin{figure}
\centering
\begin{tikzpicture}
\pgfmathsetmacro{\kpin}{0.30}
\pgfmathsetmacro{\kpsep}{0.40}			%pin to pin distance
\pgfmathsetmacro{\kul}{0.40}			%edge clearance along vertical edges
\pgfmathsetmacro{\kdimX}{2*\kul+0*\kpsep}
\pgfmathsetmacro{\kdimY}{2*\kul+2*\kpsep}	
\pgfmathsetmacro{\ksepX}{2.00}
\pgfmathsetmacro{\ksepY}{1.25}
\pgfmathsetmacro{\kpinA}{0.5}
\pgfmathsetmacro{\kpinB}{1.5}
\pgfmathsetmacro{\kmv}{0.15}
\pgfmathsetmacro{\kw}{0.20}
\kozSvA[uu1]{0}{0}
\kozSvB[uu2]{\ksepX}{0}
\kozSvA[uu3]{2*\ksepX}{0}
\kozSvB[uu4]{3*\ksepX}{0}
\draw[thin](uu1-north)node[above,yshift=1ex,rectangle,inner sep=0pt,text width=2cm,align=center]{u1\\74LS107};
\draw[thin](uu2-north)node[above,yshift=1ex,rectangle,inner sep=0pt,text width=2cm,align=center]{u1\\74LS107};
\draw[thin](uu3-north)node[above,yshift=1ex,rectangle,inner sep=0pt,text width=2cm,align=center]{u2\\74LS107};
\draw[thin](uu4-north)node[above,yshift=1ex,rectangle,inner sep=0pt,text width=2cm,align=center]{u2\\74LS107};
\kotsA[uu5]{4*\ksepX}{-\ksepY}
\kotsB[uu6]{4*\ksepX}{-2*\ksepY}
\kotsC[uu7]{4*\ksepX}{-3*\ksepY}
\kotsD[uu8]{4*\ksepX}{-4*\ksepY}
\draw[dashed,lgray]($(uu5pin|-uu5u)+(-0.5ex,2ex)$)rectangle ($(uu8pout|-uu8d)+(2ex,-1ex)$);
\draw[thin](uu5u)node[above,yshift=3ex,rectangle,inner sep=0pt,text width=1cm,align=center]{u3\\74LS126};
\draw[thin] (uu1p6)|-coordinate(cntA)(uu2p2)  (uu2p6)|-coordinate(cntB)(uu3p2)  (uu3p6)|-coordinate(cntC)(uu4p2);
\draw[thin](cntA)|-(uu8pin) (cntB)|-(uu7pin)  (cntC)|-(uu6pin)  (uu4p6)|-coordinate(kAddressLoc)(uu5pin);
\draw[thin](uu1p2)--++(-1*\kpin,0)coordinate(kClft)node[left]{$\overline{CLK}$};
\draw[thin](uu1p3)--(uu1p1)--++(0,\kpinB)coordinate(kCPl);
\draw[thin](uu2p3)--(uu2p1)--++(0,\kpinB);
\draw[thin](uu3p3)--(uu3p1)--++(0,\kpinB);
\draw[thin](uu4p3)--(uu4p1)--++(0,\kpinB)coordinate(kCPr)--(kCPl)--(kCPl-|kClft)node[left]{$C_P$};
\draw[thin](4.25*\ksepX,1*\kdimY)node[]{\text{\RL{برنامہ گنت کار}}};
\draw[thin](uu5u)--++(0,0.5ex)--++(3*\kpin,0)coordinate(kEPtop);
\draw[thin](uu6u)--++(0,0.5ex)--++(3*\kpin,0);
\draw[thin](uu7u)--++(0,0.5ex)--++(3*\kpin,0);
\draw[thin](uu8u)--++(0,0.5ex)--++(3*\kpin,0)coordinate(kEPbot);
\draw[thin](kEPbot)--(kEPtop)--(kEPtop-|kClft)node[left,yshift=-0.5ex]{$E_P$};
\draw[thin](kAddressLoc)node[left]{$A_3$};
\draw[thin](uu6pin-|kAddressLoc)node[above left]{$A_2$};
\draw[thin](uu7pin-|kAddressLoc)node[above left]{$A_1$};
\draw[thin](uu8pin-|kAddressLoc)node[above left]{$A_0$};
\draw[thin](uu4pd)--(uu1pd)--(uu1pd-|kClft)node[left,yshift=0.5ex]{$\overline{CLR}$};
\draw[thin](uu5pout)++(7*\kpin,10*\kpin)node[above]{$w_0$}coordinate(kwa)--($(uu8pout)+(7*\kpin,-35*\kpin)$)coordinate(kwbot);
\draw[thin](kwa)++(-\kw,0)coordinate(kwb)--(kwb|-kwbot);
\draw[thin](kwa)++(-2*\kw,0)coordinate(kwc)--(kwc|-kwbot);
\draw[thin](kwa)++(-3*\kw,0)coordinate(kwd)--(kwd|-kwbot);
\draw[thin](kwa)++(-4*\kw,0)coordinate(kwe)--(kwe|-kwbot);
\draw[thin](kwa)++(-5*\kw,0)coordinate(kwf)--(kwf|-kwbot);
\draw[thin](kwa)++(-6*\kw,0)coordinate(kwg)--(kwg|-kwbot);
\draw[thin](kwa)++(-7*\kw,0)node[above]{$w_7$}coordinate(kwh)--(kwh|-kwbot);
\draw[thin](uu8pout)--(uu8pout-|kwd);
\draw[thin](uu7pout)--(uu7pout-|kwc);
\draw[thin](uu6pout)--(uu6pout-|kwb);
\draw[thin](uu5pout)--(uu5pout-|kwa);
\draw[thin](kwd)node[yshift=6ex,rectangle,inner sep=0pt,text width=2cm,align=center]{\RTL{$W$ گزرگاہ}};
\draw[thick] (3*\ksepX,-8*\ksepY)coordinate(kLftBot)rectangle node[rectangle, inner sep=0pt,align=center,text width=1cm]{u4 \\ 74LS173}++(2*\kul+3*\kpsep,2*\kul+4*\kpsep);
\draw[thin] (2*\ksepX,-8*\ksepY+3*\kpsep)node[]{\text{\RL{دفتر پتہ}}};
\draw[thin](kLftBot)++(\kul+0*\kpsep,2*\kul+4*\kpsep)node[above,xshift=1.25ex]{\scriptsize{$14$}}--++(0,\kpin+4*\kw)coordinate(kd)--(kd-|kwd);
\draw[thin](kLftBot)++(\kul+1*\kpsep,2*\kul+4*\kpsep)node[above,xshift=1.25ex]{\scriptsize{$13$}}--++(0,\kpin+3*\kw)coordinate(kc)--(kc-|kwc);
\draw[thin](kLftBot)++(\kul+2*\kpsep,2*\kul+4*\kpsep)node[above,xshift=1.25ex]{\scriptsize{$12$}}--++(0,\kpin+2*\kw)coordinate(kb)--(kb-|kwb);
\draw[thin](kLftBot)++(\kul+3*\kpsep,2*\kul+4*\kpsep)node[above,xshift=1.25ex]{\scriptsize{$11$}}--++(0,\kpin+1*\kw)coordinate(ka)--(ka-|kwa);
\draw[thin](kLftBot)++(2*\kul+3*\kpsep,1*\kul+3*\kpsep)node[above right]{\scriptsize{$10$}}--++(2*\kpin,0)--++(0,\kpsep);
\draw[thin](kLftBot)++(2*\kul+3*\kpsep,1*\kul+4*\kpsep)node[above right]{\scriptsize{$9$}}--++(3*\kpin,0)node[right]{$\overline{L}_M$};
\draw[thin](kLftBot)++(2*\kul+3*\kpsep,1*\kul+3*\kpsep)++(\knshift,0)node[ocirc]{};
\draw[thin](kLftBot)++(2*\kul+3*\kpsep,1*\kul+4*\kpsep)++(\knshift,0)node[ocirc]{};
\draw[thin](kLftBot)++(2*\kul+3*\kpsep,1*\kul+2*\kpsep)node[above right]{\scriptsize{$7$}}--++(3*\kpin,0)node[right]{$CLK$};
\draw[thin](kLftBot)++(2*\kul+3*\kpsep,1*\kul+2*\kpsep)++(0,-\kmv)--++(-\kmv,\kmv)--++(\kmv,\kmv);
\draw[thin](kLftBot)++(0,1*\kul+0*\kpsep)node[above left]{\scriptsize{$15$}}--++(-2*\kpin,0)coordinate(lftMARa);
\draw[thin](kLftBot)++(0,1*\kul+1*\kpsep)node[above left]{\scriptsize{$8$}}--++(-2*\kpin,0);
\draw[thin](kLftBot)++(0,1*\kul+2*\kpsep)node[above left]{\scriptsize{$2$}}+(-\knshift,0)node[ocirc]{}--++(-2*\kpin,0);
\draw[thin](kLftBot)++(0,1*\kul+3*\kpsep)node[above left]{\scriptsize{$1$}}+(-\knshift,0)node[ocirc]{}--++(-2*\kpin,0)coordinate(lftMARb);
\draw[thin](lftMARb)--(lftMARa)node[ground]{};
\draw[thin](kLftBot)++(0,1*\kul+4*\kpsep)node[above left]{\scriptsize{$16$}}--++(-2*\kpin,0)node[left]{$\SI{5}{\volt}$};
\draw[thick] (kLftBot)++(-4*\kpsep,-4*\ksepY)coordinate(klftMUX)rectangle node[rectangle,inner sep=0pt,text width=1cm, align=center]{u5\\ 74LS157}++ (2*\kul+7*\kpsep,2*\kul+4*\kpsep);
\draw[thin](klftMUX)++(\kul+4*\kpsep,2*\kul+4*\kpsep)node[above,xshift=1.25ex]{\scriptsize{$13$}}--($(kLftBot)+(\kul+0*\kpsep,0)$)node[below,xshift=1.25ex]{\scriptsize{$3$}};
\draw[thin](klftMUX)++(\kul+5*\kpsep,2*\kul+4*\kpsep)node[above ,xshift=1.25ex]{\scriptsize{$10$}}--($(kLftBot)+(\kul+1*\kpsep,0)$)node[below,xshift=1.25ex]{\scriptsize{$4$}};
\draw[thin](klftMUX)++(\kul+6*\kpsep,2*\kul+4*\kpsep)node[above ,xshift=1.25ex]{\scriptsize{$6$}}--($(kLftBot)+(\kul+2*\kpsep,0)$)node[below,xshift=1.25ex]{\scriptsize{$5$}};
\draw[thin](klftMUX)++(\kul+7*\kpsep,2*\kul+4*\kpsep)node[above ,xshift=1.25ex]{\scriptsize{$3$}}--($(kLftBot)+(\kul+3*\kpsep,0)$)node[below,xshift=1.25ex]{\scriptsize{$6$}};
\draw[thin](klftMUX)++(\kul+0*\kpsep,2*\kul+4*\kpsep)node[above ,xshift=1.25ex]{\scriptsize{$14$}}--++(0,\kpin)--++(-5*\kpin,0) to [nos,mirror,invert]node[below]{$s1$}++(-2*\kpin,0)--++(-2*\kpin,0)coordinate(wrAddBot)node[left]{$pA_3$};
\draw[thin](klftMUX)++(\kul+1*\kpsep,2*\kul+4*\kpsep)node[above ,xshift=1.25ex]{\scriptsize{$11$}}--++(0,2.5*\kpin)--++(-5*\kpin-1*\kpsep,0) to [nos,mirror,invert]++(-2*\kpin,0)--++(-2*\kpin,0)node[left]{$pA_2$};
\draw[thin](klftMUX)++(\kul+2*\kpsep,2*\kul+4*\kpsep)node[above ,xshift=1.25ex]{\scriptsize{$5$}}--++(0,4*\kpin)--++(-5*\kpin-2*\kpsep,0) to [nos,mirror,invert]++(-2*\kpin,0)--++(-2*\kpin,0)node[left]{$pA_1$};
\draw[thin](klftMUX)++(\kul+3*\kpsep,2*\kul+4*\kpsep)node[above ,xshift=1.25ex]{\scriptsize{$2$}}--++(0,5.5*\kpin)--++(-5*\kpin-3*\kpsep,0) to [nos,mirror,invert]++(-2*\kpin,0)--++(-2*\kpin,0)coordinate(wrAddTop)node[left]{$pA_0$};
\draw[thin](wrAddTop)--(wrAddBot)node[ground]{};
\draw[thin](klftMUX)++(0,\kul+0*\kpsep)node[left,yshift=1.25ex]{\scriptsize{$15$}}+(-\knshift,0)node[ocirc]{}--++(-2*\kpin,0);
\draw[thin](klftMUX)++(0,\kul+1*\kpsep)node[left,yshift=1.25ex]{\scriptsize{$8$}}--++(-2*\kpin,0)--++(0,-\kpsep)node[ground]{};
\draw[thin](klftMUX)++(0,\kul+2*\kpsep)node[left,yshift=1.25ex]{\scriptsize{$1$}}--++(-7*\kpin,0) node[spdt,,rotate=0,anchor=out 1](sw){};
\draw[thin](sw.in)node[above]{$s2$}node[ground]{};
\draw[thin](sw.out 1)node[above]{\text{\RL{برنامہ نویسی}}};
\draw[thin](sw.out 2)node[below]{\text{\RL{دوڑ}}};
\draw[thin](klftMUX)++(\kul+4*\kpsep,0)node[left,yshift=-1.25ex]{\scriptsize{$12$}}--++(0,-2*\kpin)node[below]{$a_3$};
\draw[thin](klftMUX)++(\kul+5*\kpsep,0)node[left,yshift=-1.25ex]{\scriptsize{$9$}}--++(0,-2*\kpin)node[below]{$a_2$};
\draw[thin](klftMUX)++(\kul+6*\kpsep,0)node[left,yshift=-1.25ex]{\scriptsize{$7$}}--++(0,-2*\kpin)node[below]{$a_1$};
\draw[thin](klftMUX)++(\kul+7*\kpsep,0)node[left,yshift=-1.25ex]{\scriptsize{$4$}}--++(0,-2*\kpin)node[below]{$a_0$};
\draw[thin](klftMUX)++(2*\kul+7*\kpsep,2*\kul+2*\kpsep)node[right,xshift=2ex,rectangle,inner sep=0pt,text width=2cm,align=center]{$2\times 1$\\ \text{\RL{داخلی منتخب کار}}};
\end{tikzpicture}
\caption{برنامہ گنت کار}
\label{شکل_کمپیوٹر_برنامہ_گنتکار}
\end{figure}
%===================================================================
\begin{figure}
\centering
\begin{tikzpicture}
\pgfmathsetmacro{\kw}{0.20}
\pgfmathsetmacro{\kpin}{0.50}
\pgfmathsetmacro{\kpsep}{0.40}			%pin to pin distance
\pgfmathsetmacro{\kpinA}{0.5}
\pgfmathsetmacro{\kpinS}{0.4}
\pgfmathsetmacro{\ksepX}{5}
\pgfmathsetmacro{\ksepY}{9}
\koen[uu6]{0}{0}
\draw[thin](uu6-center)node[inner sep=0pt,text width=1cm, align=center,]{u6\\ 74189};
\foreach \n/\lbl in {1/4,2/5,3/6,4/7}\draw[thin] (uu6ap\n) to [nos,mirror,invert]++(-\kpinS,0)--++(-0.5*\kpin,0)node[left]{$D_{\lbl}$};
\draw[thin](uu6ap4)++(-\kpinS-0.5*\kpin,0)coordinate(kA)--(kA|-uu6ap1)coordinate(klft)node[ground]{};
\draw(uu6ap4)node[above left,yshift=2ex]{$S_3$};
\koen[uu7]{\ksepX}{0}
\draw[thin](uu7-center)node[inner sep=0pt,text width=1cm, align=center,]{u7\\ 74189};
\foreach \n/\lbl in {1/0,2/1,3/2,4/3}\draw[thin] (uu7ap\n) to [nos,mirror,invert]++(-\kpinS,0)--++(-0.5*\kpin,0)node[left]{$D_{\lbl}$};
\draw[thin](uu7ap4)++(-\kpinS-0.5*\kpin,0)coordinate(kA)--(kA|-uu7ap1)node[ground]{};
\draw(uu7ap4)node[above left,yshift=2ex]{$S_3$};
\foreach \n/\lbl in {0/3,1/2,2/1,3/0}\draw[thin](uu7dp\n)--++(0,2.5*\kpinA)node[above]{$a_{\lbl}$};
\draw[thin](uu6dp0)--++(0,\kpin+3*\kw)coordinate(kB)--(kB-|uu7dp0);
\draw[thin](uu6dp1)--++(0,\kpin+2*\kw)coordinate(kB)--(kB-|uu7dp1);
\draw[thin](uu6dp2)--++(0,\kpin+1*\kw)coordinate(kB)--(kB-|uu7dp2);
\draw[thin](uu6dp3)--++(0,\kpin+0*\kw)coordinate(kB)--(kB-|uu7dp3);
\draw[thin](uu6cp3)node[ground]{}  (uu7cp3)node[ground]{};
\draw[thin](uu6cp4)node[right]{$\SI{5}{\volt}$}  (uu7cp4)node[right]{$\SI{5}{\volt}$};
\draw[thin](uu7ap0)--++(0,-2*\kpin)coordinate(kC)--(kC-|uu6ap0)coordinate(kswlft)node[spdt,anchor=out 1](sw){};
\draw[thin](sw.in)node[above]{$S_4$} node[ground]{} (sw.out 2)node[below,xshift=-2ex]{\text{\RL{پڑھ}}} (sw.out 1)node[above,xshift=-2ex]{\text{\RL{لکھ}}};
\draw[thin](uu6ap0)-|(kswlft);
\draw[thin](uu7cp0)--++(0,-2*\kpin-\kw)coordinate(kD)--($(kD-|uu6ap0)+(2*\kw,0)$)--++(0,-4*\kpin)node[spdt,anchor=in,xscale=-1](sw){};
\draw[thin](sw.out 2)node[left,xshift=-2ex]{$S_2$}node[right,xshift=2ex,yshift=-3ex]{\text{\RL{برنامہ لکھائی}}}node[ground]{};
\draw[thin](sw.out 1)node[above,xshift=2ex]{\text{\RL{دوڑ}}}--++(-\kpin,0)node[left]{$\overline{CE}$};
\draw[thin](uu6cp0)--(uu6cp0|-kD);
\foreach \n in {0,1,2,3,4,5,6,7}\draw[thin](uu7dp3)++(0,2.5*\kpinA)++(3.5*\kpin+7*\kw-\n*\kw,0)coordinate(w\n)--++(0,-1.75*\ksepY);\
\draw[thin](uu7dp3)++(0,2.5*\kpinA)++(3.5*\kpin+0*\kw,0)node[above]{$w_7$}++(7*\kw,0)node[above]{$w_0$};
\draw[thin](uu7bp3)--++(0,-2*\kpin-0*\kw)coordinate(kD)--(kD-| w0);
\draw[thin](uu7bp2)--++(0,-2*\kpin-1*\kw)coordinate(kD)--(kD-| w1);
\draw[thin](uu7bp1)--++(0,-2*\kpin-2*\kw)coordinate(kD)--(kD-| w2);
\draw[thin](uu7bp0)--++(0,-2*\kpin-3*\kw)coordinate(kD)--(kD-| w3);
\draw[thin](uu6bp3)--++(0,-2*\kpin-4*\kw)coordinate(kD)--(kD-| w4);
\draw[thin](uu6bp2)--++(0,-2*\kpin-5*\kw)coordinate(kD)--(kD-| w5);
\draw[thin](uu6bp1)--++(0,-2*\kpin-6*\kw)coordinate(kD)--(kD-| w6);
\draw[thin](uu6bp0)--++(0,-2*\kpin-7*\kw)coordinate(kD)--(kD-| w7);
\koSvt[uu8]{0}{-\ksepY}
\draw[thin](uu8-center)node[inner sep=0pt,text width=1cm,align=center]{u8\\ 74LS173};
\koSvt[uu9]{\ksepX}{-\ksepY}
\draw[thin](uu9-center)node[inner sep=0pt,text width=1cm,align=center]{u9\\ 74LS173};
\draw[thin](uu9dp3)--++(0,\kpin+0*\kw)coordinate(kD)--(kD-|w0);
\draw[thin](uu9dp2)--++(0,\kpin+1*\kw)coordinate(kD)--(kD-|w1);
\draw[thin](uu9dp1)--++(0,\kpin+2*\kw)coordinate(kD)--(kD-|w2);
\draw[thin](uu9dp0)--++(0,\kpin+3*\kw)coordinate(kD)--(kD-|w3);
\draw[thin](uu8dp3)--++(0,\kpin+4*\kw)coordinate(kD)--(kD-|w4);
\draw[thin](uu8dp2)--++(0,\kpin+5*\kw)coordinate(kD)--(kD-|w5);
\draw[thin](uu8dp1)--++(0,\kpin+6*\kw)coordinate(kD)--(kD-|w6);
\draw[thin](uu8dp0)--++(0,\kpin+7*\kw)coordinate(kD)--(kD-|w7);
\draw[thin](uu8ap2)--++(-\kpin,0)--++(0,-2*\kpsep)node[ground]{}  (uu8ap1)--++(-\kpin,0)  (uu8ap0)--++(-\kpin,0); 
\draw[thin](uu8ap3)--++(-\kpin,0)coordinate(kklft)node[left]{$CLR$}  (uu8ap4)--++(-\kpin,0)node[left]{$\SI{5}{\volt}$};
\draw[thin] (uu9ap4)--++(-\kpin,0)node[left]{$\SI{5}{\volt}$};
\draw[thin](uu9cp2)--++(0,-6*\kpin)coordinate(kD)--(kD-|kklft)node[left]{$CLK$};
\draw[thin](uu8cp2)--(uu8cp2|-kD);
\draw[thin](uu9bp3)--++(0,-0*\kw)coordinate(kD)--(kD-|w0);
\draw[thin](uu9bp2)--++(0,-1*\kw)coordinate(kD)--(kD-|w1);
\draw[thin](uu9bp1)--++(0,-2*\kw)coordinate(kD)--(kD-|w2);
\draw[thin](uu9bp0)--++(0,-3*\kw)coordinate(kD)--(kD-|w3);
\draw[thin](uu9cp4)--++(\kw,0)--++(0,-6*\kpin-3*\kpsep)coordinate(kD)--(kD-|kklft)node[left]{$\overline{L}_I$};
\draw[thin](uu9cp3)--++(\kw,0);
\draw[thin](uu8cp4)--++(\kw,0)coordinate(kE)--(kE|-kD);
\draw[thin](uu8cp3)--++(\kw,0);
\draw[thin](uu9ap2)--++(-\kpin,0)--++(0,-6*\kpin-2*\kpsep)coordinate(kD)--(kD-|kklft)node[left]{$\overline{E}_I$};
\draw[thin](uu9ap1)--++(-\kpin,0);
\draw[thin](uu9ap3)--++(-2.5*\kpin,0)node[ground]{};
\draw[thin](uu9ap0)node[ground]{};
\draw[thin](uu8bp3)--++(0,-\kpin)node[below]{$I_4$};
\draw[thin](uu8bp2)--++(0,-\kpin)node[below]{$I_5$};
\draw[thin](uu8bp1)--++(0,-\kpin)node[below]{$I_6$};
\draw[thin](uu8bp0)--++(0,-\kpin)node[below]{$I_7$};
\draw(uu8ap4)node[shift={(-0.25cm,1.5cm)}]{\text{\RL{دفتر ہدایت}}};
\draw(uu6ap4)node[shift={(-0.25cm,2cm)},inner sep=0pt,align=center]{$16\times 8$\\ \text{\RL{عارضی حافظہ}}};
\end{tikzpicture}
\caption{حافظہ اور دفتر ہدایت}
\label{شکل_کمپیوٹر_حافظہ_دفتر_ہدایت}
\end{figure}
%=====================================================================
\begin{figure}
\centering
\begin{tikzpicture}
\pgfmathsetmacro{\kw}{0.20}
\pgfmathsetmacro{\kpin}{0.30}
\pgfmathsetmacro{\kpsep}{0.40}			%pin to pin distance
\pgfmathsetmacro{\kw}{0.20}
\pgfmathsetmacro{\kul}{0.40}			%edge clearance along vertical edges
\pgfmathsetmacro{\kpinA}{0.5}
\pgfmathsetmacro{\kpinS}{0.4}
\pgfmathsetmacro{\ksepX}{5}
\pgfmathsetmacro{\ksepY}{4}
\koSvtCW[uu10]{0}{0}
\koSvtCW[uu11]{0}{-\ksepY}
\kotsCW[uu12]{0-\ksepX}{0}
\kotsCW[uu13]{0-\ksepX}{-\ksepY}
\draw(uu10-center)node[inner sep=0pt,align=center]{u10\\ 74LS173};
\draw(uu11-center)node[inner sep=0pt,align=center]{u11\\ 74LS173};
\draw(uu12-center)node[inner sep=0pt,align=center]{u12\\ 74LS126};
\draw(uu13-center)node[inner sep=0pt,align=center]{u13\\ 74LS126};
\draw[thin](uu10dp3)++(2*\kul+4*\kpin+7*\kw,2.5cm)coordinate(kWtop)coordinate(w0)--++(0,-60*\kpin)coordinate(kWbot);
\foreach \n in {1,2,3,4,5,6,7} \draw[thin](kWtop)++(-\n*\kw,0)coordinate(w\n)coordinate(kA)--(kA|-kWbot);
\draw[thin](kWtop)node[above]{$w_0$} +(-7*\kw,0)node[above]{$w_7$};
\draw(uu10cp0)coordinate(kA)--(kA-| w4);
\draw(uu10cp1)coordinate(kA)--(kA-| w5);
\draw(uu10cp2)coordinate(kA)--(kA-| w6);
\draw(uu10cp3)coordinate(kA)--(kA-| w7);
\draw(uu11cp0)coordinate(kA)--(kA-| w0);
\draw(uu11cp1)coordinate(kA)--(kA-| w1);
\draw(uu11cp2)coordinate(kA)--(kA-| w2);
\draw(uu11cp3)coordinate(kA)--(kA-| w3);
\draw[thin](uu10dp3)--++(0,\kpin)--++(-4*\kpsep,0)coordinate(kA)node[ground]{} (uu10dp2)--(uu10dp2|-kA)  (uu10dp1)--(uu10dp1|-kA)   (uu10dp0)--(uu10dp0|-kA);
\draw[thin](uu11dp3)--++(0,\kpin)--++(-4*\kpsep,0)coordinate(kA)node[ground]{} (uu11dp2)--(uu11dp2|-kA)  (uu11dp1)--(uu11dp1|-kA)   (uu11dp0)--(uu11dp0|-kA);
\draw(uu10dp4)node[above]{$\SI{5}{\volt}$};
\draw(uu11dp4)node[above]{$\SI{5}{\volt}$};
\draw[thin](uu10bp3)--(uu10bp4)coordinate(kA)--($(kA-| w7)+(-2*\kw,0)$)coordinate(kB)--(kB|-uu11bp3)--++(0,-2*\kpin) -|coordinate(kD)(uu11bp3)  (kD)--++(-3*\kpin,0)coordinate(lft)node[left]{$\overline{L}_A$}  (uu11bp4)--(uu11bp4|- lft); 
\draw(uu10bp2)--++(0,-\kw)coordinate(kA)--($(kA-| w7)+(-3*\kw,0)$)coordinate(kB)--(kB|-uu11bp2)--(uu11bp2-|lft)node[left]{$CLK$};
\foreach \n in {0,1,2,3}\draw(uu12cp\n)--(uu10ap\n);
\foreach \n in {0,1,2,3}\draw(uu13cp\n)--(uu11ap\n);
\draw[thin](uu12ap3)--++(-0*\kw,0)--++(0,\kul+2*\kpin+2*\kw+0*\kpsep)coordinate(kA)--(kA-| w7);
\draw[thin](uu12ap2)--++(-1*\kw,0)--++(0,\kul+2*\kpin+3*\kw+1*\kpsep)coordinate(kA)--(kA-| w6);
\draw[thin](uu12ap1)--++(-2*\kw,0)--++(0,\kul+2*\kpin+4*\kw+2*\kpsep)coordinate(kA)--(kA-| w5);
\draw[thin](uu12ap0)--++(-3*\kw,0)--++(0,\kul+2*\kpin+5*\kw+3*\kpsep)coordinate(kA)--(kA-| w4);
\draw[thin](uu13ap3)--++(-4*\kw,0)--++(0,\ksepY+\kul+2*\kpin+6*\kw+0*\kpsep)coordinate(kA)--(kA-| w3);
\draw[thin](uu13ap2)--++(-5*\kw,0)--++(0,\ksepY+\kul+2*\kpin+7*\kw+1*\kpsep)coordinate(kA)--(kA-| w2);
\draw[thin](uu13ap1)--++(-6*\kw,0)--++(0,\ksepY+\kul+2*\kpin+8*\kw+2*\kpsep)coordinate(kA)--(kA-| w1);
\draw[thin](uu13ap0)--++(-7*\kw,0)--++(0,\ksepY+\kul+2*\kpin+9*\kw+3*\kpsep)coordinate(kA)--(kA-| w0);
\draw[thin](uu12dp0)--(uu12dp3)--(uu12dp3-|uu12cp3)coordinate(kA)--(kA |-lft)--++(0,-\kw)--++(-\kpin,0)node[left]{$E_A$};
\draw[thin](uu13dp0)--(uu13dp3)--(uu13dp3-|uu13cp3);
\draw[thin](uu12bp0)node[ground]{}  (uu13bp0)node[ground]{};
\draw[thin](uu12bp3)node[below]{$\SI{5}{\volt}$}  (uu13bp3)node[below]{$\SI{5}{\volt}$};
\foreach \n in {0,1,2,3} \draw[thin](uu11ap\n)++(-\n*\kw,0)coordinate(kA)--(kA|-  lft);
\foreach \n in {0,1,2,3} \draw[thin](uu10ap\n)++(-5*\kw-\n*\kw,0)coordinate(kA)--(kA|- lft);
\draw[thin] [decorate,decoration={brace,amplitude=5pt,mirror,raise=5pt}]{}(kA|-lft)--++(8*\kw,0)node[pos=0.5,below,inner sep=0pt,align=center,yshift=-3ex]{\text{\RL{مخارج دفتر}}\\ $A$};
\koSvtCW[uu14]{0}{-2.25*\ksepY}
\koSvtCW[uu15]{0}{-3.25*\ksepY}
\draw(uu14-center)node[inner sep=0pt,align=center]{u14\\ 74LS173};
\draw(uu15-center)node[inner sep=0pt,align=center]{u15\\ 74LS173};
\foreach \n in {0,1,2,3}\draw[thin](uu15cp\n)coordinate(kA)--(kA-| w\n);
\foreach \n/\m in {0/4,1/5,2/6,3/7}\draw[thin](uu14cp\n)coordinate(kA)--(kA-| w\m);
\draw[thin](uu14dp4)node[above]{$\SI{5}{\volt}$}   (uu15dp4)node[above]{$\SI{5}{\volt}$};
\draw[thin](uu14dp3)--++(0,\kpin)--++(-4*\kpsep,0)node[ground]{}  (uu14dp2)--++(0,\kpin) (uu14dp1)--++(0,\kpin)
(uu14dp0)--++(0,\kpin);
\draw[thin](uu15dp3)--++(0,\kpin)--++(-4*\kpsep,0)node[ground]{}  (uu15dp2)--++(0,\kpin) (uu15dp1)--++(0,\kpin)
(uu15dp0)--++(0,\kpin);
\draw[thin](uu15bp3)--++(\kpsep+\kul+2*\kpin,0)|-(uu14bp3)--++(-4*\kpin,0)node[left]{$\overline{L}_B$};
\draw[thin](uu14bp2)--++(0,-\kpsep)coordinate(kCLKa)--++(-2*\kpin,0)node[left]{$CLK$};
\draw[thin](uu15bp2)--++(0,-\kw)--++(2*\kpsep+\kul+2*\kpin+\kw,0)|-(kCLKa);
\foreach \n in {0,1,2,3} \draw[thin](uu15ap\n)--++(-\n*\kw,0)--++(0,-2*\kul-\n*\kpsep);
\foreach \n in {0,1,2,3} \draw[thin](uu14ap\n)--++(-\n*\kw-5*\kw,0)--++(0,-2*\kul-\n*\kpsep-\ksepY)coordinate(kA);
\draw[thin][decorate,decoration={brace,amplitude=5pt,raise=5pt,mirror}](kA)--++(8*\kw,0)node[pos=0.5,below,yshift=-3ex,inner sep=0pt,align=center]{\text{\RL{مخارج دفتر}}\\ $B$};
\draw[thin]($(uu10bp0)!0.25!(uu11dp0)$)node[]{$A$ \text{\RL{دفتر}}};
\draw[thin]($(uu14bp0)!0.25!(uu15dp0)$)++(-0.75*\ksepX,0)node[]{$B$ \text{\RL{دفتر}}};
\end{tikzpicture}
\caption{دفتر الف اور جمع و منفی کار}
\label{شکل_کمپیوٹر_دفتر_الف_جمع_منفی_کار}
\end{figure}
%==================================================================
%=====================================================================
\begin{figure}
\centering
\begin{tikzpicture}
\pgfmathsetmacro{\kw}{0.20}
\pgfmathsetmacro{\kpin}{0.30}
\pgfmathsetmacro{\kpsep}{0.40}			%pin to pin distance
\pgfmathsetmacro{\kw}{0.20}
\pgfmathsetmacro{\kul}{0.40}			%edge clearance along vertical edges
\pgfmathsetmacro{\kpinA}{0.5}
\pgfmathsetmacro{\kpinS}{0.4}
\pgfmathsetmacro{\ksepX}{5.5}
\pgfmathsetmacro{\ksepY}{4}
\ket[uu16]{0}{0}
\ket[uu17]{-\ksepX}{0}
\draw[thin](uu16-center)node[inner sep=0pt,align=center]{u16\\ 74LS83};
\draw[thin](uu17-center)node[inner sep=0pt,align=center]{u17\\ 74LS83};
\draw[thin](uu16dp4)--++(0,3*\kw)--++(\kpin,0)node[xor port,number inputs=2,rotate=-90,scale=0.7,anchor=out](uu19A){};
\draw[thin](uu16dp5)--++(0,2*\kw)--++(\kpin+1*\kpinA,0)--++(0,\kw)node[xor port,number inputs=2,rotate=-90,scale=0.7,anchor=out](uu19B){};
\draw[thin](uu16dp6)--++(0,1*\kw)--++(\kpin+2*\kpinA,0)--++(0,2*\kw)node[xor port,number inputs=2,rotate=-90,scale=0.7,anchor=out](uu19C){};
\draw[thin](uu16dp7)--++(0,0*\kw)--++(\kpin+3*\kpinA,0)--++(0,3*\kw)node[xor port,number inputs=2,rotate=-90,scale=0.7,anchor=out](uu19D){};

\draw[thin](uu17dp4)--++(0,3*\kw)--++(\kpin,0)node[xor port,number inputs=2,rotate=-90,scale=0.7,anchor=out](uu20A){};
\draw[thin](uu17dp5)--++(0,2*\kw)--++(\kpin+1*\kpinA,0)--++(0,\kw)node[xor port,number inputs=2,rotate=-90,scale=0.7,anchor=out](uu20B){};
\draw[thin](uu17dp6)--++(0,1*\kw)--++(\kpin+2*\kpinA,0)--++(0,2*\kw)node[xor port,number inputs=2,rotate=-90,scale=0.7,anchor=out](uu20C){};
\draw[thin](uu17dp7)--++(0,0*\kw)--++(\kpin+3*\kpinA,0)--++(0,3*\kw)node[xor port,number inputs=2,rotate=-90,scale=0.7,anchor=out](uu20D){};
\draw[thin](uu20A.in 1)--(uu19D.in 1)--++(2*\kpin,0)node[right]{$S_U$} +(-0.5*\kpin,0)coordinate(kRht);
\draw[thin](uu16cp1)-|(kRht)   (uu16ap1)--(uu17cp1);
\draw[thin](uu16ap0)node[ground]{}  (uu17ap0)node[ground]{};
\draw[thin](uu16ap2)node[left]{$\SI{5}{\volt}$}   (uu17ap2) node[left]{$\SI{5}{\volt}$};
\draw[thin](uu19A.in 2)--++(0,2*\kpin)coordinate(kSubTop);
\draw[thin](uu19B.in 2)coordinate(kA)--(kA|- kSubTop);
\draw[thin](uu19C.in 2)coordinate(kA)--(kA|- kSubTop);
\draw[thin](uu19D.in 2)coordinate(kA)--(kA|- kSubTop);
\draw[thin](uu20A.in 2)coordinate(kA)--(kA|- kSubTop);
\draw[thin](uu20B.in 2)coordinate(kA)--(kA|- kSubTop);
\draw[thin](uu20C.in 2)coordinate(kA)--(kA|- kSubTop);
\draw[thin](uu20D.in 2)coordinate(kA)--(kA|- kSubTop);
\foreach \n in {0,1,2,3}\draw(uu16dp\n)coordinate(kB)--(kB|- kSubTop);
\foreach \n in {0,1,2,3}\draw(uu17dp\n)coordinate(kB)--(kB|- kSubTop);
\end{tikzpicture}
\caption{جمع و منفی کار اور خارجی دفتر}
\label{شکل_کمپیوٹر_جمع_مفی_کار}
\end{figure}


%\immediate\closeout\tempfileTerms

%=========================================================
%\addcontentsline{toc}{chapter}{حوالہ}     %it is standard not to include References in toc. 
%\renewcommand*{\bibname}{حوالہ}      %this command has to be placed right here
%\begin{thebibliography}{99}\label{حوالہ_بیرونی_مواد}
%\begin{otherlanguage}{english}
%%\cite{حوالہ_کریزگ_الف_گیارہ}         this is how it is referenced in the text
%\bibitem{حوالہ_کریزگ_الف_سات}
 %Coddington, E. A. and N. Levinson, Theory of
%Ordinary Differential Equations. Malabar, FL: Krieger,
%1984.
%\bibitem{حوالہ_کریزگ_الف_گیارہ}
%Ince, E. L., Ordinary Differential Equations. New
%York: Dover, 1956.
%\bibitem{حوالہ_کریزگ_ب_اٹھارہ}
%Watson, G. N., A Treatise on the Theory of Bessel Functions. 2nd ed. Cambridge: University Press, 1944.
%\end{otherlanguage}
%\end{thebibliography}
%============================================================

\chapter*{جوابات}
\addcontentsline{toc}{chapter}{جوابات}

\immediate\closeout\tempfile
\begin{multicols}{2}
{\small
\input{answer}
}
\end{multicols}

%=======================
\appendix
\renewcommand{\theequation}{\arabic{equation}}  %this ensures that the chapter number is not included
%\include{./tex/calculusAppendixA}
%

%\include{./tex/emtEndOfBookTableDivergenceCurlGradientLaplacian}
%\include{./tex/toDoList}
%\include{./tex/emtQuestions}

\backmatter

\cleardoublepage
%\include{./tex/emtDataTables}        %appendices
%\include{./tex/emtLinearAlgebra}
%\include{./tex/emtCoordinatesRelations}
%================

%=================

\renewcommand*{\indexname}{فرہنگ}      %this command has to be placed right here just before printindex command
\cleardoublepage
\addcontentsline{toc}{chapter}{فرہنگ}

%\printindex


\end{document}
